%***********************************************************************
% RCS Identification
%***********************************************************************
% $Author: zender $
% $Date$
% $Id$
% $Revision$
% $Locker:  $
% $RCSfile: aca.tex,v $
% $Source: /home/zender/cvs/aca/aca.tex,v $
% $Id$
% $State: Exp $
%
% Purpose: Template for quickie papers.
%
% $Log: not supported by cvs2svn $
% Revision 1.1.1.1  1998-09-01 01:31:22  zender
% Imported sources
%
% Revision 1.2  1996/09/10 14:36:12  zender
% synchronization of locking mode checkin.
%
% Revision 1.1  1994/12/31  06:37:30  zender
% Initial revision
%***********************************************************************

%***********************************************************************
% Begin TeX notes
%***********************************************************************
% This is written for preprocessing by tib, the citation handler.
% To use all the features of tib the following commands are necessary:
% 
% alias tib       'tib -d /u4/boettner/tib ' 		;in the .cshrc file 
% alias tiblist   'tiblist -d /u4/boettner/tib '	;in the .cshrc file 
%
% tibdex clouds.ref		;creates INDEX
% tib -s jgr clouds.tex		;creates clouds-t.tex
%				;(-s = style, jgr = J. Geophys. Res. format)
% tib clouds.tex		;uses default style
% tex clouds-t.tex		;creates clouds-t.log and clouds-t.dvi
% xdvi clouds-t.dvi		;X window previewer
% dvips clouds-t.dvi		;creates clouds-t.ps
% tiblist -s jgr clouds.ref	;creates tex file directly from references
% tiblook <keyword>		;locates entry from reference file
%
% In order to include citations in the text do the following:
% <, ramaswamy detwiler ,> includes the unique citation which has both
% ramaswamy and detwiler as keywords in running text with
% the year in parenthesis, while [, reference ,] (where the ``,'' is a ``.'')
% puts the whole citation in parenthesis.
%
% This also uses the AMS TeX eplain.tex macro package for equation
% numbering.  To define an equation number, place \eqdef{eqname} right
% AFTER the equation.  To reference it in the text, use \eqref{eqname}
% The same goes for figures, only use \figdef and \figref. CAUTION:
% The figdef and eqdef macros use the same registers so naming a
% figure the same as an equation will confuse it.
%
% Note the following style points: the word ``figure'' should always
% be capitalized in text, i.e., and spelled in full at the beginning
% of a sentence but abbreviated in the middle of a sentence. ``Table''
% and ``Plate'' are the same but should never be abbreviated. And
% letters identifying subgraphs, i.e., Fig.~7{\it a\/} are italicized 
% in JGR but not JAS.
% 
% {\it Italicized\/} words should be followed by ``\/'' to adjust the
% next space correctly (except when the following character is a
% period or a comma). 
% 
% Forced spaces in math mode can be made by using the \ character,
% although that allows line breaks (unlike ~ which doesn't work in
% math mode), e.g., $\lambda = 8\ \mu$ instead of $\lambda = 8$~$\mu$.
%
%$$
%e^{i\pi}=-1
%\eqdef{eqn_name}
%$$
%
%Figure~\figref{fig_name}{\it a\/} is on the next page.
%
%$$
%e^{i\pi}=-1 \qquad\hbox{equation with text for $0 < w < 20$, 
%{\it IWP} in g/m$^2$, $w$ in cm/s} 
%\eqdef{eqn_name2}
%$$
%
% Here's an example of a bulleted outline:
%\vskip 1em
%\item{a.} First
%
%\item{b.} Second
%\itemitem{} sub-item
%\itemitem{} sub-item
%
%\vskip 1em
%
%***********************************************************************
% End TeX notes
%***********************************************************************

%***********************************************************************
% Begin Preamble
%***********************************************************************
% Define general page, paragraph, and line formatting here.
% macros for equation numbering
\input /cgd/home/zender/tex/eplain
% personal TeX definitions of item and month name
\input /cgd/home/zender/tex/utilities
% for inclusion of postscript figures /usr/local/lib/tex_family/tex/inputs/epsf.tex
\input epsf
% have tex report size of picture in output
\epsfverbosetrue
%
\magnification=\magstep1
\font\ninerm=cmr9
\font\tenrm=cmr10
\font\titlefont=cmr10 scaled\magstep1
\font\slanttitlefont=cmsl10 scaled\magstep1
\baselineskip=12pt
\null
%
%Definitions, hyphenations, and macros.
\def\reff{ {$r_{\rm e}$} }
\def\IWP{ {\it IWP\/} }
\hyphenation{va-por}
\hyphenation{depo-sition}
\hyphenation{short-wave}
\hyphenation{long-wave}
\hyphenation{anal-yzed}
\hyphenation{envi-ron-mental}
\hyphenation{cond-itions}
\hyphenation{trop-ical}
\hyphenation{meso-scale}
\hyphenation{near-ly}
\hyphenation{gas-e-ous}
\hyphenation{}
\hyphenation{}
\hyphenation{}
\hyphenation{}
%
%***********************************************************************
% End Preamble
%***********************************************************************

%***********************************************************************
% Begin Title Page
%***********************************************************************
%Skip the page number on the title page
\pageno=-1
%\headline={\tenrm {\it Draft of \today~~\timestring} \hfil \folio}
\footline={\hfil}
\nopagenumbers
\baselineskip=20pt
\ {\rm }
\vskip 2.truein
\centerline{Notes on Spectral Cloud Modelling}
\centerline{of Future TDDR flights over the ARM SGP Site}
\centerline{Charles S. Zender}
\centerline{zender@ncar.ucar.edu}
\vskip .5truein
{\obeylines\smallskip
\centerline{National Center for Atmospheric Research%
\footnote*{The National Center for Atmospheric Research is sponsored by the National Science Foundation.}}
\centerline{Boulder, Colorado 80307-3000}
\vskip 1.truein
\centerline{\today}}
%\centerline{{\it }}
\vfill\eject
%
%***********************************************************************
% End Title Page
%***********************************************************************

%***********************************************************************
% Begin Abstract 
%***********************************************************************
%Create the abstract:
\bigskip
\centerline{\it Abstract}
\smallskip
\baselineskip=14pt
{\parindent=0truein \narrower \smallskip \ninerm
\noindent
Recent measurements from a combination of {\it in situ}, satellite,
and station data lead to the conclusion that cloud models are
underestimating globally averaged cloud absorption by 20--30 W/m$^2$.
Quantifying this modelling deficiency, often called ``Anomalous
Cloud Absorption'' (ACA), and understanding its physical origin 
will be the subject of future studies including airborne TDDR
radiometers.  
The TDDR instrument currently has seven narrowband (10~nm)
channels located between .3 and 1.6~$\mu$m. 
This report attempts to address the problem of
identifying optimal spectral locations for new TDDR channels
with the goal of detecting any spectral signature to the ACA.
Toward this end a high resolution multiple scattering
radiation code is assembled and used to predict the response of
the seven existing and four new candidate channels to the cloudy
environment. 
The four new channels are picked between the existing channels
in the NIR. 
One is discarded because of poor SNR but
the remaining three are viable candidates for new TDDR channel
locations.
\smallskip}
\baselineskip=14pt
\vfill\eject
%
%***********************************************************************
% End Abstract 
%***********************************************************************

%***********************************************************************
% Begin Body of Paper
%***********************************************************************

\beginsection{1. Introduction} %the next line must be blank!

The unexplained cloud absorption known as ACA must have some spectral
signature, however broad. 
By carefully (or fortuitously) placing the narrowband TDDR channels it
is hoped to locate regions of heretofore unnoticed solar absorption in
the clouds over the Southern Great Plains (SGP) test site. 
The spectral distribution of the solar constant and
the size of the ACA already constrain the location of future TDDR
channels. 
The globally annually averaged ACA is estimated to be between 20 and
30 W/m$^2$ [.CZM94.]; this is about 5\% of the TOA solar flux.  
The solar irradiance studies of both 
<.LaN68.> and <.ThD71.> agree that less than 5\% of incoming solar
flux is redder (longer wavelength) than 2.3~$\mu$m, this constrains 
candidates to be bluer than 2.3~$\mu$m.  
Any ACA mechanism with continuum features might be observable farther
in the NIR than 2.3~$\mu$m, but should also be observable at shorter
wavelengths than 2.3~$\mu$m. 
Redder channels will have much better signal to noise ratios
(SNRs) for photon counters. 
Constraining the locations of the new channels further than this
requires detailed knowledge of the spectral structure of
atmospheric absorption in cloudy atmospheres, i.e., a radiative
transfer model.

%This will restore page numbering for rest of the manuscript. It
%should go somewhere before page two text begins.
\pageno=1
%\headline={\tenrm {\it Draft of \today~~\timestring} \hfil [\oldstyle\folio\tenrm]} 
\headline={\tenrm \hfil [\oldstyle\folio\tenrm]}
\footline={\hfil}
%
The rest of this report proceeds as follows.
Section 2 describes the workings of the radiation model and the
sources of the input data.
Section 3 presents a general overview of the atmospheric absorption
features of the troposphere followed by sensitivity
studies of spectral radiance to various configurations of clouds.
Section 4 summarizes the results and recommendations.

\beginsection{2. Radiation Model} %the next line must be blank!

The high spectral resolution radiative transfer model constructed
for these studies
accounts for multiple scattering, liquid and/or solid phase
hydrometeors, and absorption by O$_2$, O$_3$, CO$_2$, and H$_2$O
gases.  
Solar insolation has a resolution of .01~$\mu$m in the UV/visible to 
.2~$\mu$m in the NIR ([.LaN68.] or [.ThD71.]). Solar insolation has
the coarsest resolution of any of the input datasets.

Liquid cloud droplets are assumed to be present in a log normal size
distribution with a standard deviation of 1.5 and an effective radius
of 10~$\mu$m.  The optical properties of the droplet distribution
were computed from Mie theory using indices of refraction tabulated at
a resolution from .02~$\mu$m in the UV/visible to .1~$\mu$m in the
NIR.  Ice cloud crystals are also assumed to be present in a log
normal size distribution, except with a standard deviation of 1.5 and
an effective radius of 20~$\mu$m.  The optical properties of the
crystal distribution were computed from Mie theory using indices of
refraction tabulated at a resolution between .005 and .05~$\mu$m
through the solar spectrum. The ice fraction of a cloud was determined
by a simple function: warmer than -10$^\circ$C is pure liquid phase,
colder than -40$^\circ$C is pure ice phase, with a linear transition
through the mixed phase region (colder than -10$^\circ$C but warmer
than -40$^\circ$C). 

O$_3$ absorption cross-sections in the Hartley, Huggins, and Chappuis
bands, as well as O$_2$ absorption in the Herzberg continuum, is taken
from WMO (1985) data at a resolution between .02 and .05~$\mu$m. 
The WMO O$_3$ data is only input dataset that is not identical
to what Bruce P. Briegleb (NCAR) employed in his companion
study. 

Briegleb provided Malkmus random band parameters for H$_2$O (HITRAN
1992) and O$_2$ (HITRAN 1986)
at 10 cm$^{-1}$ intervals from .56~$\mu$m to 5
$\mu$m (17900 to 2000 cm$^{-1}$).  
CO$_2$ parameters (HITRAN 1986) were provided at 5 cm$^{-1}$ intervals
and averaged up to 10 cm$^{-1}$ in transmission space.
This narrow band data was used to
evaluate pseudo-monochromatic absoption coefficients for input to the
multiple scattering radiation code. 
Briegleb notes that significant
errors in these pseudo-monochromatic absorption coefficients can occur
when the cloud is optically thick enough to completely attenuate the
direct beam so that a diffusivity correction to the path length
should be utilized. 
This diffusivity effect will artificially decrease computed cloud
absorption when the solar zenith angle is less than  
the mean polar angle at which diffuse radiation propogates, usually
taken to be 50$^\circ$. 
In other words, the control solar geometry (see below) and 
computations within 3 hours of local noon 
should underpredict the cloud absorption. 
On the other hand, the
diffusivity effect should artificially increase computed cloud
absorption for solar zenith angles greater than 50$^\circ$ due the
foreshortened paths the diffuse radiation actually follows.
The diffusivity effect should be negligible near 9~AM and 3~PM.

The radiative transfer model used is DISORT, the well-known DIScrete
ORdinate Radiative Transfer program. DISORT is run in 16 stream mode
(eight per hemisphere) unless otherwise noted. 
What this report will refer to as the 
zenith and nadir intensities are actually the
azimuthally-averaged intensities at the closest computational Gaussian
angle to the zenith and nadir, about 11.5$^\circ$ for the 16 stream
mode (see below).

The atmospheric profile selected for the ARM SGP site was an 18 layer
version of the ICRCCM standard Mid-Latitude Summer (MLS) profile. 
The 300$^\circ$~K surface
temperature is perhaps a degree or two high for mid-September, and the
actual surface pressure is about 960~mb, not the 1000~mb employed by
the model. 
The availability of numerous validations to the standard MLS profile
seemed to outweigh slight benefits of these small corrections at this
early stage of modelling.
The surface is assumed to be a Lambertian reflector 
with a .2 albedo.

There are some differences in methodology and coding to consider when 
directly comparing results from this model to Briegleb's.
This model does not employ a layer extending from the MLS profile
top interface (1.5~mb) to space so computed broadband absorbed
fluxes are too small by less than .5 W/m$^2$.  
Our two models agree identically on H$_2$O absorption, but not on O$_2$
and CO$_2$ absorption, where this model absorbs about 25\% more (or
about 3~W/m$^2$) more than Briegleb's. 
The reason for this discrepency is still unexplained. 
Our Rayleigh scattering routines differ by a part in
500 in TOA clear sky albedo---probably because we use different
Rayleigh scattering parameterizations. 
Our treatments of cloud
extinction yield differences of a full percent in TOA albedo for a
standard 100~g/m$^2$ cloud configuration; mine has the higher albedo. 
This cloud discrepency could arise because Briegleb interpolates
the coarse Mie scattering properties onto the fine wavelength grid of
the spectral calculation whereas this model uses simple binning.
Briegleb differentiates clear sky from cloudy sky column
profiles by forcing the relative humidity to 100\% in all cloudy
layers. 
Despite this model giving a larger cloud albedo, it 
computes rather large SWCF ratios---as high as 1.39---for important 
cloud cases. 
There should be future intercomparisons with Briegleb's model 
to validate these large SWCF ratio results.
\medskip
\noindent{\sl Initial Conditions}\nobreak

The standard cloud case is called the control cloud.  
The control cloud is a 100 g/m$^2$ liquid water cloud with base at
1.6~km (900~mb). 
The control geometry was chosen to be local noon over the ARM SGP
Central Facility, which gives a solar zenith angle of 34$^\circ$. 
This is not a particularly useful solar angle as far as projecting
results to global averages, but it is one extrema 
of the absorption spectra: the shortest absorber paths and the highest
insolations which the flight craft could see. 
Because of the high solar elevation the downwelling radiances measured
by the up-looking TDDR beneath cloud will be much larger than
identical conditions for a lower sun. 
This improves the signal to noise ratio (SNR) for sub-cloud
TDDR but the shorter path lengths can also conceal some of the
absorption effects being sought by the TDDR.  
There is some direct evidence
from <.PiV94.> fig.~4 that the SWCF ratio
may peak near solar zenith angles of 30$^\circ$, which is very near 
the noontime sun at the ARM site. 
In any case, Briegleb extensively explores the geometry 
of a solar zenith angle near to .5 in his companion study.

\beginsection{3. Results} %the next line must be blank!

%
\medskip
\noindent{\sl Clear sky absorption}\nobreak

The transmission characteristics of the clear sky control
atmosphere are separated into the five gaseous extinction processes
considered in the model in figs.~\figref{trans_const}{\it a--e}.
Each figure was computed by turning of all processes except the one
displayed. 
O$_2$ (fig.~\figref{trans_const}{\it a}\/) has four absorption features in the shortwave: 
.63, .69, 7.6, and 1.27~$\mu$m. Since O$_2$ has such large absorber
paths, its absorption features are ubiquitous in the troposphere.
O$_3$ (fig.~\figref{trans_const}{\it b}\/) has three significant absorption features in the shortwave: 
.2--.3~$\mu$m (Hartley band), .30--33~$\mu$m (Huggins bands), and
.45--.80~$\mu$m (Chappuis bands). 
There is a slight temperature dependence to the Huggins bands---this
model employs cross-sections appropriate to -70$^\circ$C which would
tend to underestimate absorption.
Together with the O$_2$ Herzberg continuum (not shown), O$_3$ is
responsible for the extinction of all the incident UV above the
surface. 

The strongest CO$_2$ bands (fig.~\figref{trans_const}{\it c}\/) are
centered at 1.4, 1.6, and 2~$\mu$m. 
The 1.4~$\mu$m band overlaps with an H$_2$O band
(fig.~\figref{trans_const}{\it d}\/), and the 1.6~$\mu$m band 
is also strongly absorbed by ice (fig.~\figref{trans_const}{\it f}\/).
None of the TDDR channels is located in a region of significant CO$_2$
absorption. 
H$_2$O vibration-rotation bands extend from .6~$\mu$m throughout the
NIR.  
H$_2$O absorption accounts for $\sim$~90\% of all clear sky atmospheric
absorption. 
H$_2$O absorption in cloudy atmospheres certainly deserves scrutiny
with the TDDR instrument because incorrect representation of it
could lead to the large flux discrepencies necessary to explain ACA.
Since all of the existing TDDR channels are located in H$_2$O windows,
two of the proposed channels, at .91 and 1.18~$\mu$m, are placed in
the wings of H$_2$O bands. 
These two channels should help us gauge our understanding of some of
the near wing H$_2$O absorption. 
Two more channels, at 1.235 and 2.2~$\mu$m, are proposed at the center
of windows not currently seen by the TDDR.
Rayleigh scattering (fig.~\figref{trans_const}{\it e}\/), is responsible
for the clear sky optical visible depth of .15 which has been
subtracted from the optical depth reported on the figures (see
Appendix~A for the legend to the diagnostic information printed to the
left of each figure).

Figure~\figref{trans_const}{\it f}\ compares the mass absorption
coefficient of the ice crystal distribution to that of the liquid
droplet distribution.
For a given {\it CWP\/} the absorption optical depth scales inversely
with the effective radius, so we can divide the bottom figure by two
to compensate for the difference in effective radii. 
The spectral features that stand out are the very enhanced ice
absorption peaks at 1.6 and 2.0 $\mu$m; these peaks are much smaller
in the liquid water absorption profile.
\medskip
\noindent{\sl Control atmosphere and control cloud}\nobreak

The general transmission characteristics of the control
atmosphere including all gaseous and cloud constituents
are shown in figs.~\figref{hires_clr}--\figref{lores_cld}.
The high resolution figs.~\figref{hires_clr}\ and \figref{hires_cld}\
are plotted at the
computational resolution, while the low resolution
figs.~\figref{lores_clr}\ and \figref{lores_cld}\ were averaged to
10~nm (the nominal TDDR bandwidth) with a square filter.  
The 10~nm plots remove much of the fine structure of the H$_2$O bands
as well as portraying the TDDR instrument more faithfully.
It would be interesting to see what the fine structure would be if the
correct filter response functions were applied instead of boxcar
averaging. 
All figures following fig.~\figref{lores_cld}\ are plotted at the
10~nm resolution.
Note that Transmission ($T$), Absorptance ($A$), and Reflectance
($R$), are normalized so that $T + R + A = 1$ in all figures. 

The fluxes (figs.~\figref{hires_clr}{\it d}, \figref{hires_cld}{\it
d}, \figref{lores_clr}{\it d}, and \figref{lores_cld}{\it d}\/) and
radiances (figs.~\figref{hires_clr}{\it e}, \figref{hires_cld}{\it
e}, \figref{lores_clr}{\it e}, and \figref{lores_cld}{\it e}\/) are
plotted as they would be observed by tandem aircraft flying at nominal
levels of 1~km (900~mb) and 12.5~km (197~mb). 
Examination of the downwelling clear sky fluxes
(figs.~\figref{hires_clr}{\it d} or \figref{lores_clr}{\it d}\/) shows
that the H$_2$O absorption above 13~km is almost neglible because the
stratospheric H$_2$O mass path is very small.
The attentuation at 13~km also shows the fingerprints of the 1.42 and
2.0~$\mu$m CO$_2$ bands.
The flux attenuation from TOA to 13~km at .5~$\mu$m is mostly due to
Rayleigh backscattering, while in the UV and NIR the stratospheric
O$_3$ bands have extinguished $\sim$~20~W/m$^2$.

The increased flux attenuation from 13~km to 1~km is a consequence of
the exponential increase in path lengths with pressure. 
The powerful H$_2$O absorption in the NIR reflects the moisture in the
lower troposphere. 
Enhanced Rayleigh scattering in the UV--visible and more pronounced
O$_2$ absorption in the NIR stem from the dry atmosphere.
One notices the small flux in the proposed channel at 2.2~$\mu$m, 
remains unattenuated by the clear sky.
The main features of H$_2$O absorption---the windows at 1, 1.23, and
1.6~$\mu$m---are not strongly dependent on the thermodynamic phase of
the H$_2$O. 
At solar zenith angles besides twilight, the flux absorption due to 
clouds of any phase (see below) will look qualitatively like
absorption due to increased H$_2$O vapor.

The downwelling radiance below cloud base is viewed by a TDDR
looking towards its zenith and so will be referred to as the zenith
radiance.
The upwelling radiance above cloud top is viewed by a TDDR
looking towards its nadir and so will be referred to as the nadir
radiance.
These definitions agree with Briegleb's usage.
Zenith and nadir radiances 
are components of the diffuse radiation field since the sun is not
directly overhead. 
The clear sky zenith radiance
(figs.~\figref{hires_clr}{\it e}\ or \figref{lores_clr}{\it e}\/)
is proportional to the clear sky downwelling flux 
(figs.~\figref{hires_clr}{\it d}\ or \figref{lores_clr}{\it d}\/)
by a factor of $\pi/A_{\rm sfc} \sim 15$.
Rayleigh scattering dominates the nadir radiance to the
extent that the near UV is brighter than the visible.

The cloudy sky zenith radiance has the same morphology as the
the nadir radiance for the control cloud.
Examination of the slopes of the radiances in
figs.~\figref{hires_cld}{\it e}\ or \figref{lores_cld}{\it e}\ 
shows the nadir radiance is preferentially strengthened relative to
the zenith radiance in the visible, but weakened in the NIR.
The strengthening in the visible may be related to the forward
scattering peak of the hydrometeors, 
while the weakening in the NIR is caused by the H$_2$O vapor and
condensate absorption from 13~km to 1~km.

Three dimensional surface plots of the
absorbed spectral flux (figs.~\figref{spec_abs}{\it a,b}\/) give a  
qualitative overview of the structure of absorption in the
atmosphere.
Cloudy sky absorption (fig.~\figref{spec_abs}{\it b}\/) 
 is stronger than 
clear sky absorption (fig.~\figref{spec_abs}{\it a}\/) 
above the cloud because of the absorption of backscattered radiation.
The tall spike in the lower left corner is stratospheric O$_3$
absorption, while the CO$_2$ bands at 4.3~$\mu$m are seen in the lower
right.
The control cloud absorption at 800~mb is responsible for the
strongest peaks and for the long low ridge of greybody absorption in
the right rear of fig.~\figref{spec_abs}{\it b}.

Two more 3D-surface figures
verify that the azimuthally-averaged radiances being presented are 
adequate representations of the azimuthally resolved 
radiance field. 
Fig.~\figref{full_rad}{\it a}\ 
shows the full radiance field for the .5~$\mu$m channel at 1~km.
The lower TDDR will be looking at polar angle $\pi$ or 180$^\circ$,
which appears azimuthally symmetric to well within 10\%.
The high degree of azimuthal symmetry is a sign
that  the diffuse radiation beneath the cloud has been multiply
scattered. 
Variation of the radiance with the polar angle near the zenith is
negligible, justifying the use of the closest computational polar
angle to the zenith in this 16 stream computation.

Fig.~\figref{full_rad}{\it b}\ 
depicts the angularly resolved radiance field for the
visible channel above the cloud at 13~km. 
There are polar angles for which the
scattered intensity is clearly not azimuthally symmetric, with the
peak asymmetry edge-on to the cloud near a 90$^\circ$ polar angle.
However, the nadir radiance field (polar angle 0$^\circ$) is still
azimuthally symmetric.
The radiation field looking edge-on to the cloud shows peaks at the
same azimuthal angle as the sun (0$^\circ$ = 360$^\circ$) and specular
reflection-like peak at 180$^\circ$. 
These peaks are due to
singly scattered photons reflected off the top of the cloud. 
The nadir radiance
comes from multiply scattered photons which penetrated
deeper in the cloud.
These diffuse photons are azimuthally homogeneous by the time they
scatter up and out of the cloud.
Simulations of TDDR viewing angles within about
15$^\circ$ from nadir/zenith should be able to employ 
azimuthally averaged intensities from a 16~stream computation.
\medskip
\noindent{\sl Sensitivity to increasing {\it LWP\/}}\nobreak

The first sensitivity study examines the dependence of the
nadir/zenith radiances on the {\it LWP\/} in the control cloud.
The liquid control cloud at 800~mb was thickened from 0 to 300 g/m$^2$
in increments of 25 g/m$^2$. 
The column SWCF ratio decreased monotonically from 1.25 (at
25~g/m$^2$) to 1.14 (at 300~g/m$^2$).
The TDDR viewing radiances are shown for selected {\it LWP\/} in
figs.~\figref{cwp_rad}{\it a--h}.
Only the {\it LWP\/} is varied in these figures
(note that the radiance scale varies).

The nadir radiance monotonically increased with {\it LWP\/}; this
increase should be attributed to   
the higher TOA albedos of the thicker clouds. 
The nadir radiance becomes stronger than the zenith radiance when
$A_{\rm {TOA}} \sim .5$, or {\it LWP\/} $\sim 100$~g/m$^2$.
The zenith radiance increases with {\it LWP\/} until the multiple
scattering regime is reached at about $\sim 15$~g/m$^2$.
For thicker clouds
the zenith radiance monotonically decreases because less downwelling
radiation is able to penetrate the cloud.

Three of the four candidate channels should receive stronger signals
than the existing 1.64~$\mu$m channel.
Presumably these three channels would have an adequate SNR.
Fig.~\figref{cwp_rad}{\it b}\ shows
that the SNR problem in the proposed 2.2~$\mu$m channel seems to
be compounded by the insensitivity of the channel to {\it LWP\/}.  
The insensitivity is not as severe viewed from the lower TDDR
(fig.~\figref{cwp_rad}{\it d}\/), 
but the SNR problem is even more severe at 1~km because
the 2.2 $\mu$m window has completely closed for clouds heavier than 
about 200~g/m$^2$.
Therefore this 2.2~$\mu$m channel may have to be discarded
as a candidate.

Finally, all the channel radiances for this study were plotted against
the radiance in the visible channel ($\lambda = .5~\mu$m) to reproduce
the radiance relationships that Bob Cess (SUNY) has been looking at 
(figs.~\figref{cwp_rad_rat}{\it a--d}\/).
The visible channels in figs.~\figref{cwp_rad_rat}{\it a,b}\ 
(.41 and .67~$\mu$m) show no evidence of 
increasing more slowly than the .5~$\mu$m channel over the range of
{\it LWP\/} tested.
The radiance growth in the NIR channels, however, does tend to 
lag behind the visible channel as the {\it LWP\/} increases. 
This ``turning over'' in the NIR was seen by Cess. 
\medskip
\noindent{\sl Sensitivity to ice phase}\nobreak

The next sensitivity study examined what effect ice phase clouds had
on the absorption. 
For this purpose the 100 g/m$^2$ control cloud was treated as
ice ($r_{\rm e} = 20\ \mu$m), 
instead of liquid ($r_{\rm e} = 10\ \mu$m), 
even though the cloud is at 800~mb. 
The column SWCF ratio of the liquid cloud was 1.16
while the SWCF ratio for the ice cloud was surprisingly high: 1.39.
Unfortunately this sensitivity study was not performed for the
same size distribution in each phase in order to disentangle the
hydrometeor size effects from the phase effects.
Fig.~\figref{trans_const}{\it f}\ provides
insight to where the spectral
signature of the ice cloud should show up and 
figs.~\figref{ice}{\it a,b}\ compare the liquid to ice radiances at
each flight level.
The 2.0~$\mu$m region was nearly opaque due to H$_2$O vapor,
so the difference between ice and liquid condensate is difficult
to detect here.
However, by comparing the peak heights of the 1.25 and 1.6~$\mu$m
windows for both liquid and ice, one can clearly see the enhanced
absorption of ice relative to liquid throughout the 1.6~$\mu$m window. 
\medskip
\noindent{\sl Sensitivity to increasing solar zenith angle}\nobreak

The final sensitivity test in this report analyzed the dependence of
the nadir/zenith radiances on the solar zenith angle.
For this study a moderately thick (200~g/m$^2$) mixed phase cloud
($\sim 50$\% ice), was simulated in the upper troposphere
between 6.3 and 10~km (470 to 280~mb).
Starting at local noon a computation was performed each hour until
twilight. 
The column SWCF ratio decreased monotonically from 1.04 (at
local noon) to .56 (at twilight).
Figs.~\figref{sza_rad}{\it a--g}\ show that
both nadir and zenith radiances monotonically decreased with
increasing solar zenith angle.
This decrease should be attributed to the decreasing TOA flux.
Absorptance between 1~km and 13~km monotonically decreased from
15\% of TOA flux (at local noon) to $\sim 10\%$ (at twilight).
One also notices the relative enhancement of the nadir radiance to the 
zenith radiance as the shadows lengthen.
The zenith radiance is depleted because of the large number of
scatterings needed to penetrate the cloud while while the nadir
radiance is enhanced by the same effect (resulting in larger TOA
albedos for larger zenith angles).
The nadir enhancement is more pronounced at shorter wavelengths due to
more efficient Rayleigh scattering.

As the sun moved toward the horizon, the mass paths enlongated
but figs.~\figref{sza_rad_rat}{\it a--d}\/ do not show any 
of the ``turning over'' that characterized the NIR in the {\it LWP\/}
sensitivity study (figs.~\figref{cwp_rad_rat}{\it a--d}\/).
Instead the radiances in all spectral channels diminish evenly
as the solar zenith angle increases.
The absorber paths seem to be within the linear limit in this
sensitivity study.

\beginsection{4. Conclusions} %the next line must be blank!

%
The angularly resolved radiance field was examined to determine
whether nadir/zenith radiances could be approximated by the
azimuthally averaged radiance at the nearest computational angle
in a 16~stream calculation.
Figs.~\figref{full_rad}{\it a,b}\ 
support the approximation that the
azimuthally resolved nadir/zenith radiances do not differ
significantly from the azimuthally averaged nadir/zenith radiances, at
least for multiply-scattering clouds.
Sixteen streams appears to be adequate in resolving the nadir/zenith
polar angle as well.

All computations were performed assuming spherical ice crystals with
$r_{\rm e} = 20\ \mu$m. 
Recent observations by <.KKW93.> reported high numbers of small ice
crystals (total concentration $\sim 10000\ \ell^{-1}$) in tropical
cirrus.  
If the log-normal size distribution employed here undercounts small
ice crystals then the computed absorption would be underestimated near
cloud top and overestimated near cloud base [.ZeK94.].
Treating ice crystals as aspherical tends to lessen absorption
[.TaL89a.].
Computations with the CCM2 column radiation code [.Bri92a.] 
as modified to account for crystal size and shape effects were unable
to produce SWCF ratios in excess of 1.1. 
Accounting for these ice effects did improve (raise) the TOA albedo,
however. 
Those preliminary and spectrally-coarse (18 band) calculations indicate
that including detailed size 
distribution and crystal shape effects into the current model would
change $A_{\rm TOA}$ more than the cloud forcing ratio.

The SWCF ratio, the diagnostic most often used to assess ACA, was
computed for each cloud scenario.
The SWCF ratio tended to increase with decreasing {\it LWP\/} (down to
25~g/m$^2$) and decreasing zenith angle (up to 30$^\circ$).
The largest SWCF ratio simulated in these studies, 1.39, occurred 
at $\Theta \sim 34^\circ$ when the control cloud at 800~mb was
treated as 100\% ice. 
This particular case, although unphysical, should be validated by
Briegleb because of its high SWCF ratio.
<.PiV94.> observed a maxima in SWCF ratio near 30$^\circ$.

The results suggest at least two avenues for further computations:
mixed phase clouds at liquid temperatures, and varying ice and liquid
size distributions.
Could large numbers of ice phase hydrometeors
(e.g., hail or mixed-phase hydrometeors) beneath the melting level
significantly affect cloud absorption properties? 

There are two main criteria which candidate channels for the
TDDR must meet: 
First, the channel should not saturate under normal cloudy
conditions. 
Second, the channel should represent a spectral window which
could contribute substantially towards the observed 20-30~W/m$^2$ ACA.
Three of the four candidate channels (.91, 1.23, and 1.18~$\mu$m) meet
these criteria. 
The candidate at 2.2~$\mu$m fails the first criteria for the bottom
TDDR, and arguably fails the second criteria as well.
\medskip
\baselineskip=14pt
\noindent{\it Acknowledgements.}
I thank Bruce Briegleb for his generously providing the optical data,
and for his comments at every stage of this work.
This work was supported in part by Earth Observing System project
W-17,661. 
\baselineskip=14pt
%
\vfill\eject
%***********************************************************************
% End Body of Paper
%***********************************************************************

%***********************************************************************
% Begin Appendices
%***********************************************************************
\centerline{APPENDIX A}
\bigskip
Diagnostic quantities printed on the right hand side of the figures:

{\bf First line:} Geometric boundary conditions and surface albedo
\item{a.} Profile type
\item{b.} Latitude
\item{c.} Day of year
\item{d.} Hour of day (standard time)
\item{e.} Solar zenith angle
\item{f.} Cosine of the solar zenith angle
\item{g.} Surface albedo

{\bf Second line, first half:} Cloud properties 
\item{a.} Condensed Water Path ({\it CWP\/})
\item{b.} Base altitude of lowest level containing clouds
\item{c.} Thickness from highest cloud in column to lowest
\item{d.} Fraction of total {\it CWP\/} that is ice, i.e., {\it IWP}/{\it CWP}

{\bf Second line, second half:} Cloud forcings
\item{a.} Cloud forcing evaluated at TOA
\item{b.} Cloud forcing evaluated at surface
\item{c.} Cloud forcing ratio for entire column = surface cloud forcing/TOA cloud forcing
\item{d.} Cloud forcing ratio for column between 1 and 13~km = cloud
forcing at 1~km divided by cloud forcing at 13~km

{\bf Third line:} Column optical properties and broadband fluxes
(in W/m$^2$)  
\item{a.} TOA albedo
\item{b.} Cloud extinction optical depth at $\lambda = .5$~$\mu$m
\item{c.} Flux down at TOA
\item{d.} Flux down at surface
\item{e.} Flux absorbed in atmosphere
\item{f.} Flux absorbed by surface
\item{g.} Flux absorbed between 13~km and 1~km

{\bf Fourth line:} Broadband fluxes (in W/m$^2$) measureable by
aircraft at 13~km and 1~km
\item{a.} Total downwelling flux (direct + diffuse) at $z = 13$~km
\item{b.} Direct beam flux at $z = 13$~km
\item{c.} Upwelling flux at $z = 13$~km
\item{d.} Net flux at $z = 13$~km
\item{e.} Total downwelling flux (direct + diffuse) at $z = 1$~km
\item{f.} Direct beam flux at $z = 1$~km
\item{g.} Upwelling flux at $z = 1$~km
\item{h.} Net flux at $z = 1$~km
%
\vfill\eject
%***********************************************************************
% End Appendices
%***********************************************************************

%***********************************************************************
% Begin References
%***********************************************************************
% A percent followed immediately by tib citations means the citations
% will be added to the reference list but won't be cited by tib
% anywhere in the paper.  this is useful for citations which had to
% be formatted by hand.
% the following .[] is used by tib to generate the reference list
%[,extra references,],[,more references,] 
\centerline{REFERENCES}
\bigskip
.[]
\vfill\eject
%***********************************************************************
% End References
%***********************************************************************

%***********************************************************************
% Begin Tables
%***********************************************************************
\nopagenumbers
\baselineskip=16pt
% The '#' means ``stick the text of each column entry in this place''.
% The '&' is like the the tab key on the typewriter, TeX backs up to
% the beginning of the current column then advances one column exactly.
% \cr signifies the end of the row, all unfilled columns are assumed
% blank (TeXbook, p. 231). 
% Here's an example of a formatted table placed in the middle of the text:
%\midinsert
%$$
%\vbox{
%\halign{
% Here is the template for the table:
%# \hfil \qquad & # \hfil \qquad \cr
%
%\noalign{\hrule}
%\noalign{\vskip-2pt}
%\noalign{\hrule}
%\noalign{\vskip6pt}
%\multispan2{\hfil Initial Properties of Standard Cloud \hfil} \cr
%\noalign{\vskip4pt}
%\noalign{\hrule}
%\noalign{\vskip4pt}
%Cloud base (km) & 11\cr
%Maximum vertical updraft (cm-s$^{-1}$) & 5\cr
%\noalign{\vskip4pt}
%\noalign{\hrule}
%
%} % end halign
%} % end vbox
%$$
\baselineskip=12pt
%{\bf Table~1.} The initial cloud properties are shown above. 
%The {\it mean\/} properties are vertical averages within the cloud. 
\baselineskip=16pt
%\endinsert
%
%
\baselineskip=16pt
\vfill\eject
%***********************************************************************
% End Tables
%***********************************************************************

%***********************************************************************
% Begin Figure Captions
%***********************************************************************
%Print out the figure captions on their own pages in their own font.
%Skip the page number on the figure caption pages.
\nopagenumbers
\baselineskip=14pt
%
\bigskip
\vbox{
\noindent{\bf  Figure~\figdef{trans_const}}.
Atmospheric transmittance $T$ for each gaseous
constituent from $\lambda = .2$--$2.5~\mu$m.
({\it a\/}) O$_2$ absorption.
({\it b\/}) O$_3$ absorption.
({\it c\/}) CO$_2$ absorption.
({\it d\/}) H$_2$O absorption.
({\it e\/}) Rayleigh scattering.
({\it f\/}) Mass absorption coefficients for ice and liquid water
log-normal hydrometeor distributions for $\lambda = .2$--$5~\mu$m:
$\sigma = 1.5$, $r_{\rm e} = 10 \mu$m (liquid) and  $r_{\rm e} =
20~\mu$m (ice). 
}
%
\bigskip
\vbox{
\noindent{\bf  Figure~\figdef{hires_clr}}.
Clear sky optical properties, fluxes, and radiances at the full
spectral resolution of the computations.
({\it a\/}) Transmittance $T$ from $\lambda = .2$--$5~\mu$m.
({\it b\/}) Absorptance $A$ from $\lambda = .2$--$5~\mu$m.
({\it c\/}) Reflectance $R$ from $\lambda = .2$--$5~\mu$m.
({\it d\/}) Spectral downwelling flux $F^\downarrow_\lambda$ at TOA,
13~km, and 1~km from $\lambda = .2$--$2.5~\mu$m.
({\it e\/}) Spectral upwelling radiance $I^\uparrow_\lambda$ at 13~km,
and spectral downwelling radiance $I^\downarrow_\lambda$ at 1~km from $\lambda = .2$--$2.5~\mu$m.
}
%
\bigskip
\vbox{
\noindent{\bf  Figure~\figdef{lores_clr}}.
Same as figure~\figref{hires_clr}\ but averaged to 10~nm resolution.
}
%
\bigskip
\vbox{
\noindent{\bf  Figure~\figdef{hires_cld}}.
Same as figure~\figref{hires_clr}\ but for the control cloud:
100~g~m$^{-2}$ liquid droplets, base at 1.6~km (836~mb), and thickness
of .9~km (80~mb).
}
%
\bigskip
\vbox{
\noindent{\bf  Figure~\figdef{lores_cld}}.
Same as figure~\figref{hires_cld}\ but averaged to 10~nm resolution.
}
%
\bigskip
\vbox{
\noindent{\bf  Figure~\figdef{spec_abs}}.
Absorbed spectral flux at all wavelengths and levels.
Horizontal axes are level in the atmosphere (Pa) and wavelength (m). 
({\it a\/}) Clear sky.
({\it b\/}) Control cloud (100~g/m$^2$ at 800~mb).
}
%
\bigskip
\vbox{
\noindent{\bf  Figure~\figdef{full_rad}}.
Radiance field for all azimuthal and polar angles for $\lambda =
.5~\mu$m. 
Horizontal axes are polar angle (radians) and azimuthal angle
(radians). The polar angle is measured from the vertical so that
upwelling radiance has $\Theta = 0$.
The azimuthal coordinate (which wraps around on itself) is fixed by
the azimuthal angle of the sun, set to $\phi_0 = 0$.
({\it a\/}) Radiance beneath control cloud (1~km).
({\it b\/}) Radiance above control cloud (13~km).
}
%
\bigskip
\vbox{
\noindent{\bf  Figure~\figdef{cwp_rad}}.
Radiances at the flight levels for selected {\it LWP}\/s in the
control atmosphere.
({\it a\/}) {\it LWP\/}~=~0~g~m$^{-2}$.
({\it b\/}) {\it LWP\/}~=~25~g~m$^{-2}$.
({\it c\/}) {\it LWP\/}~=~50~g~m$^{-2}$.
({\it d\/}) {\it LWP\/}~=~100~g~m$^{-2}$.
({\it e\/}) {\it LWP\/}~=~150~g~m$^{-2}$.
({\it f\/}) {\it LWP\/}~=~200~g~m$^{-2}$.
({\it g\/}) {\it LWP\/}~=~250~g~m$^{-2}$.
({\it h\/}) {\it LWP\/}~=~300~g~m$^{-2}$.
}
%
\bigskip
\vbox{
\noindent{\bf  Figure~\figdef{cwp_rad_rat}}.
Radiance in the visible ($\lambda = .5~\mu$m) channel (horizontal
axis) is plotted versus radiance in the rest of the TDDR channels
(vertical axis). 
The atmospheric profile is as for the control cloud
except the total {\it LWP\/} in the cloud layer is varied from 0 to
300~g~m$^{-2}$ in steps of 25~g~m$^{-2}$.
{\it LWP\/} and channel center are labelled by hand.
({\it a\/}) 13~km upwelling radiances for existing channels.
({\it b\/}) 13~km upwelling radiances for all (existing + candidate) channels.
({\it c\/}) 1~km downwelling radiances for existing channels.
({\it d\/}) 1~km downwelling radiances for all (existing + candidate) channels.
}
%
\bigskip
\vbox{
\noindent{\bf  Figure~\figdef{ice}}.
Radiance at flight levels for the 100~g~m$^{-2}$ control cloud
treated as liquid spheres ($r_{\rm e} = 10 \mu$m) versus ice spheres
($r_{\rm e} = 20 \mu$m). 
The diagnostic figures printed on the sidebar are for the ice cloud.
({\it a\/}) Upwelling radiances at 13~km.
({\it b\/}) Downwelling radiances at 1~km.
}
%
\bigskip
\vbox{
\noindent{\bf  Figure~\figdef{sza_rad}}.
Radiances at the flight levels for hourly solar zenith angles
starting at local noon. 
The mixed phase cloud is 200~g/m$^2$, $\sim 50$\% ice, between 
6.3 and 10~km (470 to 280~mb).
({\it a\/}) Local noon, $\theta = 34^\circ$, $\cos\theta = .83$.
({\it b\/}) 1 PM or 11 AM, $\theta = 37^\circ$, $\cos\theta = .80$.
({\it c\/}) 2 PM or 10 AM, $\theta = 44^\circ$, $\cos\theta = .72$.
({\it d\/}) 3 PM or 9 AM, $\theta = 53^\circ$, $\cos\theta = .60$.
({\it e\/}) 4 PM or 8 AM, $\theta = 66^\circ$, $\cos\theta = .43$.
({\it f\/}) 5 PM or 7 AM, $\theta = 76^\circ$, $\cos\theta = .24$.
({\it g\/}) 6 PM or 6 AM, $\theta = 88^\circ$, $\cos\theta = .03$.
}
%
\bigskip
\vbox{
\noindent{\bf  Figure~\figdef{sza_rad_rat}}.
Same in fig.~\figref{cwp_rad_rat}\ but for the cloud scenarios in 
fig.~\figref{sza_rad}.
Local time and channel center are labelled by hand.
}
%
%\bigskip
%\vbox{
%\noindent{\bf  Figure~\figdef{thick}}.
%Radiances and fluxes at flight levels for the 700~g~m$^{-2}$ 
%mixed phase cloud from 1.5 to 11.5~km (830 to 200~mb).
%({\it a\/}) Spectral Flux.
%({\it b\/}) Nadir and zenith radiances.
%}
%%
%\bigskip
%\vbox{
%\noindent{\bf  Figure~\figdef{thin}}.
%Radiances and fluxes at flight levels for the 25~g~m$^{-2}$ 
%mixed phase cloud from 8 to 10~km (370 to 280~mb).
%({\it a\/}) Spectral Flux.
%({\it b\/}) Nadir and zenith radiances.
%}
%
%Reset the font from the figure caption settings to the text body:
\baselineskip=14pt
\tenrm
\vfill\eject
%***********************************************************************
% End Figure Captions
%***********************************************************************

%***********************************************************************
% Begin Figure Graphics w/o captions
%***********************************************************************
%
%\epsfxsize=6.5in
%\null\bigskip\centerline{\epsffile{/home/zender/web/papers/cir_rad_sens/figure_1.ps}}
%\bigskip\vbox{\centerline{\noindent{\bf Figure~1}.}}
%
%***********************************************************************
% End Figure Graphics w/o captions
%***********************************************************************

%***********************************************************************
% End Document
%***********************************************************************
%
\bye










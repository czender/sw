% FM95FRM.TEX -- AGU 1995 Fall Meeting abstract 
% electronic form. 

% DO NOT DELETE NEXT LINE.  It identifies the form:
% American Geophysical Union electronic abstract form. 
 
% FYI: COMMENTS are preceded by the "%"
% character. Lines that start with a "%" are
% ignored by LaTeX. 
% These comment lines must be left in the
% template that you send to AGU. 
% 
% COMMAND lines to be completed by the author 
% start at the left margin with either 
% "\" or "%\". 
% PLEASE BE SURE TO FOLLOW CAREFULLY ALL
% INSTRUCTIONS TO ENSURE THE COMPLETION OF YOUR
% ABSTRACT TEMPLATE. 
 
% If you require assistance in completing this
% template, send your questions to
% ABS-HELP@EARTH.AGU.ORG. 
% You will receive a response within 24 hours. 
 
% The \documentstyle and \nofiles command lines
% are REQUIRED.  DO NOT EDIT OR DELETE THEM. 
 
\documentstyle[11pt,fm95ab]{article} 
\nofiles 
 
% EXTENDED ABSTRACT SUBMISSION.  If you prefer 
% to send an extended abstract (28cm), remove 
% the "%" from in front of the \extendedabstract
% command.  This command must appear before the
% \begin{document}.
% DO NOT DELETE EITHER OF THESE COMMANDS. 
 
%\extendedabstract 

\begin{document} 
 
 
% AGU office staff completes the following data
% after the abstract has been received and
% accepted.

% DO NOT EDIT OR DELETE the \referencenum, 
% \received, and \sessionid lines. 
 
%\referencenum{} 
%\received{} 
%\sessionid{}{} 
 
 
% Some of the administrative information command
% lines below may require your input response. 
% In these cases, the word REQUIRED will be
% stated in the instruction above the command. 
% 
% Other command lines may only need your input 
% response if the item is applicable to your 
% submission.  IN CASES WHERE YOU DO NOT HAVE A 
% RESPONSE, LEAVE THE COMMAND LINE AS IS. 
% 
% The title of the meeting has been provided for
% you. DO NOT EDIT OR DELETE THE \meetingid
% COMMAND. 
 
\meetingid{1995 Fall Meeting}
 
% MEMBERSHIP IDENTIFICATION.  Abstracts may be 
% submitted only by AGU members; non-members must
% have the sponsorship of an AGU member.  This
% includes invited abstracts by non-members of
% AGU. 
% 
% AGU members must enter the identification
% number of one of the authors who is an AGU
% member in the % \memberid braces. Do not
% complete the \sponsor command if this applies
% to you.
% 
% If none of the authors are AGU members, give
% the name and ID number of an AGU member sponsor
% in \sponsor.  Leave the \memberid command blank
% in this case. 
%
% An AGU ID number consists of nine digits.  To
% find your ID, see the mailing label on an AGU
% journal. 
% 
% Membership information is REQUIRED. 
 
\memberid{14104858} 
\sponsor{} 
 
 
% CORRESPONDING AUTHOR INFORMATION.  Replace the 
% example input with the name and mailing 
% address of the author to whom all
% correspondence (e.g., acceptance letter) should
% be sent in the \correspondingname and address
% braces. 
% 
% Indicate line breaks using \\, (e.g.,
% "USGS\\Reston, VA 22092"). 
%
% (PLEASE NOTE: \correspondingaddress is the only
% command where the line break slashes (\\) may
% be used). 
% 
% Give this author's phone number in
% \correspondingphone, their fax phone number in
% \correspondingfax, and their e-mail address in
% \correspondingemail. 
% 
% Please delimit parts of the phone numbers with 
% hyphens, when applicable: 
% country code-area or city code-office-number,
% e.g.,  "33-1-64-84-54".  Please give the
% author's Internet address, (user@host.domain).
% 
% All corresponding author information is
% REQUIRED.
 
\correspondingname{Charles S. Zender} 
\correspondingaddress{NCAR\\P.O. Box 3000}
\correspondingcity{Boulder} 
\correspondingstate{CO} 
\correspondingpostal{80307-3000} 
\correspondingcountry{USA} 
\correspondingphone{303-497-1612} 
\correspondingfax{303-497-1324} 
\correspondingemail{zender@ncar.ucar.edu} 
 
 
% SECTION. Indicate to which AGU section the
% abstract is being submitted.  Use one of the
% following letter abbreviations: 
% 
%    U  (Union) 
%    A  (Atmospheric Sciences) 
%    G  (Geodesy) 
%    GP (Geomagnetism/Paleomagnetism)
%    H   (Hydrology) 
%    O   (Ocean Sciences) 
%    P   (Planetology) 
%    S   (Seismology) 
%    SA (SPA Aeronomy) 
%    SH (SPA Solar and Heliospheric Physics) 
%    SM (SPA Magnetospheric Physics) 
%    T   (Tectonophysics) 
%    V   (Volcanology, Geochemistry, and 
%        Petrology) 
% 
% Section code is REQUIRED 
 
\sectionid{A} 
 
 
% SPECIAL SESSION.  Input the 3- to 4-character
% special session code to which the abstract is
% being submitted, if applicable.
% Please refer to the list of special sessions in
% the FM95INS.TXT instructions file, and use the 
% corresponding code contained in the braces
% "{}". 
 
\specialsessionid{A18} 
 
 
% INDEX TERMS.  Up to three (3) index terms may
% be specified that describe the topic of your
% paper within the section. 
% 
% Please refer to the index terms with
% descriptions listed in the FM95INS.TXT
% instructions file. 

\indexterm{3314} 
\indexterm{3319} 
\indexterm{3359} 
 
 
% POSTER PRESENTATION REQUESTS.  If you prefer
% to give a poster presentation, specify by using
% one of three abbreviations (P, PC, or PV):
% 
%    P    Poster presentation requested,
%        no audiovisual equipment requirements
%    PC   Poster presentation requested,
%        computer equipment required
%    PV   Poster presentation requested,
%        video equipment required
% 
% AGU will provide standard VCRs and computers
% (MAC- or IBM-compatible) equipment in the
% poster session area, free of charge*, for 
% individuals who request them.  List the type of
% equipment needed: IBM, MAC, Internet
% connection, or UNIX workstation, e.g., PC MAC
% or PC IBM. 
% *Those requiring more sophisticated equipment 
% may need to absorb these costs directly.
%
% If you prefer an oral presentation, please
% leave the \postertype command blank.
%
% PLEASE NOTE: that the submission of an abstract
% carries with it the obligation to present the
% paper in the session or in the mode of
% presentation (whether oral or poster) as
% assigned by the Program Committee.

\postertype{} 
 

% TITLE ONLY INDICATOR.  Put the abbreviation
% "TO" in the \titleonly braces if you want the
% abstract published (if accepted), but are
% unable to present the paper at the meeting. 
 
\titleonly{} 
 
 
% PREVIOUS SUBMISSION INFORMATION. Please give
% the percentage of material previously presented or
% published as a whole number, for instance "30"
% or "50"; do not use words like "half",
% and omit the "%" sign. Indicate where the
% material was presented using a meeting
% identification (e.g., "1993 Fall Meeting") or a
% journal name/abbreviation (e.g., "JGR Oceans"
% or "WRR"). 
 
\previoussubmissioninfo{0}{} 
 
 
% PAYMENT INFORMATION.  Payment must accompany
% the abstract submission. The only acceptable
% method of payment for electronic submission is
% a credit card.  Abstracts without an
% appropriate payment method will be
% AUTOMATICALLY REJECTED and returned to the
% author. 
% 
% Please specify the \paymentmethod command by
% using one of these three abbreviations:
% 
%    AMEX    American Express 
%    MC      MasterCard 
%    VISA    Visa 
% 
% The \cardholder command should contain the name
% as it appears on the credit card.
% 
% The \accountnumber command should contain the
% credit card account number exactly as it
% appears on the card, with corresponding spaces.
% 
% The \expirationdate command should contain the
% card's expiration date using the numeric
% equivalent of month and year, e.g., 02/97 (Be
% sure to use a forward slash "/" here).
% 
% All payment information commands are REQUIRED. 
 
\paymentmethod{VISA} 
\cardholder{Charles S Zender} 
\accountnumber{4414 2000 0017 1494} 
\expirationdate{03/97} 
 
 
% SUBMITTAL FEE AMOUNT.  The abstract submittal
% fee is determined by the actual length of the
% abstract (standard vs. extended) when in the
% appropriate submission format (refer to the
% 1995 Fall Meeting call for paper's Instructions
% for Preparation of Typewritten Abstract Copy
% for details) and the submitting author's
% membership category (regular vs. student). The
% student rate is applicable only when the first
% author is a student presenting his/her own
% paper.
% 
%       Maximum         Regular        Student 
%    abstract size     AGU member     AGU member 
% 
%    11.8 cm x 18 cm      $50            $30 
%    11.8 cm x 28 cm      $70            $40 
% 
% The submittal fee is NOT refundable (including
% duplicate submissions), and must be paid in
% U.S. dollars. 
% 
% Enter the dollar amount from the preceding
% table in the \paymentamount braces below.  DO
% NOT ENTER THE DOLLAR SIGN "$" AS PART OF YOUR
% RESPONSE.
% 
% The \paymentamount command is REQUIRED. 
 
\paymentamount{30} 
 
 
% ABSTRACT SUBMISSION STATUS.  Indicate whether
% this abstract is contributed or invited by
% entering one of these two codes in the
% \papertype command:
% 
%    C   for Contributed submissions
%    I   for Invited submissions 
% 
% If this abstract is invited, the \programchair
% command MUST contain the name of the Section
% Program Chairperson who invited the abstract
% submission.  If no name is provided, this
% abstract will be treated as a contributed
% submission.  Please note that each first author
% may only submit one contributed abstract and
% one invited abstract.  Invitation of any
% additional abstracts by the same first author
% requires prior approval by the Meeting
% Chairman.
 
\papertype{C} 
\programchair{} 
 
 
% SPECIAL INSTRUCTIONS.  If you have a special
% request, such as "schedule my paper before or
% after another," enter this information in the
% \specialinstructions braces.  Authors who are
% willing to preside over a session in their
% field should indicate their interest in this
% command, as well.  You will be contacted by the
% AGU, if selected. 
% 
% PLEASE NOTE:  Requests for papers following
% each other and having the same presenter
% generally will not be honored. 
 
\specialinstructions{I am willing to preside 
over a session.} 
 
 
% STUDENT PAPER.  If a student is listed as the
% first author and is presenting his/her own
% paper, enter in YES in the \studentauthor
% braces. Otherwise, leave this command blank.
 
\studentauthor{YES}
 
 
% SCIENTIFIC INFORMATION.  This is the text
% portion of the abstract.  It includes the 
% abstract's title, author information, and body.
% 
% TITLE: The \title command should contain the
% title of your abstract submission. Refer to the
% FM95INS.TXT file for specific instructions on
% how to correctly format the title of your
% abstract.
 
\title{A Prognostic Ice Water Scheme for Anvil
Clouds} 
 
% AUTHOR BLOCK: Authors and affiliations are  
% specified within the author command sequence 
% that is given in one of the two methods below. 
% An entire author command sequence will consist 
% of the following commands: 
 
% \author{} or \presentingauthor{} 
% \authoraddress{} 
% \authorphone{} 
% \authoremail{} 
 
% You may use either one or a combination of the 
% two methods given to compose your author block.
% Please enter the authors in the order that you 
% wish them to appear on your abstract.  Refer to
% the FM95INS.TXT instructions file for examples.
% 
% It is REQUIRED that your abstract have at least
% one author complete with address, phone, and 
% e-mail information (when applicable).
% 
% When composing your author block... 
% 
% a) be sure to remove the "%" from the front 
%    of the ENTIRE author command sequence that 
%    you plan to use.  
% 
% b) be sure to enter only one author's name per 
%    \author command. 
% 
% c) authors should supply only the punctuation
%    needed within an command item. Intervening
%    punctuation marks and identifying text will
%    be added automatically.
% 
% d) identify the presenting author of the
%    abstract by replacing the \author{} command,
%    where the presenting author is indicated,
%    with a \presentingauthor{} command.
 
% METHOD 1: FOR AUTHORS WITH DIFFERENT ADDRESSES.
% Remove the "%" sign from the front
% of the following commands, and replace the
% examples between the braces "{}" with
% your author's information.  Remember to
% identify the presenting author with the
% \presentingauthor command. 
 
%\author{A First} 
%\authoraddress{Inst, City, State, Postal Code} 
%\authorphone{000-111-2222} 
%\authoremail{afirst@host.domain} 
 
% If you need to add additional authors, remove
% the "%" sign from the front of the author
% command sequence below and complete as above.
% You may add complete author command sequences
% as needed. 

%\author{} 
%\authoraddress{} 
%\authorphone{} 
%\authoremail{} 
 
% METHOD 2: FOR AUTHORS WITH THE SAME ADDRESS.
% Remove the "%" sign from the front of the  
% commands below and add as many \author commands
% as needed to accommodate each author.  You may 
% use the \presentingauthor command here to
% denote the presenting author. 
 
\presentingauthor{Charles S. Zender} 
\author{J.T. Kiehl} 
\authoraddress{NCAR, P.O. Box 3000, Boulder, CO,
80307-3000} 
\authorphone{303-497-1612} 
\authoremail{zender@ncar.ucar.edu} 
 
 
% BODY OF ABSTRACT: The body of your abstract is 
% contained in a LaTeX abstract environment,
% which begins with: 
% 
%    \begin{abstract} command on a separate line 
%    above the body of your abstract, and ends
%    with: 
% 
%    \end{abstract} command on a separate line 
%    below the body of your abstract. 
% 
% When composing the body of the abstract,... 
% 
% a) be sure to limit the lines of your text to 
%    60 characters to avoid possible text
%    deletion. 
% 
% b) use 2 hard returns to separate paragraphs. 
%    Do not use the double backslash "\\" as hard
%    returns or line spaces.  The should only be
%    used in the \correspondingaddress command.
% 
% c) refer to the sections on "LaTeX COMMANDS FOR 
%    SELECTED SPECIAL SYMBOLS" and "PROBLEMS TO
%    BE AWARE OF" in the FM95INS.TXT instructions
%    file for helpful hints. 
 
\begin{abstract} 

We have employed a cumulus ensemble model (CEM) to
parameterize the mesoscale outflow associated with deep
convection. Three steps were necessary to parameterize
results from the CEM into an ice water scheme suitable for
a GCM. Statistics from the GCM and the CEM were compared to
evaluate differences in dynamic response to similar large
scale forcings so that the anvil parameterization could be
sensibly mapped from the high to the low-resolution model.
Second, time series of the domain-averaged cloud water,
vertical updraft, and thermodynamic profiles were evaluated
to identify the most promising large scale predictors of
ice water generation and destruction. Third, a mathematical
formulation for the GCM thermodynamic tendencies was
formalized to include the most important physical processes
deduced from the CEM. We have also devised a
parameterization of ice formation under stable conditions. 

The combined convective/stable
prognostic ice scheme has been implemented into a new
version of the NCAR Community Climate Model (CCM2). One
novel feature of the convective parameterization is that is
predicts gridbox-averaged ice, thus circumventing the use
of a cloud fraction. The GCM response to the new prognostic
ice scheme indicates increased upper tropospheric heating
and associated acceleration of tropical easterlies.
Regional analysis indicates that ice water paths may be too
large at present. Results will be presented from an AMIP-type
simulation to determine the sensitivity of the current model to
tropical warm and cold events.
 
\end{abstract} 
 
% The \end{document} line is REQUIRED.  DO NOT
% EDIT OR DELETE THIS COMMAND. 

\end{document} 
 
% You have now come to the end of the template.
%
% Check to be sure that all REQUIRED information
% has been input, AGU accepts no responsibility
% for abstract inaccuracies due to author error.
%
% It is strongly recommended that you refer to
% the "PROBLEMS TO BE AWARE OF" section of the
% FM95INS.TXT instructions file for an extra
% measure of assurance.
% 
% You can submit the abstract to the AGU office
% via e-mail at FM-SUBMIT@EARTH.AGU.ORG.
%
% TIP:  Include the title (or a recognizable
% portion of your title) as the SUBJECT LINE of 
% your e-mail message to function as a handy
% indexing item for the purpose of your
% acknowledgement.
%
% You will receive an acknowledgment of the
% submission electronically within 24 hours.  If
% you do not receive an acknowledgment within
% this time frame, please contact the AGU
% Meetings Department immediately.

% Purpose: Slides for the AGU talk

\documentclass[agupp]{aguplus}
\usepackage{graphicx}
\usepackage{csz}
\begin{document}

% Title Title Title Title Title Title Title Title
\begin{center}
\Large
\textbf{Parameterization of Anvil Clouds in the NCAR GCM}

\textbf{Charles S. Zender and J.T. Kiehl}

\textbf{University of Colorado and NCAR}

\end{center}

%\title{Parameterization of Anvil Clouds in the NCAR GCM}
%\author{Charles S. Zender\\University of Colorado and NCAR \and
%J.T. Kiehl\\ NCAR}
%\date
%\maketitle

\medskip

\begin{itemize}
\Large
\item Parameterization Description
\item Global Seasonal Cycle Sensitivity
\item Regional Sensitivity
\item Equatorial Pacific 
\item Changes in Cloud Feedback in 1987 El Ni\~no
\item Conclusions
\end{itemize}

\normalsize
\pagebreak

% Heurism Heurism Heurism Heurism Heurism Heurism Heurism Heurism

\begin{figure*}
\includegraphics[height=.5\textheight,clip=true]{anv_concept.eps}
%\caption{Conceptual model of the ice budget in the ANV scheme.\label{fig:anv_concept}}
\end{figure*}

\Large
ANV Scheme:
\begin{eqnarray}
{\partial \qi \over \partial t} & 
= & 
-\vec u \cdot \nabla \qi + 
{\alpha M_{\mathrm{c}} \over \rho \Delta Z} - 
\beta \qi - 
\gamma \qi \\
{\partial \qi \over \partial t} & 
= & 
-\vec u \cdot \nabla \qi + 
A_s\Bigl(
{\alpha_{\mathrm STB} w + \beta_{\mathrm STB} {\mathrm H}(T - 240)
\over \rho \Delta z} \Bigr) -
\beta \qi - 
\gamma \qi \\
{\partial \qv \over \partial t} &
= & 
-\vec u \cdot \nabla \qv - 
{\alpha \Mc \over \rho \Delta Z} + 
\beta \qi + 
\delta \Bigl(1 - {\qi \over
\qvi}\Bigr)P_{\mathrm{i}}^{1/2} \\
{\partial T \over \partial t} &
= &
 -\vec u \cdot \nabla T +
{\mbox{L}_{\mathrm{i}} \over {\mathrm c_p}}\Biggl[
{\alpha \Mc \over \rho \Delta Z} - 
\beta \qi - 
\delta \Bigl(1 - {\qi \over \qvi}\Bigr)P_{\mathrm{i}}^{1/2}
\Biggr]
\end{eqnarray}

CCM Scheme:
\begin{eqnarray}
\LWC & = & \LWC_0 \exp (-z/H) \\
H & \propto & \ln Q \\
Q & \propto & \int q dp
\end{eqnarray}

\normalsize
\pagebreak

% Conclusions Conclusions Conclusions Conclusions Conclusions Conclusions Conclusions Conclusions 
\begin{center}
\Large
\textbf{Conclusions} 
\end{center}

\medskip

\begin{itemize}
\Large
\item $w$, $T$, and $A_c$ suffice to grossly represent stable ice
generation at the GCM scale. 
\item A threshold convective mass flux \Mc\ (i.e., \Mc\ at lowest
convective freezing level above 500~mb) can work as a proxy for \IWP\
generation. 
\item \Mc\ should be interpolated between adjacent vertical levels
to help avoid spurious changes in \Mc\ due to coarse vertical
resolution. 
\item Cloud fraction may be implicitly parameterized, at least in
cirrus anvils, where patchiness is low.  
\item Negative cloud forcing feedbacks to \SST\ forcing in deep
convective regimes is consistent with anomalous \SWCF\ increase to a
saturated \LWCF\ basic state.
\item The \LWCF\ response to large \SST\ forcing in CCM 
(tightly coupled to \SST) was superior to ANV (tightly
coupled to \Mc) because freezing level \Mc\ does not trend well 
with \SST\ (in the ANV convection scheme).
\item The \SWCF\ response to large \SST\ forcing in ANV was superior
to CCM because the CCM condensate depends on the ALWC (logarithmically
increasing with temperature change) whereas ANV depends linearly on \Mc.
\end{itemize}

\normalsize
\pagebreak

\begin{figure*}
\includegraphics*[width=.5\hsize]{/data/zender/ps/erbe_b_anom_xavg_8589x87_0112_LWCF.eps}
\includegraphics*[width=.5\hsize]{/data/zender/ps/erbe_b_anom_xavg_8589x87_0112_SWCF.eps}

\includegraphics*[width=.5\hsize]{/data/zender/ps/amip5_anom_xavg_8589x87_0112_LWCF.eps}
\includegraphics*[width=.5\hsize]{/data/zender/ps/amip5_anom_xavg_8589x87_0112_SWCF.eps}

\includegraphics*[width=.5\hsize]{/data/zender/ps/spcp_81_anom_xavg_8589x87_0112_LWCF.eps}
\includegraphics*[width=.5\hsize]{/data/zender/ps/spcp_81_anom_xavg_8589x87_0112_SWCF.eps}
\caption{Seasonal amplitude in zonally averaged longwave and shortwave
cloud forcing (\wxmS ) for (a) ERBE, (b) CCM, and
(c) ANV. 
Contour intervals are 5 and 10~\wxmS, respectively. 
Month 1 is January. 
Data are from 1985, 1986, 1988, and 1989.
\label{fig:anom_xavg_8589x87_0112}}   
\end{figure*}

\begin{figure*}
\centering
\includegraphics*[height=.3\textheight]{/data/zender/ps/anom_xyavg_reg_Pacific_Equatorial_Eastern_8589x87_0112_LWCF.eps}%
\includegraphics*[height=.3\textheight]{/data/zender/ps/anom_xyavg_reg_Pacific_Equatorial_Eastern_8589x87_0112_SWCF.eps}%

\includegraphics*[height=.3\textheight]{/data/zender/ps/anom_xyavg_reg_Pacific_Western_Warm_Pool_8589x87_0112_LWCF.eps}%
\includegraphics*[height=.3\textheight]{/data/zender/ps/anom_xyavg_reg_Pacific_Western_Warm_Pool_8589x87_0112_SWCF.eps}%

\includegraphics*[height=.3\textheight]{/data/zender/ps/anom_xyavg_reg_Atlantic_North_8589x87_0112_LWCF.eps}%
\includegraphics*[height=.3\textheight]{/data/zender/ps/anom_xyavg_reg_Atlantic_North_8589x87_0112_SWCF.eps}%
\caption{Seasonal amplitude in regionally averaged longwave and
shortwave cloud forcing (\wxmS ) over ocean only for ERBE, CCM, and
ANV. 
Month 1 is January.  
Data are from 1985, 1986, 1988, and 1989.
\label{fig:anom_xyavg_reg_8589x87_0112_ocean}}   
\end{figure*}

\begin{figure*}
\centering
\includegraphics*[height=.3\textheight]{/data/zender/ps/anom_xyavg_reg_Africa_South_8589x87_0112_LWCF.eps}%
\includegraphics*[height=.3\textheight]{/data/zender/ps/anom_xyavg_reg_Africa_South_8589x87_0112_SWCF.eps}%

\includegraphics*[height=.3\textheight]{/data/zender/ps/anom_xyavg_reg_Amazon_Basin_8589x87_0112_LWCF.eps}%
\includegraphics*[height=.3\textheight]{/data/zender/ps/anom_xyavg_reg_Amazon_Basin_8589x87_0112_SWCF.eps}%

\includegraphics*[height=.3\textheight]{/data/zender/ps/anom_xyavg_reg_Siberia_Western_8589x87_0112_LWCF.eps}%
\includegraphics*[height=.3\textheight]{/data/zender/ps/anom_xyavg_reg_Siberia_Western_8589x87_0112_SWCF.eps}%
\caption{Seasonal amplitude in regionally averaged longwave and
shortwave cloud forcing (\wxmS ) over land only for ERBE, CCM, and
ANV. 
Month 1 is January.
Data are from 1985, 1986, 1988, and 1989.
\label{fig:anom_xyavg_reg_8589x87_0112_land}}   
\end{figure*}

\begin{figure*}
\centering
\includegraphics*[height=.95\textheight]{/data/zender/ps/amip5_anom_yavg_8589_0160_LWCF.eps}%
\includegraphics*[height=.95\textheight]{/data/zender/ps/spcp_81_anom_yavg_8589_0160_LWCF.eps}%
\caption{Hovm\"oller diagrams of meridionally averaged longwave cloud
forcing (\wxmS ) in the tropics (10~\degreee S--10~\degreee N) for
(a) CCM, and (b) ANV.
Month 1 is January 1985.  
\label{fig:anom_yavg_8589_0160_a}}
\end{figure*}

\begin{figure*}
\includegraphics*[width=.45\hsize]{/data/zender/ps/erbe_b_reg_Pacific_Equatorial_Western_87m85_0305_LWCF_SWCF.eps}
\includegraphics*[width=.45\hsize]{/data/zender/ps/erbe_b_reg_Pacific_Equatorial_Central_87m85_0305_LWCF_SWCF.eps}

\includegraphics*[width=.45\hsize]{/data/zender/ps/amip5_reg_Pacific_Equatorial_Western_87m85_0305_LWCF_SWCF.eps}
\includegraphics*[width=.45\hsize]{/data/zender/ps/amip5_reg_Pacific_Equatorial_Central_87m85_0305_LWCF_SWCF.eps}

\includegraphics*[width=.45\hsize]{/data/zender/ps/spcp_81_reg_Pacific_Equatorial_Western_87m85_0305_LWCF_SWCF.eps}
\includegraphics*[width=.45\hsize]{/data/zender/ps/spcp_81_reg_Pacific_Equatorial_Central_87m85_0305_LWCF_SWCF.eps}
\caption{Differences in monthly mean \LWCF\ and \SWCF\ over ocean between 1987
and 1985 for the three months March, April, and May in the Equatorial
Western and Central Pacific for (a) ERBE, (b) CCM, and
(c) ANV.
\label{fig:reg_Pacific_Equatorial_87m85_0305_LWCF_SWCF_a}}   
\end{figure*}

\begin{figure*}
\includegraphics*[width=.45\hsize]{/data/zender/ps/erbe_b_reg_Pacific_Equatorial_Eastern_87m85_0305_LWCF_SWCF.eps}
\includegraphics*[width=.45\hsize]{/data/zender/ps/erbe_b_reg_Pacific_Equatorial_87m85_0305_LWCF_SWCF.eps}

\includegraphics*[width=.45\hsize]{/data/zender/ps/amip5_reg_Pacific_Equatorial_Eastern_87m85_0305_LWCF_SWCF.eps}
\includegraphics*[width=.45\hsize]{/data/zender/ps/amip5_reg_Pacific_Equatorial_87m85_0305_LWCF_SWCF.eps}

\includegraphics*[width=.45\hsize]{/data/zender/ps/spcp_81_reg_Pacific_Equatorial_Eastern_87m85_0305_LWCF_SWCF.eps}
\includegraphics*[width=.45\hsize]{/data/zender/ps/spcp_81_reg_Pacific_Equatorial_87m85_0305_LWCF_SWCF.eps}
\caption{Differences in monthly mean \LWCF\ and \SWCF\ over ocean between 1987
and 1985 for the three months March, April, and May in the Equatorial
Eastern Pacific and Equatorial Pacific for (a) ERBE, (b) CCM, and (c)
ANV. 
\label{fig:reg_Pacific_Equatorial_87m85_0305_LWCF_SWCF_b}}   
\end{figure*}

\begin{figure*}
\includegraphics*[width=.45\hsize]{/data/zender/ps/amip5_reg_Pacific_Equatorial_Central_87m85_0305_LWCF_SWCF.eps}
\includegraphics*[width=.45\hsize]{/data/zender/ps/spcp_81_reg_Pacific_Equatorial_Central_87m85_0305_LWCF_SWCF.eps}

\includegraphics*[width=.45\hsize]{/data/zender/ps/amip5_reg_Pacific_Equatorial_Central_87m85_0305_TS1_LWCF.eps}
\includegraphics*[width=.45\hsize]{/data/zender/ps/spcp_81_reg_Pacific_Equatorial_Central_87m85_0305_TS1_LWCF.eps}

\includegraphics*[width=.45\hsize]{/data/zender/ps/amip5_reg_Pacific_Equatorial_Central_87m85_0305_TS1_TMQ.eps}
\includegraphics*[width=.45\hsize]{/data/zender/ps/spcp_81_reg_Pacific_Equatorial_Central_87m85_0305_TS1_CMFMC.eps}
\caption{Differences in monthly mean \SST\ and \LWCF\,
\SST\ and precipitable water, and \SST\ and \Mc\ over ocean between
1987 and 1985 for the three months March, April, and May in the
Equatorial Central Pacific (a) CCM, (b) ANV.
\label{fig:reg_Pacific_Equatorial_87m85_0305_TS1_CMFMC}}   
\end{figure*}

\begin{figure*}
\centering
\includegraphics*[height=.95\textheight]{/data/zender/ps/erbe_b_anom_yavg_8589_0160_LWCF.eps}%
\caption{Hovm\"oller diagram of meridionally averaged longwave cloud
forcing (\wxmS ) in the tropics (10~\degreee S--10~\degreee N) for
ERBE.
Month 1 is January 1985.  
\label{fig:anom_yavg_8589_0160_b}}   
\end{figure*}

\end{document}

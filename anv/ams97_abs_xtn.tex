% $Id$

% Purpose: Extended abstract for 1997 AMS mtg.

% cd ~/sw/anv;make -W ams97_abs_xtn.tex ams97_abs_xtn.pdf;cd -

\documentclass[twocolumn,twoside,final,10pt]{article}

%%\ifphd{}{
%
%\documentclass[agupp]{aguplus}
%%\documentclass[agums]{aguplus}
%%\documentclass[jgrga]{aguplus}
%
%%  AGU++ OPTIONS
%\printfigures        % ADDS FIGURES AT END
%%\doublecaption{35pc} % CAPTIONS PRINTED TWICE
%\sectionnumbers      % TURNS ON SECTION NUMBERS
%%\extraabstract       % ADDS SUPPLEMENTAL ABSTRACT
%%\afour               % EUROPEAN A4 PAPER SIZE
%%\figmarkoff          % SUPPRESS MARGINAL MARKINGS
%
%%  AGUTeX OPTIONS AND ENTRIES
%%\tighten             % TURNS OFF DOUBLE SPACING, has no effect with agupp
%%\singlespace         % RESTORES SINGLE SPACING
%%\doublespace         % RESTORES DOUBLE SPACING

\usepackage{etoolbox} % \newbool, \setbool, \ifxxx
\usepackage{graphicx} % defines \includegraphics*
%\usepackage{ifthen} % Boolean and programming commands
\usepackage{natbib} % \cite commands from aguplus
%\usepackage{longtable} % for long tables, like acronyms and symbols

\usepackage{csz} % Library of personal definitions
\usepackage{abc} % Alphabet as three letter macros
\usepackage{dmn} % Dimensional units
%\usepackage{jrn_abb} % Define abbreviations used in bib.bib
%% Usage: % Usage: % Usage: \input{jgr_abb} % AGU-sanctioned journal title abbreviations

\def\aapgb{{\it Amer. Assoc. Petroleum Geologists Bull.}}
\def\adg{{\it Adv. Geophys.}}
\def\ajs{{\it Amer. J. Sci.}}
\def\amb{{\it Ambio}}
\def\amgb{{\it Arch. Meteorol. Geophys. Bioclimatl.}}
\def\ang{{\it Ann. Glaciol.}}
\def\angeo{{\it Ann. Geophys.}}
\def\apo{{\it Appl. Opt.}}
\def\areps{{\it Ann. Rev. Earth Planet. Sci.}}
\def\asr{{\it Adv. Space Res.}}
\def\ate{{\it Atmos. Environ.}}
\def\atf{{\it Atmosfera}}
\def\atms{{\it ACM Trans. Math Software}}
\def\ato{{\it Atmos. Ocean}}
\def\atr{{\it Atmos. Res.}}
\def\gbc{{\it Global Biogeochem. Cycles}} % csz
\def\blm{{\it Boundary-Layer Meteorol.}} % csz 
\def\bpa{{\it Beitr. Phys. Atmosph.}}
\def\bams{{\it Bull. Am. Meteorol. Soc.}}
\def\clc{{\it Clim. Change}}
\def\cld{{\it Clim. Dyn.}}
\def\com{{\it Computing}}
\def\dao{{\it Dyn. Atmos. Oceans}}
\def\dsr{{\it Deep-Sea Res.}}
\def\esr{{\it Earth Sci. Revs.}}
\def\gec{{\it Geosci. Canada}}
\def\gei{{\it Geofis. Int.}}
\def\gej{{\it Geogr. J.}}
\def\gem{{\it Geophys. Monogr.}}
\def\geo{{\it Geology}}
\def\grl{{\it Geophys. Res. Lett.}}
\def\ieeec{{\it IEEE Computer}}
\def\ijna{{\it IMA J. Numer. Anal.}}
\def\ijnmf{{\it Int. J. Num. Meteorol. Fl.}}
\def\jac{{\it J. Atmos. Chem.}}
\def\jacm{{\it J. Assoc. Comput. Mach.}}
\def\jam{{\it J. Appl. Meteorol.}}
\def\jas{{\it J. Atmos. Sci.}}
\def\jatp{{\it J. Atmos. Terr. Phys.}}
\def\jcam{{\it J. Climate Appl. Meteorol.}}
\def\jchp{{\it J. Chem Phys.}}
\def\jcis{{\it J. Coll. I. Sci.}}
\def\jcl{{\it J. Clim.}}
\def\jcp{{\it J. Comput. Phys.}}
\def\jfm{{\it J. Fluid Mech.}}
\def\jgl{{\it J. Glaciol.}}
\def\jgr{{\it J. Geophys. Res.}}
\def\jgs{{\it J. Geol. Soc. London}}
\def\jme{{\it J. Meteorol.}}
\def\jmr{{\it J. Marine Res.}}
\def\jmsj{{\it J. Meteorol. Soc. Jpn.}}
\def\josa{{\it J. Opt. Soc. A}}
\def\jpo{{\it J. Phys. Oceanogr.}}
\def\jqsrt{{\it J. Quant. Spectrosc. Radiat. Transfer}}
\def\jpca{{\it J. Phys. Chem. A}}
\def\lnc{{\it Lett. Nuov. C}}
\def\mac{{\it Math. Comp.}}
\def\map{{\it Meteorol. Atmos. Physics.}}
\def\mem{{\it Meteorol. Mag.}}
\def\mnras{{\it Mon. Not. Roy. Astron. Soc.}} 
\def\mwr{{\it Mon. Weather Rev.}} 
\def\nat{{\it Nature}}
\def\pac{{\it Parallel Computing}}
\def\pag{{\it Pure Appl. Geophys.}}
\def\pal{{\it Paleoceanography}}
\def\pht{{\it Physics Today}}
\def\pieee{{\it Proc. IEEE}}
\def\pla{{\it Phys. Lett. A}}
\def\ppp{{\it Paleogeogr. Paleoclim. Paleoecol.}}
\def\pra{{\it Phys. Res. A}}
\def\prd{{\it Phys. Rev. D}}
\def\prl{{\it Phys. Rev. L}}
\def\pss{{\it Planet. Space Sci.}}
\def\ptrsl{{\it Phil. Trans. R. Soc. Lond.}}
\def\qjrms{{\it Q. J. R. Meteorol. Soc.}}
\def\qres{{\it Quat. Res.}}
\def\qsr{{\it Quatern. Sci. Rev.}}
\def\reg{{\it Rev. Geophys.}}
\def\rgsp{{\it Revs. Geophys. Space Phys.}}
\def\rpp{{\it Rep. Prog. Phys.}}
\def\sca{{\it Sci. Amer.}}
\def\sci{{\it Science}}
\def\sjna{{\it SIAM J. Numer. Anal.}}
\def\sjssc{{\it SIAM J. Sci. Stat. Comput.}}
\def\tac{{\it Theor. Appl. Climatl.}}
\def\tel{{\it Tellus}}
\def\wea{{\it Weather}}

%SIAM Review: (Society for Industrial and Applied Mathematics)
%       J. on Computing
%       J. on Control and Optimization
%       J. on Algebraic and Discrete Methods
%       J. on Numerical Analysis 
%       J. on Scientific and Statistical Computing



 % AGU-sanctioned journal title abbreviations

\def\aapgb{{\it Amer. Assoc. Petroleum Geologists Bull.}}
\def\adg{{\it Adv. Geophys.}}
\def\ajs{{\it Amer. J. Sci.}}
\def\amb{{\it Ambio}}
\def\amgb{{\it Arch. Meteorol. Geophys. Bioclimatl.}}
\def\ang{{\it Ann. Glaciol.}}
\def\angeo{{\it Ann. Geophys.}}
\def\apo{{\it Appl. Opt.}}
\def\areps{{\it Ann. Rev. Earth Planet. Sci.}}
\def\asr{{\it Adv. Space Res.}}
\def\ate{{\it Atmos. Environ.}}
\def\atf{{\it Atmosfera}}
\def\atms{{\it ACM Trans. Math Software}}
\def\ato{{\it Atmos. Ocean}}
\def\atr{{\it Atmos. Res.}}
\def\gbc{{\it Global Biogeochem. Cycles}} % csz
\def\blm{{\it Boundary-Layer Meteorol.}} % csz 
\def\bpa{{\it Beitr. Phys. Atmosph.}}
\def\bams{{\it Bull. Am. Meteorol. Soc.}}
\def\clc{{\it Clim. Change}}
\def\cld{{\it Clim. Dyn.}}
\def\com{{\it Computing}}
\def\dao{{\it Dyn. Atmos. Oceans}}
\def\dsr{{\it Deep-Sea Res.}}
\def\esr{{\it Earth Sci. Revs.}}
\def\gec{{\it Geosci. Canada}}
\def\gei{{\it Geofis. Int.}}
\def\gej{{\it Geogr. J.}}
\def\gem{{\it Geophys. Monogr.}}
\def\geo{{\it Geology}}
\def\grl{{\it Geophys. Res. Lett.}}
\def\ieeec{{\it IEEE Computer}}
\def\ijna{{\it IMA J. Numer. Anal.}}
\def\ijnmf{{\it Int. J. Num. Meteorol. Fl.}}
\def\jac{{\it J. Atmos. Chem.}}
\def\jacm{{\it J. Assoc. Comput. Mach.}}
\def\jam{{\it J. Appl. Meteorol.}}
\def\jas{{\it J. Atmos. Sci.}}
\def\jatp{{\it J. Atmos. Terr. Phys.}}
\def\jcam{{\it J. Climate Appl. Meteorol.}}
\def\jchp{{\it J. Chem Phys.}}
\def\jcis{{\it J. Coll. I. Sci.}}
\def\jcl{{\it J. Clim.}}
\def\jcp{{\it J. Comput. Phys.}}
\def\jfm{{\it J. Fluid Mech.}}
\def\jgl{{\it J. Glaciol.}}
\def\jgr{{\it J. Geophys. Res.}}
\def\jgs{{\it J. Geol. Soc. London}}
\def\jme{{\it J. Meteorol.}}
\def\jmr{{\it J. Marine Res.}}
\def\jmsj{{\it J. Meteorol. Soc. Jpn.}}
\def\josa{{\it J. Opt. Soc. A}}
\def\jpo{{\it J. Phys. Oceanogr.}}
\def\jqsrt{{\it J. Quant. Spectrosc. Radiat. Transfer}}
\def\jpca{{\it J. Phys. Chem. A}}
\def\lnc{{\it Lett. Nuov. C}}
\def\mac{{\it Math. Comp.}}
\def\map{{\it Meteorol. Atmos. Physics.}}
\def\mem{{\it Meteorol. Mag.}}
\def\mnras{{\it Mon. Not. Roy. Astron. Soc.}} 
\def\mwr{{\it Mon. Weather Rev.}} 
\def\nat{{\it Nature}}
\def\pac{{\it Parallel Computing}}
\def\pag{{\it Pure Appl. Geophys.}}
\def\pal{{\it Paleoceanography}}
\def\pht{{\it Physics Today}}
\def\pieee{{\it Proc. IEEE}}
\def\pla{{\it Phys. Lett. A}}
\def\ppp{{\it Paleogeogr. Paleoclim. Paleoecol.}}
\def\pra{{\it Phys. Res. A}}
\def\prd{{\it Phys. Rev. D}}
\def\prl{{\it Phys. Rev. L}}
\def\pss{{\it Planet. Space Sci.}}
\def\ptrsl{{\it Phil. Trans. R. Soc. Lond.}}
\def\qjrms{{\it Q. J. R. Meteorol. Soc.}}
\def\qres{{\it Quat. Res.}}
\def\qsr{{\it Quatern. Sci. Rev.}}
\def\reg{{\it Rev. Geophys.}}
\def\rgsp{{\it Revs. Geophys. Space Phys.}}
\def\rpp{{\it Rep. Prog. Phys.}}
\def\sca{{\it Sci. Amer.}}
\def\sci{{\it Science}}
\def\sjna{{\it SIAM J. Numer. Anal.}}
\def\sjssc{{\it SIAM J. Sci. Stat. Comput.}}
\def\tac{{\it Theor. Appl. Climatl.}}
\def\tel{{\it Tellus}}
\def\wea{{\it Weather}}

%SIAM Review: (Society for Industrial and Applied Mathematics)
%       J. on Computing
%       J. on Control and Optimization
%       J. on Algebraic and Discrete Methods
%       J. on Numerical Analysis 
%       J. on Scientific and Statistical Computing



 % AGU-sanctioned journal title abbreviations

\def\aapgb{{\it Amer. Assoc. Petroleum Geologists Bull.}}
\def\adg{{\it Adv. Geophys.}}
\def\ajs{{\it Amer. J. Sci.}}
\def\amb{{\it Ambio}}
\def\amgb{{\it Arch. Meteorol. Geophys. Bioclimatl.}}
\def\ang{{\it Ann. Glaciol.}}
\def\angeo{{\it Ann. Geophys.}}
\def\apo{{\it Appl. Opt.}}
\def\areps{{\it Ann. Rev. Earth Planet. Sci.}}
\def\asr{{\it Adv. Space Res.}}
\def\ate{{\it Atmos. Environ.}}
\def\atf{{\it Atmosfera}}
\def\atms{{\it ACM Trans. Math Software}}
\def\ato{{\it Atmos. Ocean}}
\def\atr{{\it Atmos. Res.}}
\def\gbc{{\it Global Biogeochem. Cycles}} % csz
\def\blm{{\it Boundary-Layer Meteorol.}} % csz 
\def\bpa{{\it Beitr. Phys. Atmosph.}}
\def\bams{{\it Bull. Am. Meteorol. Soc.}}
\def\clc{{\it Clim. Change}}
\def\cld{{\it Clim. Dyn.}}
\def\com{{\it Computing}}
\def\dao{{\it Dyn. Atmos. Oceans}}
\def\dsr{{\it Deep-Sea Res.}}
\def\esr{{\it Earth Sci. Revs.}}
\def\gec{{\it Geosci. Canada}}
\def\gei{{\it Geofis. Int.}}
\def\gej{{\it Geogr. J.}}
\def\gem{{\it Geophys. Monogr.}}
\def\geo{{\it Geology}}
\def\grl{{\it Geophys. Res. Lett.}}
\def\ieeec{{\it IEEE Computer}}
\def\ijna{{\it IMA J. Numer. Anal.}}
\def\ijnmf{{\it Int. J. Num. Meteorol. Fl.}}
\def\jac{{\it J. Atmos. Chem.}}
\def\jacm{{\it J. Assoc. Comput. Mach.}}
\def\jam{{\it J. Appl. Meteorol.}}
\def\jas{{\it J. Atmos. Sci.}}
\def\jatp{{\it J. Atmos. Terr. Phys.}}
\def\jcam{{\it J. Climate Appl. Meteorol.}}
\def\jchp{{\it J. Chem Phys.}}
\def\jcis{{\it J. Coll. I. Sci.}}
\def\jcl{{\it J. Clim.}}
\def\jcp{{\it J. Comput. Phys.}}
\def\jfm{{\it J. Fluid Mech.}}
\def\jgl{{\it J. Glaciol.}}
\def\jgr{{\it J. Geophys. Res.}}
\def\jgs{{\it J. Geol. Soc. London}}
\def\jme{{\it J. Meteorol.}}
\def\jmr{{\it J. Marine Res.}}
\def\jmsj{{\it J. Meteorol. Soc. Jpn.}}
\def\josa{{\it J. Opt. Soc. A}}
\def\jpo{{\it J. Phys. Oceanogr.}}
\def\jqsrt{{\it J. Quant. Spectrosc. Radiat. Transfer}}
\def\jpca{{\it J. Phys. Chem. A}}
\def\lnc{{\it Lett. Nuov. C}}
\def\mac{{\it Math. Comp.}}
\def\map{{\it Meteorol. Atmos. Physics.}}
\def\mem{{\it Meteorol. Mag.}}
\def\mnras{{\it Mon. Not. Roy. Astron. Soc.}} 
\def\mwr{{\it Mon. Weather Rev.}} 
\def\nat{{\it Nature}}
\def\pac{{\it Parallel Computing}}
\def\pag{{\it Pure Appl. Geophys.}}
\def\pal{{\it Paleoceanography}}
\def\pht{{\it Physics Today}}
\def\pieee{{\it Proc. IEEE}}
\def\pla{{\it Phys. Lett. A}}
\def\ppp{{\it Paleogeogr. Paleoclim. Paleoecol.}}
\def\pra{{\it Phys. Res. A}}
\def\prd{{\it Phys. Rev. D}}
\def\prl{{\it Phys. Rev. L}}
\def\pss{{\it Planet. Space Sci.}}
\def\ptrsl{{\it Phil. Trans. R. Soc. Lond.}}
\def\qjrms{{\it Q. J. R. Meteorol. Soc.}}
\def\qres{{\it Quat. Res.}}
\def\qsr{{\it Quatern. Sci. Rev.}}
\def\reg{{\it Rev. Geophys.}}
\def\rgsp{{\it Revs. Geophys. Space Phys.}}
\def\rpp{{\it Rep. Prog. Phys.}}
\def\sca{{\it Sci. Amer.}}
\def\sci{{\it Science}}
\def\sjna{{\it SIAM J. Numer. Anal.}}
\def\sjssc{{\it SIAM J. Sci. Stat. Comput.}}
\def\tac{{\it Theor. Appl. Climatl.}}
\def\tel{{\it Tellus}}
\def\wea{{\it Weather}}

%SIAM Review: (Society for Industrial and Applied Mathematics)
%       J. on Computing
%       J. on Control and Optimization
%       J. on Algebraic and Discrete Methods
%       J. on Numerical Analysis 
%       J. on Scientific and Statistical Computing





%% NB: The \lefthead and \righthead will be automatically uppercased by
%% the jgrga documentstyle 
%\lefthead{Zender and Kiehl}
%\righthead{Climate sensitivity to tropical anvil representation}
%\received{date1}
%\revised{date2}
%\accepted{date3}
%\journalid{JGRD}{Journal date}
%\articleid{page1}{page2}
%\paperid{94JZ12345}
%% The $ in the following line screws up the hilit19 highlighting
%%\ccc{0000-0000/00/94JZ-12345\$05.00}
%\ccc{0000-0000/00/94JZ-12345\05.00}
%% \cpright{PD}{1994}
%% \cpright{Crown}{1994}
%% (No \ccc{} for Crown copyrights.)
%\cpright{AMS}{1996}
%
%\authoraddr{J. T. Kiehl and Charles S. Zender, National Center for Atmospheric
%Research, P.O. Box 3000, Boulder, CO 80307-3000}

%\slugcomment{Submitted to \jas, April~22, 1996. Copyright \copyright 1996 AMS.}
%\slugcomment{\today}

\oddsidemargin = 0.0in
\evensidemargin = 0.0in
\topmargin = 0.0in
\textheight = 9in
\textwidth = 6.5in
\columnsep = 0.25in
\headheight = 0in
\headsep = 0in
\footskip = 0in
\pagestyle{empty}

\renewcommand\textfraction{0.}
\setcounter{totalnumber}{10}
\setcounter{topnumber}{10}
\setcounter{dbltopnumber}{10}
\setcounter{bottomnumber}{10}
\renewcommand\topfraction{1.}
\renewcommand\dbltopfraction{1.}
\renewcommand\bottomfraction{1.}
\renewcommand\floatpagefraction{1.}
\renewcommand\dblfloatpagefraction{1.}

% latex companion p. 26
%\renewcommand{\section}{\@startsection {section}{1}{\z@}{-\baselineskip}{0.5\baselineskip}{\normalfont\normalsize}}
%\renewcommand{\subsection}{\@startsection {subsection}{2}{\z@}{-\baselineskip}{0.5\baselineskip}{\normalfont\normalsize\itshape\underline}}
% from /contrib/texmf/lib/texmf/tex/latex2e/base/article.cls
\makeatletter
%\renewcommand{\section}{\@startsection {section}{1}{\z@}{-3.5ex \@plus -1ex \@minus -.2ex}{2.3ex \@plus.2ex}{\normalfont\normalsize}}
%\renewcommand{\bibsection}{\@startsection
%{section}{1}{\z@}{-3.5ex\@plus -1ex \@minus -.2ex}{2.3ex
%\@plus.2ex}{\normalfont\normalsize References}} 
\makeatother

\begin{document}
%} % not phdcsz

% NB: title must be lowercased by hand. The documentstyle does not enforce it.
%\def\paperchaptertitle{SENSITIVITY OF A CLIMATE SIMULATION TO THE
%RADIATIVE EFFECTS OF TROPICAL CIRRUS ANVIL CLOUD}
%\ifphd{\chapter{\paperchaptertitle}\label{chap:anv}}{\title{\paperchaptertitle}}
%\ifphd{}{\author{Charles S. Zender and J. T. Kiehl}}
%\ifphd{}{\affil{National Center for Atmospheric Research, Boulder, Colorado}}

%\ifphd{\section{Abstract}}{\begin{abstract}}
%Rewrite of thesis.
%\ifphd{}{\end{abstract}}

% These don't seem to work before \begin{document}
\setlength\abovecaptionskip{0pt}
\setlength\belowcaptionskip{15pt}

\setlength\floatsep{0pt}
\setlength\textfloatsep{0pt}
\setlength\dblfloatsep{0pt}
\setlength\dbltextfloatsep{0pt}
\setlength\intextsep{0pt}

\newcommand{\amssection}[1]{\vskip 1.0\baselineskip\normalfont\normalsize\noindent{#1}\vskip 1.0\baselineskip}
\newcommand{\amssubsection}[2]{\vskip
1.0\baselineskip\normalfont\normalsize\noindent{#1} \underline{\textsl{#2}}\vskip 1.0\baselineskip}
\renewcommand{\footnotesize}{\normalsize}
\makeatletter
\renewcommand{\refname}{\vskip -1.0\baselineskip\@empty}
\global\let\@date\@empty
\makeatother

\def\paperchaptertitle{TROPICAL CLIMATE SENSITIVITY TO REPRESENTATION\\
OF CIRRUS ANVIL LIFECYCLE}  
\def\paperauthor{Charles S. Zender\footnote{\textsl{Corresponding author address:}
Charles S. Zender, National Center for Atmospheric Research,
P.O.B. 3000, Boulder, CO 80307-3000; zender@ncar.ucar.edu}~~and
J. T. Kiehl}
\def\institution{National Center for Atmospheric Research, Boulder, Colorado}
%\large{\paperchaptertitle}1.0\baselineskip\large{\paperauthor}
\title{\vskip -.3in\normalsize{\begin{flushleft}\textbf{P1.24}\end{flushleft}}\vskip -1.0\baselineskip\paperchaptertitle}
\author{\normalsize\paperauthor\\
\normalsize\institution}
%{\centering
%\noindent\textbf{P1.24} \paperchaptertitle\relax\vskip 2.0\baselineskip
%Charles S. Zender\footnote{\textsl{Corresponding author address:}
%Charles S. Zender, National Center for Atmospheric Research,
%P.O.B. 3000, Boulder, CO 80307-3000; zender@ncar.ucar.edu} and
%J. T. Kiehl\par\relax 
%\institution\relax
%\vskip 2.0\baselineskip}
\maketitle
\thispagestyle{empty}

\amssection{1. INTRODUCTION}\label{sec:anv_intro}

Radiative forcing from the extended tropical upper tropospheric cloud
known as cirrus anvil plays a dominant role in determining
the diabatic heating which drives the general circulation. 
Tropical cirrus anvil originates in the complex interaction of a
mesoscale convective system (MCS) with the environment.
A general circulation model (GCM) does not resolve this interaction
and must rely on sub-gridscale methods to diagnose or predict anvil
cloud. 
This paper presents results from a new, semi-empirical method of
predicting anvils in GCMs based on observed and modeled behavior of
tropical MCSs.
The new method is semi-empirical in that it combines constraints on
anvil formation deduced from empirical budget studies with physics
parameterized from a cloud resolving cumulus ensemble model.
The method emphasizes the following observed and modeled features of
anvil lifecycle: a strong correlation between anvil growth and
convective mass flux at anvil base, the excess of ice over supercooled
liquid for $T < -5$~\dgrc, and the vertical distribution of
condensate.  
Five year integrations of the National Center for Atmospheric Research
Community Climate Model (CCM2) show climate features sensitive to
anvil representation include tropical upper troposphere temperature
structure, Hadley cell strength, and warm pool convection.
%, and the North American flow field in winter.

\amssection{2. ANVIL PARAMETERIZATION}\label{sec:anv_mdl}

%Empirical budget studies suggest the hydrologic cycle of tropical
%anvil may be conceptually divided into convective and stratiform
%components \cite[]{LeH80,GaH83}. 
%Convective formation accounts for ice formed in deep convective 
%updrafts and detrained into the stratiform anvil.
%Stratiform formation accounts for ice formed outside convective
%cores, such as condensate formed or frozen in large-scale ascent. 
%The above studies indicate roughly 60--75\% of anvil condensate
%originates in convective cores.
%The remainder, roughly 25--40\% of anvil mass, is generated by
%circulations outside the deep convective core. 

Spatio-temporal scale mismatch between convective and stratiform
processes and GCM resolution makes anvil parameterization difficult.
GCMs currently employ two common methods to represent these processes:
(i) diagnosing anvil cloud from column thermodynamic properties (e.g.,
relative humidity) and (ii) prognosing anvil cloud by assuming an
anvil detrainment efficiency which acts on the convective mass flux
predicted by the moist convection scheme.
Diagnostic methods like (i) have difficulty representing
convective-radiative hysteresis, such as the radiative influence of 
detached anvils.
As \cite{Don93} points out, prognostic methods like (ii) often do not
explicitly account for the 25--40\% of anvil mass formed by secondary
circulations outside the deep convective core.

Cloud resolving models and mesoscale budget studies suggest the
lifecycle of MCS anvils may be simply parameterized in terms of 
large scale forcing.  
Our modified version of the \cite{LeH80} conceptual model
is shown in Figure~\ref{fig:cem}a. 
\begin{figure}
\begin{center}
\includegraphics*[width=\halfwidth]{/Users/zender/data/fgr/anv/anv_concept.eps}\vfill
\vskip .1in
\includegraphics*[width=\halfwidth,height=.25\textheight]{/Users/zender/data/fgr/anv/cem_mc_diwpdt_c.eps}\vfill
\end{center}
\caption{
(a) Conceptual model of the ice budget in ANV. 
$c_1$--$c_5$ and \pc\ are the free parameters of the scheme.
(b) Linear correlation and least squares fit of \Mc\ to \IWPdot. 
The ordinate is scaled in units of a typical GCM timestep.
\label{fig:cem}}
\end{figure}
This conceptual anvil model, denoted ANV, is implemented as follows: 
For a grid cell of density $\rho$ and ice mixing ratio \qi\ located in
a convecting column with convective mass flux at 500~mb \Mc,
the ice budget that defines ANV is:
\begin{equation}
\label{eqn:qi_cnv}
{D\qi \over Dt} = {c_1 \Mc \over \rho \Delta Z} - c_2 \qi - c_3 \qi
\end{equation}
where $\Delta Z$ is the thickness of the convecting portion of the
column in which $T < 0~^\circ$C.
The material derivative on the LHS accounts for advection.
The first term on the RHS relates the generation of total column ice
to \Mc.
Basing the generation of \qi\ throughout the anvil on the mass flux
near anvil base \Mc\ (rather than local $M$) produces a vertical
profile of \qi\ which increases or remains constant (rather than
decreasing) from anvil base up to $\sim 300$~mb, in accord with
current understanding \cite[e.g.,][]{Hou89,WSS931}.
The $c_2$ term represents local sublimation of the anvil due to 
sub-gridscale entrainment and subsaturation.
The $c_3$ term converts ice to precipitation.
The determination of $c_1$ is discussed below.
$c_2$ and $c_3$ are determined from observations \cite[]{LeH80,GaH83}
and process studies in a microphysical cloud model \cite[]{ZeK94}.
%We assume detraining anvil completely covers a gridcell, i.e., cloud  
%fraction is 1. 

\amssubsection{2.1}{Linking Anvil Growth to Anvil-base Mass Flux}

Prior studies indicate convective mass flux $M$ is the best single
parameter to characterize the formation of tropical anvil. 
\cite{XuK91} concluded $M$ best predicts tropical convective cloud
amount and the ice water content of individual anvil layers.   
Based on satellite observations, \cite{MaR93} suggested convective
mass flux at the base of cumulonimbus cores determines the mean cloud 
properties of mesoscale convective systems, including the stratiform
anvil region.

We use a cloud resolving, cumulus ensemble model (CEM) to provide a
high spatial and temporal resolution dataset which spans the range of
MCS activity from the convective to the GCM scale.
A comprehensive review of our CEM simulation is presented by
\cite{GMK95}.  
Figure~\ref{fig:cem}b shows the correlation of \Mc\
and anvil mass growth rate \IWPdot\ through the first 2.5~hr.\ of the
CEM simulation.    
This initial correlation is excellent, but anvil decay processes and
scattered convection with the CEM domain soon cause the correlation to
deteriorate.
The slope of the least squares fit between between \Mc\ and \IWPdot\  
determines the source parameter $c_1$ in (\ref{eqn:qi_cnv}).
The results of our GCM simulations with constant $c_1$ (below) show
numerous improvements in anvil climatology over a more traditional
method. 
Parameterizing $c_1$ from an ensemble of CEM integrations, perhaps as
a function of large scale forcing (e.g., wind shear, SST) is the next
logical step.

\amssubsection{2.2}{Ice Fraction}\label{sec:fice}

Recent field observations and models \cite[e.g.,][]{SLT94,GMK95,GrM96}
suggest that, above 500~mb, anvil condensate is dominated by ice.
In terms of temperature, the complete phase transition may occur over
only 5~\dgrk.  
This behavior is imposed on the GCM experiment in the present
sensitivity study.
The control model, denoted CCM, partitions diagnostic condensate
between liquid and ice via   
\begin{equation}
\ficeccm = \cases{
  0 & $T > - 10$~\dgrc \cr
  -{T+10 \over 20} & $-10 \geq T \geq -30$~\dgrc \cr
  1 & $T < -30$~\dgrc \cr}
\label{eqn:ficeccm}
\end{equation}
The experiment, denoted ANV, restricts diagnostic liquid condensate
(and, hence, mixed phase cloud) to a narrower and warmer range than
CCM (\ref{eqn:ficeccm}) according to
\begin{equation}
\ficeanv = \cases{
  0 & $T > 0$~\dgrc \cr
  -{T \over 5} & $0 \geq T \geq -5$~\dgrc \cr
  1 & $T < -5$~\dgrc \cr}
\label{eqn:ficeanv}
\end{equation}
Since CCM classifies some condensate as cold as $-30$\,\dgrc\ as
liquid (\ref{eqn:ficeccm}), while ANV has no liquid colder than
$-5$~\dgrc (\ref{eqn:ficeanv}), it is clear the ANV atmosphere
will contain much more ice than the CCM. 
Since ice hydrometeors are prescribed to be larger than liquid, classifying
more condensate as ice (\ref{eqn:ficeanv}) also decreases the
extinction efficiency per unit mass of anvil in ANV.
This proves to be an important factor in diagnosing the cause of
change in climatological cloud radiative properties in this
sensitivity study.

\amssection{3. SENSITIVITY STUDY RESULTS}\label{sec:dsc}

To assess climate sensitivity to the features of anvil lifecycle
described above, we employed a version of the National Center for
Atmospheric Research Community Climate Model (CCM2). 
Two numerical climate integrations are forced by observed 1985--1989
SST: the control, denoted CCM, with diagnostic anvil, and the 
experiment, denoted ANV, with anvil ice forecast from
(\ref{eqn:qi_cnv}), which incorporates modeled and observed
characteristics of tropical anvil lifecycle (i.e., explicitly linking
anvil condensate generation to anvil base convective mass flux,
strong vertical profiles of condensate up to 300~mb, and increased ice
fraction).
Although the results focus on ensemble mean January, the simulations
show very similar features in July.
%We stress the control model (CCM) includes diagnostic anvil---the
%sensitivity experiment (ANV) shows the impact of forecasting anvil
%based on the physics of tropical anvil.

\begin{figure}
\begin{center}
\includegraphics*[width=\halfwidth]{/Users/zender/data/fgr/anv/spcp_85_8589_amip5_8589_xavg_01_QC.eps}\vfill
\includegraphics*[width=\halfwidth]{/Users/zender/data/fgr/anv/spcp_85_8589_amip5_8589_xavg_01_QDIABAT.eps}\vfill
\includegraphics*[width=\halfwidth]{/Users/zender/data/fgr/anv/spcp_85_8589_amip5_8589_xavg_01_T.eps}\vfill
\end{center}
\caption{
Change (ANV$-$CCM) in zonal average tropical thermodynamics due to
improved anvil representation.
Panels are ensemble averages of 5 simulated Januarys from 1985--1989.
Shading indicates values $< 0$.
(a) Condensate mixing ratio \qc\ (\mgxkg), contour interval is
2~\mgxkg.
(b) Diabatic heating \QT\ (\kxday). Contour interval is .1~\kxday.  
(c) Temperature $T$ (\dgrk), contour interval is .1~\dgrk.
\label{fig:xavg_8589}}   
\end{figure}

\begin{figure}
\begin{center}
\includegraphics*[width=\halfwidth]{/Users/zender/data/fgr/anv/spcp_85_8589_amip5_8589_01_LWCF.eps}\vfill
\includegraphics*[width=\halfwidth]{/Users/zender/data/fgr/anv/spcp_85_8589_amip5_8589_pres_01_CHI.eps}\vfill
\end{center}
\caption{
As in Figure~\ref{fig:xavg_8589} but for:
(a) Longwave cloud forcing LWCF (\wxmS). Contour interval is 10~\wxmS.
(b) 200~mb velocity potential $\chi$ (\mSxs). Contour interval is $1
\times 10^6$~\mSxs.  Shading indicates less subsidence (more
divergence). 
\label{fig:8589}}   
\end{figure}

%We have investigated the sensitivity of the simulated climate to
%the radiative effects of a representation of ice cloud prognosed from 
%modeled and observed characteristics of tropical anvil. 
%In particular, we replaced a representation of tropical cloud which
%diagnoses cloud mass from column vapor with a prognostic
%representation which forecasts anvil generation from the vertical
%profile of convective mass flux and anvil precipitation from mesoscale
%budget estimates.
The direct effect of this prognostic anvil representation is to
sequester more condensate in the upper troposphere, a larger fraction
of which is ice.
Figure~\ref{fig:xavg_8589}a contours the ensemble mean January
vertical profile of change in zonal average condensate \qc\ in 
the tropics.  
The largest model differences occur in the ascending branch of
the Hadley cell.
These changes in condensate distribution and ice fraction
(\ref{eqn:ficeccm}--\ref{eqn:ficeanv}) agree with 
recent observations \cite[]{WSS931,GrM96} and simulations in cumulus 
ensemble models \cite[]{SLT94,GMK95}. 
%We stress the control model includes diagnostic ice cloud---the
%sensitivity experiment shows the impact of forecasting ice cloud based
%on the physics of tropical anvil.

Figure~\ref{fig:xavg_8589}b shows the change in zonal average total
diabatic heating \QT\ in the tropics.
Changes above 200~mb, where condensation is weak, show the radiative
heating perturbation induced by the enhanced anvil.
Enhanced anvil perturbs tropical upper troposphere heating more
strongly in winter, when the column is clearer and anvil radiatively
heats the troposphere above 200~mb.  
In the summer tropics, enhanced anvil occurs in a ``deep cloud''
environment, reducing radiative cooling up to 200~mb, and enhancing
cooling above that. 
Reduced optical depth keeps the intrinsically greater solar absorption
of ice relative to liquid from causing a ubiquitous heating increase
above 600~mb.

Beneath 200~mb, change in \QT\ is dominated by change in latent
heating \QL. 
Convection intensifies from 0--10~\dgrn\ in ANV in both seasons,
reflecting an enhanced ITCZ, notably over Micronesia, the east
Indian Ocean, and northeast of Brazil. 
Deep convection in the remainder of the ascending branch of the Hadley
circulation is reduced. 
Weaker summer hemisphere diabatic heating in ANV reduced Hadley cell
strength by 13\% in January.
Change in tropical water vapor (not shown) strongly resembles $\Delta
\QL$.

Figure~\ref{fig:xavg_8589}c shows ANV warms the 50~mb beneath the
tropical tropopause by 2--3~\dgrk, roughly 5 times the standard
deviation of  zonal average monthly $T$ from a 10~yr AMIP CCM2
simulation. 
The increase in tropical upper tropospheric temperature includes
anvil-induced increase in radiative equilibrium $T$ and decreased heat
export by the Hadley cell.
%The near meridional symmetry of the $T$ increase reflects the
%inability of the upper troposphere to maintain large temperature
%gradients \cite[]{LiH88}.
There is no significant change in tropical atmospheric stability
beneath 200~mb. 

Longwave cloud forcing (LWCF) is a radiative proxy for tropical anvil.
Figure~\ref{fig:8589}a shows the change in LWCF due to improved
representation of anvil lifecycle.
LWCF generally increased in the winter hemisphere and decreased in the
summer.
%The strongest bias of the ANV prognostic anvil scheme is too-strong
%cloud forcing over wintertime desert, due to weak sublimation in
%subsidence regimes.
Usually enhanced ice amount and fraction in the prognostic
anvil balanced the weaker mass extinction efficiency of large ice
crystals because anvil vertical location was tied to larger
hydrometeor size.  
However, LWCF significantly decreased (and improved) over the central
Indian Ocean in January despite a ubiquitous increase in anvil mass in
the tropics (cf.\ Figure~\ref{fig:xavg_8589}a).  
This was due to improvement in the circulation over the central Indian
Ocean.

Figure~\ref{fig:8589}b shows the geographic variation of the change in
January 200~mb velocity potential $\chi$.
%The prognostic anvil is less strongly coupled to SST.
%Based on the 1987 El Ni\~no, the prognostic anvil formulation improves
%longwave cloud radiative response to SST cooling but worsens response
%to warming $> 2$\,\dgrc. 
%In conjunction with weaker mass extinction of ice, this weakens the
%feedback between longwave cloud forcing and convection over SST maxima.
This change in large scale divergent motion shows the net response of
convection to the altered diabatic heating is a shift toward the
winter hemisphere.
This eliminates a persistent convective bias in January over the
central Indian Ocean.
The decreased Indian Ocean convection also decreased subsidence over
African and Arabian desert, allowing too much high cloud to form there
(Figure~\ref{fig:8589}a). 
%Moreover, increased convection and high cloud north of the equator
%propagate Rossby waves to the extratropics.  
%This causes significant ridging in the 500~mb height field over the
%west coasts of North America and Europe (not shown), substantially improving
%agreement with analysis. 

\amssection{4. CONCLUSIONS}\label{sec:anv_cnc}

We assessed climate sensitivity to tropical anvil prognosed from the
modeled and observed behavior of tropical cloud systems, emphasizing
the direct relationship between anvil growth and anvil-base convective
mass flux, the excess of ice over supercooled liquid, and the vertical
distribution of condensate. 
%Five year integrations of the National Center for Atmospheric Research
%Community Climate Model showed climate statistics from a diagnostic
%ice cloud representation based on temperature, humidity, and
%stability, differ significantly from statistics from a prognostic
%representation which forecasts cloud from modeled and observed
%characteristics of tropical anvil lifecycle.
Tropical climate features sensitive to these anvil characteristics
include upper troposphere temperature structure, Hadley cell strength,
and warm pool convection. 
Many of these responses improved the climate simulation.
Thus our study isolates some fundamental climate statistics in the 
tropics that are partially controlled by features of tropical anvil 
lifecycle which are not represented in most current GCM anvil 
parameterizations. 
Accounting for these features should be a high priority for future GCM
cloud parameterizations.

% Balance preprint columns
%\ifphd{}{\balance}

% Appendices
%\ifphd{}{\appendix}

% Acknowledgements
%\ifphd{\subsection{Acknowledgements}\label{sec:anv_ack}}{\acknowledgments}
%The authors wish to thank 

% Bibliography
\bibliographystyle{jas}
\bibliography{bib}

\end{document}

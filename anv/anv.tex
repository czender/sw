% $Id$

% Purpose: ANV parameterization and implementation paper.

% Usage:
% dvips -Ppdf -G0 -o ${DATA}/ps/anv.ps ~/anv/anv.dvi;ps2pdf ${DATA}/ps/anv.ps ${DATA}/ps/anv.pdf
% cd ~/anv;make -W anv.tex anv.dvi anv.ps;cd -
% cd ~/anv;texcln anv;make anv.pdf;bibtex anv;makeindex anv;make anv.pdf;bibtex anv;makeindex anv;make anv.pdf;cd -
% scp -p ${DATA}/ps/anv.ps ${DATA}/ps/anv.pdf dust.ess.uci.edu:/var/www/html/ppr

\documentclass[twoside,agupp]{aguplus}
%\documentclass[twoside,draft,agupp]{aguplus} % excludes figures
%\documentclass[agums]{aguplus}
%\documentclass[twoside,agums]{aguplus}
%\documentclass[jgrga]{aguplus}

%  AGU++ OPTIONS
%\printfigures        % ADDS FIGURES AT END
%\doublecaption{35pc} % CAPTIONS PRINTED TWICE
\sectionnumbers      % TURNS ON SECTION NUMBERS
%\extraabstract       % ADDS SUPPLEMENTAL ABSTRACT
%\afour               % EUROPEAN A4 PAPER SIZE
%\figmarkoff          % SUPPRESS MARGINAL MARKINGS

%  AGUTeX OPTIONS AND ENTRIES
%\tighten             % TURNS OFF DOUBLE SPACING, has no effect with agupp
%\singlespace         % RESTORES SINGLE SPACING
%\doublespace         % RESTORES DOUBLE SPACING

% Standard packages
\usepackage{ifpdf} % Define \ifpdf
\ifpdf % We are running PDFLaTeX
\usepackage[pdftex]{graphicx} % Defines \includegraphics*
\pdfcompresslevel=9
\usepackage{thumbpdf} % Generate thumbnails
\usepackage{epstopdf} % Convert .eps, if found, to .pdf when required
\else % We are not running PDFLaTeX
\usepackage[dvips]{graphicx} % Defines \includegraphics*
\fi % endif PDFLaTeX

\usepackage{longtable} % for long tables, like acronyms and symbols
\usepackage{tabularx} % for long tables, like acronyms and symbols

\usepackage{csz} % all my local definitions
\usepackage{dmn} % Dimensional units
% Usage: % Usage: % Usage: \input{jgr_abb} % AGU-sanctioned journal title abbreviations

\def\aapgb{{\it Amer. Assoc. Petroleum Geologists Bull.}}
\def\adg{{\it Adv. Geophys.}}
\def\ajs{{\it Amer. J. Sci.}}
\def\amb{{\it Ambio}}
\def\amgb{{\it Arch. Meteorol. Geophys. Bioclimatl.}}
\def\ang{{\it Ann. Glaciol.}}
\def\angeo{{\it Ann. Geophys.}}
\def\apo{{\it Appl. Opt.}}
\def\areps{{\it Ann. Rev. Earth Planet. Sci.}}
\def\asr{{\it Adv. Space Res.}}
\def\ate{{\it Atmos. Environ.}}
\def\atf{{\it Atmosfera}}
\def\atms{{\it ACM Trans. Math Software}}
\def\ato{{\it Atmos. Ocean}}
\def\atr{{\it Atmos. Res.}}
\def\gbc{{\it Global Biogeochem. Cycles}} % csz
\def\blm{{\it Boundary-Layer Meteorol.}} % csz 
\def\bpa{{\it Beitr. Phys. Atmosph.}}
\def\bams{{\it Bull. Am. Meteorol. Soc.}}
\def\clc{{\it Clim. Change}}
\def\cld{{\it Clim. Dyn.}}
\def\com{{\it Computing}}
\def\dao{{\it Dyn. Atmos. Oceans}}
\def\dsr{{\it Deep-Sea Res.}}
\def\esr{{\it Earth Sci. Revs.}}
\def\gec{{\it Geosci. Canada}}
\def\gei{{\it Geofis. Int.}}
\def\gej{{\it Geogr. J.}}
\def\gem{{\it Geophys. Monogr.}}
\def\geo{{\it Geology}}
\def\grl{{\it Geophys. Res. Lett.}}
\def\ieeec{{\it IEEE Computer}}
\def\ijna{{\it IMA J. Numer. Anal.}}
\def\ijnmf{{\it Int. J. Num. Meteorol. Fl.}}
\def\jac{{\it J. Atmos. Chem.}}
\def\jacm{{\it J. Assoc. Comput. Mach.}}
\def\jam{{\it J. Appl. Meteorol.}}
\def\jas{{\it J. Atmos. Sci.}}
\def\jatp{{\it J. Atmos. Terr. Phys.}}
\def\jcam{{\it J. Climate Appl. Meteorol.}}
\def\jchp{{\it J. Chem Phys.}}
\def\jcis{{\it J. Coll. I. Sci.}}
\def\jcl{{\it J. Clim.}}
\def\jcp{{\it J. Comput. Phys.}}
\def\jfm{{\it J. Fluid Mech.}}
\def\jgl{{\it J. Glaciol.}}
\def\jgr{{\it J. Geophys. Res.}}
\def\jgs{{\it J. Geol. Soc. London}}
\def\jme{{\it J. Meteorol.}}
\def\jmr{{\it J. Marine Res.}}
\def\jmsj{{\it J. Meteorol. Soc. Jpn.}}
\def\josa{{\it J. Opt. Soc. A}}
\def\jpo{{\it J. Phys. Oceanogr.}}
\def\jqsrt{{\it J. Quant. Spectrosc. Radiat. Transfer}}
\def\jpca{{\it J. Phys. Chem. A}}
\def\lnc{{\it Lett. Nuov. C}}
\def\mac{{\it Math. Comp.}}
\def\map{{\it Meteorol. Atmos. Physics.}}
\def\mem{{\it Meteorol. Mag.}}
\def\mnras{{\it Mon. Not. Roy. Astron. Soc.}} 
\def\mwr{{\it Mon. Weather Rev.}} 
\def\nat{{\it Nature}}
\def\pac{{\it Parallel Computing}}
\def\pag{{\it Pure Appl. Geophys.}}
\def\pal{{\it Paleoceanography}}
\def\pht{{\it Physics Today}}
\def\pieee{{\it Proc. IEEE}}
\def\pla{{\it Phys. Lett. A}}
\def\ppp{{\it Paleogeogr. Paleoclim. Paleoecol.}}
\def\pra{{\it Phys. Res. A}}
\def\prd{{\it Phys. Rev. D}}
\def\prl{{\it Phys. Rev. L}}
\def\pss{{\it Planet. Space Sci.}}
\def\ptrsl{{\it Phil. Trans. R. Soc. Lond.}}
\def\qjrms{{\it Q. J. R. Meteorol. Soc.}}
\def\qres{{\it Quat. Res.}}
\def\qsr{{\it Quatern. Sci. Rev.}}
\def\reg{{\it Rev. Geophys.}}
\def\rgsp{{\it Revs. Geophys. Space Phys.}}
\def\rpp{{\it Rep. Prog. Phys.}}
\def\sca{{\it Sci. Amer.}}
\def\sci{{\it Science}}
\def\sjna{{\it SIAM J. Numer. Anal.}}
\def\sjssc{{\it SIAM J. Sci. Stat. Comput.}}
\def\tac{{\it Theor. Appl. Climatl.}}
\def\tel{{\it Tellus}}
\def\wea{{\it Weather}}

%SIAM Review: (Society for Industrial and Applied Mathematics)
%       J. on Computing
%       J. on Control and Optimization
%       J. on Algebraic and Discrete Methods
%       J. on Numerical Analysis 
%       J. on Scientific and Statistical Computing



 % AGU-sanctioned journal title abbreviations

\def\aapgb{{\it Amer. Assoc. Petroleum Geologists Bull.}}
\def\adg{{\it Adv. Geophys.}}
\def\ajs{{\it Amer. J. Sci.}}
\def\amb{{\it Ambio}}
\def\amgb{{\it Arch. Meteorol. Geophys. Bioclimatl.}}
\def\ang{{\it Ann. Glaciol.}}
\def\angeo{{\it Ann. Geophys.}}
\def\apo{{\it Appl. Opt.}}
\def\areps{{\it Ann. Rev. Earth Planet. Sci.}}
\def\asr{{\it Adv. Space Res.}}
\def\ate{{\it Atmos. Environ.}}
\def\atf{{\it Atmosfera}}
\def\atms{{\it ACM Trans. Math Software}}
\def\ato{{\it Atmos. Ocean}}
\def\atr{{\it Atmos. Res.}}
\def\gbc{{\it Global Biogeochem. Cycles}} % csz
\def\blm{{\it Boundary-Layer Meteorol.}} % csz 
\def\bpa{{\it Beitr. Phys. Atmosph.}}
\def\bams{{\it Bull. Am. Meteorol. Soc.}}
\def\clc{{\it Clim. Change}}
\def\cld{{\it Clim. Dyn.}}
\def\com{{\it Computing}}
\def\dao{{\it Dyn. Atmos. Oceans}}
\def\dsr{{\it Deep-Sea Res.}}
\def\esr{{\it Earth Sci. Revs.}}
\def\gec{{\it Geosci. Canada}}
\def\gei{{\it Geofis. Int.}}
\def\gej{{\it Geogr. J.}}
\def\gem{{\it Geophys. Monogr.}}
\def\geo{{\it Geology}}
\def\grl{{\it Geophys. Res. Lett.}}
\def\ieeec{{\it IEEE Computer}}
\def\ijna{{\it IMA J. Numer. Anal.}}
\def\ijnmf{{\it Int. J. Num. Meteorol. Fl.}}
\def\jac{{\it J. Atmos. Chem.}}
\def\jacm{{\it J. Assoc. Comput. Mach.}}
\def\jam{{\it J. Appl. Meteorol.}}
\def\jas{{\it J. Atmos. Sci.}}
\def\jatp{{\it J. Atmos. Terr. Phys.}}
\def\jcam{{\it J. Climate Appl. Meteorol.}}
\def\jchp{{\it J. Chem Phys.}}
\def\jcis{{\it J. Coll. I. Sci.}}
\def\jcl{{\it J. Clim.}}
\def\jcp{{\it J. Comput. Phys.}}
\def\jfm{{\it J. Fluid Mech.}}
\def\jgl{{\it J. Glaciol.}}
\def\jgr{{\it J. Geophys. Res.}}
\def\jgs{{\it J. Geol. Soc. London}}
\def\jme{{\it J. Meteorol.}}
\def\jmr{{\it J. Marine Res.}}
\def\jmsj{{\it J. Meteorol. Soc. Jpn.}}
\def\josa{{\it J. Opt. Soc. A}}
\def\jpo{{\it J. Phys. Oceanogr.}}
\def\jqsrt{{\it J. Quant. Spectrosc. Radiat. Transfer}}
\def\jpca{{\it J. Phys. Chem. A}}
\def\lnc{{\it Lett. Nuov. C}}
\def\mac{{\it Math. Comp.}}
\def\map{{\it Meteorol. Atmos. Physics.}}
\def\mem{{\it Meteorol. Mag.}}
\def\mnras{{\it Mon. Not. Roy. Astron. Soc.}} 
\def\mwr{{\it Mon. Weather Rev.}} 
\def\nat{{\it Nature}}
\def\pac{{\it Parallel Computing}}
\def\pag{{\it Pure Appl. Geophys.}}
\def\pal{{\it Paleoceanography}}
\def\pht{{\it Physics Today}}
\def\pieee{{\it Proc. IEEE}}
\def\pla{{\it Phys. Lett. A}}
\def\ppp{{\it Paleogeogr. Paleoclim. Paleoecol.}}
\def\pra{{\it Phys. Res. A}}
\def\prd{{\it Phys. Rev. D}}
\def\prl{{\it Phys. Rev. L}}
\def\pss{{\it Planet. Space Sci.}}
\def\ptrsl{{\it Phil. Trans. R. Soc. Lond.}}
\def\qjrms{{\it Q. J. R. Meteorol. Soc.}}
\def\qres{{\it Quat. Res.}}
\def\qsr{{\it Quatern. Sci. Rev.}}
\def\reg{{\it Rev. Geophys.}}
\def\rgsp{{\it Revs. Geophys. Space Phys.}}
\def\rpp{{\it Rep. Prog. Phys.}}
\def\sca{{\it Sci. Amer.}}
\def\sci{{\it Science}}
\def\sjna{{\it SIAM J. Numer. Anal.}}
\def\sjssc{{\it SIAM J. Sci. Stat. Comput.}}
\def\tac{{\it Theor. Appl. Climatl.}}
\def\tel{{\it Tellus}}
\def\wea{{\it Weather}}

%SIAM Review: (Society for Industrial and Applied Mathematics)
%       J. on Computing
%       J. on Control and Optimization
%       J. on Algebraic and Discrete Methods
%       J. on Numerical Analysis 
%       J. on Scientific and Statistical Computing



 % AGU-sanctioned journal title abbreviations

\def\aapgb{{\it Amer. Assoc. Petroleum Geologists Bull.}}
\def\adg{{\it Adv. Geophys.}}
\def\ajs{{\it Amer. J. Sci.}}
\def\amb{{\it Ambio}}
\def\amgb{{\it Arch. Meteorol. Geophys. Bioclimatl.}}
\def\ang{{\it Ann. Glaciol.}}
\def\angeo{{\it Ann. Geophys.}}
\def\apo{{\it Appl. Opt.}}
\def\areps{{\it Ann. Rev. Earth Planet. Sci.}}
\def\asr{{\it Adv. Space Res.}}
\def\ate{{\it Atmos. Environ.}}
\def\atf{{\it Atmosfera}}
\def\atms{{\it ACM Trans. Math Software}}
\def\ato{{\it Atmos. Ocean}}
\def\atr{{\it Atmos. Res.}}
\def\gbc{{\it Global Biogeochem. Cycles}} % csz
\def\blm{{\it Boundary-Layer Meteorol.}} % csz 
\def\bpa{{\it Beitr. Phys. Atmosph.}}
\def\bams{{\it Bull. Am. Meteorol. Soc.}}
\def\clc{{\it Clim. Change}}
\def\cld{{\it Clim. Dyn.}}
\def\com{{\it Computing}}
\def\dao{{\it Dyn. Atmos. Oceans}}
\def\dsr{{\it Deep-Sea Res.}}
\def\esr{{\it Earth Sci. Revs.}}
\def\gec{{\it Geosci. Canada}}
\def\gei{{\it Geofis. Int.}}
\def\gej{{\it Geogr. J.}}
\def\gem{{\it Geophys. Monogr.}}
\def\geo{{\it Geology}}
\def\grl{{\it Geophys. Res. Lett.}}
\def\ieeec{{\it IEEE Computer}}
\def\ijna{{\it IMA J. Numer. Anal.}}
\def\ijnmf{{\it Int. J. Num. Meteorol. Fl.}}
\def\jac{{\it J. Atmos. Chem.}}
\def\jacm{{\it J. Assoc. Comput. Mach.}}
\def\jam{{\it J. Appl. Meteorol.}}
\def\jas{{\it J. Atmos. Sci.}}
\def\jatp{{\it J. Atmos. Terr. Phys.}}
\def\jcam{{\it J. Climate Appl. Meteorol.}}
\def\jchp{{\it J. Chem Phys.}}
\def\jcis{{\it J. Coll. I. Sci.}}
\def\jcl{{\it J. Clim.}}
\def\jcp{{\it J. Comput. Phys.}}
\def\jfm{{\it J. Fluid Mech.}}
\def\jgl{{\it J. Glaciol.}}
\def\jgr{{\it J. Geophys. Res.}}
\def\jgs{{\it J. Geol. Soc. London}}
\def\jme{{\it J. Meteorol.}}
\def\jmr{{\it J. Marine Res.}}
\def\jmsj{{\it J. Meteorol. Soc. Jpn.}}
\def\josa{{\it J. Opt. Soc. A}}
\def\jpo{{\it J. Phys. Oceanogr.}}
\def\jqsrt{{\it J. Quant. Spectrosc. Radiat. Transfer}}
\def\jpca{{\it J. Phys. Chem. A}}
\def\lnc{{\it Lett. Nuov. C}}
\def\mac{{\it Math. Comp.}}
\def\map{{\it Meteorol. Atmos. Physics.}}
\def\mem{{\it Meteorol. Mag.}}
\def\mnras{{\it Mon. Not. Roy. Astron. Soc.}} 
\def\mwr{{\it Mon. Weather Rev.}} 
\def\nat{{\it Nature}}
\def\pac{{\it Parallel Computing}}
\def\pag{{\it Pure Appl. Geophys.}}
\def\pal{{\it Paleoceanography}}
\def\pht{{\it Physics Today}}
\def\pieee{{\it Proc. IEEE}}
\def\pla{{\it Phys. Lett. A}}
\def\ppp{{\it Paleogeogr. Paleoclim. Paleoecol.}}
\def\pra{{\it Phys. Res. A}}
\def\prd{{\it Phys. Rev. D}}
\def\prl{{\it Phys. Rev. L}}
\def\pss{{\it Planet. Space Sci.}}
\def\ptrsl{{\it Phil. Trans. R. Soc. Lond.}}
\def\qjrms{{\it Q. J. R. Meteorol. Soc.}}
\def\qres{{\it Quat. Res.}}
\def\qsr{{\it Quatern. Sci. Rev.}}
\def\reg{{\it Rev. Geophys.}}
\def\rgsp{{\it Revs. Geophys. Space Phys.}}
\def\rpp{{\it Rep. Prog. Phys.}}
\def\sca{{\it Sci. Amer.}}
\def\sci{{\it Science}}
\def\sjna{{\it SIAM J. Numer. Anal.}}
\def\sjssc{{\it SIAM J. Sci. Stat. Comput.}}
\def\tac{{\it Theor. Appl. Climatl.}}
\def\tel{{\it Tellus}}
\def\wea{{\it Weather}}

%SIAM Review: (Society for Industrial and Applied Mathematics)
%       J. on Computing
%       J. on Control and Optimization
%       J. on Algebraic and Discrete Methods
%       J. on Numerical Analysis 
%       J. on Scientific and Statistical Computing





% jgrga documentstyle automatically uppercases \lefthead and \righthead
\lefthead{ZENDER AND KIEHL: CLIMATE SENSITIVITY TO ANVIL STRUCTURE}
%\righthead{Climate sensitivity to anvil condensate phase, height, and size}
%\righthead{Climate sensitivity to tropical anvil representation}
\righthead{ZENDER AND KIEHL: CLIMATE SENSITIVITY TO ANVIL STRUCTURE}
%\righthead{Climate sensitivity to tropical anvil structure}
\received{December~19, 1996}
\revised{May~5, 1997}
\accepted{July~12, 1997}
\journalid{JGRD}{October 27, 1997}
\articleid{23793}{23803}
\paperid{97JD02009}
% The $ in the following line screws up the hilit19 highlighting
\ccc{0148-0227/97/97JD-02009\$09.00}
%\ccc{0148-0227/97/97JD-02009\ 09.00}
% \cpright{PD}{1997}
% \cpright{Crown}{1997}
% (No \ccc{} for Crown copyrights.)
\cpright{AGU}{1997}

\authoraddr{J. T. Kiehl and Charles S. Zender, National Center for Atmospheric
Research, P.O. Box 3000, Boulder, CO 80307-3000. (e-mail:
jtkon@ncar.ucar.edu; zender@ncar.ucar.edu)} 

\slugcomment{JOURNAL OF GEOPHYSICAL RESEARCH, VOL.~102, NO.~D20, PAGES
23,793--23,803, OCTOBER~27,~1997}
%\slugcomment{, 102(D20), 23,793--23,803, 1997. Copyright \copyright 1997 AGU.} 
%\slugcomment{\today}

\begin{document}
%} % not phdcsz

% NB: title must be lowercased by hand. The documentstyle does not enforce it.
%\def\paperchaptertitle{Sensitivity of a climate simulation to vertical
%distribution, phase, and size of anvil condensate through radiative effects}  
%\def\paperchaptertitle{Tropical Climate Sensitivity to Representation
%of Cirrus Anvil Lifecycle}  
%\def\paperchaptertitle{Sensitivity of tropical climate simulations to
%convectively generated anvils}   
\def\paperchaptertitle{Sensitivity of climate simulations to
radiative effects of tropical anvil structure}   
\title{\paperchaptertitle}
\author{Charles S. Zender and J. T. Kiehl}
\affil{National Center for Atmospheric Research, Boulder, Colorado}

\begin{abstract}

Climate sensitivity to the representation of tropical anvil is
investigated in a version of the National Center for Atmospheric
Research Community Climate Model.
Common features of tropical anvil generation and structure, consistent
with observations and cloud resolving models, are incorporated into a
simple prognostic anvil parameterization. 
These features include anvil convective origin, vertical profile,
phase, areal extent, and lifespan.  
Two numerical climate integrations are forced by 1985--1989 sea
surface temperature (SST):
the control, with simple diagnostic anvil, and the experiment, which  
simulates tropical anvil structure prognostically.
The prognostic anvil formulation enhances ice and reduces liquid in
the tropics.    
Increase in hydrometeor size associated with anvil weakens cloud
radiative extinction per unit mass by factors of 1--3.
The weaker mass extinction efficiency approximately balances enhanced
ice amount so that anvil ice mass quadruples without biasing the mean
radiative energy balance, but significantly alters the vertical
distribution of radiative effects.
Enhanced anvil perturbs the tropical upper troposphere temperature
structure more strongly in winter, when the column is clearer and
anvil radiatively heats the troposphere above 200~mb. 
In the summer tropics, enhanced anvil reduces radiative cooling up to 
200~mb, and enhances cooling above that.
The prognostic anvil formulation improves longwave cloud radiative
response to SST cooling but worsens response to warming $> 2\
\dgr$C.  
The net response of convection is a shift toward the winter hemisphere
in solstice months.
These changes lead to a significant response in the extratropical
height field in January. 
These results emphasize the importance of representing tropical anvil
structure in climate simulations.
\end{abstract}

\section{Introduction}\label{sec:anv_intro}

% Get rid of the horizontal rule separating this footnote from text
\renewcommand{\footnoterule}{} 
\setlength{\footnotesep}{12pt} 
\footnote{\noindent Copyright 1997 by the American Geophysical Union.\\
Paper number 97JD02009\\
0148-0227/97/97JD-02009\$09.00}%
Radiative forcing from the extended tropical upper tropospheric cloud
known as cirrus anvil plays a dominant role in determining the
diabatic heating which drives the general circulation.  
Tropical cirrus anvil originates in the complex interaction of a
mesoscale convective system (MCS) with the environment.
A general circulation model (GCM) does not resolve this interaction
and must rely on sub-gridscale methods to diagnose or predict anvil
cloud. 
This study combines distinctive features of tropical anvil structure
into a parameterization suitable for GCMs, and then examines the
impact of accounting for the radiative effects of tropical anvil on
the climate system. 

The radiative effects of anvil depend on its distinctive lifecycle and
structure, which may be briefly summarized as follows:
Deep convection is the ultimate source of tropical upper tropospheric
extended clouds, i.e., tropical anvils \cite[e.g.,][]{WeS80}. 
The relative area of convecting cores to the associated anvil is
10--20\% \cite[]{LeH80,FDR90}.
Anvil lifetime, typically 6--12~hr \cite[]{ALV88,LeH80}, exceeds the
duration of deep convection by many hours \cite[]{Hou89}.
Thus, although the cumulonimbus core produces the strongest radiative
impact per unit area, the anvil region dominates the radiative impacts
of the convective cluster as a whole \cite[]{MaR93,WSS932}.
\cite{LeH80} and \cite[]{GaH83} inferred the hydrologic budgets of 
tropical convective systems from observations:
Roughly 60--75\% of anvil condensate is detrained from deep convective
updrafts. 
The remainder, roughly 25--40\% of anvil mass, is generated by
circulations outside the deep convective core, i.e., in the anvil
itself. 
Roughly 40\% of MCS precipitation comes from the stratiform region. 
Observations and numerical simulations
\cite[e.g.,][]{WSS931,SLT94,GMK96,McH96,McH971} show time mean
condensate mixing ratio \qc\ in tropical convective systems 
does not decrease significantly (but can increase) from the freezing 
level to $\sim 300$~mb, above which \qc\ decreases rapidly.  
The intrinsically mesoscale nature of these anvil features has
hindered their representation in most GCM moist convection schemes 
\cite[]{Don93}.   
%How sensitive are climate simulations, and, ultimately, climate, to
%these features of tropical anvil lifecycle? 
How explicitly should these features of anvil structure and lifecycle
be represented in GCMs?  
The answer depends on climate sensitivity to tropical anvil.

Many previous GCM studies have advanced understanding of climate
sensitivity to tropical anvils by modeling climate sensitivity to
anvil representation.
\cite{RPM83} showed non-black cirrus is crucial to maintaining
the observed tropical upper troposphere temperature structure and  
meridional temperature gradient.
\cite{ChR85} showed simulated zonal average upper troposphere
temperature increased significantly when treating frozen cloud 
particles as 20\,\um\ larger than liquid.
\cite{SlS88} showed tropical anvil not only warms the tropical upper  
troposphere, but also accelerates the subtropical jets and excites
responses in the northern hemisphere winter height field. 
%\cite{RHD89} showed detrained anvil condensate could be an important
%moistening mechanism for the upper troposphere.
\cite{RaR891} showed shortwave heating in tropical anvil is a
significant fraction of total diabatic heating above 300~mb, and plays
an important role in the maintenance of the upper tropospheric
temperature structure. 
Furthermore, they suggested the relative abundance of detached cirrus 
anvil to anvil embedded in deep convective systems determines the sign
of net radiative heating above 300~mb.
\cite{RaC91} hypothesized cirrus radiative forcing can be an important
negative feedback for stabilizing column energy changes induced by
local positive SST anomalies over warm ocean.
\cite{SeM93} showed representing ice cloud prognostically rather than
diagnostically could substantially alter modeled climate sensitivity
to CO$_2$ doubling and SST change. 
\cite{SRB94} examined climate sensitivity to anvil radiative forcing
over the west Pacific warm pool.
They showed atmospheric heating by tropical cirrus is primarily
balanced by vertical advective cooling, in accord with \cite{ALV88}.
\cite{LoR95} showed climate response to blackbody cirrus forcing
resembles the response to increased SST forcing even though the direct
heating mechanisms, enhanced cirrus radiative heating and surface
evaporation, respectively, are vertically distinct.

These GCM studies employed a variety of methods to determine anvil
condensate amount. 
Most of the aforementioned studies diagnose anvil cloud from
atmospheric thermodynamic properties (e.g., relative humidity, column
vapor, and stability).
However, prognostic schemes are more physically based than diagnostic.
\cite{SeM93} showed radiative feedbacks from prognostic cloud could be
significantly different than feedbacks from cloud diagnosed from
relative humidity. 
Newer GCMs \cite[e.g.,][]{Tie93,DYK96} prognose stratiform
anvil cloud from bulk microphysics.
These schemes detrain condensate predicted by the moist convection
parameterization into the stratiform anvil.

Our motivation in the present study is to examine the role of
tropical anvil radiative forcing in a climate where the representation
of tropical anvil agrees with the gross behavior of tropical mesoscale
convective systems, summarized above.
We replace a representation of tropical cloud which diagnoses cloud
mass from column vapor with a prognostic representation which
forecasts anvil generation from the vertical profile of convective
mass flux and anvil precipitation from mesoscale budget estimates. 
%This builds upon previous GCM cirrus sensitivity studies in an
%important way: the forecast anvil is self-consistent with
%cloud-resolving, cumulus ensemble model simulations and observations  
%of mesoscale convective systems. 
Thus, the forcing in the experiment is an integrated set of
constraints (ratio of convective condensate generation to vertical
mass flux profile, horizontal extent, vertical condensate profile, ice
fraction, ratio of mesoscale precipitation to sublimation, and
lifetime) consistent with tropical anvil structure but not present in
most current GCM cloud parameterizations.
By comparing the simulated climates to each other we can deduce
climate sensitivity to the representation of tropical anvil
structure. 

The parameterization of tropical anvil used in the numerical climate
experiment is developed in section~\ref{sec:anv_mdl}.   
Section~\ref{sec:ss} presents the mean climate response to the
tropical anvil representation.
Section~\ref{sec:enso} examines anvil response to SST forcing in
the 1987 El Ni\~no. 
Section~\ref{sec:anv_cnc} contains the conclusions. 

\section{GCM Anvil Parameterization}\label{sec:anv_mdl}

%Empirical budget studies suggest the hydrologic cycle of tropical
%anvil may be conceptually divided into convective and stratiform
%components \cite[]{LeH80,GaH83}. 
%Convective formation accounts for ice formed in deep convective 
%updrafts and detrained into the stratiform anvil.
%Stratiform formation accounts for ice formed outside convective
%cores, such as condensate formed or frozen in large-scale ascent. 
%The above studies indicate roughly 60--75\% of anvil condensate
%originates in convective cores.
%The remainder, roughly 25--40\% of anvil mass, is generated by
%circulations outside the deep convective core. 

Spatio-temporal scale mismatch between convective and stratiform
processes and GCM resolution makes anvil parameterization difficult.
GCMs currently employ two common methods to represent these processes:
(i) diagnosing anvil cloud from column thermodynamic properties (e.g.,
relative humidity) and (ii) prognosing anvil cloud by assuming an
anvil detrainment efficiency which acts on the convective mass flux
predicted by the moist convection scheme.
Diagnostic methods like (i) have difficulty representing
convective-radiative hysteresis, such as the radiative influence of 
detached anvils.
As \cite{Don93} points out, prognostic methods like (ii) often do not
explicitly account for the 25--40\% of anvil mass formed by secondary
circulations outside the deep convective core.

Cloud resolving models and mesoscale budget studies suggest the
hydrologic structure of MCS anvils may be simply parameterized in
terms of large scale forcing.  
Figure~\ref{fig:anv_concept} shows our modified version of the
\cite{LeH80} conceptual anvil model.
\begin{figure}
\begin{center}
\includegraphics*[width=\halfwidth]{/data/zender/fgr/anv/anv_concept}\vfill
\end{center}
\caption{
Conceptual model of the ice budget in ANV. 
$c_1$--$c_3$ and \pc\ are free parameters.
\label{fig:anv_concept}}
\end{figure}
We implement the parameterization of this conceptual anvil model,
denoted ANV, as follows:  
For a grid cell of density $\rho$ and ice mixing ratio \qi\ located in
a convecting column with convective mass flux \Mc\ at 500~mb,
the ice budget that defines ANV is:
\begin{equation}
\label{eqn:qi_cnv}
{D\qi \over Dt} = {c_1 \Mc \over \rho \Delta Z} - c_2 \qi - c_3 \qi
\end{equation}
where $\Delta Z$ is the thickness of the convecting portion of the
column in which $T < 0$\,\dgrc.
\qi, $\rho$, and the wind vector $\vec u$ (hidden in the material
derivative) vary in the vertical.
The material derivative on the LHS accounts for advection.
In this study the advection of the prognostic ice is computed using
the same semi-Lagrangian advection algorithm used for water vapor
\cite{WiR94}.    
The first term on the RHS relates the generation of total column ice
to \Mc.
Basing the generation of \qi\ throughout the anvil on the mass flux
near anvil base \Mc\ (rather than local $M$) produces a vertical
profile of \qi\ which increases or remains constant (rather than
significantly decreasing) from anvil base up to $\sim 300$~mb, in
accord with current understanding \cite[e.g.,][]{Hou89,WSS931}.
Justification for the $c_1$ term is described in the next section.
The $c_2$ term represents local sublimation of the anvil due to 
sub-gridscale entrainment and subsaturation.
The $c_3$ term converts ice to precipitation.

Parameters $c_1$--$c_3$ do not vary in time or space---they are the
free parameters of the parameterization.
The values of $c_1$--$c_3$ which yield a realistic climate depend on
the physical parameterizations (e.g., moist convection) used in the
host GCM.  
$c_2$ and $c_3$ are determined from observations \cite[]{LeH80,GaH83}
and process studies in a microphysical cloud model \cite[]{ZeK941}.
The present study sets $c_1 = .75 \times 10^{-3}$, $c_2 = .3 c_3$, and
$c_3 = 1.85 \times 10^{-4}$\,\xs.
We assume the detrained anvil completely covers a gridcell, i.e.,
cloud fraction is 1. 
At the GCM scale (i.e., horizontal gridscale $> 100$~km), this
assumption overestimates anvil coverage relative to observations
\cite{LeH80,Hou89,MaH93}.  
%Cloud systems larger than $100^2$\,\kmS\ account for the bulk of
%tropical cirrus cover cite[]{Hou89}.  

\subsection{Linking Anvil Growth to Anvil-base Mass Flux}\label{sec:mc}

Prior studies indicate convective mass flux $M$ is the best single
parameter to characterize the formation of tropical anvil. 
\cite{XuK91} concluded $M$ best predicts tropical convective cloud
amount and the ice water content of individual anvil layers.   
Based on satellite observations, \cite{MaR93} suggested convective
mass flux at the base of cumulonimbus cores determines the mean cloud 
properties of mesoscale convective systems, including the stratiform
anvil region.

We use a cloud resolving, cumulus ensemble model (CEM) to provide a
high spatial and temporal resolution dataset which spans the range of
MCS activity from the convective to the GCM scale.
A comprehensive review of our CEM simulation is presented by
\cite{GMK96}.  
Figure~\ref{fig:mc}a shows the variation of \Mc\ and anvil mass
growth rate \IWPdot\ through the first day of the CEM simulation.
\begin{figure}
\begin{center}
\includegraphics*[width=\halfwidth,height=.25\textheight]{/data/zender/fgr/anv/cem_mc_diwpdt_a}\vfill
\includegraphics*[width=\halfwidth,height=.25\textheight]{/data/zender/fgr/anv/cem_mc_diwpdt_c}\vfill
\end{center}
\caption{
Results of Cumulus Ensemble Model experiment.
(a) Evolution of 500~mb convective mass flux \Mc\ (\gxmS) (solid) and
\IWPdot\ (dashed) during the first 24 hours of the CEM simulation. 
\IWPdot\ is expressed in \gxmSgcmt, i.e., mass change per GCM
timestep.  
(b) Linear correlation and least squares fit of 500~mb convective mass
flux \Mc\ (\gxmSs) to anvil growth rate \IWPdot\ (\gxmSgcmt) from the
the first 2.5 hours of (a).
\label{fig:mc}}
\end{figure}
The initial anvil formation, lasting about six hours, occurred as a
prescribed thermal instability triggered concentrated convective
updrafts (cores) which detrained frozen condensate into a cirrus anvil
above 500~mb.
The variation of \IWPdot\ is strongest during convectively
active periods, and subdued during the convectively quiescent period
($400 < t < 900$~min) between the first and second generation anvils. 
Non-convective formation of anvil ice is also evident during the
quiescent period. 
Periods when \IWPdot\ is negative occur when net anvil dissipation
(due to precipitation and sublimation) exceeds production. 

In order to isolate the processes controlling ice generation
from destruction we focused on the initial hours of anvil formation,
when a single convective tower dominated the mass budget of the entire
domain. 
Figure~\ref{fig:mc}b shows the correlation of \Mc\ and \IWPdot\
through the first 2.5~hr of the CEM simulation. 
This initial correlation is excellent, but anvil decay processes and
scattered convection within the CEM domain cause the correlation to
deteriorate after 2.5~hr.
The slope of the least squares fit between between \Mc\ and \IWPdot\  
provides the initial estimate for $c_1$ in (\ref{eqn:qi_cnv}).
%Since \IWPdot\ includes convectively detrained condensate and
%condensate generated in the young anvil, the empirically determined 
%$c_1$ implicitly accounts for some anvil mass formed by secondary
%circulations.  
Note \IWPdot\ includes convectively detrained condensate as well as
condensate produced in the young anvil.
The parameter $c_1$ is intended to implicitly account for the anvil
mass formed by both convective and mesoscale circulations.
The results of our GCM simulations with constant $c_1$ (below) show
numerous improvements in anvil climatology over a more traditional
method. 
Parameterizing $c_1$ from an ensemble of CEM integrations, perhaps as
a function of large scale forcing (e.g., wind shear, SST) is the next
logical step.

\subsection{Ice Fraction and Hydrometeor Size}\label{sec:fice}

Recent field observations and models \cite[e.g.,][]{SLT94,GMK96,GrM96}
suggest that, above 500~mb, anvil condensate is dominated by ice.
In terms of temperature, the complete phase transition may occur over
less than 10\,\dgr K.
The control model, a version of the National Center for Atmospheric
Research Community Climate Model \cite[]{KBB96} denoted CCM,
partitions diagnostic condensate between liquid and ice via   
\begin{equation}
\ficeccm = \cases{
  0 & $T > - 10$\,\dgr C \cr
  -{T+10 \over 20} & $-10 \geq T \geq -30$\,\dgr C \cr
  1 & $T < -30$\,\dgr C \cr}
\label{eqn:ficeccm}
\end{equation}
The experiment, denoted ANV, restricts diagnostic liquid condensate
(and, hence, mixed phase cloud) to a narrower and warmer range than
the control model according to
\begin{equation}
\ficeanv = \cases{
  0 & $T > 0$\,\dgr C \cr
  -{T \over 5} & $0 \geq T \geq -5$\,\dgr C \cr
  1 & $T < -5$\,\dgr C \cr}
\label{eqn:ficeanv}
\end{equation}
Since CCM classifies some condensate as cold as $-30\,\dgr$C as
liquid (\ref{eqn:ficeccm}), while ANV has no liquid colder than
$-5$\,\dgr C (\ref{eqn:ficeanv}), it is clear the ANV atmosphere
will contain much more ice than the CCM. 
The effect of using (\ref{eqn:ficeccm}) vs.\ (\ref{eqn:ficeanv}) is
shown in Figure~\ref{fig:anv_fice}, which shows the vertical
distribution of \fice\ averaged over the equatorial Pacific 
(140--270~$\dgr$E, 10~$\dgr$S--10~$\dgr$N) for each GCM
layer in the mixed phase region (GCM data are from the simulations
described below).
\begin{figure}
\begin{center}
\includegraphics*[width=\halfwidth]{/data/zender/fgr/anv/anv_fice}\vfill
\end{center}
\caption[Modeled ice fraction \fice\ over the equatorial Pacific in 
July]{
Modeled ice fraction \fice\ over the equatorial Pacific in July for
the CEM (solid), CCM (dotted), and ANV (dashed).
Arrows indicate 500~mb level.
\label{fig:anv_fice}}   
% NB: ncwa -O -a lat,lon -w gw -v PRES,T,QC,QICE,QL -d lat,-10.,10. -d lon,140.,270. amip5_8589_07.nc foo.nc
% NB: ncks -C -H -v PRES,T,QICE,QC,QL foo.nc | m
\end{figure}
ANV underestimates liquid condensate near 500~mb but otherwise agrees 
with the CEM ((\ref{eqn:ficeanv}) imposes the agreement).
CCM cloud has the same mixed phase composition as ANV cloud
approximately 20\,\dgr C warmer.

For both the control and the experiment, the determination of
hydrometeor size and its radiative treatment is as described in
\cite{KBB96}: 
Model cloud droplet effective radius \rel\ is fixed at 10\,\um\ over
ocean and sea ice but varies from 5--10\,\um\ over land.
Ice crystal effective radius \rei\ represents an equivalent surface
area sphere and varies linearly with a normalized pressure coordinate.
Over ocean this results in $\rei = 10$\,\um\ for $p > 800$~mb to  
$\rei = 30$\,\um\ for $p < 400$~mb.
Solar single scattering and longwave emissivity properties are from
\cite{Sli89} (liquid) and \cite{EbC92} (ice).

Since ice hydrometeors are prescribed to be larger than liquid,
classifying more condensate as ice (\ref{eqn:ficeanv}) also decreases 
the extinction efficiency per unit mass of anvil
\cite[e.g.,][]{ZeK941}. 
This proves to be an important factor in diagnosing the cause of
change in climatological cloud radiative properties in this
sensitivity study.

\subsection{Diagnostic Condensate Formation}\label{sec:ccm}

For convenience, this section summarizes the diagnostic treatment of 
condensate employed in the control model, CCM.
For a fuller description, the reader is referred to \cite{KBB96}. 
In CCM, grid box average \qc\ is logarithmically proportional to
total column  vapor \Qv\ and linearly proportional to local cloud
fraction $A$ through
\begin{eqnarray}
\label{eqn:qc}
\qc & = & A \rho_\ell / \rho \\
\label{eqn:rhol}
\rho_\ell & = & \rho_\ell^0 e^{(-z/h_\ell)} \\
\label{eqn:hl}
h_\ell & = & 810 \ln \Qv
\end{eqnarray}
where $\rho$ is density, $\rho_\ell^0 \equiv .18$\,\gxmC\
is ``in-cloud'' condensed water density at the surface, $z$ is height,
and $h_\ell$ is the scale height of condensed water.
Note the strong coupling of \qc\ to local surface temperature by the
Clausius-Clapeyron relationship implicit in (\ref{eqn:hl}).
Inserting tropical values for \Qv\ we find $h_\ell \lesssim 4$~km so
that \qc\ monotonically decreases from the surface (for fixed $A$).
A drawback to this procedure is that the upper tropospheric peak in
\qc\ profile of an anvil system must be captured by significantly
modulating $A$ across the anvil deck (\ref{eqn:qc}).

\section{Sensitivity Study Results}\label{sec:ss}

To assess climate sensitivity to the features of convectively
generated anvil described above, we compare the results of two
numerical climate integrations forced with observed 1985--1989 SST.  
The control, denoted CCM, uses a diagnostic cloud scheme with no
special provisions for anvil \cite[]{KBB96}.
The experiment, denoted ANV, forecasts anvil ice from
(\ref{eqn:qi_cnv}), which incorporates modeled and observed 
characteristics of tropical anvil production and structure (i.e.,
explicitly linking anvil condensate generation to anvil base
convective mass flux, strong vertical profiles of condensate up to
300~mb, and increased ice fraction).
The focus of the present study is on the radiative effects of the
anvil condensate. 
Thus we restrict the effects of \qi\ in (1) to radiative heating
alone. 
The results focus on tropical climate, where anvil forcing is
greatest.   

%We stress the control model (CCM) includes diagnostic anvil---the
%sensitivity experiment (ANV) shows the impact of forecasting anvil
%based on the physics of tropical anvil.

%We have investigated the sensitivity of the simulated climate to
%the radiative effects of a representation of ice cloud prognosed from 
%modeled and observed characteristics of tropical anvil. 
%In particular, we replaced a representation of tropical cloud which
%diagnoses cloud mass from column vapor with a prognostic
%representation which forecasts anvil generation from the vertical
%profile of convective mass flux and anvil precipitation from mesoscale
%budget estimates.

\subsection{Condensate Distribution}\label{sec:qc}

The climate response to anvil representation is driven by radiative
forcing resulting from the distribution, partitioning (ice or liquid),
and size of cloud condensate. 
The direct effect of representing the structure and convective
production of anvils is to sequester more condensate in the upper
troposphere, a larger fraction of which is ice. 
Figure~\ref{fig:xavg_8589_CP} separates modeled January and July 
total condensed water path CWP by phase ($\CWP \equiv \LWP + \IWP$).
\begin{figure*}
\begin{center}
\includegraphics*[width=.5\hsize]{/data/zender/fgr/anv/xavg_8589_01_TOTLWP}%
\includegraphics*[width=.5\hsize]{/data/zender/fgr/anv/xavg_8589_07_TOTLWP}%

\includegraphics*[width=.5\hsize]{/data/zender/fgr/anv/xavg_8589_01_TOTIWP}%
\includegraphics*[width=.5\hsize]{/data/zender/fgr/anv/xavg_8589_07_TOTIWP}%

\includegraphics*[width=.5\hsize]{/data/zender/fgr/anv/xavg_8589_01_TOTCWP}%
\includegraphics*[width=.5\hsize]{/data/zender/fgr/anv/xavg_8589_07_TOTCWP}%
\end{center}
\caption[Zonal average column condensate burdens from
1985--1989 January and July simulations by CCM and ANV]{
Zonal average column condensate burdens (\gxmS) from
1985--1989 (left) January and (right) July simulations by (solid) CCM
and (dashed) ANV of (a,b) LWP, (c,d) IWP, and (e,f) CWP.   
\label{fig:xavg_8589_CP}}   
\end{figure*}
The refined anvil representation increases tropical IWP by factors
of 2--4, and reduces tropical LWP by 20\%.
The net increase in tropical CWP is 10--50\%, comprising a mean
increase of upper tropospheric condensate with a repartitioning of 
condensate from liquid to ice due to (\ref{eqn:ficeanv}).
The mean increase is partly due to implicitly accounting for anvil
formed in mesoscale circulations.
The tropical response to the experiment in July is similar to January
in the preponderance of the results.  
Thus for economy we omit showing July results in most of the following
fields. 

Figure~\ref{fig:xavg_8589_QC} contours the ensemble mean January
vertical profile of change in zonal average condensate \qc\ in 
the tropics.  
\begin{figure}
\begin{center}
\includegraphics*[width=\halfwidth]{/data/zender/fgr/anv/spcp_85_8589_amip5_8589_xavg_01_QC}\vfill
\end{center}
\caption{
Change (ANV$-$CCM) in zonal average condensate mixing ratio \qc\
(\mgxkg) due to prognostic anvil representation.
Contour interval is 2\,\mgxkg.
Shading indicates values $< 0$.
Data are from ensemble averages of 5 simulated Januarys from
1985--1989.
\label{fig:xavg_8589_QC}}   
\end{figure}
The largest model differences occur in the ascending branch of
the Hadley cell, where the 600--200~mb maxima signals enhanced
tropical anvils with better vertical definition.
These changes in condensate distribution and ice fraction
(\ref{eqn:ficeccm}--\ref{eqn:ficeanv}) agree with inferences from
recent observations \cite[]{WSS931,GrM96,McH96,McH971} and cumulus
ensemble model simulations \cite[]{SLT94,GMK96}.  
%We stress the control model includes diagnostic ice cloud---the
%sensitivity experiment shows the impact of forecasting ice cloud based
%on the physics of tropical anvil.

Before examining the climate response to changes in anvil structure,
it is of interest to estimate the relative roles of the two major
modifications to the original CCM anvil treatment in forcing the
climate.
To describe this relative forcing, we examine the terms in the
linearized net TOA energy budget, 
\begin{equation}
\label{eqn:ss}
\Delta F = {\partial F \over \partial \CWP} \Delta \CWP +
{\partial F \over \partial \fice} \Delta \fice + \ldots
\end{equation}
where $F$ represents a net radiative flux and $\Delta$ the change
between the control and the experiment.
Thus the LHS is the net radiative climate response to the forcings on
the RHS.
The first term on the RHS represents the climate forcing due to the
radiative effects of the change in condensate path \CWP\ and vertical
location which arise from the prognostic formulation of anvil 
generation (\ref{eqn:qi_cnv}) (cf.\ Figure~\ref{fig:xavg_8589_QC}).
The second term on the RHS represents the climate forcing due to the
radiative effects of the increase in ice fraction \fice\ arising from 
(\ref{eqn:ficeanv}) (cf.\ Figure~\ref{fig:anv_fice}). 
These two terms are the dominant forcing mechanisms in the experiment.
The global annual average TOA radiative budgets of the control and the
experiment balance to within .5\,\wxmS, agree between models to
within 2.5\,\wxmS\ (agreement with ERBE is within 4\,\wxmS). 
In other words $\Delta F \approx 0$\,\wxmS\ for $F$ representing TOA
flux or cloud forcing. 
In particular, the zonal average TOA radiative budgets of the models
closely agree in the tropics.
The agreement holds for the total radiative fluxes and the shortwave
and longwave components separately (the surface energy budgets are
similarly balanced). 

The sensitivity factors in (\ref{eqn:ss}) were estimated with an
offline, column version of the CCM radiation code. 
We obtained $\partial F/\partial \CWP$ by differencing the diurnal
average radiative fluxes from the control and experimental zonal
average \qc\ profiles for January at $5\ \dgr$S, using the CCM ice
fraction (\ref{eqn:ficeccm}) for both profiles.  
Similarly, $\partial F/\partial \fice$ was estimated by differencing
the fluxes obtained using the control and experimental ice fractions
on the CCM \qc\ profile. 
From these computations, the prognostic formulation for anvil
structure alone (\ref{eqn:qi_cnv}) imposes a $-19$\,\wxmS\ shortwave
forcing and a $+24$\,\wxmS\ longwave forcing in the tropics, for a net 
radiative forcing of $+5$\,\wxmS.
Thus the increased upper tropospheric condensate due to
(\ref{eqn:qi_cnv}) and shown in Figures~\ref{fig:xavg_8589_CP}\
and~\ref{fig:xavg_8589_QC}\ acts to significantly strengthen tropical
cloud radiative forcing.   
The enhanced ice fraction alone (\ref{eqn:ficeanv}) imposes a
$+10$\,\wxmS\ shortwave forcing and a $-9$\,\wxmS\ longwave forcing in
the tropics, for a net radiative forcing of $+1$\,\wxmS.
Thus the increased upper tropospheric ice fraction due to
(\ref{eqn:ficeccm}) and shown in Figures~\ref{fig:anv_fice}\
and~\ref{fig:xavg_8589_CP}\ acts to significantly weaken tropical
cloud radiative forcing.    
As mentioned in Section~\ref{sec:fice}, this is largely due to the 
larger hydrometeor size associated with ice.
The magnitude of the tropical radiative forcing by the increased upper
tropospheric condensate is 2--3~times the magnitude of the forcing due
to the increased ice fraction. 

\subsection{Tropical Upper Tropospheric Heating}\label{sec:tut} 

In this experiment the changes in tropical anvil structure 
force the circulation by altering total radiative heating \QR.
Anvil induced changes in the vertical and horizontal distribution of
\QR\ alter total diabatic heating \QT, which includes latent heating
(\QL), radiation (\QR), and diffusion (turbulence).
Figure~\ref{fig:xavg_8589_QDIABAT} shows the change in zonal average
\QR\ and \QT\ in the tropics. 
\begin{figure}
\begin{center}
\includegraphics*[width=\halfwidth]{/data/zender/fgr/anv/spcp_85_8589_amip5_8589_xavg_01_RADD}\vfill
\includegraphics*[width=\halfwidth]{/data/zender/fgr/anv/spcp_85_8589_amip5_8589_xavg_01_QDIABAT}\vfill
\end{center}
\caption{
As in Figure~\ref{fig:xavg_8589_QC} but for (a) radiative heating \QR\
only, and (b) total diabatic heating \QT.
Contour intervals are .1\,\kxday.   
\label{fig:xavg_8589_QDIABAT}}   
\end{figure}
Changes above 200~mb, where condensation is weak, are due to the
radiative heating perturbation induced by the enhanced anvil.
Enhanced anvil perturbs tropical upper troposphere heating more
strongly in winter, when the column is clearer and anvil radiatively
heats the troposphere above 200~mb.  
In the summer tropics, enhanced anvil occurs in a cloudier 
environment, reducing radiative cooling up to 200~mb, and enhancing
cooling above that. 
Thus, winter and summer tropics fall, respectively, into the ``anvil''
and ``deep'' cloud scenarios of \cite{RaR891}.
Reduced optical depth keeps the intrinsically greater solar absorption
of ice (relative to liquid) from causing a ubiquitous heating increase
above 600~mb.

Beneath 200~mb, change in \QT\ is dominated by change in latent
heating \QL. 
The response in \QL\ is, to first order, induced by the change in
\QR. 
Convection intensifies from 0--10\,\dgr N in ANV in both seasons,
reflecting an enhanced ITCZ, notably over Micronesia, the east
Indian Ocean, and northeast of Brazil. 
Deep convection in the remainder of the ascending branch of the Hadley
circulation is reduced. 
Weaker summer hemisphere diabatic heating in ANV reduced Hadley cell
strength by 13\% in January, 7\% in July.
Change in tropical water vapor (not shown) strongly resembles $\Delta
\QL$.

Figure~\ref{fig:xavg_8589_T} shows ANV warms the 50~mb beneath the
tropical tropopause by 2--3~$\dgr$K, roughly 5 times the standard
deviation of zonal average monthly $T$ from a 10~yr AMIP CCM2
simulation. 
\begin{figure}
\begin{center}
\includegraphics*[width=\halfwidth]{/data/zender/fgr/anv/spcp_85_8589_amip5_8589_xavg_01_T}\vfill
\end{center}
\caption{
As in Figure~\ref{fig:xavg_8589_QC} but for temperature $T$ (\dgr
K).
Contour interval is .1\,\dgr K. 
\label{fig:xavg_8589_T}}   
\end{figure}
The meridionally symmetric increase in tropical upper tropospheric
temperature includes anvil-induced increase in radiative equilibrium
$T$ and decreased heat export by the Hadley cell.
%The near meridional symmetry of the $T$ increase reflects the
%inability of the upper troposphere to maintain large temperature
%gradients \cite[]{LiH88}.
There is no significant change in tropical atmospheric stability
beneath 200~mb. 

\subsection{Radiative Forcing}\label{sec:cf}

We present the radiative results of the experiment in terms of
top-of-atmosphere (TOA) cloud forcing, observed by the Earth Radiation
Budget Experiment (ERBE) satellite system from 1985--1989
\cite[]{HuC92}.
Shortwave cloud forcing (SWCF) is defined as the net increase in
reflected shortwave (SW) flux at TOA due to cloud scattering and
absorption. 
Figure~\ref{fig:xavg_8589_SWCF} shows zonal average SWCF for January
and July. 
\begin{figure}
\begin{center}
\includegraphics*[width=\halfwidth]{/data/zender/fgr/anv/xavg_8589_01_SWCF}\vfill
\includegraphics*[width=\halfwidth]{/data/zender/fgr/anv/xavg_8589_07_SWCF}\vfill
\end{center}
\caption[Zonal average shortwave cloud forcing SWCF from ERBE,
CCM, and ANV for January and July]{ 
Zonal average shortwave cloud forcing SWCF (\wxmS) from ERBE
(solid), CCM (dotted), and ANV (dashed) for (a) January and (b) July.
\label{fig:xavg_8589_SWCF}}   
\end{figure}
Dramatic changes seen in condensate distribution and phase
(Figures~\ref{fig:xavg_8589_CP} and~\ref{fig:xavg_8589_QC}), are not
seen in zonal average cloud forcing.
%This introduces a central result of our investigation.
%The effects of improved representation of tropical anvil lifecycle 
The models predict similar equatorial LWP beneath 600~mb but ANV has
up to 5 times more equatorial IWP. 
Agreement in modeled tropical cloud forcing illustrates how increased
IWP can radiatively offset increased \rdsffc\ (Section~\ref{sec:fice}).
%The weak January zonal maxima in ANV at 60~$\dgr$S resembles the
%disparity in modeled CWP (Figure~\ref{fig:xavg_8589_CP}e).
% NB: This is interesting because you would infer from the 60S maxima 
% that SWCF is entirely due to CWP differences, not tau/fice changes.
% Vertical profile/mean photon path length differences might be
% important here.
ANV worsens the zonal average bias at the July ITCZ by improving
(increasing) SWCF in the equatorial east Pacific and Atlantic oceans
without reducing SWCF in the Indo-Pacific.
%ANV eliminates nearly half the 40\,\wxmS\ bias poleward of
%60~$\dgr$N in July.

The effect of cloud on terrestrial or longwave (LW) radiation, that
is, the reduction in outgoing longwave radiation (OLR) due to cloud
condensate, is called longwave cloud forcing (LWCF).
LWCF is a radiative proxy for tropical anvil.
Figure~\ref{fig:8589_LWCF} shows the geographic variation of change in
January tropical LWCF due to convectively generated anvils.
\begin{figure}
\begin{center}
\includegraphics*[width=\halfwidth]{/data/zender/fgr/anv/spcp_85_8589_amip5_8589_01_LWCF_bw}\vfill
\end{center}
\caption{
As in Figure~\ref{fig:xavg_8589_QC} but for longwave cloud forcing
LWCF (\wxmS). 
Contour interval is 10\,\wxmS. 
\label{fig:8589_LWCF}}   
\end{figure}
July results (not shown) confirm LWCF generally increased in the
winter hemisphere and decreased in the summer.
The strongest bias of the ANV prognostic anvil scheme is an
overestimate of cloud forcing over wintertime desert, due to weak
sublimation in subsidence regimes.
Usually enhanced ice amount and fraction in the prognostic
anvil balances the weaker mass extinction efficiency of large ice
crystals because hydrometeor size is specified to increase with
hydrometeor height \cite[]{KBB96}.
However, LWCF significantly decreases (and improves) over the central
Indian Ocean in January despite a ubiquitous increase in anvil mass in
the tropics (cf.\ Figure~\ref{fig:xavg_8589_QC}).  
This is due to reduced upper level divergence over the central Indian
Ocean, a region where prognostic anvil significantly alters the
vertical distribution of cloud radiative effects. 

\subsection{Tropical Circulation}\label{sec:cio}

The change in large scale divergent motion in the tropics due to
prognostic anvil representation is shown in
Figure~\ref{fig:pres_8589_01_CHI}, which depicts the geographic
response of the January 200~mb velocity potential $\chi$. 
\begin{figure}
\begin{center}
\includegraphics*[width=\halfwidth]{/data/zender/fgr/anv/spcp_85_8589_amip5_8589_pres_01_CHI}\vfill
\end{center}
\caption{
As in Figure~\ref{fig:xavg_8589_QC} but for 200~mb velocity potential
$\chi$ (\mSxs). 
Contour interval is $1 \times 10^6$\,\mSxs.  
Shading indicates less subsidence (more divergence). 
\label{fig:pres_8589_01_CHI}}   
\end{figure}
%The prognostic anvil is less strongly coupled to SST.
%Based on the 1987 El Ni\~no, the prognostic anvil formulation improves
%longwave cloud radiative response to SST cooling but worsens response
%to warming $> 2$\,\dgrc. 
%In conjunction with weaker mass extinction of ice, this weakens the
%feedback between longwave cloud forcing and convection over SST maxima.
Deep convection shifts toward the winter hemisphere
(cf.\ Figure~\ref{fig:xavg_8589_QDIABAT}).
ECMWF analyses confirm the strong maxima (reduced divergence) over the
central Indian Ocean eliminates a persistent convective bias. 
The decreased Indian Ocean convection also decreased subsidence over
African and Arabian desert, allowing too much high cloud to form there
(Figure~\ref{fig:8589_LWCF}). 
%Moreover, increased convection and high cloud north of the equator
%propagate Rossby waves to the extratropics.  
%This causes significant ridging in the 500~mb height field over the
%west coasts of North America and Europe (not shown), substantially improving
%agreement with analysis. 

As seen above, TOA cloud forcing does not reveal the full extent of 
circulation change due to anvil representation. 
Tropical circulation is sensitive to the specific vertical (and
horizontal) location of anvil heating \cite[]{RaR891,SRB94}. 
Figure~\ref{fig:xyavg_reg_Indian_Central_8589} shows
the vertical profile of simulated diabatic heating components over 
the central Indian Ocean for January conditions.
\begin{figure*}
\begin{center}
\includegraphics*[width=.5\hsize]{/data/zender/fgr/anv/amip5_xyavg_reg_Indian_Central_8589_01}%
\includegraphics*[width=.5\hsize]{/data/zender/fgr/anv/spcp_85_8589_amip5_8589_xyavg_reg_Indian_Central_01}%
\end{center}
\caption[Model simulated profiles of diabatic heating
and differences between models (ANV$-$CCM) over the central 
Indian Ocean for 1985--1989 January]{
Model simulated profiles of (a) diabatic heating
(\kxday), and (b) differences between models (ANV$-$CCM) for the
central Indian Ocean (15\,\dgr S--5\,\dgr N, 60--80\,\dgr E) 
for 1985--1989 January.
Heatings shown are total diabatic (solid), shortwave (dotted),
longwave (short dash), resolved (dash-dot), turbulent
(dash-dot-dot-dot), and convective (long dash).
Note difference in scales.
\label{fig:xyavg_reg_Indian_Central_8589}}
\end{figure*}
The CCM heating profile,
Figure~\ref{fig:xyavg_reg_Indian_Central_8589}a, is typical of deep
convective regions in both models. 
Convective heating dominates radiative from the boundary layer to
250~mb. 
Large scale heating in the upper troposphere, representing stratiform
condensation in anvil, enhances latent heating but the stratiform
precipitation evaporatively cools the lower troposphere.
SW heating is 30--60\% of LW cooling from 800--200~mb, and dominates
\QT\ from 150--100~mb \cite[]{RaR891}. 

Differences between ANV and CCM heating profiles, shown in
Figure~\ref{fig:xyavg_reg_Indian_Central_8589}b, range from 10--50\%
of mean heating rates.
The prognostic anvil representation reduces anvil formation over the
central Indian Ocean.
Reduced condensate absorptivity increases LW cooling by $\sim 30\%$
from 800--400~mb and enhances anvil-base heating near 300~mb. 
This radiative heating dipole increases atmospheric stability.
%Decreased anvil reduced both cloud radiative heating beneath 350~mb
%and cloud top cooling above, which increased column stability.
Weaker vertical motion and upper level divergence
(Figure~\ref{fig:pres_8589_01_CHI}) are accompanied by a large
reduction in convective heating and precipitation (3\,\mmxday).
Reduced convective activity also dries the column, which exacerbates
increased cooling beneath 400~mb.
Thus, a relatively small reduction in anvil heating appears to
leverage much larger reductions in latent heating.
This behavior agrees with \cite{SRB94}, who suggest compensation
between vertical motion and anvil heating is an efficient means of
restoring energy balance in a regime of weak horizontal gradients of
moist static energy. 
%This is atypical of tropical summer, when the response to prognostic
%anvil is normally enhanced cloud base heating and cloud top cooling
%(cf. Figure~\ref{fig:xavg_8589_QDIABAT}).
Differences in diabatic heating components are $< .2$\,\kxday\ in July,
when much of the central Indian Ocean is colder than 28~$\dgr$C, 
and the prognostic anvil effect is minimal.

Significant changes in precipitation and high cloud also occur in the
tropical Pacific in January.
Precipitation associated with the Australian monsoon shifts northward.
This shift enhances Micronesian rainfall by up to 7.5\,\mmxday, and 
mid-tropospheric heating rates by up to 2.7\,\kxday. 

\subsection{Extratropical Response}\label{sec:phi}

\cite{HoK81} showed extratropical stationary wave structure is 
sensitive to the distribution of tropical diabatic heating.
The prognostic representation of tropical anvil significantly alters
stationary wave patterns from the central Pacific to western Europe.
Figure~\ref{fig:pres_8589_Z2TEST} compares the observed and modeled
wintertime 500~mb height field.
\begin{figure*}
\begin{center}
\includegraphics*[width=.33\hsize]{/data/zender/fgr/anv/ecmwf_pres_9095_01_Z2TEST}%
\includegraphics*[width=.33\hsize]{/data/zender/fgr/anv/amip5_pres_8589_01_Z2TEST}%
\includegraphics*[width=.33\hsize]{/data/zender/fgr/anv/spcp_85_pres_8589_01_Z2TEST}%
\end{center}
\caption[January 500~mb geopotential height field for
30--90~$\dgr$N from ECMWF 1990--1995 analyses and model 
simulations of 1985--1989 by CCM and ANV]{
January 500~mb geopotential height field (gpm) for
30--90~$\dgr$N from (a) ECMWF 1990--1995 analyses and model 
simulations of 1985--1989 by (b) CCM and (c) ANV.
Contour interval is 10~gpm. 
\label{fig:pres_8589_Z2TEST}}   
\end{figure*}
% NB: 10~yr. ECMWF observed standard deviation of January $\Phi$ over the PNA is ???~m.
% NB: 20~yr. CCM2 388 modeled standard deviation of January $\Phi$ over the PNA is 40--80~m.
% NB: HBK94 p. 20793 Figure~6a,b is standard deviation of the daily
% differences from the monthly mean.
ANV deepens the central Pacific trough and shifts it $\sim 10 
\dgr$~E.
The associated ridge splits flow around California but reproduces 
observed ridging over the west coast of Canada, absent in CCM.
ANV strengthens the ridge over west Europe, as observed, and shifts
the central European trough $\sim 20 \dgr$~E towards analyses. 
Model differences are 1--3 times model standard deviation in the
vicinity of these ridges. 
These disturbances in extratropical planetary wave structure originate
near the tropical Indo-Pacific heating disturbances and propagate to
the extratropics. 
A similar North American response, also linked to a northward shift of
Australian monsoon precipitation, occurred in \cite{Kie942}.

\section{Cloud Response to SST Forcing in the Equatorial
Pacific}\label{sec:enso}   

The Equatorial Pacific SST anomaly associated with the 1987 El Ni\~no
provides a stringent test of model ability to mimic observed
changes in convective patterns and associated anvil cloud.
During the 1987 El Ni\~no the center of deep convection, accompanying
a large positive SST anomaly, shifted from the west to the central
equatorial Pacific. 
Cloud forcing responded by increasing in the central and east
equatorial Pacific through much of 1987, while cooler SST reduced
cloud forcing in the west.
\cite{HaM93} and \cite{Cho94} emphasize cloud enhancement from
10~$\dgr$S--10~$\dgr$N was largely compensated by clearer sky
from 10--30$\dgr$ in both hemispheres.
We will use the strong SST anomaly in the equatorial Pacific region
from $10\ \dgr$S--$10\ \dgr$N to test the deep convective
response of the differing anvil representations to transient SST
forcing.   
We focus on Springtime behavior because equatorial SST peaks in April
(when the seasonal cycle peaks), and proximity to the equinox
maximizes the hemispheric symmetry of solar forcing \cite[]{RaC91}. 
The Spring SST of the entire equatorial Pacific (10\,\dgr
S--10\,\dgr N, 140\,\dgr E--90\,\dgr W) warmed .9\,\dgr K
from 1985 to 1987, while the central equatorial Pacific alone
(10\,\dgr S--10\,\dgr N, 180--130\,\dgr W) warmed
1.2\,\dgr K. 
Differencing the cold year (1985) Spring months from the warm (1987)
removes the mean cloud forcing state and isolates the cloud forcing
sensitivity (which implicitly includes any reorganization of
convection patterns) to SST change.
Note that atmospheric response could not feed back to SST (a
prescribed boundary condition).  

Figure~\ref{fig:yavg_10S10N_87m85_MAM_QC} shows change in modeled
cloud condensate $\delta \qc$ for 1987$-$1985 over the equatorial
Pacific averaged for Spring. 
Note $\delta$ refers to $1987 - 1985$ temporal change for a given
model, not to inter-model change. 
\begin{figure}
\begin{center}
\includegraphics*[width=\halfwidth]{/data/zender/fgr/anv/amip5_yavg_10S10N_87m85_MAM_QC}\vfill
\includegraphics*[width=\halfwidth]{/data/zender/fgr/anv/spcp_85_yavg_10S10N_87m85_MAM_QC}\vfill
\end{center}
\caption[Longitude-height profile of the $1987-1985$ difference in
Spring quarter (March, April, and May) mean condensate \qc\ over the
equatorial Pacific simulated by CCM and ANV]{
Longitude-height profile of the $1987-1985$ difference in Spring
quarter (March, April, and May) mean condensate \qc\ (\mgxkg) over the
equatorial Pacific (averaged 10\,\dgr S--10\,\dgr N, 
ocean only) simulated by (a) CCM and (b) ANV.
Contour interval is 2\,\mgxkg.
Shading indicates \qc\ decrease from 1985 to 1987.
% NB: IDL gets the shading of the first -'ve contour wrong when land
% is masked so i included land points in this file, no significant
% difference from maritime only above 400 mb.
\label{fig:yavg_10S10N_87m85_MAM_QC}}   
\end{figure}
High cloud increases in both models over the central equatorial
Pacific (where $\delta$SST peaks) in 1987, and decreases over the
western equatorial Pacific.
ANV predicts $\delta \qc$ extrema at the same longitude as CCM
(145~$\dgr$E, 175~$\dgr$W), but roughly 100~mb higher,
and 2--4 times stronger.
Yearly ERBE LWCF (not shown) suggests \qc\ should increase from 
140--110~$\dgr$W, as ANV predicts.  
Thus prognostic anvil representation strengthens and appears to improve
CWP response to SST, but this cannot be demonstrated conclusively
without reliable observational estimates of CWP.

Figure~\ref{fig:reg_Pacific_Equatorial_87m85_0305_LWCF_SWCF} plots 
$\delta$LWCF vs.\ $\delta$SWCF (i.e., cloud forcing sensitivity to
SST) for the equatorial Pacific.
\begin{figure*}
\begin{center}
\includegraphics*[width=.33\hsize]{/data/zender/fgr/anv/erbe_b_reg_Pacific_Equatorial_87m85_0305_LWCF_SWCF}%
\includegraphics*[width=.33\hsize]{/data/zender/fgr/anv/amip5_reg_Pacific_Equatorial_87m85_0305_LWCF_SWCF}%
\includegraphics*[width=.33\hsize]{/data/zender/fgr/anv/spcp_85_reg_Pacific_Equatorial_87m85_0305_LWCF_SWCF}%
\end{center}
\caption[$1987-1985$ differences in Spring quarter (March, April, and
May) monthly mean maritime LWCF and SWCF over the Equatorial Pacific
for ERBE, CCM, and ANV]{ 
$1987-1985$ differences in Spring quarter (March, April, and May) 
monthly mean maritime LWCF and SWCF (\wxmS) over the Equatorial
Pacific (10\,\dgr S--10\,\dgr N, 140\,\dgr E--90\,\dgr W) 
for (a) ERBE, (b) CCM, and (c) ANV.
Solid line is least-squares fit.
\label{fig:reg_Pacific_Equatorial_87m85_0305_LWCF_SWCF}}   
\end{figure*}
Crosses represent GCM gridpoint ($\sim 300^2$\,\kmS) monthly averages. 
Note $\delta \LWCF$ is positively correlated with $\delta \SST$ so
the upper left of the scattergrams are dominated by points from the
equatorial west Pacific, and the center and lower right by points from
the central and east equatorial Pacific.
Gridpoints with $\delta \LWCF < -20$\,\wxmS\ are more copious in ANV,
but extend unrealistically beyond $\delta \LWCF < -50$\,\wxmS.
This response, and weaker response of the prognostic anvil
respresentation to $\delta \SST \gtrsim 2$\,\dgrc ($\delta \LWCF >
50$\,\wxmS), stem from the prognostic formulation of cloud mass. 
The diagnostic cloud mass, which determines CCM cloud forcing, varies
approximately exponentially with SST during convection \cite[]{KBB96}. 
In contrast, ANV cloud mass (hence cloud forcing) peaks with maximum
500~mb convective intensity, which is not necessarily collocated with
SST maxima \cite[]{Hac94}.
By decoupling cloud mass from SST, the prognostic anvil representation
couples cloud forcing more tightly to other factors determining
convective intensity, e.g., atmospheric instability, evaporation, and
surface level wind \cite[]{FDR90}. 
In summary, the prognostic representation of anvil production and
structure has generally improved LWCF response for $\delta \SST < 0\
\dgr$C and worsened LWCF response for $\delta \SST > 2\
\dgr$C. 

The slope $m \equiv \delta \SWCF / \delta \LWCF$ approximately
measures the reduction in surface insolation relative to the increase
in atmospheric heating.
The ERBE data show local net cloud forcing response to El Ni\~no is a
linear ($|r| = .94$), moderately negative feedback ($|m| \approx
1.2$), i.e., the albedo effect of anvil responds more strongly to
local SST anomalies than the greenhouse effect. 
Both models predict $m \approx -.95$, i.e., cloud forcing is a weak
positive local feedback to column energy for equatorial Pacific SST
change, rather than a moderate negative feedback, as observed.
This model agreement is surprising because \cite{RaC91} argue
$|m| > 1$ due to radiative properties of tropical anvil, which is
diagnosed in CCM, but prognosed by (\ref{eqn:qi_cnv}) in ANV.
Despite model differences in anvil representation, the trends of
$\delta$LWCF with $\delta$SST closely agree with each other and
observations: ERBE, CCM, and ANV trends of $\delta$LWCF with
$\delta$SST are $17.0$, $15.7$, and $16.0$\,\wxmSk, with correlations
$.65$, $.60$, and $.54$, respectively.  
However, ERBE, CCM, and ANV trends of $\delta$SWCF with $\delta$SST
are $-20.1$, $-12.8$, and $-15.5$\,\wxmSk, with correlations $-.59$,
$-.43$, and $-.44$, respectively. 
Thus the primary reason both model underestimate $|m|$ is their
25--45\% bias in SWCF response.

\cite{RaC91}, in their Table~1, showed $|m| > 1$ over the equatorial
Pacific independent of which non-El Ni\~no season or year is used for
$\delta$. 
Moreover, the same qualitative behavior is ubiquitous over equatorial
Pacific subregions (i.e., east, central, west) (not shown).
Thus atmospheric GCMs should adequately simulate $m$ in order to
realistically cooperate with SST changes when coupled to an oceanic 
GCM, e.g., in climate change and ENSO experiments.
Studies are currently underway at NCAR which should improve cloud
radiative response to SST change, i.e., simulation of $m$.
These include representing anvil with fully prognostic microphysical
schemes, using probability distributions of tropical cloud fraction
determined from satellite observations, and imposing observed
hydrometeor effective radius. 

%Why is modeled $\delta$SWCF too weak with modeled $\delta$LWCF so
%close to ERBE?   
%Recall that upper tropospheric \rei\ is time-invariant so $\delta \rei
%\approx 0$.  
%Thus $\delta$CWP largely determines $\delta$SWCF.

\section{Conclusions}\label{sec:anv_cnc}

Radiative forcing from cirrus anvil plays a dominant role in
determining the diabatic heating which drives the general circulation
yet many features of tropical anvil structure and lifecycle are not
represented in current GCM anvil parameterizations.   
Five year integrations of the National Center for Atmospheric Research
Community Climate Model (CCM2) were used to ellucidate climate
features sensitive to the representation of anvil production and
structure. 
Anvil features emphasized in the sensitivity study included the
direct relationship between anvil growth and anvil-base convective 
mass flux, mesoscale condensate formation, the excess of ice over
supercooled liquid, and the vertical distribution of condensate. 
%Five year integrations of the National Center for Atmospheric Research
%Community Climate Model showed climate statistics from a diagnostic
%ice cloud representation based on temperature, humidity, and
%stability, differ significantly from statistics from a prognostic
%representation which forecasts cloud from modeled and observed
%characteristics of tropical anvil lifecycle.

The direct effect of improving anvil representation is to sequester
more condensate in the upper troposphere, a larger fraction of which
is ice. 
The radiative effects of enhancing ice amount and fraction were
approximately balanced by weaker radiative extinction per unit mass
because anvil vertical location was tied to larger hydrometeor size.  
Thus top-of-atmosphere climatological radiative measures such as cloud
forcing were not biased by two to fourfold increases in tropical anvil
mass.
The thermal structure and circulation of the tropics were more
dramatically affected due to a vertical shift in heating.

Enhanced anvil perturbs the tropical upper troposphere temperature
structure more strongly in winter, when the column is clearer and
anvil radiatively heats the troposphere above 200~mb. 
In the summer tropics, enhanced anvil occurs in a ``deep cloud''
environment, reducing radiative cooling up to 200~mb, and enhancing
cooling above that. 
Reduced optical depth keeps the intrinsically greater solar absorption
of ice relative to liquid from causing a ubiquitous heating increase
above 600~mb.
Radiative heating contributes to warming the region just below the
tropical tropopause 2--3~$\dgr$K.   

Based on the 1987 El Ni\~no, the prognostic anvil formulation improves
longwave cloud radiative response to SST cooling but worsens response
to warming $> 2$\,\dgrc. 
%In conjunction with weaker radiative extinction of ice, this weakens the
%feedback between longwave cloud forcing and convection over SST maxima.
The net response of convection is a shift toward the winter hemisphere
in solstice months.
This convective reorganization reduced Hadley cell strength and
eliminated a persistent convective bias the central Indian Ocean.
Moreover, the increased convection and high cloud north of the equator
in January propagate Rossby waves to the extratropics.  
This causes significant ridging in the 500~mb height field over the
west coasts of North America and Europe, which substantially improves
agreement with analysis. 

In summary, climate features sensitive to anvil representation include
tropical upper troposphere temperature structure, Hadley cell
strength, tropical deep convection, and the northern hemisphere
wintertime flow field.   
Many of these responses improved the climate simulation.
Thus our study isolates some fundamental climate statistics in the 
tropics and extratropics that are partially controlled by features of
tropical anvil not represented in most current GCM anvil
parameterizations.   
Accounting for these features should be a high priority for future GCM
cloud parameterizations.

% Balance preprint columns
\balance

% Appendices

% Acknowledgements
\acknowledgments
CSZ gratefully acknowledges discussions with B.~Briegleb and
R.~Saravanan.
G.~Branstator provided insight interpreting
Figure~\ref{fig:pres_8589_Z2TEST}.   
P.~Rasch provided helpful guidance with cloud parameterization. 
This work was supported in part by NASA Earth Observing System project
W-17,661 and by DOE Atmospheric Radiation Measurements Program grant
DE-AI05-92ER61376.

% Bibliography
\bibliographystyle{agu}
\bibliography{bib}

\end{document}

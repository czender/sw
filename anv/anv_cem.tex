% $Id$

% Purpose: ANV parameterization development paper.

\ifphdcsz{}{

%\documentclass[twoside,agupp]{aguplus}
%\documentclass[agupp]{aguplus}
\documentclass[twoside,agums]{aguplus}
%\documentclass[jgrga]{aguplus}

%  AGU++ OPTIONS
%\printfigures        % ADDS FIGURES AT END
%\doublecaption{35pc} % CAPTIONS PRINTED TWICE
\sectionnumbers      % TURNS ON SECTION NUMBERS
%\extraabstract       % ADDS SUPPLEMENTAL ABSTRACT
%\afour               % EUROPEAN A4 PAPER SIZE
%\figmarkoff          % SUPPRESS MARGINAL MARKINGS

%  AGUTeX OPTIONS AND ENTRIES
%\tighten             % TURNS OFF DOUBLE SPACING, has no effect with agupp
%\singlespace         % RESTORES SINGLE SPACING
%\doublespace         % RESTORES DOUBLE SPACING

\usepackage{graphicx} % defines \includegraphics*
\usepackage{ifthen} % Boolean and programming commands
\usepackage{longtable} % for long tables, like acronyms and symbols
\usepackage{tabularx} % for long tables, like acronyms and symbols

\usepackage{/home/zender/tex/csz} % all my local definitions
%\usepackage{jrn_abb} % Define abbreviations used in bib.bib
% Usage: % Usage: % Usage: \input{jgr_abb} % AGU-sanctioned journal title abbreviations

\def\aapgb{{\it Amer. Assoc. Petroleum Geologists Bull.}}
\def\adg{{\it Adv. Geophys.}}
\def\ajs{{\it Amer. J. Sci.}}
\def\amb{{\it Ambio}}
\def\amgb{{\it Arch. Meteorol. Geophys. Bioclimatl.}}
\def\ang{{\it Ann. Glaciol.}}
\def\angeo{{\it Ann. Geophys.}}
\def\apo{{\it Appl. Opt.}}
\def\areps{{\it Ann. Rev. Earth Planet. Sci.}}
\def\asr{{\it Adv. Space Res.}}
\def\ate{{\it Atmos. Environ.}}
\def\atf{{\it Atmosfera}}
\def\atms{{\it ACM Trans. Math Software}}
\def\ato{{\it Atmos. Ocean}}
\def\atr{{\it Atmos. Res.}}
\def\gbc{{\it Global Biogeochem. Cycles}} % csz
\def\blm{{\it Boundary-Layer Meteorol.}} % csz 
\def\bpa{{\it Beitr. Phys. Atmosph.}}
\def\bams{{\it Bull. Am. Meteorol. Soc.}}
\def\clc{{\it Clim. Change}}
\def\cld{{\it Clim. Dyn.}}
\def\com{{\it Computing}}
\def\dao{{\it Dyn. Atmos. Oceans}}
\def\dsr{{\it Deep-Sea Res.}}
\def\esr{{\it Earth Sci. Revs.}}
\def\gec{{\it Geosci. Canada}}
\def\gei{{\it Geofis. Int.}}
\def\gej{{\it Geogr. J.}}
\def\gem{{\it Geophys. Monogr.}}
\def\geo{{\it Geology}}
\def\grl{{\it Geophys. Res. Lett.}}
\def\ieeec{{\it IEEE Computer}}
\def\ijna{{\it IMA J. Numer. Anal.}}
\def\ijnmf{{\it Int. J. Num. Meteorol. Fl.}}
\def\jac{{\it J. Atmos. Chem.}}
\def\jacm{{\it J. Assoc. Comput. Mach.}}
\def\jam{{\it J. Appl. Meteorol.}}
\def\jas{{\it J. Atmos. Sci.}}
\def\jatp{{\it J. Atmos. Terr. Phys.}}
\def\jcam{{\it J. Climate Appl. Meteorol.}}
\def\jchp{{\it J. Chem Phys.}}
\def\jcis{{\it J. Coll. I. Sci.}}
\def\jcl{{\it J. Clim.}}
\def\jcp{{\it J. Comput. Phys.}}
\def\jfm{{\it J. Fluid Mech.}}
\def\jgl{{\it J. Glaciol.}}
\def\jgr{{\it J. Geophys. Res.}}
\def\jgs{{\it J. Geol. Soc. London}}
\def\jme{{\it J. Meteorol.}}
\def\jmr{{\it J. Marine Res.}}
\def\jmsj{{\it J. Meteorol. Soc. Jpn.}}
\def\josa{{\it J. Opt. Soc. A}}
\def\jpo{{\it J. Phys. Oceanogr.}}
\def\jqsrt{{\it J. Quant. Spectrosc. Radiat. Transfer}}
\def\jpca{{\it J. Phys. Chem. A}}
\def\lnc{{\it Lett. Nuov. C}}
\def\mac{{\it Math. Comp.}}
\def\map{{\it Meteorol. Atmos. Physics.}}
\def\mem{{\it Meteorol. Mag.}}
\def\mnras{{\it Mon. Not. Roy. Astron. Soc.}} 
\def\mwr{{\it Mon. Weather Rev.}} 
\def\nat{{\it Nature}}
\def\pac{{\it Parallel Computing}}
\def\pag{{\it Pure Appl. Geophys.}}
\def\pal{{\it Paleoceanography}}
\def\pht{{\it Physics Today}}
\def\pieee{{\it Proc. IEEE}}
\def\pla{{\it Phys. Lett. A}}
\def\ppp{{\it Paleogeogr. Paleoclim. Paleoecol.}}
\def\pra{{\it Phys. Res. A}}
\def\prd{{\it Phys. Rev. D}}
\def\prl{{\it Phys. Rev. L}}
\def\pss{{\it Planet. Space Sci.}}
\def\ptrsl{{\it Phil. Trans. R. Soc. Lond.}}
\def\qjrms{{\it Q. J. R. Meteorol. Soc.}}
\def\qres{{\it Quat. Res.}}
\def\qsr{{\it Quatern. Sci. Rev.}}
\def\reg{{\it Rev. Geophys.}}
\def\rgsp{{\it Revs. Geophys. Space Phys.}}
\def\rpp{{\it Rep. Prog. Phys.}}
\def\sca{{\it Sci. Amer.}}
\def\sci{{\it Science}}
\def\sjna{{\it SIAM J. Numer. Anal.}}
\def\sjssc{{\it SIAM J. Sci. Stat. Comput.}}
\def\tac{{\it Theor. Appl. Climatl.}}
\def\tel{{\it Tellus}}
\def\wea{{\it Weather}}

%SIAM Review: (Society for Industrial and Applied Mathematics)
%       J. on Computing
%       J. on Control and Optimization
%       J. on Algebraic and Discrete Methods
%       J. on Numerical Analysis 
%       J. on Scientific and Statistical Computing



 % AGU-sanctioned journal title abbreviations

\def\aapgb{{\it Amer. Assoc. Petroleum Geologists Bull.}}
\def\adg{{\it Adv. Geophys.}}
\def\ajs{{\it Amer. J. Sci.}}
\def\amb{{\it Ambio}}
\def\amgb{{\it Arch. Meteorol. Geophys. Bioclimatl.}}
\def\ang{{\it Ann. Glaciol.}}
\def\angeo{{\it Ann. Geophys.}}
\def\apo{{\it Appl. Opt.}}
\def\areps{{\it Ann. Rev. Earth Planet. Sci.}}
\def\asr{{\it Adv. Space Res.}}
\def\ate{{\it Atmos. Environ.}}
\def\atf{{\it Atmosfera}}
\def\atms{{\it ACM Trans. Math Software}}
\def\ato{{\it Atmos. Ocean}}
\def\atr{{\it Atmos. Res.}}
\def\gbc{{\it Global Biogeochem. Cycles}} % csz
\def\blm{{\it Boundary-Layer Meteorol.}} % csz 
\def\bpa{{\it Beitr. Phys. Atmosph.}}
\def\bams{{\it Bull. Am. Meteorol. Soc.}}
\def\clc{{\it Clim. Change}}
\def\cld{{\it Clim. Dyn.}}
\def\com{{\it Computing}}
\def\dao{{\it Dyn. Atmos. Oceans}}
\def\dsr{{\it Deep-Sea Res.}}
\def\esr{{\it Earth Sci. Revs.}}
\def\gec{{\it Geosci. Canada}}
\def\gei{{\it Geofis. Int.}}
\def\gej{{\it Geogr. J.}}
\def\gem{{\it Geophys. Monogr.}}
\def\geo{{\it Geology}}
\def\grl{{\it Geophys. Res. Lett.}}
\def\ieeec{{\it IEEE Computer}}
\def\ijna{{\it IMA J. Numer. Anal.}}
\def\ijnmf{{\it Int. J. Num. Meteorol. Fl.}}
\def\jac{{\it J. Atmos. Chem.}}
\def\jacm{{\it J. Assoc. Comput. Mach.}}
\def\jam{{\it J. Appl. Meteorol.}}
\def\jas{{\it J. Atmos. Sci.}}
\def\jatp{{\it J. Atmos. Terr. Phys.}}
\def\jcam{{\it J. Climate Appl. Meteorol.}}
\def\jchp{{\it J. Chem Phys.}}
\def\jcis{{\it J. Coll. I. Sci.}}
\def\jcl{{\it J. Clim.}}
\def\jcp{{\it J. Comput. Phys.}}
\def\jfm{{\it J. Fluid Mech.}}
\def\jgl{{\it J. Glaciol.}}
\def\jgr{{\it J. Geophys. Res.}}
\def\jgs{{\it J. Geol. Soc. London}}
\def\jme{{\it J. Meteorol.}}
\def\jmr{{\it J. Marine Res.}}
\def\jmsj{{\it J. Meteorol. Soc. Jpn.}}
\def\josa{{\it J. Opt. Soc. A}}
\def\jpo{{\it J. Phys. Oceanogr.}}
\def\jqsrt{{\it J. Quant. Spectrosc. Radiat. Transfer}}
\def\jpca{{\it J. Phys. Chem. A}}
\def\lnc{{\it Lett. Nuov. C}}
\def\mac{{\it Math. Comp.}}
\def\map{{\it Meteorol. Atmos. Physics.}}
\def\mem{{\it Meteorol. Mag.}}
\def\mnras{{\it Mon. Not. Roy. Astron. Soc.}} 
\def\mwr{{\it Mon. Weather Rev.}} 
\def\nat{{\it Nature}}
\def\pac{{\it Parallel Computing}}
\def\pag{{\it Pure Appl. Geophys.}}
\def\pal{{\it Paleoceanography}}
\def\pht{{\it Physics Today}}
\def\pieee{{\it Proc. IEEE}}
\def\pla{{\it Phys. Lett. A}}
\def\ppp{{\it Paleogeogr. Paleoclim. Paleoecol.}}
\def\pra{{\it Phys. Res. A}}
\def\prd{{\it Phys. Rev. D}}
\def\prl{{\it Phys. Rev. L}}
\def\pss{{\it Planet. Space Sci.}}
\def\ptrsl{{\it Phil. Trans. R. Soc. Lond.}}
\def\qjrms{{\it Q. J. R. Meteorol. Soc.}}
\def\qres{{\it Quat. Res.}}
\def\qsr{{\it Quatern. Sci. Rev.}}
\def\reg{{\it Rev. Geophys.}}
\def\rgsp{{\it Revs. Geophys. Space Phys.}}
\def\rpp{{\it Rep. Prog. Phys.}}
\def\sca{{\it Sci. Amer.}}
\def\sci{{\it Science}}
\def\sjna{{\it SIAM J. Numer. Anal.}}
\def\sjssc{{\it SIAM J. Sci. Stat. Comput.}}
\def\tac{{\it Theor. Appl. Climatl.}}
\def\tel{{\it Tellus}}
\def\wea{{\it Weather}}

%SIAM Review: (Society for Industrial and Applied Mathematics)
%       J. on Computing
%       J. on Control and Optimization
%       J. on Algebraic and Discrete Methods
%       J. on Numerical Analysis 
%       J. on Scientific and Statistical Computing



 % AGU-sanctioned journal title abbreviations

\def\aapgb{{\it Amer. Assoc. Petroleum Geologists Bull.}}
\def\adg{{\it Adv. Geophys.}}
\def\ajs{{\it Amer. J. Sci.}}
\def\amb{{\it Ambio}}
\def\amgb{{\it Arch. Meteorol. Geophys. Bioclimatl.}}
\def\ang{{\it Ann. Glaciol.}}
\def\angeo{{\it Ann. Geophys.}}
\def\apo{{\it Appl. Opt.}}
\def\areps{{\it Ann. Rev. Earth Planet. Sci.}}
\def\asr{{\it Adv. Space Res.}}
\def\ate{{\it Atmos. Environ.}}
\def\atf{{\it Atmosfera}}
\def\atms{{\it ACM Trans. Math Software}}
\def\ato{{\it Atmos. Ocean}}
\def\atr{{\it Atmos. Res.}}
\def\gbc{{\it Global Biogeochem. Cycles}} % csz
\def\blm{{\it Boundary-Layer Meteorol.}} % csz 
\def\bpa{{\it Beitr. Phys. Atmosph.}}
\def\bams{{\it Bull. Am. Meteorol. Soc.}}
\def\clc{{\it Clim. Change}}
\def\cld{{\it Clim. Dyn.}}
\def\com{{\it Computing}}
\def\dao{{\it Dyn. Atmos. Oceans}}
\def\dsr{{\it Deep-Sea Res.}}
\def\esr{{\it Earth Sci. Revs.}}
\def\gec{{\it Geosci. Canada}}
\def\gei{{\it Geofis. Int.}}
\def\gej{{\it Geogr. J.}}
\def\gem{{\it Geophys. Monogr.}}
\def\geo{{\it Geology}}
\def\grl{{\it Geophys. Res. Lett.}}
\def\ieeec{{\it IEEE Computer}}
\def\ijna{{\it IMA J. Numer. Anal.}}
\def\ijnmf{{\it Int. J. Num. Meteorol. Fl.}}
\def\jac{{\it J. Atmos. Chem.}}
\def\jacm{{\it J. Assoc. Comput. Mach.}}
\def\jam{{\it J. Appl. Meteorol.}}
\def\jas{{\it J. Atmos. Sci.}}
\def\jatp{{\it J. Atmos. Terr. Phys.}}
\def\jcam{{\it J. Climate Appl. Meteorol.}}
\def\jchp{{\it J. Chem Phys.}}
\def\jcis{{\it J. Coll. I. Sci.}}
\def\jcl{{\it J. Clim.}}
\def\jcp{{\it J. Comput. Phys.}}
\def\jfm{{\it J. Fluid Mech.}}
\def\jgl{{\it J. Glaciol.}}
\def\jgr{{\it J. Geophys. Res.}}
\def\jgs{{\it J. Geol. Soc. London}}
\def\jme{{\it J. Meteorol.}}
\def\jmr{{\it J. Marine Res.}}
\def\jmsj{{\it J. Meteorol. Soc. Jpn.}}
\def\josa{{\it J. Opt. Soc. A}}
\def\jpo{{\it J. Phys. Oceanogr.}}
\def\jqsrt{{\it J. Quant. Spectrosc. Radiat. Transfer}}
\def\jpca{{\it J. Phys. Chem. A}}
\def\lnc{{\it Lett. Nuov. C}}
\def\mac{{\it Math. Comp.}}
\def\map{{\it Meteorol. Atmos. Physics.}}
\def\mem{{\it Meteorol. Mag.}}
\def\mnras{{\it Mon. Not. Roy. Astron. Soc.}} 
\def\mwr{{\it Mon. Weather Rev.}} 
\def\nat{{\it Nature}}
\def\pac{{\it Parallel Computing}}
\def\pag{{\it Pure Appl. Geophys.}}
\def\pal{{\it Paleoceanography}}
\def\pht{{\it Physics Today}}
\def\pieee{{\it Proc. IEEE}}
\def\pla{{\it Phys. Lett. A}}
\def\ppp{{\it Paleogeogr. Paleoclim. Paleoecol.}}
\def\pra{{\it Phys. Res. A}}
\def\prd{{\it Phys. Rev. D}}
\def\prl{{\it Phys. Rev. L}}
\def\pss{{\it Planet. Space Sci.}}
\def\ptrsl{{\it Phil. Trans. R. Soc. Lond.}}
\def\qjrms{{\it Q. J. R. Meteorol. Soc.}}
\def\qres{{\it Quat. Res.}}
\def\qsr{{\it Quatern. Sci. Rev.}}
\def\reg{{\it Rev. Geophys.}}
\def\rgsp{{\it Revs. Geophys. Space Phys.}}
\def\rpp{{\it Rep. Prog. Phys.}}
\def\sca{{\it Sci. Amer.}}
\def\sci{{\it Science}}
\def\sjna{{\it SIAM J. Numer. Anal.}}
\def\sjssc{{\it SIAM J. Sci. Stat. Comput.}}
\def\tac{{\it Theor. Appl. Climatl.}}
\def\tel{{\it Tellus}}
\def\wea{{\it Weather}}

%SIAM Review: (Society for Industrial and Applied Mathematics)
%       J. on Computing
%       J. on Control and Optimization
%       J. on Algebraic and Discrete Methods
%       J. on Numerical Analysis 
%       J. on Scientific and Statistical Computing





% NB: The \lefthead and \righthead will be automatically uppercased by
% the jgrga documentstyle 
\lefthead{Zender and Kiehl}
%\righthead{Development of the ANV Prognostic Ice Scheme for GCMs}
\righthead{Prognostic Ice Scheme for GCMs based on Tropical Anvils}
\received{date1}
\revised{date2}
\accepted{date3}
\journalid{JGRD}{Journal date}
\articleid{page1}{page2}
\paperid{94JZ12345}
% The $ in the following line screws up the hilit19 highlighting
%\ccc{0000-0000/00/94JZ-12345\$05.00}
\ccc{0000-0000/00/94JZ-12345\05.00}
% \cpright{PD}{1994}
% \cpright{Crown}{1994}
% (No \ccc{} for Crown copyrights.)
\cpright{AMS}{1996}

\authoraddr{J. T. Kiehl and Charles S. Zender, National Center for Atmospheric
Research, P.O. Box 3000, Boulder, CO 80307-3000}

%\slugcomment{Submitted to \jas, April~22, 1996. Copyright \copyright 1996 AMS.}
\slugcomment{\today}

\begin{document}
} % not phdcsz

% NB: title must be lowercased by hand. The documentstyle does not enforce it.
\def\paperchaptertitle{A prognostic ice cloud parameterization for
GCMs based on modeled and observed behavior of tropical anvils}
\ifphdcsz{\chapter{\paperchaptertitle}\label{chap:cem}}{\title{\paperchaptertitle}}
\ifphdcsz{}{\author{Charles S. Zender and J. T. Kiehl}}
\ifphdcsz{}{\affil{National Center for Atmospheric Research, Boulder, Colorado}}

\ifphdcsz{\section{Abstract}}{\begin{abstract}}
We document the development of a parameterization of cirrus cloud
suitable for implementation in general circulation models (GCMs).
The parameterization, called ANV (for anvil), comprises convective 
(sub-gridscale) and stratiform (resolved) condensation.
Both components represent the bulk processes of deposition,
sublimation, and precipitation of ice mass without assuming or
predicting the ice particle size. 
We employed a cumulus ensemble model (CEM) to determine parameters
governing ice generation in the convective case, and a size resolving
column model in the stratiform case.
As a result ANV predicts the ice budget of a GCM atmospheric column 
from the convective mass flux, updraft, temperature, humidity, and
inferred budgets of observed convective systems. 
Statistics from control runs of a GCM and the CEM revealed fundamental
distinctions in the convective behavior of each model in a tropical
convecting regime.
Sampling the CEM dataset using convective criteria which minimize
systematic differences in the mass flux profile between models permits
ANV to be mapped from the high to the low resolution model in a
physically consistent manner.
\ifphdcsz{}{\end{abstract}}

\section{Introduction}\label{sec:cem_intro}

Ice cloud has a significant impact on the radiative budget of the
earth and presents a great challenge to modeling its climate.
Ice cloud contributes more than any other cloud type to total
continental cloud cover and rank second to marine stratocumulus in
tropical oceanic coverage \cite[]{WHL86,WHL88}.
Deep convection in the tropics generates anvils in the coldest part of
the troposphere, over the warmest SST---ideal conditions for trapping
longwave emission from the surface.
In the tropics, net cloud forcing is a small residual between
shortwave and longwave cloud forcing (SWCF and LWCF) whose magnitude
largely depends on two properties: cloud mass and height
\cite[]{Kie941}. 
A small bias in either property can change the sign of feedbacks
between cloud forcing, SST, and convection which control the
hydrologic cycle. 
Satellite datasets exist that document the direct influence of cloud
on the global energy budget \cite[e.g.,][]{RCH89}, but their role in
the hydrologic cycle remains largely speculative.
\cite{SuL931} argue that the observed vertical distribution of tropical
water vapor can be maintained by sublimating hydrometeors detrained
from cirrus. 
For these reasons, the prediction of ice cloud in a general
circulation model (GCM) should include the essential physics that
determine tropical anvil vertical and geographic distribution,
longevity, and mass.

The treatment of ice cloud by GCMs falls into two classes: diagnostic
and prognostic. 
Diagnostic schemes employ empirical relationships among large scale
variables (e.g., temperature and humidity) to obtain the mixing ratio
of condensed water \qc\ \cite[e.g.,][]{KBB96}, or a proxy, such as cloud
optical depth \cite[]{HRR83}. 
Their prescribed, steady state nature makes diagnostic schemes 
computationally efficient.
Typically some form of entraining plume is implicitly assumed,
so that the \qc\ profile increases with height above cloud base more 
slowly than, but similar to, an adiabatic condensed water profile.

While variations on it can perform well for entraining plumes, an
adiabatic profile is ill-suited as a basis for predicting the vertical
distribution of condensate resulting from deep convection.
Deep convection is characterized by an active core region much smaller
then the associated mesoscale outflow (anvil).
The relative area of a core to its anvil is $< 10\%$ \cite[]{FDR90}.
Typical anvil depth is 5--10~km (resolvable by current GCMs)
\cite[]{WSS931}. 
In contrast to the monotonic adiabatic profile, observations and cloud
resolving simulations \cite[]{WSS932,SLT94,GMK95} show the
area-averaged vertical \qc\ profile of a mature (or time-averaged)
anvil system varies little (and can increase) from freezing level to
$\sim 300$~mb, above which \qc\ decreases rapidly.
Furthermore, detached anvils can project a radiative influence long
after deep convection ceases \cite[]{ALV88}.
Representing this convective-radiative hysteresis is an important
advantage of prognostic over diagnostic schemes.
The fidelity of diagnostic schemes could be improved by
transmitting information about the synoptic environment to them, but
this is a complicated, non-local procedure.

Representing a wide spectrum of cloud types is more naturally
accomplished by prognostic schemes, which forecast \qc\ each timestep 
with an equation incorporating physically based sources, sinks, and
transport. 
Our interest is the sensitivity of the climate system to tropical
anvils.
A forecast \qc\ allows hysteresis, and so can maintain long-lived
anvils after the cessation of convection.
Prognostic cloud schemes occupy a spectrum from relatively simple,
single category bulk condensate treatments \cite[e.g.,][]{Sun88}, to
complex multiple category (droplet, rain, graupel, crystal, snow etc.) 
treatments that also predict size spectra \cite[e.g.,][]{FTV89}.
Complex schemes include more physical processes whose details are
poorly understood (e.g., homogeneous nucleation) and will remain
difficult to test against observations for many years.
%The role of a specific physical process in the climate system can be
%better understood by selectively enabling or disabling the process in
%a GCM \cite[e.g., ice crystal terminal velocities,][]{SeM93}.
Prior studies of the effects of cirrus employed either diagnostic
cloud schemes \cite[e.g.,][]{SlS88,RHD89}, or prognostic schemes
\cite[e.g.,][]{SeM93,SRB94,FoR962,DYK96} that detrain (with a
specified, sometimes 100\%, efficiency) condensate predicted by a
moist convection parameterization into the stratiform anvil.
In the latter models, detrained condensate is consistent with the
moist convective parameterization, but may not include condensate
formation from secondary circulations. 
To our knowledge, only \cite{Don93} explicitly accounts for condensate
formed by mesoscale circulations outside the deep convective core
(perhaps 25--40\% of total anvil condensate \cite[]{LeH80,GaH83}).  
The main purpose of this study is to develop a prognostic cloud scheme
in which deep convection results in the generation and maintenance of
mesoscale anvils with realistic \qc\ profiles and lifetimes.

Previous studies \cite[]{Sli87,Sun88,XuK91} agree that the total 
cloud field is best parameterized as the net result of convective and
stratiform cloud processes.
Long term, multi-regional datasets on cloud microphysics, radiation,
and mesoscale dynamical features needed to develop and fully test 
prognostic ice cloud parameterizations do not exist.
In the interim many in the GCM community feel the best tool available
for studying deep convection and associated mesoscale outflow is the
cloud-resolving cumulus ensemble model (CEM) \cite[]{Bro94}.
Current computational resources allow one to define CEM domains 
that extend beyond the scale of a typical GCM grid box. 
Thus, these models serve as a bridge from the cloud scale up to the
scale of the GCM. 
For prognosing stratiform ice associated with large scale ascent and
cooling, a size resolving column model (SRCM) may suffice.
This study combines results from SRCM, CEM, and GCM simulations
with empirical estimates of convective system hydrologic budgets
into a parameterization scheme for ice cloud dubbed ANV (for anvil).  
A second paper \cite[]{ZeK962} describes the sensitivity of the
simulated climate of a new version of the National Center for
Atmospheric Research Community Climate Model (NCAR CCM) to the
radiative effects of the enhanced upper tropospheric ice amount and
fraction predicted by ANV.  

The paper is organized as follows: Section~\ref{sec:cem} summarizes
the 24~day numerical experiment performed with the CEM.
We analyze the CEM dataset and develop the convective component of ANV
in Section~\ref{sec:anv}.
The technique used to mesh the convective portion of ANV with a GCM in
a physically consistent manner is shown in Section~\ref{sec:gcm}. 
Section~\ref{sec:srcm} describes the SRCM and documents the stratiform
component of ANV.
Section~\ref{sec:cem_disc} contains concluding remarks.

\section{Cumulus Ensemble Model}\label{sec:cem}

We employed the CEM developed by \cite{Tri92} for our studies.
A more comprehensive review of our CEM simulation is presented by
\cite{GMK95}.
The model is nonhydrostatic, fully non-Boussinesq and can be run in
either a 2-dimensional or 3-dimensional configuration.
The model employs a six water category (cloud droplets, rain droplets, 
pristine ice crystals, snow crystals, aggregates and graupel) bulk
formulation for microphysics \cite[]{FTV89}.
The model also includes a parameterization for longwave and shortwave
radiation; a turbulent kinetic energy closure; and a Businger-type
parameterization for surface exchange.
The configuration of the simulations is analogous to the work of
\cite{SLT94}.
The model run is 2-dimensional with an outer domain of 900~km and a
horizontal resolution of 1~km. 
The vertical domain extends to 25~km with a variable vertical
resolution, with a nominal resolution in the mid-troposphere of 200~m.
The horizontal boundary condition is periodic.
The model time step is 10~s.
A constant surface temperature of 301~\dgr K is assumed to represent 
Western Pacific conditions.

The model is forced with steady conditions from the Marshall Islands
\cite[]{YEC73}.
The imposed zonal wind and vertical updraft profiles are shown in
Figure~\ref{fig:cem_forcing}. 
\begin{figure}
\begin{center}
\includegraphics*[width=\halfwidth]{/data/zender/fgr/phd/cem_forcing.eps}
\end{center}
\caption[Imposed vertical profiles of large scale forcing in the
CEM]{
Imposed vertical profiles of large scale forcing in the
CEM: Horizontal wind shear (solid) and large scale ascent
(dotted).\label{fig:cem_forcing}} 
\end{figure}
The imposed zonal wind profile indicates a strong low level jet
which implies strong shear in the lowest 2--3~km near the surface. 
The imposed vertical velocity profile peaks in the mid to upper
troposphere at 3~\cmxs\ upward motion.

The CEM simulation was run for 24~days. 
The length of integration was chosen to insure that the model reached an
equilibrium with the large scale forcing. 
The time evolution of domain-average temperature and precipitable
water showed the model takes roughly 15 days to reach equilibrium
with the large scale forcing \cite[Figure~4,][]{GMK95}. 
Figure~\ref{fig:cem_snapshot} shows the snapshot of total condensate
mixing ratio \qc\ at the end of day one.
\begin{figure*}
\begin{center}
\includegraphics*[width=\hsize]{/data/zender/fgr/phd/cem_snapshot.eps}
\end{center}
\caption[Snapshot of condensed water mixing ratio \qc\ at the 
end of day~1 of the CEM simulation]{Snapshot of condensed water mixing
ratio \qc\ (\gxkg) at the end of day~1 of the CEM
simulation.\label{fig:cem_snapshot}} 
\end{figure*}
The figure indicates two convective cores located near 50~km and 350~km
with extensive upper tropospheric cloud cover.  
The time-averaged (days 4--24) and domain-averaged concentrations for
liquid water, ice water, and total condensed water are shown in
Figure~\ref{fig:cem_fice}. 
\begin{figure}
\begin{center}
\includegraphics*[width=\halfwidth]{/data/zender/fgr/phd/cem_fice.eps}
\end{center}
\caption[Time- and domain-average ice, liquid, and
total condensate from the 24 day CEM simulation]{
Time- and domain-average ice (dotted), liquid (dashed), and total
(solid) condensate (\gxmS) from the 24 day CEM simulation. 
Horizontal arrows indicate the 500~mb level.\label{fig:cem_fice}}    
\end{figure}
Above 500~mb, the cloud water is dominated by ice,  with ice
concentrations between 0.1 and 0.14~\gxmC.
Note that the model predicts a very sharp transition between liquid and
ice phase.
In terms of temperature this transition occurs over only 5~\dgr K. 

As discussed in the next section, we have employed the results from 
this CEM simulation to develop a prognostic scheme for anvil cloud. 
The advantage of using this model is it provides a self-consistent and
comprehensive data base for testing and developing  parameterizations
for the GCM.  
However, we recognize the CEM model should itself be validated with
observational data. 
We hope these data will become available in the not too distant future.

\section{Parameterization of Ice Condensate Resulting from Convection}\label{sec:anv}

\subsection{Source Processes Deduced from the CEM Dataset}

A number of dynamic and thermodynamic fields were considered as
proxies for the generation of anvil ice water. 
Time series of the mean convective (cumulus) mass flux, relative
humidity, cloud amount (fraction), and ice water content were
compared to the time tendency of total column ice water path, \IWPdot.
\cite{XuK91} concluded convective mass flux $M$ best predicts
tropical convective cloud amount and the IWC of individual anvil
layers.  
Our analyses indicate $M$ also best predicts \IWPdot.
This conclusion is widely applicable to GCMs because many, if not
most, cumulus convection parameterizations predict $M$
\cite[e.g.,][]{ArS74,Tie93,ZhM952}. 
We restrict the present study to the actual parameterization of
\IWPdot\ in terms of $M$.
In the context of the CEM, $M$ is defined to be the horizontally
averaged vertical density flux from all grid cells which exceed a
critical updraft ($w > \wc = 1$~\mxs) and a critical condensate amount
($\qc > \qcc = .05$~\gxkg).
The optimal choice of \wc\ and \qcc\ depends on the GCM convective
parameterization, see Section~\ref{sec:gcm} below.

Figure~\ref{fig:cem_mc_diwpdt}\ shows the simultaneous variation of
$M$ at 500~mb and \IWPdot\ through the first day of the simulation.  
\begin{figure}
\begin{center}
\includegraphics*[height=.25\textheight]{/data/zender/fgr/phd/cem_mc_diwpdt_a.eps}\vfill
\includegraphics*[height=.25\textheight]{/data/zender/fgr/phd/cem_mc_diwpdt_b.eps}\vfill
\includegraphics*[height=.25\textheight]{/data/zender/fgr/phd/cem_mc_diwpdt_c.eps}\vfill
\end{center}
\caption[Evolution and correlation of 500~mb $M$ and \IWPdot\ during 
the first 24 hours of the CEM simulation]{
(a) Evolution of 500~mb $M$ (\gxmS) (solid) and \IWPdot\ (dashed)
during the first 24 hours of the CEM simulation. 
\IWPdot\ is expressed in \gxmSgcmt, i.e., mass change per GCM
timestep. 
(b) The first 2.5 hours of (a).
(c) Linear correlation and least squares fit of 500~mb $M$ to \IWPdot\
from (b).\label{fig:cem_mc_diwpdt}}
\end{figure}
The initial anvil formation, lasting about six hours, occured as a
prescribed thermal instability triggered concentrated convective
updrafts (cores) which detrained frozen condensate into a cirrus anvil
above 500~mb.
The variation of \IWPdot\ is strongest during convectively
active periods, and subdued during the convectively quiescent period
($400 < t < 900$~min). 
Non-convective formation of anvil ice is also evident during the
quiescent period. 
Secondary circulations and weaker convection ($w < 1$~\mxs) apparently 
result in significant deposition.
These mesoscale processes are poorly understood, but consistent with
budget studies \cite[]{LeH80}.
At $t \approx 900$~min two widely separated, westward travelling
convective systems developed and maintained distinct anvils.
Deep convective activity similar to $900 < t < 1300$~min occurred
approximately daily and maintained extensive anvil coverage for the
remaining three weeks.

In order to isolate the processes controlling ice generation
from destruction we focused on the initial hours of anvil formation,
when a single convective tower dominated the mass budget of the entire
domain. 
Figures~\ref{fig:cem_mc_diwpdt}b and \ref{fig:cem_mc_diwpdt}c show
initial anvil formation is attributable to the strength of $M$ near
the freezing level. 
We define \pc\ as the pressure $p$ at which $M$ most strongly
correlates with \IWPdot. 
Similar analyses (not shown) which varied $p$ from 750 to 250~mb,
confirmed $\pc \approx 500$~mb (roughly 50~mb above the tropical
freezing level).
Deep convection was perennial in the three week simulation but 
$\Mc \equiv M(\pc)$ never exceeded 80~\gxmSs.
This adds confidence the correlation between \Mc\ and \IWPdot\ in
Figure~\ref{fig:cem_mc_diwpdt}c is robust over the realizable range of
convective intensities.
On this basis we adopted \Mc\ as the fundamental predictor of anvil
ice generation by convective processes.   
This is implemented in ANV as the term $c_1\Mc$ in forecast
equation (\ref{eqn:qi_cnv}) for \qi (shown below).
Ice generated by convection may be augmented by stratiform condensate
(detailed in Section~\ref{sec:srcm}) in non-convecting layers.

The current ANV implementation apportions new IWP ($c_1\Mc\Delta t$)
evenly among all convecting layers where $T < 0~^\circ$C.
Experimental versions of ANV, employing the convection scheme of
\cite{Hac94}, apportioned new IWP weighted by local $M$, so that
IWP was distributed nonuniformly.
This method tended to overpredict \qi\ at near freezing level at the
expense of too little \qi\ in the upper levels of anvils.
%Figures~\ref{cem_snapshot} and \ref{cem_fice} show that anvil ice
%appears to peak well above the freezing level.
Other methods of distributing IWP nonuniformly in the vertical are
expected to become more practical as data from field observations,
especially in the tropics, accumulates.
For example, new IWP could be vertically partitioned based on observed
IWC profiles.

\subsection{Sink Processes Deduced from the CEM Dataset and
Observations} 

A forecast equation for \qi\ requires adequate representation of ice 
removal processes to balance the convective detrainment.
We divide sink processes into three terms: precipitation, sublimation,
and sublimation of precipitation. 
Isolating the specific processes causing anvil decay in the CEM
simulation was difficult due to persistent convection and secondary
circulations in the domain (Figure~\ref{fig:cem_mc_diwpdt}a). 
Nevertheless, regressions of relative humidity, saturation deficit,
cloud amount, and IWC against \IWPdot\ and \IWCdot\ were performed
to identify the most promising proxies for anvil decay. 

Precipitation (including virga and mesoscale downdrafts) is the 
strongest sink process in anvils \cite[]{LeH80,GaJ91}. 
Cloud amount at 500~mb (determined using the method of \cite{SLT94})
was clearly related to \IWPdot\ throughout the convectively quiescent
period.
Despite this, cloud amount was discarded as a proxy for precipitation 
because it is difficult to define consistently from CEM to GCM. 
There are three advantages meriting \IWC\ as the chief predictor for
the rate of precipitation: 
First, \cite{Hey77} found the precipitation rate in deep continental
ice cloud was proportional to $\IWC^{1.16}$.
Second, \IWC\ often nicely (anti-)correlated with \IWCdot\ in the CEM
(correlation was often enhanced by letting \IWCdot\ lag \IWC\ by
40--60~min).
However, no useful rate coefficient could be isolated because the
archived data lacked necessary fields.
Third, \IWC\ and precipitation rate are independent of the horizontal 
extent of a homogeneous anvil; thus a rate coefficient derived from a
SRCM applies to a GCM. 
On this basis ANV converts ice to precipitation (snow) at a rate
proportional to anvil mass, $c_3$\qi.  
The value of the free parameter $c_3^{-1}$, the $e$-folding time for
ice removal due to precipitation, is discussed in
Section~\ref{sec:srcm} below.   
Currently, precipitation in ANV instantly falls through the column.
\cite{Smi90} describes a simple method for implementing realistic 
fall-speeds, which can significantly change the amount of ice in a GCM
\cite[]{SeM93}. 
A Kessler-style formulation sublimes precipitation (virga)
\cite[e.g.,][]{Sun88}.

Anvil decay through local sublimation includes the effects of
entrainment of dry air into cloudy, and sub-gridscale distributions
of cloudiness and subsaturated air.
Neither the ice saturation deficit ($\qvi - \qv$) nor the relative
humidity (\RH) correlated with \IWCdot\ after the upper troposphere
equilibrated to $\RH \sim 100\%$ \cite[]{GMK95}.
Rather than resorting to a sub-gridscale model of \qvi\ distribution
(i.e., cloud fraction), we take a semi-empirical approach like
\cite{Don93}. 
Following \cite{LeH80}, many studies have inverted observations
of tropical and continental convective systems to infer the spatially
and temporally averaged moisture balance between the convective and
stratiform regions.
\cite{GaJ91}, in their Table~2, summarize the ratio between mesoscale
(anvil) evaporation ($E_{me}$) and precipitation ($R_m$) from five
such studies. 
The average $E_{me}/R_m$ in tropical and continental systems is .22
and .28, respectively.
On this basis ANV assumes that the ratio of anvil sublimation to
precipitation is  $c_2/c_3 = .25$ in every layer and timestep.

\subsection{Prognostic Thermodynamic Equations of ANV}

Figure~\ref{fig:anv_concept} shows a conceptual model of the ice
budget in ANV.  
\begin{figure}
\begin{center}
\includegraphics*[width=\halfwidth]{/home/zender/anv/anv_concept.eps}
\end{center}
\caption[Conceptual model of the ice budget in ANV]{
Conceptual model of the ice budget in ANV. 
Free parameters $c_1$--$c_5$ and \pc\ are defined and discussed in
Sections~\ref{sec:anv}--\ref{sec:srcm} and used in 
equations
(\ref{eqn:qi_cnv}--\ref{eqn:qi_stb}).\label{fig:anv_concept}}   
\end{figure}
For a grid cell of density $\rho$,
thickness $\Delta z$, ice mixing ratio \qi, vapor mixing
ratio \qv, temperature $T$, ice saturation mixing ratio \qvi,
precipitation (snow) flux \Psnow, and large scale wind field 
$\vec u$, the thermodynamic couplings that define ANV are:
\begin{eqnarray}
\label{eqn:qi_cnv}
{D\qi \over Dt} & 
= & 
{c_1 \Mc \over \rho \Delta Z} - 
c_2 \qi - 
c_3 \qi \\
\label{eqn:qv}
{D\qv \over Dt} &
= & 
-{c_1 \Mc \over \rho \Delta Z} + 
c_2 \qi + 
c_4 \Bigl(1 - {\qi \over
\qvi}\Bigr)\Psnow^{1/2} \\
\label{eqn:t}
{DT \over Dt} &
= &
{\Lvi \over \cp}\Biggl[
{c_1 \Mc \over \rho \Delta Z} - 
c_2 \qi - 
c_4 \Bigl(1 - {\qi \over \qvi}\Bigr)\Psnow^{1/2}
\Biggr]
\end{eqnarray}
where $\Delta Z$ is the thickness of the convecting portion of the
column in which $T < 0~^\circ$C, \Lvi\ is the latent heat of
sublimation, and \cp\ is the specific heat of air at constant pressure.
The material derivatives on the LHS of
(\ref{eqn:qi_cnv}--\ref{eqn:t}) account for advection by $\vec u$. 
The first term on the RHS of (\ref{eqn:qi_cnv}) relates the generation
of total column ice to \Mc.
The second term represents local sublimation of the anvil due to
sub-gridscale entrainment and subsaturation.
The $c_3$ term in (\ref{eqn:qi_cnv}) converts ice to precipitation.
The $c_4$ term in (\ref{eqn:qv}) and (\ref{eqn:t}) acts to sublime
precipitation.
Terms not involving ice have been omitted from the RHS of
(\ref{eqn:qv}) and (\ref{eqn:t}).
Note that the detraining anvil completely covers a gridcell, i.e.,
cloud fraction is 1.
Table~\ref{tbl:anv_free_parm} summarizes all the parameters used by ANV.
%\begin{planotable}{cllr}
%% See CZPIII p. 89,90,136
%%\tablewidth{\hsize} % works for normal tables
%\tablewidth{41pc} % works for full width tables
%\tablecaption{Free parameters in ANV\label{tbl:anv_free_parm}}
%\tablehead{\colhead{} & \colhead{Meaning} & \colhead{Suggested Value} & \colhead{Source}}
%\tablenotetext{\textit{a}}{Average of values from Table~2, \cite{GaJ91}.}
%\tablenotetext{\textit{b}}{\cite{Sun88}. Value from NCAR CCM3.}
%\startdata
%$c_1$ & Relates \Mc\ to mesoscale detrainment & $2.4 \times 10^{-3}$ & CEM, Figure~\ref{fig:cem_mc_diwpdt}c \nl
%$c_2$ & Anvil sublimation rate & $.25c_3$ s$^{-1}$ & Empirical\tablenotemark{\textit{a}} \nl
%$c_3$ & Precipitation rate & $2.8 \times 10^{-4}$ s$^{-1}$ & SRCM, Figure~\ref{fig:srcm_decay_ss}\nl
%$c_4$ & Precipitation sublimation rate & $1 \times 10^{-5}$ m$^{-2}$~kg$^{1/2}$~s$^{-1/2}$ & Empirical\tablenotemark{\textit{b}} \nl
%$c_5$ & Relates stratiform \qidot\ to $w$, $T$ & -1.15 & SRCM, Figures~\ref{fig:srcm_w_ss},\ref{fig:srcm_t_ss} \nl
%\wc\ & Threshold $w$ for convection in CEM & 1 \mxs & CEM, GCM, Figure~\ref{fig:mc_hist} \nl
%\qcc\ & Threshold \qc\ for convection in CEM & .05~\gxkg & CEM, Empirical \nl
%\pc\ & Level to evaluate $M$ for (\ref{eqn:qi_cnv}--\ref{eqn:t}) & 500 mb & CEM, Figure~\ref{fig:cem_mc_diwpdt}b \nl
%\end{planotable}
%\begin{minipage}[t]{\hsize}
%\begin{sidewaystable}
%\begin{table*}
\ifphdcsz{\begin{sidewaystable}}{\begin{table*}}
\begin{center}
%\caption{Free parameters in ANV\label{tbl:anv_free_parm}}
\vspace{5pt}
\begin{tabularx}{7in}{cllr}
\hline
& & & \\[-5pt]
& Meaning & Suggested Value & Source \\[4pt]
\hline
& & & \\[-6pt]
$c_1$ & Relates \Mc\ to mesoscale detrainment & $2.4 \times 10^{-3}$ & CEM, Figure~\ref{fig:cem_mc_diwpdt}c \\
%$c_2$ & Anvil sublimation rate & $.25c_3$ s$^{-1}$ & Empirical \\
% XXX: reinsert footnotes for preprints
$c_2$ & Anvil sublimation rate & $.25c_3$ s$^{-1}$ &
Empirical\ifphdcsz{\footnote{Average of values from Table~2,
\cite{GaJ91}.}}{\tablenotemark{\textit{a}}} \\
%$c_2$ & Anvil sublimation rate & $.25c_3$ s$^{-1}$ &
%Empirical\footnote{Average of values from Table~2, \cite{GaJ91}.} \\ 
$c_3$ & Precipitation rate & $2.8 \times 10^{-4}$ s$^{-1}$ & SRCM, Figure~\ref{fig:srcm_decay_ss} \\
%$c_4$ & Precipitation sublimation rate & $1 \times 10^{-5}$
%m~kg$^{-1/2}$~s$^{-1/2}$ & Empirical \\
$c_4$ & Precipitation sublimation rate & $1 \times 10^{-5}$
m~kg$^{-1/2}$~s$^{-1/2}$ &
Empirical\ifphdcsz{\footnote{\cite{Sun88}. Value from NCAR CCM3
\cite[]{KBB96}.}}{\tablenotemark{\textit{b}}} \\  
$c_5$ & Relates stratiform \qidot\ to $w$, $T$ & $-1.15$ & SRCM, Figures~\ref{fig:srcm_w_ss},\ref{fig:srcm_t_ss} \\
\wc\ & Threshold $w$ for convection in CEM & 1 \mxs & CEM, GCM, Figure~\ref{fig:mc_hist} \\
\qcc\ & Threshold \qc\ for convection in CEM & .05~\gxkg & CEM, Empirical \\
\pc\ & Level to evaluate $M$ for (\ref{eqn:qi_cnv}--\ref{eqn:t}) & 500 mb & CEM, Figure~\ref{fig:cem_mc_diwpdt}b \\[4pt]
\hline
& & & \\[-8pt]
\end{tabularx}
\ifphdcsz{}{\tablenotetext{\textit{a}}{Average of values from Table~2, \cite{GaJ91}.}}
\ifphdcsz{}{\tablenotetext{\textit{b}}{\cite{Sun88}. Value from NCAR CCM3
\cite[]{KBB96}.}}
\end{center}
\caption{Free parameters in ANV\label{tbl:anv_free_parm}}
\ifphdcsz{\end{sidewaystable}}{\end{table*}}
%\end{minipage}
%\end{table*}
%\end{sidewaystable}

\section{Meshing ANV with a GCM}\label{sec:gcm}

Meshing an anvil parameterization from a CEM into a GCM only makes
sense if an equivalence can be demonstrated between the predictive
parameters in the respective models.
Since \Mc\ is the sole source of convectively generated ice in ANV
(\ref{eqn:qi_cnv}), the CEM $M$ profile should match the GCM $M$
profile to ensure ANV complements, not co-opts, the GCM convection
scheme.
The $M$ predicted by a GCM moist convection scheme sums the mass flux
from all sub-gridscale (non-resolved) convective elements (clouds).
Convection in the GCM context includes all unresolved mass flux
generated by moist adiabatic instabilities.
However, the continuous physical processes in a CEM do not distinguish 
between convection, mesoscale eddies, and large scale ascent.
As mentioned above, ANV employs a threshold updraft \wc\
and threshold condensate mixing ratio \qcc\ to select GCM-type
convection in the CEM; only gridcells exceeding these thresholds
contribute to CEM $M$.
The main criterion for setting these thresholds is to obtain
consistent $M$ between models (and, hopefully, observations) so that
implementations of (\ref{eqn:qi_cnv}--\ref{eqn:t}) in the GCM will
behave predictably.

Intercomparing the convective behavior of a CEM to a GCM raises a 
number of questions.
How can the influence of differing boundary conditions between the CEM 
(periodic, fixed SST) and a GCM region (open, prescribed SST) be
minimized? 
How might the internal variance of the models (their sensitivity to
initial conditions) skew the analyses?
The models run at physically incommensurate temporal and spatial
scales; is it feasible to use the instantaneous GCM \Mc\ in
(\ref{eqn:qi_cnv}--\ref{eqn:t})?
We attempt to address these issues in three ways.
First, we isolate a common synoptic regime to intercompare.
Second, we look for systematic convection differences in averages over
many convective system lifetimes.
Third, we intercompare frequency distributions of the instantaneous
\Mc\ used in (\ref{eqn:qi_cnv}--\ref{eqn:t}).

\subsection{GCM Control Simulation}

We employ a new version of the NCAR CCM2 (hereafter the GCM) to 
demonstrate the technique of mapping (\ref{eqn:qi_cnv}--\ref{eqn:t})
to a GCM in a physically consistent manner. 
The GCM contains four improvements over the standard CCM2
\cite[]{HBB93} in the area of cloud physics:
Ice phase condensate is treated with the radiative scheme of
\cite[]{EbC92}.
A continental/maritime distinction is made in the effective radius of
cloud droplets \cite[]{Kie942}.
LWP is based on column precipitable water rather than prescribed by
latitude. 
Large scale precipitation is able to evaporate between the cloud and
surface.

Steady conditions from the Marshall Islands region studied by
\cite{YEC73} were used to force the CEM simulation
(Figure~\ref{fig:cem_forcing}). 
July GCM simulations best matched the strong low-level convergence and
intense precipitation of the CEM simulation in a $10\dgr \times
15\dgr$ region centered at 125~\dgr E, 15~\dgr N
(in the Philippine Sea, 40~\dgr W of the Marshalls). 
This control region consists of 19 maritime and one land gridpoint
at T42 resolution ($\sim$~2.8\dgr).
The GCM simulation was run for 14~days and instantaneous data were
archived every timestep ($\Delta t = 20$~min). 
%Two filters were applied to the GCM dataset:
%Second, gridpoints not experiencing intense convective precipitation
%($P > 15$~\mmxday) were excluded (XXX replot figure~\ref{fig:precip}
%or disqualify it from this statement).
%This excluded points atypical of the CEM forcing conditions.
Land-locked precipitation ($10\%$ of the regional total) was removed 
by excising the land gridpoint (the CEM simulation was also entirely
maritime). 
Figure~\ref{fig:precip} compares the resulting instantaneous,
domain-averaged precipitation timeseries between CEM and GCM.
\begin{figure}
\begin{center}
\includegraphics*[width=\halfwidth]{/data/zender/fgr/phd/precip.eps}
\end{center}
\caption[Domain-averaged precipitation timeseries from tropical
simulations]{
Domain-averaged precipitation timeseries (\mmxday) from tropical
simulations.   
CEM instantaneous hourly values from days 4--24 (solid) and GCM
instantaneous values every 20~m for 14 days (dashed). 
Offset along the abscissa is arbitrary.\label{fig:precip}}
\end{figure}
The mean precipitation rates (and associated latent heating) are
similar: 14.4 and 17.0~\mmxday, respectively.
However, the strong diurnal component in the CEM contrasts markedly
with the low-frequency synoptic events in the GCM.
The significant difference in the spectral characteristics of the
latent heating raised doubts that the $M$ profiles would be
comparable. 

\subsection{Threshold Convective Velocity}

Intercomparing temporally and spatially averaged $M$ profiles
reduces, to the extent possible, model differences due to internal
variability.  
Figure~\ref{fig:mc_profile} compares the vertical structure of $M$ in
the GCM control region, the CEM simulation, and the Marshall Islands
as inferred by \cite{YEC73}.  
\begin{figure}
\begin{center}
\includegraphics*[width=\halfwidth]{/data/zender/fgr/phd/mc_profile.eps}
\end{center}
\caption[Time- and domain-average convective mass flux $M$ over the
Marshall Islands region from observations, GCM, and CEM]{
Time- and domain-average convective mass flux $M$ (\mbxh) over the
Marshall Islands region from observations (solid), GCM (dotted), and  
CEM (dashed).
Updrafts are defined to be positive.
Observations are taken from \cite{YEC73},
Figure~13.\label{fig:mc_profile}}
\end{figure}
Observed $M$ peaked at the surface with a secondary maxima near
250~mb, indicative of anvil circulations.
Strong low level convergence and shear cause $M$ to peak near the
surface in the Yanai and GCM $M$ profiles. 
In contrast, the closed domain forces CEM $M$ to zero at the surface 
(cf.\ Figure~18 of \cite{SLT94}). 
The GCM and CEM reproduce observed $M$ from the boundary layer to
500 and 600~mb, respectively. 
The weak GCM $M$ in the upper troposphere is a known bias of the
convection scheme, see \cite{MRP95}.
CEM $M$ is sensitive to \wc\ and \qcc\ (all figures in this study use
$\wc = 1$~\mxs\ and $\qcc = .05$~\gxkg). 
Decreasing \qcc\ from .1 to 0~\gxkg\ nearly doubled $M$ beneath
500~mb. 
Above 500~mb however, $M$ is relatively insensitive to $\qcc <
.1$~\gxkg\ because the CEM simulation produced persistent anvil
coverage in this region (Figure~\ref{fig:cem_fice}). 
The modeled 800--200~mb $M$ gradient is too strong.
Halving \wc\ from 1 to .5~\mxs\ reduced the CEM gradient from 2.5:1 to 
1.25:1, in agreement with Yanai.
The relaxed gradient was caused by including a strong, perhaps
radiatively driven, circulation ($.5 < w < 1$~\mxs) near anvil top.

The radiative and dynamical feedbacks from prognosed cloud are
nonlinear. 
Therefore it is important that the temporal distribution of GCM $M$  
match the distribution of CEM $M$ used to derive
(\ref{eqn:qi_cnv}--\ref{eqn:t}) and to pick $c_1$.  
Figure~\ref{fig:mc_hist} shows histograms of the instantaneous,
domain-average $M$ from the CEM and GCM.
\begin{figure}
\begin{center}
\includegraphics*[height=.2\textheight]{/data/zender/fgr/phd/cem_mc_hist_500mb.eps}\vfill
\includegraphics*[height=.2\textheight]{/data/zender/fgr/phd/ccm_mc_hist_500mb.eps}\vfill
\includegraphics*[height=.2\textheight]{/data/zender/fgr/phd/cem_mc_hist_200mb.eps}\vfill
\includegraphics*[height=.2\textheight]{/data/zender/fgr/phd/ccm_mc_hist_200mb.eps}\vfill
\end{center}
\caption[Comparison of frequency distribution of instantaneous $M$ 
between CEM and GCM]{
Comparison of frequency distribution of instantaneous $M$ (\gxmSs)
between CEM and GCM. 
The left hand axes indicate the percentage of the data contained in
each bin.
The dashed lines and right hand axes indicate the percentage of the
data with $M$ less than the abscissa.
(a) CEM $\sim$~500~mb. (b) GCM 500~mb. (c) CEM $\sim$~200~mb. (d) GCM
200~mb. 
\label{fig:mc_hist}}
\end{figure}
The frequency distributions of $M$ in the two models were most
similar from 500--200~mb using $w_{\mathrm{c}} = 1$~\mxs.
Relaxing \wc\ increases the mean and mode of \Mc.
Typically only 1--2~\% of the (2-dimensional) CEM domain satisfied $w 
> 1$~\mxs.
This fraction is at the low end of satellite-derived estimates
of the areal extent of deep convective cloudiness \cite[]{FDR90}. 

The overall agreement in the simulated $M$ profiles is somewhat
unexpected, considering the differences between the models and their
precipitation timeseries (Figure~\ref{fig:precip}).
The agreement in mean $M$ (Figure~\ref{fig:mc_profile}) partially
results from tuning \qcc\ and \wc, but the underlying distributions of
$M$ (Figure~\ref{fig:mc_hist}) were already similar.
Both \qcc\ and \wc\ depend on the GCM convection scheme
with which ANV is to be used, and possibly on the synoptic regime. 

\section{Stratiform Ice Cloud Generation}\label{sec:srcm}

Orographically forced uplift, warm frontal overrunning, and large
scale ascent and cooling can form ice cloud in the absence of any
parameterized convection; we denote such ice as stratiform.
The GCM-resolved thermodynamic fields that control the development
of stratiform (stable) ice cloud are temperature $T$, humidity \qv,
and updraft $w$ \cite[]{HeD90}.
As with convective condensation, the temporal and spatial scales of
stratiform condensation differ vastly from the scales on which GCMs
predict the thermodynamic fields. 
GCM schemes for diagnosis and prognosis of stratiform ice span a
corresponding range of complexity:
\cite{HeP84} empirically relate \qi\ to $w$ and $T$.
\cite{Sun88} predicts \qi\ from highly parameterized physical
processes which assume exact saturation in the cloudy fraction of a
gridbox.
\cite{HeD90} diagnose \qi\ by trajectory integrations of the
thermodynamics governing single parcel behavior.
\cite{Smi90} predicts \qc\ by assuming a functional form for the
sub-gridscale distribution conserved quantities.
\cite{FeH92} diagnose \qi\ from fits to a detailed microphysical
scheme to determine number concentration, effective radius, and \qc\
for liquid (or spherical ice crystal) cloud.
Experience with the above methods suggests detailed sub-gridscale
modelling may not be necessary to predict free-atmosphere stratiform
condensation at GCM resolution. 
Our strategy for modelling stratiform condensation is to replace the
mass flux-based creation process from (\ref{eqn:qi_cnv}) with a
quiescent deposition process, but to keep the same destruction and
removal processes.

\subsection{Closure and Microphysical Assumptions}

Parameterization of stratiform condensation in large scale models
involves closure assumptions and microphysical assumptions.
Allowing cloud formation in subsaturated gridcells requires a closure 
assumption to relate the resolved thermodynamic fields to the
sub-gridscale distribution of condensate.
Stratiform cloud fraction $A_s$ is naturally suited to map column (and
parcel) models of stratiform condensation, which is horizontally homogeneous,
to specified horizontal fractions of a gridcell.
Numerical studies by \cite{XuK91} suggest relative humidity is best
suited to diagnose $A_s$.
At the same time, diagnoses of $A_s$ based on local or synoptic
atmospheric state continually improve as field, model, and satellite 
data accumulate.
To accommodate future advances in $A_s$ diagnosis, and to maintain
internal simplicity, ANV requires the GCM to make the closure
assumptions for, and to specify, $A_s$. 
The rest of this section describes the microphysical assumptions 
we make to model condensate formation and removal processes. 

A supersaturated vapor field quickly equilibrates by deposition,
assuming adequate ice freezing nuclei (IFN) are present.
The IFN assumption is justified in most cases for $T < -15$~\dgr C  
\cite[]{HeD90}.
Since supercooled droplets should be represented by the liquid
component in a full GCM cloud scheme, ANV assumes all vapor condensed
at $T < 0$~\dgr C to form ice.
Complications arise from droplet curvature (Kelvin) effects, water
inhomogeneity (solute) effects and radiation (especially at cloud base
and top).
These processes slightly change the equilibrium vapor pressure over
crystals and lead to size-based competition for vapor among ice
crystals \cite[]{RaD86}.
Bulk (non size-resolving) condensate models such as ANV may ignore
vapor competition so long as the timescale for the vapor field to
equilibrate with ice crystals through diffusional processes is much
faster than a typical GCM timestep ($\Delta t \sim 20$~min). 

The balance between \qi\ and \qv\ in a fixed, horizontally
homogeneous, exactly saturated layer in an updraft $w$ is: 
\begin{equation}
{\partial \qi \over \partial t} = 
-{\partial \qvi \over \partial t} 
-w{\partial \qvi \over \partial z} 
-w{\partial \qi \over \partial z}
- c_2\qi 
- c_3\qi 
\label{eqn:qidot_1}
\end{equation}
The first two terms on the RHS of (\ref{eqn:qidot_1}) describe the
changes in vapor available for deposition in large-scale ascent due
to local temperature change (e.g., adiabatic or radiative cooling of
the layer), and convergence of supersaturated vapor, respectively.
The difficulty of separating the contributions of $w$ and $T$ to
deposition in an Eulerian framework is apparent in these two terms: 
vapor flux and adiabatic cooling are linear in $w$, but \qvi\ is 
exponential in $T$ due to the Clausius-Clapeyron relationship.
The next term is the large scale convergence of \qi.
The remaining terms are local sublimation and precipitation,
parameterized in terms of \qi\ from (\ref{eqn:qi_cnv}).
The behavior of \qidot\ in the limit of resolved and
saturated large-scale ascent (\ref{eqn:qidot_1}) guides
our parameterization of sub-gridscale effects. 

\subsection{Size Resolving Column Model (SRCM) Simulations}

Sensitivity studies were performed with the SRCM of \cite{ZeK94} in
order to isolate the effects of updraft speed $w$ from temperature
$T$ in (\ref{eqn:qidot_1}).
The SRCM explicitly represents nucleation, deposition, sublimation,
sedimentation, and radiation, but does not attempt represent
diffusion, convection, collision, liquid processes, or entrainment.   
The SRCM used 50 geometrically spaced mass bins for ice particles
corresponding to hexagonal column lengths 3--2000~\um.  
The time step is 3~s for microphysics and transport, 1~min for
radiation.
The thermodynamic profile was a standard 35-level tropical
atmospheric profile from ICRCCM \cite[]{EEF91} with an embedded
130-level ($\Delta z = 100$~m) computational grid from 5--18~km.
The ice crystal distribution from a decaying tropical anvil
\cite[]{KKW93} was replicated to form a 2~km thick ($\sim$ one
GCM layer) homogeneous anvil with initial $\IWP \approx 73$~\gxmS. 
Most uncertain in the Knollenberg distribution is the abundance of
small ice crystals (which contribute little to cloud mass). 
The following sensitivity studies are presented in terms of IWP
rather than \qi\ ($= \IWP/(\rho \Delta z)$) for direct
comparison to Figure~\ref{fig:cem_mc_diwpdt}, and to simplify
deductions about the radiative import of the sensitivities.

\subsection{Timescale for Precipitation}

As mentioned in Section~\ref{sec:anv}, the precipitation rate of
cirrus is taken to be proportional to cloud mass itself.
The proportionality constant, $c_3$ in (\ref{eqn:qi_cnv}), is the
inverse of the $e$-folding timescale $\tau$ for sedimentation of ice
crystals in the cloud.
Modelling the sedimentation of a tropical anvil in the SRCM provides an
understanding of the range of $\tau$ in the atmosphere.
Here we use the term \textit{sedimentation} rather than
\textit{precipitation} to emphasize the SRCM does not account for
collection processes.
This study monitored the IWP change in the initial cloud through
a six hour integration in still, saturated air.
Figure~\ref{fig:srcm_decay_ss} shows the sedimentation of the same
anvil from three different cloud base heights.
\begin{figure}
\begin{center}
\includegraphics*[width=\halfwidth]{/data/zender/fgr/phd/srcm_decay_ss.eps}
\end{center}
\caption[Time variation of IWP (\gxmS) for the same initial 2~km
thick anvil at three different cloud base heights]{
Time variation of IWP (\gxmS) for the same initial 2~km
thick anvil at three different cloud base heights: 5~km (dotted),
10~km (dashed), and 15~km (dot-dashed).
Solid curves represent pure exponential decay with timescale $\tau =
.5$~hr (lower) and 2~hr (upper),
respectively.\label{fig:srcm_decay_ss}}    
\end{figure}
The two solid curves represent purely exponential decay of the initial
anvil, $\IWP(t) = \IWP_0 e^{-t/\tau}$, with timescales $\tau = .5$
and 2~hr. 
The SRCM employs terminal velocities from \cite{Hey72}, which increase 
with height and crystal length.
Sedimentation is very efficient for the first hour, as the large
crystals ($L > 200$~\um) quickly fall from the anvil.
Differential sedimentation causes the separation of the dashed
curves before $t = 2$~hr, and leaves more vapor available for
deposition to small crystals afterwards.
The remaining mass descends slowly ($< 30$~\cmxs) in smaller crystals. 
Figure~\ref{fig:srcm_decay_ss} shows that 
Near GCM resolution in the free troposphere ($\Delta z < 2$~km), 
figure~\ref{fig:srcm_decay_ss} shows $\tau < .5$~hr overestimates 
sedimentation for the entire life of the anvil while $\tau >
2$~hr underestimates sedimentation of large crystals, and hence
cloud mass.
A compromise value of $\tau = c_3^{-1} = 1$~hr is recommended.
This should be contrasted with observed ``typical'' anvil lifetimes of
$\sim 6$~hr over Panama \cite[]{ALV88}. 
Of course observations implicitly include condensate resupply
through convective detrainment, mesoscale generation, and cloud top 
cooling. 

\subsection{Dependence of Deposition Rate on Updraft}

This sensitivity study isolates the effect of updraft on stratiform
condensation by examining the growth of the initial cloud in an
atmosphere exactly saturated with respect to ice. 
Ice crystals were not allowed to sediment.
Figure~\ref{fig:srcm_w_ss}a shows the dependence of \IWPdot\ on the
ascent speed $w$ of vapor for the same initial anvil at three
different cloud base heights.  
\begin{figure}
\begin{center}
\includegraphics*[width=\halfwidth]{/data/zender/fgr/phd/srcm_w_ss.eps}\vfill
\includegraphics*[width=\halfwidth]{/data/zender/fgr/phd/srcm_w_ss_c5.eps}\vfill
\end{center}
\caption[Sensitivity of deposition rate \IWPdot\ and $c_5$ to updraft
$w$ in a 2~km thick anvil in a saturated column]{ 
Sensitivity of (a) deposition rate \IWPdot\ and (b) $c_5$ to updraft 
$w$ (\cmxs) in a 2~km thick anvil in a saturated column.
\IWPdot\ is expressed in \gxmSgcmt, i.e., mass change per GCM
timestep. 
Results are shown for three different cloud base heights: 5~km
(solid), 10~km (dotted), and 15~km (dashed).\label{fig:srcm_w_ss}} 
\end{figure}
\IWPdot\ is computed from the difference between the final and
initial (73~\gxmS) IWP from one hour integrations of the SRCM. 
\IWPdot\ is linear with $w$ for all cloud heights, and a nonlinear
dependence on cloud height is evident.
Most of the difference ($> 90\%$) in ice deposition rates stems
from the exponential decrease in $-\partial \qvi/\partial z$ in
(\ref{eqn:qidot_1}) from 1.4 to .02~\gxkgkm\ between 5~and
15~km. 
The ambient density decrease from .63 to .24~\kgxmC\ between 5 and
15~km accounts for the remaining factor of $\sim 3$ difference in the
deposition rates. 
The \IWPdot\ shown is an upper bound because the initial atmosphere
was saturated and all removal processes were disabled. 

Evidently a simple linear dependence of \IWPdot\ on $w$ will not
capture the effects of $T$.
Instead, we define a single parameter approximation that includes both
$T$ and $w$ effects.
We implicitly define the dimensionless empirical parameter $c_5$ based
on the second term on the RHS of (\ref{eqn:qidot_1}):
\begin{equation}
{\partial \qi \over \partial t} \equiv 
c_5 w {\partial \qvi \over \partial z} 
\label{eqn:c5}
\end{equation}
Thus $c_5$ expresses the ice formation tendency as a function of
supersaturated vapor convergence alone.
Note that \dqvidz\ is analytically tractable; it depends only on $T$
\cite[e.g.,][]{WaP86}.  
Figure~\ref{fig:srcm_w_ss}b shows $c_5$ computed using the initial, not
the instantaneous, $T$ profile.
This removes the feedback of heating of the cloud environment on
\dqvidz\ (but not on \dqidt). 
Clearly $c_5$ is relatively constant because the linear $w$ and
exponential $T$ dependencies are factored out. 

% What causes the spread in c_5 here?
%Lapse rates decrease with altitude, which causes advective cooling to
%increase, so that the final temperature at 15 km has cooled more than
%the final temperature at 5 km after an hour of updraft.
%When you ignore this temperature feedback and compute dqvidz using
%the initial T, you therefore overestimate the actual dqvidz felt by
%the model (and present in dqidt) at 15 km relative to 5 km. 
%Dividing by this artificially inflated dqvidz reduces (in magnitude)
%c_5  at 15 km relative to 5 km.

\subsection{Dependence of Deposition Rate on Temperature}

This sensitivity study isolates the effect of cloud temperature on
stratiform deposition.
As in the updraft sensitivity study, the atmosphere is exactly
saturated with respect to ice and sedimentation of ice crystals was
disabled.
Figure~\ref{fig:srcm_t_ss}a shows the dependence of \IWPdot\ on the
mid-cloud temperature $T$.
\begin{figure}
\begin{center}
\includegraphics*[width=\halfwidth]{/data/zender/fgr/phd/srcm_t_ss.eps}\vfill
\includegraphics*[width=\halfwidth]{/data/zender/fgr/phd/srcm_t_ss_c5.eps}\vfill
\end{center}
\caption[Sensitivity of deposition rate \IWPdot\ and $c_5$ to
mid-cloud temperature $T$ in a 2~km thick anvil in a saturated
column]{ 
Sensitivity of (a) deposition rate \IWPdot\ and (b) $c_5$ to mid-cloud
temperature $T$ in a 2~km thick anvil in a saturated column.  
\IWPdot\ computed and scaled as in Figure~\ref{fig:srcm_w_ss}.
Results are shown for vertical motions of 2, 4, 6, 8, and
10~\cmxs.\label{fig:srcm_t_ss}} 
\end{figure}
As expected from (\ref{eqn:qidot_1}), the deposition tendency is
exponential in $T$, while the spacing between the curves is linear in 
$w$. 
Exponential dependence of IWC on $T$ characterizes observations of
cirrus formed by large scale ascent \cite[]{Hey77}.

Figure~\ref{fig:srcm_t_ss}b shows the empirical parameter $c_5$ from 
(\ref{eqn:c5}) falls in nearly the same range as the updraft
sensitivity study (cf. Figure~\ref{fig:srcm_w_ss}b).
Indeed, $c_5$ varies $\lesssim 20\%$ for typical upper tropospheric
ice cloud.
The spread in $c_5$ is dominated by variations in the lapse rate in
the ICRCCM $T$ profile, which changes local advective cooling, and by
the associated saturation vapor density, which determines the mass
available for deposition. 
When $c_5$ is computed using instantaneous $T$ (not shown),
the advectively cooling environment reduces $|\dqvidz|$ and increases
$!c_5|$ by $\sim 20\%$ to $-1.1 < c_5 < -1.2$.
This effect is more pronounced at higher altitudes where tropical
lapse rate and vapor density are smaller, but can be
ameliorated by horizontal circulation and diabatic heating. 

In summary, the slight variation of $c_5$ over a wide range of $w$ and
$T$ is appealing for parameterization purposes.
Therefore ANV models gridbox stratiform deposition due to large scale
ascent as $A_s c_5 w \dqvidz$.
We recommend $c_5 = -1.1$ for representing light ($w < 10$~\cmxs)
ascent between 5--10~km in altitude, where a GCM stratiform ice scheme
is most active. 

The full equation for stratiform ice in ANV is
\begin{eqnarray}
\label{eqn:qi_stb}
{D\qi \over Dt} & 
= & 
A_s c_5 w {\partial \qvi \over \partial z} -
c_2 \qi - 
c_3 \qi
\end{eqnarray}
where $A_s$ is the GCM-supplied stratiform cloud fraction.
In non-convecting gridcells with $w > 0$, (\ref{eqn:qi_stb}) is used
in place of  (\ref{eqn:qi_cnv}), along with (\ref{eqn:qv}) and
(\ref{eqn:t}). 
$A_s$ fills the role of $\partial \qvi / \partial t$ in
(\ref{eqn:qidot_1}), to some extent, because local changes in
temperature will presumably affect $A_s$ and thus \qidot\ through 
(\ref{eqn:qi_stb}).
Note (\ref{eqn:qi_stb}) does not produce ice in still air ($w = 0$).  
In practice stable cloud fraction $A_s$ produced by many GCM
cloud schemes is also zero in still or subsiding air.
Set $A_s = 1$ in subsiding air to allow sublimation.
Stratiform cloud in moist, radiative cooled regions above anvils may
be underpredicted by ANV because (\ref{eqn:qi_stb}) lacks stability
dependence and $w \rightarrow 0$ near the tropopause.
As mentioned in Section~\ref{sec:anv}, the sublimation rate $c_2\qi$
is determined (as a fraction of the precipitation rate $c_3\qi$) from
budget studies which implicitly contain the effects of partial
cloudiness. 
Thus the precipitation and sublimation rates in (\ref{eqn:qi_stb}) act 
independently of cloud fraction $A_s$, exactly as in (\ref{eqn:qi_cnv}).

The stratiform cloud ANV attempts to represent in a GCM gridbox
can be envisioned as maintained by some unknown, underlying
distribution of sub-gridscale updrafts (and downdrafts) with 
mean speed $w$.
The range of $w$ in Figure~\ref{fig:srcm_w_ss}, 0--20~\cmxs, is
approximately that reported by \cite{HeP84}.
Large scale ascent in nonconvecting gridcells at the GCM scale is
typically $< 5$~\cmxs.
The sub-gridscale distribution composing $w$ could, if known (e.g.,
a Poisson distribution with a certain variance), be convolved
through the $c_5$ term in (\ref{eqn:c5}) to obtain a more realistic
\qidot.
Lacking this information, we assume the effects of sub-gridscale
distributions of $w$ on \qidot\ are incorporated through $A_s$.

\section{Discussion}\label{sec:cem_disc}

We have presented the rationale for, and development of, an ice cloud
parameterization suitable for a GCM representation of ice cloud
based on modeled and observed behavior of tropical anvils.
The methodology of the development is explicit parameterization of the
physical processes of ice cloud formation and removal: detrainment of 
convectively formed ice into a stratiform anvil, formation of ice
through large scale ascent, sublimation, precipitation, and
evaporation of precipitation.
Each process was simply modeled in terms of a prognostic GCM variable
and a single free parameter and incorporated into GCM forecast equations
for cloud ice, water vapor, and temperature.  
Each free parameter was determined by a sensitivity study that
attempted to isolate the respective process.
The exception is the parameter determining local sublimation ($c_2$);
it was chosen to match empirical estimates of anvil contributions to
tropical convective system hydrologic budgets. 

ANV is intended to be the ice phase component of a GCM cloud scheme. 
It can be wedded to the liquid component of a full cloud scheme by use 
of a parameterization for the fraction of condensate that is ice.
Many of the physical processes and assumptions underpinning ANV would 
need only with minor modifications to be suitable for a prognostic
liquid water scheme.
Although we have not done so here, one could perform analogous
single parameter sensitivity studies for \qldot\ in terms of \Mc, $w$
and $T$.

We have taken advantage of a hierarchy of models (SRCM, CEM, and GCM)
and the empirical budget estimates because no single tool was wholly
adequate. 
This methodology demonstrated the feasibility of combining the results 
from a hierarchy of models and observations into a coherent cloud
parameterization.
We eschewed incorporating detailed sub-gridscale models into ANV
although we recognize such models (e.g., cloud fraction) enhance the
internal consistency of a condensate forecast. 
The emphasis has been on forecasting ice sequestered in mesoscale
anvils associated with deep convection because anvils'
combination of areal extent, longevity, and IWP produces strong
radiative forcing.
However, budget studies infer 25--60\% of anvil ice comes from
mesoscale updrafts outside the convective core region
\cite[]{LeH80,GaJ91}. 
This ice is only implicitly parameterized by the $c_1$ term in
(\ref{eqn:qi_cnv}) and the $c_5$ term in (\ref{eqn:qi_stb}).  
Future studies should attempt to explicitly parameterize the ice
formation by intra-anvil circulations, perhaps in terms of radiative
heating gradients, stability, \qi, and $w$.

% Balance preprint columns
\ifphdcsz{}{\balance}

% Appendices
\ifphdcsz{}{\appendix}

% Acknowledgements
\ifphdcsz{\subsection{Acknowledgements}\label{sec:cem_ack}}{\acknowledgments}
The authors wish to thank G.~Tripoli for supplying the CEM, and
J.~Doetzl for programming support.
CSZ gratefully acknowledges discussions with W.~Grabowski and
J.~Petch, and guidance in cloud parameterization from P.~Rasch. 
This work was supported in part by DOE Atmospheric Radiation
Measurements Program grant DE-AI05-92ER61376 and by NASA
Earth Observing System project W-17,661. 

% Bibliography
\ifphdcsz{}{\bibliographystyle{jas}}
\ifphdcsz{}{\bibliography{/home/zender/tex/bib}}

\ifphdcsz{}{\end{document}}

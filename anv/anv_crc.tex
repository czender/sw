% $Id$

% Purpose: Correct figures which appeared in ANV paper in JGR

% Usage:
% make -W anv_crc.tex anv_crc.ps
% gzip -f /data/zender/ps/anv_crc.ps
% /bin/cp /data/zender/ps/anv_crc.ps.gz /web/web-data/cms/zender/papers/anv_crc.ps.gz

\documentclass[twocolumn,final]{article}
\usepackage{natbib} % \cite commands from aguplus
\usepackage[figuresright]{rotating} % allows sideways figures and tables
\usepackage{graphicx} % defines \includegraphics*
\usepackage{longtable} % multi-page tables, e.g., acronyms and symbols
\usepackage{ifthen} % Boolean and programming commands

\usepackage{/home/zender/tex/csz} % all my local definitions
\input{/home/zender/tex/jgr_abb}

\topmargin -48pt   \headheight 12pt \headsep 18pt
\textheight 723pt \textwidth 470pt
\oddsidemargin 0pt \evensidemargin 0pt
\marginparwidth 72pt \marginparsep 7pt
\footskip 0pt
\footnotesep=14pt

%\renewcommand\textfraction{0.}
\setcounter{totalnumber}{10}
\setcounter{topnumber}{10}
\setcounter{dbltopnumber}{10}
\setcounter{bottomnumber}{10}
\renewcommand\topfraction{1.}
\renewcommand\dbltopfraction{1.}
\renewcommand\bottomfraction{1.}
\renewcommand\floatpagefraction{1.}
\renewcommand\dblfloatpagefraction{1.}

\begin{document}

% These don't seem to work before \begin{document}
\setlength\abovecaptionskip{9pt}
\setlength\belowcaptionskip{9pt}

\setlength\floatsep{0pt}
\setlength\textfloatsep{0pt}
\setlength\dblfloatsep{0pt}
\setlength\dbltextfloatsep{0pt}
\setlength\intextsep{0pt}
%\setlength\floatsep{18pt \@plus 2pt \@minus 4pt}
%\setlength\textfloatsep{18pt \@plus 2pt \@minus 4pt}
%\setlength\dblfloatsep{18pt \@plus 2pt \@minus 4pt}
%\setlength\dbltextfloatsep{18pt \@plus 2pt \@minus 4pt}
%\setlength\intextsep{20pt \@plus 4pt \@minus 4pt}

\pagestyle{myheadings}
%\thispagestyle{empty}
\pagenumbering{arabic}
\setcounter{page}{1}
\markright{Corrigenda to Figures 5,~6,~7,~9,~10, and~13 of \cite{Zek972}}

\section{Contents}
Important greyscales were inadvertently omitted from Figures
5,~6,~7,~9,~10, and~13 of \cite{Zek972}.   
The omission does not affect the results of the study, but does make 
interpretation of these figures very difficult.
These corrigenda show the figures with the greyscales correctly
rendered.   

\setcounter{figure}{4}
\begin{figure}
\begin{center}
\includegraphics*[width=\halfwidth]{/data/zender/fgr/anv/spcp_85_8589_amip5_8589_xavg_01_QC.eps}\vfill
\end{center}
\caption{
Change (ANV$-$CCM) in zonal average condensate mixing ratio \qc\
(\mgxkg) due to prognostic anvil representation.
Contour interval is 2~\mgxkg.
Shading indicates values $< 0$.
Data are from ensemble averages of 5 simulated Januarys from
1985--1989.
\label{fig:xavg_8589_QC}}   
\end{figure}

\begin{figure}
\begin{center}
\includegraphics*[width=\halfwidth]{/data/zender/fgr/anv/spcp_85_8589_amip5_8589_xavg_01_RADD.eps}\vfill
\includegraphics*[width=\halfwidth]{/data/zender/fgr/anv/spcp_85_8589_amip5_8589_xavg_01_QDIABAT.eps}\vfill
\end{center}
\caption{
As in Figure~\ref{fig:xavg_8589_QC} but for (a) radiative heating \QR\
only, and (b) total diabatic heating \QT.
Contour intervals are .1~\kxday.   
\label{fig:xavg_8589_QDIABAT}}   
\end{figure}

\begin{figure}
\begin{center}
\includegraphics*[width=\halfwidth]{/data/zender/fgr/anv/spcp_85_8589_amip5_8589_xavg_01_T.eps}\vfill
\end{center}
\caption{
As in Figure~\ref{fig:xavg_8589_QC} but for temperature $T$ (\degreee
K).
Contour interval is .1~\degreee K. 
\label{fig:xavg_8589_T}}   
\end{figure}

\setcounter{figure}{8}
\begin{figure}
\begin{center}
\includegraphics*[width=\halfwidth]{/data/zender/fgr/anv/spcp_85_8589_amip5_8589_01_LWCF_bw.eps}\vfill
\end{center}
\caption{
As in Figure~\ref{fig:xavg_8589_QC} but for longwave cloud forcing
LWCF (\wxmS). 
Contour interval is 10~\wxmS. 
\label{fig:8589_LWCF}}   
\end{figure}

\begin{figure}
\begin{center}
\includegraphics*[width=\halfwidth]{/data/zender/fgr/anv/spcp_85_8589_amip5_8589_pres_01_CHI.eps}\vfill
\end{center}
\caption{
As in Figure~\ref{fig:xavg_8589_QC} but for 200~mb velocity potential
$\chi$ (\mSxs). 
Contour interval is $1 \times 10^6$~\mSxs.  
Shading indicates less subsidence (more divergence). 
\label{fig:pres_8589_01_CHI}}   
\end{figure}
\clearpage

\setcounter{figure}{12}
\begin{figure}
\begin{center}
\includegraphics*[width=\halfwidth]{/data/zender/fgr/anv/amip5_yavg_10S10N_87m85_MAM_QC.eps}\vfill
\includegraphics*[width=\halfwidth]{/data/zender/fgr/anv/spcp_85_yavg_10S10N_87m85_MAM_QC.eps}\vfill
\end{center}
\caption[Longitude-height profile of the $1987-1985$ difference in
Spring quarter (March, April, and May) mean condensate \qc\ over the
equatorial Pacific simulated by CCM and ANV]{
Longitude-height profile of the $1987-1985$ difference in Spring
quarter (March, April, and May) mean condensate \qc\ (\mgxkg) over the
equatorial Pacific (averaged 10~\degreee S--10~\degreee N, 
ocean only) simulated by (a) CCM and (b) ANV.
Contour interval is 2~\mgxkg.
Shading indicates \qc\ decrease from 1985 to 1987.
% NB: IDL gets the shading of the first -'ve contour wrong when land
% is masked so i included land points in this file, no significant
% difference from maritime only above 400 mb.
\label{fig:yavg_10S10N_87m85_MAM_QC}}   
\end{figure}

\nocite{Zek972}
\bibliographystyle{agu}
\bibliography{/home/zender/tex/bib}

\end{document}

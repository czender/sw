% $Id$

% Purpose: ANV parameterization implementation paper.

\ifphdcsz{}{

%\documentclass[twoside,agupp]{aguplus}
%\documentclass[twoside,draft,agupp]{aguplus} % excludes figures
\documentclass[agums]{aguplus}
%\documentclass[twoside,agums]{aguplus}
%\documentclass[jgrga]{aguplus}

%  AGU++ OPTIONS
%\printfigures        % ADDS FIGURES AT END
%\doublecaption{35pc} % CAPTIONS PRINTED TWICE
\sectionnumbers      % TURNS ON SECTION NUMBERS
%\extraabstract       % ADDS SUPPLEMENTAL ABSTRACT
%\afour               % EUROPEAN A4 PAPER SIZE
%\figmarkoff          % SUPPRESS MARGINAL MARKINGS

%  AGUTeX OPTIONS AND ENTRIES
%\tighten             % TURNS OFF DOUBLE SPACING, has no effect with agupp
%\singlespace         % RESTORES SINGLE SPACING
%\doublespace         % RESTORES DOUBLE SPACING

\usepackage{graphicx} % defines \includegraphics*
\usepackage{ifthen} % Boolean and programming commands
\usepackage{longtable} % for long tables, like acronyms and symbols
\usepackage{tabularx} % for long tables, like acronyms and symbols

\usepackage{/home/zender/tex/csz} % all my local definitions
% Usage: % Usage: % Usage: \input{jgr_abb} % AGU-sanctioned journal title abbreviations

\def\aapgb{{\it Amer. Assoc. Petroleum Geologists Bull.}}
\def\adg{{\it Adv. Geophys.}}
\def\ajs{{\it Amer. J. Sci.}}
\def\amb{{\it Ambio}}
\def\amgb{{\it Arch. Meteorol. Geophys. Bioclimatl.}}
\def\ang{{\it Ann. Glaciol.}}
\def\angeo{{\it Ann. Geophys.}}
\def\apo{{\it Appl. Opt.}}
\def\areps{{\it Ann. Rev. Earth Planet. Sci.}}
\def\asr{{\it Adv. Space Res.}}
\def\ate{{\it Atmos. Environ.}}
\def\atf{{\it Atmosfera}}
\def\atms{{\it ACM Trans. Math Software}}
\def\ato{{\it Atmos. Ocean}}
\def\atr{{\it Atmos. Res.}}
\def\gbc{{\it Global Biogeochem. Cycles}} % csz
\def\blm{{\it Boundary-Layer Meteorol.}} % csz 
\def\bpa{{\it Beitr. Phys. Atmosph.}}
\def\bams{{\it Bull. Am. Meteorol. Soc.}}
\def\clc{{\it Clim. Change}}
\def\cld{{\it Clim. Dyn.}}
\def\com{{\it Computing}}
\def\dao{{\it Dyn. Atmos. Oceans}}
\def\dsr{{\it Deep-Sea Res.}}
\def\esr{{\it Earth Sci. Revs.}}
\def\gec{{\it Geosci. Canada}}
\def\gei{{\it Geofis. Int.}}
\def\gej{{\it Geogr. J.}}
\def\gem{{\it Geophys. Monogr.}}
\def\geo{{\it Geology}}
\def\grl{{\it Geophys. Res. Lett.}}
\def\ieeec{{\it IEEE Computer}}
\def\ijna{{\it IMA J. Numer. Anal.}}
\def\ijnmf{{\it Int. J. Num. Meteorol. Fl.}}
\def\jac{{\it J. Atmos. Chem.}}
\def\jacm{{\it J. Assoc. Comput. Mach.}}
\def\jam{{\it J. Appl. Meteorol.}}
\def\jas{{\it J. Atmos. Sci.}}
\def\jatp{{\it J. Atmos. Terr. Phys.}}
\def\jcam{{\it J. Climate Appl. Meteorol.}}
\def\jchp{{\it J. Chem Phys.}}
\def\jcis{{\it J. Coll. I. Sci.}}
\def\jcl{{\it J. Clim.}}
\def\jcp{{\it J. Comput. Phys.}}
\def\jfm{{\it J. Fluid Mech.}}
\def\jgl{{\it J. Glaciol.}}
\def\jgr{{\it J. Geophys. Res.}}
\def\jgs{{\it J. Geol. Soc. London}}
\def\jme{{\it J. Meteorol.}}
\def\jmr{{\it J. Marine Res.}}
\def\jmsj{{\it J. Meteorol. Soc. Jpn.}}
\def\josa{{\it J. Opt. Soc. A}}
\def\jpo{{\it J. Phys. Oceanogr.}}
\def\jqsrt{{\it J. Quant. Spectrosc. Radiat. Transfer}}
\def\jpca{{\it J. Phys. Chem. A}}
\def\lnc{{\it Lett. Nuov. C}}
\def\mac{{\it Math. Comp.}}
\def\map{{\it Meteorol. Atmos. Physics.}}
\def\mem{{\it Meteorol. Mag.}}
\def\mnras{{\it Mon. Not. Roy. Astron. Soc.}} 
\def\mwr{{\it Mon. Weather Rev.}} 
\def\nat{{\it Nature}}
\def\pac{{\it Parallel Computing}}
\def\pag{{\it Pure Appl. Geophys.}}
\def\pal{{\it Paleoceanography}}
\def\pht{{\it Physics Today}}
\def\pieee{{\it Proc. IEEE}}
\def\pla{{\it Phys. Lett. A}}
\def\ppp{{\it Paleogeogr. Paleoclim. Paleoecol.}}
\def\pra{{\it Phys. Res. A}}
\def\prd{{\it Phys. Rev. D}}
\def\prl{{\it Phys. Rev. L}}
\def\pss{{\it Planet. Space Sci.}}
\def\ptrsl{{\it Phil. Trans. R. Soc. Lond.}}
\def\qjrms{{\it Q. J. R. Meteorol. Soc.}}
\def\qres{{\it Quat. Res.}}
\def\qsr{{\it Quatern. Sci. Rev.}}
\def\reg{{\it Rev. Geophys.}}
\def\rgsp{{\it Revs. Geophys. Space Phys.}}
\def\rpp{{\it Rep. Prog. Phys.}}
\def\sca{{\it Sci. Amer.}}
\def\sci{{\it Science}}
\def\sjna{{\it SIAM J. Numer. Anal.}}
\def\sjssc{{\it SIAM J. Sci. Stat. Comput.}}
\def\tac{{\it Theor. Appl. Climatl.}}
\def\tel{{\it Tellus}}
\def\wea{{\it Weather}}

%SIAM Review: (Society for Industrial and Applied Mathematics)
%       J. on Computing
%       J. on Control and Optimization
%       J. on Algebraic and Discrete Methods
%       J. on Numerical Analysis 
%       J. on Scientific and Statistical Computing



 % AGU-sanctioned journal title abbreviations

\def\aapgb{{\it Amer. Assoc. Petroleum Geologists Bull.}}
\def\adg{{\it Adv. Geophys.}}
\def\ajs{{\it Amer. J. Sci.}}
\def\amb{{\it Ambio}}
\def\amgb{{\it Arch. Meteorol. Geophys. Bioclimatl.}}
\def\ang{{\it Ann. Glaciol.}}
\def\angeo{{\it Ann. Geophys.}}
\def\apo{{\it Appl. Opt.}}
\def\areps{{\it Ann. Rev. Earth Planet. Sci.}}
\def\asr{{\it Adv. Space Res.}}
\def\ate{{\it Atmos. Environ.}}
\def\atf{{\it Atmosfera}}
\def\atms{{\it ACM Trans. Math Software}}
\def\ato{{\it Atmos. Ocean}}
\def\atr{{\it Atmos. Res.}}
\def\gbc{{\it Global Biogeochem. Cycles}} % csz
\def\blm{{\it Boundary-Layer Meteorol.}} % csz 
\def\bpa{{\it Beitr. Phys. Atmosph.}}
\def\bams{{\it Bull. Am. Meteorol. Soc.}}
\def\clc{{\it Clim. Change}}
\def\cld{{\it Clim. Dyn.}}
\def\com{{\it Computing}}
\def\dao{{\it Dyn. Atmos. Oceans}}
\def\dsr{{\it Deep-Sea Res.}}
\def\esr{{\it Earth Sci. Revs.}}
\def\gec{{\it Geosci. Canada}}
\def\gei{{\it Geofis. Int.}}
\def\gej{{\it Geogr. J.}}
\def\gem{{\it Geophys. Monogr.}}
\def\geo{{\it Geology}}
\def\grl{{\it Geophys. Res. Lett.}}
\def\ieeec{{\it IEEE Computer}}
\def\ijna{{\it IMA J. Numer. Anal.}}
\def\ijnmf{{\it Int. J. Num. Meteorol. Fl.}}
\def\jac{{\it J. Atmos. Chem.}}
\def\jacm{{\it J. Assoc. Comput. Mach.}}
\def\jam{{\it J. Appl. Meteorol.}}
\def\jas{{\it J. Atmos. Sci.}}
\def\jatp{{\it J. Atmos. Terr. Phys.}}
\def\jcam{{\it J. Climate Appl. Meteorol.}}
\def\jchp{{\it J. Chem Phys.}}
\def\jcis{{\it J. Coll. I. Sci.}}
\def\jcl{{\it J. Clim.}}
\def\jcp{{\it J. Comput. Phys.}}
\def\jfm{{\it J. Fluid Mech.}}
\def\jgl{{\it J. Glaciol.}}
\def\jgr{{\it J. Geophys. Res.}}
\def\jgs{{\it J. Geol. Soc. London}}
\def\jme{{\it J. Meteorol.}}
\def\jmr{{\it J. Marine Res.}}
\def\jmsj{{\it J. Meteorol. Soc. Jpn.}}
\def\josa{{\it J. Opt. Soc. A}}
\def\jpo{{\it J. Phys. Oceanogr.}}
\def\jqsrt{{\it J. Quant. Spectrosc. Radiat. Transfer}}
\def\jpca{{\it J. Phys. Chem. A}}
\def\lnc{{\it Lett. Nuov. C}}
\def\mac{{\it Math. Comp.}}
\def\map{{\it Meteorol. Atmos. Physics.}}
\def\mem{{\it Meteorol. Mag.}}
\def\mnras{{\it Mon. Not. Roy. Astron. Soc.}} 
\def\mwr{{\it Mon. Weather Rev.}} 
\def\nat{{\it Nature}}
\def\pac{{\it Parallel Computing}}
\def\pag{{\it Pure Appl. Geophys.}}
\def\pal{{\it Paleoceanography}}
\def\pht{{\it Physics Today}}
\def\pieee{{\it Proc. IEEE}}
\def\pla{{\it Phys. Lett. A}}
\def\ppp{{\it Paleogeogr. Paleoclim. Paleoecol.}}
\def\pra{{\it Phys. Res. A}}
\def\prd{{\it Phys. Rev. D}}
\def\prl{{\it Phys. Rev. L}}
\def\pss{{\it Planet. Space Sci.}}
\def\ptrsl{{\it Phil. Trans. R. Soc. Lond.}}
\def\qjrms{{\it Q. J. R. Meteorol. Soc.}}
\def\qres{{\it Quat. Res.}}
\def\qsr{{\it Quatern. Sci. Rev.}}
\def\reg{{\it Rev. Geophys.}}
\def\rgsp{{\it Revs. Geophys. Space Phys.}}
\def\rpp{{\it Rep. Prog. Phys.}}
\def\sca{{\it Sci. Amer.}}
\def\sci{{\it Science}}
\def\sjna{{\it SIAM J. Numer. Anal.}}
\def\sjssc{{\it SIAM J. Sci. Stat. Comput.}}
\def\tac{{\it Theor. Appl. Climatl.}}
\def\tel{{\it Tellus}}
\def\wea{{\it Weather}}

%SIAM Review: (Society for Industrial and Applied Mathematics)
%       J. on Computing
%       J. on Control and Optimization
%       J. on Algebraic and Discrete Methods
%       J. on Numerical Analysis 
%       J. on Scientific and Statistical Computing



 % AGU-sanctioned journal title abbreviations

\def\aapgb{{\it Amer. Assoc. Petroleum Geologists Bull.}}
\def\adg{{\it Adv. Geophys.}}
\def\ajs{{\it Amer. J. Sci.}}
\def\amb{{\it Ambio}}
\def\amgb{{\it Arch. Meteorol. Geophys. Bioclimatl.}}
\def\ang{{\it Ann. Glaciol.}}
\def\angeo{{\it Ann. Geophys.}}
\def\apo{{\it Appl. Opt.}}
\def\areps{{\it Ann. Rev. Earth Planet. Sci.}}
\def\asr{{\it Adv. Space Res.}}
\def\ate{{\it Atmos. Environ.}}
\def\atf{{\it Atmosfera}}
\def\atms{{\it ACM Trans. Math Software}}
\def\ato{{\it Atmos. Ocean}}
\def\atr{{\it Atmos. Res.}}
\def\gbc{{\it Global Biogeochem. Cycles}} % csz
\def\blm{{\it Boundary-Layer Meteorol.}} % csz 
\def\bpa{{\it Beitr. Phys. Atmosph.}}
\def\bams{{\it Bull. Am. Meteorol. Soc.}}
\def\clc{{\it Clim. Change}}
\def\cld{{\it Clim. Dyn.}}
\def\com{{\it Computing}}
\def\dao{{\it Dyn. Atmos. Oceans}}
\def\dsr{{\it Deep-Sea Res.}}
\def\esr{{\it Earth Sci. Revs.}}
\def\gec{{\it Geosci. Canada}}
\def\gei{{\it Geofis. Int.}}
\def\gej{{\it Geogr. J.}}
\def\gem{{\it Geophys. Monogr.}}
\def\geo{{\it Geology}}
\def\grl{{\it Geophys. Res. Lett.}}
\def\ieeec{{\it IEEE Computer}}
\def\ijna{{\it IMA J. Numer. Anal.}}
\def\ijnmf{{\it Int. J. Num. Meteorol. Fl.}}
\def\jac{{\it J. Atmos. Chem.}}
\def\jacm{{\it J. Assoc. Comput. Mach.}}
\def\jam{{\it J. Appl. Meteorol.}}
\def\jas{{\it J. Atmos. Sci.}}
\def\jatp{{\it J. Atmos. Terr. Phys.}}
\def\jcam{{\it J. Climate Appl. Meteorol.}}
\def\jchp{{\it J. Chem Phys.}}
\def\jcis{{\it J. Coll. I. Sci.}}
\def\jcl{{\it J. Clim.}}
\def\jcp{{\it J. Comput. Phys.}}
\def\jfm{{\it J. Fluid Mech.}}
\def\jgl{{\it J. Glaciol.}}
\def\jgr{{\it J. Geophys. Res.}}
\def\jgs{{\it J. Geol. Soc. London}}
\def\jme{{\it J. Meteorol.}}
\def\jmr{{\it J. Marine Res.}}
\def\jmsj{{\it J. Meteorol. Soc. Jpn.}}
\def\josa{{\it J. Opt. Soc. A}}
\def\jpo{{\it J. Phys. Oceanogr.}}
\def\jqsrt{{\it J. Quant. Spectrosc. Radiat. Transfer}}
\def\jpca{{\it J. Phys. Chem. A}}
\def\lnc{{\it Lett. Nuov. C}}
\def\mac{{\it Math. Comp.}}
\def\map{{\it Meteorol. Atmos. Physics.}}
\def\mem{{\it Meteorol. Mag.}}
\def\mnras{{\it Mon. Not. Roy. Astron. Soc.}} 
\def\mwr{{\it Mon. Weather Rev.}} 
\def\nat{{\it Nature}}
\def\pac{{\it Parallel Computing}}
\def\pag{{\it Pure Appl. Geophys.}}
\def\pal{{\it Paleoceanography}}
\def\pht{{\it Physics Today}}
\def\pieee{{\it Proc. IEEE}}
\def\pla{{\it Phys. Lett. A}}
\def\ppp{{\it Paleogeogr. Paleoclim. Paleoecol.}}
\def\pra{{\it Phys. Res. A}}
\def\prd{{\it Phys. Rev. D}}
\def\prl{{\it Phys. Rev. L}}
\def\pss{{\it Planet. Space Sci.}}
\def\ptrsl{{\it Phil. Trans. R. Soc. Lond.}}
\def\qjrms{{\it Q. J. R. Meteorol. Soc.}}
\def\qres{{\it Quat. Res.}}
\def\qsr{{\it Quatern. Sci. Rev.}}
\def\reg{{\it Rev. Geophys.}}
\def\rgsp{{\it Revs. Geophys. Space Phys.}}
\def\rpp{{\it Rep. Prog. Phys.}}
\def\sca{{\it Sci. Amer.}}
\def\sci{{\it Science}}
\def\sjna{{\it SIAM J. Numer. Anal.}}
\def\sjssc{{\it SIAM J. Sci. Stat. Comput.}}
\def\tac{{\it Theor. Appl. Climatl.}}
\def\tel{{\it Tellus}}
\def\wea{{\it Weather}}

%SIAM Review: (Society for Industrial and Applied Mathematics)
%       J. on Computing
%       J. on Control and Optimization
%       J. on Algebraic and Discrete Methods
%       J. on Numerical Analysis 
%       J. on Scientific and Statistical Computing




%\usepackage{/home/zender/tex/jrn_abb} % Define abbreviations used in bib.bib

% NB: The \lefthead and \righthead will be automatically uppercased by
% the jgrga documentstyle 
\lefthead{Zender and Kiehl}
%\righthead{Climate sensitivity to anvil condensate phase, height, and size}
\righthead{Climate sensitivity to tropical anvil representation}
\received{date1}
\revised{date2}
\accepted{date3}
\journalid{JGRD}{Journal date}
\articleid{page1}{page2}
\paperid{94JZ12345}
% The $ in the following line screws up the hilit19 highlighting
%\ccc{0000-0000/00/94JZ-12345\$05.00}
\ccc{0000-0000/00/94JZ-12345\05.00}
% \cpright{PD}{1994}
% \cpright{Crown}{1994}
% (No \ccc{} for Crown copyrights.)
\cpright{AGU}{1994}

\authoraddr{J. T. Kiehl and Charles S. Zender, National Center for Atmospheric
Research, P.O. Box 3000, Boulder, CO 80307-3000}

%\slugcomment{Submitted to \jcl, May~13, 1996. Copyright \copyright 1996 AMS.}
\slugcomment{\today}

\begin{document}
} % not phdcsz

% NB: title must be lowercased by hand. The documentstyle does not enforce it.
%\def\paperchaptertitle{Sensitivity of a climate simulation to vertical
%distribution, phase, and size of anvil condensate through radiative effects}  
\def\paperchaptertitle{Sensitivity of a climate simulation to
representation of tropical anvil}  
\ifphdcsz{\chapter{\paperchaptertitle}\label{chap:gcm}}{\title{\paperchaptertitle}}
\ifphdcsz{}{\author{Charles S. Zender and J. T. Kiehl}}
\ifphdcsz{}{\affil{National Center for Atmospheric Research, Boulder, Colorado}}

\ifphdcsz{\section{Abstract}}{\begin{abstract}}
Climate sensitivity to the representation of tropical anvil is
investigated using a simple prognostic ice cloud scheme in a 
version of the National Center for Atmospheric Research Community
Climate Model.
Two numerical climate integrations are forced by 1985--1989 sea
surface temperature (SST):
the control, with diagnostic ice cloud, and the experiment, with
ice cloud forecast from modeled and observed characteristics of
tropical anvil lifecycle (i.e., convective origin, vertical profile,
phase, mesoscale precipitation, and lifetime).
The sensitivity experiment predicts enhanced ice and reduced liquid
over most latitudes.    
Increase in hydrometeor size associated with anvil weakens cloud
radiative extinction per unit mass by factors of 1--3.
The weaker mass extinction efficiency approximately balances enhanced
ice amount so that anvil ice mass quadruples without biasing the mean
radiative energy balance.

Enhanced anvil perturbs the tropical upper troposphere temperature
structure more strongly in winter, when the column is clearer and
anvil radiatively heats the troposphere above 200~mb. 
In the summer tropics, enhanced anvil reduces radiative cooling up to 
200~mb, and enhances cooling above that.
The prognostic anvil is less strongly coupled to SST.
Based on the 1987 El Ni\~no, the underprediction of shortwave cloud
forcing sensitivity to SST indicates anvil ice crystal size or habit
may be sensitive to SST.
The prognostic anvil formulation improves longwave cloud radiative
response to SST cooling but worsens response to warming $> 2\
\dgr$C.  
In conjunction with weaker mass extinction efficiency of large ice,
this weakens the feedback between longwave cloud forcing and
convection over SST maxima.  
The net response of convection is a shift toward the winter hemisphere
in solstice months.
These changes lead to a significant response in the extratropical
height field in January. 

% Reread the paper, find the clearest, strongest points, and put them
% in the abstract
\ifphdcsz{}{\end{abstract}}

\section{Introduction}\label{sec:gcm_intro}

The role of cirrus anvil in determining the diabatic heating which
drives the general circulation is widely recognized.
Previous studies \cite[e.g.,][]{WeS80,ALV88,RaR891} show the sign and
magnitude of radiative heating in the upper troposphere depend
strongly on cirrus height, thickness, mass, and microphysical
characteristics such as particle size and shape. 
Cirrus anvils produce a radiative influence long after deep convection
ceases \cite[]{ALV88}, and respond to equatorial sea surface
temperature (SST) increase by radiatively reducing column energy
\cite[]{RaC91}. 
Anvils further influence the climate system by processing moisture 
\cite[]{LeH80} and storing water in the upper troposphere
\cite[]{SuL931}.  
This study examines how condensate amount, ice fraction, and particle
size interact to radiatively force the climate system in a general
circulation model (GCM) when the anvil ice distribution is determined
by a prognostic scheme.

%The prevalence of non-convective cirrus in the atmosphere is
%documented by \cite{WHL88} and \cite{RoS91}.

In \cite{ZeK961} we developed a simple prognostic ice scheme, called
ANV, that encapsulates the following characteristics of tropical
anvils:
Deep convection is the ultimate source of tropical upper tropospheric
extended clouds, i.e., tropical anvils \cite[e.g.,][]{WeS80}. 
The relative area of convecting cores to the associated anvil is
$\lesssim 10\%$ \cite[]{FDR90}.
Anvil lifetime, typically 6~hr \cite[]{ALV88}, exceeds the duration 
of deep convection by many hours.
Anvil generates roughly 40\% of total precipitation in tropical
convective systems \cite[]{GaH83}. 
Observations and numerical simulations
\cite[e.g.,][]{WSS931,SLT94,GMK95} show time mean condensate mixing
ratio \qc\ in tropical convective systems varies little (and can
increase), from the freezing level to $\sim 300$~mb, above which \qc\
decreases rapidly.  

In the present study we replace the diagnostic ice cloud in the 
National Center for Atmospheric Research Community Climate Model (NCAR
CCM) with prognostic ice cloud from ANV to study climate sensitivity
to the radiative effects of these observed and modeled anvil features. 
Changes in condensate distribution due to the above details of anvil
specification are shown to significantly impact simulated climate
through three mechanisms: ice fraction, crystal size, and vertical
distribution. 

%Advantages of a prognostic, rather than diagnostic, condensate
%treatment in a GCM include allowing cloud to act as a tropospheric
%moisture reservoir \cite{SuL931}, allowing water transport to occur in
%the condensed phase (where its radiative effects are enhanced), and

The CCM and ANV models are summarized in section~\ref{sec:gcm_model}. 
Section~\ref{sec:vld} compares the models' mean global simulated
climate statistics to each other and analyses.
Section~\ref{sec:sst} examines response to SST forcing in the 1987 El
Ni\~no. 
Section~\ref{sec:reg} describes the seasonal cycle of the energy
budget of the central Indian Ocean.
Section~\ref{sec:gcm_disc} contains a discussion and conclusions.

\section{Model Description}\label{sec:gcm_model}

The control model, denoted CCM, contains four improvements over CCM2
\cite[]{HBB93} in the area of cloud physics:   
Ice phase condensate is treated with the radiative scheme of
\cite{EbC92}.
A continental/maritime distinction is made in the effective radius of
cloud droplets \cite[]{Kie942}.
Total condensed water path (CWP) is diagnosed from column precipitable
water rather than prescribed by latitude. 
Large scale precipitation can evaporate between the cloud and surface.

The sensitivity experiment, denoted ANV, is the prognostic ice scheme
of \cite{ZeK961} implemented in this version of the CCM (details in
Appendix~\ref{app:anv_imp}).  
Only radiative feedbacks of prognostic ice are enabled, thermodynamic
feedbacks (i.e., latent heating and moisture) are turned off in ANV.
This section presents the CCM and ANV formulations for cloud
condensate, ice fraction, hydrometeor size, and their radiative
implications.

\subsection{Condensate Formation}\label{sec:qc}

CCM utilizes a diagnostic treatment of condensate.
In CCM, grid box average \qc\ is logarithmically proportional to
total column  vapor \Qv\ and linearly proportional to local cloud
fraction $A$ through
\begin{eqnarray}
\label{eqn:qc}
\qc & = & A \rho_\ell / \rho \\
\label{eqn:rhol}
\rho_\ell & = & \rho_\ell^0 e^{(-z/h_\ell)} \\
\label{eqn:hl}
h_\ell & = & 810 \ln \Qv
\end{eqnarray}
where $\rho$ is density, $\rho_\ell^0 \equiv .18$~\gxmC\
is ``in-cloud'' condensed water density at the surface, $z$ is height,
and $h_\ell$ is the scale height of condensed water.
Note the strong coupling of \qc\ to local surface temperature by the
Clausius-Clapeyron relationship implicit in (\ref{eqn:hl}).
Inserting tropical values for \Qv\ we find $h_\ell \lesssim 4$~km so
that \qc\ monotonically decreases from the surface (for fixed $A$).
A drawback to this procedure is that the upper tropospheric peak in
\qc\ profile of an anvil system \cite[e.g., Figure~3 of][]{ZeK961}
must be captured by significantly modulating $A$ across the anvil deck
(\ref{eqn:qc}).  

ANV treats ice condensate as a prognostic variable.
In ANV, ice production for the entire convecting column above freezing
level is proportional to convective mass flux at 500~mb.
New ice is homogeneously distributed to all freezing, convecting
layers. 
This apportionment increases condensate sequestered in the upper
troposphere, a primary goal of this study.  
Precipitation is forecast from mesoscale budget estimates.
Stratiform ice generation is prognosed from vertical updraft,
temperature, and cloud fraction.  

\subsection{Ice Fraction}\label{sec:fice}

\cite{SuS94} analyze the importance of the cloud ice mass fraction,
\fice, to radiative fluxes. 
They show treating mixed phase cloud as 100\% liquid can result in
greater error than neglecting the ice mass completely.
Insufficient data exist to know the climatological \fice\ of anvil
with certainty.
Metastable supercooled liquid in cloud is known to occur at $-40\
\dgr$C \cite[]{HeS89}, but not in significant quantities at GCM
scale (i.e., over areas of $\sim 100^2$~\kmS\ for $\sim 1$~hr).   
Motivated by data of \cite{RaH94}, CCM partitions diagnostic
condensate between liquid and ice via  
\begin{equation}
\ficeccm = \cases{
  0 & $T > - 10$~\dgr C \cr
  -{T+10 \over 20} & $-10 \geq T \geq -30$~\dgr C \cr
  1 & $T < -30$~\dgr C \cr}
\label{eqn:ficeccm}
\end{equation}
Motivated by recent mid-latitude observations \cite[]{MoJ94} and
tropical simulations \cite[e.g., Figure~3 of][]{ZeK961}, ANV restricts
diagnostic liquid condensate (and, hence, mixed phase cloud) to a
narrower and warmer range than the CCM (\ref{eqn:ficeccm}) according
to
\begin{equation}
\ficeanv = \cases{
  0 & $T > 0$~\dgr C \cr
  -{T \over 5} & $0 \geq T \geq -5$~\dgr C \cr
  1 & $T < -5$~\dgr C \cr}
\label{eqn:ficeanv}
\end{equation}
Since the CCM classifies some condensate as cold as $-30~\dgr$C as
liquid (\ref{eqn:ficeccm}), while ANV has no liquid colder than
$-5$~\dgr C (\ref{eqn:ficeanv}), it is clear the ANV atmosphere
will contain much more ice than the CCM. 
The coldest liquid water in ANV, at $-5\ \dgr$C, is warmer than 
observations and is further discussed in Section~\ref{sec:cld}.

\subsection{Hydrometeor Size}\label{sec:rdsffc}

Model cloud droplet effective radius \rel\ is fixed at 10~\um\ over
ocean and sea ice but varies from 5--10~\um\ over land.
Ice crystal effective radius \rei\ represents an equivalent surface
area sphere and varies linearly with a normalized pressure coordinate.
Over ocean this results in $\rei = 10$~\um\ for $p > 800$~mb to  
$\rei = 30$~\um\ for $p < 400$~mb.

Due to enlargement of the glaciated atmosphere in ANV
(\ref{eqn:ficeccm}--\ref{eqn:ficeanv}), and the pressure/phase
dependence of \rdsffc, hydrometeor size plays as important a role as
condensate amount in determining net radiative effects of 
condensate differences in the present experiment.
The simulated tropical temperature structure led to the following
distribution of ice fraction and \rdsffc: 
The majority of \qc\ is large ice crystal ($\rei = 25$--30~\um)
for $p < 550$~mb in ANV and $p < 300$~mb in CCM. 
CCM has half liquid droplet ($\rel = 10$~\um) and half large ice for
$300 < p < 400$~mb.
The majority of \qc\ is liquid droplet for $p > 400$~mb in the CCM and
$p > 550$~mb in ANV.

We realize prescribing \rdsffc\ based on pressure and phase (i.e.,
temperature) is not physically consistent with mass-conserving
microphysics because it ignores the relation between size
distribution, \rdsffc, and \qc.
We employ diagnostic \rdsffc\ in the interim as prognostic
parameterizations for \rdsffc\ (which require assumptions about, e.g.,
size distribution and aerosol effects) mature. 

\subsection{Cloud Radiative Properties}\label{sec:ebc}

The radiative treatment of liquid, ice, and mixed phase cloud in CCM
is reported by \cite{Kie961}.
We summarize the salient features here.
The radiative properties of a cloud layer depend to first order on
cloud mass (expressed as condensed water path CWP in \gxmS), phase
(ice, liquid, or mixed), and hydrometeor effective radius (\rdsffc). 
Solar single scattering and longwave emissivity properties are from
\cite{Sli89} (liquid) and \cite{EbC92} (ice).
Cloud extinction optical depth $\tau$ is defined as
\begin{equation}
\label{eqn:tau}
\tau = \CWP (a + b\rdsffc^{-1})
\end{equation}
where $a$ and $b$ are empirical constants with spectral dependence.
For ice, $a$ is small so $\tau$ varies inversely with \rdsffc.
Thus increased CWP can offset increased \rdsffc\ and visa versa. 
Absorption and forward scattering increase with \rdsffc.
Ice crystals of a given (equivalent area) \rdsffc\ absorb more strongly
(in the visible and near-infrared) and scatter more isotropically than
liquid droplets.
Thus ice and mixed phase cloud absorbs more than liquid cloud for a
given $\tau$.
\cite{Kie961} describes the change in heating rate due to ice vs.\
liquid radiative properties for an upper tropospheric ice cloud as a
function of \rdsffc.

Optical depth of a given anvil \qc\ is usually \textit{less} in ANV
than CCM because ice crystals are up to 6 times larger than liquid
droplets in the models. 
In other words, tropical upper tropospheric ice in both models is
significantly more transparent than lower tropospheric (e.g., arctic)
ice and liquid.  
% CZP III p. 121
Consider the example of a single layer cloud of mass 25~\gxmS, which
is comparable to the difference in monthly average CWP between ANV and
CCM over regions of deep convection (Figure~\ref{fig:8589_TOTCWP}
below).  
The visible optical depth of this cloud is 2.1 if upper tropospheric 
ice, 6.1 if lower tropospheric ice, and more than 4.0 if liquid.

Longwave radiative effects of cloud also depend upon CWP, phase,
and \rdsffc. These determine cloud emissivity through
\begin{eqnarray}
\label{eqn:emissivity}
\epsilon & = & 1 - \exp[-1.66(\kappal \LWP + \kappai \IWP)] \\
\label{eqn:kappai1}
\kappai & = & .005 + \rei^{-1}
\end{eqnarray}
where LWP and IWP are the liquid and ice water path, respectively,
the $\kappa$ are longwave mass absorption coefficients (\mSxg), \rei\
is in \um, and $\kappal \equiv .090361$. 
The emissivity of the example cloud is .796 if upper tropospheric ice,
.987 if lower tropospheric ice, and .976 if liquid (approximately the
emissivity of a 65~\gxmS\ upper tropospheric ice cloud).
This scaling is highly non-linear, and diminishes as IWP increases and
the cloud becomes opaque ($\epsilon \rightarrow 1$).

In summary, condensate phase and size are inextricably linked to 
temperature and pressure through simple parameterizations which
neglect mesoscale conditions. 
We refer to the decrease in $\tau$ (\ref{eqn:tau}) and $\kappa$
(\ref{eqn:kappai1}) per unit mass in 20--30~\um\ ice crystal relative
to 5--10~\um\ liquid droplet as the weaker mass extinction (extinction
efficiency per unit mass) of ice.
Weaker mass extinction characterizes ANV since $\ficeanv > \ficeccm$
(\ref{eqn:ficeccm}--\ref{eqn:ficeanv}).

\subsection{Observational Analyses}\label{sec:obs}

\cite{HBK94} and \cite{KHB94} analyze the climate simulation produced
by CCM2, upon which the present models are based.
We use European Center for Medium Range Weather Forecasts (ECMWF)
analyses archived at T42 resolution on 14 pressure levels
\cite[]{Tre92} for validation of dynamical fields.
Six Januarys and Julys between July, 1989 and January, 1995,
inclusive, were averaged into the climatological analyses below.
The models employ a hybrid vertical coordinate which smoothly
transitions from a sigma (terrain-following) coordinate near the
surface to a pressure coordinate above 200~mb.
Model fields shown are vertically interpolated to pressure levels for
comparison to ECMWF analyses (e.g., $T$, $U$).
Vertical fields without an ECMWF counterpart (e.g., \qc) are plotted
in the model hybrid coordinate but labeled with the approximate
pressure for clarity.

Earth Radiation Budget Experiment (ERBE) data archived at T42
resolution \cite[]{HuC92} are available for 52 months from from
February, 1985 to May, 1989, inclusive. 
Thus ERBE climatologies shown below include five (February--May) or
four (June--January) years of monthly data.
The difference between clear sky and all-sky flux measures the impact
of cloud absorption and scattering on the surface-atmosphere system,
known as cloud forcing. 
The effect of cloud on terrestrial or longwave (LW) radiation, that
is, the reduction in outgoing longwave radiation (OLR) due to cloud
condensate, is called the longwave cloud forcing (LWCF).
Shortwave cloud forcing (SWCF) is defined as the net increase in
reflected shortwave (SW) flux at the top of atmosphere due to
cloud scattering and absorption.

Clear sky data was sometimes unavailable for a given pixel and month,
especially near tropical convective centers.
The climatological values presented average all months of available
data. 
Estimated uncertainty in monthly averaged flux is $\sim 10$~\wxmS.  
The ERBE scene identification algorithm has difficulty distinguishing
ice- and snow-covered surfaces from cloud. 
Therefore cloud forcing data poleward of 60\dgr\ is uncertain.

\section{Model Results and Validation}\label{sec:vld}

CCM and ANV were integrated from September 1, 1984 to January 1, 1990 
using monthly varying SST from AMIP \cite[]{Gat92} as a lower
boundary condition. 
The AMIP dataset is thought to be accurate to 1~\dgr K.
As described above, CCM diagnoses \qi\ while ANV forecasts \qi\ based
on observed and modeled anvil behavior.
Significant differences in the model results show the climate response
to the radiative effects of changed \qc, \fice, and \rdsffc. 

This section illustrates changes in the simulated climate in the
order: cloud, radiation, thermodynamics, dynamics.
The focus is ensemble mean January and July.
Most results are shown as multi-panel figures, January on the left,
July (if included) on the right. 
Observational analyses, if available, are in the top panel.
Models follow in the order CCM then ANV and $\Delta \psi$ refers to 
the inter-model difference (ANV$-$CCM) in predicted $\psi$. 

\subsection{Cloud Climatology}\label{sec:cld}

A global climatology of \qc\ does not exist at present, so
validations of GCM predictions of \qc\ rely on indirect measurements, 
notably top of atmosphere cloud radiative properties. 
We illustrate the following aspects of simulated \qc:
geographic distribution, vertical distribution, and ice fraction.
Hydrometeor size is described in Section~\ref{sec:rdsffc} above.

ANV sequesters more total condensate in the upper troposphere.
Figure~\ref{fig:xavg_8589_QC} contours the ensemble mean January and
July vertical profiles of zonal average condensate \qc.
\begin{figure*}
\ifams{}{\begin{center}
\includegraphics*[width=.5\hsize]{/data/zender/fgr/phd/amip5_xavg_8589_01_QC.eps}%
\includegraphics*[width=.5\hsize]{/data/zender/fgr/phd/amip5_xavg_8589_07_QC.eps}%

\includegraphics*[width=.5\hsize]{/data/zender/fgr/phd/spcp_85_xavg_8589_01_QC.eps}%
\includegraphics*[width=.5\hsize]{/data/zender/fgr/phd/spcp_85_xavg_8589_07_QC.eps}%

\includegraphics*[width=.5\hsize]{/data/zender/fgr/phd/spcp_85_8589_amip5_8589_xavg_01_QC.eps}%
\includegraphics*[width=.5\hsize]{/data/zender/fgr/phd/spcp_85_8589_amip5_8589_xavg_07_QC.eps}%
\end{center}}
\ifams{\figbox{.5\hsize}{4in}{
\includegraphics*[width=.5\hsize]{/data/zender/fgr/phd/amip5_xavg_8589_01_QC.eps}%
\includegraphics*[width=.5\hsize]{/data/zender/fgr/phd/amip5_xavg_8589_07_QC.eps}\vfill
\includegraphics*[width=.5\hsize]{/data/zender/fgr/phd/spcp_85_xavg_8589_01_QC.eps}%
\includegraphics*[width=.5\hsize]{/data/zender/fgr/phd/spcp_85_xavg_8589_07_QC.eps}\vfill
\includegraphics*[width=.5\hsize]{/data/zender/fgr/phd/spcp_85_8589_amip5_8589_xavg_01_QC.eps}%
\includegraphics*[width=.5\hsize]{/data/zender/fgr/phd/spcp_85_8589_amip5_8589_xavg_07_QC.eps}%
}}{}
\caption[Condensate mixing ratio \qc\ for January and July
1985--1989 simulated by CCM, ANV, and ANV$-$CCM]{
Condensate mixing ratio \qc\ (\mgxkg) for (left) January and (right)
July 1985--1989 simulated by (top) CCM, (middle) ANV, and (bottom)
ANV$-$CCM: (a) January CCM, (b) July CCM, (c) January ANV, (d) July
ANV, (e) January ANV$-$CCM, and (f) July ANV$-$CCM. 
For a--d contours are .5, 1, 2, 4, 6, 8, 10, 15, 20, 30, 40, 50, 60,
and 70~\mgxkg\ and shading indicates $\qc > 6$~\mgxkg.
For e--f contour interval is 2~\mgxkg\ and shading indicates where ANV
has less condensate than CCM.  
\label{fig:xavg_8589_QC}}   
\end{figure*}
Contour resolution increases with height to allow comparison of
radiatively important upper tropospheric \qc\ differences.
The strong meridional structure of \qc\ peaks in summertime tropics,
mid-latitudes, and the polar boundary layer. 
The largest model differences occur in the ascending branch of
the Hadley cell, and in winter mid-latitudes.
ANV exceeds CCM \qc\ by up to 10~\mgxkg\ ($\sim 200\%$) from
600--200~mb except in summer polar latitudes. 
Strongly reduced polar \qc, and an associated relaxed winter
inversion, improved simulated LWCF and surface temperature, as 
discussed below.    
We stress ANV did not require more vigorous vertical motion to
increase upper tropospheric \qc\ because ANV can predict more ice than
CCM for a given mass flux profile.

To first order, LWCF varies as $\Ts^4 - \Tc^4$, where \Ts\ and \Tc\
are surface and cloud top temperatures, respectively. 
Small differences in \qc\ near the tropical tropopause can thereby
cause large differences in LWCF.
ANV predicts .1--2~\mgxkg\ more \qc\ than CCM from 100--200~mb;
this corresponds to only .1~\gxmS\ CWP in the zonal monthly average. 

Figure~\ref{fig:8589_TOTCWP} shows the zonal variation of \qc\ in
terms of CWP, the sum of IWP and LWP.
\begin{figure*}
\begin{center}
\includegraphics*[width=.5\hsize]{/data/zender/fgr/phd/amip5_8589_01_TOTCWP.eps}%
\includegraphics*[width=.5\hsize]{/data/zender/fgr/phd/amip5_8589_07_TOTCWP.eps}%

\includegraphics*[width=.5\hsize]{/data/zender/fgr/phd/spcp_85_8589_01_TOTCWP.eps}%
\includegraphics*[width=.5\hsize]{/data/zender/fgr/phd/spcp_85_8589_07_TOTCWP.eps}%
\end{center}
\caption[Geographic distribution of condensed water path from 
January and July 1985--1989 simulated by CCM and ANV]{
Geographic distribution of condensed water path (\gxmS) from 
(left) January and (right) July 1985--1989 simulated by (top) CCM and
(bottom) ANV: (a) January CCM, (b) July CCM, (c) January ANV, and (d)
July ANV. 
Contour interval is 25~\gxmS. 
%Shading indicates $\CWP > 75$~\gxmS.
\label{fig:8589_TOTCWP}}   
\end{figure*}
CWP exhibits stronger zonal structure in the tropics than
mid-latitudes. 
The weakness of mid-oceanic tropical CWP maxima is symptomatic of poor
simulation of deep convection over the maritime ITCZ. 
This deficiency is shown more clearly in the LWCF field below.
ANV enhances CWP by up to 100~\gxmS\ over the American, African, and
Indonesian convective centers.

In January ANV reduces CWP by 10--40~\gxmS\ over a significant
fraction of the Northern continents and by 5~\gxmS\ over the southern
ocean (south of 40~$\dgr$S).
This CWP reduction is largely from reduced stratiform, not convective,
condensate.
CWP maxima over the Andes and Tibetan Plateau are enhanced by
unrealistic locking of precipitating circulations over steep
orography, a known model bias \cite[]{HBK94}. 
%Regions of deep convective activity with $\Delta \CWP > 25$~\gxmS\
%(i.e., Melanesia, West Australia, West China, the Congo and Amazon
%basins in January, and Thailand, Ethiopia, and central America in
%July) show mixed but generally positive changes in LWCF, and generate
%locally enhanced convection due to greatly associated radiative
%heating from 300--200~mb (see below). 

Comparable CWP belies dramatic differences in condensate phase.
Figure~\ref{fig:xavg_8589_CP} distinguishes modeled condensate
burdens by phase.
\begin{figure*}
\begin{center}
\includegraphics*[width=.5\hsize]{/data/zender/fgr/phd/xavg_8589_01_TOTLWP.eps}%
\includegraphics*[width=.5\hsize]{/data/zender/fgr/phd/xavg_8589_07_TOTLWP.eps}%

\includegraphics*[width=.5\hsize]{/data/zender/fgr/phd/xavg_8589_01_TOTIWP.eps}%
\includegraphics*[width=.5\hsize]{/data/zender/fgr/phd/xavg_8589_07_TOTIWP.eps}%

\includegraphics*[width=.5\hsize]{/data/zender/fgr/phd/xavg_8589_01_TOTCWP.eps}%
\includegraphics*[width=.5\hsize]{/data/zender/fgr/phd/xavg_8589_07_TOTCWP.eps}%
\end{center}
\caption[Zonal average column condensate burdens from
1985--1989 January and July simulations by CCM and ANV]{
Zonal average column condensate burdens (\gxmS) from
1985--1989 (left) January and (right) July simulations by (solid) CCM
and (dashed) ANV of (a,b) LWP, (c,d) IWP, and (e,f) CWP.   
\label{fig:xavg_8589_CP}}   
\end{figure*}
Since \qi\ was not archived (or prognosed) in the CCM simulation, we
used monthly mean $T$ and \qc\ to compute \qi\ from
(\ref{eqn:ficeccm}), and determined \ql\ as the residual.
This method accurately estimates \qi\ in the mixed phase region
since $T$ and $\qi/\qc$ are linearly related (\ref{eqn:ficeccm}).

Maritime LWP from the 1987--1995 Special Sensor Microwave/Imager (SSMI)
data of \cite{WeG94} (not shown) has been compared to the simulated data.
Both models underpredict LWP in the ITCZ and winter hemisphere.
CCM and ANV predictions of LWP over the zonally symmetric southern
ocean (Figure~\ref{fig:xavg_8589_CP}a,b) are $\sim 20$~\gxmS\ high and
low, respectively. 
Global satellite data to validate CWP does not exist, but the
prediction of higher CWP in mid-latitudes than tropics is consistent
with SSMI data.  
We also compared the vertical profile of ice fraction in the tropics  
to the three week cumulus ensemble simulation used to develop ANV.
Table~\ref{tbl:fice} shows ensemble July \fice\ averaged over the
equatorial Pacific (140--270~$\dgr$E,
10~$\dgr$S--10~$\dgr$N) for each GCM layer in the mixed phase
region.
\begin{table}
\begin{center}
\caption{Modeled ice fraction \fice\ over the equatorial Pacific in 
July\label{tbl:fice}} 
\vspace{5pt}
\begin{tabularx}{\hsize}{ccccc}
\hline
& & & & \\[-5pt]
%$p$~mb & $T\ \dgr$C & CCM & ANV & CEM \\[4pt]  
% XXX: reinsert footnotes for preprints
$p$~mb & $T\ \dgr$C & CCM & ANV & CEM\ifphdcsz{\footnote{values at
same $T$ from cumulus ensemble model integration
\cite[]{GMK95}.}}{\tablenotemark{\textit{a}}} \\[4pt]  
\hline
% NB: ncwa -O -a lat,lon -w gw -v PRES,T,QC,QICE,QL -d lat,-10.,10. -d lon,140.,270. amip5_8589_07.nc foo.nc
% NB: ncks -C -H -v PRES,T,QICE,QC,QL foo.nc | m
& & & & \\[-6pt]
251 & $-42$ & 1 & 1 & 1 \\ % CEM lvl 27
325 & $-27$ & .84 & 1 & 1 \\ % CEM lvl 24
409 & $-15$ & .24 & 1 & 1 \\ % CEM lvl 20
501 & $-5$ & 0 & .98 & .91 \\ % CEM lvl 16
598 & 2 & 0 & .12 & .25 \\ % CEM lvl 13
695 & $9$ & 0 & 0 & .04 \\ % CEM lvl 11
787 & 15 & 0 & 0 & 0 \\[4pt] % CEM lvl 8
\hline
& & & & \\[-8pt]
\end{tabularx}
\ifphdcsz{}{\tablenotetext{\textit{a}}{values at same $T$ from cumulus
ensemble model integration \cite[]{GMK95}.}}
\end{center}
\end{table}
ANV underestimates liquid condensate near 500~mb but otherwise agrees
with the CEM.
Allowing mixed phase cloud to occur as cold as roughly $-10\
\dgr$C (\ref{eqn:ficeccm}--\ref{eqn:ficeanv}) would alleviate 
biases against both SSMI observations and CEM simulations.
\cite{GrM96} improved the simulated mid-latitude radiation budget by
narrowing the temperature range of mixed phase cloud from
$-15$--0~\dgr C to $-9$--0~\dgr C.

Cloud drives circulation by altering total diabatic heating \QT.
Diabatic heating includes latent heating (\QL), radiation (\QR), and
diffusion (turbulence).
Figure~\ref{fig:xavg_8589_QDIABAT} shows the change in zonal average
\QT.
\begin{figure}
\begin{center}
\includegraphics*[width=\halfwidth]{/data/zender/fgr/phd/spcp_85_8589_amip5_8589_xavg_01_QDIABAT.eps}\vfill
\includegraphics*[width=\halfwidth]{/data/zender/fgr/phd/spcp_85_8589_amip5_8589_xavg_07_QDIABAT.eps}\vfill
\end{center}
\caption[Difference (ANV$-$CCM) in simulated zonal average diabatic
heating \QT\ between ANV and CCM for January and July 1985--1989]{ 
Difference (ANV$-$CCM) in simulated zonal average diabatic heating
\QT\ (\kxday) between ANV and CCM for (a) January and (b) July
1985--1989. 
Contour interval is .1~\kxday. 
Shading indicates where ANV cools more (heats less) than CCM. 
\label{fig:xavg_8589_QDIABAT}}   
\end{figure}
Above the boundary layer on monthly timescales $\QT \approx \QR +
\QL$.
Above 200~mb, where \QL\ is negligible, $\Delta \QT$ is due to $\Delta
\QR$ (not shown) which may be inferred from $\Delta \qc$
(Figure~\ref{fig:xavg_8589_QC}e,f).
Reduction of highly absorbing liquid beneath 400~mb increases the net
infrared flux to the upper troposphere.
This allows ANV to cool more efficiently beneath 400~mb but enhances
anvil base heating from 400--200~mb.
Enhanced anvil ice above 200~mb increases cloud top cooling in summer
deep cloud regions but increases anvil heating in the clearer regions
of winter subsidence.  

Examination of solar heating (not shown) reveals reduced optical depth
keeps the intrinsically greater solar absorption of ice relative to
liquid (Section~\ref{sec:ebc}) from causing a ubiquitous heating
increase above 600~mb.  
Solar heating change in the upper troposphere was secondary to LW
heating change, but ameliorated LW cooling above 250~mb in summer and
bolstered LW heating in winter.   
% We know the winter hemisphere TUT $\Delta \QR$ above 200~mb is > 0
% because of enhanced \qc\ absorption and not weaker optical
% properties leading to enhanced net (upwards positive) LW flux at the
% level because QSW is also positive above 200~mb, which can only be
% due to enhanced \qc\ (recall fice = 1 above 300~mb in both models,
% so intrinsically greater absorption (1-omegahat) of ice cannot
% explain QSW differences above 300~mb).

Thus above 200~mb the \QR\ signal from enhanced tropical anvil is the
heating/cooling dipole straddling the equator in
Figure~\ref{fig:xavg_8589_QDIABAT}. 
The warming pole is $\gtrsim 2$ times stronger than the cooling in
most months.
The dipole structure shows enhanced anvil ice
(Figure~\ref{fig:xavg_8589_QC}) modulates mean \QR\ by the \QR\
profile of ``deep'' cloud in the ascending Hadley region, and of
``anvil'' cloud in the descending region \cite[terminology of][their
Figure~16]{RaR891}. 
The radiative relaxation time of anvil from 200--100~mb is
15--75~days.  

Change in extratropical \QT\ is also dominated by \QR.
Reduced \qc\ (Figure~\ref{fig:xavg_8589_QC}e,f) and optical depth
reduce shortwave heating above 500~mb in the summer hemisphere
extratropics; excess insolation and stronger absorption per unit mass
of ice increase solar heating beneath 500~mb. 
Longwave cooling in the lower extratropical troposphere is enhanced by
reduced condensate.
Consequently a layer of the mid-troposphere is heated (relatively) by
exposure to warmer radiating surfaces beneath.
The CCM confines cloud to the polar boundary layer over Antarctica in 
July while ANV predicts more \qc\ above the polar night inversion and
less beneath (Figure~\ref{fig:xavg_8589_QC}). 
Cloud in the inversion enhances cooling since it is warmer than the
surface (there is no compensating SW heating in polar night). 
Thus ANV radiatively weakens the Antarctic winter inversion.
% NB: Magnitude of polar nite heating signal is 10 K/day in SHW, 1.5
% K/day in NHW.

Change in tropical \QT\ beneath 200~mb is dominated by change in \QL.
Convection intensifies from 0--10~\dgr N in ANV in both seasons,
reflecting an enhanced ITCZ, notably over Micronesia, the east
Indian Ocean, and northeast of Brazil. 
Deep convection in the remainder of the ascending branch of the Hadley
circulation is reduced. 
The cause of the convective reorganization involves a weaker linking
between SST, \qc, LWCF, and convection; this is discussed below.
Change in tropical water vapor (not shown) strongly resembles $\Delta
\QL$. 

\subsection{Radiation Fields}\label{sec:rad}

In many respects the present CCM radiative climatology is similar to,
but improved upon, NCAR CCM2's \cite[]{KHB94}. 
The present analysis focuses on changes from CCM to ANV.
We show cloud forcing without albedo or OLR since the direct radiative
effects of changes in water vapor were $\mathcal{O}(10\%)$ of the 
effect from changes in \qc\ and its radiative properties.  
The near-global increase in ice fraction prescribed
(\ref{eqn:ficeccm}--\ref{eqn:ficeanv}) and predicted
(Figure~\ref{fig:xavg_8589_CP}) by ANV tends to reduce cloud forcing 
by increasing \rdsffc\ which weakens mass extinction efficiency
(\ref{eqn:tau}--\ref{eqn:kappai1}). 
Thus in regions of cloud forcing constancy or increase, ANV increased
CWP (or, for LWCF, possibly cloud height) enough to overcome weaker
mass extinction.

Qualitatively, SWCF convolves CWP, \rdsffc, and insolation.
Thus SWCF peaks over convective centers and summertime extratropics. 
Figure~\ref{fig:8589_SWCF} compares modeled to observed SWCF in
January and July conditions.
\begin{figure*}
\begin{center}
\includegraphics*[width=.5\hsize]{/data/zender/fgr/phd/erbe_b_8589_01_SWCF.eps}%
\includegraphics*[width=.5\hsize]{/data/zender/fgr/phd/erbe_b_8589_07_SWCF.eps}%

\includegraphics*[width=.5\hsize]{/data/zender/fgr/phd/amip5_8589_01_SWCF.eps}%
\includegraphics*[width=.5\hsize]{/data/zender/fgr/phd/amip5_8589_07_SWCF.eps}%

\includegraphics*[width=.5\hsize]{/data/zender/fgr/phd/spcp_85_8589_01_SWCF.eps}%
\includegraphics*[width=.5\hsize]{/data/zender/fgr/phd/spcp_85_8589_07_SWCF.eps}%
\end{center}
\caption[Geographic distribution of shortwave cloud forcing SWCF
for 1985--1989 January and July ERBE, CCM, and ANV]{
Geographic distribution of shortwave cloud forcing SWCF (\wxmS) for
1985--1989 January and July (a,b) ERBE, (c,d) CCM, and (e,f) ANV.  
\label{fig:8589_SWCF}}   
\end{figure*}
Neither model captures the narrow, mid-oceanic ITCZ tongue (also true
for LWCF, shown below).
The simulated South Pacific convergence zone is more diffuse in ANV.

ANV increases SWCF (decreases insolation) over West Australia,
Southeast Asia, and North Africa in January and over the US, Panama,
Australia, and equatorial oceans in July. 
ANV decreases SWCF over the central Indian Ocean and circumpolar
region in January, and over the central Pacific (from
0--35~$\dgr$N) and Arctic in July. 

Excessive cloud forcing over winter desert in ANV has two causes.
First, subsidence over African and Arabian desert decreased as upper
level divergence in the Indo-Pacific decreased (shown below).
This allows more cloud to form.
Second, ANV removes all ice cloud with parameterized sublimation and
precipitation processes with fixed timescales that have no explicit
dependence on saturation vapor deficit \cite[]{ZeK961}.
The timescales of these removal processes are inappropriately long for
winter desert, leading to excessively persistent cloud.

Figure~\ref{fig:xavg_8589_SWCF} shows zonal average SWCF for January
and July. 
\begin{figure}
\begin{center}
\includegraphics*[width=\halfwidth]{/data/zender/fgr/phd/xavg_8589_01_SWCF.eps}\vfill
\includegraphics*[width=\halfwidth]{/data/zender/fgr/phd/xavg_8589_07_SWCF.eps}\vfill
\end{center}
\caption[Zonal average shortwave cloud forcing SWCF from ERBE,
CCM, and ANV for January and July]{ 
Zonal average shortwave cloud forcing SWCF (\wxmS) from ERBE
(solid), CCM (dotted), and ANV (dashed) for (a) January and (b) July.
\label{fig:xavg_8589_SWCF}}   
\end{figure}
The models predict similar equatorial LWP beneath 600~mb 
(Figures~\ref{fig:xavg_8589_QC}c,d and~\ref{fig:xavg_8589_CP}a,b),
but ANV has up to 5 times more equatorial IWP.
Thus agreement in modeled tropical zonal SWCF illustrates how
increased IWP can radiatively offset increased \rdsffc\ (\ref{eqn:tau}).
The weak January zonal maxima in ANV at 60~$\dgr$S resembles the
disparity in modeled CWP (Figure~\ref{fig:xavg_8589_CP}e).
% NB: This is interesting because you would infer from the 60S maxima 
% that SWCF is entirely due to CWP differences, not tau/fice changes.
% Vertical profile/mean photon path length differences might be
% important here.
ANV worsens the zonal bias at the July ITCZ by improving (increasing)
SWCF in the equatorial east Pacific and Atlantic oceans without
reducing SWCF in the Indo-Pacific.
ANV eliminates nearly half the 40~\wxmS\ bias poleward of
60~$\dgr$N in July.

Qualitatively, LWCF convolves CWP with $\Ts^4 - \Tc^4$.
Thus LWCF peaks in the tropics and declines toward the poles.
Figure~\ref{fig:8589_LWCF} compares modeled to observed LWCF.
\begin{figure*}[p]
\begin{center}
\includegraphics*[width=.5\hsize]{/data/zender/fgr/phd/erbe_b_8589_01_LWCF.eps}%
\includegraphics*[width=.5\hsize]{/data/zender/fgr/phd/erbe_b_8589_07_LWCF.eps}%

\includegraphics*[width=.5\hsize]{/data/zender/fgr/phd/amip5_8589_01_LWCF.eps}%
\includegraphics*[width=.5\hsize]{/data/zender/fgr/phd/amip5_8589_07_LWCF.eps}%

\includegraphics*[width=.5\hsize]{/data/zender/fgr/phd/spcp_85_8589_01_LWCF.eps}%
\includegraphics*[width=.5\hsize]{/data/zender/fgr/phd/spcp_85_8589_07_LWCF.eps}%
\end{center}
\caption[Geographic distribution of longwave cloud forcing LWCF
for 1985--1989 January and July ERBE, CCM, and ANV]{
Geographic distribution of longwave cloud forcing LWCF (\wxmS) for
1985--1989 January and July (a,b) ERBE, (c,d) CCM, and (e,f) ANV. 
\label{fig:8589_LWCF}}   
\end{figure*}
Neither model captures the mid-oceanic ITCZ.
ANV introduced or worsened LWCF biases in January over southeast Asia,
greater Bangladesh, the northern hemisphere storm track, and North
African and Arabian desert.
ANV ameliorated or eliminated biases in January over Eurasia (north of
40~$\dgr$N), Micronesia, the west Indian and circumpolar oceans
in January, and over Greenland and the northeast Atlantic and central
Pacific oceans in July.
Cirrus reduction over the Indian Ocean in January accompanies a
significant improvement in regional upper level divergence (shown
below).  

The change in maritime LWCF is of particular interest since SST
forcing was identical in the models.
The strongest tropical maritime LWCF reductions occurred over the
center of the Pacific and Indian Ocean warm pools, where SST gradients
(not shown) are small.
Recall CCM \qc, which determines LWCF, varies approximately
exponentially with SST during convection
(\ref{eqn:qc}--\ref{eqn:hl}). 
The SST boundary condition damps short timescale negative feedbacks to
convection (surface evaporation and SWCF) while allowing positive
feedbacks (LWCF, latent heating, low-level convergence).  
Thus (\ref{eqn:qc}--\ref{eqn:hl}) enhance cooperation between \qc,
LWCF, and convection, attracting CCM LWCF maxima to SST maxima. 
In contrast ANV \qc\ (hence LWCF) peaks with maximum 500~mb convective
intensity, which is not necessarily collocated with SST maxima. 
Thus decoupling SST from \qc\ couples LWCF more tightly to other
factors determining convective intensity, e.g., atmospheric
instability, evaporation, and surface level wind \cite[]{FDR90,Hac94}.
Weaker mass extinction of ice further weakens the feedback between
convection and LWCF in ANV.

Significant reductions in LWCF occur in the extratropics.
Reduced LWCF over extratropical ocean is a climatological
improvement both seasons, though too strong in northern winter.
The verisimilitude of LWCF poleward of 60~$\dgr$N is difficult to
assess due to unreliable and missing ERBE data.
ANV may have improved LWCF over Asia in January (this is supported by
surface temperature, discussed below) and over July in Greenland.

Figure~\ref{fig:xavg_8589_LWCF} shows zonal average LWCF for January
and July. 
\begin{figure}
\begin{center}
\includegraphics*[width=\halfwidth]{/data/zender/fgr/phd/xavg_8589_01_LWCF.eps}\vfill
\includegraphics*[width=\halfwidth]{/data/zender/fgr/phd/xavg_8589_07_LWCF.eps}\vfill
\end{center}
\caption[Zonal average longwave cloud forcing LWCF from ERBE,
CCM, and ANV for January and July]{
Zonal average longwave cloud forcing LWCF (\wxmS) from ERBE
(solid), CCM (dotted), and ANV (dashed) for (a) January and (b) July.
\label{fig:xavg_8589_LWCF}}   
\end{figure}
The weakened equatorial maxima in ANV in both seasons stems from the
Indo-Pacific, discussed further in Section~\ref{sec:reg}.
In the tropics, where ANV has more high cloud than CCM
(Figure~\ref{fig:xavg_8589_CP}), weaker absorption by large ice
crystals (\ref{eqn:emissivity}) causes weaker LWCF. 
The prognostic stratiform ice significantly improves the northern
hemisphere July simulation due to reduced \qc\
(Figure~\ref{fig:xavg_8589_QC}f) and weaker mass extinction efficiency
of the enhanced ice fraction. 

The zonal symmetry of the southern ocean is conducive to demonstrating
the sensitivity of LWCF to ice fraction using zonal average quantities
(although the arguments also apply regionally and in the tropics). 
LWCF over the circumpolar ocean (40--70~$\dgr$S) in July is weaker
in ANV (by $\sim 3$~\wxmS), which has up to 100\% more condensate
above 600~mb than CCM (Figure~\ref{fig:xavg_8589_QC}d).
This ice abundance is nearly balanced by reduced liquid so $\Delta \CWP
\approx 4$~\gxmS\ over the region, i.e., ANV sequesters a larger
fraction of a fixed CWP in the upper troposphere as ice. 
%(We speculate the reduced low-level meridional temperature gradient in
%SHW (see below) reduces baroclinic activity and associated condensate
%formation beneath 550~mb while \qc\ above 550~mb increases in ANV due
%to temperature sensitivity in its stratiform ice generation).
For condensate of the same phase and size this vertical redistribution
would increase LWCF.
However, ice is more transmissive to upwelling terrestrial radiation
per unit mass than liquid due to intrinsic optical properties
(\ref{eqn:kappai1}) coupled to larger \rdsffc.
Figure~\ref{fig:xavg_8589_CP} shows the liquid to ice mass
ratio reverses from 2:1 in CCM to 1:2 in ANV.
Thus the greater transmittance of ice relative to liquid actually
causes increased high level cloud to reduce LWCF.
A combination of this mechanism and reduced upper tropospheric \qc\
causes stronger LWCF reduction over the southern ocean in January
and over NH landmasses in July.

\subsubsection{Seasonal Cycle}\label{sec:sc}

% Perhaps cite the Cess et al. 92 study (JGR v. 97, p. 7613) on ERBE
% seasonal cycle.
% Where does Roeckner get his ERBE data? it looks nothing like ours!
% In LoR95 he shows ERBE data ostensibly from 8906--8912
% In ChR96 he shows a distinctly different SWCF seasonal cycle graph
% than we get with our data here.

We illustrate the seasonal sensitivity of cloud radiative forcing to
anvil representation by plotting the difference between the zonal
average of the ensemble monthly mean and the 5-year mean.
Note the seasonal cycle contains nearly half the power of the tropical
SST boundary condition and all the power of top of atmosphere
insolation.
Figure~\ref{fig:anom_xavg_8589_0112_SWCF} shows the seasonal cycle of
SWCF. 
\begin{figure}
\begin{center}
\setlength{\widthdim}{\halfwidth}\ifphdcsz{\setlength{\widthdim}{.9\widthdim}}{}
\includegraphics*[width=\widthdim]{/data/zender/fgr/phd/erbe_b_anom_xavg_8589_0112_SWCF.eps}\vfill
\includegraphics*[width=\widthdim]{/data/zender/fgr/phd/amip5_anom_xavg_8589_0112_SWCF.eps}\vfill
\includegraphics*[width=\widthdim]{/data/zender/fgr/phd/spcp_85_anom_xavg_8589_0112_SWCF.eps}\vfill
\end{center}
\caption[Seasonal amplitude of zonal average shortwave cloud forcing
for ERBE, CCM, and ANV]{
Seasonal amplitude of zonal average shortwave cloud 
forcing (\wxmS) for (a) ERBE, (b) CCM, and (c) ANV. 
Shading indicates ensemble monthly value is less than the 5~year
mean. 
Contour interval is 10~\wxmS.
Tickmarks represent mid-month values (i.e., N represents
November~15). 
\label{fig:anom_xavg_8589_0112_SWCF}}   
\end{figure}
The tropical quadrupole pattern marks the biannual crossing of the
equator by the ITCZ. 
The seasonal cycle in ANV is slightly weaker in general, but the
models are hard to distinguish. 
Both skillfully predict the north-south and winter-summer
asymmetries. 

The seasonal cycle of LWCF, shown in
Figure~\ref{fig:anom_xavg_8589_0112_LWCF}, reveals significant model
differences.
\begin{figure}
\begin{center}
\setlength{\widthdim}{\halfwidth}\ifphdcsz{\setlength{\widthdim}{.9\widthdim}}{}
\includegraphics*[width=\widthdim]{/data/zender/fgr/phd/erbe_b_anom_xavg_8589_0112_LWCF.eps}\vfill
\includegraphics*[width=\widthdim]{/data/zender/fgr/phd/amip5_anom_xavg_8589_0112_LWCF.eps}\vfill
\includegraphics*[width=\widthdim]{/data/zender/fgr/phd/spcp_85_anom_xavg_8589_0112_LWCF.eps}\vfill
\end{center}
\caption[Same as Figure~\ref{fig:anom_xavg_8589_0112_SWCF} but for 
longwave cloud forcing]{
Same as Figure~\ref{fig:anom_xavg_8589_0112_SWCF} but for longwave
cloud forcing. 
Contour interval is 4~\wxmS.
\label{fig:anom_xavg_8589_0112_LWCF}}   
\end{figure}
ANV exacerbates the weak bias in tropical LWCF at both solstices.
Unrealistically persistent cloud over desert, mentioned above, weakens
and worsens the seasonal amplitude in ANV by reducing inter-solstitial
LWCF change. 
This land influence makes Figure~\ref{fig:anom_xavg_8589_0112_LWCF}
a poor estimator of the linear response of LWCF to the SST boundary
condition, discussed in Section~\ref{sec:sst}.
Northern hemisphere tropical extrema in both models are determined
more by strong, diffuse forcing (or lack thereof) from the
Indo-Pacific than by ITCZ variation, as observed.

\subsection{Dynamic and Thermodynamic Fields}\label{sec:dyn}

As described above, ANV reduced diabatic heating in the summer
hemisphere tropics (Figure~\ref{fig:xavg_8589_QDIABAT}), especially
over zonally cohesive SST maxima (Figure~\ref{fig:8589_LWCF}).
Figure~\ref{fig:pres_xavg_8589_01_MPSI} shows the January meridional
stream function \mpsi.  
\begin{figure}
\begin{center}
\includegraphics*[width=\halfwidth]{/data/zender/fgr/phd/ecmwf_pres_xavg_9095_01_MPSI.eps}\vfill
\includegraphics*[width=\halfwidth]{/data/zender/fgr/phd/amip5_pres_xavg_8589_01_MPSI.eps}\vfill
\includegraphics*[width=\halfwidth]{/data/zender/fgr/phd/spcp_85_pres_xavg_8589_01_MPSI.eps}\vfill
\end{center}
\caption[January meridional stream function \mpsi\ for ECMWF, CCM, and
ANV]{ 
January meridional stream function \mpsi\ (\kgxs)
for (a) ECMWF, (b) CCM, and (c) ANV.
Features below $\sim 800$~mb may be artifacts of orography.
Contour interval is $2 \times 10^{10}$~\kgxs.  
\label{fig:pres_xavg_8589_01_MPSI}}
\end{figure}
Weaker summer diabatic heating in ANV reduced Hadley cell strength by
13\% in January, and 7\% in July (not shown).
A weaker Hadley cell accompanied weaker latent and radiative heating
in the tropical mid-troposphere in previous studies
\cite[e.g.,][]{RHD89,SRB94}.
Associated with reduced \mpsi\ is reduced transport of angular
momentum and heat to the winter hemisphere extratropics
\cite[e.g.,][]{LiH88}. 
Reduced subsidence is conducive to enhanced convection in the
wintertime tropics.

The January 200~mb velocity potential $\chi$ in
Figure~\ref{fig:pres_8589_01_CHI} shows the geographic variation of
large scale divergent motion associated with the altered diabatic
heating.  
\begin{figure}
\begin{center}
\includegraphics*[width=\halfwidth]{/data/zender/fgr/phd/ecmwf_pres_9095_01_CHI.eps}\vfill
\includegraphics*[width=\halfwidth]{/data/zender/fgr/phd/amip5_pres_8589_01_CHI.eps}\vfill
\includegraphics*[width=\halfwidth]{/data/zender/fgr/phd/spcp_85_pres_8589_01_CHI.eps}\vfill
\end{center}
\caption[January 200~mb velocity potential $\chi$ for
ECMWF, CCM, and ANV]{
January 200~mb velocity potential $\chi$ (\mSxs) for (a)  
ECMWF, (b) CCM, and (c) ANV.
Contour interval is $3 \times 10^6$~\mSxs. 
Shading indicates divergence ($\chi < 0$).
\label{fig:pres_8589_01_CHI}}
\end{figure}
As mentioned in regards to LWCF, ANV eliminates the divergence center
over the Indian Ocean warm pool in January.
The local energetics of this circulation change are detailed in
Section~\ref{sec:reg} below. 
Examination of $\Delta \chi$ (not shown) reveals compensating
subsidence is reduced over the North African and Arabian deserts,
worsening LWCF there (Figure~\ref{fig:8589_LWCF}). 
Divergence associated with the Australian monsoon circulation shifts
northeastward from the Pacific warm pool.
This improves the extent of divergence over the north Pacific.
Changes in July $\chi$ (not shown) are less significant.

%ANV improves the location of the descending branch of the Pacific
%Walker cell, shifting it 20~$\dgr$W of CCM on the equator. 
%ANV worsens (shirts east) the intensity west Pacific center of
%divergence. 
%The models situate the center of TWP divergence $\sim 20\ \dgr$E
%of the analyzed position, and overpredict its magnitude.
%Too-strong subsidence over Africa occurs in January also (not
%shown) and contributes to the suppression of high cloud over
%central Africa (Figure~\ref{fig:8589_LWCF}) in both models
%(Figure~\ref{fig:8589_LWCF}). 
%NB: Interestingly, the January Walker cell, as deduced from the 200~mb
%divergence field, shows no lessening in extent or strength, although
%the reduction in LWCF, and Hadley intensity, is much stronger during
%that season than July. Why?

Figure~\ref{fig:pres_xavg_8589_T} shows the CCM zonal average
temperature $T$ bias against ECMWF analyses together with $\Delta T$
for January and July.    
\begin{figure*}
\begin{center}
\includegraphics*[width=.5\hsize]{/data/zender/fgr/phd/amip5_8589_ecmwf_9095_pres_xavg_01_T.eps}%
\includegraphics*[width=.5\hsize]{/data/zender/fgr/phd/amip5_8589_ecmwf_8994_pres_xavg_07_T.eps}%

\includegraphics*[width=.5\hsize]{/data/zender/fgr/phd/spcp_85_8589_amip5_8589_xavg_01_T.eps}%
\includegraphics*[width=.5\hsize]{/data/zender/fgr/phd/spcp_85_8589_amip5_8589_xavg_07_T.eps}%
\end{center}
\caption[CCM zonal average temperature $T$ bias against ECMWF analyses
and $T$ difference (ANV$-$CCM) for 1985--1989 January and July]{
CCM zonal average temperature $T$ bias (\dgr K) against 1989--1995
ECMWF analyses and $T$ difference (ANV$-$CCM) (left) January and
(right) July: (a) January CCM$-$ECMWF, (b) July CCM$-$ECMWF, (c)
January ANV$-$CCM, and (d) July ANV$-$CCM.
Contour interval is (a,b)~1 and (c,d) .5~\dgr K. 
Shading indicates where (a,b) $\CCM < \ECMWF$ (i.e., CCM is colder
than ECMWF) and (c,d) $\ANV < \CCM$.
\label{fig:pres_xavg_8589_T}}   
\end{figure*}
A systematic tropospheric cold bias of 0--2~\dgr K is apparent in
both models, with larger biases of 6--12~\dgr K at the summer
polar tropopause. 
ANV warms the 50~mb beneath the tropical tropopause by
2--3~$\dgr$K, roughly 5 times the standard deviation of 
zonal average monthly $T$ from a 10~yr AMIP CCM2 simulation.
The increase in tropical upper tropospheric temperature includes
anvil-induced increase in radiative equilibrium $T$
(Figure~\ref{fig:xavg_8589_QDIABAT}) and decreased heat export by the
Hadley cell (Figure~\ref{fig:pres_xavg_8589_01_MPSI}). 
The near meridional symmetry of the $T$ increase reflects the
inability of the upper troposphere to maintain large temperature
gradients \cite[]{LiH88}.
There is no significant change in tropical atmospheric stability
beneath 200~mb. 

ANV is 1--9~$\dgr$K warmer than CCM near the Antarctic surface in
July, a significant change even in this highly variable region. 
This is due to radiative weakening of the inversion because of less
cloud during polar night, as discussed above.
The resulting reduction in the low-level equator to pole temperature
gradient reduces the energy available for baroclinic activity.

Changes in extratropical stratiform cloud distribution and cloud
optical properties can significantly alter land surface temperature
\Ts\ \cite[e.g.,][]{Kie942}.
The most significant \Ts\ change in January occurred in central and
north Asia, where ANV reduced CWP by 10--50~\gxmS\ 
(Figure~\ref{fig:xavg_8589_CP}).
Reduced cloud cover enhanced LW cooling here by 5--15~\wxmS.
Wintertime insolation was too weak to compensate so \Ts\ declined
by $\gtrsim 5\ \dgr$C over large regions, compared to model
gridpoint variability of $\lesssim 4\ \dgr$C.
% NB: Case 422 TS1s over US, Siberia in January is < 4 K
In July, increased low and mid-level CWP over the US interior reduced
surface solar absorption by 10--35~\wxmS. 
Reduced surface absorption decreased \Ts\ 3--4~$\dgr$C, compared to
model variability of $\lesssim 1.5\ \dgr$C.  
% NB: Case 422 TS1s over US in July is 0--1.5 K
Both \Ts\ decreases improve model climatological biases relative to
\cite{LeW90} (not shown).

Figure~\ref{fig:pres_xavg_8589_U} shows ECMWF analyzed and model
simulated zonal average zonal wind $U$.
\begin{figure*}
\begin{center}
\includegraphics*[width=.5\hsize]{/data/zender/fgr/phd/ecmwf_pres_xavg_9095_01_U.eps}%
\includegraphics*[width=.5\hsize]{/data/zender/fgr/phd/ecmwf_pres_xavg_8994_07_U.eps}%

\includegraphics*[width=.5\hsize]{/data/zender/fgr/phd/amip5_pres_xavg_8589_01_U.eps}%
\includegraphics*[width=.5\hsize]{/data/zender/fgr/phd/amip5_pres_xavg_8589_07_U.eps}%

\includegraphics*[width=.5\hsize]{/data/zender/fgr/phd/spcp_85_pres_xavg_8589_01_U.eps}%
\includegraphics*[width=.5\hsize]{/data/zender/fgr/phd/spcp_85_pres_xavg_8589_07_U.eps}%
\end{center}
\caption[Zonal average zonal wind $U$ for January and July ECMWF, CCM,
and ANV]{ 
Zonal average zonal wind $U$ (\mxs) for (left) January and
(right) July (a,b) 1989--1994 ECMWF, (c,d) 1985--1989 CCM, and (e,f)
1985--1989 ANV.  
Contour interval is 5~\mxs.
Shading indicates easterly zonal winds ($U < 0$).
\label{fig:pres_xavg_8589_U}}
\end{figure*}
Model differences are $< 6$~\mxs\ and model standard deviation is  
$< 3$~\mxs.
Most biases in CCM appear in ANV, notably a 2--12~\mxs\ westerly bias
in the tropics above 800~mb.
ANV worsens (accelerates) the 100-200~mb westerly bias by 2--6~\mxs\
from 10~$\dgr$S--10~$\dgr$N in both seasons. 
This enhanced westerly bias cannot be explained by thermal adjustment 
(Figure~\ref{fig:pres_xavg_8589_T}) but is consistent with two changes
in the Hadley circulation.
First, a weaker Hadley cell reduces export of eastward momentum. 
Second, the equatorward shift of tropical heating
(Figure~\ref{fig:xavg_8589_QDIABAT}) decreases the asymmetry of the
meridional circulation about the equator, to which the strength of
upper level tropical easterlies is strongly sensitive \cite[]{LiH88}.
Extratropical $U$ changes fall within the range of model variability.

%Weakened upper level westerlies and retreat of the SHS jet have been 
%correlated to reduced LWCF (Figure~\ref{fig:xavg_8589_LWCF}) in
%previous studies \cite[e.g.,][]{SlS88}. 

%Examination of the geographic distribution of upper level winds (not
%shown) reveals the following:
%SHS jet intensity weakens due to favorable broadening of the jet over
%the southwest Atlantic. 
%NHW jet simulation appears unchanged but results from 
%improvements in upper level winds over Asia compensating for
%unrealistic split flow over the central Pacific. 
%NHS jet location is improved, due to improved simulation of the
%North American and East Asian jets.

Strong diabatic heating sources in the tropics can excite
extratropical planetary waves \cite[e.g.,][]{HoK81}.  
In January ANV formed two distinct heating sources in the Micronesian
region associated with a northward shift of Australian Monsoon
precipitation. 
First, enhanced precipitation of up to 7.5~\mmxday\ from
145--170~$\dgr$E to 3~$\dgr$S--7~$\dgr$N increased
mid-tropospheric heating rates by up to 2.7~\kxday.
This precipitation reduces a dry bias relative to the climatology of 
\cite{LeW90} (not shown).
Second, enhanced anvil cloud from 165--195~$\dgr$E to
5--15~$\dgr$N increased 250--100~mb diabatic heating by up to
1.5~\kxday.
The latter heating was entirely due to radiative absorption by
increased upper level anvil, not condensation. 

% NB: 1 mm/day = 29 W/m2
The response of the extratropical circulation to the altered
heating appears in Figure~\ref{fig:pres_8589_Z2TEST}, which depicts
the 500~mb height field $\Phi$.  
\begin{figure*}
\begin{center}
\includegraphics*[width=.33\hsize]{/data/zender/fgr/phd/ecmwf_pres_9095_01_Z2TEST.eps}%
\includegraphics*[width=.33\hsize]{/data/zender/fgr/phd/amip5_pres_8589_01_Z2TEST.eps}%
\includegraphics*[width=.33\hsize]{/data/zender/fgr/phd/spcp_85_pres_8589_01_Z2TEST.eps}%
\end{center}
\caption[January 500~mb geopotential height field $\Phi$ for
30--90~$\dgr$N from ECMWF 1990--1995 analyses and model 
simulations of 1985--1989 by CCM and ANV]{
January 500~mb geopotential height field $\Phi$ (gpm) for
30--90~$\dgr$N from (a) ECMWF 1990--1995 analyses and model 
simulations of 1985--1989 by (b) CCM and (c) ANV.
Contour interval is 10~gpm. 
\label{fig:pres_8589_Z2TEST}}   
\end{figure*}
% NB: 10~yr. ECMWF observed standard deviation of January $\Phi$ over the PNA is ???~m.
% NB: 20~yr. CCM2 388 modeled standard deviation of January $\Phi$ over the PNA is 40--80~m.
% NB: HBK94 p. 20793 Figure~6a,b is standard deviation of the daily
% differences from the monthly mean.
ANV significantly alters stationary wave patterns from the central
Pacific to west Europe.
ANV deepens the central Pacific trough and shifts it $\sim 10 
\dgr$~E.
The associated ridge splits flow around California but reproduces 
observed ridging over the west coast of Canada, absent in CCM.
ANV strengthens the ridge over west Europe, as observed, and shifts
the central European trough $\sim 20 \dgr$~E towards analyses. 
Model differences are 1--3 times model standard deviation in the
vicinity of these ridges. 
A similar North American response, also linked to a northward shift of
Australian monsoon precipitation, occurred in \cite{Kie942}.

\section{Cloud Response to SST Forcing in the Equatorial
Pacific}\label{sec:sst}

The Equatorial Pacific SST anomaly associated with the 1987 El Ni\~no
provides a stringent test of model ability to mimic observed
changes in convective patterns and associated anvil cloud.
ANV physics, which explicitly represent tropical anvil lifecycle,  
differ significantly from the diagnostic cloud physics in CCM.
Thus differences in cloud forcing response to SST anomalies will
reveal differences in model coupling of SST to convection.

During the 1987 El Ni\~no the center of deep convection, accompanying
a large positive SST anomaly, shifted from the west to the central
equatorial Pacific. 
Cloud forcing responded by increasing in the central and east
equatorial Pacific through much of 1987, while cooler SST reduced
cloud forcing in the west.
\cite{HaM93} and \cite{Cho94} emphasize cloud enhancement from
10~$\dgr$S--10~$\dgr$N was largely compensated by clearer sky
from 10--30$\dgr$ in both hemispheres.
We focus on the equatorial Pacific region from $10\
\dgr$S--$10\ \dgr$N and use the strong SST signal there to
test the deep convective response of the models. 
Note atmospheric response could not feed back to SST (a prescribed
boundary condition).  

Figure~\ref{fig:anom_yavg_10S10N_8589_0160_SWCF} shows the 
evolution of the SWCF anomaly (from the 5-year mean) of this region in
Hovm\"oller (longitude-time) format.
\begin{figure}
\begin{center}
\ifphdcsz{\setlength{\heightdim}{.9\textheight}}{\setlength{\heightdim}{\textheight}}
\includegraphics*[width=\halfwidth,height=.31\heightdim]{/data/zender/fgr/phd/erbe_b_anom_yavg_10S10N_8589_0160_SWCF.eps}\vfill
\includegraphics*[width=\halfwidth,height=.28\heightdim]{/data/zender/fgr/phd/amip5_anom_yavg_10S10N_8589_0160_SWCF.eps}\vfill
\includegraphics*[width=\halfwidth,height=.31\heightdim]{/data/zender/fgr/phd/spcp_85_anom_yavg_10S10N_8589_0160_SWCF.eps}\vfill
\end{center}
\caption[Hovm\"oller diagrams of shortwave cloud forcing 
anomaly in the Equatorial Pacific for ERBE, CCM, and ANV]{
Hovm\"oller diagrams of shortwave cloud forcing anomaly (\wxmS) in the
Equatorial Pacific (averaged 10\dgr S--10\dgr N) for (a)
ERBE, (b) CCM, and (c) ANV.  
Month 1 is January 1985.  
Contour interval is 10~\wxmS. 
\label{fig:anom_yavg_10S10N_8589_0160_SWCF}}
\end{figure}
The El Ni\~no SST anomaly appears as the SWCF minimum which passes
from West to East Pacific in late 1986.
Both models capture the morphology of the 1987 changes reasonably
well, but exaggerate zonal structure. 
ANV produces more anomalies $> 20$~\wxmS\ than CCM, except near
90~$\dgr$W, where ANV variability is too weak.
The minima near 170~$\dgr$W in January 1987 is too weak in CCM but
too far east in ANV.
In January 1988 the minima is too weak in CCM, too strong in ANV.

Figure~\ref{fig:anom_yavg_10S10N_8589_0160_LWCF} shows the
corresponding LWCF anomaly.
\begin{figure}
\begin{center}
\ifphdcsz{\setlength{\heightdim}{.9\textheight}}{\setlength{\heightdim}{\textheight}}
\includegraphics*[width=\halfwidth,height=.31\heightdim]{/data/zender/fgr/phd/erbe_b_anom_yavg_10S10N_8589_0160_LWCF.eps}\vfill
\includegraphics*[width=\halfwidth,height=.28\heightdim]{/data/zender/fgr/phd/amip5_anom_yavg_10S10N_8589_0160_LWCF.eps}\vfill
\includegraphics*[width=\halfwidth,height=.31\heightdim]{/data/zender/fgr/phd/spcp_85_anom_yavg_10S10N_8589_0160_LWCF.eps}\vfill
\end{center}
\caption{Same as Figure~\ref{fig:anom_yavg_10S10N_8589_0160_SWCF} but
for longwave cloud forcing.
\label{fig:anom_yavg_10S10N_8589_0160_LWCF}}
\end{figure}
As with SWCF, ANV produces larger LWCF anomalies than CCM---this is
most apparent in the west and central Pacific. 
The 40--50~\wxmS\ anomaly in late 1986 is again too weak in CCM but
too far east in ANV. 
In summary, CCM and ANV manifest distinct but realistic convection
changes in response to El Ni\~no: 
ANV produces cloud forcing anomalies closer in maximum strength to
ERBE than CCM, but positioned too far east. 

In the spirit of \cite{RaC91} and \cite{Cho94}, we examine the
response of cloud forcing and \qc\ to SST forcing by subtracting the
equatorial Pacific climate of Spring (average of March, April, and
May) 1985 from Spring 1987. 
They chose Spring because equatorial SST peaks in April (when the
seasonal cycle peaks), and proximity to the equinox maximizes the
hemispheric symmetry of solar forcing. 
The Spring SST of the entire equatorial Pacific (10~\dgr
S--10~\dgr N, 140~\dgr E--90~\dgr W) warmed .9~\dgr K
from 1985 to 1987, while the central equatorial Pacific alone
(10~\dgr S--10~\dgr N, 180--130~\dgr W) warmed
1.2~\dgr K. 
Differencing the cold year (1985) Spring from the warm (1987) removes
the mean cloud forcing state and isolates the cloud forcing
sensitivity (which implicitly includes any reorganization of
convection patterns) to SST change.
Mean ANV and CCM cloud forcing in this region are approximately equal 
(Figures~\ref{fig:8589_SWCF} and~\ref{fig:8589_LWCF}).

Figure~\ref{fig:reg_Pacific_Equatorial_87m85_0305_LWCF_SWCF} plots 
$\delta$LWCF vs.\ $\delta$SWCF for the equatorial Pacific
($\delta$ refers to $1987 - 1985$ temporal change for a given model,
not to inter-model change). 
\begin{figure*}
\begin{center}
\includegraphics*[width=.33\hsize]{/data/zender/fgr/phd/erbe_b_reg_Pacific_Equatorial_87m85_0305_LWCF_SWCF.eps}%
\includegraphics*[width=.33\hsize]{/data/zender/fgr/phd/amip5_reg_Pacific_Equatorial_87m85_0305_LWCF_SWCF.eps}%
\includegraphics*[width=.33\hsize]{/data/zender/fgr/phd/spcp_85_reg_Pacific_Equatorial_87m85_0305_LWCF_SWCF.eps}%
\end{center}
\caption[$1987-1985$ differences in Spring quarter (March, April, and
May) mean maritime LWCF and SWCF over the Equatorial Pacific for ERBE,
CCM, and ANV]{ 
$1987-1985$ differences in Spring quarter (March, April, and May) 
mean maritime LWCF and SWCF (\wxmS) over the Equatorial Pacific 
(10~\dgr S--10~\dgr N, 140~\dgr E--90~\dgr W)
for (a) ERBE, (b) CCM, and (c) ANV.
Solid line is least-squares fit.
\label{fig:reg_Pacific_Equatorial_87m85_0305_LWCF_SWCF}}   
\end{figure*}
Crosses represent GCM gridpoints ($\sim 300^2$~\kmS).
Gridpoints with $\delta \LWCF < -20$~\wxmS\ are more copious in ANV,
but extend unrealistically beyond $\delta \LWCF < -50$~\wxmS.
The slope $m \equiv \delta \SWCF / \delta \LWCF$ approximately
measures the reduction in surface insolation relative to the increase
in atmospheric heating. 
ERBE shows local net cloud forcing response to El Ni\~no is a linear
($|r| = .94$), moderately negative feedback ($|m| \approx 1.2$), i.e.,
the albedo effect of anvil responds more strongly to local SST
anomalies than the greenhouse effect. 
\cite{RaC91}, in their Table~1, showed $|m| > 1$ over the equatorial
Pacific independent of which non-El Ni\~no season or year is used for
$\delta$. 
Moreover, the same qualitative behavior is ubiquitous over equatorial
Pacific subregions (i.e., east, central, west) (not shown).
Thus predicting $m$ is a necessary condition for coupled
ocean-atmosphere GCMs used for climate change research. 

ANV and CCM predict $m \approx -.95$, i.e., cloud forcing is a weak
positive local feedback to column energy for equatorial Pacific SST
change, rather than a moderate negative feedback, as observed.
This model agreement is surprising because \cite{RaC91} argue
$|m| > 1$ due to radiative properties of tropical anvil, which is
diagnosed in CCM, but prognosed in ANV.
Analyzing the dependence of cloud forcing on SST provides insight
here.
Figure~\ref{fig:reg_Pacific_Equatorial_87m85_0305_TS1_LWCF} shows the
trend of $\delta$LWCF with $\delta$SST is itself linear.
\begin{figure*}
\begin{center}
\includegraphics*[width=.33\hsize]{/data/zender/fgr/phd/erbe_b_reg_Pacific_Equatorial_87m85_0305_TS1_LWCF.eps}%
\includegraphics*[width=.33\hsize]{/data/zender/fgr/phd/amip5_reg_Pacific_Equatorial_87m85_0305_TS1_LWCF.eps}%
\includegraphics*[width=.33\hsize]{/data/zender/fgr/phd/spcp_85_reg_Pacific_Equatorial_87m85_0305_TS1_LWCF.eps}%
\end{center}
\caption[Same as Figure~\ref{fig:reg_Pacific_Equatorial_87m85_0305_LWCF_SWCF}
but for SST and LWCF]{
Same as Figure~\ref{fig:reg_Pacific_Equatorial_87m85_0305_LWCF_SWCF}
but for SST (\dgr K) and LWCF.
\label{fig:reg_Pacific_Equatorial_87m85_0305_TS1_LWCF}}   
\end{figure*}
The identical abscissa, AMIP $\delta$SST, appears in all three panels.
Modeled LWCF sensitivities to $\delta$SST are most distinct for
$\delta \SST < 0$ and $> 2\ \dgr$C, where ANV response is too
strong and too weak, respectively.
Despite model differences in anvil representation, the trends of
$\delta$LWCF with $\delta$SST closely agree with each other and
observations: ERBE, CCM, and ANV slopes are 17.0, 16.0, and
15.7~\wxmSk, respectively.   
Corresponding trends of $\delta$SWCF with $\delta$SST (not shown)
are $-20.1$, $-12.8$, and $-15.5$~\wxmSk, with correlations  $-.59$,
$-.43$, and $-.44$, respectively. 

If we assume nonlinear coupling of cloud forcing sensitivity to the
ocean-atmosphere system may be neglected, the conclusion is the models
underpredict $|m|$ largely due to underpredicting $\delta$SWCF rather
than overpredicting $\delta$LWCF. 
Since observed cloud forcing response is linear with $\delta$SST, the
major uncertainty in this conclusion is whether the mean cloud forcing 
state in the modeled equatorial Pacific (Figures
\ref{fig:8589_SWCF}--\ref{fig:8589_LWCF}) is sufficiently realistic. 
%A new version of CCM, CCM3 \cite[]{KBB96}, improves mean cloud forcing but
%not cloud forcing response to $\delta$SST.

Why is modeled $\delta$SWCF too weak with modeled $\delta$LWCF so
close to ERBE?   
Recall that upper tropospheric \rdsffc\ is time-invariant
(Section~\ref{sec:rdsffc}) so $\delta \rdsffc \approx 0$.
Thus $\delta$CWP largely determines $\delta$SWCF (\ref{eqn:tau}).
We examine $\delta \qc$ to see if CWP response to $\delta$SST is too
weak.
Figure~\ref{fig:yavg_10S10N_87m85_MAM_QC} shows modeled $\delta \qc$
for 1987$-$1985 over the equatorial Pacific averaged for Spring. 
\begin{figure}
\begin{center}
\includegraphics*[width=\halfwidth]{/data/zender/fgr/phd/amip5_yavg_10S10N_87m85_MAM_QC.eps}\vfill
\includegraphics*[width=\halfwidth]{/data/zender/fgr/phd/spcp_85_yavg_10S10N_87m85_MAM_QC.eps}\vfill
\end{center}
\caption[Longitude-height profile of the $1987-1985$ difference in
Spring quarter (March, April, and May) mean condensate \qc\ over the
equatorial Pacific simulated by CCM and ANV]{
Longitude-height profile of the $1987-1985$ difference in Spring
quarter (March, April, and May) mean condensate \qc\ (\mgxkg) over the
equatorial Pacific (averaged 10~\dgr S--10~\dgr N, 
ocean only) simulated by (a) CCM and (b) ANV.
Contour interval is 2~\mgxkg.
Shading indicates \qc\ decrease from 1985 to 1987.
% NB: IDL gets the shading of the first -'ve contour wrong when land
% is masked so i included land points in this file, no significant
% difference from maritime only above 400 mb.
\label{fig:yavg_10S10N_87m85_MAM_QC}}   
\end{figure}
High cloud increases in both models over the central equatorial
Pacific (where $\delta$SST peaks), and decreases over the western
equatorial Pacific.
ANV predicts $\delta \qc$ extrema at the same longitude as CCM
(145~$\dgr$E, 175~$\dgr$W), but roughly 100~mb higher,
and 2--4 times stronger.
Yearly LWCF (not shown) suggests \qc\ should increase from
140--110~$\dgr$W, as ANV predicts.  
CWP response does not appear too weak in either model, but reliable
observational estimates do not exist to verify this.

In summary, the model coupling of $\delta$CWP to $\delta$SST is
distinct and strong in both models.
Thus both models predict the necessary combination of change in cloud
height and $\delta$CWP to obtain reasonable $\delta$LWCF.   
We speculate that ERBE data may show a significant sensitivity of
shortwave hydrometeor optical properties (\rdsffc\ or habit) to equatorial
Pacific SST change, and neglecting this sensitivity biases modeled
SWCF response.  
A physical mechanism to cause hydrometeor optical properties to
respond to $\delta$SST is not obvious.
One plausible mechanism is the presence of small ice crystals: 
Cloud forcing is strongly linked to $\delta$SST in the equatorial
region. 
The cloud forcing response is due to tropical anvil, whose cloud top
temperature presumably decreases with SST \cite[e.g.,][]{RaC91}.  
\cite{KKW93} observed large numbers of small ice crystals in the tops 
of tropical anvil, and proposed homogeneous nucleation as their
source. 
Small ice crystals enhance SWCF relative to LWCF in opaque cloud 
(i.e., $\epsilon \sim 1$) \cite[]{ZeK94}.
The coefficients employed in (\ref{eqn:tau}--\ref{eqn:kappai1})
neglect crystals smaller than 20~\um.
Neglecting small crystals could therefore account for some of the bias
in $\delta$SWCF. 
Methods of testing this are discussed in Section~\ref{sec:gcm_disc}.

% NB: CRM sensitivity studies in tropics at noon show:
% 1 mg/kg condensate difference at p = 150 mb, T=200, 
% rho = .25 kg/m3 is .25 g/m2 for a 1 km thick cloud. 
% Deconvolving the spatial and temporal averaging to obtain what the
% radiation code feels instaneously and you might get 2.5 g/m2 more in
% CCM than ANV.
% .25 g/m2 changes LWCF by 4 W/m2 by itself, or by .5 W/m2 when on top
% of a 70 g/m2 cloud between 550--200 mb.
% 2.5 g/m2 changes LWCF by 30 W/m2 by itself, or by 5 W/m2 when on top
% of a 70 g/m2 cloud between 550--200 mb.
% Vapor is ~ 10x less efficient at reducing OLR in clear sky than
% condensate at the tropopause (per unit mixing ratio). 

\section{Central Indian Ocean}\label{sec:reg}

The central Indian Ocean (15~$\dgr$S--5~$\dgr$N,
60--80~$\dgr$E)  
stands out as a region where the explicit parameterization of
tropical anvil significantly altered local cloud forcing and
dynamics (Figures~\ref{fig:8589_SWCF}, \ref{fig:8589_LWCF}, and 
\ref{fig:pres_8589_01_CHI}). 
In this section we examine the seasonal cycle of the central Indian
Ocean to identify processes which maintain energy balance in the
changed regional climate.

The simulated seasonal cycle of cloud in the central Indian Ocean is
shown in Figure~\ref{fig:reg_xyavg_Indian_Central_0112_QC}.
\begin{figure}
\begin{center}
\includegraphics*[width=\halfwidth]{/data/zender/fgr/phd/amip5_xyavg_reg_Indian_Central_8589_0112_QC.eps}\vfill
\includegraphics*[width=\halfwidth]{/data/zender/fgr/phd/spcp_85_xyavg_reg_Indian_Central_8589_0112_QC.eps}\vfill
\end{center}
\caption[Seasonal cycle of simulated condensate \qc\ over the central
Indian Ocean for 1985--1989]{
Seasonal cycle simulated condensate \qc\ (\mgxkg) over the central
Indian Ocean (15~\dgr S--5~\dgr N, 60--80~\dgr E) for
1985--1989 (a) CCM and (b) ANV.  
Contour interval is 2~\mgxkg.
Shading indicates $\qc > 6$~\mgxkg.
\label{fig:reg_xyavg_Indian_Central_0112_QC}}   
\end{figure}
The figure shows $\Delta \qc > 0$ from 550--200~mb for all seasons.
The factor of 2--3 increase in anvil ice tends to improve LWCF but not
SWCF (shown below).
$\Delta \qc$ is a minimum in January, when CCM is relatively cloudy
and ANV is relatively clear.
The coolest maximum SST in the central Indian Ocean occurs in January,
although the minimum mean SST occurs in August.
Reduced deep convection in ANV in January
(cf. Figure~\ref{fig:pres_8589_01_CHI}) is probably a consequence of
weaker specified coupling of \qc\ to SST (\ref{eqn:qc}--\ref{eqn:hl})
discussed above.  

Figure~\ref{fig:xyavg_reg_8589_0112_ocean_CF} shows the seasonal cycle
of central Indian Ocean cloud forcing.
\begin{figure}
\begin{center}
\includegraphics*[width=\halfwidth]{/data/zender/fgr/phd/xyavg_reg_Indian_Central_8589_0112_SWCF.eps}\vfill
\includegraphics*[width=\halfwidth]{/data/zender/fgr/phd/xyavg_reg_Indian_Central_8589_0112_LWCF.eps}\vfill
\end{center}
\caption[Seasonal amplitude in regional shortwave and longwave
cloud forcing over central Indian Ocean for ERBE, CCM, and ANV]{ 
Seasonal amplitude in regional (a) shortwave and (b) longwave
cloud forcing (\wxmS ) over the central Indian Ocean
(15~\dgr S--5~\dgr N,60--80~\dgr E) for ERBE (solid), CCM
(dotted), and ANV (dashed).
\label{fig:xyavg_reg_8589_0112_ocean_CF}}
\end{figure}
The models predict a seasonal cycle peaking in January, of amplitude
15--25~\wxmS\ stronger than observed.
The bias in total cloud (SWCF) is present from October--May, while the
bias in high cloud (LWCF) peaks from May--October.
% NB: XXX check ISCCP high cloud
ANV weakens the amplitude of the cloud forcing cycles, with the
greatest difference from CCM occuring in January.
ANV improves high cloud from December--April, but this has little
effect on the SWCF bias.
As a result, ANV worsens net monthly cloud forcing $\NCF \equiv \LWCF
+ \SWCF$. 
ERBE has $0 < \NCF < -10$~\wxmS\ in the region
(this small magnitude is typical of the tropics \cite[]{Kie941}). 
CCM and ANV have NCF $\approx -20$ and $-30$~\wxmS, respectively. 

Figure~\ref{fig:xyavg_reg_Indian_Central_8589} shows
the vertical profile of simulated diabatic heating components over 
the central Indian Ocean for January conditions.
\begin{figure*}
\begin{center}
\includegraphics*[width=.5\hsize]{/data/zender/fgr/phd/amip5_xyavg_reg_Indian_Central_8589_01.eps}%
\includegraphics*[width=.5\hsize]{/data/zender/fgr/phd/spcp_85_8589_amip5_8589_xyavg_reg_Indian_Central_01.eps}%
\end{center}
\caption[Model simulated profiles of diabatic heating
and differences between models (ANV$-$CCM) over the central 
Indian Ocean for 1985--1989 January]{
Model simulated profiles of (a) diabatic heating
(\kxday), and (b) differences between models (ANV$-$CCM) for the
central Indian Ocean (15~\dgr S--5~\dgr N, 60--80~\dgr E) 
for 1985--1989 January.
Heatings shown are total diabatic (solid), shortwave (dotted),
longwave (short dash), resolved (dash-dot), turbulent
(dash-dot-dot-dot), and convective (long dash).
Note difference in scales.
\label{fig:xyavg_reg_Indian_Central_8589}}
\end{figure*}
The heating profile is typical of deep convective regions in the
models.
Convective heating dominates radiative from the boundary layer to
250~mb. 
Large scale heating in the upper troposphere, representing stratiform
condensation in anvil, enhances latent heating but the stratiform
precipitation evaporatively cools the lower troposphere.
SW heating is 30--60\% of LW cooling from 800--200~mb, and dominates
\QT\ from 150--100~mb \cite[]{RaR891}. 

Differences between ANV and CCM profiles range from 10--50\% of mean
heating rates.
Weaker vertical motion and upper level divergence
(Figure~\ref{fig:pres_8589_01_CHI}) compensate the large reduction
in convective heating and precipitation (3~\mmxday).
The profile of LW heating change is always negative due to significant
reduction of cloud mass in January.
This is atypical of tropical summer, when the response to prognostic
anvil is normally enhanced cloud base heating and cloud top cooling
(cf. Figure~\ref{fig:xavg_8589_QDIABAT}).
Reduced column vapor and condensate absorptivity increase LW cooling 
by $\sim 30\%$ from 800--400~mb and enhance anvil-base heating near
300~mb. 
Differences in diabatic heating components are $< .2$~\kxday\ in July,
when much of the central Indian Ocean is colder than 28~$\dgr$C, 
and modeled cloud forcing agrees
(Figure~\ref{fig:xyavg_reg_8589_0112_ocean_CF}).  

%Figure~\ref{fig:xyavg_reg_Indian_Central_8589_0112_SEB} shows surface
%energy budget components from the central Indian Ocean. 
%\begin{figure}
%\begin{center}
%\includegraphics*[width=\halfwidth,height=.135\textheight]{/data/zender/fgr/phd/xyavg_reg_Indian_Central_8589_0112_FSNS.eps}\vfill
%\includegraphics*[width=\halfwidth,height=.12\textheight]{/data/zender/fgr/phd/xyavg_reg_Indian_Central_8589_0112_FLNS.eps}\vfill
%\includegraphics*[width=\halfwidth,height=.12\textheight]{/data/zender/fgr/phd/xyavg_reg_Indian_Central_8589_0112_SHFLX.eps}\vfill
%\includegraphics*[width=\halfwidth,height=.12\textheight]{/data/zender/fgr/phd/xyavg_reg_Indian_Central_8589_0112_LHFLX.eps}\vfill
%\includegraphics*[width=\halfwidth,height=.12\textheight]{/data/zender/fgr/phd/xyavg_reg_Indian_Central_8589_0112_E_P.eps}\vfill
%\includegraphics*[width=\halfwidth,height=.135\textheight]{/data/zender/fgr/phd
%/xyavg_reg_Indian_Central_8589_0112_NET.eps}\vfill
%\end{center}
%\caption[Simulated seasonal cycle of central Indian Ocean 
%surface energy budget components from CCM and ANV]{ 
%Simulated seasonal cycle of central Indian Ocean 
%surface energy budget components (a) net shortwave
%flux, (b) net longwave flux, (c) sensible heat flux, (d) latent heat
%flux, (e) evaporation$-$precipitation, and (f) net surface energy from
%CCM (solid), and ANV (dashed).  
%Positive fluxes heat the surface.
%Units are \wxmS\ except E$-$P, which is \mmxday.
%\label{fig:xyavg_reg_Indian_Central_8589_0112_SEB}}   
%\end{figure}
%Inter-model differences in surface flux components (panels a--e) tend
%to compensate so that net surface energy (panel f) changes $< 20$~\wxmS. 
%Interestingly, the greatest inter-model differences in the seasonal
%cycle of surface fluxes occur in January, when net surface flux does
%not change at all.
%The region is more reflective in ANV during southern hemisphere
%winter, though always more condensate-laden
%(Figures~\ref{fig:reg_xyavg_Indian_Central_0112_QC}--\ref{fig:xyavg_reg_8589_0112_ocean_CF}).
%This reduces surface insolation May--December.
%Longwave cooling modestly increases January--April, owing to weaker
%summertime convection. 
%Variations in other surface fluxes are not as significant.
%In southern summer, cooler, drier boundary layer air increases latent
%and sensible fluxes in ANV. 
%Monthly differences in surface wind speed are $ < 1$~\mxs.

In summary, the central Indian Ocean exemplifies three characteristics
of the Indo-Pacific response to prognostic anvil.
First, anvil condensate is consistently greater.
Second, enhanced anvil mass roughly balances weaker mass
extinction efficiency---this prevents increased bias in SWCF and LWCF 
individually, but not necessarily their sum.
Third, convection is generally weaker in summer and stronger in
winter.  

\section{Discussion and Conclusions}\label{sec:gcm_disc}

Radiative forcing from cirrus anvil plays a dominant role in
determining the diabatic heating which drives the general circulation.
We have investigated the sensitivity of the simulated climate to
the radiative effects of a representation of ice cloud prognosed from 
modeled and observed characteristics of tropical anvil. 
In particular, we replaced a representation of tropical cloud which
diagnoses cloud mass from column vapor with a prognostic
representation which forecasts anvil generation from the vertical
profile of convective mass flux and anvil precipitation from mesoscale
budget estimates.
The direct effect of this prognostic anvil representation is to
sequester more condensate in the upper troposphere, a larger fraction
of which is ice.
These changes in condensate distribution and ice fraction agree with
recent observations \cite[]{WSS931,GrM96} and simulations in cumulus
ensemble models \cite[]{SLT94,GMK95}. 
We stress the control model includes diagnostic ice cloud---the
sensitivity experiment shows the impact of forecasting ice cloud based
on the physics of tropical anvil.

Cloud radiative forcing could have changed by three mechanisms in this
experiment (modulo total condensate amount): intrinsic differences in
ice vs.\ liquid optical properties (for a given effective radius
hydrometeor), coupling of hydrometeor size to temperature,
pressure, and phase, and changes in the vertical distribution of
condensate. 
The results showed all three mechanisms were important in preserving
a realistic climate simulation: 
The enhanced ice amount and fraction in the prognostic anvil were
approximately balanced by weaker mass extinction only because
anvil vertical location was tied to larger hydrometeor size.
The southern ocean in July and the central Indian Ocean in January 
were used to demonstrate this balance is important globally.
In these regions LWCF decreased (and improved) although upper
tropospheric cloud actually increased.
The strongest bias of the ANV prognostic anvil scheme is too-strong
cloud forcing over wintertime desert, due to weak sublimation in
subsidence regimes.
This causes significant weakening of the annual cycle of zonal average
LWCF.

Enhanced anvil perturbs the tropical upper troposphere temperature
structure more strongly in winter, when the column is clearer and
anvil radiatively heats the troposphere above 200~mb. 
In the summer tropics, enhanced anvil occurs in a ``deep cloud''
environment, reducing radiative cooling up to 200~mb, and enhancing
cooling above that. 
Reduced optical depth keeps the intrinsically greater solar absorption
of ice relative to liquid from causing a ubiquitous heating increase
above 600~mb.
Radiative heating contributes to warming the region just below the
tropical tropopause 2--3~$\dgr$K.   

The prognostic anvil is less strongly coupled to SST.
Based on the 1987 El Ni\~no, the prognostic anvil formulation improves
longwave cloud radiative response to SST cooling but worsens response
to warming $> 2\ \dgr$C. 
In conjunction with weaker mass extinction of ice, this weakens the
feedback between longwave cloud forcing and convection over SST maxima.
The net response of convection is a shift toward the winter hemisphere
in solstice months.
This eliminates a persistent convective bias in January over the
central Indian Ocean.
Moreover, increased convection and high cloud north of the equator
propagate Rossby waves to the extratropics.  
This causes significant ridging in the 500~mb height field over the
west coasts of North America and Europe, substantially improving
agreement with analysis. 

%Large changes in non-anvil, stratiform cloud cover and optical
%properties occurred in the extratropics, notably northern hemisphere
%landmass.
%In January, reduced cloud enhanced surface longwave cooling
%by 5--15~\wxmS in central and north Asia; this cooled surface
%temperature by $\gtrsim 5\ \dgr$C.
%In July, increased cloud reduced reduced surface solar absorption by 
%$10--35$~\wxmS in the US interior, cooling the surface
%3--4~$\dgr$C. 

%The ratio of ice to liquid condensate from 600--300~mb and the 
%difference between ice and liquid hydrometeor size are two
%microphysical quantities which are prescribed by the models
%(Sections~\ref{sec:fice} and~\ref{sec:rdsffc}) but for which great
%uncertainty exists.
%Had the former not reversed (Figure~\ref{fig:xavg_8589_CP}), or 
%the latter been smaller, enhanced anvils could have unrealistically 
%increased optical depth, emissivity, and cloud forcing.

The underprediction of shortwave cloud forcing sensitivity to SST
indicates anvil ice crystal size or habit may be sensitive to SST.
As mentioned previously, the parameterizations which link \taui\ and
\kappai\ to \rei\ (\ref{eqn:tau}--\ref{eqn:kappai1}) neglect the
presence of small ice crystals, whose abundance is also uncertain. 
\cite{EbC92} based the coefficients of
(\ref{eqn:tau}--\ref{eqn:kappai1}) on ice crystal distributions of
hexagonal columns with major axis length $L > 20$~\um, the limit of
particle detectors at that time. 
%We extended optical properties and size distributions from that study
%to $L = 10$~\um\ to account for the presence of smaller ice crystals.
%We find \kappai\ increases to
%\begin{equation}
%\label{eqn:kappai2}
%\kappai = .005 + (\rei - 5)^{-1}.
%\end{equation}
%For $\rei = 30$~\um, parameterizations (\ref{eqn:kappai1}) and
%(\ref{eqn:kappai2}) yield $\kappai = .038$ and .045~\mSxg,
%respectively, a 15\% increase.
Accounting for crystals $L < 20$~\um\ increases both LWCF and SWCF for
a given \qi\ \cite[]{ZeK94}. 
\cite{SuS95} developed parameterizations analogous to
\cite{EbC92} but which account for small crystals, and include
strong temperature dependence for cold ($T < -40\ \dgr$C) ice.
This temperature dependence, which they base on observed temperature
trends of small crystal number and crystal habit, strongly increases
the mass extinction efficiency of cold cirrus.
Ice crystal optical property parameterizations suitable for GCMs are 
also needed for more complex crystal habits than solid hexagonal
columns \cite[e.g.,][]{TaL95} as these habits may dominate in anvil.  
GCMs should test if such parameterizations rectify underprediction of
SWCF response relative to LWCF response demonstrated in El Ni\~no
experiments in Section~\ref{sec:sst}.  
Moreover, observations which simultaneously constrain IWP, \rei, and
radiative fluxes are required to test these parameterizations.
Programs such as ARM are currently assembling such data.
%NB: Which EOS platforms are doing likewise?

In summary, the prognostic representation of ice cloud based on anvil
behavior produced a radiative climatology in overall agreement with
the control model (and observations) despite a large increase in
anvil condensate.
Climate features sensitive to anvil representation include 
tropical upper troposphere temperature structure, cloud forcing
response to SST change, Hadley cell strength, warm pool convection,
and the North American flow field in winter.
Obviously, these results are based on a particular anvil
representation in a single GCM and must be viewed as a sensitivity 
study. 
More physically based tropical anvil for future studies should
include diagnostic (or prognostic) effective hydrometeor radius based
on observed crystal size distributions.
Such studies, which will further constrain the mechanisms determining
cloud radiative response to SST change, are underway at NCAR.

% Balance preprint columns
\ifphdcsz{}{\balance}

% Appendices
\ifphdcsz{}{\appendix\section{Implementation of ANV}\label{app:anv_imp}\ifphdcsz{\thispagestyle{myheadings}}{}
The ANV prognostic ice scheme \cite[][to which the following
equation numbers refer]{ZeK961} was implemented in CCM in the
following manner:
Closing the heat and moisture budgets self-consistently (without
having to forecast liquid) required disabling the direct thermodynamic
feedbacks (2--3).
Instead, heat and moisture tendencies of ice cloud formation are
accounted for by CCM stratiform and convective cloud physics.
This ensures condensational heating is not double-counted.
Convective ice generation is forecast with~(1).
Stratiform ice is forecast with an earlier version of (6), namely 
\begin{eqnarray}
\label{eqn:qi_stb_old}
{D\qi \over Dt} & 
= & 
A_s \Bigl(
{\alpha w + \beta {\mathrm H}(T - T_0) \over \rho \Delta z} 
\Bigr) -
c_2 \qi - 
c_3 \qi
\end{eqnarray}
where ${\mathrm H}$ is the Heavyside step function and $\alpha$,
$\beta$, and $T_0$ are empirical constants set to $6.3 \times
10^{-5}$~\kgxmC, $4 \times 10^{-7}$~\kgxmSsk, and 240~$\dgr$K,  
respectively. 
%Table~\ref{tbl:anv_parm_imp} lists the parameters as suggested in 
%\cite{ZeK961} Table~1 and as implemented in the present study.
%\begin{planotable}{cll}
%% See CZPIII p. 89,90,136
%\tablewidth{\hsize} % works for normal tables
%\tablecaption{Parameters in ANV\label{tbl:anv_parm_imp}}
%\tablehead{\colhead{} & \colhead{Suggested Value} & \colhead{Implemented}}
%\startdata
%$c_1$ & $2.4 \times 10^{-3}$ & $.75 \times 10^{-3}$ \nl
%$c_2$ & $.25c_3$ & $.3c_3$~\xs\ \nl
%$c_3$ & $2.8 \times 10^{-4}$ & $1.85 \times 10^{-4}$~\xs\ \nl
%$c_4$ & $1 \times 10^{-5}$ & $1 \times 10^{-5}$ m$^{-2}$~kg$^{1/2}$~s$^{-1/2}$ \nl
%$c_5$ & $7.6 \times 10^{-6}$ & $6.3 \times 10^{-5}$ \kgxmC \nl
%$c_6$ & $1.75 \times 10^{-5}$ & $4 \times 10^{-7}$ \kgxmSsk \nl
%\end{planotable}
%\begin{table*}
%% See CZPIII p. 89,90,136
%\caption{Parameters implemented in ANV\label{tbl:anv_parm_imp}} 
%\begin{center}
%\vspace{5pt}
%\begin{tabular}{lll}
%\tableline
%& & \\[-5pt]
%& Suggested Value & Implemented \\[4pt]
%\tableline
%& & \\[-6pt]
%$c_1$ & $2.4 \times 10^{-3}$ & $.75 \times 10^{-3}$ \\
%$c_2$ & $.25c_3$ & $.3c_3$~\xs\ \\
%$c_3$ & $2.8 \times 10^{-4}$ & $1.85 \times 10^{-4}$~\xs\ \\
%$c_4$ & $1 \times 10^{-5}$ & $1 \times 10^{-5}$ m$^{-2}$~kg$^{1/2}$~s$^{-1/2}$ \\
%%$c_5$ & $7.6 \times 10^{-6}$ & $6.3 \times 10^{-5}$ \kgxmC \\
%%$c_6$ & $1.75 \times 10^{-5}$ & $4 \times 10^{-7}$ \kgxmSsk \\[4pt]
%\tableline
%& & \\[-8pt]
%\end{tabular}
%\end{center}
%\end{table*}
To produce a more realistic model climate, parameters $c_1$--$c_3$ 
were ``tuned'' from the values suggested in \cite{ZeK961} Table~1.
The values of $c_1$--$c_4$ implemented in the present study are: $.75
\times 10^{-3}$, $.3c_3$~\xs, $1.85 \times 10^{-4}$~\xs, and 
$1 \times 10^{-5}$~m$^{-2}$~kg$^{1/2}$~s$^{-1/2}$, respectively. 

CCM physics \cite[]{HBB93} are used to diagnose liquid condensate and
stratiform cloud fraction. 
We use a potential relative humidity $\RHp \equiv (\qv + \qi)/\qvi$
rather than $\RH \equiv \qv/\qvi$ to compute cloud fraction.
We adjusted the extinction optical depth correction for randomly
overlapped clouds from $\tau' = \tau A^{3/2}$ to $\tau' = \tau
A^{1/2}$ and used gridbox IWP (rather than in-cloud IWP) to compute
$\taui$. 
For longwave radiation we transformed prognostic gridbox IWP to
in-cloud IWP using diagnosed cloud fraction $A$ (we ignored $A < 1\%$)
before computing $\epsilon$; then used $A\epsilon$ as the effective
cloud fraction. 


}

%\section{Symbols}\label{app:sym}
%\iftwocol{\begin{tabular}{l p{.7\hsize}}}{\begin{tabular}{r l}}
%$A_s$ & stratiform cloud fraction \\
%\end{tabular}

%\section{Acronyms}\label{app:abb}
%\iftwocol{\begin{tabular}{l p{.6\hsize}}}{\begin{tabular}{r l}}
%WPWP & West Pacific Warm Pool \\
%\end{tabular}

% Acknowledgements
\ifphdcsz{\subsection{Acknowledgements}\label{sec:gcm_ack}}{\acknowledgments}
CSZ gratefully acknowledges discussions with B.~Briegleb,
R.~Saravanan, and D.-Z.~Sun.
G.~Branstator provided insight interpreting
Figure~\ref{fig:pres_8589_Z2TEST}.  
P.~Rasch provided helpful guidance with cloud parameterization. 
This work was supported in part by NASA Earth Observing System project
W-17,661 and by DOE Atmospheric Radiation Measurements Program grant
DE-AI05-92ER61376.

% Bibliography
\ifphdcsz{}{\bibliographystyle{jas}}
\ifphdcsz{}{\bibliography{/home/zender/tex/bib}}

\ifphdcsz{}{\end{document}}


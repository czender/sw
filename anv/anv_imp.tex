\ifphdcsz{\thispagestyle{myheadings}}{}
The ANV prognostic ice scheme \cite[][to which the following
equation numbers refer]{ZeK961} was implemented in CCM in the
following manner:
Closing the heat and moisture budgets self-consistently (without
having to forecast liquid) required disabling the direct thermodynamic
feedbacks (2--3).
Instead, heat and moisture tendencies of ice cloud formation are
accounted for by CCM stratiform and convective cloud physics.
This ensures condensational heating is not double-counted.
Convective ice generation is forecast with~(1).
Stratiform ice is forecast with an earlier version of (6), namely 
\begin{eqnarray}
\label{eqn:qi_stb_old}
{D\qi \over Dt} & 
= & 
A_s \Bigl(
{\alpha w + \beta {\mathrm H}(T - T_0) \over \rho \Delta z} 
\Bigr) -
c_2 \qi - 
c_3 \qi
\end{eqnarray}
where ${\mathrm H}$ is the Heavyside step function and $\alpha$,
$\beta$, and $T_0$ are empirical constants set to $6.3 \times
10^{-5}$~\kgxmC, $4 \times 10^{-7}$~\kgxmSsk, and 240~$\dgr$K,  
respectively. 
%Table~\ref{tbl:anv_parm_imp} lists the parameters as suggested in 
%\cite{ZeK961} Table~1 and as implemented in the present study.
%\begin{planotable}{cll}
%% See CZPIII p. 89,90,136
%\tablewidth{\hsize} % works for normal tables
%\tablecaption{Parameters in ANV\label{tbl:anv_parm_imp}}
%\tablehead{\colhead{} & \colhead{Suggested Value} & \colhead{Implemented}}
%\startdata
%$c_1$ & $2.4 \times 10^{-3}$ & $.75 \times 10^{-3}$ \nl
%$c_2$ & $.25c_3$ & $.3c_3$~\xs\ \nl
%$c_3$ & $2.8 \times 10^{-4}$ & $1.85 \times 10^{-4}$~\xs\ \nl
%$c_4$ & $1 \times 10^{-5}$ & $1 \times 10^{-5}$ m$^{-2}$~kg$^{1/2}$~s$^{-1/2}$ \nl
%$c_5$ & $7.6 \times 10^{-6}$ & $6.3 \times 10^{-5}$ \kgxmC \nl
%$c_6$ & $1.75 \times 10^{-5}$ & $4 \times 10^{-7}$ \kgxmSsk \nl
%\end{planotable}
%\begin{table*}
%% See CZPIII p. 89,90,136
%\caption{Parameters implemented in ANV\label{tbl:anv_parm_imp}} 
%\begin{center}
%\vspace{5pt}
%\begin{tabular}{lll}
%\tableline
%& & \\[-5pt]
%& Suggested Value & Implemented \\[4pt]
%\tableline
%& & \\[-6pt]
%$c_1$ & $2.4 \times 10^{-3}$ & $.75 \times 10^{-3}$ \\
%$c_2$ & $.25c_3$ & $.3c_3$~\xs\ \\
%$c_3$ & $2.8 \times 10^{-4}$ & $1.85 \times 10^{-4}$~\xs\ \\
%$c_4$ & $1 \times 10^{-5}$ & $1 \times 10^{-5}$ m$^{-2}$~kg$^{1/2}$~s$^{-1/2}$ \\
%%$c_5$ & $7.6 \times 10^{-6}$ & $6.3 \times 10^{-5}$ \kgxmC \\
%%$c_6$ & $1.75 \times 10^{-5}$ & $4 \times 10^{-7}$ \kgxmSsk \\[4pt]
%\tableline
%& & \\[-8pt]
%\end{tabular}
%\end{center}
%\end{table*}
To produce a more realistic model climate, parameters $c_1$--$c_3$ 
were ``tuned'' from the values suggested in \cite{ZeK961} Table~1.
The values of $c_1$--$c_4$ implemented in the present study are: $.75
\times 10^{-3}$, $.3c_3$~\xs, $1.85 \times 10^{-4}$~\xs, and 
$1 \times 10^{-5}$~m$^{-2}$~kg$^{1/2}$~s$^{-1/2}$, respectively. 

CCM physics \cite[]{HBB93} are used to diagnose liquid condensate and
stratiform cloud fraction. 
We use a potential relative humidity $\RHp \equiv (\qv + \qi)/\qvi$
rather than $\RH \equiv \qv/\qvi$ to compute cloud fraction.
We adjusted the extinction optical depth correction for randomly
overlapped clouds from $\tau' = \tau A^{3/2}$ to $\tau' = \tau
A^{1/2}$ and used gridbox IWP (rather than in-cloud IWP) to compute
$\taui$. 
For longwave radiation we transformed prognostic gridbox IWP to
in-cloud IWP using diagnosed cloud fraction $A$ (we ignored $A < 1\%$)
before computing $\epsilon$; then used $A\epsilon$ as the effective
cloud fraction. 



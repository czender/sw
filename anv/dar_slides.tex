%Define general page, paragraph, and line formatting here.
\input /cgd/home/zender/tex/eplain
\input /cgd/home/zender/tex/utilities
\magnification=\magstep3
\nopagenumbers
\parskip=12pt
\parindent=1em
\tolerance=10000
\overfullrule=0pt
\raggedright
%%%%%%%%%%%%%% Insert scratch TeX commands below here %%%%%%%%%%%%%

%%%%%%%%%%%%%%%%%%%%%%%%%%%%%%%%%%%%%%%%%%%%%%%%%%%%%%%%%%%%%%%%%%%%%%%%
% Begin page
%%%%%%%%%%%%%%%%%%%%%%%%%%%%%%%%%%%%%%%%%%%%%%%%%%%%%%%%%%%%%%%%%%%%%%%%
\centerline{\bf Parameterization of Anvil Clouds in the NCAR GCM}
\centerline{\bf Charles S. Zender and J.T. Kiehl}
\centerline{\bf University of Colorado and NCAR}
\medskip

\item{1.} Motivation
\item{2.} Parameterization development
\item{3.} Climate sensitivity
\item{4.} Cloud sensitivity to El Nino
\item{5.} Conclusions

\vfill\eject
%%%%%%%%%%%%%%%%%%%%%%%%%%%%%%%%%%%%%%%%%%%%%%%%%%%%%%%%%%%%%%%%%%%%%%%%
% End page
%%%%%%%%%%%%%%%%%%%%%%%%%%%%%%%%%%%%%%%%%%%%%%%%%%%%%%%%%%%%%%%%%%%%%%%%

%%%%%%%%%%%%%%%%%%%%%%%%%%%%%%%%%%%%%%%%%%%%%%%%%%%%%%%%%%%%%%%%%%%%%%%%
% Begin page
%%%%%%%%%%%%%%%%%%%%%%%%%%%%%%%%%%%%%%%%%%%%%%%%%%%%%%%%%%%%%%%%%%%%%%%%
\centerline{\bf Parameterization Properties}
\medskip

\item{1.} Shows correct radiative feedbacks
\item{2.} Non-prescribed IWP (allow deviations from Adiabatic Liquid Water Content)
\item{3.} Captures IWC profile of anvils (-> ice -> SWCF > LWCF)
\item{4.} Sensitive to local convective properties
\item{5.} Prognostic

\vfill\eject
%%%%%%%%%%%%%%%%%%%%%%%%%%%%%%%%%%%%%%%%%%%%%%%%%%%%%%%%%%%%%%%%%%%%%%%%
% End page
%%%%%%%%%%%%%%%%%%%%%%%%%%%%%%%%%%%%%%%%%%%%%%%%%%%%%%%%%%%%%%%%%%%%%%%%

%%%%%%%%%%%%%%%%%%%%%%%%%%%%%%%%%%%%%%%%%%%%%%%%%%%%%%%%%%%%%%%%%%%%%%%%
% Begin page
%%%%%%%%%%%%%%%%%%%%%%%%%%%%%%%%%%%%%%%%%%%%%%%%%%%%%%%%%%%%%%%%%%%%%%%%
\centerline{\bf Current CCM Cloud Properties}
\medskip

\item{1.} Prescribed IWP as a function of $\int Q$ (Adiabatic Liquid Water Content)
\item{2.} Does not quite saturate the LWCF
\item{3.} Does not know about anvils
\item{4.} Sensitive to total column moisture (e.g., SST)
\item{5.} Prescribed

\vfill\eject
%%%%%%%%%%%%%%%%%%%%%%%%%%%%%%%%%%%%%%%%%%%%%%%%%%%%%%%%%%%%%%%%%%%%%%%%
% End page
%%%%%%%%%%%%%%%%%%%%%%%%%%%%%%%%%%%%%%%%%%%%%%%%%%%%%%%%%%%%%%%%%%%%%%%%

%%%%%%%%%%%%%%%%%%%%%%%%%%%%%%%%%%%%%%%%%%%%%%%%%%%%%%%%%%%%%%%%%%%%%%%%
% Begin page
%%%%%%%%%%%%%%%%%%%%%%%%%%%%%%%%%%%%%%%%%%%%%%%%%%%%%%%%%%%%%%%%%%%%%%%%
\centerline{\bf Motivation }
\medskip

Tropical cirrus are especially important because among all cloud types
they exist in the coldest part of the troposphere directly over the
warmest oceans and thus produce the strongest longwave forcing.

CCM2 captures mean climatological radiative fields quite well, but a
more stringent test is the model's ability to respond to an external
forcing. 

The 1987 El Nino and contemporaneous ERBE data are such a test.
Observed radiative sensitivity to SST forcing can test whether a 
GCM cloud parameterization predicts the correct anomalies in both the
LW (cloud top temperature, emissivity) and the SW (optical depth,
particle size and phase).

Analyses show that $\Delta SWCF / \Delta LWCF > 1.1$ in the tropics
during El Nino. However the CCM predicts the ratio is $\approx$ 1. 

A prescribed {\it IWP\/} is unable to fully adjust cloud mass to
convective events. A prognosed {\it IWP\/} should take into account
the most important bulk predictors of anvil ice.

\vfill\eject
%%%%%%%%%%%%%%%%%%%%%%%%%%%%%%%%%%%%%%%%%%%%%%%%%%%%%%%%%%%%%%%%%%%%%%%%
% End page
%%%%%%%%%%%%%%%%%%%%%%%%%%%%%%%%%%%%%%%%%%%%%%%%%%%%%%%%%%%%%%%%%%%%%%%%

%%%%%%%%%%%%%%%%%%%%%%%%%%%%%%%%%%%%%%%%%%%%%%%%%%%%%%%%%%%%%%%%%%%%%%%%
% Begin page
%%%%%%%%%%%%%%%%%%%%%%%%%%%%%%%%%%%%%%%%%%%%%%%%%%%%%%%%%%%%%%%%%%%%%%%%
\centerline{\bf SPCP = Semi-Prognostic Cirrus Parameterization = }
\medskip
For a grid cell of density $\rho$,
thickness $\Delta z$, ice mixing ratio $q_{\rm i}$, vapor mixing
ratio $q_{\rm v}$, temperature $T$, ice saturation specific humidity
$q_{\rm s}$, precipitation (snow) flux $P_{\rm i}$, and large scale wind field
$\vec u$, the thermodynamic couplings that define the SPCP are:

Convective:
$$
{\partial q_{\rm i} \over \partial t} = -\vec u \cdot \nabla q_{\rm i} + 
{\alpha M_{\rm c} \over \rho \Delta Z} - 
\beta q_{\rm i} - 
\gamma q_{\rm i}
\eqdef{q_i_eqn}
$$

Stable:
$$
{\partial q_{\rm i} \over \partial t} = -\vec u \cdot \nabla q_{\rm i} + 
{\alpha_{\rm STB} w + \beta_{\rm STB} {\rm H}(T - 240) \over \rho
\Delta Z} -
\beta q_{\rm i} - 
\gamma q_{\rm i}
\eqdef{q_i_stb_eqn}
$$

$$
{\partial q_{\rm v} \over \partial t} = -\vec u \cdot \nabla q_{\rm v} - 
{\alpha M_{\rm c} \over \rho \Delta Z} + 
\beta q_{\rm i} + 
{\delta \over \rho} \Bigl(1 - {q_{\rm i} \over q_{\rm s}}\Bigr)P_{\rm i}^{1/2}
\eqdef{q_v_eqn}
$$
$$
{\partial T \over \partial t} = -\vec u \cdot \nabla T +
{\hbox{L}_{\rm i} \over {\rm c_p}}\Biggl[
{\alpha M_{\rm c} \over \rho \Delta Z} - 
\beta q_{\rm i} - 
{\delta \over \rho} \Bigl(1 - {q_{\rm i} \over q_{\rm s}}\Bigr)P_{\rm i}^{1/2}
\Biggr]
\eqdef{T_eqn}
$$
where $\Delta Z$ is the thickness of the convecting portion of the
column for which $T < 0~^\circ$C, L$_{\rm i}$ is the latent heat of
fusion, and c$_{\rm p}$ is the specific heat of air at constant pressure. 

\vfill\eject
%%%%%%%%%%%%%%%%%%%%%%%%%%%%%%%%%%%%%%%%%%%%%%%%%%%%%%%%%%%%%%%%%%%%%%%%
% End page
%%%%%%%%%%%%%%%%%%%%%%%%%%%%%%%%%%%%%%%%%%%%%%%%%%%%%%%%%%%%%%%%%%%%%%%%

%%%%%%%%%%%%%%%%%%%%%%%%%%%%%%%%%%%%%%%%%%%%%%%%%%%%%%%%%%%%%%%%%%%%%%%%
% Begin page
%%%%%%%%%%%%%%%%%%%%%%%%%%%%%%%%%%%%%%%%%%%%%%%%%%%%%%%%%%%%%%%%%%%%%%%%
\centerline{\bf $\Omega_{.5}$ (Control Model) = CCM2 + }
\medskip
\item{1.} Ice radiative properties
\item{2.} Continental/Maritime cloud drop size distinction
\item{3.} Locally diagnosed cloud water content
\item{4.} Background aerosol
\item{5.} Trace gases
\item{6.} LSM Land Surface Model
\item{7.} modifications to PBL and moist convection

\vfill\eject
%%%%%%%%%%%%%%%%%%%%%%%%%%%%%%%%%%%%%%%%%%%%%%%%%%%%%%%%%%%%%%%%%%%%%%%%
% End page
%%%%%%%%%%%%%%%%%%%%%%%%%%%%%%%%%%%%%%%%%%%%%%%%%%%%%%%%%%%%%%%%%%%%%%%%

%%%%%%%%%%%%%%%%%%%%%%%%%%%%%%%%%%%%%%%%%%%%%%%%%%%%%%%%%%%%%%%%%%%%%%%%
% Begin page
%%%%%%%%%%%%%%%%%%%%%%%%%%%%%%%%%%%%%%%%%%%%%%%%%%%%%%%%%%%%%%%%%%%%%%%%
\vsize=11.0truein
\centerline{\bf Conclusions} 
\medskip
\item{$\bullet$} Negative cloud forcing feedbacks in deep convective
regimes may be caused by the anomalous SWCF increase to a saturated 
LWCF basic state.

\item{$\bullet$} Convective mass flux $M_c$ can work as a proxy for
IWP generation...but using a cloud base $M_c$ ignores the integrated
strength of penetrative convection.

\item{$\bullet$} Cloud fraction may be implicitly parameterized, at
least in cirrus anvils, where patchiness is low. 

\item{$\bullet$} Dividing ice condensate generation into convective
and stable components works well. Further division into multiple
synoptic cloud regimes looks promising.

\item{$\bullet$} $w$, $T$, and $A_c$ suffice to grossly represent 
stable ice generation at the GCM scale.

\item{$\bullet$} Spatial distributions of {\it IWP\/}, and the
location and strength of {\it IWP\/} extrema, are needed to see whether
higher {\it IWP\/} in storm tracks than ITCZ is real.

\item{$\bullet$} Vertical distribution of $f_{\rm ice}(T)$ is 
needed to cleanly mesh ice with liquid.

\vfill\eject
%%%%%%%%%%%%%%%%%%%%%%%%%%%%%%%%%%%%%%%%%%%%%%%%%%%%%%%%%%%%%%%%%%%%%%%%
% End page
%%%%%%%%%%%%%%%%%%%%%%%%%%%%%%%%%%%%%%%%%%%%%%%%%%%%%%%%%%%%%%%%%%%%%%%%

%%%%%%%%%%%%%% Insert scratch TeX commands above here %%%%%%%%%%%%%
\bye

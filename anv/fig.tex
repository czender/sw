% $Id$

% Purpose: Illustrate model climatology

\documentclass[twocolumn,final]{article}
\usepackage{natbib} % \cite commands from aguplus
\usepackage[figuresright]{rotating} % allows sideways figures and tables
\usepackage{graphicx} % defines \includegraphics*
\usepackage{longtable} % multi-page tables, e.g., acronyms and symbols
\usepackage{ifthen} % Boolean and programming commands

\usepackage{/home/zender/tex/csz} % all my local definitions
% Usage: % Usage: % Usage: \input{jgr_abb} % AGU-sanctioned journal title abbreviations

\def\aapgb{{\it Amer. Assoc. Petroleum Geologists Bull.}}
\def\adg{{\it Adv. Geophys.}}
\def\ajs{{\it Amer. J. Sci.}}
\def\amb{{\it Ambio}}
\def\amgb{{\it Arch. Meteorol. Geophys. Bioclimatl.}}
\def\ang{{\it Ann. Glaciol.}}
\def\angeo{{\it Ann. Geophys.}}
\def\apo{{\it Appl. Opt.}}
\def\areps{{\it Ann. Rev. Earth Planet. Sci.}}
\def\asr{{\it Adv. Space Res.}}
\def\ate{{\it Atmos. Environ.}}
\def\atf{{\it Atmosfera}}
\def\atms{{\it ACM Trans. Math Software}}
\def\ato{{\it Atmos. Ocean}}
\def\atr{{\it Atmos. Res.}}
\def\gbc{{\it Global Biogeochem. Cycles}} % csz
\def\blm{{\it Boundary-Layer Meteorol.}} % csz 
\def\bpa{{\it Beitr. Phys. Atmosph.}}
\def\bams{{\it Bull. Am. Meteorol. Soc.}}
\def\clc{{\it Clim. Change}}
\def\cld{{\it Clim. Dyn.}}
\def\com{{\it Computing}}
\def\dao{{\it Dyn. Atmos. Oceans}}
\def\dsr{{\it Deep-Sea Res.}}
\def\esr{{\it Earth Sci. Revs.}}
\def\gec{{\it Geosci. Canada}}
\def\gei{{\it Geofis. Int.}}
\def\gej{{\it Geogr. J.}}
\def\gem{{\it Geophys. Monogr.}}
\def\geo{{\it Geology}}
\def\grl{{\it Geophys. Res. Lett.}}
\def\ieeec{{\it IEEE Computer}}
\def\ijna{{\it IMA J. Numer. Anal.}}
\def\ijnmf{{\it Int. J. Num. Meteorol. Fl.}}
\def\jac{{\it J. Atmos. Chem.}}
\def\jacm{{\it J. Assoc. Comput. Mach.}}
\def\jam{{\it J. Appl. Meteorol.}}
\def\jas{{\it J. Atmos. Sci.}}
\def\jatp{{\it J. Atmos. Terr. Phys.}}
\def\jcam{{\it J. Climate Appl. Meteorol.}}
\def\jchp{{\it J. Chem Phys.}}
\def\jcis{{\it J. Coll. I. Sci.}}
\def\jcl{{\it J. Clim.}}
\def\jcp{{\it J. Comput. Phys.}}
\def\jfm{{\it J. Fluid Mech.}}
\def\jgl{{\it J. Glaciol.}}
\def\jgr{{\it J. Geophys. Res.}}
\def\jgs{{\it J. Geol. Soc. London}}
\def\jme{{\it J. Meteorol.}}
\def\jmr{{\it J. Marine Res.}}
\def\jmsj{{\it J. Meteorol. Soc. Jpn.}}
\def\josa{{\it J. Opt. Soc. A}}
\def\jpo{{\it J. Phys. Oceanogr.}}
\def\jqsrt{{\it J. Quant. Spectrosc. Radiat. Transfer}}
\def\jpca{{\it J. Phys. Chem. A}}
\def\lnc{{\it Lett. Nuov. C}}
\def\mac{{\it Math. Comp.}}
\def\map{{\it Meteorol. Atmos. Physics.}}
\def\mem{{\it Meteorol. Mag.}}
\def\mnras{{\it Mon. Not. Roy. Astron. Soc.}} 
\def\mwr{{\it Mon. Weather Rev.}} 
\def\nat{{\it Nature}}
\def\pac{{\it Parallel Computing}}
\def\pag{{\it Pure Appl. Geophys.}}
\def\pal{{\it Paleoceanography}}
\def\pht{{\it Physics Today}}
\def\pieee{{\it Proc. IEEE}}
\def\pla{{\it Phys. Lett. A}}
\def\ppp{{\it Paleogeogr. Paleoclim. Paleoecol.}}
\def\pra{{\it Phys. Res. A}}
\def\prd{{\it Phys. Rev. D}}
\def\prl{{\it Phys. Rev. L}}
\def\pss{{\it Planet. Space Sci.}}
\def\ptrsl{{\it Phil. Trans. R. Soc. Lond.}}
\def\qjrms{{\it Q. J. R. Meteorol. Soc.}}
\def\qres{{\it Quat. Res.}}
\def\qsr{{\it Quatern. Sci. Rev.}}
\def\reg{{\it Rev. Geophys.}}
\def\rgsp{{\it Revs. Geophys. Space Phys.}}
\def\rpp{{\it Rep. Prog. Phys.}}
\def\sca{{\it Sci. Amer.}}
\def\sci{{\it Science}}
\def\sjna{{\it SIAM J. Numer. Anal.}}
\def\sjssc{{\it SIAM J. Sci. Stat. Comput.}}
\def\tac{{\it Theor. Appl. Climatl.}}
\def\tel{{\it Tellus}}
\def\wea{{\it Weather}}

%SIAM Review: (Society for Industrial and Applied Mathematics)
%       J. on Computing
%       J. on Control and Optimization
%       J. on Algebraic and Discrete Methods
%       J. on Numerical Analysis 
%       J. on Scientific and Statistical Computing



 % AGU-sanctioned journal title abbreviations

\def\aapgb{{\it Amer. Assoc. Petroleum Geologists Bull.}}
\def\adg{{\it Adv. Geophys.}}
\def\ajs{{\it Amer. J. Sci.}}
\def\amb{{\it Ambio}}
\def\amgb{{\it Arch. Meteorol. Geophys. Bioclimatl.}}
\def\ang{{\it Ann. Glaciol.}}
\def\angeo{{\it Ann. Geophys.}}
\def\apo{{\it Appl. Opt.}}
\def\areps{{\it Ann. Rev. Earth Planet. Sci.}}
\def\asr{{\it Adv. Space Res.}}
\def\ate{{\it Atmos. Environ.}}
\def\atf{{\it Atmosfera}}
\def\atms{{\it ACM Trans. Math Software}}
\def\ato{{\it Atmos. Ocean}}
\def\atr{{\it Atmos. Res.}}
\def\gbc{{\it Global Biogeochem. Cycles}} % csz
\def\blm{{\it Boundary-Layer Meteorol.}} % csz 
\def\bpa{{\it Beitr. Phys. Atmosph.}}
\def\bams{{\it Bull. Am. Meteorol. Soc.}}
\def\clc{{\it Clim. Change}}
\def\cld{{\it Clim. Dyn.}}
\def\com{{\it Computing}}
\def\dao{{\it Dyn. Atmos. Oceans}}
\def\dsr{{\it Deep-Sea Res.}}
\def\esr{{\it Earth Sci. Revs.}}
\def\gec{{\it Geosci. Canada}}
\def\gei{{\it Geofis. Int.}}
\def\gej{{\it Geogr. J.}}
\def\gem{{\it Geophys. Monogr.}}
\def\geo{{\it Geology}}
\def\grl{{\it Geophys. Res. Lett.}}
\def\ieeec{{\it IEEE Computer}}
\def\ijna{{\it IMA J. Numer. Anal.}}
\def\ijnmf{{\it Int. J. Num. Meteorol. Fl.}}
\def\jac{{\it J. Atmos. Chem.}}
\def\jacm{{\it J. Assoc. Comput. Mach.}}
\def\jam{{\it J. Appl. Meteorol.}}
\def\jas{{\it J. Atmos. Sci.}}
\def\jatp{{\it J. Atmos. Terr. Phys.}}
\def\jcam{{\it J. Climate Appl. Meteorol.}}
\def\jchp{{\it J. Chem Phys.}}
\def\jcis{{\it J. Coll. I. Sci.}}
\def\jcl{{\it J. Clim.}}
\def\jcp{{\it J. Comput. Phys.}}
\def\jfm{{\it J. Fluid Mech.}}
\def\jgl{{\it J. Glaciol.}}
\def\jgr{{\it J. Geophys. Res.}}
\def\jgs{{\it J. Geol. Soc. London}}
\def\jme{{\it J. Meteorol.}}
\def\jmr{{\it J. Marine Res.}}
\def\jmsj{{\it J. Meteorol. Soc. Jpn.}}
\def\josa{{\it J. Opt. Soc. A}}
\def\jpo{{\it J. Phys. Oceanogr.}}
\def\jqsrt{{\it J. Quant. Spectrosc. Radiat. Transfer}}
\def\jpca{{\it J. Phys. Chem. A}}
\def\lnc{{\it Lett. Nuov. C}}
\def\mac{{\it Math. Comp.}}
\def\map{{\it Meteorol. Atmos. Physics.}}
\def\mem{{\it Meteorol. Mag.}}
\def\mnras{{\it Mon. Not. Roy. Astron. Soc.}} 
\def\mwr{{\it Mon. Weather Rev.}} 
\def\nat{{\it Nature}}
\def\pac{{\it Parallel Computing}}
\def\pag{{\it Pure Appl. Geophys.}}
\def\pal{{\it Paleoceanography}}
\def\pht{{\it Physics Today}}
\def\pieee{{\it Proc. IEEE}}
\def\pla{{\it Phys. Lett. A}}
\def\ppp{{\it Paleogeogr. Paleoclim. Paleoecol.}}
\def\pra{{\it Phys. Res. A}}
\def\prd{{\it Phys. Rev. D}}
\def\prl{{\it Phys. Rev. L}}
\def\pss{{\it Planet. Space Sci.}}
\def\ptrsl{{\it Phil. Trans. R. Soc. Lond.}}
\def\qjrms{{\it Q. J. R. Meteorol. Soc.}}
\def\qres{{\it Quat. Res.}}
\def\qsr{{\it Quatern. Sci. Rev.}}
\def\reg{{\it Rev. Geophys.}}
\def\rgsp{{\it Revs. Geophys. Space Phys.}}
\def\rpp{{\it Rep. Prog. Phys.}}
\def\sca{{\it Sci. Amer.}}
\def\sci{{\it Science}}
\def\sjna{{\it SIAM J. Numer. Anal.}}
\def\sjssc{{\it SIAM J. Sci. Stat. Comput.}}
\def\tac{{\it Theor. Appl. Climatl.}}
\def\tel{{\it Tellus}}
\def\wea{{\it Weather}}

%SIAM Review: (Society for Industrial and Applied Mathematics)
%       J. on Computing
%       J. on Control and Optimization
%       J. on Algebraic and Discrete Methods
%       J. on Numerical Analysis 
%       J. on Scientific and Statistical Computing



 % AGU-sanctioned journal title abbreviations

\def\aapgb{{\it Amer. Assoc. Petroleum Geologists Bull.}}
\def\adg{{\it Adv. Geophys.}}
\def\ajs{{\it Amer. J. Sci.}}
\def\amb{{\it Ambio}}
\def\amgb{{\it Arch. Meteorol. Geophys. Bioclimatl.}}
\def\ang{{\it Ann. Glaciol.}}
\def\angeo{{\it Ann. Geophys.}}
\def\apo{{\it Appl. Opt.}}
\def\areps{{\it Ann. Rev. Earth Planet. Sci.}}
\def\asr{{\it Adv. Space Res.}}
\def\ate{{\it Atmos. Environ.}}
\def\atf{{\it Atmosfera}}
\def\atms{{\it ACM Trans. Math Software}}
\def\ato{{\it Atmos. Ocean}}
\def\atr{{\it Atmos. Res.}}
\def\gbc{{\it Global Biogeochem. Cycles}} % csz
\def\blm{{\it Boundary-Layer Meteorol.}} % csz 
\def\bpa{{\it Beitr. Phys. Atmosph.}}
\def\bams{{\it Bull. Am. Meteorol. Soc.}}
\def\clc{{\it Clim. Change}}
\def\cld{{\it Clim. Dyn.}}
\def\com{{\it Computing}}
\def\dao{{\it Dyn. Atmos. Oceans}}
\def\dsr{{\it Deep-Sea Res.}}
\def\esr{{\it Earth Sci. Revs.}}
\def\gec{{\it Geosci. Canada}}
\def\gei{{\it Geofis. Int.}}
\def\gej{{\it Geogr. J.}}
\def\gem{{\it Geophys. Monogr.}}
\def\geo{{\it Geology}}
\def\grl{{\it Geophys. Res. Lett.}}
\def\ieeec{{\it IEEE Computer}}
\def\ijna{{\it IMA J. Numer. Anal.}}
\def\ijnmf{{\it Int. J. Num. Meteorol. Fl.}}
\def\jac{{\it J. Atmos. Chem.}}
\def\jacm{{\it J. Assoc. Comput. Mach.}}
\def\jam{{\it J. Appl. Meteorol.}}
\def\jas{{\it J. Atmos. Sci.}}
\def\jatp{{\it J. Atmos. Terr. Phys.}}
\def\jcam{{\it J. Climate Appl. Meteorol.}}
\def\jchp{{\it J. Chem Phys.}}
\def\jcis{{\it J. Coll. I. Sci.}}
\def\jcl{{\it J. Clim.}}
\def\jcp{{\it J. Comput. Phys.}}
\def\jfm{{\it J. Fluid Mech.}}
\def\jgl{{\it J. Glaciol.}}
\def\jgr{{\it J. Geophys. Res.}}
\def\jgs{{\it J. Geol. Soc. London}}
\def\jme{{\it J. Meteorol.}}
\def\jmr{{\it J. Marine Res.}}
\def\jmsj{{\it J. Meteorol. Soc. Jpn.}}
\def\josa{{\it J. Opt. Soc. A}}
\def\jpo{{\it J. Phys. Oceanogr.}}
\def\jqsrt{{\it J. Quant. Spectrosc. Radiat. Transfer}}
\def\jpca{{\it J. Phys. Chem. A}}
\def\lnc{{\it Lett. Nuov. C}}
\def\mac{{\it Math. Comp.}}
\def\map{{\it Meteorol. Atmos. Physics.}}
\def\mem{{\it Meteorol. Mag.}}
\def\mnras{{\it Mon. Not. Roy. Astron. Soc.}} 
\def\mwr{{\it Mon. Weather Rev.}} 
\def\nat{{\it Nature}}
\def\pac{{\it Parallel Computing}}
\def\pag{{\it Pure Appl. Geophys.}}
\def\pal{{\it Paleoceanography}}
\def\pht{{\it Physics Today}}
\def\pieee{{\it Proc. IEEE}}
\def\pla{{\it Phys. Lett. A}}
\def\ppp{{\it Paleogeogr. Paleoclim. Paleoecol.}}
\def\pra{{\it Phys. Res. A}}
\def\prd{{\it Phys. Rev. D}}
\def\prl{{\it Phys. Rev. L}}
\def\pss{{\it Planet. Space Sci.}}
\def\ptrsl{{\it Phil. Trans. R. Soc. Lond.}}
\def\qjrms{{\it Q. J. R. Meteorol. Soc.}}
\def\qres{{\it Quat. Res.}}
\def\qsr{{\it Quatern. Sci. Rev.}}
\def\reg{{\it Rev. Geophys.}}
\def\rgsp{{\it Revs. Geophys. Space Phys.}}
\def\rpp{{\it Rep. Prog. Phys.}}
\def\sca{{\it Sci. Amer.}}
\def\sci{{\it Science}}
\def\sjna{{\it SIAM J. Numer. Anal.}}
\def\sjssc{{\it SIAM J. Sci. Stat. Comput.}}
\def\tac{{\it Theor. Appl. Climatl.}}
\def\tel{{\it Tellus}}
\def\wea{{\it Weather}}

%SIAM Review: (Society for Industrial and Applied Mathematics)
%       J. on Computing
%       J. on Control and Optimization
%       J. on Algebraic and Discrete Methods
%       J. on Numerical Analysis 
%       J. on Scientific and Statistical Computing





\topmargin -48pt   \headheight 12pt \headsep 18pt
\textheight 723pt \textwidth 470pt
\oddsidemargin 0pt \evensidemargin 0pt
\marginparwidth 72pt \marginparsep 7pt
\footskip 0pt
\footnotesep=14pt

\renewcommand\textfraction{0.}
\setcounter{totalnumber}{10}
\setcounter{topnumber}{10}
\setcounter{dbltopnumber}{10}
\setcounter{bottomnumber}{10}
\renewcommand\topfraction{1.}
\renewcommand\dbltopfraction{1.}
\renewcommand\bottomfraction{1.}
\renewcommand\floatpagefraction{1.}
\renewcommand\dblfloatpagefraction{1.}

\begin{document}

% These don't seem to work before \begin{document}
\setlength\abovecaptionskip{9pt}
\setlength\belowcaptionskip{9pt}

\setlength\floatsep{0pt}
\setlength\textfloatsep{0pt}
\setlength\dblfloatsep{0pt}
\setlength\dbltextfloatsep{0pt}
\setlength\intextsep{0pt}
%\setlength\floatsep{18pt \@plus 2pt \@minus 4pt}
%\setlength\textfloatsep{18pt \@plus 2pt \@minus 4pt}
%\setlength\dblfloatsep{18pt \@plus 2pt \@minus 4pt}
%\setlength\dbltextfloatsep{18pt \@plus 2pt \@minus 4pt}
%\setlength\intextsep{20pt \@plus 4pt \@minus 4pt}

\pagenumbering{roman}
\setcounter{page}{1}
\pagestyle{myheadings}
\thispagestyle{empty}
\onecolumn
\listoffigures
\twocolumn
\pagenumbering{arabic}
\setcounter{page}{1}
\markright{Climatologies of Analyses, CCM $\Omega_{.5}$, and ANV}

\begin{figure}
\begin{center}
\includegraphics*[height=\halfwidth,angle=90]{/data2/zender/ps/spcp_85_8589_amip5_8589_01_reg_IndoPacific_PRECT.eps}\vfill
\includegraphics*[height=\halfwidth,angle=90]{/data2/zender/ps/spcp_85_8589_amip5_8589_01_reg_IndoPacific_LWCF.eps}\vfill
\end{center}
\caption[Geographic distribution of difference (ANV$-$CCM) in
simulated precipitation $P$ and longwave cloud forcing LWCF between
ANV and CCM over the Indo-Pacific for January 1985--1989]{ 
Geographic distribution of difference (ANV$-$CCM) in simulated
(a) precipitation $P$ (\mmxday) and (b) longwave cloud forcing LWCF
(\wxmS) for January 1985--1989.
\label{fig:8589_01_reg_IndoPacific}}
\end{figure}
\clearpage

\begin{figure*}
\includegraphics*[width=.5\hsize,height=.14\textheight]{/data2/zender/ps/spcp_85_8589_amip5_8589_yavg_00N10N_01_QRS.eps}
\includegraphics*[width=.5\hsize,height=.14\textheight]{/data2/zender/ps/spcp_85_8589_amip5_8589_yavg_00N10N_07_QRS.eps}

\includegraphics*[width=.5\hsize,height=.14\textheight]{/data2/zender/ps/spcp_85_8589_amip5_8589_yavg_00N10N_01_QRL.eps}
\includegraphics*[width=.5\hsize,height=.14\textheight]{/data2/zender/ps/spcp_85_8589_amip5_8589_yavg_00N10N_07_QRL.eps}

\includegraphics*[width=.5\hsize,height=.14\textheight]{/data2/zender/ps/spcp_85_8589_amip5_8589_yavg_00N10N_01_RADD.eps}
\includegraphics*[width=.5\hsize,height=.14\textheight]{/data2/zender/ps/spcp_85_8589_amip5_8589_yavg_00N10N_07_RADD.eps}

\includegraphics*[width=.5\hsize,height=.14\textheight]{/data2/zender/ps/spcp_85_8589_amip5_8589_yavg_00N10N_01_HGS.eps}
\includegraphics*[width=.5\hsize,height=.14\textheight]{/data2/zender/ps/spcp_85_8589_amip5_8589_yavg_00N10N_07_HGS.eps}

\includegraphics*[width=.5\hsize,height=.14\textheight]{/data2/zender/ps/spcp_85_8589_amip5_8589_yavg_00N10N_01_CMFDT.eps}
\includegraphics*[width=.5\hsize,height=.14\textheight]{/data2/zender/ps/spcp_85_8589_amip5_8589_yavg_00N10N_07_CMFDT.eps}

\includegraphics*[width=.5\hsize,height=.16\textheight]{/data2/zender/ps/spcp_85_8589_amip5_8589_yavg_00N10N_01_QDIABAT.eps}
\includegraphics*[width=.5\hsize,height=.16\textheight]{/data2/zender/ps/spcp_85_8589_amip5_8589_yavg_00N10N_07_QDIABAT.eps}
\caption[Zonal distribution of difference (ANV$-$CCM) in simulated
meridional average (0--10~\degreee N) diabatic heating for 1985--1989
January and July.]{
Zonal distribution of difference (ANV$-$CCM) in simulated meridional
average (0--10~\degreee N) diabatic heating for 1985--1989 (left)
January and (right) July. 
Shown are (a) \QRL, (b) \QRS, (c) \QR, (d) \QLS, (e) \QC, and (f) \QT.
\label{fig:yavg_00N10N_8589_QT}}
\end{figure*}
\clearpage

\begin{figure*}
\includegraphics*[width=.5\hsize,height=.14\textheight]{/data2/zender/ps/spcp_85_8589_amip5_8589_yavg_00N30N_01_QRS.eps}
\includegraphics*[width=.5\hsize,height=.14\textheight]{/data2/zender/ps/spcp_85_8589_amip5_8589_yavg_00N30N_07_QRS.eps}

\includegraphics*[width=.5\hsize,height=.14\textheight]{/data2/zender/ps/spcp_85_8589_amip5_8589_yavg_00N30N_01_QRL.eps}
\includegraphics*[width=.5\hsize,height=.14\textheight]{/data2/zender/ps/spcp_85_8589_amip5_8589_yavg_00N30N_07_QRL.eps}

\includegraphics*[width=.5\hsize,height=.14\textheight]{/data2/zender/ps/spcp_85_8589_amip5_8589_yavg_00N30N_01_RADD.eps}
\includegraphics*[width=.5\hsize,height=.14\textheight]{/data2/zender/ps/spcp_85_8589_amip5_8589_yavg_00N30N_07_RADD.eps}

\includegraphics*[width=.5\hsize,height=.14\textheight]{/data2/zender/ps/spcp_85_8589_amip5_8589_yavg_00N30N_01_HGS.eps}
\includegraphics*[width=.5\hsize,height=.14\textheight]{/data2/zender/ps/spcp_85_8589_amip5_8589_yavg_00N30N_07_HGS.eps}

\includegraphics*[width=.5\hsize,height=.14\textheight]{/data2/zender/ps/spcp_85_8589_amip5_8589_yavg_00N30N_01_CMFDT.eps}
\includegraphics*[width=.5\hsize,height=.14\textheight]{/data2/zender/ps/spcp_85_8589_amip5_8589_yavg_00N30N_07_CMFDT.eps}

\includegraphics*[width=.5\hsize,height=.16\textheight]{/data2/zender/ps/spcp_85_8589_amip5_8589_yavg_00N30N_01_QDIABAT.eps}
\includegraphics*[width=.5\hsize,height=.16\textheight]{/data2/zender/ps/spcp_85_8589_amip5_8589_yavg_00N30N_07_QDIABAT.eps}
\caption[Zonal distribution of difference (ANV$-$CCM) in simulated
meridional average (0--30~\degreee N) diabatic heating for 1985--1989
January and July.]{
Zonal distribution of difference (ANV$-$CCM) in simulated meridional
average (0--30~\degreee N) diabatic heating for 1985--1989 (left)
January and (right) July. 
Shown are (a) \QRL, (b) \QRS, (c) \QR, (d) \QLS, (e) \QC, and (f) \QT.
\label{fig:yavg_00N30N_8589_QT}}
\end{figure*}
\clearpage

\begin{figure*}
\includegraphics*[width=.5\hsize,height=.14\textheight]{/data2/zender/ps/spcp_85_8589_amip5_8589_yavg_30S00N_01_QRS.eps}
\includegraphics*[width=.5\hsize,height=.14\textheight]{/data2/zender/ps/spcp_85_8589_amip5_8589_yavg_30S00N_07_QRS.eps}

\includegraphics*[width=.5\hsize,height=.14\textheight]{/data2/zender/ps/spcp_85_8589_amip5_8589_yavg_30S00N_01_QRL.eps}
\includegraphics*[width=.5\hsize,height=.14\textheight]{/data2/zender/ps/spcp_85_8589_amip5_8589_yavg_30S00N_07_QRL.eps}

\includegraphics*[width=.5\hsize,height=.14\textheight]{/data2/zender/ps/spcp_85_8589_amip5_8589_yavg_30S00N_01_RADD.eps}
\includegraphics*[width=.5\hsize,height=.14\textheight]{/data2/zender/ps/spcp_85_8589_amip5_8589_yavg_30S00N_07_RADD.eps}

\includegraphics*[width=.5\hsize,height=.14\textheight]{/data2/zender/ps/spcp_85_8589_amip5_8589_yavg_30S00N_01_HGS.eps}
\includegraphics*[width=.5\hsize,height=.14\textheight]{/data2/zender/ps/spcp_85_8589_amip5_8589_yavg_30S00N_07_HGS.eps}

\includegraphics*[width=.5\hsize,height=.14\textheight]{/data2/zender/ps/spcp_85_8589_amip5_8589_yavg_30S00N_01_CMFDT.eps}
\includegraphics*[width=.5\hsize,height=.14\textheight]{/data2/zender/ps/spcp_85_8589_amip5_8589_yavg_30S00N_07_CMFDT.eps}

\includegraphics*[width=.5\hsize,height=.16\textheight]{/data2/zender/ps/spcp_85_8589_amip5_8589_yavg_30S00N_01_QDIABAT.eps}
\includegraphics*[width=.5\hsize,height=.16\textheight]{/data2/zender/ps/spcp_85_8589_amip5_8589_yavg_30S00N_07_QDIABAT.eps}
\caption[Zonal distribution of difference (ANV$-$CCM) in simulated
meridional average (0--30~\degreee S) diabatic heating for 1985--1989
January and July.]{
Zonal distribution of difference (ANV$-$CCM) in simulated meridional
average (0--30~\degreee S) diabatic heating for 1985--1989 (left)
January and (right) July. 
Shown are (a) \QRL, (b) \QRS, (c) \QR, (d) \QLS, (e) \QC, and (f) \QT.
\label{fig:yavg_30S00N_8589_QT}}
\end{figure*}
\clearpage

\begin{figure*}
\begin{center}
\includegraphics*[width=.5\hsize]{/data2/zender/ps/legates_2080_01_TS1.eps}%
\includegraphics*[width=.5\hsize]{/data2/zender/ps/legates_2080_07_TS1.eps}%

\includegraphics*[width=.5\hsize]{/data2/zender/ps/amip5_8589_legates_2080_01_TS1.eps}%
\includegraphics*[width=.5\hsize]{/data2/zender/ps/amip5_8589_legates_2080_07_TS1.eps}%

\includegraphics*[width=.5\hsize]{/data2/zender/ps/spcp_85_8589_legates_2080_01_TS1.eps}%
\includegraphics*[width=.5\hsize]{/data2/zender/ps/spcp_85_8589_legates_2080_07_TS1.eps}%
\end{center}
\caption[Surface temperature \Ts\ for January and
July Legates analyses, CCM biases, and ANV biases]{
Surface temperature \Ts\ (\degreee K) for (left) January
and (right) July (top) Legates analyses, (middle) CCM biases, and
(bottom) ANV biases: (a) January Legates 1920--1980, (b) July Legates
1920--1980,
(c) January CCM$-$Legates, (d) July CCM$-$Legates, (e) January
ANV$-$Legates, and (f) July ANV$-$Legates.
Contour intervals are (a,b) 5~\degreee K and (c--f) 1~\degreee K. 
Shading indicates where the simulation is colder than analyzed.
\label{fig:pres_8589_TS1}}   
\end{figure*}
\clearpage

\begin{figure}
\begin{center}
\includegraphics*[width=\halfwidth]{/data2/zender/ps/spcp_85_8589_amip5_8589_01_FSNS.eps}\vfill
\includegraphics*[width=\halfwidth]{/data2/zender/ps/spcp_85_8589_amip5_8589_07_FSNS.eps}\vfill
\end{center}
\caption[Geographic distribution of difference (ANV$-$CCM) in
simulated surface absorbed solar flux between ANV and CCM for
January and July 1985--1989]{ 
Geographic distribution of difference (ANV$-$CCM) in simulated
surface absorbed solar flux (\wxmS) for 1985--1989 (top) January and
(bottom) July. 
\label{fig:spcp_85_8589_amip5_8589_FSNS}}
\end{figure}

\begin{figure}
\begin{center}
\includegraphics*[width=\halfwidth]{/data2/zender/ps/spcp_85_8589_amip5_8589_01_LHFLX.eps}\vfill
\includegraphics*[width=\halfwidth]{/data2/zender/ps/spcp_85_8589_amip5_8589_07_LHFLX.eps}\vfill
\end{center}
\caption[Geographic distribution of difference (ANV$-$CCM) in
simulated surface latent heat flux between ANV and CCM for
January and July 1985--1989]{ 
Geographic distribution of difference (ANV$-$CCM) in simulated
latent heat flux (\wxmS) (defined positive into the atmosphere) for
1985--1989 (top) January and (bottom) July. 
\label{fig:spcp_85_8589_amip5_8589_LHFLX}}
\end{figure}

\begin{figure}
\begin{center}
\includegraphics*[width=\halfwidth]{/data2/zender/ps/spcp_85_8589_amip5_8589_01_FLNS.eps}\vfill
\includegraphics*[width=\halfwidth]{/data2/zender/ps/spcp_85_8589_amip5_8589_07_FLNS.eps}\vfill
\end{center}
\caption[Geographic distribution of difference (ANV$-$CCM) in
simulated net surface longwave cooling between ANV and CCM for
January and July 1985--1989]{ 
Geographic distribution of difference (ANV$-$CCM) in simulated
net surface longwave cooling (\wxmS) for 1985--1989 (top) January and
(bottom) July. 
\label{fig:spcp_85_8589_amip5_8589_FLNS}}
\end{figure}

\begin{figure}
\begin{center}
\includegraphics*[width=\halfwidth]{/data2/zender/ps/spcp_85_8589_amip5_8589_01_SHFLX.eps}\vfill
\includegraphics*[width=\halfwidth]{/data2/zender/ps/spcp_85_8589_amip5_8589_07_SHFLX.eps}\vfill
\end{center}
\caption[Geographic distribution of difference (ANV$-$CCM) in
simulated surface sensible heat flux between ANV and CCM for
January and July 1985--1989]{ 
Geographic distribution of difference (ANV$-$CCM) in simulated
sensible heat flux (\wxmS) (defined positive into the atmosphere) for
1985--1989 (top) January and (bottom) July. 
\label{fig:spcp_85_8589_amip5_8589_SHFLX}}
\end{figure}
\clearpage

\begin{figure}
\begin{center}
\includegraphics*[width=\halfwidth]{/data2/zender/ps/spcp_85_8589_amip5_8589_01_NET.eps}\vfill
\includegraphics*[width=\halfwidth]{/data2/zender/ps/spcp_85_8589_amip5_8589_07_NET.eps}\vfill
\end{center}
\caption[Geographic distribution of difference (ANV$-$CCM) in
simulated net surface energy between ANV and CCM for January and July
1985--1989]{  
Geographic distribution of difference (ANV$-$CCM) in simulated
net surface energy (\wxmS) for 1985--1989 (top) January and (bottom)
July.  
\label{fig:spcp_85_8589_amip5_8589_NET}}
\end{figure}

\begin{figure}
\begin{center}
\includegraphics*[width=\halfwidth]{/data2/zender/ps/spcp_85_8589_amip5_8589_01_VMAGSFC.eps}\vfill
\includegraphics*[width=\halfwidth]{/data2/zender/ps/spcp_85_8589_amip5_8589_07_VMAGSFC.eps}\vfill
\end{center}
\caption[Geographic distribution of difference (ANV$-$CCM) in
surface wind speed $|V|$ between ANV and CCM for January and July
1985--1989]{  
Geographic distribution of difference (ANV$-$CCM) in simulated
surface wind speed $|V|$ (\mxs) for 1985--1989 (top) January and
(bottom) July.  
\label{fig:spcp_85_8589_amip5_8589_VMAGSFC}}
\end{figure}

\begin{figure}
\begin{center}
\includegraphics*[width=\halfwidth]{/data2/zender/ps/spcp_85_8589_amip5_8589_01_TS1.eps}\vfill
\includegraphics*[width=\halfwidth]{/data2/zender/ps/spcp_85_8589_amip5_8589_07_TS1.eps}\vfill
\end{center}
\caption[Geographic distribution of difference (ANV$-$CCM) in
simulated surface temperature \Ts\ between ANV and CCM for
January and July 1985--1989]{ 
Geographic distribution of difference (ANV$-$CCM) in simulated
surface temperature \Ts\ (\degreee K) for 1985--1989 (top) January and
(bottom) July. 
\label{fig:spcp_85_8589_amip5_8589_TS1}}
\end{figure}

\begin{figure}
\begin{center}
\includegraphics*[width=\halfwidth]{/data2/zender/ps/spcp_85_8589_amip5_8589_01_PRECT.eps}\vfill
\includegraphics*[width=\halfwidth]{/data2/zender/ps/spcp_85_8589_amip5_8589_07_PRECT.eps}\vfill
\end{center}
\caption[Geographic distribution of difference (ANV$-$CCM) in
simulated precipitation $P$ between ANV and CCM for January and July
1985--1989]{  
Geographic distribution of difference (ANV$-$CCM) in simulated
precipitation $P$ (\mmxday) for 1985--1989 (top) January and (bottom)
July.  
\label{fig:spcp_85_8589_amip5_8589_PRECT}}
\end{figure}
\clearpage

\begin{figure}
\begin{center}
\includegraphics*[width=\halfwidth]{/data2/zender/ps/spcp_85_8589_amip5_8589_01_PS.eps}\vfill
\includegraphics*[width=\halfwidth]{/data2/zender/ps/spcp_85_8589_amip5_8589_07_PS.eps}\vfill
\end{center}
\caption[Geographic distribution of difference (ANV$-$CCM) in
simulated surface pressure \ps\ between ANV and CCM for January and
July 1985--1989]{  
Geographic distribution of difference (ANV$-$CCM) in simulated
surface pressure \ps\ (mb) for 1985--1989 (top) January and
(bottom) July. 
\label{fig:spcp_85_8589_amip5_8589_PS}}
\end{figure}

\begin{figure}
\begin{center}
\includegraphics*[width=\halfwidth]{/data2/zender/ps/spcp_85_8589_amip5_8589_pres_01_Z2TEST.eps}\vfill
\includegraphics*[width=\halfwidth]{/data2/zender/ps/spcp_85_8589_amip5_8589_pres_07_Z2TEST.eps}\vfill
\end{center}
\caption[Geographic distribution of difference (ANV$-$CCM) in
simulated 500~mb geopotential height $\Phi$ between ANV and CCM for
January and July 1985--1989]{  
Geographic distribution of difference (ANV$-$CCM) in simulated
500~mb geopotential height $\Phi$ (gpm) for 1985--1989 (top) January
and (bottom) July. 
\label{fig:spcp_85_8589_amip5_8589_pres_Z2TEST}}
\end{figure}

\begin{figure}
\begin{center}
\includegraphics*[width=\halfwidth]{/data2/zender/ps/spcp_85_8589_amip5_8589_01_U.eps}\vfill
\includegraphics*[width=\halfwidth]{/data2/zender/ps/spcp_85_8589_amip5_8589_07_U.eps}\vfill
\end{center}
\caption[Geographic distribution of difference (ANV$-$CCM) in
simulated surface zonal wind $U$ between ANV and CCM for
January and July 1985--1989]{  
Geographic distribution of difference (ANV$-$CCM) in simulated
surface zonal wind $U$ (\mxs) for 1985--1989 (top) January
and (bottom) July. 
\label{fig:spcp_85_8589_amip5_8589_U}}
\end{figure}

\begin{figure}
\begin{center}
\includegraphics*[width=\halfwidth]{/data2/zender/ps/spcp_85_8589_amip5_8589_01_TS1.eps}\vfill
\includegraphics*[width=\halfwidth]{/data2/zender/ps/spcp_85_8589_amip5_8589_07_TS1.eps}\vfill
\end{center}
\caption[Geographic distribution of difference (ANV$-$CCM) in
simulated surface temperature \Ts\ between ANV and CCM for
January and July 1985--1989]{ 
Geographic distribution of difference (ANV$-$CCM) in simulated
surface temperature \Ts\ (\degreee K) for 1985--1989 (top) January and
(bottom) July. 
\label{fig:spcp_85_8589_amip5_8589_TS1}}
\end{figure}
\clearpage

\begin{figure*}
\begin{center}
\includegraphics*[width=.5\hsize]{/data2/zender/ps/amip5_xavg_8589_01_QC.eps}%
\includegraphics*[width=.5\hsize]{/data2/zender/ps/amip5_xavg_8589_07_QC.eps}%

\includegraphics*[width=.5\hsize]{/data2/zender/ps/spcp_85_xavg_8589_01_QC.eps}%
\includegraphics*[width=.5\hsize]{/data2/zender/ps/spcp_85_xavg_8589_07_QC.eps}%

\includegraphics*[width=.5\hsize]{/data2/zender/ps/spcp_85_8589_amip5_8589_xavg_01_QC.eps}%
\includegraphics*[width=.5\hsize]{/data2/zender/ps/spcp_85_8589_amip5_8589_xavg_07_QC.eps}%
\end{center}
\caption[Condensate mixing ratio \qc\ for January and July
1985--1989 simulated by CCM, ANV, and ANV$-$CCM]{
Condensate mixing ratio \qc\ (\mgxkg) for (left) January and (right)
July 1985--1989 simulated by (top) CCM, (middle) ANV, and (bottom)
ANV$-$CCM: (a) January CCM, (b) July CCM, (c) January ANV, (d) July
ANV, (e) January ANV$-$CCM, and (f) July ANV$-$CCM. 
For a--d contours are .5, 1, 2, 4, 6, 8, 10, 15, 20, 30, 40, 50, 60,
and 70~\mgxkg\ and shading indicates $\qc > 6$~\mgxkg.
For e--f contours are $-40$, $-30$, $-25$, $-20$, $-15$, $-10$, $-8$,
$-6$, $-4$, $-2$, $-1$, $-.5$, $-.1$, 0., .1, .5, 1, 2, 4, 6, 8, 10,
15, 20, 25, 30 and 40~\mgxkg\ and shading indicates where ANV has less
condensate than CCM.  
\label{fig:xavg_8589_QC}}   
\end{figure*}
\clearpage

\begin{figure*}
\begin{center}
\includegraphics*[width=.5\hsize]{/data2/zender/ps/spcp_85_8589_amip5_8589_xavg_01_QL.eps}%
\includegraphics*[width=.5\hsize]{/data2/zender/ps/spcp_85_8589_amip5_8589_xavg_07_QL.eps}%

\includegraphics*[width=.5\hsize]{/data2/zender/ps/spcp_85_8589_amip5_8589_xavg_01_QICE.eps}%
\includegraphics*[width=.5\hsize]{/data2/zender/ps/spcp_85_8589_amip5_8589_xavg_07_QICE.eps}%

\includegraphics*[width=.5\hsize]{/data2/zender/ps/spcp_85_8589_amip5_8589_xavg_01_QC.eps}%
\includegraphics*[width=.5\hsize]{/data2/zender/ps/spcp_85_8589_amip5_8589_xavg_07_QC.eps}%
\end{center}
\caption[Difference (ANV$-$CCM) in simulated liquid, ice and total
condensate mixing ratios \ql, \qi, and \qc\ for January and July
1985--1989]{ 
Difference (ANV$-$CCM) in simulated liquid, ice and total
condensate mixing ratios (top) \ql, (middle) \qi, and (bottom) \qc\
for (left) January and (right) July 1985--1989.
Contours are $-40$, $-30$, $-25$, $-20$, $-15$, $-10$, $-8$,
$-6$, $-4$, $-2$, $-1$, $-.5$, $-.1$, 0., .1, .5, 1, 2, 4, 6, 8, 10,
15, 20, 25, 30 and 40~\mgxkg\ and shading indicates where ANV has less
condensate than CCM.  
\label{fig:xavg_8589_QLQIQC}}   
\end{figure*}
\clearpage

\begin{figure*}
\begin{center}
\includegraphics*[width=.5\hsize]{/data2/zender/ps/amip5_8589_01_TOTCWP.eps}%
\includegraphics*[width=.5\hsize]{/data2/zender/ps/amip5_8589_07_TOTCWP.eps}%

\includegraphics*[width=.5\hsize]{/data2/zender/ps/spcp_85_8589_01_TOTCWP.eps}%
\includegraphics*[width=.5\hsize]{/data2/zender/ps/spcp_85_8589_07_TOTCWP.eps}%

\includegraphics*[width=.5\hsize]{/data2/zender/ps/spcp_85_8589_amip5_8589_01_TOTCWP.eps}%
\includegraphics*[width=.5\hsize]{/data2/zender/ps/spcp_85_8589_amip5_8589_07_TOTCWP.eps}%
\end{center}
\caption[Geographic distribution of condensed water path CWP from 
January and July 1985--1989 simulated by CCM and ANV]{
Geographic distribution of condensed water path CWP (\gxmS) from 
(left) January and (right) July 1985--1989 simulated by (top) CCM and
(bottom) ANV: (a) January CCM, (b) July CCM, (c) January ANV, and (d)
July ANV. 
Contour interval is 25~\gxmS. 
Shading indicates $\CWP > 75$~\gxmS.
\label{fig:8589_TOTCWP}}   
\end{figure*}
\clearpage

\begin{figure*}
\begin{center}
\includegraphics*[width=.5\hsize]{/data2/zender/ps/amip5_8589_01_TOTCWP.eps}%
\includegraphics*[width=.5\hsize]{/data2/zender/ps/amip5_8589_07_TOTCWP.eps}%

\includegraphics*[width=.5\hsize]{/data2/zender/ps/spcp_85_8589_01_TOTCWP.eps}%
\includegraphics*[width=.5\hsize]{/data2/zender/ps/spcp_85_8589_07_TOTCWP.eps}%
\end{center}
\caption[Geographic distribution of difference (ANV$-$CCM) in
simulated condensed water path between ANV and CCM for January and
July 1985--1989]{ 
Geographic distribution of difference (ANV$-$CCM) in condensed water
path CWP (\gxmS) simulated by ANV and CCM for 1985--1989 (a) January
and (b) July.
Contour interval is 25~\gxmS. 
Shading indicates $\CWP > 75$~\gxmS.
\label{fig:8589_TOTCWP}}   
\end{figure*}
\clearpage

\begin{figure*}
\begin{center}
\includegraphics*[width=.5\hsize]{/data2/zender/ps/ssmi_8795_01_TOTLWP.eps}%
\includegraphics*[width=.5\hsize]{/data2/zender/ps/ssmi_8791_07_TOTLWP.eps}%

\includegraphics*[width=.5\hsize]{/data2/zender/ps/amip5_8589_01_TOTLWP.eps}%
\includegraphics*[width=.5\hsize]{/data2/zender/ps/amip5_8589_07_TOTLWP.eps}%

\includegraphics*[width=.5\hsize]{/data2/zender/ps/spcp_85_8589_01_TOTLWP.eps}%
\includegraphics*[width=.5\hsize]{/data2/zender/ps/spcp_85_8589_07_TOTLWP.eps}%
\end{center}
\caption[Geographic distribution of January and July column liquid
water path LWP from 1987--1995 SSMI, and 1985--1989 CCM and
ANV]{ 
Maritime distribution of (left) January and (right) July
column liquid water path LWP (\gxmS) from (a,b) SSMI 1987--1995
\cite[]{WeG94}, SSMI 1987--1991 (Greenwald) (c,d) CCM 1985--1989, and
(e,f) ANV 1985--1989.  
\label{fig:8589_TOTLWP}}
\end{figure*}
\clearpage

\begin{figure*}
\begin{center}
\includegraphics*[width=.5\hsize]{/data2/zender/ps/xavg_8589_01_TOTLWP.eps}%
\includegraphics*[width=.5\hsize]{/data2/zender/ps/xavg_8589_07_TOTLWP.eps}%

\includegraphics*[width=.5\hsize]{/data2/zender/ps/xavg_8589_01_TOTIWP.eps}%
\includegraphics*[width=.5\hsize]{/data2/zender/ps/xavg_8589_07_TOTIWP.eps}%

\includegraphics*[width=.5\hsize]{/data2/zender/ps/xavg_8589_01_TOTCWP.eps}%
\includegraphics*[width=.5\hsize]{/data2/zender/ps/xavg_8589_07_TOTCWP.eps}%
\end{center}
\caption[Zonal average column condensate burdens LWP, IWP, and CWP
from 1985--1989 January and July simulations by CCM and ANV]{ 
Zonal average column condensate burdens LWP, IWP, and CWP (\gxmS) from
1985--1989 (left) January and (right) July simulations by (solid) CCM
and (dashed) ANV of (a,b) LWP, (c,d) IWP, and (e,f) CWP.   
\label{fig:xavg_8589_CP}}   
\end{figure*}
\clearpage

\begin{figure}
\begin{center}
\includegraphics*[width=\halfwidth]{/data2/zender/ps/spcp_85_8589_amip5_8589_xavg_01_QC.eps}\vfill
\includegraphics*[width=\halfwidth]{/data2/zender/ps/spcp_85_8589_amip5_8589_xavg_07_QC.eps}\vfill
\end{center}
\caption[Difference (ANV$-$CCM) in simulated zonal average condensate
mixing ratio \qc\ between ANV and CCM for January and July
1985--1989]{
Difference (ANV$-$CCM) in simulated zonal average condensate mixing
ratio \qc\ (\mgxkg) between ANV and CCM for (a) January and (b) July 
1985--1989. 
Contour interval is irregular and shading indicates where ANV has less
condensate than CCM.
\label{fig:xavg_8589_QC}}   
\end{figure}

\begin{figure}
\begin{center}
\includegraphics*[width=\halfwidth]{/data2/zender/ps/spcp_85_8589_amip5_8589_xavg_01_QRS.eps}\vfill
\includegraphics*[width=\halfwidth]{/data2/zender/ps/spcp_85_8589_amip5_8589_xavg_07_QRS.eps}\vfill
\end{center}
\caption[Difference (ANV$-$CCM) in simulated zonal average shortwave
radiative heating \QRS\ between ANV and CCM for January and
July 1985--1989]{
Difference (ANV$-$CCM) in simulated zonal average shortwave radiative
heating \QR\ (\kxday) between ANV and CCM for (a) January and (b) July
1985--1989. 
Contour interval is .1~\kxday. 
Shading indicates where ANV heats less than CCM. 
\label{fig:xavg_8589_QRS}}   
\end{figure}

\begin{figure}
\begin{center}
\includegraphics*[width=\halfwidth]{/data2/zender/ps/spcp_85_8589_amip5_8589_xavg_01_RADD.eps}\vfill
\includegraphics*[width=\halfwidth]{/data2/zender/ps/spcp_85_8589_amip5_8589_xavg_07_RADD.eps}\vfill
\end{center}
\caption[Difference (ANV$-$CCM) in simulated zonal average radiative
heating \QR\ between ANV and CCM for January and July 1985--1989]{
Difference (ANV$-$CCM) in simulated zonal average radiative heating
\QR\ (\kxday) between ANV and CCM for (a) January and (b) July
1985--1989. 
Contour interval is .1~\kxday. 
Shading indicates where ANV cools more (heats less) than CCM. 
\label{fig:xavg_8589_RADD}}   
\end{figure}

\begin{figure}
\begin{center}
\includegraphics*[width=\halfwidth]{/data2/zender/ps/spcp_85_8589_amip5_8589_xavg_01_QRL.eps}\vfill
\includegraphics*[width=\halfwidth]{/data2/zender/ps/spcp_85_8589_amip5_8589_xavg_07_QRL.eps}\vfill
\end{center}
\caption[Difference (ANV$-$CCM) in simulated zonal average longwave
radiative heating \QRL\ between ANV and CCM for January and
July 1985--1989]{
Difference (ANV$-$CCM) in simulated zonal average shortwave radiative
heating \QR\ (\kxday) between ANV and CCM for (a) January and (b) July
1985--1989. 
Contour interval is .1~\kxday. 
Shading indicates where ANV cools more (heats less) than CCM. 
\label{fig:xavg_8589_QRL}}   
\end{figure}
\clearpage

\begin{figure}
\begin{center}
\includegraphics*[width=\halfwidth]{/data2/zender/ps/spcp_85_8589_amip5_8589_xavg_01_RADD.eps}\vfill
\includegraphics*[width=\halfwidth]{/data2/zender/ps/spcp_85_8589_amip5_8589_xavg_07_RADD.eps}\vfill
\end{center}
\caption[Difference (ANV$-$CCM) in simulated zonal average radiative
heating \QR\ between ANV and CCM for January and July
1985--1989]{
Difference (ANV$-$CCM) in simulated zonal average radiative heating
\QR\ (\kxday) between ANV and CCM for (a) January and (b) July
1985--1989. 
Contour interval is .1~\kxday. 
Shading indicates where ANV cools more (heats less) than CCM. 
\label{fig:xavg_8589_RADD}}   
\end{figure}

\begin{figure}
\begin{center}
\includegraphics*[width=\halfwidth]{/data2/zender/ps/spcp_85_8589_amip5_8589_xavg_01_HGS.eps}\vfill
\includegraphics*[width=\halfwidth]{/data2/zender/ps/spcp_85_8589_amip5_8589_xavg_07_HGS.eps}\vfill
\end{center}
\caption[Difference (ANV$-$CCM) in simulated zonal average large scale
condensational heating \QL\ between ANV and CCM for January
and July 1985--1989]{
Difference (ANV$-$CCM) in simulated zonal average large scale
condensational heating \QL\ (\kxday) between ANV and CCM for (a)
January and (b) July 1985--1989. 
Contour interval is .1~\kxday. 
Shading indicates where ANV cools more (heats less) than CCM. 
\label{fig:xavg_8589_HGS}}   
\end{figure}

\begin{figure}
\begin{center}
\includegraphics*[width=\halfwidth]{/data2/zender/ps/spcp_85_8589_amip5_8589_xavg_01_CMFDT.eps}\vfill
\includegraphics*[width=\halfwidth]{/data2/zender/ps/spcp_85_8589_amip5_8589_xavg_07_CMFDT.eps}\vfill
\end{center}
\caption[Difference (ANV$-$CCM) in simulated zonal average convective
heating \QC\ between ANV and CCM for January and July 1985--1989]{
Difference (ANV$-$CCM) in simulated zonal average convective heating
\QC\ (\kxday) between ANV and CCM for (a) January and (b) July
1985--1989. 
Contour interval is .1~\kxday. 
Shading indicates where ANV cools more (heats less) than CCM. 
\label{fig:xavg_8589_CMFDT}}   
\end{figure}

\begin{figure}
\begin{center}
\includegraphics*[width=\halfwidth]{/data2/zender/ps/spcp_85_8589_amip5_8589_xavg_01_QDIABAT.eps}\vfill
\includegraphics*[width=\halfwidth]{/data2/zender/ps/spcp_85_8589_amip5_8589_xavg_07_QDIABAT.eps}\vfill
\end{center}
\caption[Difference (ANV$-$CCM) in simulated zonal average diabatic
heating \QT\ between ANV and CCM for January and July 1985--1989]{
Difference (ANV$-$CCM) in simulated zonal average diabatic heating
\QT\ (\kxday) between ANV and CCM for (a) January and (b) July
1985--1989. 
Contour interval is .1~\kxday. 
Shading indicates where ANV cools more (heats less) than CCM. 
\label{fig:xavg_8589_QDIABAT}}   
\end{figure}
\clearpage

\begin{figure*}
\begin{center}
\includegraphics*[width=.5\hsize]{/data2/zender/ps/gpcp_8894_01_PRECT.eps}%
\includegraphics*[width=.5\hsize]{/data2/zender/ps/gpcp_8793_07_PRECT.eps}%

\includegraphics*[width=.5\hsize]{/data2/zender/ps/amip5_8589_01_PRECT.eps}%
\includegraphics*[width=.5\hsize]{/data2/zender/ps/amip5_8589_07_PRECT.eps}%

\includegraphics*[width=.5\hsize]{/data2/zender/ps/spcp_85_8589_01_PRECT.eps}%
\includegraphics*[width=.5\hsize]{/data2/zender/ps/spcp_85_8589_07_PRECT.eps}%
\end{center}
\caption[Geographic distribution of total precipitation $P$ 
for January and July GPCP, CCM, and ANV]{ 
Geographic distribution of total precipitation $P$ (\mmxday)
for (left) January and (right) July (a,b) 1987--1994 GPCP, (c,d)
1985--1989 CCM, and (e,f) 1985--1989 ANV.   
\label{fig:8589_PRECT}}   
\end{figure*}
\clearpage

\begin{figure*}
\begin{center}
\includegraphics*[width=.5\hsize]{/data2/zender/ps/legates_2080_01_PRECT.eps}%
\includegraphics*[width=.5\hsize]{/data2/zender/ps/legates_2080_07_PRECT.eps}%

\includegraphics*[width=.5\hsize]{/data2/zender/ps/amip5_8589_legates_2080_01_PRECT.eps}%
\includegraphics*[width=.5\hsize]{/data2/zender/ps/amip5_8589_legates_2080_07_PRECT.eps}%

\includegraphics*[width=.5\hsize]{/data2/zender/ps/spcp_85_8589_legates_2080_01_PRECT.eps}%
\includegraphics*[width=.5\hsize]{/data2/zender/ps/spcp_85_8589_legates_2080_07_PRECT.eps}%
\end{center}
\caption[Geographic distribution of total precipitation $P$ 
for January and July Legates analyses, CCM biases, and ANV biases]{  
Geographic distribution of total precipitation $P$ (\mmxday)
for (left) January and (right) July (top) Legates analyses, 
(middle) CCM biases, and (bottom) ANV biases: (a,b) 1920--1980 Legates
analyses, (c,d) CCM$-$Legates, and (e,f) ANV$-$Legates.   
\label{fig:8589_PRECT}}   
\end{figure*}
\clearpage

\begin{figure*}
\begin{center}
\includegraphics*[width=.5\hsize]{/data2/zender/ps/spcp_85_8589_amip5_8589_01_PRECC.eps}%
\includegraphics*[width=.5\hsize]{/data2/zender/ps/spcp_85_8589_amip5_8589_07_PRECC.eps}%

\includegraphics*[width=.5\hsize]{/data2/zender/ps/spcp_85_8589_amip5_8589_01_PRECL.eps}%
\includegraphics*[width=.5\hsize]{/data2/zender/ps/spcp_85_8589_amip5_8589_07_PRECL.eps}%

\includegraphics*[width=.5\hsize]{/data2/zender/ps/spcp_85_8589_amip5_8589_01_PRECT.eps}%
\includegraphics*[width=.5\hsize]{/data2/zender/ps/spcp_85_8589_amip5_8589_07_PRECT.eps}%
\end{center}
\caption[Geographic distribution of difference (ANV$-$CCM) in
simulated precipitation between ANV and CCM for January and
July 1985--1989]{   
Geographic distribution of difference in (ANV$-$CCM) in simulated
precipitation (\mmxday) between ANV and CCM for (left) January and
(right) July (a,b) convective precipitation, (c,d) large scale
precipitation, and (e,f) total precipitation.
\label{fig:spcp_85_8589_amip5_8589_PREC}}   
\end{figure*}
\clearpage

\begin{figure*}
\begin{center}
\includegraphics*[width=.5\hsize]{/data2/zender/ps/erbe_b_8589_01_SWCF.eps}%
\includegraphics*[width=.5\hsize]{/data2/zender/ps/erbe_b_8589_07_SWCF.eps}%

\includegraphics*[width=.5\hsize]{/data2/zender/ps/amip5_8589_01_SWCF.eps}%
\includegraphics*[width=.5\hsize]{/data2/zender/ps/amip5_8589_07_SWCF.eps}%

\includegraphics*[width=.5\hsize]{/data2/zender/ps/spcp_85_8589_01_SWCF.eps}%
\includegraphics*[width=.5\hsize]{/data2/zender/ps/spcp_85_8589_07_SWCF.eps}%
\end{center}
\caption[Geographic distribution of shortwave cloud forcing SWCF
for 1985--1989 January and July ERBE, CCM, and ANV]{
Geographic distribution of shortwave cloud forcing SWCF (\wxmS) for
1985--1989 January and July (a,b) ERBE, (c,d) CCM, and (e,f) ANV.  
\label{fig:8589_SWCF}}   
\end{figure*}
\clearpage

\begin{figure*}
\begin{center}
\includegraphics*[width=.5\hsize]{/data2/zender/ps/erbe_b_8589_01_LWCF.eps}%
\includegraphics*[width=.5\hsize]{/data2/zender/ps/erbe_b_8589_07_LWCF.eps}%

\includegraphics*[width=.5\hsize]{/data2/zender/ps/amip5_8589_01_LWCF.eps}%
\includegraphics*[width=.5\hsize]{/data2/zender/ps/amip5_8589_07_LWCF.eps}%

\includegraphics*[width=.5\hsize]{/data2/zender/ps/spcp_85_8589_01_LWCF.eps}%
\includegraphics*[width=.5\hsize]{/data2/zender/ps/spcp_85_8589_07_LWCF.eps}%
\end{center}
\caption[Geographic distribution of longwave cloud forcing LWCF
for 1985--1989 January and July ERBE, CCM, and ANV]{
Geographic distribution of longwave cloud forcing LWCF (\wxmS) for
1985--1989 January and July (a,b) ERBE, (c,d) CCM, and (e,f) ANV. 
\label{fig:8589_LWCF}}   
\end{figure*}
\clearpage

\begin{figure*}
\begin{center}
\includegraphics*[width=.5\hsize]{/data2/zender/ps/spcp_85_8589_amip5_8589_01_SWCF.eps}%
\includegraphics*[width=.5\hsize]{/data2/zender/ps/spcp_85_8589_amip5_8589_01_LWCF.eps}%

\includegraphics*[width=.5\hsize]{/data2/zender/ps/spcp_85_8589_amip5_8589_07_SWCF.eps}%
\includegraphics*[width=.5\hsize]{/data2/zender/ps/spcp_85_8589_amip5_8589_07_LWCF.eps}%
\end{center}
\caption[Geographic distribution of difference (ANV$-$CCM) in
simulated shortwave and longwave cloud forcing between ANV and CCM for
January and July 1985--1989]{ 
Geographic distribution of difference (ANV$-$CCM) in simulated
(left) shortwave and (right) longwave cloud forcing SWCF and LWCF
(\wxmS) for 1985--1989 (top) January and (bottom) July. 
\label{fig:spcp_85_8589_amip5_8589_CF}}
\end{figure*}
\clearpage

\begin{figure}
\begin{center}
\includegraphics*[width=\halfwidth]{/data2/zender/ps/erbe_b_422_amip5_sld012d_xavg_8589_01_SWCF.eps}\vfill
\includegraphics*[width=\halfwidth]{/data2/zender/ps/erbe_b_422_amip5_sld012d_xavg_8589_07_SWCF.eps}\vfill
\end{center}
\caption[Zonal average shortwave cloud forcing SWCF from ERBE,
CCM2, CCM$\Omega_{.5}$, and CCM3 for 1985--1989 January and July]{  
Zonal average shortwave cloud forcing SWCF (\wxmS) from ERBE
(solid), CCM2 (dotted), CCM$\Omega_{.5}$ (dashed), and CCM3 (dash-dot)
for (a) January and (b) July 1985--1989. 
\label{fig:xavg_8589_SWCF}}   
\end{figure}

\begin{figure}
\begin{center}
\includegraphics*[width=\halfwidth]{/data2/zender/ps/xavg_8589_01_ALBEDO.eps}\vfill
\includegraphics*[width=\halfwidth]{/data2/zender/ps/xavg_8589_07_ALBEDO.eps}\vfill
\end{center}
\caption[Zonal average planetary albedo $A$ from ERBE, CCM, and
ANV for 1985--1989 January and July]{ 
Zonal average planetary albedo $A$ (\%) from ERBE (solid), CCM
(dotted), and ANV (dashed) for 1985--1989 (a) January and (b) July.  
\label{fig:xavg_8589_ALBEDO}}   
\end{figure}

\begin{figure}
\begin{center}
\includegraphics*[width=\halfwidth]{/data2/zender/ps/erbe_b_422_amip5_sld012d_xavg_8589_01_LWCF.eps}\vfill
\includegraphics*[width=\halfwidth]{/data2/zender/ps/erbe_b_422_amip5_sld012d_xavg_8589_07_LWCF.eps}\vfill
\end{center}
\caption[Zonal average longwave cloud forcing LWCF from ERBE,
CCM, and ANV for 1985--1989 January and July]{
Zonal average longwave cloud forcing LWCF (\wxmS) from ERBE
(solid), CCM2 (dotted), CCM$\Omega_{.5}$ (dashed), and CCM3 (dash-dot)
for (a) January and (b) July 1985--1989. 
\label{fig:xavg_8589_LWCF}}   
\end{figure}

\begin{figure}
\begin{center}
\includegraphics*[width=\halfwidth]{/data2/zender/ps/xavg_8589_01_FLNTC.eps}\vfill
\includegraphics*[width=\halfwidth]{/data2/zender/ps/xavg_8589_07_FLNTC.eps}\vfill
\end{center}
\caption[Zonal average outgoing longwave radiation in clear sky
\OLRc\ from ERBE, CCM, and ANV for 1985--1989 January and
July]{ 
Zonal average outgoing longwave radiation in clear sky
\OLRc\ (\wxmS) from ERBE (solid), CCM (dotted), and ANV (dashed) for
1985--1989 (a) January and (b) July. 
\label{fig:xavg_8589_FLNTC}}   
\end{figure}
\clearpage

\begin{figure*}
\begin{center}
\includegraphics*[width=.5\hsize]{/data2/zender/ps/erbe_b_8589_01_GCLD.eps}%
\includegraphics*[width=.5\hsize]{/data2/zender/ps/erbe_b_8589_07_GCLD.eps}%

\includegraphics*[width=.5\hsize]{/data2/zender/ps/amip5_8589_01_GCLD.eps}%
\includegraphics*[width=.5\hsize]{/data2/zender/ps/amip5_8589_07_GCLD.eps}%

\includegraphics*[width=.5\hsize]{/data2/zender/ps/spcp_85_8589_01_GCLD.eps}%
\includegraphics*[width=.5\hsize]{/data2/zender/ps/spcp_85_8589_07_GCLD.eps}%
\end{center}
\caption[Geographic distribution of all-sky greenhouse effect $G$
for 1985--1989 January and July ERBE, CCM, and ANV]{
Geographic distribution of all-sky greenhouse effect $G$ (\wxmS) for
1985--1989 January and July (a,b) ERBE, (c,d) CCM, and (e,f) ANV.  
\label{fig:8589_GCLD}}   
\end{figure*}
\clearpage

\begin{figure}
\begin{center}
\includegraphics*[width=\halfwidth]{/data2/zender/ps/erbe_b_anom_xavg_8589_0112_SWCF.eps}\vfill
\includegraphics*[width=\halfwidth]{/data2/zender/ps/amip5_anom_xavg_8589_0112_SWCF.eps}\vfill
\includegraphics*[width=\halfwidth]{/data2/zender/ps/spcp_85_anom_xavg_8589_0112_SWCF.eps}\vfill
\end{center}
\caption[Seasonal amplitude of zonal average shortwave cloud forcing
SWCF for ERBE, CCM, and ANV]{
Seasonal amplitude of zonal average shortwave cloud 
forcing (\wxmS) for (a) ERBE, (b) CCM, and (c) ANV. 
Shading indicates ensemble monthly value is less than the 5~year
mean. 
Contour interval is 10~\wxmS.
Tickmarks represent mid-month values (i.e., N represents
November~15). 
\label{fig:anom_xavg_8589_0112_SWCF}}   
\end{figure}

\begin{figure}
\begin{center}
\includegraphics*[width=\halfwidth]{/data2/zender/ps/erbe_b_anom_xavg_8589_0112_LWCF.eps}\vfill
\includegraphics*[width=\halfwidth]{/data2/zender/ps/amip5_anom_xavg_8589_0112_LWCF.eps}\vfill
\includegraphics*[width=\halfwidth]{/data2/zender/ps/spcp_85_anom_xavg_8589_0112_LWCF.eps}\vfill
\end{center}
\caption[Seasonal amplitude of zonal average longwave cloud forcing 
LWCF for ERBE, CCM, and ANV]{
Seasonal amplitude of zonal average longwave cloud 
forcing (\wxmS) for (a) ERBE, (b) CCM, and (c) ANV. 
Shading indicates ensemble monthly value is less than the 5~year
mean. 
Contour interval is 4~\wxmS.
Tickmarks represent mid-month values (i.e., N represents
November~15). 
\label{fig:anom_xavg_8589_0112_LWCF}}   
\end{figure}
\clearpage

\begin{figure*}
\begin{center}
\includegraphics*[width=.5\hsize]{/data2/zender/ps/ecmwf_pres_xavg_9095_01_T.eps}%
\includegraphics*[width=.5\hsize]{/data2/zender/ps/ecmwf_pres_xavg_8994_07_T.eps}%

\includegraphics*[width=.5\hsize]{/data2/zender/ps/amip5_8589_ecmwf_9095_pres_xavg_01_T.eps}%
\includegraphics*[width=.5\hsize]{/data2/zender/ps/amip5_8589_ecmwf_8994_pres_xavg_07_T.eps}%

\includegraphics*[width=.5\hsize]{/data2/zender/ps/spcp_85_8589_ecmwf_9095_pres_xavg_01_T.eps}%
\includegraphics*[width=.5\hsize]{/data2/zender/ps/spcp_85_8589_ecmwf_8994_pres_xavg_07_T.eps}%
\end{center}
\caption[Zonal average temperature $T$ for January and
July ECMWF analyses, CCM biases, and ANV biases]{
Zonal average temperature $T$ (\degreee K) for (left) January
and (right) July (top) ECMWF analyses, (middle) CCM biases, and
(bottom) ANV biases: (a) January ECMWF 1990--1995, (b) July ECMWF
1989-1994, 
(c) January CCM$-$ECMWF, (d) July CCM$-$ECMWF, (e) January
ANV$-$ECMWF, and (f) July ANV$-$ECMWF.
Contour intervals are (a,b) 5~\degreee K and (c--f) 1~\degreee K. 
Shading indicates where the simulation is colder than analyzed.
\label{fig:pres_xavg_8589_T}}   
\end{figure*}
\clearpage

\begin{figure*}
\begin{center}
\includegraphics*[width=.5\hsize]{/data2/zender/ps/ecmwf_pres_xavg_9095_01_U.eps}%
\includegraphics*[width=.5\hsize]{/data2/zender/ps/ecmwf_pres_xavg_8994_07_U.eps}%

\includegraphics*[width=.5\hsize]{/data2/zender/ps/amip5_pres_xavg_8589_01_U.eps}%
\includegraphics*[width=.5\hsize]{/data2/zender/ps/amip5_pres_xavg_8589_07_U.eps}%

\includegraphics*[width=.5\hsize]{/data2/zender/ps/spcp_85_pres_xavg_8589_01_U.eps}%
\includegraphics*[width=.5\hsize]{/data2/zender/ps/spcp_85_pres_xavg_8589_07_U.eps}%
\end{center}
\caption[Zonal average zonal wind $U$ for January and July ECMWF, CCM,
and ANV]{ 
Zonal average zonal wind $U$ (\mxs) for (left) January and
(right) July (a,b) 1989--1994 ECMWF, (c,d) 1985--1989 CCM, and (e,f)
1985--1989 ANV.  
Contour interval is 5~\mxs.
Shading indicates easterly zonal winds ($U < 0$).
\label{fig:pres_xavg_8589_U}}
\end{figure*}
\clearpage

\begin{figure*}
\begin{center}
\includegraphics*[width=.5\hsize]{/data2/zender/ps/ecmwf_pres_xavg_9095_01_U.eps}%
\includegraphics*[width=.5\hsize]{/data2/zender/ps/ecmwf_pres_xavg_8994_07_U.eps}%

\includegraphics*[width=.5\hsize]{/data2/zender/ps/amip5_8589_ecmwf_9095_pres_xavg_01_U.eps}%
\includegraphics*[width=.5\hsize]{/data2/zender/ps/amip5_8589_ecmwf_8994_pres_xavg_07_U.eps}%

\includegraphics*[width=.5\hsize]{/data2/zender/ps/spcp_85_8589_ecmwf_9095_pres_xavg_01_U.eps}%
\includegraphics*[width=.5\hsize]{/data2/zender/ps/spcp_85_8589_ecmwf_8994_pres_xavg_07_U.eps}%
\end{center}
\caption[Zonal average zonal wind $U$ for January
and July ECMWF analyses, CCM biases, and ANV biases]{
Zonal average zonal wind $U$ (\mxs) for (left) January
and (right) July (top) ECMWF analyses, (middle) CCM biases, and (bottom)
ANV biases: (a) January ECMWF 1990--1995, (b) July ECMWF 1989-1994,
(c) January CCM$-$ECMWF, (d) July CCM$-$ECMWF, (e) January
ANV$-$ECMWF, and (f) July ANV$-$ECMWF.
Contour intervals are (a,b) 5~\mxs\ and (c--f) 2~\mxs. 
Shading indicates (a,b) easterly zonal winds ($U < 0$) or (c--f) where
the simulation is less (more easterly) than analyzed.
\label{fig:pres_xavg_8589_U}}   
\end{figure*}
\clearpage

\begin{figure*}
\begin{center}
\includegraphics*[width=.33\hsize]{/data2/zender/ps/ecmwf_pres_9095_01_Z2TEST.eps}%
\includegraphics*[width=.33\hsize]{/data2/zender/ps/amip5_pres_8589_01_Z2TEST.eps}%
\includegraphics*[width=.33\hsize]{/data2/zender/ps/spcp_85_pres_8589_01_Z2TEST.eps}%
\end{center}
\caption[January 500~mb geopotential height $\Phi$ for
30--90~$\degreee$N from ECMWF 1990--1995 analyses and model 
simulations of 1985--1989 by CCM and ANV]{
January 500~mb geopotential height $\Phi$ (gpm) for
30--90~$\degreee$N from (a) ECMWF 1990--1995 analyses and model 
simulations of 1985--1989 by (b) CCM and (c) ANV.
Contour interval is 10~gpm. 
\label{fig:pres_8589_Z2TEST}}   
\end{figure*}
\clearpage

\begin{figure*}
\begin{center}
\includegraphics*[width=.5\hsize]{/data2/zender/ps/ecmwf_pres_9095_01_Z2TEST.eps}%
\includegraphics*[width=.5\hsize]{/data2/zender/ps/ecmwf_pres_8994_07_Z2TEST.eps}%

\includegraphics*[width=.5\hsize]{/data2/zender/ps/amip5_pres_8589_01_Z2TEST.eps}%
\includegraphics*[width=.5\hsize]{/data2/zender/ps/amip5_pres_8589_07_Z2TEST.eps}%

\includegraphics*[width=.5\hsize]{/data2/zender/ps/spcp_85_pres_8589_01_Z2TEST.eps}%
\includegraphics*[width=.5\hsize]{/data2/zender/ps/spcp_85_pres_8589_07_Z2TEST.eps}%
\end{center}
\caption[500~mb geopotential height $\Phi$ for January and
July ECMWF, CCM, and ANV]{
500~mb geopotential height $\Phi$(gpm) for (left) January and
(right) July (a,b) ECMWF, (c,d) CCM, and (e,f) ANV.
Contour interval is 10~gpm. 
Shading indicates $\Phi > 560$~gpm.
\label{fig:pres_8589_Z2TEST}}
\end{figure*}
\clearpage

\begin{figure*}
\begin{center}
\includegraphics*[width=.5\hsize]{/data2/zender/ps/ecmwf_pres_9095_01_Z2TEST.eps}%
\includegraphics*[width=.5\hsize]{/data2/zender/ps/ecmwf_pres_8994_07_Z2TEST.eps}%

\includegraphics*[width=.5\hsize]{/data2/zender/ps/amip5_8589_ecmwf_9095_pres_01_Z2TEST.eps}%
\includegraphics*[width=.5\hsize]{/data2/zender/ps/amip5_8589_ecmwf_8994_pres_07_Z2TEST.eps}%

\includegraphics*[width=.5\hsize]{/data2/zender/ps/spcp_85_8589_ecmwf_9095_pres_01_Z2TEST.eps}%
\includegraphics*[width=.5\hsize]{/data2/zender/ps/spcp_85_8589_ecmwf_8994_pres_07_Z2TEST.eps}%
\end{center}
\caption[500~mb geopotential height $\Phi$ for January
and July ECMWF analyses, CCM biases, and ANV biases]{
500~mb geopotential height $\Phi$ (gpm) for (left) January
and (right) July (top) ECMWF analyses, (middle) CCM biases, and (bottom)
ANV biases: (a) January ECMWF 1990--1995, (b) July ECMWF 1989-1994,
(c) January CCM$-$ECMWF, (d) July CCM$-$ECMWF, (e) January
ANV$-$ECMWF, and (f) July ANV$-$ECMWF.
Contour intervals are (a,b) 10~gpm\ and (c--f) 2~gpm.
Shading indicates (a,b) $\Phi > 560$~gpm or (c--f) where simulations 
predict lower height (less potential energy) than observed. 
\label{fig:pres_8589_Z2TEST}}   
\end{figure*}
\clearpage

\begin{figure}
\begin{center}
\includegraphics*[width=\halfwidth]{/data2/zender/ps/spcp_85_8589_amip5_8589_xavg_01_T.eps}\vfill
\includegraphics*[width=\halfwidth]{/data2/zender/ps/spcp_85_8589_amip5_8589_xavg_07_T.eps}\vfill
\end{center}
\caption[Difference (ANV$-$CCM) in zonal average temperature $T$
for 1985--1989 January and July]{
Difference (ANV$-$CCM) in zonal average temperature $T$
(\degreee K) for 1985--1989 (a) January and (b) July.
Contour interval is .5~\degreee K. 
Shading indicates where ANV is cooler than CCM.
\label{fig:8589_T}}   
\end{figure}

\begin{figure}
\begin{center}
\includegraphics*[width=\halfwidth]{/data2/zender/ps/spcp_85_8589_amip5_8589_xavg_01_U.eps}\vfill
\includegraphics*[width=\halfwidth]{/data2/zender/ps/spcp_85_8589_amip5_8589_xavg_07_U.eps}\vfill
\end{center}
\caption[Difference (ANV$-$CCM) in zonal average zonal wind $U$
for 1985--1989 January and July]{
Difference (ANV$-$CCM) in zonal average zonal wind $U$ (\mxs) for
1985--1989 (a) January and (b) July.
Contour interval is 1~\mxs. 
Shading indicates where ANV is more easterly than CCM.
\label{fig:8589_U}}   
\end{figure}

\begin{figure}
\begin{center}
\includegraphics*[width=\halfwidth]{/data2/zender/ps/spcp_85_8589_amip5_8589_pres_xavg_01_MPSI.eps}\vfill
\includegraphics*[width=\halfwidth]{/data2/zender/ps/spcp_85_8589_amip5_8589_pres_xavg_07_MPSI.eps}\vfill
\end{center}
\caption[Difference (ANV$-$CCM) in meridional stream function \mpsi\ 
for 1985--1989 January and July]{
Difference (ANV$-$CCM) in meridional stream function \mpsi\
(\kgxs) for 1985--1989 (a) January and (b) July.
Contour interval is $1 \times 10^{10}$~\kgxs.  
\label{fig:pres_xavg_8589_MPSI}}
\end{figure}

\begin{figure}
\begin{center}
\includegraphics*[width=\halfwidth]{/data2/zender/ps/spcp_85_8589_amip5_8589_pres_01_CHI.eps}\vfill
\includegraphics*[width=\halfwidth]{/data2/zender/ps/spcp_85_8589_amip5_8589_pres_07_CHI.eps}\vfill
\end{center}
\caption[Difference (ANV$-$CCM) in 200~mb velocity potential $\chi$
for 1985--1989 January and July]{
Difference (ANV$-$CCM) in 200~mb velocity potential $\chi$ (\mSxs)
for 1985--1989 (a) January and (b) July.
Contour interval is $1 \times 10^6$~\mSxs. 
Shading indicates less subsidence (more divergence).
\label{fig:pres_8589_CHI}}
\end{figure}
\clearpage

\begin{figure*}
\begin{center}
\includegraphics*[width=.5\hsize]{/data2/zender/ps/ecmwf_pres_9095_01_U.eps}%
\includegraphics*[width=.5\hsize]{/data2/zender/ps/ecmwf_pres_8994_07_U.eps}%

\includegraphics*[width=.5\hsize]{/data2/zender/ps/amip5_8589_ecmwf_9095_pres_01_U.eps}%
\includegraphics*[width=.5\hsize]{/data2/zender/ps/amip5_8589_ecmwf_8994_pres_07_U.eps}%

\includegraphics*[width=.5\hsize]{/data2/zender/ps/spcp_85_8589_ecmwf_9095_pres_01_U.eps}%
\includegraphics*[width=.5\hsize]{/data2/zender/ps/spcp_85_8589_ecmwf_8994_pres_07_U.eps}%
\end{center}
\caption[200~mb zonal wind $U$ for January and July ECMWF analyses,
CCM biases, and ANV biases]{ 
200~mb zonal wind $U$ (\mxs) for (left) January and (right) July (top)
ECMWF analyses, (middle) CCM biases, and (bottom) ANV biases: (a)
January ECMWF 1990--1995, (b) July ECMWF 1989-1994, (c) January
CCM$-$ECMWF, (d) July CCM$-$ECMWF, (e) January ANV$-$ECMWF, and (f)
July ANV$-$ECMWF. 
Contour intervals are (a,b) 5~\mxs\ and (c--f) 2~\mxs. 
Shading indicates (a,b) easterly zonal winds ($U < 0$) or (c--f) where
the simulation is less (more easterly) than analyzed.
\label{fig:pres_8589_U}}   
\end{figure*}
\clearpage

\begin{figure*}
\begin{center}
\includegraphics*[width=.5\hsize]{/data2/zender/ps/ecmwf_pres_9095_01_U.eps}%
\includegraphics*[width=.5\hsize]{/data2/zender/ps/ecmwf_pres_8994_07_U.eps}%

\includegraphics*[width=.5\hsize]{/data2/zender/ps/amip5_pres_8589_01_U.eps}%
\includegraphics*[width=.5\hsize]{/data2/zender/ps/amip5_pres_8589_07_U.eps}%

\includegraphics*[width=.5\hsize]{/data2/zender/ps/spcp_85_pres_8589_01_U.eps}%
\includegraphics*[width=.5\hsize]{/data2/zender/ps/spcp_85_pres_8589_07_U.eps}%
\end{center}
\caption[200~mb zonal wind $U$ for January and July ECMWF, CCM,
and ANV]{ 
200~mb zonal wind $U$ (\mxs) for (left) January and (right) July (a,b)
ECMWF, (c,d) CCM, and (e,f) ANV. 
Contour interval is 5~\mxs.
Shading indicates easterlies ($U < 0$).
\label{fig:pres_8589_U}}
\end{figure*}
\clearpage

\begin{figure*}
\begin{center}
\includegraphics*[width=.5\hsize]{/data2/zender/ps/ecmwf_pres_xavg_9095_01_MPSI.eps}%
\includegraphics*[width=.5\hsize]{/data2/zender/ps/ecmwf_pres_xavg_8994_07_MPSI.eps}%

\includegraphics*[width=.5\hsize]{/data2/zender/ps/amip5_pres_xavg_8589_01_MPSI.eps}%
\includegraphics*[width=.5\hsize]{/data2/zender/ps/amip5_pres_xavg_8589_07_MPSI.eps}%

\includegraphics*[width=.5\hsize]{/data2/zender/ps/spcp_85_pres_xavg_8589_01_MPSI.eps}%
\includegraphics*[width=.5\hsize]{/data2/zender/ps/spcp_85_pres_xavg_8589_07_MPSI.eps}%
\end{center}
\caption[Meridional stream function \mpsi\ for January and
July ECMWF, CCM, and ANV]{
Meridional stream function \mpsi\ (\kgxs) for (left) January and
(right) July (a,b) ECMWF, (c,d) CCM, and (e,f) ANV. 
Features below $\sim 800$~mb may be artifacts of orography.
Contour interval is $2 \times 10^{10}$~\kgxs.  
\label{fig:pres_xavg_8589_MPSI}}   
\end{figure*}
\clearpage

\begin{figure*}
\begin{center}
\includegraphics*[width=.5\hsize]{/data2/zender/ps/ecmwf_pres_xavg_9095_01_MPSI.eps}%
\includegraphics*[width=.5\hsize]{/data2/zender/ps/ecmwf_pres_xavg_8994_07_MPSI.eps}%

\includegraphics*[width=.5\hsize]{/data2/zender/ps/amip5_8589_ecmwf_9095_pres_xavg_01_MPSI.eps}%
\includegraphics*[width=.5\hsize]{/data2/zender/ps/amip5_8589_ecmwf_8994_pres_xavg_07_MPSI.eps}%

\includegraphics*[width=.5\hsize]{/data2/zender/ps/spcp_85_8589_ecmwf_9095_pres_xavg_01_MPSI.eps}%
\includegraphics*[width=.5\hsize]{/data2/zender/ps/spcp_85_8589_ecmwf_8994_pres_xavg_07_MPSI.eps}%
\end{center}
\caption[Meridional stream function \mpsi\ for January
and July ECMWF analyses, CCM biases, and ANV biases]{
Meridional stream function \mpsi\ (\kgxs) for (left) January
and (right) July (top) ECMWF analyses, (middle) CCM biases, and (bottom)
ANV biases: (a) January ECMWF 1990--1995, (b) July ECMWF 1989-1994,
(c) January CCM$-$ECMWF, (d) July CCM$-$ECMWF, (e) January
ANV$-$ECMWF, and (f) July ANV$-$ECMWF.
Contour intervals are (a,b) $2 \times 10^{10}$~\kgxs\ and (c--f) $1
\times 10^{10}$~\kgxs.  
Shading indicates (a,b) net southward mass transport beneath the
indicated level ($\mpsi < 0$) or (c--f) where simulations predict less
northward (more southward) mass transport beneath a given level than
observed.   
\label{fig:pres_xavg_8589_MPSI}}   
\end{figure*}
\clearpage

\begin{figure*}
\begin{center}
\includegraphics*[width=.5\hsize]{/data2/zender/ps/ecmwf_pres_9095_01_CHI.eps}%
\includegraphics*[width=.5\hsize]{/data2/zender/ps/ecmwf_pres_8994_07_CHI.eps}%

\includegraphics*[width=.5\hsize]{/data2/zender/ps/amip5_pres_8589_01_CHI.eps}%
\includegraphics*[width=.5\hsize]{/data2/zender/ps/amip5_pres_8589_07_CHI.eps}%

\includegraphics*[width=.5\hsize]{/data2/zender/ps/spcp_85_pres_8589_01_CHI.eps}%
\includegraphics*[width=.5\hsize]{/data2/zender/ps/spcp_85_pres_8589_07_CHI.eps}%
\end{center}
\caption[200~mb velocity potential $\chi$ for January and
July ECMWF, CCM, and ANV]{
200~mb velocity potential $\chi$ (\mSxs) for (left) January and
(right) July (a,b) ECMWF, (c,d) CCM, and (e,f) ANV.
Contour interval is $3 \times 10^6$~\mSxs. 
Shading indicates divergence ($\chi < 0$).
\label{fig:pres_8589_CHI}}
\end{figure*}
\clearpage

\begin{figure*}
\begin{center}
\includegraphics*[width=.5\hsize]{/data2/zender/ps/ecmwf_pres_9095_01_CHI.eps}%
\includegraphics*[width=.5\hsize]{/data2/zender/ps/ecmwf_pres_8994_07_CHI.eps}%

\includegraphics*[width=.5\hsize]{/data2/zender/ps/amip5_8589_ecmwf_9095_pres_01_CHI.eps}%
\includegraphics*[width=.5\hsize]{/data2/zender/ps/amip5_8589_ecmwf_8994_pres_07_CHI.eps}%

\includegraphics*[width=.5\hsize]{/data2/zender/ps/spcp_85_8589_ecmwf_9095_pres_01_CHI.eps}%
\includegraphics*[width=.5\hsize]{/data2/zender/ps/spcp_85_8589_ecmwf_8994_pres_07_CHI.eps}%
\end{center}
\caption[200~mb velocity potential $\chi$ for January
and July ECMWF analyses, CCM biases, and ANV biases]{
200~mb velocity potential $\chi$ (\mSxs) for (left) January
and (right) July (top) ECMWF analyses, (middle) CCM biases, and (bottom)
ANV biases: (a) January ECMWF 1990--1995, (b) July ECMWF 1989-1994,
(c) January CCM$-$ECMWF, (d) July CCM$-$ECMWF, (e) January
ANV$-$ECMWF, and (f) July ANV$-$ECMWF.
Contour intervals are (a,b) $3 \times 10^6$~\mSxs\ and (c--f) $1
\times 10^6$~\mSxs.
Shading indicates (a,b) divergence ($\chi < 0$) or (c--f) where
simulations predict less subsidence (more divergence) than observed.
\label{fig:pres_8589_CHI}}   
\end{figure*}
\clearpage

\begin{figure*}
\begin{center}
\includegraphics*[width=.5\hsize]{/data2/zender/ps/ecmwf_pres_9095_01_PSI.eps}%
\includegraphics*[width=.5\hsize]{/data2/zender/ps/ecmwf_pres_8994_07_PSI.eps}%

\includegraphics*[width=.5\hsize]{/data2/zender/ps/amip5_pres_8589_01_PSI.eps}%
\includegraphics*[width=.5\hsize]{/data2/zender/ps/amip5_pres_8589_07_PSI.eps}%

\includegraphics*[width=.5\hsize]{/data2/zender/ps/spcp_85_pres_8589_01_PSI.eps}%
\includegraphics*[width=.5\hsize]{/data2/zender/ps/spcp_85_pres_8589_07_PSI.eps}%
\end{center}
\caption[300~mb streamfunction $\psi$ for January and
July ECMWF, CCM, and ANV]{
300~mb streamfunction $\psi$ (\mSxs) for (left) January and
(right) July (a,b) ECMWF, (c,d) CCM, and (e,f) ANV.
Contour interval is $20 \times 10^6$~\mSxs. 
Shading indicates $\psi < 0$.
\label{fig:pres_8589_PSI}}
\end{figure*}
\clearpage

\begin{figure*}
\begin{center}
\includegraphics*[width=.5\hsize]{/data2/zender/ps/ecmwf_pres_9095_01_PSI.eps}%
\includegraphics*[width=.5\hsize]{/data2/zender/ps/ecmwf_pres_8994_07_PSI.eps}%

\includegraphics*[width=.5\hsize]{/data2/zender/ps/amip5_8589_ecmwf_9095_pres_01_PSI.eps}%
\includegraphics*[width=.5\hsize]{/data2/zender/ps/amip5_8589_ecmwf_8994_pres_07_PSI.eps}%

\includegraphics*[width=.5\hsize]{/data2/zender/ps/spcp_85_8589_ecmwf_9095_pres_01_PSI.eps}%
\includegraphics*[width=.5\hsize]{/data2/zender/ps/spcp_85_8589_ecmwf_8994_pres_07_PSI.eps}%
\end{center}
\caption[300~mb streamfunction $\psi$ for January
and July ECMWF analyses, CCM biases, and ANV biases]{
300~mb streamfunction $\psi$ (\mSxs) for (left) January
and (right) July (top) ECMWF analyses, (middle) CCM biases, and (bottom)
ANV biases: (a) January ECMWF 1990--1995, (b) July ECMWF 1989-1994,
(c) January CCM$-$ECMWF, (d) July CCM$-$ECMWF, (e) January
ANV$-$ECMWF, and (f) July ANV$-$ECMWF.
Contour intervals are (a,b) $20 \times 10^6$~\mSxs\ and (c--f) $2
\times 10^6$~\mSxs.
Shading indicates (a,b) $\psi < 0$ or (c--f) where
simulations predict smaller $\psi$ than observed.
\label{fig:pres_8589_PSI}}   
\end{figure*}
\clearpage

\begin{figure}
\begin{center}
\includegraphics*[width=\halfwidth]{/data2/zender/ps/spcp_85_8589_amip5_8589_pres_01_PSI.eps}\vfill
\includegraphics*[width=\halfwidth]{/data2/zender/ps/spcp_85_8589_amip5_8589_pres_07_PSI.eps}\vfill
\end{center}
\caption[Difference (ANV$-$CCM) in 300~mb streamfunction $\psi$
for 1985--1989 January and July]{
Difference (ANV$-$CCM) in 300~mb streamfunction $\psi$ (\mSxs)
for 1985--1989 (a) January and (b) July.
Contour interval is $2 \times 10^6$~\mSxs. 
Shading indicates $\ANV < \CCM$.
\label{fig:pres_8589_PSI}}
\end{figure}
\clearpage

\begin{figure}
\begin{center}
\includegraphics*[width=\halfwidth,height=.3\textheight]{/data2/zender/ps/erbe_b_anom_yavg_10S10N_8589_0160_SWCF.eps}\vfill
\includegraphics*[width=\halfwidth,height=.3\textheight]{/data2/zender/ps/amip5_anom_yavg_10S10N_8589_0160_SWCF.eps}\vfill
\includegraphics*[width=\halfwidth,height=.3\textheight]{/data2/zender/ps/spcp_85_anom_yavg_10S10N_8589_0160_SWCF.eps}\vfill
\end{center}
\caption[Hovm\"oller diagrams of shortwave cloud forcing 
anomaly in the Equatorial Pacific for 1985--1989 ERBE, CCM, and ANV]{ 
Hovm\"oller diagrams of shortwave cloud forcing anomaly (\wxmS) in the
Equatorial Pacific (averaged 10~\degreee S--10~\degreee N) for
1985--1989 (a) ERBE, (b) CCM, and (c) ANV.  
Month 1 is January 1985.  
Contour interval is 10~\wxmS. 
\label{fig:anom_yavg_10S10N_8589_0160_SWCF}}
\end{figure}

\begin{figure}
\begin{center}
\includegraphics*[width=\halfwidth,height=.3\textheight]{/data2/zender/ps/erbe_b_anom_yavg_10S10N_8589_0160_LWCF.eps}\vfill
\includegraphics*[width=\halfwidth,height=.3\textheight]{/data2/zender/ps/amip5_anom_yavg_10S10N_8589_0160_LWCF.eps}\vfill
\includegraphics*[width=\halfwidth,height=.3\textheight]{/data2/zender/ps/spcp_85_anom_yavg_10S10N_8589_0160_LWCF.eps}\vfill
\end{center}
\caption{Same as Figure~\ref{fig:anom_yavg_10S10N_8589_0160_SWCF} but
for longwave cloud forcing.
\label{fig:anom_yavg_10S10N_8589_0160_LWCF}}
\end{figure}
\clearpage

\begin{figure*}
\begin{center}
\includegraphics*[width=.33\hsize]{/data2/zender/ps/erbe_b_reg_Pacific_Equatorial_87m85_0305_LWCF_SWCF.eps}%
\includegraphics*[width=.33\hsize]{/data2/zender/ps/amip5_reg_Pacific_Equatorial_87m85_0305_LWCF_SWCF.eps}%
\includegraphics*[width=.33\hsize]{/data2/zender/ps/spcp_85_reg_Pacific_Equatorial_87m85_0305_LWCF_SWCF.eps}%
\end{center}
\caption[$1987-1985$ differences in Spring quarter (March, April, and
May) mean maritime LWCF and SWCF over the Equatorial Pacific for ERBE,
CCM, and ANV]{ 
$1987-1985$ differences in Spring quarter (March, April, and May) 
mean maritime LWCF and SWCF (\wxmS) over the Equatorial Pacific 
(10~\degreee S--10~\degreee N, 140~\degreee E--90~\degreee W)
for (a) ERBE, (b) CCM, and (c) ANV.
Solid line is least-squares fit.
\label{fig:reg_Pacific_Equatorial_87m85_0305_LWCF_SWCF}}   
\end{figure*}

\begin{figure*}
\begin{center}
\includegraphics*[width=.33\hsize]{/data2/zender/ps/erbe_b_reg_Pacific_Equatorial_87m85_0305_TS1_LWCF.eps}%
\includegraphics*[width=.33\hsize]{/data2/zender/ps/amip5_reg_Pacific_Equatorial_87m85_0305_TS1_LWCF.eps}%
\includegraphics*[width=.33\hsize]{/data2/zender/ps/spcp_85_reg_Pacific_Equatorial_87m85_0305_TS1_LWCF.eps}%
\end{center}
\caption[$1987-1985$ differences in Spring quarter (March, April, and May)
mean maritime SST and LWCF over the Equatorial Pacific for ERBE, CCM,
and ANV]{ 
$1987-1985$ differences in Spring quarter (March, April, and May)
mean maritime SST (\degreee K) and LWCF (\wxmS) over the Equatorial
Pacific  (10~\degreee S--10~\degreee N, 140~\degreee E--90~\degreee W)
for (a) ERBE, (b) CCM, and (c) ANV.
Solid line is least-squares fit.
\label{fig:reg_Pacific_Equatorial_87m85_0305_TS1_LWCF}}   
\end{figure*}

\begin{figure*}
\begin{center}
\includegraphics*[width=.33\hsize]{/data2/zender/ps/erbe_b_reg_Pacific_Equatorial_87m85_0305_TS1_SWCF.eps}%
\includegraphics*[width=.33\hsize]{/data2/zender/ps/amip5_reg_Pacific_Equatorial_87m85_0305_TS1_SWCF.eps}%
\includegraphics*[width=.33\hsize]{/data2/zender/ps/spcp_85_reg_Pacific_Equatorial_87m85_0305_TS1_SWCF.eps}%
\end{center}
\caption[$1987-1985$ differences in Spring quarter (March, April, and May)
mean maritime SST and SWCF over the Equatorial Pacific for ERBE, CCM,
and ANV]{ 
$1987-1985$ differences in Spring quarter (March, April, and May)
mean maritime SST (\degreee K) and SWCF (\wxmS) over the Equatorial
Pacific  (10~\degreee S--10~\degreee N, 140~\degreee E--90~\degreee W)
for (a) ERBE, (b) CCM, and (c) ANV.
Solid line is least-squares fit.
\label{fig:reg_Pacific_Equatorial_87m85_0305_TS1_SWCF}}   
\end{figure*}
\clearpage

\begin{figure}
\begin{center}
\includegraphics*[width=\halfwidth]{/data2/zender/ps/amip5_yavg_10S10N_87m85_MAM_QC.eps}\vfill
\includegraphics*[width=\halfwidth]{/data2/zender/ps/spcp_85_yavg_10S10N_87m85_MAM_QC.eps}\vfill
\end{center}
\caption[Longitude-height profile of the $1987-1985$ difference in
Spring quarter (March, April, and May) mean condensate \qc\ over the
maritime Equatorial Pacific simulated by CCM and ANV]{
Longitude-height profile of the $1987-1985$ difference in Spring quarter
(March, April, and May) mean condensate \qc\ (\mgxkg) over the
maritime Equatorial Pacific (averaged 10~\degreee S--10~\degreee N, 
ocean only) simulated by (a) CCM and (b) ANV.
Contour interval is 1~\mgxkg.
Shading indicates \qc\ decrease from 1985 to 1987.
\label{fig:yavg_10S10N_87m85_MAM_QC}}   
\end{figure}
\clearpage

\begin{figure}
\begin{center}
\includegraphics*[width=\halfwidth]{/data2/zender/ps/spcp_85_8589_amip5_8589_yavg_00N30N_07_QC.eps}%
\end{center}
\caption[Longitude-height cross-section of difference (ANV$-$CCM) in
simulated condensate \qc\ over the Northwest Equatorial Pacific
for ensemble (1985--1989) July]{
Longitude-height cross-section of difference (ANV$-$CCM) in
simulated condensate \qc\ (\mgxkg) over the Northwest Equatorial Pacific
for ensemble (1985--1989) July. 
Data are averaged from 0--30~$\degreee$N.
%NB: this figure currently includes land points.
Contour interval is 1~\mgxkg.
Shading indicates ANV has less \qc\ than CCM.
\label{fig:spcp_85_8589_amip5_8589_yavg_00N30N_07_QC}}   
\end{figure}
\clearpage

\begin{figure}
\begin{center}
\includegraphics*[width=\halfwidth]{/data2/zender/ps/amip5_xyavg_reg_Indian_Central_8589_0112_QC.eps}\vfill
\includegraphics*[width=\halfwidth]{/data2/zender/ps/spcp_85_xyavg_reg_Indian_Central_8589_0112_QC.eps}\vfill
\end{center}
\caption[Seasonal cycle of simulated condensate \qc\ over the central
Indian Ocean for 1985--1989]{
Seasonal cycle simulated condensate \qc\ (\mgxkg) over the central
Indian Ocean (15~\degreee S--5~\degreee N, 60--80~\degreee E) for
1985--1989 (a) CCM and (b) ANV.  
Contour interval is 2~\mgxkg.
Shading indicates $\qc > 6$~\mgxkg.
\label{fig:reg_xyavg_Indian_Central_0112_QC}}   
\end{figure}

\begin{figure}
\begin{center}
\includegraphics*[width=\halfwidth]{/data2/zender/ps/xyavg_reg_Indian_Central_8589_0112_SWCF.eps}\vfill
\includegraphics*[width=\halfwidth]{/data2/zender/ps/xyavg_reg_Indian_Central_8589_0112_LWCF.eps}\vfill
\end{center}
\caption[Seasonal amplitude in regional shortwave and longwave
cloud forcing over the maritime Central Indian Ocean
for ERBE, CCM, and ANV]{
Seasonal amplitude in regional (a) shortwave and (b) longwave
cloud forcing (\wxmS ) over the maritime Central Indian Ocean
(15~\degreee S--5~\degreee N,60--80~\degreee E) for ERBE (solid), CCM
(dotted), and ANV (dashed).
\label{fig:xyavg_reg_8589_0112_ocean_CF}}
\end{figure}

\begin{figure}
\begin{center}
\includegraphics*[width=\halfwidth,height=.12\textheight]{/data2/zender/ps/xyavg_reg_Indian_Central_8589_0112_FSNS.eps}\vfill
\includegraphics*[width=\halfwidth,height=.10\textheight]{/data2/zender/ps/xyavg_reg_Indian_Central_8589_0112_TS1.eps}\vfill
\includegraphics*[width=\halfwidth,height=.10\textheight]{/data2/zender/ps/xyavg_reg_Indian_Central_8589_0112_FLNS.eps}\vfill
\includegraphics*[width=\halfwidth,height=.10\textheight]{/data2/zender/ps/xyavg_reg_Indian_Central_8589_0112_VMAGSFC.eps}\vfill
\includegraphics*[width=\halfwidth,height=.10\textheight]{/data2/zender/ps/xyavg_reg_Indian_Central_8589_0112_SHFLX.eps}\vfill
\includegraphics*[width=\halfwidth,height=.10\textheight]{/data2/zender/ps/xyavg_reg_Indian_Central_8589_0112_LHFLX.eps}\vfill
\includegraphics*[width=\halfwidth,height=.10\textheight]{/data2/zender/ps/xyavg_reg_Indian_Central_8589_0112_E_P.eps}\vfill
\includegraphics*[width=\halfwidth,height=.12\textheight]{/data2/zender/ps/xyavg_reg_Indian_Central_8589_0112_NET.eps}\vfill
\end{center}
\caption[Simulated seasonal cycle of Central Indian Ocean 
maritime surface energy budget (SEB) components from CCM and ANV]{ 
Simulated seasonal cycle of Central Indian Ocean 
maritime surface energy budget (SEB) components (a) net shortwave
flux, (b) net longwave flux, (c) sensible heat flux, (d) latent heat
flux, (e) evaporation$-$precipitation, and (f) net SEB from CCM
(solid), and ANV (dashed).  
Positive fluxes heat the surface.
Units are \wxmS\ except E$-$P, which is \mmxday.
\label{fig:xyavg_reg_Indian_Central_8589_0112_SEB}}   
\end{figure}
\clearpage

\begin{figure*}
\begin{center}
\includegraphics*[width=.5\hsize]{/data2/zender/ps/amip5_xyavg_reg_Indian_Central_8589_01.eps}%
\includegraphics*[width=.5\hsize]{/data2/zender/ps/spcp_85_8589_amip5_8589_xyavg_reg_Indian_Central_01.eps}%

\includegraphics*[width=.5\hsize]{/data2/zender/ps/amip5_xyavg_reg_Indian_Central_8589_07.eps}%
\includegraphics*[width=.5\hsize]{/data2/zender/ps/spcp_85_8589_amip5_8589_xyavg_reg_Indian_Central_07.eps}%
\end{center}
\caption[Model simulated profiles of maritime diabatic heating
and differences between models (ANV$-$CCM) for the
Central Indian Ocean from 1985--1989]{
Model simulated profiles of maritime diabatic heating
(\kxday), and differences between models (ANV$-$CCM) for the
Central Indian Ocean (15~\degreee S--5~\degreee
N,60--80~\degreee E) from 1985--1989: (a) January CCM, (b) January
ANV$-$CCM, (c) July 
CCM, and  (d) July ANV$-$CCM. 
Heatings shown are total diabatic (solid), shortwave (dotted),
longwave (short dash), resolved (dash-dot), turbulent
(dash-dot-dot-dot), and convective (long dash).
Note difference in scales.
\label{fig:xyavg_reg_Indian_Central_8589}}
\end{figure*}
\clearpage

\begin{figure}
\begin{center}
\includegraphics*[width=\halfwidth]{/data2/zender/ps/xyavg_reg_Pacific_ITCZ_West_8589_0112_SWCF.eps}\vfill
\includegraphics*[width=\halfwidth]{/data2/zender/ps/xyavg_reg_Pacific_ITCZ_West_8589_0112_LWCF.eps}\vfill
\end{center}
\caption[Seasonal amplitude in regional shortwave and longwave
cloud forcing over the maritime West Pacific ITCZ
for ERBE, CCM, and ANV]{
Seasonal amplitude in regional (a) shortwave and (b) longwave
cloud forcing (\wxmS ) over the maritime West Pacific ITCZ
(0--15~\degreee N,140--200~\degreee E) for ERBE (solid), CCM
(dotted), and ANV (dashed).
\label{fig:xyavg_reg_8589_0112_ocean_CF}}
\end{figure}

\begin{figure}
\begin{center}
\includegraphics*[width=\halfwidth,height=.135\textheight]{/data2/zender/ps/xyavg_reg_Pacific_ITCZ_West_8589_0112_FSNS.eps}\vfill
\includegraphics*[width=\halfwidth,height=.12\textheight]{/data2/zender/ps/xyavg_reg_Pacific_ITCZ_West_8589_0112_FLNS.eps}\vfill
\includegraphics*[width=\halfwidth,height=.12\textheight]{/data2/zender/ps/xyavg_reg_Pacific_ITCZ_West_8589_0112_SHFLX.eps}\vfill
\includegraphics*[width=\halfwidth,height=.12\textheight]{/data2/zender/ps/xyavg_reg_Pacific_ITCZ_West_8589_0112_LHFLX.eps}\vfill
\includegraphics*[width=\halfwidth,height=.12\textheight]{/data2/zender/ps/xyavg_reg_Pacific_ITCZ_West_8589_0112_E_P.eps}\vfill
\includegraphics*[width=\halfwidth,height=.135\textheight]{/data2/zender/ps/xyavg_reg_Pacific_ITCZ_West_8589_0112_NET.eps}\vfill
\end{center}
\caption[Simulated seasonal cycle of West Pacific ITCZ 
maritime surface energy budget (SEB) components from CCM and ANV]{ 
Simulated seasonal cycle of West Pacific ITCZ 
maritime surface energy budget (SEB) components (a) net shortwave
flux, (b) net longwave flux, (c) sensible heat flux, (d) latent heat
flux, (e) evaporation$-$precipitation, and (f) net SEB from CCM
(solid), and ANV (dashed).  
Positive fluxes heat the surface.
Units are \wxmS\ except E$-$P, which is \mmxday.
\label{fig:xyavg_reg_Pacific_ITCZ_West_8589_0112_SEB}}   
\end{figure}
\clearpage

\begin{figure*}
\begin{center}
\includegraphics*[width=.5\hsize]{/data2/zender/ps/amip5_xyavg_reg_Pacific_ITCZ_West_8589_01.eps}%
\includegraphics*[width=.5\hsize]{/data2/zender/ps/spcp_85_8589_amip5_8589_xyavg_reg_Pacific_ITCZ_West_01.eps}%

\includegraphics*[width=.5\hsize]{/data2/zender/ps/amip5_xyavg_reg_Pacific_ITCZ_West_8589_07.eps}%
\includegraphics*[width=.5\hsize]{/data2/zender/ps/spcp_85_8589_amip5_8589_xyavg_reg_Pacific_ITCZ_West_07.eps}%
\end{center}
\caption[Model simulated profiles of maritime diabatic heating
and differences between models (ANV$-$CCM) for the
West Pacific ITCZ from 1985--1989]{
Model simulated profiles of maritime diabatic heating
(\kxday), and differences between models (ANV$-$CCM) for the
West Pacific ITCZ (0--15~\degreee N,140--200~\degreee E)
from 1985--1989: (a) January CCM, (b) January
ANV$-$CCM, (c) July 
CCM, and  (d) July ANV$-$CCM. 
Heatings shown are total diabatic (solid), shortwave (dotted),
longwave (short dash), resolved (dash-dot), turbulent
(dash-dot-dot-dot), and convective (long dash).
Note difference in scales.
\label{fig:xyavg_reg_Pacific_ITCZ_West_8589}}
\end{figure*}
\clearpage

% bibliography
\bibliographystyle{agu} 
\bibliography{/home/zender/tex/bib}

\end{document}


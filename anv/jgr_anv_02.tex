% $Id$

% Purpose: Revision of ZeK97 for JGR

\documentclass[12pt,twoside]{article}
\usepackage{graphicx} % defines \includegraphics*
\usepackage{natbib} % \cite commands from aguplus
\usepackage{graphicx} % defines \includegraphics*
\usepackage{ifthen} % Boolean and programming commands

\usepackage{/home/zender/tex/csz} % all my local definitions
% Usage: % Usage: % Usage: \input{jgr_abb} % AGU-sanctioned journal title abbreviations

\def\aapgb{{\it Amer. Assoc. Petroleum Geologists Bull.}}
\def\adg{{\it Adv. Geophys.}}
\def\ajs{{\it Amer. J. Sci.}}
\def\amb{{\it Ambio}}
\def\amgb{{\it Arch. Meteorol. Geophys. Bioclimatl.}}
\def\ang{{\it Ann. Glaciol.}}
\def\angeo{{\it Ann. Geophys.}}
\def\apo{{\it Appl. Opt.}}
\def\areps{{\it Ann. Rev. Earth Planet. Sci.}}
\def\asr{{\it Adv. Space Res.}}
\def\ate{{\it Atmos. Environ.}}
\def\atf{{\it Atmosfera}}
\def\atms{{\it ACM Trans. Math Software}}
\def\ato{{\it Atmos. Ocean}}
\def\atr{{\it Atmos. Res.}}
\def\gbc{{\it Global Biogeochem. Cycles}} % csz
\def\blm{{\it Boundary-Layer Meteorol.}} % csz 
\def\bpa{{\it Beitr. Phys. Atmosph.}}
\def\bams{{\it Bull. Am. Meteorol. Soc.}}
\def\clc{{\it Clim. Change}}
\def\cld{{\it Clim. Dyn.}}
\def\com{{\it Computing}}
\def\dao{{\it Dyn. Atmos. Oceans}}
\def\dsr{{\it Deep-Sea Res.}}
\def\esr{{\it Earth Sci. Revs.}}
\def\gec{{\it Geosci. Canada}}
\def\gei{{\it Geofis. Int.}}
\def\gej{{\it Geogr. J.}}
\def\gem{{\it Geophys. Monogr.}}
\def\geo{{\it Geology}}
\def\grl{{\it Geophys. Res. Lett.}}
\def\ieeec{{\it IEEE Computer}}
\def\ijna{{\it IMA J. Numer. Anal.}}
\def\ijnmf{{\it Int. J. Num. Meteorol. Fl.}}
\def\jac{{\it J. Atmos. Chem.}}
\def\jacm{{\it J. Assoc. Comput. Mach.}}
\def\jam{{\it J. Appl. Meteorol.}}
\def\jas{{\it J. Atmos. Sci.}}
\def\jatp{{\it J. Atmos. Terr. Phys.}}
\def\jcam{{\it J. Climate Appl. Meteorol.}}
\def\jchp{{\it J. Chem Phys.}}
\def\jcis{{\it J. Coll. I. Sci.}}
\def\jcl{{\it J. Clim.}}
\def\jcp{{\it J. Comput. Phys.}}
\def\jfm{{\it J. Fluid Mech.}}
\def\jgl{{\it J. Glaciol.}}
\def\jgr{{\it J. Geophys. Res.}}
\def\jgs{{\it J. Geol. Soc. London}}
\def\jme{{\it J. Meteorol.}}
\def\jmr{{\it J. Marine Res.}}
\def\jmsj{{\it J. Meteorol. Soc. Jpn.}}
\def\josa{{\it J. Opt. Soc. A}}
\def\jpo{{\it J. Phys. Oceanogr.}}
\def\jqsrt{{\it J. Quant. Spectrosc. Radiat. Transfer}}
\def\jpca{{\it J. Phys. Chem. A}}
\def\lnc{{\it Lett. Nuov. C}}
\def\mac{{\it Math. Comp.}}
\def\map{{\it Meteorol. Atmos. Physics.}}
\def\mem{{\it Meteorol. Mag.}}
\def\mnras{{\it Mon. Not. Roy. Astron. Soc.}} 
\def\mwr{{\it Mon. Weather Rev.}} 
\def\nat{{\it Nature}}
\def\pac{{\it Parallel Computing}}
\def\pag{{\it Pure Appl. Geophys.}}
\def\pal{{\it Paleoceanography}}
\def\pht{{\it Physics Today}}
\def\pieee{{\it Proc. IEEE}}
\def\pla{{\it Phys. Lett. A}}
\def\ppp{{\it Paleogeogr. Paleoclim. Paleoecol.}}
\def\pra{{\it Phys. Res. A}}
\def\prd{{\it Phys. Rev. D}}
\def\prl{{\it Phys. Rev. L}}
\def\pss{{\it Planet. Space Sci.}}
\def\ptrsl{{\it Phil. Trans. R. Soc. Lond.}}
\def\qjrms{{\it Q. J. R. Meteorol. Soc.}}
\def\qres{{\it Quat. Res.}}
\def\qsr{{\it Quatern. Sci. Rev.}}
\def\reg{{\it Rev. Geophys.}}
\def\rgsp{{\it Revs. Geophys. Space Phys.}}
\def\rpp{{\it Rep. Prog. Phys.}}
\def\sca{{\it Sci. Amer.}}
\def\sci{{\it Science}}
\def\sjna{{\it SIAM J. Numer. Anal.}}
\def\sjssc{{\it SIAM J. Sci. Stat. Comput.}}
\def\tac{{\it Theor. Appl. Climatl.}}
\def\tel{{\it Tellus}}
\def\wea{{\it Weather}}

%SIAM Review: (Society for Industrial and Applied Mathematics)
%       J. on Computing
%       J. on Control and Optimization
%       J. on Algebraic and Discrete Methods
%       J. on Numerical Analysis 
%       J. on Scientific and Statistical Computing



 % AGU-sanctioned journal title abbreviations

\def\aapgb{{\it Amer. Assoc. Petroleum Geologists Bull.}}
\def\adg{{\it Adv. Geophys.}}
\def\ajs{{\it Amer. J. Sci.}}
\def\amb{{\it Ambio}}
\def\amgb{{\it Arch. Meteorol. Geophys. Bioclimatl.}}
\def\ang{{\it Ann. Glaciol.}}
\def\angeo{{\it Ann. Geophys.}}
\def\apo{{\it Appl. Opt.}}
\def\areps{{\it Ann. Rev. Earth Planet. Sci.}}
\def\asr{{\it Adv. Space Res.}}
\def\ate{{\it Atmos. Environ.}}
\def\atf{{\it Atmosfera}}
\def\atms{{\it ACM Trans. Math Software}}
\def\ato{{\it Atmos. Ocean}}
\def\atr{{\it Atmos. Res.}}
\def\gbc{{\it Global Biogeochem. Cycles}} % csz
\def\blm{{\it Boundary-Layer Meteorol.}} % csz 
\def\bpa{{\it Beitr. Phys. Atmosph.}}
\def\bams{{\it Bull. Am. Meteorol. Soc.}}
\def\clc{{\it Clim. Change}}
\def\cld{{\it Clim. Dyn.}}
\def\com{{\it Computing}}
\def\dao{{\it Dyn. Atmos. Oceans}}
\def\dsr{{\it Deep-Sea Res.}}
\def\esr{{\it Earth Sci. Revs.}}
\def\gec{{\it Geosci. Canada}}
\def\gei{{\it Geofis. Int.}}
\def\gej{{\it Geogr. J.}}
\def\gem{{\it Geophys. Monogr.}}
\def\geo{{\it Geology}}
\def\grl{{\it Geophys. Res. Lett.}}
\def\ieeec{{\it IEEE Computer}}
\def\ijna{{\it IMA J. Numer. Anal.}}
\def\ijnmf{{\it Int. J. Num. Meteorol. Fl.}}
\def\jac{{\it J. Atmos. Chem.}}
\def\jacm{{\it J. Assoc. Comput. Mach.}}
\def\jam{{\it J. Appl. Meteorol.}}
\def\jas{{\it J. Atmos. Sci.}}
\def\jatp{{\it J. Atmos. Terr. Phys.}}
\def\jcam{{\it J. Climate Appl. Meteorol.}}
\def\jchp{{\it J. Chem Phys.}}
\def\jcis{{\it J. Coll. I. Sci.}}
\def\jcl{{\it J. Clim.}}
\def\jcp{{\it J. Comput. Phys.}}
\def\jfm{{\it J. Fluid Mech.}}
\def\jgl{{\it J. Glaciol.}}
\def\jgr{{\it J. Geophys. Res.}}
\def\jgs{{\it J. Geol. Soc. London}}
\def\jme{{\it J. Meteorol.}}
\def\jmr{{\it J. Marine Res.}}
\def\jmsj{{\it J. Meteorol. Soc. Jpn.}}
\def\josa{{\it J. Opt. Soc. A}}
\def\jpo{{\it J. Phys. Oceanogr.}}
\def\jqsrt{{\it J. Quant. Spectrosc. Radiat. Transfer}}
\def\jpca{{\it J. Phys. Chem. A}}
\def\lnc{{\it Lett. Nuov. C}}
\def\mac{{\it Math. Comp.}}
\def\map{{\it Meteorol. Atmos. Physics.}}
\def\mem{{\it Meteorol. Mag.}}
\def\mnras{{\it Mon. Not. Roy. Astron. Soc.}} 
\def\mwr{{\it Mon. Weather Rev.}} 
\def\nat{{\it Nature}}
\def\pac{{\it Parallel Computing}}
\def\pag{{\it Pure Appl. Geophys.}}
\def\pal{{\it Paleoceanography}}
\def\pht{{\it Physics Today}}
\def\pieee{{\it Proc. IEEE}}
\def\pla{{\it Phys. Lett. A}}
\def\ppp{{\it Paleogeogr. Paleoclim. Paleoecol.}}
\def\pra{{\it Phys. Res. A}}
\def\prd{{\it Phys. Rev. D}}
\def\prl{{\it Phys. Rev. L}}
\def\pss{{\it Planet. Space Sci.}}
\def\ptrsl{{\it Phil. Trans. R. Soc. Lond.}}
\def\qjrms{{\it Q. J. R. Meteorol. Soc.}}
\def\qres{{\it Quat. Res.}}
\def\qsr{{\it Quatern. Sci. Rev.}}
\def\reg{{\it Rev. Geophys.}}
\def\rgsp{{\it Revs. Geophys. Space Phys.}}
\def\rpp{{\it Rep. Prog. Phys.}}
\def\sca{{\it Sci. Amer.}}
\def\sci{{\it Science}}
\def\sjna{{\it SIAM J. Numer. Anal.}}
\def\sjssc{{\it SIAM J. Sci. Stat. Comput.}}
\def\tac{{\it Theor. Appl. Climatl.}}
\def\tel{{\it Tellus}}
\def\wea{{\it Weather}}

%SIAM Review: (Society for Industrial and Applied Mathematics)
%       J. on Computing
%       J. on Control and Optimization
%       J. on Algebraic and Discrete Methods
%       J. on Numerical Analysis 
%       J. on Scientific and Statistical Computing



 % AGU-sanctioned journal title abbreviations

\def\aapgb{{\it Amer. Assoc. Petroleum Geologists Bull.}}
\def\adg{{\it Adv. Geophys.}}
\def\ajs{{\it Amer. J. Sci.}}
\def\amb{{\it Ambio}}
\def\amgb{{\it Arch. Meteorol. Geophys. Bioclimatl.}}
\def\ang{{\it Ann. Glaciol.}}
\def\angeo{{\it Ann. Geophys.}}
\def\apo{{\it Appl. Opt.}}
\def\areps{{\it Ann. Rev. Earth Planet. Sci.}}
\def\asr{{\it Adv. Space Res.}}
\def\ate{{\it Atmos. Environ.}}
\def\atf{{\it Atmosfera}}
\def\atms{{\it ACM Trans. Math Software}}
\def\ato{{\it Atmos. Ocean}}
\def\atr{{\it Atmos. Res.}}
\def\gbc{{\it Global Biogeochem. Cycles}} % csz
\def\blm{{\it Boundary-Layer Meteorol.}} % csz 
\def\bpa{{\it Beitr. Phys. Atmosph.}}
\def\bams{{\it Bull. Am. Meteorol. Soc.}}
\def\clc{{\it Clim. Change}}
\def\cld{{\it Clim. Dyn.}}
\def\com{{\it Computing}}
\def\dao{{\it Dyn. Atmos. Oceans}}
\def\dsr{{\it Deep-Sea Res.}}
\def\esr{{\it Earth Sci. Revs.}}
\def\gec{{\it Geosci. Canada}}
\def\gei{{\it Geofis. Int.}}
\def\gej{{\it Geogr. J.}}
\def\gem{{\it Geophys. Monogr.}}
\def\geo{{\it Geology}}
\def\grl{{\it Geophys. Res. Lett.}}
\def\ieeec{{\it IEEE Computer}}
\def\ijna{{\it IMA J. Numer. Anal.}}
\def\ijnmf{{\it Int. J. Num. Meteorol. Fl.}}
\def\jac{{\it J. Atmos. Chem.}}
\def\jacm{{\it J. Assoc. Comput. Mach.}}
\def\jam{{\it J. Appl. Meteorol.}}
\def\jas{{\it J. Atmos. Sci.}}
\def\jatp{{\it J. Atmos. Terr. Phys.}}
\def\jcam{{\it J. Climate Appl. Meteorol.}}
\def\jchp{{\it J. Chem Phys.}}
\def\jcis{{\it J. Coll. I. Sci.}}
\def\jcl{{\it J. Clim.}}
\def\jcp{{\it J. Comput. Phys.}}
\def\jfm{{\it J. Fluid Mech.}}
\def\jgl{{\it J. Glaciol.}}
\def\jgr{{\it J. Geophys. Res.}}
\def\jgs{{\it J. Geol. Soc. London}}
\def\jme{{\it J. Meteorol.}}
\def\jmr{{\it J. Marine Res.}}
\def\jmsj{{\it J. Meteorol. Soc. Jpn.}}
\def\josa{{\it J. Opt. Soc. A}}
\def\jpo{{\it J. Phys. Oceanogr.}}
\def\jqsrt{{\it J. Quant. Spectrosc. Radiat. Transfer}}
\def\jpca{{\it J. Phys. Chem. A}}
\def\lnc{{\it Lett. Nuov. C}}
\def\mac{{\it Math. Comp.}}
\def\map{{\it Meteorol. Atmos. Physics.}}
\def\mem{{\it Meteorol. Mag.}}
\def\mnras{{\it Mon. Not. Roy. Astron. Soc.}} 
\def\mwr{{\it Mon. Weather Rev.}} 
\def\nat{{\it Nature}}
\def\pac{{\it Parallel Computing}}
\def\pag{{\it Pure Appl. Geophys.}}
\def\pal{{\it Paleoceanography}}
\def\pht{{\it Physics Today}}
\def\pieee{{\it Proc. IEEE}}
\def\pla{{\it Phys. Lett. A}}
\def\ppp{{\it Paleogeogr. Paleoclim. Paleoecol.}}
\def\pra{{\it Phys. Res. A}}
\def\prd{{\it Phys. Rev. D}}
\def\prl{{\it Phys. Rev. L}}
\def\pss{{\it Planet. Space Sci.}}
\def\ptrsl{{\it Phil. Trans. R. Soc. Lond.}}
\def\qjrms{{\it Q. J. R. Meteorol. Soc.}}
\def\qres{{\it Quat. Res.}}
\def\qsr{{\it Quatern. Sci. Rev.}}
\def\reg{{\it Rev. Geophys.}}
\def\rgsp{{\it Revs. Geophys. Space Phys.}}
\def\rpp{{\it Rep. Prog. Phys.}}
\def\sca{{\it Sci. Amer.}}
\def\sci{{\it Science}}
\def\sjna{{\it SIAM J. Numer. Anal.}}
\def\sjssc{{\it SIAM J. Sci. Stat. Comput.}}
\def\tac{{\it Theor. Appl. Climatl.}}
\def\tel{{\it Tellus}}
\def\wea{{\it Weather}}

%SIAM Review: (Society for Industrial and Applied Mathematics)
%       J. on Computing
%       J. on Control and Optimization
%       J. on Algebraic and Discrete Methods
%       J. on Numerical Analysis 
%       J. on Scientific and Statistical Computing





\begin{document}
 
\begin{center}\normalsize
Revisions to Manuscript JGRd-96152, formerly \\
``Tropical Climate Sensitivity to Representation of Cirrus
Anvil Lifecycle'', \\
now titled \\
``Sensitivity of
climate simulations to radiative effects of tropical anvil structure''
\end{center}
\bigskip
\begin{center}\normalsize
Authors: C.~S.~Zender and J.~T.~Kiehl
\end{center}\normalsize
\bigskip
Date: \today
\bigskip\normalsize

\section{Response to Reviewer A}

\begin{enumerate}

% 1.
\item Most of the terms in (1) do not vary in the vertical.
The revised manuscript states this more explicitly in the text beneath
(1). 
The reviewer may overestimate the complexity of the parameterization. 
Parameters $c_1$--$c_3$ do not vary in time or space---they are the
free parameters of the parameterization.
Their values are quoted just before the beginning of Section~2.1.
\Mc\ is the convective mass flux in the lowest convecting layer above
500~mb, so it is fixed for a given column.  
\qi, $\rho$, and the wind vector $\vec u$ (which is hidden in the
material derivative) are the only variables which are functions of
height in (1). 
The vertical variation of \qi\ is defined by (1) and discussed in 
the paragraph following (1), and in Sections~2.3 (new) and~3.1.

The $c_4$ label in Figure~1 was inadvertently included and has been
removed from the revised manuscript.  
In the experiment, latent heating (including evaporation of
precipitation) operates on the condensation and precipitation
diagnosed by the original CCM routines, not on the prognostic \qi,
which is felt through its radiative effects only. 
A completely prognostic liquid water scheme would be needed in
addition to the ice water scheme in order to fully couple the latent
heat of \qi\ to the system.
The manuscript describes a prognostic ice water scheme only, and
thus focuses on the radiative effects of the anvil condensate.
This is now explicitly stated in the first paragraph of Section~3 and
emphasized in the new title.

%In the manuscript, the latent heating of anvil formation is determined
%by the deep convection scheme in both the control and the experiment
%\cite[]{Hac94}. 

% 2.
\item The disputed claim on p.~5 has been removed.
However, we are concerned the reviewer feels the justification for the
ice parameterization is \textit{weak} whereas we feel it is
conceptually \textit{simple} (but powerful).  
So allow us to recapitulate the justification:
The fundamental idea of the prognostic parameterization is that the
production of ice in anvil is, to first order, proportional to the
convective mass flux at anvil base.  
The application of this idea in a GCM is, to our knowledge, new
although the idea itself is not.
For example, the observational study of \cite{MaR93} concludes
``\ldots [our findings and others] suggest that the magnitude of the
convective updraft and mass (and water) flux at the base of the Cb may
determine the average vertical and horizontal extent and mean cloud
properties of all CS [tropical high cloud systems].''
The first paragraph of Section~2.1 cites this and the similar findings
of the modeling study of \cite{XuK91} (who state the \textit{local}
convective mass flux is the best predictor of \textit{local} anvil
mass, a point we return to in the next item). 
Finally, Figure~2 shows the results of a CEM simulation intended to
qualitatively demonstrate the modeled relationship between \Mc\ and
\IWPdot.

% 3.
\item Reviewer~A raises many points in revision~3, we address them in
order.

\begin{enumerate}
\item To incorporate a better sense of the wide spectrum of anvil 
behavior, we now use ranges instead of single values to describe the 
lifetime and relative core area estimates of a typical anvil.
The revised manuscript says:
``Anvil lifetime, typically 6--12~hr \cite[]{ALV88,LeH80} \ldots'' and
``The relative area of convecting cores to the associated anvil is
10--20\% \cite[]{FDR90,LeH80}.''

\item Reviewer~A states ``\ldots the vertical structure of the
condensate is discussed with the statement that model and observations
show little variation or increase from the freezing level to
300~mb. Vertical variation of this type is not indicated by
observations but only by models \ldots''.
Further, Reviewer~A implies time-mean \qi\ in MCSs significantly
decreases above anvil base both in reality and in model studies.
We have reviewed many studies and think Reviewer~A will agree the
balance of evidence supports the new phrasing ``\ldots \qc\ does not 
decrease significantly (but can increase) from the freezing level
\ldots''.     
The studies which definitively bear on this subject are cited in the
revised manuscript, and we summarize their results in greater detail
here so that Reviewer~A may weigh the evidence:

First, the EMEX anvils simulated by runs 7 and~8 in Figure~3 of
\cite{TLS96} do not appear to us to contradict our statement.
However, the figure is not time or domain averaged so it is difficult
to tell.
We note that $\IWC = \rho \qi$ so that \qi\ increases with height as
long as IWC decreases with height slower than $\rho$.
A GCM requires the prediction of \qi\ on large time and space scales.
Anvil \qi\ contributes to the time average from a wide region until
the anvil dissipates whereas convective region \qi\ contributes to the
time average over a smaller area and for a shorter period.   
This is evident in Figure~13 of \cite{GMK96}, where \qi\ increases by
roughly 100\% between freezing level and 275~mb.
Figure~13 of \cite{SLT94} shows \qi\ decreasing by only $\sim 20\%$
from freezing level to 300~mb. 

Direct observations of the vertical profile of \qi\ (or IWC) in
tropical anvils are scarce.
\cite{WSS931} Figure~6 shows radar reflectivity of an anvil and
Figure~7b shows LWC (not IWC) as a function of height. 
However, in modeling the same system, \cite{WSS932} claim their
Figures~2 and~3 agree with radar measurements and are consistent with
typical MCS structure proposed by \cite{Hou89}.  
Observational data from CEPEX also supports the paradigm of slowly
varying \qi\ up to 300~mb.
When converted from IWC to \qi, \cite{McH96} Figure~6 shows \qi\
increases slowly from 8--10~km (400--300~mb)\footnote{The 8-10~km
flight occurred later in the lifecycle of the anvil.}, decreases
slowly from 10--12~km (300-220~mb), and decreases quickly above
12~km. 
\cite{McH971} Figure~1 shows the vertical structure of IWC averaged
over all anvils sampled in CEPEX as a function of temperature. 
There is a very wide spread in the measurements, but the median \qi\
appears relatively constant from $-10 < T < -40$~C (450--250~mb). 
The relatively small sample size encompassed by these observations
cannot be used to construct a climatological value. 

\item The reviewer correctly notes the importance of the difference 
between the parameterization, in which apportions new IWP
($c_1\Mc\Delta t$) evenly among all convecting layers where $T <
0~^\circ$C, and reality, in which local anvil condensate is 
produced or detrained by the local convective mass flux.
In developing ANV, we performed many experiments where the vertical
distribution of the predicted condensate was weighted by local mass
flux $M$ (which is close to what the reviewer suggests). 
The upper level convective mass fluxes were very weak compared to the
anvil base mass fluxes and so the anvils were ``bottom heavy'' and
lacked enough high-level condensate to produce realistic LWCF. 
The deep convection scheme employed in CCM2 produces weaker upper
level convection than most competing schemes \cite[]{Hac94,MRP95}, and
GCMs which employ different deep convection schemes might well produce
realistic climates using the method suggested by the reviewer. 
Finally, we emphasize the observational study of \cite{MaR93} (quoted
in item~2 above) and Figure~2 provide qualitative motivation for our
procedure. 

%Other methods of distributing IWP nonuniformly in the vertical are
%expected to become more practical as data from field observations,
%especially in the tropics, accumulates.
%For example, new IWP could be vertically partitioned based on observed
%IWC profiles.

\item As indicated in (1) the vertical distribution of condensate by
using \Mc\ instead of local mass flux $M$.
The total anvil condensate generated in one timestep is $\Delta \IWP =
c_1 \Mc \Delta t$.
The fraction of $\Delta \IWP$ which is apportioned into a model layer
of thickness $\Delta z$ is $\Delta z/\Delta Z$ where $\Delta Z$ is the
thickness of the portion of the total column that is both freezing and
convectively active.

\end{enumerate}

% 4.
\item Done.
% The stats in Hou89, FDR90, and Mar93 show that anvil cloud fraction
% approaches 1 when GCM horizontal resolution approaches $100^2$~\kmS.

% 5.
\item At the reviewer's request we have added a panel to Figure~2
which shows the variation of \Mc\ and \IWPdot\ for the first day of
the CEM simulation.
We elected to show the timeseries rather than a scatterplot because
we feel the former is more illustrative of the mesoscale physics
occuring in the domain.
As stated in the manuscript the excellent correlation between
\Mc\ and \IWPdot\ in the CEM simulation only lasts for the first
2.5~hr of the simulation, shown in Figure~2b.
As is clear from Figure~2a and the text, the correlation vanishes for
longer time periods (the linear correlation coefficient is 0.24 for
the first day of the simulation). 

%During this time there is only one convective core in the domain, and 
%condensate removal processes (precipitation and evaporation) are
%negligible compared to production processes (convective detrainment
%and stratiform condensation). 
%Thus the slope of the correlation presented in Figure~2 represents
%the partial derivative of IWP with respect to \Mc, i.e., $c_1$.
%Once significant precipitation, evaporation, or convection elsewhere
%in the CEM domain commences, a plot like Figure~2 would show the total
%derivative of IWP with respect to \Mc and thus be unsuitable for
%estimating $c_1$. 
%Often \Mc\ is positive when \IWPdot\ is negative! 
%This always occurs net anvil dissipation exceeds production.
%It also occurs when the decay of a mature anvil in one region of the
%domain exceeds the formation of anvil in another.
%(1) is capable of parameterizing these circumstances as long as $c_1$
%represents the partial derivative of IWP with respect to \Mc, which is
%always positive and qualitatively shown in Figure~2.

% 6.
\item In terms of the approximation suggested by the reviewer,
$$
{\IWPdot \over \Delta Z} = {\qi \Mc \over \Delta Z} +
\mathrm{Mesoscale~generation}
$$
we intend $c_1$ to represent the entire RHS of the equation.
As suggested, we computed the mesoscale condensate source as the
residual between the actual anvil mass change \IWPdot\ and the
convective source predicted by the product $\qi\Mc$.
We employed this method on the hour of anvil formation ending at
$t = 150$~min.
The contribution of the inferred mesoscale production to the 
total anvil generation averaged 48\%, with a range from 0--95\%.  
The wide range is presumably due to the approximation.  
\cite{LeH80} and \cite{GaH83} estimated mesoscale processes
contribute 25--40\% of anvil condensate. 
Unfortunately, the data archived from the CEM simulation lacks
the necessary information\footnote{The data required are separate
physics tendencies, or rates, for all condensate source and sink
mechanisms, i.e., condensation, evaporation, removal.
Only absolute mixing ratios, not their tendencies, were archived.} 
to accurately determine the relative strengths of convective vs.\
mesoscale source and sink processes, i.e., to perform the same
rigorous budget analysis as \cite{LeH80} or \cite{SLT94}.
Since we are unable to perform the more rigorous budget analysis 
on the CEM simulation we have opted to not include the above estimates
in the revised manuscript. 
We changed the paragraph to read 
``Note \IWPdot\ includes convectively detrained condensate as well as
condensate produced in the young anvil.
The parameter $c_1$ is intended to implicitly account for the anvil
mass formed by both convective and mesoscale circulations.''  

% 7.
\item The 3-dimensional transport of ice mixing ratio is accomplished 
by the semi-Lagrangian technique of \cite{WiR93}. 
This is now mentioned under~(1).

% 8.
\item Reviewer~A points out that the sensitivity study presented
consists of two fundamentally different changes to the treatment of
ice cloud in the model.
The first change is the prognostic formulation itself (1), and the
second is changing the ice fraction from (2) to~(3).
Reviewer~A then asks the what the relative role of these two
modifications is.   
This question is essentially identical to the question asked by
Reviewer~B and our response applies to both:
 
Isolating the climate responses due to (1) and (3) separately in 
independent GCM sensitivity studies would not yield conclusive
results on the their role in forcing the response in the present,
combined experiment. 
This is because increasing ice fraction alone, for example, would
result in a large net climate response not present in our experiment.
Therefore we approached this question by performing offline
sensitivity studies to estimate the relative strengths of the
forcings due to the two modifications. 
The magnitude of the tropical radiative forcing by the increased upper
tropospheric condensate is 2--3~times the magnitude of the forcing due
to the increased ice fraction. 
Section~3.3 of the revised manuscript contains two paragraphs which 
summarize summarize these findings.

% 9.
\item We claimed the increased abundance of zonal average condensate
in the upper troposphere agrees with recent observations and models. 
There is no 3-dimensional observational climatology of condensate
against which to validate Figure~5.
Therefore we have diluted our claim about the modeled zonal average
condensate from ``\ldots agree with recent observations \ldots'' to
``\ldots agree with inferences from recent observations \ldots''.  
The additional observations by \cite{McH96} and \cite{McH971} we now
cite are described in our response to item~3, but do not a climatology
make.  
The use of a CEMs with large domains (ours was 900~km in longitude) is
intended bridge the gap between MCS models and GCMs. 
The improving success of many CEMs at reproducing observed features of
MCSs (GATE, EMEX, TOGA-COARE) is the (admittedly hopeful)
justification for extending the \qi\ predictions of CEMs from local to
zonal averages.  

% 10.
\item We used ``lifecycle'' to distinguish between the treatment of
anvil as a time-evolving entity with characteristic, prognostic source
and sink mechanisms as in (1), and the diagnostic treatment where
anvil is determined as present or absent based on the instantaneous
state of other large scale variables without regard to previous
timesteps as in CCM (4)--(6).  
Others prefer using ``lifecycle'' to characterize the stages of anvil
development in a mesoscale environment (e.g., the four stages of MCS
lifecycle idealized in Figure~1 of \cite{Hou89}).    
At Reviewer~A's suggestion, we have eliminated most instances of
``anvil lifecycle'' throughout the manuscript. 
In its place, we usually use ``anvil production'' and/or ``anvil
structure'', terms that more concisely reflect the major differences
between CCM and ANV.
The manuscript is now titled ``Sensitivity of tropical climate
simulations to radiative effects of anvil structure'', which continues
to emphasize the manuscript as a sensitivity study, but also
identifies the major difference in the treatment of anvil between the
two parameterizations.  

It is true that an improved diagnostic approach might well cause the
same changes as our prognostic approach.
Yes, the paper is a comparison between two different
parameterizations.   

\end{enumerate}

% Bibliography
\nocite{ZeK972}
\bibliographystyle{agu}
\bibliography{/home/zender/tex/bib}

\clearpage

\setcounter{section}{1}
\setcounter{page}{1}

\begin{center}\normalsize
Revisions to Manuscript JGRd-96152, formerly \\
``Tropical Climate Sensitivity to Representation of Cirrus
Anvil Lifecycle'', \\
now titled \\
``Sensitivity of
climate simulations to radiative effects of tropical anvil structure''
\end{center}
\bigskip
\begin{center}\normalsize
Authors: C.~S.~Zender and J.~T.~Kiehl
\end{center}\normalsize
\bigskip
Date: \today
\bigskip\normalsize

\section{Response to Reviewer B}

\begin{enumerate}

% 1.
\item Both reviewers have asked essentially the same question, that
is, to identify the relative roles of the two modifications, (1) and
(3), in causing the model response.
We did not mean to convey the impression that the prognostic
generation of anvil (1), by itself, caused the model response.
In fact, the last sentence of Section~2.2 reads ``This [increased ice
fraction] proves to be an important factor in diagnosing the cause of
change in climatological cloud radiative properties in this
sensitivity study''.

Isolating the climate responses due to (1) and (3) separately in 
independent GCM sensitivity studies would not yield conclusive
results on the their role in forcing the response in the present,
combined experiment. 
This is because increasing ice fraction alone, for example, would
result in a large net climate response not present in our experiment.
Therefore we approached this question by performing offline
sensitivity studies to estimate the relative strengths of the
forcings due to the two modifications. 
The magnitude of the tropical radiative forcing by the increased upper
tropospheric condensate is 2--3~times the magnitude of the forcing due
to the increased ice fraction. 
Section~3.3 of the revised manuscript contains two paragraphs which 
summarize these findings.

% 2.
\item At Reviewer~B's request we have isolated the radiative
response to the anvil parameterization and included it as Figure~6a,
and relabeled the total diabatic heating response as Figure~6b. 
The discussion of Figure~6 is expanded and clarified.

% 3. 
\item Yes, the adjective ``improved'' was too ambiguous.
By ``improved representation of tropical anvil'' we meant the
our prognostic representation of anvil generation and structure.
The manuscript has been changed accordingly.

% 4.
\item The determination of hydrometeor size and its radiative
treatment is now summarized at the end of Section~2.2.
Section~2.3 was added to describe the diagnostic cirrus
parameterization. 

\end{enumerate}

\end{document}



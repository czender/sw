% $Id$

% Purpose: Assemble one-shot graphics

%\documentclass[onecolumn,final,10pt]{report}
%\documentclass[final]{article}
\documentclass[twocolumn,final,11pt]{article}
\usepackage{natbib} % \cite commands from aguplus
\usepackage[figuresright]{rotating} % allows sideways figures and tables
\usepackage{graphicx} % defines \includegraphics*
\usepackage{longtable} % multi-page tables, e.g., acronyms and symbols
\usepackage{ifthen} % Boolean and programming commands

\usepackage{/home/zender/tex/csz} % all my local definitions
% Usage: % Usage: % Usage: \input{jgr_abb} % AGU-sanctioned journal title abbreviations

\def\aapgb{{\it Amer. Assoc. Petroleum Geologists Bull.}}
\def\adg{{\it Adv. Geophys.}}
\def\ajs{{\it Amer. J. Sci.}}
\def\amb{{\it Ambio}}
\def\amgb{{\it Arch. Meteorol. Geophys. Bioclimatl.}}
\def\ang{{\it Ann. Glaciol.}}
\def\angeo{{\it Ann. Geophys.}}
\def\apo{{\it Appl. Opt.}}
\def\areps{{\it Ann. Rev. Earth Planet. Sci.}}
\def\asr{{\it Adv. Space Res.}}
\def\ate{{\it Atmos. Environ.}}
\def\atf{{\it Atmosfera}}
\def\atms{{\it ACM Trans. Math Software}}
\def\ato{{\it Atmos. Ocean}}
\def\atr{{\it Atmos. Res.}}
\def\gbc{{\it Global Biogeochem. Cycles}} % csz
\def\blm{{\it Boundary-Layer Meteorol.}} % csz 
\def\bpa{{\it Beitr. Phys. Atmosph.}}
\def\bams{{\it Bull. Am. Meteorol. Soc.}}
\def\clc{{\it Clim. Change}}
\def\cld{{\it Clim. Dyn.}}
\def\com{{\it Computing}}
\def\dao{{\it Dyn. Atmos. Oceans}}
\def\dsr{{\it Deep-Sea Res.}}
\def\esr{{\it Earth Sci. Revs.}}
\def\gec{{\it Geosci. Canada}}
\def\gei{{\it Geofis. Int.}}
\def\gej{{\it Geogr. J.}}
\def\gem{{\it Geophys. Monogr.}}
\def\geo{{\it Geology}}
\def\grl{{\it Geophys. Res. Lett.}}
\def\ieeec{{\it IEEE Computer}}
\def\ijna{{\it IMA J. Numer. Anal.}}
\def\ijnmf{{\it Int. J. Num. Meteorol. Fl.}}
\def\jac{{\it J. Atmos. Chem.}}
\def\jacm{{\it J. Assoc. Comput. Mach.}}
\def\jam{{\it J. Appl. Meteorol.}}
\def\jas{{\it J. Atmos. Sci.}}
\def\jatp{{\it J. Atmos. Terr. Phys.}}
\def\jcam{{\it J. Climate Appl. Meteorol.}}
\def\jchp{{\it J. Chem Phys.}}
\def\jcis{{\it J. Coll. I. Sci.}}
\def\jcl{{\it J. Clim.}}
\def\jcp{{\it J. Comput. Phys.}}
\def\jfm{{\it J. Fluid Mech.}}
\def\jgl{{\it J. Glaciol.}}
\def\jgr{{\it J. Geophys. Res.}}
\def\jgs{{\it J. Geol. Soc. London}}
\def\jme{{\it J. Meteorol.}}
\def\jmr{{\it J. Marine Res.}}
\def\jmsj{{\it J. Meteorol. Soc. Jpn.}}
\def\josa{{\it J. Opt. Soc. A}}
\def\jpo{{\it J. Phys. Oceanogr.}}
\def\jqsrt{{\it J. Quant. Spectrosc. Radiat. Transfer}}
\def\jpca{{\it J. Phys. Chem. A}}
\def\lnc{{\it Lett. Nuov. C}}
\def\mac{{\it Math. Comp.}}
\def\map{{\it Meteorol. Atmos. Physics.}}
\def\mem{{\it Meteorol. Mag.}}
\def\mnras{{\it Mon. Not. Roy. Astron. Soc.}} 
\def\mwr{{\it Mon. Weather Rev.}} 
\def\nat{{\it Nature}}
\def\pac{{\it Parallel Computing}}
\def\pag{{\it Pure Appl. Geophys.}}
\def\pal{{\it Paleoceanography}}
\def\pht{{\it Physics Today}}
\def\pieee{{\it Proc. IEEE}}
\def\pla{{\it Phys. Lett. A}}
\def\ppp{{\it Paleogeogr. Paleoclim. Paleoecol.}}
\def\pra{{\it Phys. Res. A}}
\def\prd{{\it Phys. Rev. D}}
\def\prl{{\it Phys. Rev. L}}
\def\pss{{\it Planet. Space Sci.}}
\def\ptrsl{{\it Phil. Trans. R. Soc. Lond.}}
\def\qjrms{{\it Q. J. R. Meteorol. Soc.}}
\def\qres{{\it Quat. Res.}}
\def\qsr{{\it Quatern. Sci. Rev.}}
\def\reg{{\it Rev. Geophys.}}
\def\rgsp{{\it Revs. Geophys. Space Phys.}}
\def\rpp{{\it Rep. Prog. Phys.}}
\def\sca{{\it Sci. Amer.}}
\def\sci{{\it Science}}
\def\sjna{{\it SIAM J. Numer. Anal.}}
\def\sjssc{{\it SIAM J. Sci. Stat. Comput.}}
\def\tac{{\it Theor. Appl. Climatl.}}
\def\tel{{\it Tellus}}
\def\wea{{\it Weather}}

%SIAM Review: (Society for Industrial and Applied Mathematics)
%       J. on Computing
%       J. on Control and Optimization
%       J. on Algebraic and Discrete Methods
%       J. on Numerical Analysis 
%       J. on Scientific and Statistical Computing



 % AGU-sanctioned journal title abbreviations

\def\aapgb{{\it Amer. Assoc. Petroleum Geologists Bull.}}
\def\adg{{\it Adv. Geophys.}}
\def\ajs{{\it Amer. J. Sci.}}
\def\amb{{\it Ambio}}
\def\amgb{{\it Arch. Meteorol. Geophys. Bioclimatl.}}
\def\ang{{\it Ann. Glaciol.}}
\def\angeo{{\it Ann. Geophys.}}
\def\apo{{\it Appl. Opt.}}
\def\areps{{\it Ann. Rev. Earth Planet. Sci.}}
\def\asr{{\it Adv. Space Res.}}
\def\ate{{\it Atmos. Environ.}}
\def\atf{{\it Atmosfera}}
\def\atms{{\it ACM Trans. Math Software}}
\def\ato{{\it Atmos. Ocean}}
\def\atr{{\it Atmos. Res.}}
\def\gbc{{\it Global Biogeochem. Cycles}} % csz
\def\blm{{\it Boundary-Layer Meteorol.}} % csz 
\def\bpa{{\it Beitr. Phys. Atmosph.}}
\def\bams{{\it Bull. Am. Meteorol. Soc.}}
\def\clc{{\it Clim. Change}}
\def\cld{{\it Clim. Dyn.}}
\def\com{{\it Computing}}
\def\dao{{\it Dyn. Atmos. Oceans}}
\def\dsr{{\it Deep-Sea Res.}}
\def\esr{{\it Earth Sci. Revs.}}
\def\gec{{\it Geosci. Canada}}
\def\gei{{\it Geofis. Int.}}
\def\gej{{\it Geogr. J.}}
\def\gem{{\it Geophys. Monogr.}}
\def\geo{{\it Geology}}
\def\grl{{\it Geophys. Res. Lett.}}
\def\ieeec{{\it IEEE Computer}}
\def\ijna{{\it IMA J. Numer. Anal.}}
\def\ijnmf{{\it Int. J. Num. Meteorol. Fl.}}
\def\jac{{\it J. Atmos. Chem.}}
\def\jacm{{\it J. Assoc. Comput. Mach.}}
\def\jam{{\it J. Appl. Meteorol.}}
\def\jas{{\it J. Atmos. Sci.}}
\def\jatp{{\it J. Atmos. Terr. Phys.}}
\def\jcam{{\it J. Climate Appl. Meteorol.}}
\def\jchp{{\it J. Chem Phys.}}
\def\jcis{{\it J. Coll. I. Sci.}}
\def\jcl{{\it J. Clim.}}
\def\jcp{{\it J. Comput. Phys.}}
\def\jfm{{\it J. Fluid Mech.}}
\def\jgl{{\it J. Glaciol.}}
\def\jgr{{\it J. Geophys. Res.}}
\def\jgs{{\it J. Geol. Soc. London}}
\def\jme{{\it J. Meteorol.}}
\def\jmr{{\it J. Marine Res.}}
\def\jmsj{{\it J. Meteorol. Soc. Jpn.}}
\def\josa{{\it J. Opt. Soc. A}}
\def\jpo{{\it J. Phys. Oceanogr.}}
\def\jqsrt{{\it J. Quant. Spectrosc. Radiat. Transfer}}
\def\jpca{{\it J. Phys. Chem. A}}
\def\lnc{{\it Lett. Nuov. C}}
\def\mac{{\it Math. Comp.}}
\def\map{{\it Meteorol. Atmos. Physics.}}
\def\mem{{\it Meteorol. Mag.}}
\def\mnras{{\it Mon. Not. Roy. Astron. Soc.}} 
\def\mwr{{\it Mon. Weather Rev.}} 
\def\nat{{\it Nature}}
\def\pac{{\it Parallel Computing}}
\def\pag{{\it Pure Appl. Geophys.}}
\def\pal{{\it Paleoceanography}}
\def\pht{{\it Physics Today}}
\def\pieee{{\it Proc. IEEE}}
\def\pla{{\it Phys. Lett. A}}
\def\ppp{{\it Paleogeogr. Paleoclim. Paleoecol.}}
\def\pra{{\it Phys. Res. A}}
\def\prd{{\it Phys. Rev. D}}
\def\prl{{\it Phys. Rev. L}}
\def\pss{{\it Planet. Space Sci.}}
\def\ptrsl{{\it Phil. Trans. R. Soc. Lond.}}
\def\qjrms{{\it Q. J. R. Meteorol. Soc.}}
\def\qres{{\it Quat. Res.}}
\def\qsr{{\it Quatern. Sci. Rev.}}
\def\reg{{\it Rev. Geophys.}}
\def\rgsp{{\it Revs. Geophys. Space Phys.}}
\def\rpp{{\it Rep. Prog. Phys.}}
\def\sca{{\it Sci. Amer.}}
\def\sci{{\it Science}}
\def\sjna{{\it SIAM J. Numer. Anal.}}
\def\sjssc{{\it SIAM J. Sci. Stat. Comput.}}
\def\tac{{\it Theor. Appl. Climatl.}}
\def\tel{{\it Tellus}}
\def\wea{{\it Weather}}

%SIAM Review: (Society for Industrial and Applied Mathematics)
%       J. on Computing
%       J. on Control and Optimization
%       J. on Algebraic and Discrete Methods
%       J. on Numerical Analysis 
%       J. on Scientific and Statistical Computing



 % AGU-sanctioned journal title abbreviations

\def\aapgb{{\it Amer. Assoc. Petroleum Geologists Bull.}}
\def\adg{{\it Adv. Geophys.}}
\def\ajs{{\it Amer. J. Sci.}}
\def\amb{{\it Ambio}}
\def\amgb{{\it Arch. Meteorol. Geophys. Bioclimatl.}}
\def\ang{{\it Ann. Glaciol.}}
\def\angeo{{\it Ann. Geophys.}}
\def\apo{{\it Appl. Opt.}}
\def\areps{{\it Ann. Rev. Earth Planet. Sci.}}
\def\asr{{\it Adv. Space Res.}}
\def\ate{{\it Atmos. Environ.}}
\def\atf{{\it Atmosfera}}
\def\atms{{\it ACM Trans. Math Software}}
\def\ato{{\it Atmos. Ocean}}
\def\atr{{\it Atmos. Res.}}
\def\gbc{{\it Global Biogeochem. Cycles}} % csz
\def\blm{{\it Boundary-Layer Meteorol.}} % csz 
\def\bpa{{\it Beitr. Phys. Atmosph.}}
\def\bams{{\it Bull. Am. Meteorol. Soc.}}
\def\clc{{\it Clim. Change}}
\def\cld{{\it Clim. Dyn.}}
\def\com{{\it Computing}}
\def\dao{{\it Dyn. Atmos. Oceans}}
\def\dsr{{\it Deep-Sea Res.}}
\def\esr{{\it Earth Sci. Revs.}}
\def\gec{{\it Geosci. Canada}}
\def\gei{{\it Geofis. Int.}}
\def\gej{{\it Geogr. J.}}
\def\gem{{\it Geophys. Monogr.}}
\def\geo{{\it Geology}}
\def\grl{{\it Geophys. Res. Lett.}}
\def\ieeec{{\it IEEE Computer}}
\def\ijna{{\it IMA J. Numer. Anal.}}
\def\ijnmf{{\it Int. J. Num. Meteorol. Fl.}}
\def\jac{{\it J. Atmos. Chem.}}
\def\jacm{{\it J. Assoc. Comput. Mach.}}
\def\jam{{\it J. Appl. Meteorol.}}
\def\jas{{\it J. Atmos. Sci.}}
\def\jatp{{\it J. Atmos. Terr. Phys.}}
\def\jcam{{\it J. Climate Appl. Meteorol.}}
\def\jchp{{\it J. Chem Phys.}}
\def\jcis{{\it J. Coll. I. Sci.}}
\def\jcl{{\it J. Clim.}}
\def\jcp{{\it J. Comput. Phys.}}
\def\jfm{{\it J. Fluid Mech.}}
\def\jgl{{\it J. Glaciol.}}
\def\jgr{{\it J. Geophys. Res.}}
\def\jgs{{\it J. Geol. Soc. London}}
\def\jme{{\it J. Meteorol.}}
\def\jmr{{\it J. Marine Res.}}
\def\jmsj{{\it J. Meteorol. Soc. Jpn.}}
\def\josa{{\it J. Opt. Soc. A}}
\def\jpo{{\it J. Phys. Oceanogr.}}
\def\jqsrt{{\it J. Quant. Spectrosc. Radiat. Transfer}}
\def\jpca{{\it J. Phys. Chem. A}}
\def\lnc{{\it Lett. Nuov. C}}
\def\mac{{\it Math. Comp.}}
\def\map{{\it Meteorol. Atmos. Physics.}}
\def\mem{{\it Meteorol. Mag.}}
\def\mnras{{\it Mon. Not. Roy. Astron. Soc.}} 
\def\mwr{{\it Mon. Weather Rev.}} 
\def\nat{{\it Nature}}
\def\pac{{\it Parallel Computing}}
\def\pag{{\it Pure Appl. Geophys.}}
\def\pal{{\it Paleoceanography}}
\def\pht{{\it Physics Today}}
\def\pieee{{\it Proc. IEEE}}
\def\pla{{\it Phys. Lett. A}}
\def\ppp{{\it Paleogeogr. Paleoclim. Paleoecol.}}
\def\pra{{\it Phys. Res. A}}
\def\prd{{\it Phys. Rev. D}}
\def\prl{{\it Phys. Rev. L}}
\def\pss{{\it Planet. Space Sci.}}
\def\ptrsl{{\it Phil. Trans. R. Soc. Lond.}}
\def\qjrms{{\it Q. J. R. Meteorol. Soc.}}
\def\qres{{\it Quat. Res.}}
\def\qsr{{\it Quatern. Sci. Rev.}}
\def\reg{{\it Rev. Geophys.}}
\def\rgsp{{\it Revs. Geophys. Space Phys.}}
\def\rpp{{\it Rep. Prog. Phys.}}
\def\sca{{\it Sci. Amer.}}
\def\sci{{\it Science}}
\def\sjna{{\it SIAM J. Numer. Anal.}}
\def\sjssc{{\it SIAM J. Sci. Stat. Comput.}}
\def\tac{{\it Theor. Appl. Climatl.}}
\def\tel{{\it Tellus}}
\def\wea{{\it Weather}}

%SIAM Review: (Society for Industrial and Applied Mathematics)
%       J. on Computing
%       J. on Control and Optimization
%       J. on Algebraic and Discrete Methods
%       J. on Numerical Analysis 
%       J. on Scientific and Statistical Computing





\topmargin -48pt   \headheight 12pt \headsep 18pt
\textheight 723pt \textwidth 470pt
\oddsidemargin 0pt \evensidemargin 0pt
\marginparwidth 72pt \marginparsep 7pt
\footskip 0pt
\footnotesep=14pt
%\textwidth 330pt

\renewcommand\textfraction{0.}
\setcounter{totalnumber}{10}
\setcounter{topnumber}{10}
\setcounter{dbltopnumber}{10}
\setcounter{bottomnumber}{10}
\renewcommand\topfraction{1.}
\renewcommand\dbltopfraction{1.}
\renewcommand\bottomfraction{1.}
\renewcommand\floatpagefraction{1.}
\renewcommand\dblfloatpagefraction{1.}

\begin{document}

% These don't seem to work before \begin{document}
\setlength\abovecaptionskip{9pt}
\setlength\belowcaptionskip{9pt}

\setlength\floatsep{0pt}
\setlength\textfloatsep{0pt}
\setlength\dblfloatsep{0pt}
\setlength\dbltextfloatsep{0pt}
\setlength\intextsep{0pt}
%\setlength\floatsep{18pt \@plus 2pt \@minus 4pt}
%\setlength\textfloatsep{18pt \@plus 2pt \@minus 4pt}
%\setlength\dblfloatsep{18pt \@plus 2pt \@minus 4pt}
%\setlength\dbltextfloatsep{18pt \@plus 2pt \@minus 4pt}
%\setlength\intextsep{20pt \@plus 4pt \@minus 4pt}

\pagenumbering{roman}
\setcounter{page}{1}
\pagestyle{myheadings}
\thispagestyle{empty}
\onecolumn
\listoffigures
\twocolumn
\pagenumbering{arabic}
\setcounter{page}{1}
\markright{Climatologies of Analyses, CCM2, CCM$\mathit{\Omega}_{.5}$, and CCM3}

%***********************************************************************
% Insert scratch commands below here
%***********************************************************************

%\onecolumn
%\thispagestyle{empty}
%Peer-reviewed publications by Charles S. Zender, grad student and CGD
%visitor on EOS program:
%\cite[]{ZeK94,ZeK961,ZeK962}
%\bibliographystyle{jas} 
%\bibliography{/home/zender/tex/bib}
%\twocolumn

\begin{figure*}
\begin{center}
\includegraphics*[width=.5\hsize]{/data/zender/ps/erbe_b_8589_01_SWCF.eps}%
\includegraphics*[width=.5\hsize]{/data/zender/ps/erbe_b_8589_07_SWCF.eps}%

\includegraphics*[width=.5\hsize]{/data/zender/ps/422_8589_01_SWCF.eps}%
\includegraphics*[width=.5\hsize]{/data/zender/ps/422_8589_07_SWCF.eps}%

\includegraphics*[width=.5\hsize]{/data/zender/ps/amip5_8589_01_SWCF.eps}%
\includegraphics*[width=.5\hsize]{/data/zender/ps/amip5_8589_07_SWCF.eps}%

\includegraphics*[width=.5\hsize]{/data/zender/ps/sld012d_8589_01_SWCF.eps}%
\includegraphics*[width=.5\hsize]{/data/zender/ps/sld012d_8589_07_SWCF.eps}%
\end{center}
\caption[Geographic distribution of shortwave cloud forcing SWCF
for 1985--1989 January and July ERBE, CCM2, CCM$\Omega_{.5}$, and CCM3]{
Geographic distribution of shortwave cloud forcing SWCF (\wxmS) for
1985--1989 January and July (a,b) ERBE, (c,d) CCM2, (e,f)
CCM$\Omega_{.5}$, and (g,h) CCM3.   
\label{fig:8589_SWCF}}   
\end{figure*}
\clearpage

\begin{figure*}
\begin{center}
\includegraphics*[width=.5\hsize]{/data/zender/ps/erbe_b_8589_01_LWCF.eps}%
\includegraphics*[width=.5\hsize]{/data/zender/ps/erbe_b_8589_07_LWCF.eps}%

\includegraphics*[width=.5\hsize]{/data/zender/ps/422_8589_01_LWCF.eps}%
\includegraphics*[width=.5\hsize]{/data/zender/ps/422_8589_07_LWCF.eps}%

\includegraphics*[width=.5\hsize]{/data/zender/ps/amip5_8589_01_LWCF.eps}%
\includegraphics*[width=.5\hsize]{/data/zender/ps/amip5_8589_07_LWCF.eps}%

\includegraphics*[width=.5\hsize]{/data/zender/ps/sld012d_8589_01_LWCF.eps}%
\includegraphics*[width=.5\hsize]{/data/zender/ps/sld012d_8589_07_LWCF.eps}%
\end{center}
\caption[Geographic distribution of longwave cloud forcing LWCF
for 1985--1989 January and July ERBE, CCM2, CCM$\Omega_{.5}$, and CCM3]{
Geographic distribution of longwave cloud forcing LWCF (\wxmS) for
1985--1989 January and July (a,b) ERBE, (c,d) CCM2, (e,f)
CCM$\Omega_{.5}$, and (g,h) CCM3.  
\label{fig:8589_LWCF}}   
\end{figure*}
\clearpage

\begin{figure}
\begin{center}
\includegraphics*[width=\halfwidth]{/data/zender/ps/erbe_b_422_amip5_sld012d_xavg_8589_01_LWCF.eps}\vfill
\includegraphics*[width=\halfwidth]{/data/zender/ps/erbe_b_422_amip5_sld012d_xavg_8589_07_LWCF.eps}\vfill
\end{center}
\caption[Zonal average longwave cloud forcing LWCF from ERBE,
CCM, and ANV for 1985--1989 January and July]{
Zonal average longwave cloud forcing LWCF (\wxmS) from ERBE
(solid), CCM2 (dotted), CCM$\Omega_{.5}$ (dashed), and CCM3 (dash-dot)
for (a) January and (b) July 1985--1989. 
\label{fig:xavg_8589_LWCF}}   
\end{figure}

\begin{figure}
\begin{center}
\includegraphics*[width=\halfwidth]{/data/zender/ps/erbe_b_422_amip5_sld012d_xavg_8589_01_SWCF.eps}\vfill
\includegraphics*[width=\halfwidth]{/data/zender/ps/erbe_b_422_amip5_sld012d_xavg_8589_07_SWCF.eps}\vfill
\end{center}
\caption[Zonal average shortwave cloud forcing SWCF from ERBE,
CCM2, CCM$\Omega_{.5}$, and CCM3 for 1985--1989 January and July]{  
Zonal average shortwave cloud forcing SWCF (\wxmS) from ERBE
(solid), CCM2 (dotted), CCM$\Omega_{.5}$ (dashed), and CCM3 (dash-dot)
for (a) January and (b) July 1985--1989. 
\label{fig:xavg_8589_SWCF}}   
\end{figure}
\clearpage

\begin{figure}
\begin{center}
\includegraphics*[width=\halfwidth]{/data/zender/ps/erbe_b_anom_xavg_8589_0112_SWCF.eps}\vfill
\includegraphics*[width=\halfwidth]{/data/zender/ps/422_anom_xavg_8589_0112_SWCF.eps}\vfill
\includegraphics*[width=\halfwidth]{/data/zender/ps/amip5_anom_xavg_8589_0112_SWCF.eps}\vfill
\includegraphics*[width=\halfwidth]{/data/zender/ps/sld012d_anom_xavg_8589_0112_SWCF.eps}\vfill
\end{center}
\caption[Seasonal amplitude of zonal average shortwave cloud forcing
SWCF for ERBE, CCM2, CCM$\Omega_{.5}$, and CCM3]{
Seasonal amplitude of zonal average shortwave cloud 
forcing (\wxmS) for (a) ERBE, (b) CCM2, (c) CCM$\Omega_{.5}$, and (d)
CCM3.  
Shading indicates ensemble monthly value is less than the 5~year
mean. 
Contour interval is 10~\wxmS.
Tickmarks represent mid-month values (i.e., N represents
November~15). 
\label{fig:anom_xavg_8589_0112_SWCF}}   
\end{figure}

\begin{figure}
\begin{center}
\includegraphics*[width=\halfwidth]{/data/zender/ps/erbe_b_anom_xavg_8589_0112_LWCF.eps}\vfill
\includegraphics*[width=\halfwidth]{/data/zender/ps/422_anom_xavg_8589_0112_LWCF.eps}\vfill
\includegraphics*[width=\halfwidth]{/data/zender/ps/amip5_anom_xavg_8589_0112_LWCF.eps}\vfill
\includegraphics*[width=\halfwidth]{/data/zender/ps/sld012d_anom_xavg_8589_0112_LWCF.eps}\vfill
\end{center}
\caption[Seasonal amplitude of zonal average longwave cloud forcing 
LWCF for ERBE, CCM2, CCM$\Omega_{.5}$, and CCM3]{
Seasonal amplitude of zonal average longwave cloud 
forcing (\wxmS) for (a) ERBE, (b) CCM2, (c) CCM$\Omega_{.5}$, and (d)
CCM3.  
Shading indicates ensemble monthly value is less than the 5~year
mean. 
Contour interval is 4~\wxmS.
Tickmarks represent mid-month values (i.e., N represents
November~15). 
\label{fig:anom_xavg_8589_0112_LWCF}}   
\end{figure}
\clearpage

\begin{figure*}
\begin{center}
\includegraphics*[width=.5\textwidth,height=.45\textheight]{/data/zender/ps/erbe_b_anom_yavg_10S10N_8589_0160_SWCF.eps}%
\includegraphics*[width=.5\textwidth,height=.45\textheight]{/data/zender/ps/422_anom_yavg_10S10N_8589_0160_SWCF.eps}%

\includegraphics*[width=.5\textwidth,height=.45\textheight]{/data/zender/ps/amip5_anom_yavg_10S10N_8589_0160_SWCF.eps}%
\includegraphics*[width=.5\textwidth,height=.45\textheight]{/data/zender/ps/sld012d_anom_yavg_10S10N_8589_0160_SWCF.eps}%
\end{center}
\caption[Hovm\"oller diagrams of shortwave cloud forcing 
anomaly in the equatorial Pacific for 1985--1989 ERBE, CCM2,
CCM$\Omega_{.5}$, and CCM3]{ 
Hovm\"oller diagrams of shortwave cloud forcing anomaly (\wxmS) in the
equatorial Pacific (averaged 10~\degreee S--10~\degreee N) for
1985--1989 (a) ERBE, (b) CCM2, (c) CCM$\Omega_{.5}$ and (d) 
CCM3.
Month 1 is January 1985.  
Contour interval is 10~\wxmS. 
\label{fig:anom_yavg_10S10N_8589_0160_SWCF}}
\end{figure*}
\clearpage

\begin{figure*}
\begin{center}
\includegraphics*[width=.5\textwidth,height=.45\textheight]{/data/zender/ps/erbe_b_anom_yavg_10S10N_8589_0160_LWCF.eps}%
\includegraphics*[width=.5\textwidth,height=.45\textheight]{/data/zender/ps/422_anom_yavg_10S10N_8589_0160_LWCF.eps}%

\includegraphics*[width=.5\textwidth,height=.45\textheight]{/data/zender/ps/amip5_anom_yavg_10S10N_8589_0160_LWCF.eps}%
\includegraphics*[width=.5\textwidth,height=.45\textheight]{/data/zender/ps/sld012d_anom_yavg_10S10N_8589_0160_LWCF.eps}%
\end{center}
\caption[Hovm\"oller diagrams of longwave cloud forcing 
anomaly in the equatorial Pacific for 1985--1989 ERBE, CCM2,
CCM$\Omega_{.5}$, and CCM3]{ 
Hovm\"oller diagrams of longwave cloud forcing anomaly (\wxmS) in the
equatorial Pacific (averaged 10~\degreee S--10~\degreee N) for
1985--1989 (a) ERBE, (b) CCM2, (c) CCM$\Omega_{.5}$ and (d) 
CCM3.
Month 1 is January 1985.  
Contour interval is 10~\wxmS. 
\label{fig:anom_yavg_10S10N_8589_0160_LWCF}}
\end{figure*}
\clearpage

\begin{figure*}
\begin{center}
\includegraphics*[width=.5\textwidth,height=.45\textheight]{/data/zender/ps/erbe_b_yavg_10S10N_8589_0160_SWCF.eps}%
\includegraphics*[width=.5\textwidth,height=.45\textheight]{/data/zender/ps/422_yavg_10S10N_8589_0160_SWCF.eps}%

\includegraphics*[width=.5\textwidth,height=.45\textheight]{/data/zender/ps/amip5_yavg_10S10N_8589_0160_SWCF.eps}%
\includegraphics*[width=.5\textwidth,height=.45\textheight]{/data/zender/ps/sld012d_yavg_10S10N_8589_0160_SWCF.eps}%
\end{center}
\caption[Hovm\"oller diagrams of shortwave cloud forcing 
in the equatorial Pacific for 1985--1989 ERBE, CCM2,
CCM$\Omega_{.5}$, and CCM3]{ 
Hovm\"oller diagrams of shortwave cloud forcing (\wxmS) in the
equatorial Pacific (averaged 10~\degreee S--10~\degreee N) for
1985--1989 (a) ERBE, (b) CCM2, (c) CCM$\Omega_{.5}$ and (d) 
CCM3.
Month 1 is January 1985.  
Contour interval is 10~\wxmS. 
\label{fig:yavg_10S10N_8589_0160_SWCF}}
\end{figure*}
\clearpage

\begin{figure*}
\begin{center}
\includegraphics*[width=.5\textwidth,height=.45\textheight]{/data/zender/ps/erbe_b_yavg_10S10N_8589_0160_LWCF.eps}%
\includegraphics*[width=.5\textwidth,height=.45\textheight]{/data/zender/ps/422_yavg_10S10N_8589_0160_LWCF.eps}%

\includegraphics*[width=.5\textwidth,height=.45\textheight]{/data/zender/ps/amip5_yavg_10S10N_8589_0160_LWCF.eps}%
\includegraphics*[width=.5\textwidth,height=.45\textheight]{/data/zender/ps/sld012d_yavg_10S10N_8589_0160_LWCF.eps}%
\end{center}
\caption[Hovm\"oller diagrams of longwave cloud forcing 
in the equatorial Pacific for 1985--1989 ERBE, CCM2,
CCM$\Omega_{.5}$, and CCM3]{ 
Hovm\"oller diagrams of longwave cloud forcing (\wxmS) in the
equatorial Pacific (averaged 10~\degreee S--10~\degreee N) for
1985--1989 (a) ERBE, (b) CCM2, (c) CCM$\Omega_{.5}$ and (d) 
CCM3.
Month 1 is January 1985.  
Contour interval is 10~\wxmS. 
\label{fig:yavg_10S10N_8589_0160_LWCF}}
\end{figure*}
\clearpage

\begin{figure*}
\begin{center}
\includegraphics*[width=.5\textwidth,height=.45\textheight]{/data/zender/ps/erbe_b_anom_yavg_10S10N_8589_0160_FLNT.eps}%
\includegraphics*[width=.5\textwidth,height=.45\textheight]{/data/zender/ps/422_anom_yavg_10S10N_8589_0160_FLNT.eps}%

\includegraphics*[width=.5\textwidth,height=.45\textheight]{/data/zender/ps/amip5_anom_yavg_10S10N_8589_0160_FLNT.eps}%
\includegraphics*[width=.5\textwidth,height=.45\textheight]{/data/zender/ps/sld012d_anom_yavg_10S10N_8589_0160_FLNT.eps}%
\end{center}
\caption[Hovm\"oller diagrams of outgoing longwave radiation
anomaly in the equatorial Pacific for 1985--1989 ERBE, CCM2,
CCM$\Omega_{.5}$, and CCM3]{ 
Hovm\"oller diagrams of outgoing longwave radiation anomaly (\wxmS) in
the equatorial Pacific (averaged 10~\degreee S--10~\degreee N) for 
1985--1989 (a) ERBE, (b) CCM2, (c) CCM$\Omega_{.5}$ and (d) 
CCM3.
Month 1 is January 1985.  
Contour interval is 10~\wxmS. 
\label{fig:anom_yavg_10S10N_8589_0160_FLNT}}
\end{figure*}
\clearpage

\begin{figure*}
\begin{center}
\includegraphics*[width=.5\textwidth,height=.45\textheight]{/data/zender/ps/noaa_anom_yavg_10S10N_8589_0160_FLNT.eps}%
\includegraphics*[width=.5\textwidth,height=.45\textheight]{/data/zender/ps/422_anom_yavg_10S10N_8589_0160_FLNT.eps}%

\includegraphics*[width=.5\textwidth,height=.45\textheight]{/data/zender/ps/amip5_anom_yavg_10S10N_8589_0160_FLNT.eps}%
\includegraphics*[width=.5\textwidth,height=.45\textheight]{/data/zender/ps/sld012d_anom_yavg_10S10N_8589_0160_FLNT.eps}%
\end{center}
\caption[Hovm\"oller diagrams of outgoing longwave radiation
anomaly in the equatorial Pacific for 1985--1989 NOAA, CCM2,
CCM$\Omega_{.5}$, and CCM3]{ 
Hovm\"oller diagrams of outgoing longwave radiation anomaly (\wxmS) in
the equatorial Pacific (averaged 10~\degreee S--10~\degreee N) for 
1985--1989 (a) NOAA, (b) CCM2, (c) CCM$\Omega_{.5}$ and (d) 
CCM3.
Month 1 is January 1985.  
Contour interval is 10~\wxmS. 
\label{fig:anom_yavg_10S10N_8589_0160_FLNT}}
\end{figure*}
\clearpage

\begin{figure*}
\begin{center}
\includegraphics*[width=.5\textwidth,height=.9\textheight]{/data/zender/ps/amip5_yavg_10S10N_8589_0160_TS1.eps}%
\includegraphics*[width=.5\textwidth,height=.9\textheight]{/data/zender/ps/amip5_anom_yavg_10S10N_8589_0160_TS1.eps}%
\end{center}
\caption[Hovm\"oller diagrams of sea surface temperature (SST) and SST
anomaly in the equatorial Pacific for 1985--1989 CCM$\Omega_{.5}$]{  
Hovm\"oller diagrams of (a) sea surface temperature (SST) and (b) SST
anomaly (\degreee K) in the equatorial Pacific (averaged 10~\degreee
S--10~\degreee N) for 1985--1989 CCM$\Omega_{.5}$.
Month 1 is January 1985.  
Contour interval is .5~\degreee K. 
\label{fig:anom_yavg_10S10N_8589_0160_TS1}}
\end{figure*}
\clearpage

\begin{figure*}
\begin{center}
\includegraphics*[width=.5\textwidth,height=.9\textheight]{/data/zender/ps/amip5_anom_yavg_10S10N_8589_0160_CLDTOT.eps}%
\includegraphics*[width=.5\textwidth,height=.9\textheight]{/data/zender/ps/sld012d_anom_yavg_10S10N_8589_0160_CLDTOT.eps}%
\end{center}
\caption[Hovm\"oller diagrams of total cloud anomaly in the equatorial
Pacific for 1985--1989 CCM$\Omega_{.5}$ and CCM3]{ 
Hovm\"oller diagrams of total cloud anomaly (\%) in the equatorial
Pacific (averaged 10~\degreee S--10~\degreee N) for 1985--1989
(a) CCM$\Omega_{.5}$ and (b) CCM3.
Month 1 is January 1985.  
Contour interval is 5~\%. 
\label{fig:anom_yavg_10S10N_8589_0160_CLDTOT}}
\end{figure*}
\clearpage

\begin{figure*}
\begin{center}
\includegraphics*[width=.5\textwidth,height=.9\textheight]{/data/zender/ps/amip5_anom_yavg_10S10N_8589_0160_CLDHGH.eps}%
\includegraphics*[width=.5\textwidth,height=.9\textheight]{/data/zender/ps/sld012d_anom_yavg_10S10N_8589_0160_CLDHGH.eps}%
\end{center}
\caption[Hovm\"oller diagrams of high cloud anomaly in the equatorial
Pacific for 1985--1989 CCM$\Omega_{.5}$ and CCM3]{ 
Hovm\"oller diagrams of high cloud anomaly (\%) in the equatorial
Pacific (averaged 10~\degreee S--10~\degreee N) for 1985--1989
(a) CCM$\Omega_{.5}$ and (b) CCM3.
Month 1 is January 1985.  
Contour interval is 5~\%. 
\label{fig:anom_yavg_10S10N_8589_0160_CLDHGH}}
\end{figure*}
\clearpage

\begin{figure*}
\begin{center}
\includegraphics*[width=.5\textwidth,height=.9\textheight]{/data/zender/ps/amip5_anom_yavg_10S10N_8589_0160_CLDMED.eps}%
\includegraphics*[width=.5\textwidth,height=.9\textheight]{/data/zender/ps/sld012d_anom_yavg_10S10N_8589_0160_CLDMED.eps}%
\end{center}
\caption[Hovm\"oller diagrams of middle cloud anomaly in the equatorial
Pacific for 1985--1989 CCM$\Omega_{.5}$ and CCM3]{ 
Hovm\"oller diagrams of middle cloud anomaly (\%) in the equatorial
Pacific (averaged 10~\degreee S--10~\degreee N) for 1985--1989
(a) CCM$\Omega_{.5}$ and (b) CCM3.
Month 1 is January 1985.  
Contour interval is 5~\%. 
\label{fig:anom_yavg_10S10N_8589_0160_CLDMED}}
\end{figure*}
\clearpage

\begin{figure*}
\begin{center}
\includegraphics*[width=.5\textwidth,height=.9\textheight]{/data/zender/ps/amip5_anom_yavg_10S10N_8589_0160_CLDLOW.eps}%
\includegraphics*[width=.5\textwidth,height=.9\textheight]{/data/zender/ps/sld012d_anom_yavg_10S10N_8589_0160_CLDLOW.eps}%
\end{center}
\caption[Hovm\"oller diagrams of low cloud anomaly in the equatorial
Pacific for 1985--1989 CCM$\Omega_{.5}$ and CCM3]{ 
Hovm\"oller diagrams of low cloud anomaly (\%) in the equatorial
Pacific (averaged 10~\degreee S--10~\degreee N) for 1985--1989
(a) CCM$\Omega_{.5}$ and (b) CCM3.
Month 1 is January 1985.  
Contour interval is 5~\%. 
\label{fig:anom_yavg_10S10N_8589_0160_CLDLOW}}
\end{figure*}
\clearpage

\begin{figure*}
\begin{center}
\includegraphics*[width=.5\textwidth,height=.9\textheight]{/data/zender/ps/amip5_anom_yavg_10S10N_8589_0160_TOTCWP.eps}%
\includegraphics*[width=.5\textwidth,height=.9\textheight]{/data/zender/ps/sld012d_anom_yavg_10S10N_8589_0160_TOTCWP.eps}%
\end{center}
\caption[Hovm\"oller diagrams of condensed water path anomaly in
the equatorial Pacific for 1985--1989 CCM$\Omega_{.5}$ and CCM3]{ 
Hovm\"oller diagrams of condensed water path anomaly (\gxmS) in the
equatorial Pacific (averaged 10~\degreee S--10~\degreee N) for
1985--1989 (a) CCM$\Omega_{.5}$ and (b) CCM3.
Month 1 is January 1985.  
Contour interval is 5~\gxmS. 
\label{fig:anom_yavg_10S10N_8589_0160_TOTCWP}}
\end{figure*}
\clearpage

\begin{figure*}
\begin{center}
\includegraphics*[width=.5\textwidth,height=.9\textheight]{/data/zender/ps/amip5_yavg_10S10N_8589_0160_CLDTOT.eps}%
\includegraphics*[width=.5\textwidth,height=.9\textheight]{/data/zender/ps/sld012d_yavg_10S10N_8589_0160_CLDTOT.eps}%
\end{center}
\caption[Hovm\"oller diagrams of total cloud in the equatorial
Pacific for 1985--1989 CCM$\Omega_{.5}$ and CCM3]{ 
Hovm\"oller diagrams of total cloud (\%) in the equatorial
Pacific (averaged 10~\degreee S--10~\degreee N) for 1985--1989
(a) CCM$\Omega_{.5}$ and (b) CCM3.
Month 1 is January 1985.  
Contour interval is 5~\%. 
\label{fig:yavg_10S10N_8589_0160_CLDTOT}}
\end{figure*}
\clearpage

\begin{figure*}
\begin{center}
\includegraphics*[width=.5\textwidth,height=.9\textheight]{/data/zender/ps/amip5_yavg_10S10N_8589_0160_CLDHGH.eps}%
\includegraphics*[width=.5\textwidth,height=.9\textheight]{/data/zender/ps/sld012d_yavg_10S10N_8589_0160_CLDHGH.eps}%
\end{center}
\caption[Hovm\"oller diagrams of high cloud in the equatorial
Pacific for 1985--1989 CCM$\Omega_{.5}$ and CCM3]{ 
Hovm\"oller diagrams of high cloud (\%) in the equatorial
Pacific (averaged 10~\degreee S--10~\degreee N) for 1985--1989
(a) CCM$\Omega_{.5}$ and (b) CCM3.
Month 1 is January 1985.  
Contour interval is 5~\%. 
\label{fig:yavg_10S10N_8589_0160_CLDHGH}}
\end{figure*}
\clearpage

\begin{figure*}
\begin{center}
\includegraphics*[width=.5\textwidth,height=.9\textheight]{/data/zender/ps/amip5_yavg_10S10N_8589_0160_CLDMED.eps}%
\includegraphics*[width=.5\textwidth,height=.9\textheight]{/data/zender/ps/sld012d_yavg_10S10N_8589_0160_CLDMED.eps}%
\end{center}
\caption[Hovm\"oller diagrams of middle cloud in the equatorial
Pacific for 1985--1989 CCM$\Omega_{.5}$ and CCM3]{ 
Hovm\"oller diagrams of middle cloud (\%) in the equatorial
Pacific (averaged 10~\degreee S--10~\degreee N) for 1985--1989
(a) CCM$\Omega_{.5}$ and (b) CCM3.
Month 1 is January 1985.  
Contour interval is 5~\%. 
\label{fig:yavg_10S10N_8589_0160_CLDMED}}
\end{figure*}
\clearpage

\begin{figure*}
\begin{center}
\includegraphics*[width=.5\textwidth,height=.9\textheight]{/data/zender/ps/amip5_yavg_10S10N_8589_0160_CLDLOW.eps}%
\includegraphics*[width=.5\textwidth,height=.9\textheight]{/data/zender/ps/sld012d_yavg_10S10N_8589_0160_CLDLOW.eps}%
\end{center}
\caption[Hovm\"oller diagrams of low cloud in the equatorial
Pacific for 1985--1989 CCM$\Omega_{.5}$ and CCM3]{ 
Hovm\"oller diagrams of low cloud (\%) in the equatorial
Pacific (averaged 10~\degreee S--10~\degreee N) for 1985--1989
(a) CCM$\Omega_{.5}$ and (b) CCM3.
Month 1 is January 1985.  
Contour interval is 5~\%. 
\label{fig:yavg_10S10N_8589_0160_CLDLOW}}
\end{figure*}
\clearpage

\begin{figure*}
\begin{center}
\includegraphics*[width=.5\textwidth,height=.9\textheight]{/data/zender/ps/amip5_yavg_10S10N_8589_0160_TOTCWP.eps}%
\includegraphics*[width=.5\textwidth,height=.9\textheight]{/data/zender/ps/sld012d_yavg_10S10N_8589_0160_TOTCWP.eps}%
\end{center}
\caption[Hovm\"oller diagrams of condensed water path in
the equatorial Pacific for 1985--1989 CCM$\Omega_{.5}$ and CCM3]{ 
Hovm\"oller diagrams of condensed water path (\gxmS) in the
equatorial Pacific (averaged 10~\degreee S--10~\degreee N) for
1985--1989 (a) CCM$\Omega_{.5}$ and (b) CCM3.
Month 1 is January 1985.  
Contour interval is 5~\gxmS. 
\label{fig:yavg_10S10N_8589_0160_TOTCWP}}
\end{figure*}
\clearpage

\begin{sidewaysfigure*}
\begin{center}
\includegraphics*[width=.25\textheight]{/data/zender/ps/erbe_b_reg_Pacific_Equatorial_87m85_0305_LWCF_SWCF.eps}%
\includegraphics*[width=.25\textheight]{/data/zender/ps/422_reg_Pacific_Equatorial_87m85_0305_LWCF_SWCF.eps}%
\includegraphics*[width=.25\textheight]{/data/zender/ps/amip5_reg_Pacific_Equatorial_87m85_0305_LWCF_SWCF.eps}%
\includegraphics*[width=.25\textheight]{/data/zender/ps/sld012d_reg_Pacific_Equatorial_87m85_0305_LWCF_SWCF.eps}%

\includegraphics*[width=.25\textheight]{/data/zender/ps/erbe_b_reg_Pacific_Equatorial_87m85_0305_TS1_LWCF.eps}%
\includegraphics*[width=.25\textheight]{/data/zender/ps/422_reg_Pacific_Equatorial_87m85_0305_TS1_LWCF.eps}%
\includegraphics*[width=.25\textheight]{/data/zender/ps/amip5_reg_Pacific_Equatorial_87m85_0305_TS1_LWCF.eps}%
\includegraphics*[width=.25\textheight]{/data/zender/ps/sld012d_reg_Pacific_Equatorial_87m85_0305_TS1_LWCF.eps}%

\includegraphics*[width=.25\textheight]{/data/zender/ps/erbe_b_reg_Pacific_Equatorial_87m85_0305_TS1_SWCF.eps}%
\includegraphics*[width=.25\textheight]{/data/zender/ps/422_reg_Pacific_Equatorial_87m85_0305_TS1_SWCF.eps}%
\includegraphics*[width=.25\textheight]{/data/zender/ps/amip5_reg_Pacific_Equatorial_87m85_0305_TS1_SWCF.eps}%
\includegraphics*[width=.25\textheight]{/data/zender/ps/sld012d_reg_Pacific_Equatorial_87m85_0305_TS1_SWCF.eps}%
\end{center}
\caption[$1987-1985$ differences in Spring quarter (March, April, and
May) mean maritime LWCF, SWCF, and SST over the equatorial Pacific for
ERBE, CCM2, CCM$\Omega_{.5}$, and CCM3]{ 
$1987-1985$ differences in Spring quarter (March, April, and May) 
mean maritime LWCF, SWCF (\wxmS), and SST (\degreee K) over the
equatorial Pacific (10~\degreee S--10~\degreee N, 140~\degreee
E--90~\degreee W) for (a,e,i) ERBE, (b,f,j) CCM2, (c,g,k)
CCM$\Omega_{.5}$, and (d,h,l) CCM3: (a--d) $\delta$LWCF vs.\
$\delta$SWCF, (e--h) $\delta$SST vs. $\delta$LWCF, and (i--l)
$\delta$SST vs. $\delta$SWCF.
Solid line is least-squares fit.
\label{fig:reg_Pacific_Equatorial_87m85_0305_LWCF_SWCF}}   
\end{sidewaysfigure*}
\clearpage

\begin{sidewaysfigure}
\begin{center}
\includegraphics*[width=.25\hsize]{/data/zender/ps/erbe_b_reg_Pacific_Equatorial_87m85_0305_LWCF_SWCF.eps}%
\includegraphics*[width=.25\hsize]{/data/zender/ps/422_reg_Pacific_Equatorial_87m85_0305_LWCF_SWCF.eps}%
\includegraphics*[width=.25\hsize]{/data/zender/ps/amip5_reg_Pacific_Equatorial_87m85_0305_LWCF_SWCF.eps}%
\includegraphics*[width=.25\hsize]{/data/zender/ps/sld012d_reg_Pacific_Equatorial_87m85_0305_LWCF_SWCF.eps}%
\end{center}
\caption[$1987-1985$ differences in Spring quarter (March, April, and
May) mean maritime LWCF and SWCF over the equatorial Pacific for ERBE,
CCM2, CCM$\Omega_{.5}$, and CCM3]{ 
$1987-1985$ differences in Spring quarter (March, April, and May) 
mean maritime LWCF and SWCF (\wxmS) over the equatorial Pacific 
(10~\degreee S--10~\degreee N, 140~\degreee E--90~\degreee W)
for (a) ERBE, (b) CCM2, (c) CCM$\Omega_{.5}$, and (d) CCM3.
Solid line is least-squares fit.
\label{fig:reg_Pacific_Equatorial_87m85_0305_LWCF_SWCF}}   
\end{sidewaysfigure}

\begin{sidewaysfigure}
\begin{center}
\includegraphics*[width=.25\hsize]{/data/zender/ps/erbe_b_reg_Pacific_Equatorial_87m85_0305_TS1_LWCF.eps}%
\includegraphics*[width=.25\hsize]{/data/zender/ps/422_reg_Pacific_Equatorial_87m85_0305_TS1_LWCF.eps}%
\includegraphics*[width=.25\hsize]{/data/zender/ps/amip5_reg_Pacific_Equatorial_87m85_0305_TS1_LWCF.eps}%
\includegraphics*[width=.25\hsize]{/data/zender/ps/sld012d_reg_Pacific_Equatorial_87m85_0305_TS1_LWCF.eps}%
\end{center}
\caption[$1987-1985$ differences in Spring quarter (March, April, and May)
mean maritime SST and LWCF over the equatorial Pacific for ERBE, CCM2, CCM$\Omega_{.5}$,
and CCM3]{ 
$1987-1985$ differences in Spring quarter (March, April, and May)
mean maritime SST (\degreee K) and LWCF (\wxmS) over the equatorial
Pacific  (10~\degreee S--10~\degreee N, 140~\degreee E--90~\degreee W)
for (a) ERBE, (b) CCM2, (c) CCM$\Omega_{.5}$, and (d) CCM3.
Solid line is least-squares fit.
\label{fig:reg_Pacific_Equatorial_87m85_0305_TS1_LWCF}}   
\end{sidewaysfigure}

\begin{sidewaysfigure}
\begin{center}
\includegraphics*[width=.25\hsize]{/data/zender/ps/erbe_b_reg_Pacific_Equatorial_87m85_0305_TS1_SWCF.eps}%
\includegraphics*[width=.25\hsize]{/data/zender/ps/422_reg_Pacific_Equatorial_87m85_0305_TS1_SWCF.eps}%
\includegraphics*[width=.25\hsize]{/data/zender/ps/amip5_reg_Pacific_Equatorial_87m85_0305_TS1_SWCF.eps}%
\includegraphics*[width=.25\hsize]{/data/zender/ps/sld012d_reg_Pacific_Equatorial_87m85_0305_TS1_SWCF.eps}%
\end{center}
\caption[$1987-1985$ differences in Spring quarter (March, April, and May)
mean maritime SST and SWCF over the equatorial Pacific for ERBE, CCM2, CCM$\Omega_{.5}$,
and CCM3]{ 
$1987-1985$ differences in Spring quarter (March, April, and May)
mean maritime SST (\degreee K) and SWCF (\wxmS) over the equatorial
Pacific  (10~\degreee S--10~\degreee N, 140~\degreee E--90~\degreee W)
for (a) ERBE, (b) CCM2, (c) CCM$\Omega_{.5}$, and (d) CCM3.
Solid line is least-squares fit.
\label{fig:reg_Pacific_Equatorial_87m85_0305_TS1_SWCF}}   
\end{sidewaysfigure}
\clearpage

\begin{figure*}
\begin{center}
\includegraphics*[width=.5\hsize,height=.9\textheight]{/data/zender/ps/noaa_anom_yavg_10S10N_7901_9312_FLNT.eps}
\end{center}
\caption[Hovm\"oller diagram of outgoing longwave radiation
anomaly in the equatorial Pacific for 1979--1993 NOAA satellites]{ 
Hovm\"oller diagram of outgoing longwave radiation anomaly (\wxmS) in
the equatorial Pacific (averaged 10~\degreee S--10~\degreee N) for 
1979--1993 NOAA satellites.
Month 1 is January 1979.  
Contour interval is 10~\wxmS. 
\label{fig:anom_yavg_10S10N_7901_9312_FLNT}}
\end{figure*}
\clearpage

\begin{figure*}
\begin{center}
\includegraphics*[height=.5\hsize,angle=90]{/data/zender/tmp/SST_01.ps}%
\includegraphics*[height=.5\hsize,angle=90]{/data/zender/tmp/dLWCF_01.ps}%

\includegraphics*[height=.5\hsize,angle=90]{/data/zender/tmp/SST_04.ps}%
\includegraphics*[height=.5\hsize,angle=90]{/data/zender/tmp/dLWCF_04.ps}%

\includegraphics*[height=.5\hsize,angle=90]{/data/zender/tmp/SST_07.ps}%
\includegraphics*[height=.5\hsize,angle=90]{/data/zender/tmp/dLWCF_07.ps}%

\includegraphics*[height=.5\hsize,angle=90]{/data/zender/tmp/SST_10.ps}%
\includegraphics*[height=.5\hsize,angle=90]{/data/zender/tmp/dLWCF_10.ps}%
\end{center}
\caption[SST and difference (ANV$-$CCM) in simulated LWCF for
1985--1989 January, April, July, and October]{
(left) SST (\degreee K) and (right) difference (ANV$-$CCM) in
simulated LWCF (\wxmS) for 1985--1989 (a,b) January, (c,d) April,
(e,f) July, and (g,h) October 1985--1989.
\label{fig:SST_dLWCF}}   
\end{figure*}

\begin{figure*}
\begin{center}
\includegraphics*[height=.5\hsize,angle=90]{/data/zender/tmp/SST_01.ps}%
\includegraphics*[height=.5\hsize,angle=90]{/data/zender/tmp/LWCF_01.ps}%

\includegraphics*[height=.5\hsize,angle=90]{/data/zender/tmp/SST_04.ps}%
\includegraphics*[height=.5\hsize,angle=90]{/data/zender/tmp/LWCF_04.ps}%

\includegraphics*[height=.5\hsize,angle=90]{/data/zender/tmp/SST_07.ps}%
\includegraphics*[height=.5\hsize,angle=90]{/data/zender/tmp/LWCF_07.ps}%

\includegraphics*[height=.5\hsize,angle=90]{/data/zender/tmp/SST_10.ps}%
\includegraphics*[height=.5\hsize,angle=90]{/data/zender/tmp/LWCF_10.ps}%
\end{center}
\caption[SST and ERBE LWCF for
1985--1989 January, April, July, and October]{
(left) SST (\degreee K) and (right) ERBE LWCF (\wxmS) for 1985--1989
(a,b) January, (c,d) April, (e,f) July, and (g,h) October 1985--1989. 
\label{fig:SST_dLWCF}}   
\end{figure*}

%***********************************************************************
% Insert scratch commands above here
%***********************************************************************

\end{document}


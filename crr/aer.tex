% $Id$

% Purpose: Aerosol physics

% Copyright (c) 1998--present, Charles S. Zender
% Permission is granted to copy, distribute and/or modify this document
% under the terms of the GNU Free Documentation License (GFDL), Version 1.3
% or any later version published by the Free Software Foundation;
% with no Invariant Sections, no Front-Cover Texts, and no Back-Cover Texts.
% GFDL: http://www.gnu.org/copyleft/fdl.html

% The original author of this software, Charlie Zender, seeks to improve
% it with your suggestions, contributions, bug-reports, and patches.
% Charlie Zender <zender at uci dot edu>
% Department of Earth System Science
% University of California, Irvine
% Irvine, CA 92697-3100

% URL: http://dust.ess.uci.edu/facts/aer/aer.pdf

% Usage (see also end of file):
% cd ~/sw/crr;make -W aer.tex aer.pdf;cd -

\documentclass[12pt,twoside]{book}

% Standard packages
\usepackage{ifpdf} % Define \ifpdf
\ifpdf % PDFLaTeX
\usepackage{graphicx} % Defines \includegraphics*
%\pdfcompresslevel=9
\usepackage{thumbpdf} % Generate thumbnails
\usepackage{epstopdf} % Convert .eps, if found, to .pdf when required
\else % !PDFLaTeX
\usepackage{graphicx} % Defines \includegraphics*
\fi % !PDFLaTeX
% fxm: 20001030 amsmath breaks hyperref in some documents
\usepackage{amsmath} % \subequations, \eqref, \align
\usepackage{array} % Table and array extensions, e.g., column formatting
\usepackage[dayofweek]{datetime} % \xxivtime, \ordinal
\usepackage{epigraph} % Pithy quotes at section starts, \epigraph
\usepackage{etoolbox} % \newbool, \setbool, \ifxxx
\usepackage{longtable} % Multi-page tables, e.g., acronyms and symbols
\usepackage{lscape} % Landscape environment
\usepackage{makeidx} % Index keyword processor: \printindex and \see
\usepackage{mdwlist} % Compact list formats \itemize*, \enumerate*
\usepackage{natbib} % \cite commands from aguplus
% fxm: 20001028 /usr/share/texmf/tex/latex/base/showidx.sty breaks hyperref
%\usepackage{showidx} % Print index entries as marginal notes
\usepackage{times} % Postscript Times-Roman font KoD99 p. 375
\usepackage{tocbibind} % Add Bibliography and Index to Table of Contents
\usepackage{url} % Typeset URLs and e-mail addresses

% hyperref is last package since it redefines other packages' commands
% hyperref options, assumed true unless =false is specified:
% backref       List citing sections after bibliography entries
% baseurl       Make all URLs in document relative to this
% bookmarksopen Unknown
% breaklinks    Wrap links onto newlines
% colorlinks    Use colored text for links, not boxes
% hyperindex    Link index to text
% plainpages=false Suppress warnings caused by duplicate page numbers
% pdftex        Conform to pdftex conventions
% Colors used when colorlinks=true:
% linkcolor     Color for normal internal links
% anchorcolor   Color for anchor text
% citecolor     Color for bibliographic citations in text
% filecolor     Color for URLs which open local files
% menucolor     Color for Acrobat menu items
% pagecolor     Color for links to other pages
% urlcolor      Color for linked URLs
\ifpdf % PDFLaTeX
\usepackage[backref,breaklinks,colorlinks,citecolor=blue,linkcolor=blue,urlcolor=blue,hyperindex,plainpages=false]{hyperref} % Hyper-references
%\pdfcompresslevel=9
\else % !PDFLaTeX
\usepackage[backref=false,breaklinks,colorlinks=false,hyperindex,plainpages=false]{hyperref} % Hyper-references
\fi % !PDFLaTeX

% preview-latex recommends it be last-activated package
\usepackage[showlabels,sections,floats,textmath,displaymath]{preview} % preview-latex equation extraction

% Personal packages
\usepackage{csz} % Library of personal definitions
\usepackage{abc} % Alphabet as three letter macros
\usepackage{dmn} % Dimensional units
\usepackage{chm} % Commands generic to chemistry
\usepackage{dyn} % Commands generic to fluid dynamics
\usepackage{aer} % Commands specific to aerosol physics
\usepackage{rt} % Commands specific to radiative transfer
\usepackage{psd} % Particle size distributions
\usepackage{hyp} % Hyphenation exception list
\usepackage{jrn_agu} % AGU-sanctioned journal title abbreviations

% Commands which must be executed in preamble
\makeglossary % Glossary described on KoD95 p. 221
\makeindex % Index described on KoD95 p. 220

% Commands specific to this file
\newcounter{reaction} % Create separate counter for chemical reactions (dchem.sty)
\renewcommand{\thereaction}{R\arabic{reaction}} % Style for reaction numbering (dchem.sty)
\makeatletter
\renewcommand{\fnum@table}{\textbf{\tablename~\thetable}} % Boldface ``Table: #''
\renewcommand{\fnum@figure}{\textbf{\figurename~\thefigure}} % Boldface ``Figure: #''
\makeatother
\hyphenation{} % Hyphenation exception list

% 1. Primary commands
% 2. Derived commands
% 3. Doubly-derived commands

% 19990316: From Donald Arseneau, 
% ``Here also is a redefinition of caption that only typesets once, and
% gives the default formatting (plus \bf) if you are inclined to try.''
% Without this re-definition, footnotes in table captions appear twice.
\makeatletter
\renewcommand{\fnum@table}{\textbf{\tablename~\thetable}}
\def\onecaptflag{268 }
\long\def\@makecaption#1#2{\let\@tempa\relax
   \ifdim\prevdepth>-99\p@ \vskip\abovecaptionskip \relax 
   \else \def\@tempa{\vbox to\topskip{}}\fi
   {#1: }\@tempa \vadjust{\penalty \onecaptflag}#2\par
   \ifnum\lastpenalty=\onecaptflag
      \unpenalty \setbox\@tempboxa\lastbox
      \nointerlineskip
      \hbox to\hsize{\hskip\parfillskip\unhbox\@tempboxa}%
   \fi \vskip\belowcaptionskip}
\makeatother

% Margins
\topmargin -24pt \headheight 12pt \headsep 12pt
\textheight 9in \textwidth 6.5in
\oddsidemargin 0in \evensidemargin 0in
%\marginparwidth 0pt \marginparsep 0pt
\setlength{\marginparwidth}{1.5in} % Width of callouts of index terms and page numbers KoD95 p. 220
\setlength{\marginparsep}{12pt} % Add separation for index terms KoD95 p. 220
\footskip 24pt
\footnotesep=0pt

% Float placement
% NB: Placement of figures is very sensitive to \textfraction
\renewcommand\textfraction{0.0} % Minimum fraction of page that is text
\setcounter{totalnumber}{73} % Maximum number of floats per page
\setcounter{topnumber}{73} % Maximum number of floats at top of page
\setcounter{dbltopnumber}{73} % Maximum number of floats at top of two-column page
\setcounter{bottomnumber}{73} % Maximum number of floats at bottom of page
\renewcommand\topfraction{1.0} % Maximum fraction of top of page occupied by floats
\renewcommand\dbltopfraction{1.0} % Maximum fraction of top of two-column page occupied by floats
\renewcommand\bottomfraction{1.0} % Maximum fraction of bottom of page occupied by floats
\renewcommand\floatpagefraction{1.0} % Fraction of float page filled with floats
\renewcommand\dblfloatpagefraction{1.0} % Fraction of double column float page filled with floats

\begin{document} % End preamble

\ifpdf % PDFLaTeX
\pdfinfo{ % A command provided by PDFLaTeX
/Title (Natural Aerosols in the Climate System)
/Subject (Natural Aerosols in the Climate System)
/Author (Charlie Zender)
/Keywords (aer mnr dst)
} % end \pdfinfo
\fi % !PDFLaTeX
\hypersetup{ % A command provided by \hyperref
pdftitle={Natural Aerosols in the Climate System},
pdfsubject={Natural Aerosols in the Climate System},
pdfauthor={Charlie Zender},
pdfkeywords={aer mnr dst}
} % end \hypersetup

\begin{center}
Online: \url{http://dust.ess.uci.edu/facts} \hfill Built: \shortdate\today, \xxivtime\\
\bigskip
{\Large \textbf{Natural Aerosols in the Climate System}}\\
\bigskip
by Charlie Zender\\
University of California, Irvine\\
\end{center}
Department of Earth System Science \hfill \url{zender@uci.edu}\\
University of California \hfill Voice: (949)\thinspace 891-2429\\
Irvine, CA~~92697-3100 \hfill Fax: (949)\thinspace 824-3256

\begin{figure*}[b]
\centering
\includegraphics[width=\hsize]{/Users/zender/data/fgr/aer/esh_fgr_cpy}\vfill
\caption[Mineral Dust Lifecycle]{
Mineral dust lifecycle. 
Illustration by Robynn Zender.
\label{fgr:esh}}
\end{figure*}

\frontmatter % Preface, tables of contents, changes numbering to Roman
\pagenumbering{roman}
\setcounter{page}{1}
\pagestyle{headings}
\thispagestyle{empty}

% Frontmatter and copyrights
\noindent We gratefully acknowledge \href{http://arts.ucsc.edu/gdead/agdl}{The Annotated Grateful Dead Lyrics} site by David Dodd.\\
\noindent Lyric to ``Box of Rain'' (p.~\pageref{BoR}), \copyright\ Ice Nine Publishing Company. Used with permission.\\

\tableofcontents
\listoffigures
\listoftables
\clearpage

\mainmatter % Returns numbering to Arabic
\pagenumbering{arabic}
\setcounter{page}{1}
%\markleft{Natural Aerosols in the Climate System}
%\markright{}

\chapter{Introduction}\label{sxn:ntr}

This monograph describes mathematical, physical, chemical, and
computational considerations pertinent to understanding and simulating 
the distribution and effects of natural aerosols in Earth's
atmosphere.
Much of the content applies generically to any aerosol, but the
majority of the aerosol-specific sections focus on mineral dust.
There are also minor sections on sea salt mobilization and gaseous
uptake on sulfate particles.

\section[Acknowledgements]{Acknowledgements}\label{sxn:ack}
This monograph benefits from my discussions with many scientists.
Their names appear in citations whenever possible.
However, many of the their ideas, presented at meetings or in private
conversations, are recapitulated here without acknowledgement. 
These people include Drs.\ 
Stephane Alfaro (Universit\'{e} Paris), 
Richard Arimoto (New Mexico State University),
Vicki Grassian (University of Iowa),
Zev Levin (Tel Aviv University),
Natalie Mahowald (National Center for Atmospheric Research), 
Bill Nickling (Guelph University),
Greg Okin (University of Virgina), 
Kevin Perry (University of Utah),
Yaping Shao (University of New South Wales), and 
Richard Washington (Oxford University). 

\section[Literature Review]{Literature Review}\label{sxn:ltr_rvw_mdl}
\cite{Hei89} reviewed the state of knowledge of tropospheric aerosol
composition gleaned mainly from boundary layer observations.
\cite{GBC90} report the observed size distributions and elemental
compositions of mineral aerosols measured by a cascade impactor in the
Sahara. 
\cite{GBW96} report measurements of natural saltation (at Pismo Beach,
California) that form a useful dataset for testing saltation theories.
\cite{AGG97} describe results of wind-tunnel experiments to deduce the
dependence of the emitted dust size distribution on the saltation
intensity and $\wndfrc$.
\cite{PRM99} and \cite{LRG99} show how multi-angle and microwave
satellite sensors can adequately retrieve land surface properties such
as \trmidx{LAI} and roughness length, key to determining dust
mobilization.  
\cite{STB98} show the important role of infrared absorption by mineral dust.
\cite{SoT99} analyze the effects of mineral composition on dust
optical properties.
\cite{CPH99} compared the in situ observations of mineral dust with
TOMS satellite retrievals.
\cite{LFC99} counted mineral dust particles smaller than 2\,\um\ as
cloud droplet condensation nuclei.
\cite{RFM991} and \cite{RFM992} analyzed changes between mineral dust
climatologies during the Last Glacial Maximum and the present.
\cite{Gil99} describes the factors contributing to the recurrence of 
dust emission ``hot spots'' as seen from TOMS.
\cite{HCB99} characterize the vertical structure of Saharan dust
exported to the Mediterranean basin.
\cite{CSB99} combined the FAO soil map of the world with surface
mineralogy of specific samples to create predictive relationships
linking soil type to surface mineralogy on a global scale.
\cite{LBS99} showed is it possible to identify specific soil types
from as few as six narrow-band infrared measurements.
\cite{BaP99} measured threshold friction velocities for beds of
pebbles 5--50\,\mm\ in diameter.
\cite{IAA99} describe an intensive field campaign in which radiative, 
microphysical, and chemical properties of various aerosols in the
Negev desert were measured and inter-correlated.
\cite{KKT99} present an overview of the potential of current and
future space-borne platforms to measure tropospheric aerosols
including dust.
\cite{RWA99} discovered that the proportion of soil aggregates larger
than 250\,\um\ increases linearly with \COd\ concentration in certain
grasslands due to biological effects. 
\cite{WUH00} describe a Kosa (yellow dust) deflation model and
evaluations its fidelity over East Asia.
\cite{AlG01} describe how to estimate the size distribution of the
emitted dust by accounting for the size-dependent binding energy of
the saltating particles.
\cite{Ari01} present a broad overview of the climate factors
influencing the abundance of atmospheric dust, as well as the
radiative properties controlling the climate impact of dust.
\cite{MyS01} performed sensitivity tests of the global radiative
forcing of anthropogenic mineral dust.
\cite{GZC02} discuss the stability, accuracy, and behavior of
sandblasting fluxes determined by the \cite{AlG01} model.
\cite{LuV02} develop the Hadley Centre dust model and evaluate it
against the standard suite of observations available on Earth.
\cite{Nam03} conclude from saltation measurements and modeling that
the ``launch speed'' of saltators remains relatively constant during
saltation, rather than increasing with friction speed.
\cite{LeL03} combined visible and infrared satellite measurements to
identify dust sources and track dust plumes near the north Indian
Ocean. 
\cite{Van03} directly measure chemical composition of aerosol,
including mineral dust, from Asia which dominates the mass
concentration of remote, high altitude sites as far east as the
western United States.
\cite{VHS03} examine anthropogenic erosion on military bases in the
Mojave Desert.
\cite{KuM03} discovered that increased frequency of strong winds
explains much of the observed increased in 
\trmidx{Dust Storm Frequency} (DSF) in East Asia from 2000--2002 
relative to the previous decade.
\cite{CMT04} show how sub-gridscale gustiness, driven largely dry
convection, explains dust emissions in regions where mean winds
are otherwise too weak to generate observed emissions.
\cite{KuM04} derive an empirical Snow Cover Factor (SCF) that accounts
for the influence of snow on the threshold wind velocity for dust
mobilization. 
\cite{GrZ04} apply show that accounting for saltation, sandblasting, 
and wind-speed PDFs improves the simulated size distribution of long
range transported dust in a global model.
\cite{MSM05} further discuss the problems with determining
sandblasting fluxes highlighted by \cite{GZC02}, and present a new
numerically stable scheme for their evaluation.
\cite{BCD05} describe the interaction between climate and society in
the Sahara.
\cite{AKK06} summarize dust measurements and modeling during 
\trmidx{ACE Asia}.
Many studies examine the possible role of dust as a vector for disease
organisms affecting humans \cite[]{ZeT06b} and downwind ecosystems such
as coral reefs \cite[]{SSP00,PBM05}.
\cite{YGH06} quantify the sensitivity of global dust mobilization, 
loading, and deposition to assumed size distribution.
\cite{LSR06} characterize the effects of iron oxides on dust optical properties.

Many researchers have investigated the Martian dust cycle.
In fact, until the 1990s, probably more dust research was performed by
researchers more concerned with Mars than Earth.
Recent global dust simulations on Mars are described in \cite{PaI02}, 
\cite{NLR02}, and \cite{BRW04}.

\subsection[Meteoric Dust]{Meteoric Dust}\label{sxn:ltr_rvw_mtr}
Deposition of \trmidx{Meteoric dust} has occurred since Earth's
formation. 
Annual deposits of \trmidx{extra-terrestrial} dust are much less than 
terrestrial.
However, meteoric dust has unique mineralogical characteristics that
make even a small flux interesting.
\cite{PRH94} estimates a flux of $40 \pm 20$\,\ktxyr\ based on Osmium
isotopes in seawater.
\cite{GBP04} estimate a flux of $14 \pm 5$\,\ktxyr\ based on iridium
and platinum measurements from a Greenland ice core.
\cite{MRG06} fxm.

\chapter{Boundary Layer Physics}\label{sxn:blp}

\section{Definitions}

\subsection[Wind Stress]{Wind Stress}\label{sxn:wnd_str}
Surfaces dissipate the momentum of the wind blowing over them.
This dissipation is the result of tangential shear stress between the
wind and the surface elements.
The rate of change of atmospheric momentum $\mssatm \vlcvec$ defines a
stress force $\wndstrvec$ and the magnitude of this stress force
$\wndstr = (\wndstrvec \cdot \wndstrvec)^{1/2}$ expresses the total
momentum the surface extracts from the wind per unit surface area per
unit time.
Hence the surface wind stress is also called the surface 
\trmidx{momentum flux}.
Some fraction of this wind stress $\wndstr$ does work on the surface
in the form of moving the surface elements, e.g., moving leaves,
or causing waves.
Over bare or nearly bare ground much of the wind stress must go into
aeolian abrasion (over stony surfaces) or soil movement since there is
little else to absorb the force.
The remainder of the wind stress may be converted to frictional
heating of the surface, or small scale atmospheric turbulence.

We define the horizontal wind stress $\wndstr$ by appealing to basic
principles of fluid dynamics.
A fluid of density $\dnsatm$ moving at speed $\wndspd$ exerts a
pressure $\prs$ (force per unit area) of $\frac{1}{2} \dnsatm
\wndspd^{2}$ on a stationary object transverse to the flow. 
The wind stress $\wndstr$ tangential to the surface takes a similar
form,
\begin{eqnarray}
\wndstr & = & \xchcffmmn \dnsatm \wndspd^{2}
\label{eqn:wnd_str_dfn}
\end{eqnarray}
where $\xchcffmmn$ is the \trmdfn{exchange coefficient for momentum}.
The dimensionless $\xchcffmmn$ may be thought of as twice the
equivalent vertical obstruction created by a given horizontal surface.  
$\xchcffmmn$, in turn, depends on the properties and distribution of
surface elements.
The quadratic dependence of $\wndstr$ on $\wndspd$ results from the
\ldots (fxm: look this up in Kundu, Bernoulli's equation).

The total stress to the surface $\wndstrvec$ is the vector sum of  
individual components representing stress dissipated by the plant
canopy, stress dissipated by airborne (saltating and suspended)
particles, and, finally, wind stress dissipated by the solid surface
itself. 
This \trmdfn{stress partition} or \trmdfn{drag partition} has 
important implications for dust studies.
\begin{eqnarray}
\wndstr & = & \wndstratm + \wndstrslt
\label{eqn:wnd_str_prt}
\end{eqnarray}
Drag partitioning is discussed further in \S\ref{sxn:drg} and 
\S\ref{sxn:hrz}. 

Using (\ref{eqn:wnd_frc_dfn_2}) to express the wind stress $\wndstr$
(\ref{eqn:wnd_str_dfn}) solely in terms of $\wndfrc$ we obtain
\begin{eqnarray}
\wndstr & = & \dnsatm \wndfrc^{2}
\label{eqn:wnd_str_dfn_2}
\end{eqnarray}

\subsection[Friction Velocity]{Friction Velocity}
We now consider the wind speed profile $\wndspd(\hgt)$ between the
free atmosphere and the surface.
The planetary surface is the interface between the fluid atmosphere
and the ``solid'' surface (soil, ocean, etc). 
A solid land surface requires a no slip boundary condition, i.e., the
wind speed is zero exactly at the surface.
To a good approximation, the ocean may also be treated with the no
slip boundary condition since the atmospheric wind speed $\wndspdatm$
is usually much larger than the surface current in the ocean
$\vlcocn$, i.e., $\wndspdatm \gg \vlcocn$.
Let us assume that we know the measured or predicted wind speed
$\wndspdatm$ at a height $\hgtatm$ above the surface.

Knowing the wind with speed $\wndspd$ at height $\hgt$ exerts a stress
$\wndstr$ on the surface, 
\begin{eqnarray}
\wndfrc & = & \sqrt{ \frac{\wndstr}{\dnsatm} }
\label{eqn:wnd_frc_dfn}
\end{eqnarray}
$\wndfrc$ is called the \trmdfn{friction velocity}, \trmdfn{drag
velocity}, or, more appropriately, the \trmdfn{friction speed}.

Substituting (\ref{eqn:wnd_str_dfn}) into (\ref{eqn:wnd_frc_dfn}) we
see that 
\begin{eqnarray}
\wndfrc & = & \xchcffmmn^{1/2} \wndspd
\label{eqn:wnd_frc_dfn_2}
\end{eqnarray}

The friction velocity $\wndfrc$ is the fundamental quantity
determining the flux of dust into the atmosphere.
Nevertheless, it is difficult to attach a simple physical
interpretation to the friction velocity.
However we now demonstrate two important physical properties of
$\wndfrc$.  
First, the mean horizontal wind speed at the top of the laminar layer
is $\wndfrc$. 
Thus immediately after uplift, a particle is embedded in a horizontal
wind of speed $\wndfrc$.
In \S\ref{sxn:hrz} we use this property of $\wndfrc$ to explain
the observed cubic dependence of the horizontal mass flux of saltating
particles on the wind speed. 

Secondly, $\wndfrc$ is proportional to the mean velocity gradient
$\frac{\partial \uuu}{\partial \zzz}$ near the surface.

\subsection[Conversions]{Conversions}
There are many other useful relations which can be established between
$\wndspd$, $\wndfrc$, $\rssmmn$, $\xchcffmmn$, and $\wndstr$.
These relations are simple, but tedious, to derive.
Table~\ref{tbl:cnv} lists many of the relations between frequently
occurring boundary layer parameters.
\begin{table}
\begin{minipage}{\hsize} % Minipage necessary for footnotes KoD95 p. 110 (4.10.4)
\renewcommand{\footnoterule}{\rule{\hsize}{0.0cm}\vspace{-0.0cm}} % KoD95 p. 111
\begin{center}
\caption[Boundary Layer Physics Conversion Table]{\textbf{Conversion Factors
Between Quantities In The Boundary Layer}%
\footnote{Note that the numerators of the logarithmic profiles should
actually be the difference between atmospheric height and zero plane 
displacement height $\hgtatm - \hgtzpd$.
For convenience we define $\wndstr \equiv |\wndstrvec|$.
Stability corrections are omitted for brevity.}%
\label{tbl:cnv}}   
\vspace{\cpthdrhlnskp}
\begin{tabular}{ >{$\dpysty}r<{$} *{5}{>{$\dpysty}c<{$}} } % KoD95 p. 94 describes '*' notation
\hline \rule{0.0ex}{\hlntblhdrskp}% 
& \wndspd & \wndfrc & \rssmmn & \xchcffmmn & \wndstr \\[0.0ex]
& \mbox{\mxs} & \mbox{\mxs} & \mbox{\sxm} & & \mbox{\NxmS} \\[0.0ex]
\hline \rule{0.0ex}{\hlntblntrskp}%
\wndspd & - & \wndfrc \xchcffmmn^{-1/2} \quad \frac{\wndfrc}{\vonkrmcst}
\ln \left( \frac{\hgtatm}{\rghmmn} \right) & & 
\sqrt{\frac{\wndfrc^{2}}{\xchcffmmn}} & \\[1.0ex]

\wndfrc & \wndspd \xchcffmmn^{1/2} & - & & 
& \sqrt{\frac{\wndstr}{\dnsatm}} \\[1.0ex]

\rssmmn & & \frac{1}{\vonkrmcst \wndfrc} \ln \left(\frac{\hgtatm}{
\rghmmn} \right) & - & (\xchcffmmn \wndspd)^{-1} & \\[1.0ex]

\xchcffmmn & \frac{\wndstr}{\dnsatm \wndspd^{2}} & \frac{\wndfrc^{2}}{\wndspd^{2}} & (\rssmmn \wndspd )^{-1} & - & \\[1.0ex] % Rau92 (1)

\wndstr & -\frac{\dnsatm \wndspd}{\rssmmn} & \dnsatm \wndfrc^{2} & & 
-\xchcffmmn \dnsatm \wndspd^{2} & -
\\[1.0ex] % Rau92 (1)
\hline
\end{tabular}
\end{center}
\end{minipage}
\end{table}

\subsection[Neutral Stability]{Neutral Stability}
The thermodynamic properties of the boundary layer determine the
vertical gradient of fluxes within the boundary layer.
In neutral conditions the wind speed varies logarithmically with
height according to
\begin{eqnarray}
\wndntr ( \hgtatm ) & = & 
\frac{\wndfrc}{\vonkrmcst} \ln \left( \frac{\hgtatm}{\rghmmn} \right)
\label{eqn:wnd_ntr_dfn}
\end{eqnarray}
where $\vonkrmcst$ is the \trmidx{Von~Karman constant}.
The ``\ntrsbs'' superscript indicates neutral conditions.
Strictly speaking, a \trmdfn{logarithmic wind profile} refers to a
wind profile which obeys (\ref{eqn:wnd_ntr_dfn}).
Loosely used, the term refers to any wind profile in the lowest
hundred meters or so of the atmosphere.
The \trmdfn{neutral exchange coefficient for momentum}, also called
the \trmdfn{neutral drag coefficient}, is then
\begin{eqnarray}
\xchcffmmnntr ( \hgtatm, \rghmmn+\hgtzpd ) & = & 
\vonkrmcst^{2} \left[ \ln \left( \frac{\hgtatm - \hgtzpd}{\rghmmn} \right) \right]^{-2}
\label{eqn:cff_xch_mmn_ntr_dfn}
\end{eqnarray}

Finally, it is sometimes useful to invert
(\ref{eqn:cff_xch_mmn_ntr_dfn}) in order to obtain $\rghmmn$ in terms
of $\xchcffmmnntr$
\begin{eqnarray}
\rghmmn & = & \hgtatm \exp \left( - \frac{\vonkrmcst}{\sqrt { \xchcffmmnntr}} \right)
\label{eqn:rgh_mmn_ntr_dfn}
\end{eqnarray}

\section[Surface Fluxes]{Surface Fluxes}\label{sxn:flx_sfc}
The surface fluxes for momentum, heat, and vapor transfer are
coupled by micrometeorological exchanges between the surface and
the atmosphere in the surface (constant flux) layer.
Determination of these fluxes from observation is possible via eddy
flux correlation techniques.
From a modeling perspective, the fluxes may determined by solving
coupled non-linear differential equations in the surface layer.
This technique is employed in \trmdfn{Large Eddy Simulation} (LES)
models. 
LES solutions resolve, as exactly as practical, the complex, turbulent
eddies which determine the thermodynamic behavior of the boundary
layer. 
However, large scale atmospheric models cannot afford to solve the
continuous equations of motions throughout the boundary layer.
Instead, a class of bi-level solutions for boundary layer fluxes has
been developed based on \trmdfn{Monin-Obukhov similarity theory}.

Monin-Obukhov similarity theory is usually applied in terms of
\trmidx{resistance} $\rrr$ and \trmidx{conductance} 
$\CCC \equiv \rrr^{-1}$ which describe the transfer of scalar
quantities between two levels within the boundary layer.
For simplicity, one of these levels is taken as $\hgtmdp$ the midpoint
height of the lowest atmospheric layer in the large scale atmospheric
model.
A host model provides the potential temperature $\tptptnmdp$, pressure 
$\prsmdp$, specific humidity $\qvprmdp$, and meridional and zonal
winds $\wndmrdmdp$ and $\wndznlmdp$.
The subscript $\mdpsbs$ indicates the quantities are defined at the
height $\hgtmdp$. 
The momentum fluxes $\wndstrx$ and $\wndstry$\,[\kgxmsS], sensible heat
flux $\flxsns$\,[\wxmS], and moisture flux $\flxvpr$\,[\kgxmSs] are
defined by the vertical gradient of the appropriate thermodynamic
quantity between $\hgtatm = \hgtmdp$ and $\hgtatm = \hgtsfc$, where
$\hgtsfc$ is the ``surface height'' (defined below).
The fluxes are expressed as
\begin{eqnarray}
% Bon96 p. 47
\label{eqn:wnd_str_x_dfn}
\wndstrx & = & -\dnsatm \frac{( \wndznlmdp - \wndznlsfc )}{\rssmmn} \\
\label{eqn:wnd_str_y_dfn}
\wndstry & = & -\dnsatm \frac{( \wndmrdmdp - \wndmrdsfc )}{\rssmmn} \\
\label{eqn:flx_sns_dfn}
\flxsns & = & -\dnsatm \heatcpcspcprs \frac{( \tptptnmdp - \tptptnsfc )
}{ \rssheat} \\
\label{eqn:flx_vpr_dfn}
\flxvpr & = & -\dnsatm \frac{( \qvprmdp - \qvprsfc )}{\rssvpr} 
\end{eqnarray}
The similarity of these expressions to one another arises from the
definitions of the resistances $\rssmmn$, $\rssheat$, and $\rssvpr$.  
These resistances depend implicitly on the fluxes $\wndstr$,
$\flxsns$, and $\flxvpr$, through Monin-Obukhov similarity theory.
Thus (\ref{eqn:wnd_str_x_dfn})--(\ref{eqn:flx_vpr_dfn}) must be solved   
iteratively.

Solutions to (\ref{eqn:wnd_str_x_dfn})--(\ref{eqn:flx_vpr_dfn}) must
balance the surface energy budget.
In other words, power absorbed by the surface must be dissipated by
surface heating/cooling, and energy divergence to the atmosphere or
soil in the form of latent, sensible, and radiative heating or
cooling.

\subsection[Bulk Aerodynamic Approximation]{Bulk Aerodynamic Approximation}\label{sxn:bad}
The \trmdfn{turbulent surface fluxes}, also called \trmdfn{Reynolds
fluxes}, are the fluxes of heat, moisture, and momentum between the
surface and the atmosphere.
These fluxes arise as the atmosphere and the surface attempt to reach
equilibrium states with the prevailing conditions.
Because they are usually unresolved, the turbulent fluxes are usually 
determined by the application of Monin-Obukhov theory to the
prevailing mean conditions.
One simple and popular method, the 
\trmdfn{bulk aerodynamic approximation}, is of particular utility to
large scale atmospheric models. 
We shall describe the essential physics for determining the turbulent
surface fluxes, and related parameters, using the bulk aerodynamic
approximation and more complex approximations.

There are three turbulent fluxes of interest: the momentum flux (also
called the \trmdfn{surface stress} or \trmdfn{wind stress})
$\wndstr$\,[\kgxmsS], and the sensible and latent heat fluxes
$\flxsns$ and $\flxltn$, respectively, both measured in \wxmS. 
These fluxes are defined in terms of the \trmdfn{eddy fluxes} of the
appropriate fields.
Any scalar field $\xxx(\tm)$ may be decomposed into time-mean and
fluctuating  components, $\xxxbar$ and $\xxxprm$, respectively 
\begin{eqnarray}
\xxx(\tm) & = & \xxxbar + \xxxprm(\tm) % 
\label{eqn:flc_dfn}
\end{eqnarray}
By definition the time-mean component does not depend on time~$\tm$ 
and the time average of the fluctuating component vanishes.
\begin{subequations}
\label{eqn:flc_prp_dfn}
\begin{align}
\label{eqn:flc_prp_avg_dfn}
\overline{\xxxbar} & = \xxxbar \\
\label{eqn:flc_prp_flc_dfn}
\overline{\xxxprm(\tm)} & = 0 
\end{align}
\end{subequations}
Atmospheric models always predict the time-mean components of state
variables like wind speed and temperature.
The time-mean components satisfy the mass, energy, and momentum
conservation equations (i.e., the \textit{primitive equations})
which underpin fluid mechanics.
Rarely do models predict the variance about the mean.
Parameterizations of this variance usually invoke an extra degree of
freedom to describe the PDF of the variance about the mean.

Eddy fluxes arise from the fluctuating components of state variables.
Consider the vertical fluxes of the scalar quantity~$\xxx$
(\ref{eqn:flc_dfn}). 
For concreteness, imagine that $\xxx$ represents horizontal wind speed
$\wndspd$, temperature $\tpt$, or specific humidity $\mmrvpr$.
Using (\ref{eqn:flc_dfn}) we see that the instantaneous vertical flux
of $\xxx$ is 
 \begin{eqnarray}
\wndvrt(\tm)\xxx(\tm) & = & 
[\wndvrtbar + \wndvrtprm(\tm)][\xxxbar + \xxxprm(\tm)] \nonumber \\
& \equiv & (\wndvrtbar + \wndvrtprm)(\xxxbar + \xxxprm) \nonumber \\
& = & \wndvrtbar\xxxbar + \wndvrtprm\xxxbar + \xxxprm\wndvrtbar + \xxxprm\wndvrtprm
\label{eqn:flx_sfc_vrt_nst_dfn}
\end{eqnarray}
The time-mean surface flux of $\xxx$ is obtained by applying the
time-average operator to (\ref{eqn:flx_sfc_vrt_nst_dfn})  
\begin{eqnarray}
\overline{\wndvrt(\tm)\xxx(\tm)} & = & 
\overline{\wndvrtbar\xxxbar + \wndvrtprm\xxxbar + \xxxprm\wndvrtbar +
  \wndvrtprm\xxxprm} \nonumber \\
& = & 
\overline{\wndvrtbar\xxxbar} + \overline{\wndvrtprm\xxxbar} +
\overline{\xxxprm\wndvrtbar} + \overline{\wndvrtprm\xxxprm} \nonumber \\
& = & 
\wndvrtbar\xxxbar + \overline{\wndvrtprm\xxxprm} \nonumber \\
& = & 
\overline{\wndvrtprm\xxxprm}
\label{eqn:flx_sfc_vrt_nst_dfn}
\end{eqnarray}
where we have used the time-mean properties of $\xxxbar$ and $\xxxprm$
(\ref{eqn:flc_prp_dfn}) and the further property that the vertical
wind vanishes at the surface $\wwwbar(\hgt = 0) \equiv 0$.

The eddy fluxes are multiplied by a pre-factor to obtain the
conventional units
\begin{subequations}
\label{eqn:flx_trb_sfc_dfn}
\begin{align}
\label{eqn:flx_trb_mmn_dfn}
\flxmmn & = - \dnsatm \overline{ \wndvrtprm \wndspdprm } \\ % LaP81 p. 325 (1)
\label{eqn:flx_trb_heat_dfn}
\flxsns & = \heatcpcspcprs \dnsatm \overline{ \wndvrtprm \tptprm } \\ % LaP81 p. 325 (1)
\label{eqn:flx_trb_vpr_dfn}
\flxltn & = \ltnspc \dnsatm \overline{ \wndvrtprm \mmrvprprm } % LaP81 p. 325 (1)
\end{align}
\end{subequations}

The \trmdfn{bulk aerodynamic approximation} for turbulent fluxes
defines the eddy fluxes in terms of the time-mean state variables. 
The eddy correlations are assumed to be proportional to the product of
the horizontal wind speed $\wndspd$ and change of the appropriate
state variable ($\wndspd$, $\tptptn$, or $\mmrvpr$) between the
surface and the height of interest. 
\begin{subequations}
\label{eqn:flx_sfc_blk_aer}
\begin{align}
\label{eqn:flx_sfc_blk_aer_mmn}
-\overline{ \wndvrtprm \wndspdprm } & = \xchcffmmn \wndspd \wndspddlt = \xchcffmmn \wndspd^{2} \\ % LaP81 p. 327 (13)
\label{eqn:flx_sfc_blk_aer_heat}
\overline{ \wndvrtprm \tptprm } & = \xchcffheat \wndspd \tptptndlt \\ % LaP81 p. 327 (13)
\label{eqn:flx_sfc_blk_aer_vpr}
\overline{ \wndvrtprm \mmrvprprm } & = \xchcffvpr \wndspd \mmrvprdlt %  LaP81 p. 327 (13)
\end{align}
\end{subequations}
where $\xchcffmmn$, $\xchcffheat$, and $\xchcffvpr$ are dimensionless.   
These constants of proportionality are the called \trmdfn{exchange
coefficients} for momentum, heat, and moisture, respectively. 
The atmosphere to surface changes in potential temperature and
specific humidity may be written $\tptptndlt = \tptptn - \tptptnsfc$
and $\mmrvprdlt = \mmrvpr - \mmrvprsfc$, respectively.
We shall take (\ref{eqn:flx_sfc_blk_aer}) to be the definition of the
exchange coefficients, although many other definitions are possible. 
Note that the vertical eddy correlation flux of temperature is
parameterized in terms of the potential temperature.

Combining (\ref{eqn:flx_trb_sfc_dfn}) with (\ref{eqn:flx_sfc_blk_aer}) we
obtain 
\begin{equation}
\begin{array}{ r l l r l }
\flxmmn & = & - \dnsatm \overline{ \wndvrtprm \wndspdprm } 
& = & \dnsatm \xchcffmmn \wndspd^{2} \\ % LaP81 p. 325 (1)
\flxsns & = & \heatcpcspcprs \dnsatm \overline{ \wndvrtprm \tptprm} 
& = & \dnsatm \heatcpcspcprs \xchcffheat \wndspd \tptptndlt \\ % LaP81 p. 325 (1)
\flxltn & = & \ltnspc \dnsatm \overline{ \wndvrtprm \mmrvprprm} 
& = & \ltnspc \dnsatm \xchcffvpr \wndspd \mmrvprdlt % LaP81 p. 325 (1)
\end{array}
\label{eqn:blk_flx_dfn}
\end{equation}
Each of the variables in (\ref{eqn:blk_flx_dfn}) is height dependent. 
However, it is very common to evaluate the exchange coefficients at a
particular height known as the \trmdfn{reference height} $\hgtrfr$.
The reference height is usually taken to be 10~m.
Shifting the exchange coefficients between $\hgtrfr$ an arbitrary
height $\hgt$ is useful for putting measurements from a variety of
heights into a common framework for analysis.

Table~\ref{tbl:drg} describes the defining relations of many of the
related quantities which prove useful in boundary layer meteorology. 
\begin{table}
\begin{minipage}{\hsize} % Minipage necessary for footnotes KoD95 p. 110 (4.10.4)
\renewcommand{\footnoterule}{\rule{\hsize}{0.0cm}\vspace{-0.0cm}} % KoD95 p. 111
\begin{center}
\caption[Drag Coefficients]{\textbf{Drag Coefficients}%
\footnote{\emph{Sources:} \cite{LaP81,LaP82}}%
\footnote{Usually the reference height is 10~m but we use $\hgtrfr$
for generality.}%
\label{tbl:drg}}   
\vspace{\cpthdrhlnskp}
\begin{tabular}{ >{\raggedright}p{9.0em}<{} >{$}l<{$} >{$\dpysty}l<{$} }
\hline \rule{0.0ex}{\hlntblhdrskp}% 
Name & \mbox{Symbol} & \mbox{Defining Relationship} \\[0.0ex]
\hline \rule{0.0ex}{\hlntblntrskp}%
Drag coefficient & \xchcffmmn & \wndstr = \xchcffmmn \dnsatm \wndspd^{2} \\[0.5ex]
Drag coefficient at reference height & \xchcffmmnrfr & \xchcffmmnrfr
\wndrfr^{2}  = \flxmmn / \dnsatm = - \overline{ \wndvrtprm \wndspdprm
}\\[0.5ex] % LaP81 p. 327 (14) 
Neutral drag coefficient at $\hgtrfr = 10$~m & \xchcffmmnntr & \xchcffmmnntr =
\vonkrmcst^{2} \left[ \ln \left( \frac{\hgtrfr}{\rghmmn} \right)
\right]^{-2} \\[0.5ex] % LaP82 p. 466 (9)
Wind speed at $\hgtatm = \hgtrfr$ & \wndrfr & 
\wndspd - 
\frac{\wndfrc}{\vonkrmcst} \left[
\ln \left( \frac{\hgtatm - \zpdmmn}{\hgtrfr - \zpdmmn} \right) 
- \psimmn \left( \frac{\hgtatm - \zpdmmn}{\mnolng} \right)
+ \psimmn \left( \frac{\hgtrfr - \zpdmmn}{\mnolng} \right)
\right]
\\[0.5ex] % LaP81 p. 327 (14)
Neutral wind speed at $\hgtatm = \hgtrfr$ & \wndrfrntr & \wndrfrntr
\sqrt{\xchcffmmnntr} = \wndspd \sqrt{\xchcffmmn} = \wndfrc \\[0.5ex] % CCM:dom/flxoce()
\hline
\end{tabular}
\end{center}
\end{minipage}
\end{table}
\cite{Gar771} reviews drag coefficient terminology, relationships,
measurements, and constraints. 

\subsection[Roughness Length]{Roughness Length}\label{sxn:rgh}
We assume aerosol has the same flux properties in the boundary layer
as momentum.
The \trmdfn{roughness length} for momentum transfer $\rghmmn$ is a 
property of the surface characteristics only, i.e., $\rghmmn$ is
independent of wind speed when the following conditions are met:
\begin{enumerate}
\item All roughness elements are inelastic.
Inelastic elements, such as rocks, do not deform under wind stress 
whereas elastic elements, such as some vegetation does.
For flexible vegetation, $\rghmmn$ increases with wind speed
$\wndspd$ up to a critical speed $\wndspdcrt$, and decreases
thereafter. 
Ordinary lawn grass displays this behavior as it flutters in gentle
breezes and bows before stiff winds.
\item Saltation is not occuring. 
Saltation roughens the boundary layer once it commences
(Section~\ref{sxn:drg}).
\end{enumerate}
Rougher surfaces tend to absorb more wind stress into non-erodible
elements. 
Thus saltation decreases as $\rghmmn$ increases, and visa versa.
% Paris: Marticorena's talk
The frequency of saltation events follows the same pattern, since
smooth surface initiate saltation more readily. 

\cite{Rau94} derived simple analytic relations for the roughness
length $\rghmmn$ and the zero-plane displacement $\hgtzpd$ of
vegetated surfaces as functions of vegetation height $\hgtvgt$ and
area index $\areaidx$.
Microwave radar data may be inverted to obtain high resolution
roughness length (and soil moisture) data of bare ground globally
\cite[e.g.,][]{PRM99,LRG99,ZrD03}. 

There are two roughness lengths pertinent to wind erosion over bare
ground.
The first is the aerodynamic roughness length of the bare ground
including the non-erodible elements such as pebbles, rocks, and
vegetation. 
This is what is traditionally known as the roughness length for
momentum transfer, $\rghmmn$.
The second roughness length is the so-called ``smooth'' roughness
length, $\rghmmnsmt$ \cite[]{MaB95}.
$\rghmmnsmt$ is the roughness length of a bed of potentially erodible 
particles without any non-erodible elements. 
The roughness length most easily measured in laboratory wind
tunnel experiments is $\rghmmnsmt$.
Wind tunnel experiments over uniform beds comprised of known particle
sizes show that 
\begin{eqnarray}
\rghmmnsmt & \approx & \dmt / 30 % MaB95 p. 16425 (44), Sep97 p. , GrI85 p. 
\label{eqn:rgh_mmn_smt_dfn}
\end{eqnarray}
However, uniform beds of purely erodible particles are rare in
Nature. 

It is useful to distinguish between the susceptibility of soil to
erosion, called \trmdfn{erodibility}, from the power of the wind
(or some other force) to cause erosion, called \trmdfn{erosivity}.
Erodibility depends on the microphysical, chemical, and mechanical
properties of the the soil, vegetation, and topography
\cite[][]{DYO01}. 
Erosivity depends on the wind speed, intermittency, shear, and
turbulence. 

Discounting erodible particles which are sheltered by non-erodible
elements, the roughness length felt by the atmosphere over erodible
particles is $\rghmmnsmt$.
Moreover, our theoretical understanding of threshold wind velocities
is based on $\rghmmnsmt$, while most large scale atmospheric models
are concerned with total momentum flux, and thus tend to compute
$\rghmmn$.  
Thus a theory is necessary to connect the $\rghmmnsmt$ to $\rghmmn$.
This is the theory of \trmdfn{drag partition}.
The increase $\frcthrncrdrg$ in threshold friction velocity for
saltation $\wndfrcsltthr$ due to drag partition is \cite[]{MaB95}
\begin{eqnarray}
\frcthrncrdrg & = & \left[ 1.0 - 
\left( \frac{\ln (\rghmmnmbl/\rghmmnsmt) }
{ \ln \{ 0.35 [ (0.1/\rghmmnsmt)^{0.8} ] \} } \right)
\right]^{-1}
\label{eqn:frc_thr_ncr_drg}
\end{eqnarray}
The inverse of $\frcthrncrdrg$ is the fraction of momentum transferred
that is available for inducing saltation, called the \trmdfn{wind
friction efficiency}, $\wndfrcfshfrc = \frcthrncrdrg^{-1}$.
The roughness lengths $\rghmmn$ and $\rghmmnsmt$ are properties 
of the surface characteristics only, i.e., they are independent 
of wind speed so long as the surface is not in motion.
\cite{GMB98} present corrections to this assumption for saltating
surfaces. 

Strong saltation can modify $\rghmmn$ because the saltators
provide a sink for momentum distinct from the surface.
Consider a saltator ejected from the surface with an initial speed
proportional to $\wndfrc$. 
It is common assumption that, after launching, a saltating particle
experiences no vertical acceleration except gravity.  
Such trajectories are called \trmdfn{ballistic}.
In a ballistic trajectory, the vertical velocity decreases linearly
with time, and the initial upwards velocity equals the final downwards
velocity. 
It is easy to show the height reached by a ballistic saltator is
proportional to $\wndfrc^{2}/\grv$. 
Strong saltation causes an effective thickening of the roughness
length also in proportion to $\wndfrc^{2}/\grv$.
For strongly saltating surfaces with $\wndfrc \gg \wndfrcthr$, 
\cite{Cha83} suggests
\begin{eqnarray}
\rghmmn & = & c_{0} \wndfrc^{2} / \grv % LeR91 p. 540 (2), Cha64 p. 406 (1)
\label{eqn:rgh_lmt}
\end{eqnarray}
For moderate wind friction speeds $\wndfrc \sim \wndfrcthr$, such
effects may be neglected \cite[]{LeR91}.  

Table~\ref{tbl:rgh} shows typical roughness lengths of non-vegetated 
surface types. 
\begin{table}
\begin{minipage}{\hsize} % Minipage necessary for footnotes KoD95 p. 110 (4.10.4)
\renewcommand{\footnoterule}{\rule{\hsize}{0.0cm}\vspace{-0.0cm}} % KoD95 p. 111
\begin{center}
\caption[Momentum Roughness Length]{\textbf{Momentum Roughness Lengths
of Non-Vegetated Surface Types}% 
\footnote{\emph{Sources:} \cite{Bon96}, p.~59; \cite{BKL97}, p.~F-3}%
\label{tbl:rgh}}   
\vspace{\cpthdrhlnskp}
\begin{tabular}{r >{$}l<{$}}
\hline \rule{0.0ex}{\hlntblhdrskp}% 
Surface Type & \rghmmn \\[0.0ex]
& \mbox{m} \\[0.0ex]
\hline \rule{0.0ex}{\hlntblntrskp}%
Glacial ice & 0.05 \\[0.5ex] % Bon96 p. 59
Lake ice & 0.04 \\[0.5ex] % Bon96 p. 59
Warm lake & 0.001 \\[0.5ex] % Bon96 p. 59
\footnote{More appropriate for newly formed, seasonal ice. For ridged, 
multi-year ice, $\rghmmn \sim 0.05$~m.}%
Sea ice & 0.0005 \\[0.5ex] % BKL97 p. F-4 (updated)
%Sea ice & 0.05 \\[0.5ex] % BKL97 p. F-4 (old)
%Sea ice & 0.04 \\[0.5ex] % CCM:dom/parpbl.h
Bare ground & 0.05 \\[0.5ex] % Bon96 p. 59
Snow & 0.04 \\[0.5ex] % Bon96 p. 59 CCM:lsm/snoconi.F
\footnote{Maritime $\rghmmn$ depends strongly on surface conditions, see, e.g., \cite{LaP81}}%
Ocean & 0.0001 \\[0.5ex] % CCM:dom/parpbl.h
\hline
\end{tabular}
\end{center}
\end{minipage}
\end{table}

The roughness length of fluids depends on the surface wind shear.
The shear stress deforms the fluid and generate ripples or waves at
higher wind speeds.
These waves, in turn, modify the surface roughness length.
\cite{Cha83} pointed out that, in certain circumstances, wind drag
entrains very similar amounts of surface mass into the atmosphere over
many different surface types.
He assembled a variety of observational data which showed that
saltation of sand and snow was analogous to entrainment of sea-spray
over the ocean.
This agreement could be explained by assuming $\rghmmn$ was
proportional to~$\wndfrc^{2}/\grv$ over each of the surfaces.
Over oceans, we adopt the dependence of~$\rghmmn$ on~$\wndspd$
proposed by \cite{LaP82}
\begin{eqnarray}
\hgtrfr & = & 10.0\ \mbox{m} \\
\wndrfr & = & \wndspd ( \hgtrfr ) = \wndten \\
\wndspddps & = & \max ( 1.0 , \wndten) \\
\xchcffmmnntr & = & \frac{0.0027}{\wndspddps} + 0.000142 + 
0.0000764 \wndspddps \\ % LaP82 CCM:dom/flxoce(), NOS97 p. I-2
\rghmmn & = & 10.0 \exp( - \vonkrmcst / \sqrt{\xchcffmmnntr} ) \\ % BKL97 p. F-4 fxm
\zpdmmn & = & 0.0 %
\label{eqn:rgh_mmn_ocn_dfn}
\end{eqnarray}
The reference height~$\hgtrfr$ is the height at which the neutral
exchange coefficient~$\xchcffmmnntr$ is determined. 
In theory this could be any height but in practice $\xchcffmmnntr$ is
measured and parameterized as a function of the wind speed at $10$\,m
\cite[e.g.,][]{LaP81, NOS97, BKL97}.
Thus we use $\wndten$ instead of~$\wndrfr$.
The constraint that $\wndspddps > 1$\,\mxs\ prevents surface exchanges
from vanishing at small wind speeds.

Note that $\xchcffmmnntr$ (\ref{eqn:rgh_mmn_ocn_dfn}) is 
the \textit{neutral} drag coefficient, not the drag coefficient.
Stability-based corrections must be applied to~$\xchcffmmnntr$ in
order to obtain~$\xchcffmmn$. 
\cite{LaP81} summarize the procedure used to convert~$\xchcffmmnntr$
to a (non-neutral) drag coefficient, shifted to any height~$\hgt$:
\begin{eqnarray}
% LaP81 p. 327 (15), LaP82 p. 466 (10)
\xchcffmmn(\hgt) & = & \xchcffmmnntr \left\{ 1 + 
\frac{\sqrt { \xchcffmmnntr}}{\vonkrmcst} 
\left[ \ln \left( \frac{\hgtatm}{\hgtrfr} \right) 
- \psimmn \left( \frac{\hgtatm}{\mnolng} \right) \right] \right\}^{-2}
\label{eqn:cff_xch_cff_xch_ntr}
\end{eqnarray}
When $\hgtatm = \hgtrfr$, then $\xchcffmmn$ is the drag coefficient at
the reference height.

Exchange coefficients such as the drag coefficient $\xchcffmmn$
(\ref{eqn:cff_xch_cff_xch_ntr}) are positive-definite by definition,
e.g., (\ref{eqn:wnd_str_dfn}). 
Some care must be taken to ensure numerical procedures do not
erroneously predict negative-valued exchange coefficients.
For example, (\ref{eqn:cff_xch_cff_xch_ntr}) predicts
$\xchcffmmn(\hgt) < 0$ when \ldots fxm % fxm

\subsection[Stability Corrections]{Stability Corrections}
As mentioned above, the vertical profile of momentum in the boundary
layer is, to a first approximation, logarithmic with height
(\ref{eqn:wnd_ntr_dfn}). 
However, the stability properties of the atmosphere introduce a
correction 
\begin{eqnarray}
\wndspd ( \hgtatm_{2} ) - \wndspd ( \hgtatm_{1} ) & = & 
\frac{\wndfrc}{\vonkrmcst} \left[
\ln \left( \frac{\hgtatm_{2} - \zpdmmn}{\hgtatm_{1} - \zpdmmn} \right) 
- \psimmn \left( \frac{\hgtatm_{2} - \zpdmmn}{\mnolng} \right)
+ \psimmn \left( \frac{\hgtatm_{1} - \zpdmmn}{\mnolng} \right)
\right]
\end{eqnarray}
where $\psimmn$ is the \trmdfn{stability correction factor} for
momentum. 

The \trmdfn{stability parameter} $\stbprm$ is the ratio of the height
$\hgtatm$ to the Monin-Obukhov length $\mnolng$\,[\m] (defined below).
\begin{eqnarray}
\stbprm & = & \hgtatm / \mnolng
\label{eqn:stb_prm_dfn}
\end{eqnarray}
The stability parameter is the non-dimensionalized height in the
surface turbulence equations.  

% SeP97 p. 868
In the surface layer, the stability parameter equals the 
\trmidx{flux Richardson number}, $\stbprm = \rchnbrflx$.
The flux Richardson number is the ratio of the production (or loss) of
turbulent kinetic energy (\trmidx{TKE}) by buoyancy to the production
of turbulent kinetic energy by shear stresses.
When $\rchnbrflx = 1$, turbulence is consumed by buoyancy as fast as
it is produced by shear stress.
This is why the similarity functions defined below have discontinuous
first derivatives at $\stbprm = 1$.
This behavior is well-documented in nature.

It is convenient to define separate stability parameters for the
processes of momentum, vapor, and heat transfer in the boundary
layer: 
\begin{eqnarray}
\label{eqn:stb_prm_mmn_dfn}
\stbprmmmn & = & \rghmmn / \mnolng \nonumber \\
\stbprmvpr & = & \rghvpr / \mnolng \\
\stbprmheat & = & \rghheat / \mnolng \nonumber
\end{eqnarray}

The correction factor $\psimmn$ is defined in terms of the 
\trmdfn{similarity function} $\smlfnc$ via
\begin{eqnarray}
\psimmn(\stbprm) & = & \int_{\rghmmn}^{\stbprm} 
\frac{[1-\smlfncmmn(\xxx)]}{\xxx} \,\dfr\xxx
\label{eqn:stb_crc_dfn}
\end{eqnarray}
Correction factors $\psivpr$ and $\psiheat$ for vapor and heat
transfer are defined analogously to (\ref{eqn:stb_crc_dfn}).

\subsection[Flux-Gradient Relationships]{Flux-Gradient Relationships}
\textit{Monin-Obukhov similarity theory} is a more physically accurate
description of surface turbulent fluxes than the bulk aerodynamic
formulation (\ref{eqn:blk_flx_dfn}).
We may re-write in terms of the mean gradients of the scalar
properties and their respective transfer resistances.
% fxm make better looking
\begin{subequations}
% Ary88 p. 167, Bru82 p. 71, SeP97 p. 869, Bon96 p. 52, Bon02 p. 229
\label{eqn:flx_sfc_rss_dfn}
\begin{align}
\label{eqn:flx_sfc_rss_mmn_dfn}
\flxmmn & = \dnsatm (\wndrfr-\wndsfc)/\rssmmn \\ % Bon02 p. 229
\label{eqn:flx_sfc_rss_heat_dfn}
\flxsns & = -\heatcpcspcprs \dnsatm (\tptrfr-\tptsfc)/\rssheat \\ % Bon02 p. 229
\label{eqn:flx_sfc_rss_vpr_dfn}
\flxltn & = -\ltnspc \dnsatm (\mmrvprrfr-\mmrvprsfc)/\rssvpr % Bon02 p. 229
\end{align}
\end{subequations}

The Monin-Obukhov similarity theory relates surface turbulent flux to
the mean gradients of the scalar quantities in the surface (constant
flux) layer.
\begin{subequations}
% Bon02 p. 229
\label{eqn:flx_trb_sfc_flx_grd_dfn}
\begin{align}
\label{eqn:flx_trb_sfc_flx_grd_mmn_dfn}
\flxmmn & = \dnsatm \left[ \frac{\vonkrmcst\wndfrc(\hgt-\hgtzpd)}{\smlfncmmn(\stbprm)} \right] \frac{\partial\wndspd}{\partial\hgt} \\ % Bon02 p. 230
\label{eqn:flx_trb_sfc_flx_grd_heat_dfn}
\flxsns & = -\heatcpcspcprs \dnsatm \left[ \frac{\vonkrmcst\wndfrc(\hgt-\hgtzpd)}{\smlfncheat(\stbprm)} \right] \frac{\partial\tpt}{\partial\hgt} \\ % Bon02 p. 230
\label{eqn:flx_trb_sfc_flx_grd_vpr_dfn}
\flxltn & = -\ltnspc \dnsatm \left[ \frac{\vonkrmcst\wndfrc(\hgt-\hgtzpd)}{\smlfncheat(\stbprm)} \right] \frac{\partial\mmrvpr}{\partial\hgt} % Bon02 p. 230
\end{align}
\end{subequations}
The relationship between surface fluxes and vertical gradients is
expressed in terms of similarity functions which express the
theoretical forms of the transfer resistances in
(\ref{eqn:flx_sfc_rss_dfn}). 

\subsection[Similarity Functions]{Similarity Functions}
Many studies have constructed empirical similarity functions from 
boundary layer experiments.
We adopt forms for the similarity functions summarized by
\cite[p.~71]{Bru82} and \cite{ZZD98}:
\begin{subequations}
% Ary88 p. 167, Bru82 p. 71, SeP97 p. 869, Bon96 p. 52
\label{eqn:sml_fnc_dfn}
\begin{align}
\label{eqn:sml_fnc_dfn_stb}
\smlfncmmn(\stbprm) = \smlfncheat(\stbprm) = \smlfncvpr(\stbprm) & = 
1 + 5 \stbprm \qquad \qquad \stbprm \ge 0 \\
\label{eqn:sml_fnc_dfn_mmn_uns}
\smlfncmmn(\stbprm) & = 
( 1 - 16 \stbprm )^{-1/4} \qquad \stbprm < 0 \\
\label{eqn:sml_fnc_dfn_heat_uns}
\smlfncheat(\stbprm) = \smlfncvpr(\stbprm) & = 
( 1 - 16 \stbprm )^{-1/2} \qquad \stbprm < 0
\end{align}
\end{subequations}
Hence the similarity function for momentum, heat, and vapor are equal  
in stable conditions, and differ slightly in unstable conditions.
For lack of evidence to the contrary, the forms for heat and vapor are
identical and only the forms for heat will be explicitly presented
in the following.
The numerical constants appearing in (\ref{eqn:sml_fnc_dfn}) are
subject to slight alteration depending on the exact value of the
\trmidx{Von~Karman constant} (we assume $\vonkrmcst = 0.4$) and the
dataset employed.  
\cite[pp.~162--165]{Ary88} shows that (\ref{eqn:sml_fnc_dfn}) fits
data from Kansas with astonishing fidelity.

\cite{ZZD98} find that the similarity functions for very stable
($\stbprm > 1$) and very unstable ($\stbprmmmn < -1.574$, 
$\stbprmheat < -0.465$) conditions approach different limits than
(\ref{eqn:sml_fnc_dfn}).
For completeness, we present the \cite{ZZD98} recommendations for
the entire range of $\stbprm$ for momentum
\begin{eqnarray}
% ZZD98 p. 2629 (2)-(6)
\smlfncmmn(\stbprm) & = & \left\{
\begin{array}{l@{\quad:\quad}r}
0.7\vonkrmcst^{2/3}(-\stbprm)^{-1/3} & \stbprm < -1.574 \\
( 1 - 16 \stbprm )^{-1/2} & -1.574 \le \stbprm < 0 \\
1 + 5 \stbprm & 0 \le \stbprm < 1 \\
5 + \stbprm & 1 \le \stbprm  \\
\end{array} \right.
\label{eqn:sml_fnc_mmn_ttl_dfn}
\end{eqnarray}
and for heat (and vapor)
\begin{eqnarray}
% ZZD98 p. 2629 (2)-(6)
\smlfncheat(\stbprm) & = & \left\{
\begin{array}{l@{\quad:\quad}r}
0.9\vonkrmcst^{4/3}(-\stbprm)^{-1/3} & \stbprm < -0.465 \\
( 1 - 16 \stbprm )^{-1/2} & -0.465 \le \stbprm < 0 \\
1 + 5 \stbprm & 0 \le \stbprm < 1 \\
5 + \stbprm & 1 \le \stbprm  \\
\end{array} \right.
\label{eqn:sml_fnc_heat_ttl_dfn}
\end{eqnarray}
The locations of the functional interfaces in
(\ref{eqn:sml_fnc_mmn_ttl_dfn}) and (\ref{eqn:sml_fnc_heat_ttl_dfn})
were, to some degree, chosen to ensure the stability functions evenly
match eachother. 

\subsection[Monin-Obukhov Length]{Monin-Obukhov Length}\label{sxn:mno_lng}
The \trmidx{Monin-Obukhov length} $\mnolng$\,[\m] characterizes the
stability of a fluid.
There are many equivalent definitions of $\mnolng$, such as
\begin{eqnarray}
\label{eqn:mno_lng_dfn}
% ZZD98 p. 2629 (4)
\mnolng & = & \frac{\tptptnvrt\wndfrc^{2}}{\vonkrmcst\grvsfc\tptptnvrtstr}
\end{eqnarray}
where $\tptptnvrt$\,[\K] is the \trmidx{virtual potential temperature} and
$\tptptnvrtstr$\,[\K] is the scaling parameter for temperature.
The Monin-Obukhov length is the height above the ground at which
the production of turbulence by mechanical (shear) and thermal
(buoyancy) forces are equal.
$\mnolng$ increases with friction velocity (shear turbulence), and
decreases with buoyancy flux (convective turbulence).
Positive and negative $\mnolng$ indicate stable and unstable
atmospheres, respectively.
When the magnitude of $\mnolng$ is very large, e.g., 
$|\mnolng| > 10^{5}$, the atmosphere is neutrally stable.

The procedure for estimating the surface turbulent fluxes often begins
with an initial estimate of the Monin-Obukhov length~$\mnolng$.
The following method is based on \cite{ZZD98} and is used in
\trmidx{CLM} \cite[][]{DZD03}.
We use five quantities to estimate~$\mnolng$:
First, the wind speed at reference height $\wndrfr$.
Second and third, the virtual potential temperatures at reference
height $\tptptnvrtrfr$ and at surface $\tptptnvrtsfc$\,[\K] (or,
equivalently, their difference 
$\dlttptptnvrt \equiv \tptptnvrtrfr-\tptptnvrtsfc$).
Fourth, the reference height $\hgtrfr$\,[\m].
Fifth and finally, the roughness length for momentum $\rghmmn$\,[\m]. 
These five quantities may be measured or, in a model, their values
from the previous timestep may be used.

Given these quantities, we first estimate a \trmidx{flux wind speed}
$\wndflx$\,[\mxs] which depends on stability.
We define the term ``flux wind speed'' operationally as the wind speed
used to compute surface fluxes.
In stable conditions, the flux wind speed is simply the vector
magnitude of the mean zonal and meridional components, $\wndznl$
and $\wndmrd$. 
\begin{eqnarray}
% ZZD98 p. 2630 (18,19), CLM MoninObukIni.F90
\wndflx(\dlttptptnvrt) & = & \left\{
\begin{array}{l@{\quad:\quad}l}
\max(\sqrt{\wndznl^{2}+\wndmrd^{2}},0.1) & \dlttptptnvrt \ge 0 
\quad \mbox{(Neutral or Stable)} \\
\sqrt{\wndznl^{2}+\wndmrd^{2}+(\beta\wndcnvscl)^{2}} & \dlttptptnvrt <
0 \quad \mbox{(Unstable)}
\end{array} \right.
\label{eqn:wnd_flx_dfn}
\end{eqnarray}
The minimal value of $0.1$\,\mxs\ avoids singularities and represents
the small eddies present even in a becalmed, stable atmosphere.
In unstable conditions, $\wndflx$ includes the 
\trmidx{convective velocity scale} $\wndcnvscl$ as well. 
The inclusion of $\wndcnvscl$ (\ref{eqn:wnd_flx_dfn}) helps 
$\wndflx$ account for the contributions of large eddies in the
convective boundary to the surface fluxes \cite[][]{ZZD98}. 

After the flux wind speed is estimated (\ref{eqn:wnd_flx_dfn}), 
we obtain the bulk \trmidx{gradient Richardson number} $\rchnbr$ 
\begin{eqnarray}
\label{eqn:rch_nbr_dfn}
% CLM MoninObukIni.F90
\rchnbr & = & \frac{\grvsfc\dlthgt\dlttptptnvrt}{\tptptnvrt\wndflx^{2}}
\end{eqnarray}
where $\dlthgt \equiv \hgtrfr-\zpdmmn$ is the difference between
the reference height~$\hgtrfr$ and the \trmidx{zero plane
  displacement} height~$\zpdmmn$.
The gradient Richardson number characterizes the turbulence intensity.
Formally, $\rchnbr$ is the ratio of convectively available potential 
energy (\trmidx{CAPE}) to the magnitude of low-level shear. 
Hence (\ref{eqn:rch_nbr_dfn}) is simply a first estimate
for~$\rchnbr$. 

We use $\rchnbr$ to form an initial guess for~$\stbprmmmn$
(\ref{eqn:stb_prm_mmn_dfn}) 
\begin{eqnarray}
% CLM MoninObukIni.F90 ZZD98 p. 2629 (4)
\stbprmmmn(\rchnbr) & = & \left\{
\begin{array}{l@{\quad:\quad}l}
\rchnbr \ln(\dlthgt/\rghmmn) / [1.0-5.0\min(\rchnbr,0.19)] & \rchnbr \ge 0 
\quad \mbox{(Neutral or Stable)} \nonumber \\
\rchnbr \ln(\dlthgt/\rghmmn) & \rchnbr < 0
\quad \mbox{(Unstable)} \nonumber
\end{array} \right.
\label{eqn:stb_prm_est}
\end{eqnarray}
However, these guesses allow $\stbprmmmn$ outside the range of
measured values. 
Hence we contrain the results of (\ref{eqn:stb_prm_est}) as follows:
\begin{eqnarray}
% CLM MoninObukIni.F90 ZZD98 p. 2629 (4)
\stbprmmmn(\rchnbr) & = & \left\{
\begin{array}{l@{\quad:\quad}l}
\min[2.0,\max(\stbprmmmn,0.01)] & \rchnbr \ge 0 
\quad \mbox{(Neutral or Stable)} \nonumber \\
\max[-100.0,\min(\rchnbr,-0.01)] & \rchnbr < 0
\quad \mbox{(Unstable)} \nonumber
\end{array} \right.
\label{eqn:stb_prm_est_bnd}
\end{eqnarray}

Finally, we estimate $\mnolng$ from $\stbprmmmn$
(\ref{eqn:stb_prm_est_bnd}) and $\rghmmn$ using
(\ref{eqn:stb_prm_mmn_dfn})  
\begin{eqnarray}
% CLM MoninObukIni.F90 ZZD98 p. 2629 (4)
\mnolng = (\hgtrfr-\zpdmmn)/\stbprmmmn 
\label{eqn:mno_lng_est}
\end{eqnarray}
This initial estimate for~$\mnolng$ (\ref{eqn:mno_lng_est}) is refined
in subsequent iterations. 

\csznote{
\subsection[Latent Heat Flux]{Latent Heat Flux}\label{sxn:lhf}
The latent heat flux fxm
} % end csznote

\section[Wind Speed]{Wind Speed}

\subsection[Reference Level Wind Speed]{Reference Level Wind Speed}
The single most important parameter determining surface erosion is the
wind speed. 
In practice we predict or measure the wind speed $\wndspd$ at a
certain height above the ground and wish to relate $\wndspd$ to an
observed or predicted dust flux.
It is convenient to define a reference height $\hgtrfr$ so that
$\wndspd$ and $\flxmsshrz$ may be intercompared at disparate locations. 
By convention $\hgtrfr$ is set to $10$~m, which is a standard height
both for field observations and for numerical model output.
As discussed below, it is actually the wind friction speed which is
directly linked to dust mobilization.

\chapter[Dust Source Processes]{Dust Source Processes}\label{sxn:mbl}
\cszepigraph{%
A lot of real estate changin' hands today.}
{}{Saying in West Texas}{Anonymous} 
% 20082003: Contributed by Dick Bowdon '74 SSP Executive Director
% along with "Lubbock is the only place you can be up to your ass in
% mud and still get sand in your eyes."

This section describes the wind driven processes which determine
the flux of long-lived mineral dust into the atmosphere.
The atmospheric dust burden is the end result of a chain of processes
which begins with \trmdfn{saltation}.
Saltation is the wind-initiated movement of large soil particles in
the downstream direction.
Particles large enough to be entrained into motion directly by wind
are called \trmdfn{sand}.
Sand ranges in \trmdfn{texture} (size) from \trmdfn{very fine} to
\trmdfn{coarse}, $1/16 < \dmt < 2$\,\mm\ ($62 < \dmt < 2000$\,\um) 
respectively. 
Smaller particles of crustal material, those susceptible to long term
suspension and transport in the atmosphere, are collectively referred
to as \trmdfn{dust}.
When the drag on the particles is large enough to overcome the
inertial and cohesive forces attaching the particle to the soil bed,
the particle suddenly lifts from the bed in a nearly vertical
trajectory.
Blown from behind and unable to resist gravitational settling for very
long, the particle arcs back to the surface along a shallower
trajectory.  
The energy dissipated by the impact of the saltating particles on the
surface, may, in turn, break adhesive bonds and liberate much smaller
dust particles known as dust from the surface.
Although the distinction is somewhat arbitrary, dust particles are
smaller than the population of saltators which liberate them. 

The size of dust particles is generally taken to be $\dmt < 50$\,\um.
Since the time a particle requires to settle out of the atmosphere
under the influence of gravitation alone is proportional to the square
of the particle diameter (see \S\ref{sxn:dry_dps} below), dust
particles have significantly longer atmospheric residence times than
sand.

\section{Literature Review}\label{mbl_ltr}
Two books are devoted to physical mechanisms of wind erosion, 
\cite{Bag41} and \cite{Sha00}.
\cite{RaL04} provide an excellent review of all aspects of dust
mobilization and dry deposition.

\section{Threshold Wind Friction Speed}\label{sxn:wnd_frc_thr}
The conditions necessary to initiate or maintain the movement of
particles depend on many properties, including wind speed, particle
size, surface roughness, stability, sheltering effects, surface
moisture, and interparticle forces.
Laboratory experiments have been performed over a wide range of
ambient pressures and particle densities to isolate and define these
dependencies \cite[]{IvW82,SRL96}.
We now characterize the aerodynamic forces on particles which
initiate and maintain particle motion which in turn causes saltation, 
dust mobilization and wind erosion.

\subsection[Bagnold's Original Theory]{Bagnold's Original Theory}\label{sxn:thr_Bag41}
\cite{Bag41} first derived the size dependence of the threshold wind
velocity by considering the balance of forces on a particle.
His lucid derivation, though lacking in completeness, nevertheless
provides the starting point for more complete theory of mobilization.
Consider a particle of density $\dnsprt$ resting atop a bed of similar
particles in a fluid of density $\dnsatm$ traveling above and around
the particles with speed $\wndfrc$. 
Define the \trmdfn{packing angle} $\nglpck$ as the angle from the
vertical subtended by the downstream point of contact $\pnt$ of the
particle with the bed it lies atop.
If the particles are all spherical and regularly packed, then $\nglpck$
is the angle from the center of the upper sphere to the center of one
of the lower spheres.

The forces considered by \cite{Bag41} are drag and gravity.
Bagnold assumes particle movement initiates when the aerodynamic
drag exerted by the wind on the particle overcomes the component of the
particle's weight directed opposite to the wind stream.
A more complete physical statement of the balance of forces leading to
saltating is that particle motion commences when the sum of the
forces acting on the particle result in a net moment of zero at the
downstream point of contact $\pnt$. 
We shall use this more general principle to include the cohesive and
lift forces later; for now, we retrace Bagnold's original derivation.

The mass of the particle is $\mpi \dmt^{3} \dnsprt / 6$ and its
net weight relative to the surrounding fluid is $\mpi \grv \dmt^{3}
(\dnsprt - \dnsatm) / 6$. 
The contact point $\pnt$ is the pivot, or axis of support, about which
gravity attempts to pull the particle downwards and backwards.
The gravitational torque acts through the center of mass of the
particle to create a \trmdfn{moment} of force. 
The gravitational moment $\mmngrv$ is the product of the gravitational
force and the distance between the axis through which gravity acts
(i.e., the center of mass) and the point $\pnt$.
This distance is $(\dmt \sin \nglpck)/2$, and thus 
\begin{eqnarray}
% Bag41 p. 86 
\mmngrv & = & \frac{\mpi \grv \dmt^{3} (\dnsprt - \dnsatm)}{6} \times 
\frac{\dmt}{2} \sin \nglpck \nonumber \\
& = & \frac{\mpi \grv \dmt^{4} (\dnsprt - \dnsatm) \sin \nglpck}{12}
\label{eqn:mmn_grv_dfn}
\end{eqnarray}

The horizontal drag force $\mmndrg$ exerted on the particle by the
fluid is proportional to the exposed cross sectional area of the
particle normal to the fluid flow, $\mmndrg \propto \dmt^{2}$. 
The drag per unit particle surface area is proportional to $\dnsatm  
\wndfrc^{2}$. 
Assuming the net drag force is directed through the center of mass of
the particle, then the distance from the axis of drag force to $\pnt$ 
is $(\dmt \cos \nglpck)/2$, and thus the drag moment is
\begin{eqnarray}
% Bag41 p. 86 
\mmndrg & = & \cst_{1} \times \dnsatm \wndfrc^{2} \times \dmt^{2}  \times
\frac{\dmt}{2} \cos \nglpck \nonumber \\
& = & \cst_{2} \dnsatm \wndfrc^{2} \dmt^{3} \cos \nglpck
\label{eqn:mmn_drg_dfn}
\end{eqnarray}
where $\cst_{2} = \cst_{1} / 2$ is a constant of proportionality which
depends upon the exact grain geometry, micrometeorology, and grain
packing. 

At the threshold velocity $\wndfrcthr$, the gravitational moment
equals the drag moment on the particle so that any small perturbation
in wind speed may initiate particle motion, i.e., saltation.  
Equating $\mmngrv$ (\ref{eqn:mmn_grv_dfn}) and $\mmndrg$
(\ref{eqn:mmn_drg_dfn}) 
\begin{eqnarray}
\frac{\mpi \grv \dmt^{4} (\dnsprt - \dnsatm) \sin \nglpck}{12} 
& = & \cst_{2} \dnsatm \wndfrcthr^{2} \dmt^{3} \cos \nglpck
\nonumber \\
\wndfrcthr^{2} & = & \left( \frac{\mpi}{12 \cst_{2}} \right) \tan \nglpck
\frac{\grv \dmt (\dnsprt - \dnsatm)}{\dnsatm} 
\nonumber \\
\wndfrcthr & = & \AAA
\sqrt{ \frac{\grv \dmt (\dnsprt - \dnsatm)}{\dnsatm}}
\label{eqn:wnd_frc_thr_bgn_dfn}
\end{eqnarray}
where $\AAA^{2} \equiv ( \frac{\mpi}{12 \cst_{2}} ) \tan \nglpck$.
For particles in air, $\dnsprt \gg \dnsatm$ and so
(\ref{eqn:wnd_frc_thr_bgn_dfn}) becomes
\begin{equation}
% IvW82 p. 114 (4) 
\wndfrcthrdmt \approx \AAA \sqrt{ \frac{\dnsprt \grv \dmt}{\dnsatm}}
\label{eqn:wnd_frc_thr_bgn_prx}
\end{equation}

The preceding discussion applies to the initiation of particle
motion in a bed of grains initially at rest.
Thus $\wndfrcthr$ defined by (\ref{eqn:wnd_frc_thr_bgn_dfn}) is called
the \trmdfn{fluid threshold friction velocity} or \trmdfn{aerodynamic
threshold friction velocity}.  
Once particle motion begins, the threshold velocity actually
\textit{decreases} since some of the momentum needed to initiate
further particle motion is supplied by particles already in motion. 
The friction velocity in an environment of particles already in motion
is known as the \trmdfn{impact threshold friction velocity}
$\wndfrcthrimp$.
For large sand grains, \cite{Bag41} found that 
$\wndfrcthrimp \sim 0.8 \wndfrcthr$
The effect of particle motion on the surface wind speed and its
feedbacks to further particle motion will be discussed further in
\S\ref{sxn:hrz} below. 

Subsequent developers of this theory have continued to use the
parameter $\AAA$ as it appears in (\ref{eqn:wnd_frc_thr_bgn_dfn}) and 
in (\ref{eqn:wnd_frc_thr_bgn_prx}), i.e., as the proportionality
factor between $\wndfrcthr$ and the factor containing the square root
of the particle diameter. 
\begin{equation}
\AAA \approx \wndfrcthr \sqrt{ \frac{\dnsatm}{\grv \dnsprt \dmt}}
\label{eqn:AAA_dfn}
\end{equation}
$\AAA$ is called the \trmdfn{dimensionless threshold friction speed},
or simply the \trmdfn{threshold parameter} \cite[]{IvW82}.
Theoretical approximations and empirical parameterizations of $\AAA$
will be discussed in \S\ref{sxn:thr_GrI85} below.

\cite{Bag41} developed (\ref{eqn:wnd_frc_thr_bgn_dfn}) for particles large
enough to appear as isolated elements to the fluid flow. 
In particular, experiments confirm the predictions of
(\ref{eqn:wnd_frc_thr_bgn_dfn}) only for particle sizes larger than about
200\,\um. 
$\AAA \sim 0.1$ for large particles in air and $\AAA \sim 0.2$ for
particles in water \cite[][p. 88]{Bag41}. 
In this size range $\wndfrcthr$ increases linearly with $\dmt^{1/2}$.
For smaller particles, however, it is found that cohesive and lift
forces may not be neglected, and that these forces combine to produce
an optimum size for particle saltation, i.e., a minimum in
$\wndfrcthr(\dmt)$ not suggested by (\ref{eqn:wnd_frc_thr_bgn_dfn}).

\subsection[Reynolds Number]{Reynolds Number}
In order to develop a more complete theory of saltation and dust
emissions, let us first examine the physics of airflow around a single 
particle. 
The balance of these forces determine when particle motion is
initiated by airflow.
Consider an isolated particle of size $\dmt$ at rest in a
\trmdfn{Newtonian fluid} of density $\dnsatm$ moving with velocity
$\vlcvec$.  
For simplicity, we consider the component of motion in the $\xxx$
direction. 
As shown in \S\ref{sxn:fld_mch}, the equation of continuity  
(conservation of mass) for an incompressible fluid
(\ref{eqn:ncm_dfn_crt}) requires   
\begin{equation}
\frac{\partial \vlcxxx}{\partial \xxx} + 
\frac{\partial \vlcyyy}{\partial \yyy} + 
\frac{\partial \vlczzz}{\partial \zzz} = 0 
\label{eqn:ncm_crt}
\end{equation}
and momentum conservation in the $\xxx$ direction
(\ref{eqn:elr_eom_vec}) requires (\ref{eqn:mmn_cns_scl})
\begin{equation}
\frac{\partial \vlcxxx}{\partial \tm} + 
\left( 
\vlcxxx \frac{\partial \vlcxxx}{\partial \xxx} +
\vlcyyy \frac{\partial \vlcxxx}{\partial \yyy} +
\vlczzz \frac{\partial \vlcxxx}{\partial \zzz} 
\right) =  
-\frac{1}{\dnsatm} \frac{\partial \prs}{\partial \xxx} +
\frac{\vscdyn}{\dnsatm} 
\left( 
\frac{\partial^{2} \vlcxxx}{\partial \xxx^{2}} +
\frac{\partial^{2} \vlcxxx}{\partial \yyy^{2}} +
\frac{\partial^{2} \vctxxx}{\partial \zzz^{2}} 
\right) 
\label{eqn:mmn_cns_ryn}
\end{equation}

We can gain insight into the relevant physical processes by
\trmdfn{non-dimensionalizing} the equations.
We do this by changing to a non-dimensional coordinate space which is
scaled by the relevant physical dimensions of the fluid flow.
For fluid flow around an isolated particle at rest, the relevant
velocity is the upstream velocity $\vlcinf$ far from the particle.
The relevant length scale is $\dmt$.
\begin{equation}
\xxxtld = \frac{\xxx}{\dmt}, \qquad
\yyytld = \frac{\yyy}{\dmt}, \qquad
\zzztld = \frac{\zzz}{\dmt}, \qquad
\vlcxxxtld = \frac{\vlcxxx}{\vlcinf}, \qquad
\vlcyyytld = \frac{\vlcyyy}{\vlcinf}, \qquad
\vlczzztld = \frac{\vlczzz}{\vlcinf}
\label{eqn:cov_ndm_1}
\end{equation}
The dimensionless time and pressure variables are 
\begin{equation}
\tmtld = \frac{\tm \vlcinf}{\dmt} \qquad\mbox{and}\qquad
\prstld = \frac{\prs}{\dnsatm \vlcinf^{2}}
\label{eqn:cov_ndm_2}
\end{equation}
Thus, by definition, none of the transformed coordinates has any
physical dimensions.
Substituting the non-dimensional coordinate definitions into
(\ref{eqn:ncm_crt}) we obtain
\begin{eqnarray}
\left( \vlcinf \times \frac{1}{\dmt} \right) \left(
\frac{\partial \vlcxxxtld}{\partial \xxxtld} + 
\frac{\partial \vlcyyytld}{\partial \yyytld} + 
\frac{\partial \vlczzztld}{\partial \zzztld} \right) & = & 0 \nonumber \\
\frac{\partial \vlcxxxtld}{\partial \xxxtld} + 
\frac{\partial \vlcyyytld}{\partial \yyytld} + 
\frac{\partial \vlczzztld}{\partial \zzztld} & = & 0
\label{eqn:ncm_crt_ndm}
\end{eqnarray}
where the constant factor $\vlcinf / \dmt$ results from the coordinate 
transformation. 
Thus the incompressible continuity equation does not change in the
transformed coordinate system.
Substituting the non-dimensional coordinate definitions into
(\ref{eqn:mmn_cns_ryn}), and using $\vscknm \equiv \vscdyn/\dnsatm$
(\ref{eqn:vsc_knm_dfn}),  
\begin{eqnarray}
\lefteqn{ % KoD95 p. 136
\left( \vlcinf \times \frac{\vlcinf}{\dmt} \right) 
\frac{\partial \vlcxxxtld}{\partial \tmtld} + 
\left( \vlcinf \times \vlcinf \times \frac{1}{\dmt} \right)
\left( 
\vlcxxxtld \frac{\partial \vlcxxxtld}{\partial \xxxtld} +
\vlcyyytld \frac{\partial \vlcxxxtld}{\partial \yyytld} +
\vlczzztld \frac{\partial \vlcxxxtld}{\partial \zzztld} 
\right) = } \nonumber \\
& &
- \frac{1}{\dnsatm}
\left( \dnsatm \vlcinf^{2} \times \frac{1}{\dmt} \right)
\frac{\partial \prstld}{\partial \xxxtld} +
\vscknm
\left( \vlcinf \times \frac{1}{\dmt^{2}} \right)
\left( 
\frac{\partial^{2} \vlcxxxtld}{\partial \xxxtld^{2}} +
\frac{\partial^{2} \vlcxxxtld}{\partial \yyytld^{2}} +
\frac{\partial^{2} \vlcxxxtld}{\partial \zzztld^{2}} 
\right)
\label{eqn:mmn_cns_ryn_ndm}
\end{eqnarray}
Factoring the physical scales from each side we obtain
\begin{equation}
\frac{\vlcinf^{2}}{\dmt} \left[
\frac{\partial \vlcxxxtld}{\partial \tmtld} + 
\left( 
\vlcxxxtld \frac{\partial \vlcxxxtld}{\partial \xxxtld} +
\vlcyyytld \frac{\partial \vlcxxxtld}{\partial \yyytld} +
\vlczzztld \frac{\partial \vlcxxxtld}{\partial \zzztld} 
\right) \right] 
= 
- \frac{\vlcinf^{2}}{\dmt} 
\frac{\partial \prstld}{\partial \xxxtld} +
\frac{\vlcinf \vscknm}{\dmt^{2}} 
\left( 
\frac{\partial^{2} \vlcxxxtld}{\partial \xxxtld^{2}} +
\frac{\partial^{2} \vlcxxxtld}{\partial \yyytld^{2}} +
\frac{\partial^{2} \vlcxxxtld}{\partial \zzztld^{2}} 
\right)
\nonumber
\end{equation}
Multiplying each side by $\dmt / \vlcinf^{2}$ we find
\begin{equation}
\frac{\partial \vlcxxxtld}{\partial \tmtld} + 
\vlcxxxtld \frac{\partial \vlcxxxtld}{\partial \xxxtld} +
\vlcyyytld \frac{\partial \vlcxxxtld}{\partial \yyytld} +
\vlczzztld \frac{\partial \vlcxxxtld}{\partial \zzztld} 
= 
- \frac{\partial \prstld}{\partial \xxxtld} +
\frac{\vscknm}{\vlcinf \dmt} 
\left( 
\frac{\partial^{2} \vlcxxxtld}{\partial \xxxtld^{2}} +
\frac{\partial^{2} \vlcxxxtld}{\partial \yyytld^{2}} +
\frac{\partial^{2} \vlcxxxtld}{\partial \zzztld^{2}} 
\right)
\label{eqn:mmn_cns_ndm}
\end{equation}
Note that all physical scales in the problem appear in a group which
multiplies the diffusion term.
This single dimensionless factor must determine the behavior of the
entire system. 
The \trmdfn{Reynolds number} $\rynnbr$ is defined as the inverse of
this factor
\begin{equation}
\rynnbr = \vlcinf \dmt / \vscknm 
\label{eqn:ryn_nbr_dfn}
\end{equation}
The Reynolds number expresses the ratio of inertial to viscous
forces. 
The nature of the solutions to (\ref{eqn:mmn_cns_ndm}) is strongly
sensitive to whether $\rynnbr < 1$ or $\rynnbr > 1$.
When $\rynnbr \ll 1$, viscous forces dominate and the LHS of
(\ref{eqn:mmn_cns_ndm}) is small relative to the RHS (because
$\rynnbr^{-1} \gg 1$) and may be neglected.
The steady state behavior of (\ref{eqn:mmn_cns_ndm}) then approaches 
\begin{equation}
\frac{\partial \prstld}{\partial \xxxtld} =
\frac{1}{\rynnbr} 
\left( 
\frac{\partial^{2} \vlcxxxtld}{\partial \xxxtld^{2}} +
\frac{\partial^{2} \vlcxxxtld}{\partial \yyytld^{2}} +
\frac{\partial^{2} \vlcxxxtld}{\partial \zzztld^{2}} 
\right)
\label{eqn:eom_sph}
\end{equation}
The solution to (\ref{eqn:eom_sph}) for a sphere at rest in a fluid
is one of the central results of fluid mechanics.

The Reynolds number of an arbitrary flow is defined as 
\begin{equation}
\rynnbr = \wndspd \dmt / \vscknm 
\label{eqn:ryn_nbr_dfn_2}
\end{equation}
so that $\rynnbr$ varies with position.
The \trmidx{gravitational settling} speed $\vlcgrv$ of particles
smaller than $10$\,\um\ is less than $10^{-2}$\,\mxs.
This implies $\rynnbr < 10^{-2}$ for $\dmt < 10$\,\um\
(\ref{eqn:ryn_nbr_dfn_2}). 
Thus inertial effects may be neglected for most aerosols falling at
\trmidx{terminal velocity}.
However, $\vlcgrv$ increases as the square of the particle diameter
so that $\rynnbr > 1$ for $\dmt > 50$\,\um. 
When $\rynnbr \gg 1$, then inertia plays a large in the motion.
In this case, the flows described by (\ref{eqn:mmn_cns_ndm}) become 
highly turbulent.

The relative velocity between strong winds and an undisturbed sand grain
at the surface, however, easily exceeds $\vlcgrv$.
The relevant wind speed for entraining particles from the
surface is not the atmospheric wind speed, but the \trmidx{friction
velocity} $\wndfrc$.  
Section~\ref{sxn:blp} defines and describes the friction velocity as
the characteristic wind velocity dissipated by shear stress and small
scale turbulent interactions between the surface and the atmosphere.
In complete analogy with (\ref{eqn:ryn_nbr_dfn_2}) we define the
\trmdfn{friction Reynolds number} as
\begin{equation}
\rynnbrfrc = \wndfrc \dmt / \vscknm 
\label{eqn:ryn_nbr_frc_dfn}
\end{equation}
Windy conditions in the free atmosphere, e.g., wind speeds of
5--10\,\mxs, typically correspond to $\wndfrc$ between
10--50\,\cmxs\ over barren surfaces. 
In such conditions, $\rynnbr > 1$ for particles larger than about
20\,\um, and inertial forces should not be neglected.

The friction Reynolds number at the threshold friction velocity is
called the \trmdfn{threshold friction Reynolds number}
\begin{equation}
% IvW82 p. 111 variable ``B''
\rynnbrfrcthr = \wndfrcthr \dmt / \vscknm
\label{eqn:ryn_nbr_frc_thr_dfn}
\end{equation}
The literature of particle mobilization quite often uses 
$\BBB$ instead of $\rynnbrfrcthr$.
The symbols are equivalent, and the following sections shall
use both notations. 

\subsection[Iversen's Theory]{Iversen's Theory}\label{sxn:thr_GrI85}
Our presentation follows the detailed summary of \cite{GrI85}, which
remains the state-of-the-art in theoretical modeling of particle
mobilization in planetary atmospheres.  
Most of the empirical elements were presented in a unified theory in
\cite{IvW82}. 
We begin by reconsidering a loose particle at rest atop a bed of
similar particles.
We shall account for the effects of five moments acting on this
particle.  
The forces creating these moments are drag $\frcdrg$, gravity
$\frcgrv$, lift $\frclft$, interparticle cohesion $\frcipc$, and
rotational inertia.
Drag and gravity were explicitly considered by \cite{Bag41} and were
discussed in \S\ref{sxn:thr_Bag41}. 

The lift force is due to the extremely large velocity gradient above 
the particle. 
For $\rynnbr$ smaller than about $5$, the particle resides in a thin
(1--10\,\mm) quasi-laminar sublayer where the velocity profile is 
\begin{equation}
\wndspd(\hgt) = \frac{\wndfrc^{2} \hgt}{\vscknm}
\label{eqn:wnd_spd_lmn_dfn}
\end{equation}
Thus the velocity shear is constant in the quasi-laminar sublayer 
\begin{equation}
\frac{\partial \wndspd}{\partial \hgt} = \frac{\wndfrc^{2}}{\vscknm}
\label{eqn:wnd_spd_shr_dfn}
\end{equation}
In typical dust-prone conditions on Earth, $\wndfrc = 0.25$\,\mxs\ and 
$\vscknm = 1.3 \times 10^{-5}$\,\mSxs, so that the velocity shear
(\ref{eqn:wnd_spd_shr_dfn}) is nearly $5 \times 10^{3}$\,\mxsm.
The strong velocity shear is an indication that the pressure
distribution along the top of the particle generates a lift force akin 
to an airfoil.
The lift force acts through the center of mass of the particle in the
vertical direction, i.e., opposite to the gravitational force.

Once the particle is lifted out of the quasi-laminar sublayer the lift
force is expected to decrease significantly.
Some particles receive their initial vertical momentum from the impact
of being struck by other saltating particles.
However, the vertical momentum of particles not released due to
ballistic impact is thought to be due to the lift force.
Experiments (Chepil, 1958; Einstein and El-Samni, 1949, MFC) show
that lift and drag forces on hemispheres are of the same order of
magnitude.  
Experiments show qualitatively similar behavior in the lift force
occurs in fully turbulent boundary layers ($\rynnbrfrc > 5$).

Originally, it was thought that the observed optimal size for
initiation of particle saltation was due to aerodynamic effects.
The observations suggested the existence of forces which act in
opposition to the drag on small particles (\ref{eqn:mmn_drg_dfn}). 
\cite{IvW82} demonstrated that \trmdfn{interparticle cohesive forces}
are important for small particles. 
Interparticle cohesive forces include a number of processes including 
moisture, suction, electric charge, and chemical reactions.
Many of these forces may be conceived as acting along the line
connecting the centers of adjacent particles.
The sum of all interparticle cohesive forces is denoted $\frcipc$.

The \trmdfn{angle of repose} of a soil is the maximum inclination at
which the soil does not undergo spontaneous slippage. 
Thus the angle of repose is a qualitative estimate of the importance
of interparticle cohesion. 
The greater the tilt, the more important interparticle cohesion is in
preventing slippage.
The angle between the leeward face of a sand dune and the horizontal
is an excellent proxy for the angle of repose.
For ordinary dune sand, the angle is $34\dgr$ but, for very small 
particles, the angle can approach the vertical \cite[]{IvW82}.   
\cite{Bag41} conjectured that the packing angle $\nglpck$
(\S\ref{sxn:thr_Bag41}) is, on average, close to the angle of repose.

The forces described above act on the particle along axes at varying
distances from the downstream point of contact $\pnt$.  
The distance between the axis of force and $\pnt$ is called the
\trmdfn{moment arm}. 
From the figure we see that the moment arm for drag is $\aaa$, 
for gravity and lift is $\bbb$, and for interparticle cohesion is
$\ccc$. 
Thus the moments associated with each force are  
\begin{eqnarray}
\mmndrg & = & \aaa \frcdrg \nonumber \\
\mmngrv & = & \bbb \frcgrv \nonumber \\
\mmnlft & = & \bbb \frclft \nonumber \\
\mmnipc & = & \ccc \frcipc
\label{eqn:mmn_dfn}
\end{eqnarray}
At the threshold velocity, these moments are assumed to sum to zero 
so that any small perturbation in wind speed may initiate particle
motion, i.e., saltation.
The balance at threshold is 
\begin{eqnarray}
% GrI85 p. 72 (3.3)
\mmndrg + \mmnlft + \mmnrsh & = & \mmngrv + \mmnipc \nonumber \\
\aaa \frcdrg + \bbb \frclft + \mmnrsh & = & \bbb \frcgrv + \ccc \frcipc
\label{eqn:mmn_eqm}
\end{eqnarray}
where the RHS contains the moments which tend to dislodge the
particles and the LHS contains the moments which stabilize the
particles. 
Note the \trmdfn{moment of rotational inertia} $\mmnrsh$ has been 
introduced on the LHS. 
The rotational inertia is not a force per se, rather, it measures
of resistance of an object to angular acceleration about a specified
axis. 
The rotational inertia $\rsh$ of an object is defined as 
$\int \rds^{2} \,\dfr\mss$, where $\rds$ is the distance of the element of
mass $\dfr\mss$ from the origin of coordinates.
A solid sphere has a rotational inertia 
$\rsh = 2 \mss \rds^{2} / 5 = \mss \dmt^{2} / 10$
about its center. 
Assuming the particle is spherical, we apply the parallel axis theorem
to obtain the rotational inertia of the particle about $\pnt$   
\begin{eqnarray}
\mmnrsh & = & { \mss \dmt^{2} / 10} + \mss \left( \frac{\dmt}{2}
\right)^{2} \nonumber \\ 
& = & { 7 \mss \dmt^{2} / 20} \nonumber \\ 
& = & { 7 \mpi \dnsprt \dmt^{5} / 120}
\label{eqn:mmn_rsh_dfn}
\end{eqnarray}
where we have expressed the particle mass in terms of its diameter in
the last step.

It is possible to measure the relative strength of lift, drag, and
rotational inertia by non-dimensionalizing the forces in
(\ref{eqn:mmn_dfn}) and (\ref{eqn:mmn_rsh_dfn}).
Following \cite{GrI85}, the small particle shear flow force
coefficients $\frcprmdrg$, $\frcprmlft$, $\frcprmrsh$, for drag, lift,
and rotational inertia, respectively, are defined as  
\begin{eqnarray}
\label{eqn:mmn_ndm_dfn}
\frcdrg & = & \frcprmdrg \dnsatm \wndfrc^{2} \dmt^{2} \nonumber \\
\frclft & = & \frcprmlft \dnsatm \wndfrc^{2} \dmt^{2} \\
\mmnipc & = & \frcprmrsh \dnsatm \wndfrc^{2} \dmt^{3} \nonumber
\end{eqnarray}
\cite{GrI85} summarize the agreement between the measured and
theoretical values of $\frcprmdrg$, $\frcprmlft$, and $\frcprmrsh$.
Relatively good agreement is found for measured and theoretical values
of the drag parameter, with all values falling in the range $4.65 <
\frcprmdrg < 9.82$ for $\rynnbr = 0.95$.
Wind tunnel inferences of $\frcprmlft$ are much larger (factor of 40)
than theoretical values (Saffman, 1965, 1968, MFC).

All three moment arms in (\ref{eqn:mmn_dfn}) are proportional to
$\dmt$, and so may be defined as $\aaa = \aaatld \dmt$, $\bbb =
\bbbtld \dmt$, and $\ccc = \ccctld \dmt$.  
The non-dimensional moment arms $\aaatld$, $\bbbtld$, and $\ccctld$
may be obtained for a specified packing geometry.  
Coleman (1967, MFC) showed that, for closely packed spheres of
identical size,
\begin{eqnarray}
\aaatld & = & \frac{1}{\sqrt{6}} \nonumber \\
\bbbtld & = & \frac{1}{4 \sqrt{3}} \nonumber \\
\ccctld & = & ???
\label{eqn:arm_ndm_dfn}
\end{eqnarray}

Empirical constraints on $\frcprmdrg$, $\frcprmlft$, and $\frcprmrsh$ 
were inferred by \cite{IvW82} using data from the Mars Surface Wind
Tunnel, MARSWIT.
Wind tunnels allow direct measurement of $\tpt$, $\prs$, $\dmt$,
$\dnsprt$, $\wndspd(\hgt)$, and, most importantly,
$\wndspdthr(\hgt)$. 
From these, one uses theory to infer $\rghmmn$, $\dnsatm$,
$\vscknm$, $\wndfrc$, $\wndfrcthr$, and $\rynnbrfrcthr$.
$\fff(\rynnbrfrcthr)$.
\cite{IvW82} performed wind tunnel observations with MARSWIT over wide
ranges of $\dnsatm$, $\wndspd$, and $\dmt$.
The data were expressed in terms of $\AAA$, the non-dimensional  
threshold friction speed from (\ref{eqn:wnd_frc_thr_bgn_dfn}).

Substituting (\ref{eqn:mmn_dfn}), (\ref{eqn:mmn_ndm_dfn}) and
(\ref{eqn:arm_ndm_dfn}) into (\ref{eqn:mmn_eqm}) we obtain
\begin{eqnarray}
% GrI85 p. 72 (3.3)
\aaatld \dmt \times \frcprmdrg \dnsatm \wndfrc^{2} \dmt^{2} + 
\bbbtld \dmt \times \frcprmlft \dnsatm \wndfrc^{2} \dmt^{2} + 
\frcprmrsh \dnsatm \wndfrc^{2} \dmt^{3} & = & 
\bbbtld \dmt \times \frac{\mpi \dnsprt \grv \dmt^{3}}{6} + 
\ccctld \dmt \times \frcipc
\\
\dnsatm \wndfrc^{2} \dmt^{3} 
( \aaatld \frcprmdrg + \bbbtld \frcprmlft + \frcprmrsh ) & = & 
\dnsprt \grv \dmt
\left( \frac{\mpi \bbbtld \dmt^{3}}{6} + 
\frac{\ccctld \frcipc}{\dnsprt \grv}
\right)
\label{eqn:dtl_bln}
\end{eqnarray}
With the atmospheric approximation $\dnsprt \gg \dnsatm$, 
$\AAA^{2} \approx \wndfrcthr^{2} \dnsatm / ( \grv \dnsprt \dmt )$
(\ref{eqn:wnd_frc_thr_bgn_prx}).
Thus we manipulate (\ref{eqn:dtl_bln}) to solve for $\AAA^{2}$
\begin{eqnarray}
% GrI85 p. 72 (3.5)
\frac{\dnsatm \wndfrc^{2}}{\dnsprt \grv \dmt} & = &
\frac{1}{\dmt^{3}}
\left( \frac{\mpi \bbbtld \dmt^{3}}{6} + 
\frac{\ccctld \frcipc}{\dnsprt \grv}
\right) 
( \aaatld \frcprmdrg + \bbbtld \frcprmlft + \frcprmrsh )^{-1} 
\nonumber \\
\AAA^{2} & = & \frac{\mpi \bbbtld}{6}
\left( 1 + \frac{6 \ccctld \frcipc}{\mpi \bbbtld \dnsprt \grv \dmt^{3}}
\right) 
( \aaatld \frcprmdrg + \bbbtld \frcprmlft + \frcprmrsh )^{-1}
\label{eqn:AAA_dfn_2}
\end{eqnarray}

Substituting (\ref{eqn:arm_ndm_dfn}) into (\ref{eqn:AAA_dfn_2}) we
obtain 
\begin{eqnarray}
\label{eqn:AAA_dfn_3}
\AAA^{2} & = & \frac{\mpi}{24 \sqrt{3}}
\left( \frac{\frcprmdrg}{\sqrt{6}} + \frac{\frcprmlft}{4 \sqrt{3}}
+ \frcprmrsh \right)^{-1} 
\left( 1 + \frac{6 \ccctld \frcipc}{\mpi \bbbtld \dnsprt \grv \dmt^{3}}
\right) 
\end{eqnarray}
The final term term on the RHS represents the effects of cohesion.
The $\dmt^{-3}$ dependence of the cohesion term suggest that $\AAA^{2}$  
may approach an asymptotic value for large $\dmt$ as long as
$\frcprmlft$ is not strongly dependent on $\dmt$.

\cite{IvW82} defined the \trmdfn{cohesionless threshold coefficient}
$\AAA_{1}$ by setting the cohesion force $\frcipc = 0$ in
(\ref{eqn:AAA_dfn_3}) 
\begin{eqnarray}
\AAA_{1}^{2} & = & \frac{\mpi}{24}
\left( \frac{\frcprmdrg}{\sqrt{2}} + \frac{\frcprmlft}{4} + \sqrt{3}
\frcprmrsh \right)^{-1} 
\label{eqn:AAA_1_dfn}
\end{eqnarray}
Note that $\AAA_{1}$ does not depend on $\dmt$, and thus, in principle, 
may be inferred from measurements of $\AAA^{2}$ when the influence of 
cohesion on (\ref{eqn:AAA_dfn_3}) is known.  

Greeley et al. (1980, MFC) and \cite{IvW82} used experiments with
MARSWIT to infer $\frcprmdrg$, $\frcprmlft$, and $\frcprmrsh$ 
(\ref{eqn:mmn_ndm_dfn}). 
They removed the effects of cohesion from the data and performed a
least squares fit to find
\begin{eqnarray}
% GrI85 p. 74 (3.7)
\label{eqn:frc_cff_obs}
\frcprmdrg + \sqrt{6} \frcprmrsh & = & 4.65 \nonumber \\
\frcprmlft & = & 32.8 \rynnbrfrcthr
\end{eqnarray}
for $0.03 < \rynnbrfrcthr < 0.3$.
Note that the lift parameter $\frcprmlft$ was found to explicitly
depend on $\rynnbrfrcthr$. 
This dependence on $\rynnbrfrcthr$ suggests that the form of 
$\frcprmlft$ chosen to non-dimensionalize $\frclft$ in
(\ref{eqn:mmn_ndm_dfn}) could be improved.

If the measured values in (\ref{eqn:frc_cff_obs}) are used in 
(\ref{eqn:AAA_1_dfn}), we find 
\begin{equation}
\AAA_{1}^{2} = 0.43 + 1.07 \rynnbrfrcthr
\label{eqn:AAA_1_obs}
\end{equation}
Equation (\ref{eqn:AAA_1_obs}) predicts $\AAA_{1} > 0.66$.
As previously mentioned, many observations suggest $\AAA \approx 0.2$ 
for large particles \cite[e.g.,][p. 88]{Bag41} where cohesive force
are presumably small.   
Thus the assumptions leading to (\ref{eqn:AAA_1_obs}) have rendered
the theory a qualitative, rather than an exact, description of
saltation initiation. 

In order to reconcile theory with observations, Iversen, Greeley, and 
colleagues set forth to isolate each functional dependence in
(\ref{eqn:AAA_dfn_3}) which could be independently measured.
They assumed $\AAA$ could be expressed as the product of three factors:
the cohesionless threshold $\AAA_{1}$ (\ref{eqn:AAA_1_dfn}), 
a function $\fff(\rynnbrfrcthr)$ (\ref{eqn:ryn_nbr_frc_thr_dfn}), 
and a function $\ggg(\dmt)$ which accounts for all interparticle
cohesive forces (Iversen et al., 1976a,b, MFC) 
\begin{equation}
\AAA = \AAA_{1} \fff(\rynnbrfrcthr) \ggg(\dmt) 
\label{eqn:AAA_dfn_4}
\end{equation}
The absence of interparticle cohesion corresponds to $\ggg = 1$.

Based on the theoretical influence of $\frcipc$ on $\AAA$
(\ref{eqn:AAA_dfn_3}), $\ggg$ is assumed to have a square-root
relationship with $\AAA$ and to be a function of particle size to an
unknown power $\nnn$   
\begin{equation}
\ggg(\dmt) = \left( 1 + \frac{\fctipc}{\dnsprt \grv \dmt^{\nnn}} \right)^{1/2}
\label{eqn:ggg_dfn}
\end{equation}
where $\nnn$ and $\fctipc$ are to be determined empirically.
Clearly $\ggg(\dmt)$ is a generalized version of the final term in 
(\ref{eqn:AAA_dfn_3}):
the factor $6 \ccctld \frcipc / (\mpi \bbbtld)$ is combined into
$\fctipc$, and $\nnn$ allows cohesion to depend on a non-integer power
of $\dmt$.
The latter assumption is intuitively appealing because cohesive forces
depend on surface properties (e.g., van~der~Waal's forces,
capillarity) as well as volume properties (e.g., electrostatic
charge). 

In order to determine $\fctipc$ and $\nnn$, \cite{IvW82} grouped
together many observations of $\AAA^{2}(\dmt)$ for a fixed value of
$\rynnbrfrcthr$.  
Each such curve showed $\AAA$ was strongly dependent on particle size
for $\dmt < 80$\,\um, but confirmed that $\AAA$ approaches an
asymptotic value of about $0.02$ for $\dmt > 150$\,\um.
\cite{IvW82} found that these data were best fit by 
$\fctipc = 6 \times 10^{-7}$\,kg\,m$^{1/2}$\,sec$^{-2}$ and $\nnn = 2.5$
\begin{equation}
\ggg(\dmt) = \left( 1 + \frac{6 \times 10^{-7}}{\dnsprt \grv \dmt^{2.5}} \right)^{1/2}
\label{eqn:ggg_obs}
\end{equation}
where all quantities are expressed in MKS units.
Taken together with (\ref{eqn:AAA_dfn_3}), the observed of best fit
value $\nnn = 2.5$ implies the $\frcipc \propto \sqrt{\dmt}$.
The quality of the fit in (\ref{eqn:ggg_obs}) adds confidence to the
form of parameterization chosen for $\AAA$ (\ref{eqn:AAA_dfn_4}).

Once $\ggg(\dmt)$ was known from (\ref{eqn:ggg_obs}), 
\cite{IvW82} were able to infer the product $\AAA_{1}
\fff(\rynnbrfrcthr)$ from $\AAA$ (\ref{eqn:AAA_dfn_4}).
The data were found to obey different functional relations depending
on the value of $\rynnbrfrcthr$,
\begin{eqnarray}
\AAA_{1} \fff(\BBB) & = & \left\{
\begin{array}{l@{\quad:\quad}r}
0.1291 ( -1 + 1.928 \BBB^{0.0922} )^{-1/2} & 0.03 \le \BBB \le 10 \\
0.120 ( 1 - 0.0858 \me^{ -0.0617 ( \BBB - 10 ) } ) & \BBB > 10
\end{array} \right.
\label{eqn:A1B_dfn}
\end{eqnarray}
where, following many of the source references, we have used $\BBB$
instead of $\rynnbrfrcthr$. 
Note that $\AAA$ approaches a limiting value near 0.120 for for $\BBB
> 10$.

Using (\ref{eqn:AAA_dfn_4}), (\ref{eqn:ggg_obs}), and
(\ref{eqn:A1B_dfn}) to compute $\wndfrcthr(\dmt,\BBB)$
(\ref{eqn:wnd_frc_thr_bgn_prx}),
\begin{eqnarray}
\wndfrcthr & = & \left\{
\begin{array}{l@{\quad:\quad}r}
\left( 1 + \frac{6 \times 10^{-7}}{\dnsprt \grv \dmt^{2.5} } \right)^{1/2}
0.1291 ( -1 + 1.928 \BBB^{0.0922} )^{-1/2} 
\sqrt{ \frac{\dnsprt \grv \dmt}{\dnsatm }}
& 0.03 \le \BBB \le 10 \\
\left( 1 + \frac{6 \times 10^{-7}}{\dnsprt \grv \dmt^{2.5} } \right)^{1/2}
0.120 ( 1 - 0.0858 \me^{ -0.0617 ( \BBB - 10 ) } ) 
\sqrt{ \frac{\dnsprt \grv \dmt}{\dnsatm }}
& \BBB > 10
\end{array} \right.
\label{eqn:wnd_frc_thr_obs}
\end{eqnarray}
Since $\rynnbrfrcthr$ is defined in terms of $\wndfrcthr$
(\ref{eqn:ryn_nbr_frc_thr_dfn}), (\ref{eqn:wnd_frc_thr_obs}) is an
implicit definition of $\wndfrcthr$ which must be solved numerically.
Note that the semi-empirical $\wndfrcthr$ defined by
(\ref{eqn:wnd_frc_thr_obs}) is a fluid threshold friction velocity, as
opposed to an impact threshold friction velocity
(cf. \S\ref{sxn:hrz}). 
A computationally efficient form of (\ref{eqn:wnd_frc_thr_obs}) is
given in (\ref{eqn:wnd_frc_thr_obs_2}).

Rather than solving (\ref{eqn:wnd_frc_thr_obs}) iteratively, 
\cite{MaB95} parameterized $\rynnbrfrcthr(\dmt)$ using $\dnsatm$ and
$\vscknm$ typical of dust source regions 
\begin{eqnarray}
% MaB95 p. 16417 (5)
\rynnbrfrcthr(\dmt) & = & 0.38 + 1331 (100\dmt)^{1.56}
\label{eqn:ryn_nbr_frc_thr_prx_MaB95}
\end{eqnarray}
The RHS of (\ref{eqn:ryn_nbr_frc_thr_prx_MaB95}) is usually known.
Thus $\rynnbrfrcthr(\dmt)$ may be evaluated without iteration, a
considerable advantage.   
However, we are unable to reproduce the accuracy of % fxm: Why? 
(\ref{eqn:ryn_nbr_frc_thr_prx_MaB95}) demonstrated in their Figure~1. 

The numerical solution of (\ref{eqn:wnd_frc_thr_obs}) shows that
$\wndfrcthrdmt$ has a fairly shallow minima at $\wndfrcthrmin$ which  
defines the optimal particle diameter for saltation, $\dmtopt$.
For a steady friction velocity $\wndfrc > \wndfrcthrmin$ over dry,
bare ground, we expect saltation to initiate with particles of size
$\dmtopt$.
% IvW82 p. 117 Fgr. 8, Pye87 p. 31, MBA97 p. 4388, SRL96 (2)
For typical arid regions of Earth, $\dmtopt \sim 75$\,\um\
\cite[][]{IvW82,Pye87,SRL96,MBA97}.
The minima is not symmetric in $\dmt$ or $\wndfrc$, however, due to
the rapid increase of cohesive forces with decreasing size.
Broadly speaking, particles in the range $40 < \dmt < 200$\,\um\ are
susceptible to saltation.

Predictions of (\ref{eqn:wnd_frc_thr_obs}) agree remarkably well with
wind tunnel observations. 
\cite{IvW82} tested (\ref{eqn:wnd_frc_thr_obs}) over a wide range of
$\rynnbrfrcthr$ for particles as small as $\dmt = 12$\,\um. 
Agreement was within 5\% for particles larger than $\dmt = 40$\,\um.
Uncertainties in both the model and the measurements become
significant for $\dmt < 40$\,\um, which is outside the saltation
range. 
Thus, these uncertainties need not be worrisome.
As previously mentioned (\ref{eqn:wnd_frc_thr_obs}) agrees with 
Bagnold's formulation for particles larger than $200$\,\um.

For particles smaller than $\dmtopt$, $\wndfrcthr$ increases very
quickly. 
In fact, for particles $\dmt < $\,\um, (\ref{eqn:wnd_frc_thr_obs})
predicts $\wndfrcthr > 1$\,\mxs, i.e., the threshold speed exceeds
values plausible for terrestrial conditions.

\cite{ShL00} derive an alternative theory for $\wndfrcthr$ that fits
the wind tunnel data of \cite{IvW82} but which also has the virtue of 
resulting in simpler expressions than (\ref{eqn:wnd_frc_thr_obs}).
% fxm: Fill in discussion of ShL00 here

To summarize, Bagnold's formulation (\ref{eqn:wnd_frc_thr_bgn_prx})
predicts that $\wndfrcthr$ decreases with $\dmt$ so that very small
particles should be most efficiently mobilized. 
Noting, however, that all observations show that $\wndfrcthrdmt$
decreases with $\dmt$ until a critical particle size $\dmt =
\dmtopt$, Iversen and colleagues developed a semi-empirical
formulation for $\wndfrcthr$ (\ref{eqn:wnd_frc_thr_obs}) which
accounts for lift and cohesive forces. 
Their formulation agrees well with wind tunnel measurements for $12 <
\dmt < 1000$\,\um. 

\section{Horizontal Dust Flux}\label{sxn:hrz}
Until now we have concentrated on the determination of the threshold
friction velocity $\wndfrcthr$ required to initiate motion of
particles of diameter $\dmt$ initially at rest. 
A useful theory of dust mobilization also requires specification of
the horizontal mass flux of all particle sizes and at all heights,
since this quantity can be measured in a wind tunnel.
\begin{equation}
\qqq(\hgt) = \int_{\dmt = 0}^{\dmt = \infty} 
\dst(\dmt,\hgt) \mss(\dmt) \vlc(\dmt,\hgt) \,\dfr\dmt
\label{eqn:flx_hrz_dfn_4}
\end{equation}
where $\dst$ is the number distribution of particles, $\mss$ their
mass, and $\vlc$ their horizontal velocity.
The units of $\qqq$ are \kgxmSs.
In addition to $\wndfrcthrdmt$, 
a useful theory of dust mobilization
requires specification of the vertically integrated horizontal mass
flux due to saltation 
\begin{equation}
\flxmsshrz = \int_{\hgt = 0}^{\hgt = \hgtslt} \int_{\dmt = 0}^{\dmt =
\infty} \dst(\dmt,\hgt) \mss(\dmt) \vlc(\dmt,\hgt) \,\dfr\dmt \,\dfr\hgt 
\label{eqn:flx_hrz_dfn}
\end{equation}
where $\hgtslt$ is the height of the saltation layer, 

The units of $\flxmsshrz$ are \kgxms\ rather than \kgxmSs\ since a
vertical integration of the streamwise mass flux has been performed.
The integration over $\dmt$ ensures $\flxmsshrz$ includes all sizes of
saltating particles, while the integration over $\hgt$ ensures
$\flxmsshrz$ includes particles (including suspended dust) at all levels.  
Thus $\flxmsshrz$ is the mass crossing orthoganally through an
infinitely tall column per unit width of the column.
$\flxmsshrz$ is often called the \trmdfn{streamwise mass flux} since
it measures the movement of crustal material in the direction of the
prevailing winds. 

The total streamwise movement of surface material $\flxmsshrz$ is the
result of three intertwined processes, \trmdfn{saltation},
\trmdfn{suspension}, and \trmdfn{surface creep}. 
Saltation includes the movement of all airborne particles which are
too large to become suspended.
Suspension, on the other hand, includes the movement of all particles 
The prevailing aerodynamic conditions play a role in determining what
particles are susceptible to long term suspension and transport,
so there is no single size below which particles are always suspended
and above which never suspended.
However, as a rule of thumb we call particles larger than 60\,\um\ in
diameter sand, other particles are collectively referred to as
\trmidx{dust}.  

Surface creep accounts for the movement of particles which are pushed
along the surface.
The creep may be caused by the wind directly or by nudges from
saltator impacts.
The important distinction is that saltating particles remove momentum
from the near surface wind which decreases the friction velocity felt
by the surface. 
Creeping particles, on the other hand, do not directly remove any
momentum from the near surface wind, and thus do not lessen the 
surface friction velocity.

With the above discussion in mind, we may also write $\flxmsshrz$ 
(\ref{eqn:flx_hrz_dfn}) as the sum of the individual streamwise fluxes
due to saltation $\flxmsshrzslt$, surface creep $\flxmsshrzcrp$, and
suspended dust $\flxmsshrzdst$ thusly 
\begin{eqnarray}
% Bag41 p. 65 (5)
\flxmsshrz & = & \flxmsshrzslt + \flxmsshrzcrp + \flxmsshrzdst
\label{eqn:flx_slt_crp}
\end{eqnarray}
According to \cite{Bag41}, 
$\flxmsshrzcrp \sim \frac{1}{4} \flxmsshrzslt$ in typical conditions.
Observations by \cite{SRF93} indicate that 
$\flxmsshrzdst \lesssim \flxmsshrzslt$.
As described in \S\ref{sxn:hrz_SRF93},
$\flxmsshrzdst/\flxmsshrzslt$ is called the 
\trmdfn{bombardment efficiency}.

A primary goal of mineral dust studies is the prediction of the
lifecycle of dust particles which begins with $\flxmsshrzdst$.
As shown above, wind tunnel studies \cite[e.g.,][]{IvW82} show that
the threshold velocity $\wndfrcthr$ (\ref{eqn:wnd_frc_thr_obs})
increases rapidly (due to cohesive forces) for particles smaller than
about 60\,\um, and exceeds terrestrial conditions for $\dmt < 10$\,\um.
Thus most long-lived dust particles are thought to be initially lofted
by impacts from more massive saltators rather than directly lofted by
the wind. 
Hence particles in suspension are said to be \trmdfn{secondary
particles}.  
For this reason $\flxmsshrzdst$ depends intimately on $\flxmsshrzslt$.  
Thus we concentrate on $\flxmsshrzslt$ before turning our attention to
the link between $\flxmsshrzslt$ and $\flxmsshrzdst$ in \S\ref{sxn:vrt}.

Many expressions have been proposed which express $\flxmsshrz$ in terms
of $\wndfrc$ and $\wndfrcthr$.
\cite{GrI85} summarize these expressions in their Table~3.5.
These expressions are based on theories or observations of particle
saltation.
The theories must predict the concentration and motion of particles
set into motion by $\wndfrc$, as well as particles released by the
impact of saltating particles.
Complicating these theories is the interaction between particles of
different sizes.

Theoretical and empirical evidence strongly suggests the horizontal
flux of saltating particles $\flxmsshrz$ varies with the cube of the
wind friction velocity during saltation $\wndfrcslt$
\cite{Owe64,SRF93} 
\begin{eqnarray}
\flxmsshrz & = & \left\{
\begin{array}{r@{\quad:\quad}l}
0 & \wndfrcslt < \wndfrcsltthr \\
\cst \wndfrcslt^{3} \left( 1 - \frac{\wndfrcsltthr^{2}}{\wndfrcslt^{2}} \right) & \wndfrcslt > \wndfrcsltthr
\end{array} \right.
\label{eqn:flx_hrz_dfn_2}
\end{eqnarray}
For now $\cst$ is a dimensional constant which is a function of
aerodynamic, surface, and soil properties to be defined below. 
Note that (\ref{eqn:flx_hrz_dfn_2}) is equivalent to (41) of
\cite{Owe64}. 

Since the wind speed is related to the wind friction speed by 
$\wndspd = \xchcffmmn^{1/2} \wndfrc$, we may rewrite
(\ref{eqn:flx_hrz_dfn_2}) as 
\begin{eqnarray}
\flxmsshrz & = & \left\{
\begin{array}{r@{\quad:\quad}l}
0 & \wndspdslt < \wndspdsltthr \\
\cst \wndspdslt^{3} \left( 1 - \frac{\wndspdsltthr^{2}}{\wndspdslt^{2}} \right) & \wndspdslt > \wndspdsltthr
\end{array} \right.
\label{eqn:flx_hrz_dfn_3}
\end{eqnarray}

\cite{LeR91} used a portable wind erosion tunnel to measure 
$\wndfrcthr(\dmt)$ and total streamwise mass flux $\flxmsshrz$ in field
conditions in Australia. 
The observed $\flxmsshrz$ was well described by (\ref{eqn:flx_hrz_dfn}).
They also predicted $\wndfrcthr$ with four competing methods,
including Equation~(\ref{eqn:wnd_frc_thr_obs}).
Unfortunately, no method, including (\ref{eqn:wnd_frc_thr_obs}),
adequately predicted the observed $\wndfrcthr$.

We now show the development of many theories for~$\flxmsshrz$,
including those of Bagnold, Kawamura, Owen, and the Australian school.
Unfortunately, authors in this field have often inadvertently
incorporated typographical errors into their papers (and models)
\cite[][]{NaS97}.
\cite{Baa05} covers this subject with more up-to-date examples, 
and intercompares some of these theories.
Papers with known typographical errors include 
Blumberg (1993, equation~3.2), 
Lancaster (1995, equation~2.12), 
\cite{Whi79} (equation~22),  
Greeley et~al. (1996, table~2, equation~4), 
Pye and Tsoar (1990, equation~4.48),
\cite{PaI02} (equation~19), and 
\cite{ZBN03} (equation~10).
Equation~27 in the influential paper \cite{MaB95} (and subsequently in
\cite{LuV02} (equation~6), \cite{THK02b} (equation~3) and \cite{ShL11}
(equation~11)), is correct even though it may appear at first glance
to be algebraically distinct.  
Researchers are urged to verify the correctness of their saltation
formulations by comparison to the original formulations.

\subsection[Original Theory]{Original Theory}\label{sxn:hrz_Bag41}
\cite{Bag41} developed the original theory relating $\flxmsshrz$ to
wind speed based on the energetics of an idealized, steady state,
linear, saltation zone.
A few qualitative observations motivate the strategy of Bagnold's
theory of saltation. 
The first, alluded to in \S\ref{sxn:thr_Bag41}, is that once
saltation has initiated by winds exceeding the fluid threshold
($\wndfrc > \wndfrcthr$), the downstream surface wind speed drops
because some fraction of the drag is now exerted on suspended
particles rather than directly on the surface. 
Downstream saltation continues as long as the surface wind speed
exceeds the impact threshold, i.e.,  $\wndfrc > \wndfrcthrimp$.

Once saltation initiates, moreover, the ambient wind speed $\wndspd$ 
may reach any strength but the surface wind speed is relatively constant.
In other words the velocity gradient near the surface can grow without 
bound, but the surface wind speed at about 3\,\mm\ height is insensitive
to this gradient.
These qualitative observations strongly suggest that drag in excess of
that necessary to maintain saltation is dissipated in the atmosphere
by the saltating grains.
Note that this observation is the basis of the second hypothesis of
\cite{Owe64} (cf. \S\ref{sxn:hrz_Owe64}).  

These observations underpin Bagnold's simple energetic explanation of
the cubic dependence of $\flxmsshrz$ on $\wndfrc$.
Consider the following idealized scenario:
A bed of particles of identical size $\dmt$ and mass $\mssslt$ saltates
in a steady state wind with friction velocity $\wndfrc$. 
The wind (rather than impacts by other particles) is responsible for
lifting each particle into the atmosphere with an initial horizontal
velocity $\uuusrt$. 
The wind transports each particle a mean horizontal distance $\lngscl$  
before the particle fall to the surface.
During this transport, the wind accelerates the particle to a final
horizontal velocity $\uuuend$ which is completely dissipated in the
impact.  

We now impose energy conservation constraints on this system.
The initial and final momenta of the particle are 
$\mssslt \uuusrt$ and $\mssslt \uuuend$, respectively.
The difference between these momenta, $\mssslt (\uuuend - \uuusrt)$, is
extracted from the atmosphere over the distance $\lngscl$.
Therefore the rate of loss of atmospheric momentum per unit area due
to a total horizontal mass flux of saltating particles
$\flxmsshrzslt$ must be 
\begin{equation}
\frac{\dfr\mmn}{\dfr\tm} = \frac{\flxmsshrzslt (\uuuend - \uuusrt)}{\lngscl }
\label{eqn:mmn_roc}
\end{equation}
Newton's second law (\ref{eqn:2nd_law_1}) tells us that the rate of
change of momentum is equivalent to the applied force.
In this case, the applied force is the surface wind stress due to
particle the drag of the particles on the wind $\wndstrslt$. 
\begin{eqnarray}
\frac{\flxmsshrzslt (\uuuend - \uuusrt)}{\lngscl } & = & \wndstrslt
\label{eqn:wnd_str_slt_dfn}
\end{eqnarray}
We note that $\wndstrslt < \wndstr$ since some of the total wind
stress $\wndstr$ (\ref{eqn:wnd_str_dfn}) goes directly into the
surface rather than into increasing the momentum of airborne
particles. 
Subsequent theories of $\flxmsshrzslt$, presented below, explicitly account 
for the distinction between $\wndstr$ and $\wndstrslt$.
Inserting (\ref{eqn:wnd_str_dfn_2}) into (\ref{eqn:wnd_str_slt_dfn}) 
\begin{eqnarray}
\frac{\flxmsshrzslt (\uuuend - \uuusrt)}{\lngscl } & = & \dnsatm \wndfrc^{2}
\label{eqn:mmn_wnd_frc}
\end{eqnarray}
To progress further, \cite{Bag41} made use of two qualitative
observations. 
First, the initial trajectory of a particle uplifted by wind is nearly
vertical, but the trajectory at impact in nearly horizontal which
implies $\uuuend \gg \uuusrt$ so that   
\begin{eqnarray}
\frac{\flxmsshrzslt \uuuend}{\lngscl } & = & \dnsatm \wndfrc^{2}
\label{eqn:mmn_wnd_frc_2}
\end{eqnarray}
Second, Bagnold observed that $\uuuend / \lngscl \approx \grv /
\wwwsrt$, where $\wwwsrt$ is the mean initial vertical velocity of a
saltating particle.
This observation is consistent with the approximation that sand
particles undergo ballistic trajectories. 
The atmospheric residence time $\Delta \tm$ of a sand particle ejected
into the atmosphere at speed $\wwwsrt$ is 
$\Delta \tm = 2 \wwwsrt / \grv$.
If the mean horizontal velocity of the particle is $\uuubar$ then
the total streamwise distance traversed is 
\begin{eqnarray}
\lngscl & = & \frac{2 \uuubar \wwwsrt}{\grv }
\label{eqn:lng_scl_dfn}
\end{eqnarray}
Inserting (\ref{eqn:lng_scl_dfn}) in (\ref{eqn:mmn_wnd_frc_2}) and
assuming $\uuubar \sim \uuuend$ leads to 
\begin{eqnarray}
% Bag41 p. 65 (5)
\frac{\flxmsshrzslt \grv}{\wwwsrt } & = & \dnsatm \wndfrc^{2} \nonumber \\
\flxmsshrzslt & = & \frac{\dnsatm \wwwsrt \wndfrc^{2}}{\grv }
\label{eqn:flx_slt_dfn}
\end{eqnarray}
where we have dropped the factor of~2 in (\ref{eqn:lng_scl_dfn}) for
consistency with Bagnold's original formulation.

To proceed further than (\ref{eqn:flx_slt_dfn}), is necessary to
discover or formulate a relationship between $\wwwsrt$ and $\wndfrc$.
Bagnold argued that, on average, $\wwwsrt = \cst_{1} \wndfrc$
where $\cst_{1}$ is called the \trmdfn{impact coefficient}.
In a perfectly elastic reflection, or ricochet, $\cst_{1} = 1$.
An elastic collision may also eject multiple saltators as products, in
which case $\cst_{1} < 1$ for each product. 
However, it is quite possible to have $\cst_{1} > 1$ when large
particles eject smaller particles. 
Bagnold reasoned that particles are ejected with a velocity
proportional to the incident velocity of the impacting particle which,
he argued, ought to be $\wndfrc$ on average.  
This reasoning is somewhat difficult to defend, but the assumption
turns out to be correct.
A better justification for this assumption is that fxm.
This leads to our first derivation of the well known phenomena that
the horizontal mass flux due to saltation is proportional to the cube
of the friction velocity
\begin{eqnarray}
% Bag41 p. 65 (7)
\flxmsshrzslt & = & \frac{\cst_{1} \dnsatm \wndfrc^{3}}{\grv }
\label{eqn:flx_slt_dfn_2}
\end{eqnarray}
Bagnold's observations fit (\ref{eqn:flx_slt_dfn_2}) best with $\cst_{1}
\sim 0.8$.

Based on careful analysis, Bagnold empirically modified
(\ref{eqn:flx_slt_dfn_2}) to include a factor of $(\dmt)^{1/2}$.
With this correction, and subsuming $\cst_{1}$ into a new empirical
constant $\cstslt$ 
\begin{eqnarray}
% Bag41 p. 67 (9)
\flxmsshrz & = & \cstslt \left( \frac{\dnsatm}{\grv } \right) 
\left( \frac{\dmt}{250 \times 10^{-6} } \right)^{1/2} \wndfrcslt^{3} 
\label{eqn:flx_hrz_bgn}
\end{eqnarray}
where all parameters are specified in MKS.

Expressing (\ref{eqn:flx_hrz_bgn}) in terms of $\wndspd$ rather than
$\wndfrc$, Bagnold found
\begin{eqnarray}
% Bag41 p. 69 (10)
\label{eqn:flx_hrz_bgn_2}
\flxmsshrz & = & \cstslt \left( \frac{\dnsatm}{\grv } \right) 
\left( \frac{\dmt}{250 \times 10^{-6} } \right)^{1/2} \wndfrcslt^{3} \\
\label{eqn:flx_hrz_bgn_3}
& \sim & 1.5 \times 10^{-9} ( \wndspd - \wndspdthr )^{3}
\end{eqnarray}
where (\ref{eqn:flx_hrz_bgn_2}) yields (\ref{eqn:flx_hrz_bgn_3}) for
typical conditions in natural dunes.

\subsection[Owen's Theory]{Owen's Theory}\label{sxn:hrz_Owe64}
\cite{Owe64} developed a physical theory describing the saltation of
uniform particles in air.
His theory continues to serve as the best mathematical definition and
description of saltation, as well as being a valuable exposition of
mathematical physics in its own right.

Owen's two hypotheses are:
\begin{quotation}
I.~The saltation layer behaves, so far as the flow outside it is
concerned, as an aerodynamic roughness whose height is proportional to
the thickness of the layer.

II.~The concentration of particles within the saltation layer is
governed by the condition that the shearing stress borne by the fluid
falls, as the surface is approached, to a value just sufficient to
ensure that the surface grains are in a mobile state.
\end{quotation}

\subsection[Kawamura/White Formulation]{Kawamura/White Formulation}\label{sxn:hrz_Whi79}
Kawamura's work on sand transport in the late 1940s and early 1950s
is not widely known.
During this time Kawamura developed a rather complete theory for the
streamwise saltation flux (1951) which appeared in translated form
in 1964.
White based his formulation for the streamwise saltation flux
\cite[]{WGI76,Whi79} squarely on Kawamura's work.  
Thus, White appears to be the first to recognize the efficacy of
Kawamura's work for modern dust models. 
Unfortunately (but understandably), many references inadvertently
attribute Kawamura's formulation of streamwise mass transport to
\cite{Whi79}.
In recognition of White's association, we sometimes refer to this
as the Kawamura/White formulation. 

Rather uniquely, White and colleagues performed numerical simulations
of the saltation jumps of individual particles based on first
principles, i.e., the equations of motion.   
These simulations, in aggregate, later serve to validate simplifying
assumptions he makes in his bulk theory.
The equations of motion for a particle saltation take the form
\begin{eqnarray}
% Whi79 p. 4644 (5)
\mssslt \xxxddot = - \frcdrg \frac{\xxxdot - \uuu}{\vlcrlt } 
+ \frclft \frac{\yyydot}{\vlcrlt } \\
% Whi79 p. 4644 (6)
\mssslt \yyyddot = - \frclft \frac{\xxxdot - \uuu}{\vlcrlt }  
- \frcdrg \frac{\yyy}{\vlcrlt } - \mssslt \grv 
\end{eqnarray}
where $(\xxxdot,\yyydot)$ and $(\xxxddot,\yyyddot)$ are the $\xxx$ and
$\yyy$ components of the particle's velocity and acceleration,
respectively, $\frclft$ and $\frcdrg$ are the lift and drag forces on
the particle, $\uuu$ is the streamwise wind speed, and $\vlcrlt$ is the
relative speed of the particle to the wind.
By definition the mean vertical wind speed is zero in the saltation
layer so that 
\begin{eqnarray}
% Whi79 p. 4644 (7)
\vlcrlt & = & \sqrt{ ( \xxxdot - \uuu )^{2} + \yyydot^{2} }
\end{eqnarray}

The lift and drag forces on the particle are expressed in terms of the
lift and drag coefficients $\cfflft$ and $\cffdrg$ as
\begin{eqnarray}
% Whi79 p. 4644 (8), SeP97 p. 462 (8.31)
\frclft & = & \frac{1}{8} \mpi \dnsatm \cfflft \vlcrlt^{2} \dmt^{2} \\
\frcdrg & = & \frac{1}{8} \mpi \dnsatm \cffdrg \vlcrlt^{2} \dmt^{2}
\end{eqnarray}

The drag force on a particle is usually expressed in terms of the
density of the medium, the projected area of the particle, and the
square of the particle's velocity.
\begin{eqnarray}
% SeP97 p. 462 (8.31)
\frcdrg & = & \frac{1}{2} \dnsatm \cffdrg \xsa \vlcinf^{2}
\label{eqn:frc_drg_gnr_dfn}
\end{eqnarray}
where $\xsa$ is the cross-sectional area of the particle.
This form of relationship between $\cffdrg$ and $\frcdrg$ has been
chosen for several reasons. 
First, the solution to the problem of flow over a sphere tells that
$\frcdrg \propto \vlcinf$ when $\rynnbr \ll 1$.
However, we also know that the pressure, or force per unit area, 
should vary as $\dnsatm \vlcinf^{2}$ (from Bernoulli's theorem?)
as $\rynnbr \rightarrow 1$, $\frcdrg$.
For spherical particles, (\ref{eqn:frc_drg_gnr_dfn}) implies
\begin{eqnarray}
% SeP97 p. 462 (8.31)
\frcdrg & = & \frac{1}{8} \mpi \dnsatm \cffdrg \vlcinf^{2} \dmt^{2}
\label{eqn:frc_drg_sph_dfn}
\end{eqnarray}

As presented in \S\ref{sxn:wnd_str}, the total stress to the surface
$\wndstr$ is the sum of a particle drag and an aerodynamic drag.
The particle drag on the surface, $\wndstrslt$, is caused by the
horizontal deceleration of the impactors by the surface.
The aerodynamic drag on the surface, $\wndstratm$, is the drag
directly due to gas flow over the surface (\ref{eqn:wnd_str_prt}).  
We know $\wndstr$ (\ref{eqn:wnd_str_dfn_2}) and can obtain
$\wndstratm$ using Owen's second hypothesis (\S\ref{sxn:hrz_Owe64}) 
\newline\parbox{6in}{ % KoD95 p. 138
\begin{eqnarray*}
\wndstr & = & \dnsatm \wndfrc^{2} \\
\wndstratm & = & \dnsatm \wndfrcthr^{2}
\end{eqnarray*}
}\hfill % end parbox KoD95 p. 138
\parbox{1cm}{\begin{eqnarray}\label{eqn:wnd_str_Whi79}\end{eqnarray}}\newline
Together with (\ref{eqn:wnd_str_prt}), this implies
\begin{eqnarray}
% Whi79 p. 4648 (11)
\wndstrslt & = & \wndstr - \wndstratm \nonumber \\
% MaB9595 p. 16421 (24)
& = & \dnsatm ( \wndfrc^{2} - \wndfrcthr^{2} ) \nonumber \\
% Whi79 p. 4648 (12)
& = & \dnsatm ( \wndfrc + \wndfrcthr ) ( \wndfrc - \wndfrcthr )
\label{eqn:wnd_str_slt_Whi79_1}
\end{eqnarray}
Thus $\wndstrslt$ appears as the difference between two fluid
stresses, $\dnsatm \wndfrc^{2}$ and $\dnsatm \wndfrcthr^{2}$. 
The former is the total stress available to do turbulent work, 
and the latter is the threshold required to initiate saltation.
Note that these fluid stresses are dimensionally equivalent to
volumetric energy densities. 

Equation~(\ref{eqn:wnd_str_slt_Whi79_1}) defines $\wndstrslt$ as a
residual between the total surface stress and the aerodynamic stress. 
We may also define $\wndstrslt$ directly from kinematic considerations
as the rate of deposition of streamwise momentum to the surface by the
saltating particles. 
If $\flxmssvrtslt$ is the downward mass flux of saltating particles
per unit area per unit time, then the total streamwise momentum
deposited to the bed by the particles is 
\begin{eqnarray}
% Whi79 p. 4648 (13)
\wndstrslt & = & \flxmssvrtslt (\uuuend - \uuusrt)
\label{eqn:wnd_str_slt_Whi79_2}
\end{eqnarray}
where $\uuusrt$ and $\uuuend$ are the mean initial and final
streamwise velocities during a saltator jump.
Conceptually, we may view $\uuuend - \uuusrt$ as the velocity change
of the \textit{same} saltator as it repeatedly skips of the surface,
depositing some momentum with each impact.
Then the particulate mass flux $\flxmssvrtslt$ times the mean change
in particle velocity is the momentum flux.
Note that the initial saltator velocity $\uuusrt$ contributes to
particle momentum rather than surface stress, so it is subtracted from 
$\wndstrslt$.

Let us denote the mean initial vertical velocity of saltating
particles as $\wwwsrt$.
Then the mean initial vertical momentum of saltating particles is
$\wwwsrt \flxmssvrtslt$.
Clearly the initial vertical momentum of saltating particles varies
with the intensity of saltation.
One possibility is that $\wwwsrt \flxmssvrtslt$ obeys a functional
relationship with the surface saltation stress $\wndstrslt$.
For example, one might hypothesize that $\wndstrslt$ is converted into
vertical momentum with some non-unity efficiency.
More specifically, we shall assume that $\wwwsrt \flxmssvrtslt$ is
linearly proportional to $\wndstrslt$.
Then (\ref{eqn:wnd_str_slt_Whi79_1}) and
(\ref{eqn:wnd_str_slt_Whi79_2}) imply  
\begin{eqnarray}
% Whi79 p. 4648 (14)
\wwwsrt \flxmssvrtslt & \propto & \wndstrslt \nonumber \\
& \propto & \dnsatm ( \wndfrc^{2} - \wndfrcthr^{2} ) \nonumber \\
& \propto & \flxmssvrtslt (\uuuend - \uuusrt)
\label{eqn:www_srt_Whi79}
\end{eqnarray}
Note that (\ref{eqn:www_srt_Whi79}) assumes
If the initial vertical momentum of a saltating particle is
proportional to the kinetic energy released by surface bombardment 
then (or is this assuming the vertical momentum flux of ejected
particles must vary as the horizontal momentum flux deposited by
bombarding particles) 

The relationship in (\ref{eqn:www_srt_Whi79}) assumes that the kinetic 
energy of bombardment converts to vertical momentum of the product
saltators with an imperfect, but constant, efficiency. 
In other words the bombardment process is inelastic.
The disposition of the unaccounted-for energy is not specified.
The theory of \cite{SRF93}, presented below, extends this treatment of
energy conversion in developing a theory for the vertical flux of
small dust particles.

The crucial advance of Kawamura's theory is made possible by the
assumption of a relation between $\flxmssvrtslt$ and the friction
speeds. 
MARSWIT data and the saltation model of \cite{Whi79} (described above)
both support the following empirical relationship 
\begin{eqnarray}
% Whi79 p. 4648 (15)
\flxmssvrtslt & \propto & \dnsatm ( \wndfrc - \wndfrcthr )
\label{eqn:flx_mss_vrt_slt_Whi79}
\end{eqnarray}
Combining (\ref{eqn:flx_mss_vrt_slt_Whi79}) with
(\ref{eqn:www_srt_Whi79}) we obtain 
\begin{eqnarray}
% Whi79 p. 4648 (16)
\label{eqn:www_srt_Whi79_2}
\wwwsrt & \propto & \wndfrc + \wndfrcthr \\
% Whi79 p. 4648 (17)
\label{eqn:lng_scl_Whi79}
\lngscl & \propto & ( \wndfrc + \wndfrcthr )^{2} / \grv
\end{eqnarray}
\cite{Whi79} noted that the values of $\wwwsrt$ and $\lngscl$
predicted by (\ref{eqn:www_srt_Whi79_2})--(\ref{eqn:lng_scl_Whi79})
agreed with direct numerical integration of the equations of motions
for saltating particles. 

The total streamwise saltation flux is simply the product of
$\flxmssvrtslt$ and $\lngscl$.
Using (\ref{eqn:flx_mss_vrt_slt_Whi79}) and (\ref{eqn:lng_scl_Whi79}),  
\begin{eqnarray}
% fxm: Should use \wndfrcslt rather than \wndfrc in White's
% expressions to be consistent with rest of manuscript.
% \wndfrcslt is wind friction speed during saltation.
\flxmsshrzslt & = & \flxmssvrtslt \times \lngscl \nonumber \\
% Whi79 p. 4648 (18)
& = & \cstslt \times \dnsatm (\wndfrc - \wndfrcthr) \times 
(\wndfrc + \wndfrcthr)^{2} / \grv \nonumber \\
% Whi79 p. 4648 (19)
& = & \frac{\cstslt \dnsatm}{\grv} (\wndfrc - \wndfrcthr)(\wndfrc + \wndfrcthr)^{2} \nonumber \\
& = & \frac{\cstslt \dnsatm \wndfrc^{3}}{\grv} 
\left( 1 - \frac{\wndfrcthr}{\wndfrc} \right) 
\left( 1 + \frac{\wndfrcthr}{\wndfrc} \right)^{2}
\label{eqn:flx_mss_hrz_slt_Whi79}
\end{eqnarray}
where, as before, $\cstslt$ is the dimensionless constant of
proportionality between saltation mass flux and the factors
proportional to the cube of the friction speed. 
Factoring out the $\wndfrc^{3}$ factor in the final step brings the form
of (\ref{eqn:flx_mss_hrz_slt_Whi79}) into closer agreement with
Bagnold (\ref{eqn:flx_hrz_bgn}) and Owens. % fxm: Include Owens expression
Note that (\ref{eqn:flx_mss_hrz_slt_Whi79}) was derived and tested
for monodisperse soil distributions. 

\cite{NaS97} and \cite{Baa05} document that Equation~(19) in
\cite{Whi79}, which corresponds to our
Equation~(\ref{eqn:flx_mss_hrz_slt_Whi79}), has a typographical error
which propagated into some dust emission models apparently including 
\cite{PaI02}.
If adopted, this error would cause models to overpredict dust emissions.
In practice, however, the erroneous equation performs no worse than
other (correct) mobilization implementations. 
By coincidence, \cite{ZBN03} also contains a typo in its version of 
Equation~(\ref{eqn:flx_mss_hrz_slt_Whi79})\footnote{
This typo was not present in the submitted manuscript.
The error was introduced by the \textit{JGR} copy editor.
The mistake was never present in DEAD, and so it never affected any 
simulations.}.  

\cite{Whi79} used MARSWIT to replicate a variety of Earth and Martian 
saltation conditions in order to determine $\cstslt$.
With small glass beads ($\dmt = 0.208$\,\mm) as saltators, they found
$\cstslt = 2.61$ under a wide range of conditions.
This differed by only 6\% from Kawamura's original estimate of
$\cstslt = 2.78$ \cite[][]{NaS97,Baa05}.
From this we may infer that (\ref{eqn:flx_mss_hrz_slt_Whi79}) contains 
all the relevant physics of saltation for both the Earth and Mars
simulations. 
Of course, such complicating factors as moisture, heterogeneous soil
sizes, and vegetation were not considered in the tests.
Nevertheless, the consistency of the experiments of \cite{Whi79} with
(\ref{eqn:flx_mss_hrz_slt_Whi79}) are very encouraging as a point from
which to begin to include the effects of more complicating factors.

\subsection[Australian School]{Australian School}\label{sxn:hrz_SRF93}
Researchers in Australia published theories for saltation and
sandblasting beginning in the 1990s. 
This research involved many groups, though Michael Raupach and Yaping
Shao appear to be most consistently behind it.
I refer to their approach as the Australian School, since it is nicely
synthesized and originally evaluated with Australian models.
\cite{SRF93} present a theory for the streamwise saltation flux which 
differs slightly from the Kawamura/White formulation
(\S\ref{sxn:hrz_Whi79}). 
\cite{LuS99} and \cite{Sha01} summarize the full development of this
theory. 
\cite{RaL04} includes these theories in the context of more general
treatments of mobilization and dry deposition. 

The Australians' approach considers uniform particles of mass
$\mssslt$ in steady state saltation. 
In steady state, the mean rates of saltator bombardment and ejection 
per unit surface area are equal, and denoted by $\flxnbrvrtslt$ which has
units of \xmSs. 
With these assumptions the vertically integrated streamwise saltation
flux defined in (\ref{eqn:flx_hrz_dfn}) may be rewritten as  
\begin{eqnarray}
% SRF93 p. 12723 (8)
\flxmsshrzslt & = & \mssslt \flxnbrvrtslt \lngscl
\label{eqn:flx_slt_SRF93}
\end{eqnarray}
where, as before, $\lngscl$ is the downstream projection of the mean
particle jump. 

As in the Kawamura/White formulation (\S\ref{sxn:hrz_Whi79}), 
we assume $\wndstr$ and $\wndstratm$ are given by
(\ref{eqn:wnd_str_Whi79}).
Combining Bagnold's expression for $\wndstrslt$
(\ref{eqn:wnd_str_slt_dfn}) with (\ref{eqn:flx_slt_SRF93}) we obtain
\begin{eqnarray}
\wndstrslt & = & \frac{\flxmsshrzslt (\uuuend - \uuusrt)}{\lngscl }
\nonumber \\
& = & \frac{\mssslt \flxnbrvrtslt \lngscl (\uuuend - \uuusrt)}{\lngscl }
\nonumber \\
& = & \mssslt \flxnbrvrtslt (\uuuend - \uuusrt) 
\label{eqn:wnd_str_slt_SRF93}
\end{eqnarray}
Finally we obtain a relation for $\flxnbrvrtslt$ by substituting
values from (\ref{eqn:wnd_str_Whi79}) and
(\ref{eqn:wnd_str_slt_SRF93}) into  each term in
(\ref{eqn:wnd_str_prt}) 
\begin{eqnarray}
\dnsatm \wndfrc^{2} & = & \dnsatm \wndfrcthr^{2} + 
\mssslt \flxnbrvrtslt (\uuuend - \uuusrt) \nonumber \\
\dnsatm \wndfrc^{2} \left( 1 - \frac{\wndfrcthr^{2}}{\wndfrc^{2} } \right)
& = & 
\mssslt \flxnbrvrtslt (\uuuend - \uuusrt) \nonumber \\
\flxnbrvrtslt & = & 
\dnsatm \wndfrc^{2} \left( 1 - \frac{\wndfrcthr^{2}}{\wndfrc^{2} } \right)
[ \mssslt ( \uuuend - \uuusrt ) ]^{-1}
\label{eqn:flx_nbr_vrt_slt_SRF93}
\end{eqnarray}
Following Bagnold, \cite{SRF93} note that $\lngscl = \uuubar \Delta
\tm$ where $\uuubar$ is the mean streamwise velocity during the jump.
For ballistic trajectories, $\Delta \tm = 2 \wwwsrt / \grv$ so 
$\lngscl = 2 \wwwsrt \uuubar / \grv$.
Thus, using the ballistic assumption for $\lngscl$ and
(\ref{eqn:flx_nbr_vrt_slt_SRF93}) for $\flxnbrvrtslt$ in
(\ref{eqn:flx_slt_SRF93}) 
\begin{eqnarray}
\flxmsshrzslt & = & 
\mssslt \times
\dnsatm \wndfrc^{2} \left( 1 - \frac{\wndfrcthr^{2}}{\wndfrc^{2} } \right)
\frac{\uuubar}{( \uuuend - \uuusrt ) } \times
\frac{2 \wwwsrt}{\grv } \nonumber \\
\label{eqn:flx_slt_SRF93_2}
& = & \frac{\dnsatm \wndfrc^{3}}{\grv }
\left( 1 - \frac{\wndfrcthr^{2}}{\wndfrc^{2} } \right) \times
\frac{\uuubar}{( \uuuend - \uuusrt ) } \times
\frac{2 \wwwsrt}{\wndfrc } \\
\label{eqn:flx_slt_SRF93_3}
& = & 
\frac{\cstslt \dnsatm \wndfrc^{3}}{\grv }
\left( 1 - \frac{\wndfrcthr^{2}}{\wndfrc^{2} } \right)
\end{eqnarray}
where $\cstslt$ is a dimensionless coefficient which incorporates the
last two factors on the RHS of (\ref{eqn:flx_slt_SRF93_2}). 
Following reasoning similar to that presented in the development of
(\ref{eqn:lng_scl_dfn}) and (\ref{eqn:flx_slt_dfn_2}), \cite{SRF93}
assert $\uuubar \sim \uuuend$ and $\wwwsrt \sim \wndfrc$\footnote{ 
This is equivalent to assuming Bagnold's impact coefficient 
$\cst_{1} \sim 1$ in (\ref{eqn:flx_slt_dfn_2}).}.
If both factors comprising $\cstslt$ are of order unity, then $\cstslt$ is
$\mathcal{O}(1)$ as well.

The parameter $\cstslt$ (\ref{eqn:flx_slt_SRF93_3}) appears in most
theories of $\flxmsshrzslt$ \cite[e.g.,][p.~100]{GrI85}. 
As presented in \S\ref{sxn:hrz_Bag41}, \cite{Bag41} estimated
$\cstslt \sim 0.8$. 
Equation~(47) of \cite{Owe64} estimates $\cstslt$ empirically to be 
\begin{eqnarray}
% Owe64 p. 239 (47), SRL96 p. 315 (5)
\cstslt & = & 0.25 + \frac{\vlcgrv (\dmt)}{3 \wndfrc} 
\label{eqn:flx_slt_prm_Owe64}
\end{eqnarray}
where $\vlcgrv$ (\ref{eqn:vlc_grv_dfn}) is the terminal fall speed of
the particle.  
For $\dmt \sim 150$\,\um\ and $\wndfrc \sim 0.8$\,\mxs,
(\ref{eqn:flx_slt_prm_Owe64}) predicts $\cstslt \sim 0.8$, 
which agrees with Bagnold's estimate.

Note that $\flxnbrvrtslt$ (\ref{eqn:flx_nbr_vrt_slt_SRF93}) can be
expressed in terms of $\flxmsshrzslt$ using (\ref{eqn:flx_slt_SRF93}) and
the ballistic approximation for $\lngscl$   
\begin{eqnarray}
\flxnbrvrtslt & = & \frac{\flxmsshrzslt}{\lngscl \mssslt } \nonumber \\
& = & \frac{\grv \flxmsshrzslt}{2 \wwwsrt \uuubar \mssslt }
\label{eqn:flx_nbr_vrt_slt_SRF93_2}
\end{eqnarray}

\subsection[My Theory]{My Theory}\label{sxn:hrz_csz}
If a saltating particle undergoes constant streamwise acceleration
while airborne then its mean horizontal velocity is 
$\uuubar = (\uuusrt + \uuuend) / 2$.
Constant acceleration is a plausible assumption for some particles
(fxm: which???). 
The total streamwise distance traversed is then
\begin{eqnarray}
\lngscl & = & \frac{\uuusrt + \uuuend}{2} \times 
\frac{2 \wwwsrt}{\grv } \nonumber \\
& = & \frac{\wwwsrt ( \uuusrt + \uuuend )}{\grv }
\label{eqn:lng_scl_csz}
\end{eqnarray}
Employing (\ref{eqn:lng_scl_csz}) for $\lngscl$ and
(\ref{eqn:flx_nbr_vrt_slt_SRF93}) for $\flxnbrvrtslt$ in 
(\ref{eqn:flx_slt_SRF93}) we find  
\begin{eqnarray}
\flxmsshrzslt & = & 
\mssslt \times
\dnsatm \wndfrc^{2} \left( 1 - \frac{\wndfrcthr^{2}}{\wndfrc^{2} } \right)
[ \mssslt ( \uuuend - \uuusrt ) ]^{-1} \times
\frac{\wwwsrt ( \uuusrt + \uuuend )}{\grv } \nonumber \\
& = & 
\frac{\dnsatm \wndfrc^{2}}{\grv }
\left( 1 - \frac{\wndfrcthr^{2}}{\wndfrc^{2} } \right)
\frac{\wwwsrt ( \uuusrt + \uuuend )}{( \uuuend - \uuusrt ) }  \nonumber \\
& = & 
\frac{\dnsatm \wndfrc^{3}}{\grv }
\left( 1 - \frac{\wndfrcthr^{2}}{\wndfrc^{2} } \right)
\frac{\wwwsrt ( \uuusrt + \uuuend )}{\wndfrc ( \uuuend - \uuusrt ) } 
\label{eqn:flx_hrz_csz}
\end{eqnarray}
As far as I can tell, no one, not even \cite{Owe64}, has derived a
relationship between $\wwwsrt$ and $\wndfrc$ from first principles.   
All theories seem to base the assumption that 
$\wwwsrt \sim \wndfrc$ on dimensional analysis and observations.

\section{Vertical Dust Flux}\label{sxn:vrt}
% IvW82 LeR91
% GrI85 p.~81, Bag41 p.~84, SRL96 p.~313, Pye87 p.~31, MaB95 p.~16417, 
Until now we have concentrated on predicting the wind-initiated motion
of saltating particles but we have put aside consideration of the mass
flux of the smaller, suspended particles known as dust.
Bombardment by saltating particles, or \trmdfn{sandblasting}, is
thought to be the ultimate source of most fine dust emissions.
The theory of sandblasting has been extensively developed by Alfaro,
Gomes and coworkers \cite[]{GBC90,AlG95,AGG97,AGG98,AlG01}.
Applications of these theories in regional scale models is infrequent
\cite[][]{SRL96,ShL97,GZZ03}.
Saltation-sandblasting is beginning to appear in global scale models 
\cite[][]{GrZ04,GMZ05}.
The small size and long atmospheric residence time of dust particles
causes them to exert significant influences over climate.

Saltation is summarized by the streamwise saltation flux $\flxmsshrzslt$.
Likewise, the streamwise mass flux of dust is denoted $\flxmsshrzdst$.
The vertical mass flux of dust particles through a horizontal plane is 
\begin{equation}
\flxmssvrt(\hgt) = \int_{\dmt = 0}^{\dmt = \infty} 
\dst(\dmt,\hgt) \mss(\dmt) \www(\dmt,\hgt) \,\dfr\dmt
\label{eqn:flx_vrt_dfn}
\end{equation}
When $\hgt > \hgtslt$, then (\ref{eqn:flx_vrt_dfn}) includes only dust
mass so that $\flxmssvrt = \flxmssvrtdst$.
The units of $\flxmssvrt$ are \kgxmSs.
To distinguish dust particles from saltating particles, we shall use
$\mssdst$ and $\mssslt$, respectively.
Dust, by definition, is suspended in the atmosphere and does not
immediately settle back to the surface.
Thus $\flxmssvrt$ is somewhat insensitive to $\hgt$ once $\hgt >
\hgtslt$. 
Many large scale atmospheric models assume turbulence uniformly mixes 
all dust emissions into the lowest atmospheric layer, so that the net
dust source term is taken to be $\flxmssvrt(\hgt = \hgtslt)$.
Note that the mean streamwise distance from ejection to impact
$\lngscl$, which proved useful in defining $\flxmsshrzslt$
(\ref{eqn:flx_slt_SRF93}), does not appear in (\ref{eqn:flx_vrt_dfn}).   

Prediction of $\flxmssvrtdst$ is the crux of mineral dust aerosol models.
To date, all theories are based on establishing a relation between 
$\flxmsshrz$ (\ref{eqn:flx_hrz_dfn_2}) and $\flxmssvrtdst$
(\ref{eqn:flx_vrt_dfn}). 
There is good observational evidence to support this link
\cite[e.g.,]{SRF93,GFG97}, but the lack of theoretical support is
somewhat discomfiting.  

The most successful theories are based on the energetics of the
impact-ejection mechanism.
Each ballistic impact a saltating particle with the surface results in 
a transfer and conversion of momentum and kinetic energy from the
saltating particle to the surface.
We have seen that some horizontal momentum is transferred into
soil creep.
The vertical momentum may be reflected into the next bounce of the
saltator, or it may initiate the ejection of another saltator.
When the product of an impact is the ejection of dust, however, it is
likely that some energy has been used to break the cohesive bonds
binding the dust particle to the surface.
Thus some fraction of the energy from particle bombardment is
responsible for the injection of dust into the atmosphere.   
The theory of \cite{SRF93} explains many, but not all, of the observed
features of the $\flxmssvrtdst$-$\flxmsshrzslt$ relationship.

\subsection[Australian School]{Australian School}\label{sxn:vrt_SRF93}
\cite{SRF93} developed a theoretical framework for the rupture of
the interparticle bonds between dust particles and the surface.
They allow for a mean interparticle binding energy of $\nrgbnd$.
$\nrgbnd$ is the depth of the energy potential well which must be
surmounted in order to free the dust grain from the surface.
Thus $\nrgbnd$ accounts for the forces of cohesion $\frcipc$ and
gravity $\frcgrv$ discussed in \S\ref{sxn:thr_GrI85}.
$\nrgbnd$ is also related to the \trmdfn{modulus of surface rupture}
discussed by \cite{GiP88}.

They consider a scenario where a saltator impacts the surface and
ruptures the bonds of, on average, $\dstprdfsh$ dust grains which are
then ejected from the surface along with zero or more saltators.
We call this the \trmdfn{impact-rupture-ejection} scenario. 
Let the kinetic energy of the impacting saltator be $\nrgend$ and the 
total kinetic energy of the product saltators (which include the
original when reflection occurs) be $\nrgprd$.
Thus the energy available for freeing dust bonds is 
$\dltnrg = \nrgend - \nrgprd$.
Let us define $\fshfct$ as the mean fraction of $\dltnrg$ which is
channeled to rupturing dust bonds.
Energy conservation requires $\fshfct < 1$ since, in addition to other
sinks, some of $\nrgend$ is converted into the kinetic energy of the
ejected dust particles.   
Then the energy balance of each saltator impact is
\begin{eqnarray}
% SRF93 p. 12724 (12)
\dstprdfsh \nrgbnd & = & \fshfct \dltnrg
\label{eqn:dst_nbr_dfn}
\end{eqnarray}

By definition, dust is suspended once it is emitted, so there is no
downward flux of dust particles in the saltation layer.
Therefore the net vertical dust flux is simply the dust emission flux, 
which is the product of the areal rate of saltator impacts
$\flxnbrvrtslt$ and the dust production efficiency per impact
$\dstprdfsh$. 
With these assumptions the areal rate of emission of dust mass from
the surface defined in (\ref{eqn:flx_vrt_dfn}) may be rewritten as  
\begin{eqnarray}
\flxmssvrtdst & = & \mssdst \flxnbrvrtdst \\
& = & \mssdst \flxnbrvrtslt \dstprdfsh
\label{eqn:flx_vrt_SRF93_1}
\end{eqnarray}
Substituting $\dstprdfsh$ from (\ref{eqn:dst_nbr_dfn}) into
(\ref{eqn:flx_vrt_SRF93_1}) we obtain
\begin{eqnarray}
% SRF93 p. 12724 (12)
\flxmssvrtdst & = & \mssdst \flxnbrvrtslt \fshfct \dltnrg / \nrgbnd
\label{eqn:flx_vrt_SRF93_2}
\end{eqnarray}
We use (\ref{eqn:flx_nbr_vrt_slt_SRF93}) for $\flxnbrvrtslt$ to
rewrite the RHS of (\ref{eqn:flx_vrt_SRF93_2}) in terms of $\flxmsshrzslt$  
\begin{eqnarray}
\flxmssvrtdst 
& = & \frac{\mssdst \fshfct \dltnrg}{\nrgbnd } \times
\dnsatm \wndfrc^{2} \left( 1 - \frac{\wndfrcthr^{2}}{\wndfrc^{2} } \right) \times
[ \mssslt ( \uuuend - \uuusrt ) ]^{-1} \nonumber \\
& = & 
\frac{\mssdst \fshfct \dltnrg}{\nrgbnd } \times
\frac{\grv \flxmsshrzslt}{\cstslt \wndfrc } \times
[ \mssslt ( \uuuend - \uuusrt ) ]^{-1}
\label{eqn:flx_vrt_SRF93_3}
\end{eqnarray}

Let us now examine the $\dltnrg$ term.
If $\uuuend$ and $\wwwend$ are the mean streamwise and vertical
components of a saltator's velocity at impact, then its kinetic energy
is $\nrgend = \frac{1}{2} \mssslt ( \uuuend^{2} + \wwwend^{2} )$.
\cite{SRF93} assume that $\uuuend \gg \wwwend$ so that 
$\nrgend \sim \frac{1}{2} \mssslt \uuuend^{2}$.
Then, they assume each bombardment ejects only one saltator.
We call this \trmdfn{conservative bombardment} since, on average,
it is equivalent to an inelastic saltator reflection at each
impact-rupture-ejection event. 
In conservative bombardment every sand grain is the sole product of
one bombardment, i.e., $\nrgprd = \nrgsrt$.
Clearly this assumption is only plausible for steady state saltation.
In these conditions
\begin{eqnarray}
\dltnrg & = & \nrgend - \nrgprd \nonumber \\
& = & \frac{\mssslt ( \uuuend^{2} - \uuusrt^{2} )}{2}
\label{eqn:dlt_nrg_cns_bmb}
\end{eqnarray}

If the conditions for conservative bombardment are met, then we may
substitute (\ref{eqn:dlt_nrg_cns_bmb}) for $\dltnrg$ so that
\begin{eqnarray}
% SRF93 p. 12725 (14a)
\flxmssvrtdst 
& = & 
\frac{\mssdst \fshfct}{\nrgbnd } \times
\frac{\mssslt ( \uuuend + \uuusrt ) ( \uuuend - \uuusrt )}{2} \times
\frac{\grv \flxmsshrzslt}{\cstslt \wndfrc } \times
[ \mssslt ( \uuuend - \uuusrt ) ]^{-1} \nonumber \\
& = & 
\frac{\mssdst \fshfct}{\nrgbnd } \times
\frac{( \uuuend + \uuusrt )}{2} \times
\frac{\grv \flxmsshrzslt}{\cstslt \wndfrc } \nonumber \\
& = & 
\label{eqn:flx_vrt_SRF93_4}
\frac{\mssdst \grv \flxmsshrzslt}{\nrgbnd } \times
\frac{\fshfct}{\cstslt } \times
\frac{\uuuend + \uuusrt}{2 \wndfrc } \\
& \equiv & 
\label{eqn:hrz_vrt_prp_fct_SRF93}
\hrzvrtprpfct \flxmsshrzslt 
\end{eqnarray}
where $\hrzvrtprpfct$, which has dimensions of \xm, is the ratio of 
the dust vertical mass flux $\flxmssvrtdst$ to the streamwise saltation 
flux $\flxmsshrzslt$.
$\hrzvrtprpfct$ is called the \trmdfn{flux ratio}%
\footnote{This definition of
$\hrzvrtprpfct$ is identical to $\alpha$ in Equation~(40) of
\cite{MaB95}.}.
As described below, it is possible to infer $\hrzvrtprpfct$ from
measurements of $\flxmsshrzdst$ and $\flxmsshrzslt$ \cite[]{SRF93}.
Thus we shall compare the $\hrzvrtprpfct$ defined by
(\ref{eqn:hrz_vrt_prp_fct_SRF93}) to the $\hrzvrtprpfct$ predicted  
using alternate theories.

Inserting (\ref{eqn:flx_slt_SRF93_3}) back into
(\ref{eqn:flx_vrt_SRF93_4}), \cite{SRF93} defined a new parameter
$\alpha_{1}$ 
\begin{eqnarray}
% SRF93 p. 12725 (14b)
\flxmssvrtdst 
& = & 
\frac{\cstslt \dnsatm \wndfrc^{3}}{\grv }
\left( 1 - \frac{\wndfrcthr^{2}}{\wndfrc^{2} } \right) \times
\frac{\mssdst \grv}{\nrgbnd } \times
\frac{\fshfct}{\cstslt } \times
\frac{\uuuend + \uuusrt}{2 \wndfrc } \nonumber \\
& = & 
\wndfrc^{3} \left( 1 - \frac{\wndfrcthr^{2}}{\wndfrc^{2} } \right) \times
\frac{\mssdst \dnsatm \fshfct}{\nrgbnd } \times
\frac{\uuuend + \uuusrt}{2 \wndfrc } \nonumber \\
& \equiv & 
\alpha_{1} \wndfrc^{3}
\left( 1 - \frac{\wndfrcthr^{2}}{\wndfrc^{2} } \right)
\label{eqn:flx_vrt_SRF93_5}
\end{eqnarray}
where $\alpha_{1}$, whose units are \kgsSxmF, subsumes the final two
factors on the RHS of the preceding equation.
Note that the RHS of (\ref{eqn:flx_vrt_SRF93_4}) and
(\ref{eqn:flx_vrt_SRF93_5}) is a function of the size of the saltating
particles $\dmtslt$. 
Since surface soils comprise a continuous size distribution of
saltators, (\ref{eqn:flx_vrt_SRF93_4})--(\ref{eqn:flx_vrt_SRF93_5})
should be discretized into bins which each represent the dust
emissions produced by a given $\dmtslt$.
An example of this discretization in a global mineral dust model is
described in \S\ref{sxn:mpl}. 

The factors defining $\hrzvrtprpfct$ (\ref{eqn:hrz_vrt_prp_fct_SRF93}) 
and $\alpha_{1}$ (\ref{eqn:flx_vrt_SRF93_5}) are similar and bear
further examination.
Consider first the dimensionless speed factor   
$\gamma = (\uuuend + \uuusrt) / (2 \wndfrc) $.
Equations (19) and (20) of \cite{Owe64} are formulae for $\uuusrt$
and $\uuuend$ in terms of $\wndfrc$ and $\dmt$.
\cite{SRL96} evaluated these expressions numerically and found that
$\gamma \approx 2.5$ for most conditions. % SRL96 p. 316, p. 338
Hence the proportionality parameters from
(\ref{eqn:hrz_vrt_prp_fct_SRF93}) and (\ref{eqn:flx_vrt_SRF93_5}) are,
respectively, 
\begin{eqnarray}
\hrzvrtprpfct & = & \gamma \mssdst \grv \fshfct / ( \nrgbnd \cstslt ) \\
\alpha_{1} & = & \gamma \mssdst \dnsatm \fshfct / \nrgbnd 
\label{eqn:alpha_prx}
\end{eqnarray}
where $\gamma \approx 2.5$.
Knowing all the terms in (\ref{eqn:alpha_prx}) would allow 
us to predict vertical dust emissions at all scales.
The two terms which remain ill-defined are the mean kinetic energy
transfer efficiency, $\fshfct$, and the mean energy binding dust
particles to the surface, $\nrgbnd$ (\ref{eqn:dst_nbr_dfn}).  
Estimating $\fshfct$ and $\nrgbnd$ from first principles is currently
one of the most pressing challenges in theoretical studies of aeolian
erosion. 

\cite{SRL96} introduced a semi-empirical method for obtaining
$\fshfct$ and $\nrgbnd$.
First, they assumed that $\nrgbnd$ is the product of a length scale
determined by the dust particle size and the mean drag force due to
the friction wind  
\begin{eqnarray}
% SRL96 p. 316 (7)
\nrgbnd & = & \cstnrgbnd \dmt \times \frac{\mpi \dmt^{2}}{4} 
\dnsatm [ \wndfrcthr ( \dmtdst ) ]^{2} \nonumber \\
& = & \frac{\mpi \cstnrgbnd \dnsatm [ \wndfrcthr ( \dmtdst ) ]^{2} \dmtdst^{3}}{4}
\label{eqn:nrg_bnd_dfn}
\end{eqnarray}
where $\cstnrgbnd$ is the dimensionless length scale and is
$\mathcal{O}(1)$. 
Substituting (\ref{eqn:nrg_bnd_dfn}) into (\ref{eqn:alpha_prx}) 
\begin{eqnarray}
% SRL96 p. 318 (10)
\flxmssvrtdst 
& = & 
\frac{\gamma \mssdst \grv \fshfct}{\cstslt } \times 
\frac{4}{\mpi \cstnrgbnd \dnsatm [ \wndfrcthr ( \dmtdst ) ]^{2} \dmtdst^{3} } \times
\flxmsshrzslt
\nonumber \\
& = & 
\frac{\gamma \mssdst \grv \fshfct}{\cstslt } \times 
\frac{4}{\cstnrgbnd \dnsatm [ \wndfrcthr ( \dmtdst ) ]^{2} } \times 
\frac{\dnsprt}{6 \mssdst } \times
\flxmsshrzslt
\nonumber \\
& = & 
\frac{2}{3} \times 
\frac{\dnsprt}{\dnsatm } \times
\frac{\gamma \grv \fshfct}{\cstslt \cstnrgbnd [ \wndfrcthr ( \dmtdst ) ]^{2} } \times 
\flxmsshrzslt
\nonumber \\
& \equiv & 
\frac{2}{3} \times 
\frac{\dnsprt}{\dnsatm } \times
\frac{\bmbprm \gamma \grv}{[ \wndfrcthr ( \dmtdst ) ]^{2} } \times 
\flxmsshrzslt
\label{eqn:flx_vrt_SRF93_6}
\end{eqnarray}
where we have defined the \trmdfn{bombardment parameter}
$\bmbprm \equiv \fshfct / ( \cstslt \cstnrgbnd )$ in the last step.
The bombardment parameter contains most of the uncertainties in the
problem. 

We may invert (\ref{eqn:flx_vrt_SRF93_6}) to solve for $\bmbprm$
\begin{eqnarray}
% SRL96 p. 318 (12)
\bmbprm
& \equiv & 
\frac{3}{2} \times 
\frac{\dnsatm}{\dnsprt } \times
\frac{[ \wndfrcthr ( \dmtdst ) ]^{2}}{\gamma \grv } \times 
\frac{\flxmssvrtdst}{\flxmsshrzslt }
\label{eqn:bmb_prm_dfn}
\end{eqnarray}
In a wind tunnel experiment, all quantities on the RHS of
(\ref{eqn:bmb_prm_dfn}) are inputs known \textit{a priori} $(\dnsatm,
\dnsprt, \grv, \dmtdst, \wndspd)$, are determined by theory and/or
measurements $(\wndfrcthr, \gamma)$, or are directly measured after
the experiment $(\flxmsshrzslt,\flxmssvrtdst)$. 
The best empirical fit to the wind tunnel dataset gathered by
\cite{SRF93} is 
\begin{eqnarray}
% SRL96 p. 319 (16)
\bmbprm & = & 
[ 0.125 \times 10^{-4} \ln ( \dmtslt ) +
0.328 \times 10^{-4} ] 
\exp ( -140.7 \dmtdst + 0.37 )
\label{eqn:bmb_prm_obs}
\end{eqnarray}
where $\dmtdst, \dmtslt$ are in mm and $\bmbprm > 0$.

\cite{ShL99} further developed these theories and applied
them in a Large Eddy Simulation.

\subsection[Marticorena's Theory]{Marticorena's Theory}\label{sxn:vrt_MaB95}
\cite{MaB95} have synthesized many theories and observations into a
comprehensive global mineral dust emission model.
Many components of their model are described in the sections above.
We focus now on the components which are native to their model.

\cite{MaB95} begin by adapting the streamwise saltation results of
\cite{Whi79} for $\flxmsshrzslt$ (\ref{eqn:flx_mss_hrz_slt_Whi79}) to
account for more realistic soil conditions.
First, they accounted for the fraction of surface that actually
consists of erodible soils.
This fraction, $\sfcfrcslt$, includes only bare ground or sparsely
vegetated surfaces susceptible to saltation.
For example, $\sfcfrcslt$ excludes bare stone surfaces, swamps, and
lakes. 
However, $\sfcfrcslt$ does not exclude surfaces which require only
large friction speeds to initiate saltation, e.g., wetted soils.
Clearly $\flxmsshrzslt$ is linearly proportional to $\sfcfrcslt$.
Defining $\wndfrcrat$ as the ratio of threshold friction speed to
friction speed, (\ref{eqn:flx_mss_hrz_slt_Whi79}) may be rewritten as 
\begin{eqnarray}
\wndfrcrat & = & \wndfrcthr / \wndfrc \nonumber \\
\flxmsshrzslt(\dmt) & = & \frac{\cstslt \sfcfrcslt \dnsatm \wndfrc^{3}}{\grv } 
( 1 - \wndfrcrat ) ( 1 + \wndfrcrat )^{2}
\label{eqn:flx_mss_hrz_slt_MaB95_1}
\end{eqnarray}
where the functional dependence of $\flxmsshrzslt$ on $\dmt$ indicates 
that (\ref{eqn:flx_mss_hrz_slt_MaB95_1}) applies only to a
monodisperse distribution of particles of size $\dmt$.

In order to apply (\ref{eqn:flx_mss_hrz_slt_MaB95_1}) to a continuous 
size distribution $\dst(\dmt)$ in the source soil, some simplifying
assumptions are required.
First, \cite{MaB95} assume that the value $\cstslt = 2.61$,
empirically derived from monodisperse saltation experiments in MARSWIT 
\cite[]{Whi79}, is size-independent and applies equally to
heterogeneous, polydisperse size distributions.
Second, they assume that the mass flux $\flxmsshrzslt(\dmt)$ arises
only from motions of particle sizes between $\dmt$ and $\dmt +
\dltdmt$. 
In other words they assume that interactions between particles of
different sizes do not contribute significant errors to
(\ref{eqn:flx_mss_hrz_slt_MaB95_1}), which is based on monodisperse
assumptions.
This approximation breaks down in the limit of dust production by
ballistic impacts, since the impact of one large particle is assumed
to eject many small dust particles.  
\cite{SRL96} \csznote{p.317} contains a more thorough discussion of the
validity of these approximations.
Nevertheless, with these two assumptions, 
\begin{eqnarray}
% SRL96 p. 317 (8)
\flxmsshrzslt & = & \int_{0}^{\infty} \flxmsshrzslt(\dmt)
\pdffnc(\dmt) \,\dfr\dmt
\label{eqn:flx_mss_hrz_slt_MaB95_2}
\end{eqnarray}
where $\pdffnc(\dmt)$ is the PDF which defines the normalized,
fractional contribution of each size $\dmt$ to the total mass flux. 
These assumptions imply that $\cstslt = 2.61$ defines $\flxmsshrzslt$
through (\ref{eqn:flx_mss_hrz_slt_MaB95_1}), regardless of the
underlying size distribution of the parent soil. 
The only remaining difficulty is in determining $\pdffnc(\dmt)$.

\cite{MaB95} assume that $\pdffnc(\dmt)$ is best represented by the
fractional cross-sectional area distribution of the soil. 
This appears to be a reasonable assumption because, in the absence of
information to the contrary, the fractional surface area covered by
grains of a given size should vary linearly with the cross-sectional
area of the grains.  
Moreover, the exposure of saltation grains to wind also is
proportional to the cross-sectional area of the grains.
Another plausible assumption is made by \cite{SRL96}, who, as
described below, set $\pdffnc(\dmt)$ to the fractional mass
distribution of the soil.

The discretization of (\ref{eqn:flx_mss_hrz_slt_MaB95_2}) proceeds as
follows. 
Continuous soil size distributions are most often approximated as
multi-modal log-normal distributions, i.e., 
The three parameters required to define each mode.
The parameters usually available from soil sieving techniques are 
the mass median diameter $\dmtvma$, the geometric standard
deviation $\gsd$, and the mass fraction $\mssfrc$.
Thus it will be convenient to express $\pdffnc(\dmt)$ in terms of
$\mss(\dmt)$.  
To do so, we first put down the geometric relations between the
cross-sectional area, mass, and volume of spherical particles
\begin{equation}
\begin{array}{ r l >{\displaystyle}l<{} r >{\displaystyle}l<{} }
\xsa & = & \frac{\mpi \dmt^{2}}{4} & = & \frac{3}{2 \dnsprt \dmt } \mss \\[2.0ex]
\mss & = & \frac{\mpi \dmt^{3} \dnsprt}{6} & = & \frac{2 \dnsprt \dmt}{3} \xsa \\[2.0ex]
\end{array}
\label{eqn:area_mss_rlt}
\end{equation}
We now take the limit of (\ref{eqn:area_mss_rlt}) to define the
differential changes of area and mass in terms of eachother
\begin{equation}
\begin{array}{ r l >{\displaystyle}l<{} r >{\displaystyle}l<{} }
\dfr\xsa & = & \frac{\mpi \dmt}{2} \,\dfr\dmt & = & \frac{3}{2 \dnsprt
\dmt } \,\dfr\mss - \frac{3 \mss}{2 \dnsprt \dmt^{2} } \,\dfr\dmt \\[2.0ex]
\dfr\mss & = & \frac{\mpi \dmt^{2} \dnsprt}{2} \,\dfr\dmt & = & \frac{2 \dnsprt
\dmt}{3} \,\dfr\xsa + \frac{2 \dnsprt \xsa}{3} \,\dfr\dmt \\
\end{array}
% fxm: Make sure you understand the differential here: why no chain-rule?
\label{eqn:pdf_MaB95_1}
\end{equation}
We may normalize the increments of area and of mass by dividing
(\ref{eqn:pdf_MaB95_1}) by $\xsa$ and $\mss$, respectively
\begin{equation}
\begin{array}{ r l >{\displaystyle}l<{} r >{\displaystyle}l<{} r >{\displaystyle}l<{} }
\dfr\xsanrm & = & \frac{\dfr\xsa}{\xsa} & = & 
\frac{4}{\mpi \dmt^{2} } \times \frac{\mpi \dmt}{2} \,\dfr\dmt
& = & \frac{2}{\dmt } \,\dfr\dmt
\\[2.0ex]
\dfr\mssnrm & = & \frac{\dfr\mss}{\mss} & = & 
\frac{6}{\mpi \dnsprt \dmt^{3} } \times \frac{\mpi \dmt^{2} \dnsprt}{2} \,\dfr\dmt
& = & \frac{3}{\dmt } \,\dfr\dmt
\\[2.0ex]
\end{array}
% fxm: Make sure you understand the differential here: why no chain-rule?
\label{eqn:pdf_MaB95_2}
\end{equation}
By definition, $\int \dfr\xsanrm = 1$ and $\int \dfr\mssnrm = 1$.
Thus $\xsanrm$ and $\mssnrm$ are properly normalized PDFs and either
may take the place of $\pdffnc(\dmt)$ in (\ref{eqn:flx_mss_hrz_slt_MaB95_2}). 

Using (\ref{eqn:pdf_MaB95_2}), we may discretize
(\ref{eqn:flx_mss_hrz_slt_MaB95_2}) 
\begin{eqnarray}
% SRL96 p. 317 (9)
\sum_{\iii = 0}^{\iii = \NNN} \dltxsanrm_{\iii} & = & 1 \nonumber \\
\flxmsshrzslt & = & \sum_{\iii = 0}^{\iii = \NNN} \flxmsshrzslt \dltxsanrm_{\iii}
\label{eqn:flx_mss_hrz_slt_dsc_MaB95}
\end{eqnarray}

To convert this to a vertical dust flux $\flxmssvrtdst$, \cite{MaB95}
assume the relation 
\begin{eqnarray}
\flxmssvrtdst & = & \hrzvrtprpfct \flxmsshrzslt
\label{eqn:flx_vrt_MaB95}
% fxm: This needs work to keep labeling consistent \mbox{\hfill(\ref{eqn:hrz_vrt_prp_fct_SRF93})}  
\end{eqnarray}
Subject to the approximations discussed in \S\,\ref{sxn:vrt_SRF93},
(\ref{eqn:flx_vrt_MaB95}) was proved by \cite{SRF93}
(\ref{eqn:hrz_vrt_prp_fct_SRF93}).  
Instead of attempting to evaluate $\hrzvrtprpfct$ from first
principles, they take an empirical approach.

\cite{Gil79,Gil81} describes a dataset comprised of 
measurements of $\flxmssvrtdst$ and $\flxmsshrzslt$ at numerous
(anywhere from two to ten) friction velocities for each of nine
distinct soils. 
These data are reproduced in Table~\ref{tbl:txt_Gil79}. 
\begin{landscape}
\begin{table}
\begin{minipage}{\hsize} % Minipage necessary for footnotes KoD95 p. 110 (4.10.4)
\renewcommand{\footnoterule}{\rule{\hsize}{0.0cm}\vspace{-0.0cm}} % KoD95 p. 111
\begin{center}
\caption[Erodibility Properties of Nine Soils]{\textbf{Erodibility Properties of Nine Soils}% 
\footnote{\emph{Source:} \cite{Gil79,Gil81}, Tables~4.3 and~3,
respectively.}%  
\label{tbl:txt_Gil79}}   
\vspace{\cpthdrhlnskp}
\begin{tabular}{ *{9}{>{$}r<{$}} } % KoD95 p. 94 describes '*' notation
\hline \rule{0.0ex}{\hlntblhdrskp}% 
& & & & & & \multicolumn{3}{c}{Texture} \\[0.0ex]
\mbox{Soil} & \mbox{Soil moisture} & \mbox{Cloddiness} &
\mbox{Vegetative residue} & \mbox{Ridge roughness} & \mbox{Erosion
fetch} & \mbox{Sand} & \mbox{Silt} & \mbox{Clay} \\[0.0ex] 
& \vwc & & & & & \mssfrcsndpct & \mssfrcsltpct & \mssfrcclypct \\[0.0ex]
& \% & \% & \mbox{\gxmS} & \mbox{cm} & \mbox{km} & \% & \% & \% \\[0.0ex]
\hline \rule{0.0ex}{\hlntblntrskp}%
% Gil79 p. 87 Table 4.3
1 & 0.52 \pm 0.61 & 95.0 \pm 2.78 & 26.67 & 2.5 & 1.6 & 96.0 & 0.5 & 3.5 \\[0.5ex]
2 & 0.99 & 98.9 & 8.25 & 2.5 & 0.8 & 95.5 & 1.0 & 3.5 \\[0.5ex]
3 & 1.29 \pm 0.20 & 89.1 \pm 5.0 & 3.67 & 2.5 & 1.6 & 81.5 & 8.5 & 10.0 \\[0.5ex]
4 & 0.41 \pm 0.10 & 95.9 \pm 1.6 & 91.6 & 3.7 & 1.6 & 96.8 & 1.4 & 1.8 \\[0.5ex]
5 & 0.52 \pm 0.13 & 98.8 \pm 0.6 & 19.1 & 2.5 & 0.2 & 93.1 & 1.0 & 5.9 \\[0.5ex]
6 & 0.75 & 36.7 & 3.5 & 5.0 & 0.5 & 77.7 & 3.3 & 19.0 \\[0.5ex]
7 & 0.6 & 60.0 & 161.0 & 5.0 & 0.5 & 88.0 & 3.2 & 8.8 \\[0.5ex]
8 & 0.6 & 53.0 & 39.0 & 22.5 & 0.5 & 88.0 & 3.2 & 8.8 \\[0.5ex]
9 & 6.6 & 9.3 & 2.9 & 2.5 & 0.1 & 28.0 & 20.0 & 52.0 \\[0.5ex]
\hline
\end{tabular}
\end{center}
\end{minipage}
\end{table}
\end{landscape}
All soils except soil~9 (a clay) are very dry, with volumetric
water contents $\vwc < 0.013$ ($\vwc$ is defined in \S\,\ref{sxn:swc}).
These data show that more finely textured soils (i.e., with higher
clay content) produce more dust per unit $\flxmsshrzslt$ than coarser
soils (i.e., with higher sand content).

\cite{MaB95} showed that, in this dataset, the percentage of clay
particles ($\dmt < 2$\,\um) in the source soil explains more than 90\%
of the covariation of $\flxmssvrtdst$ with $\flxmsshrzslt$. 
Their best linear fit is 
\begin{eqnarray}
% MaB95 p. 16423 (47)
\log_{10} ( \flxmssvrtdst / \flxmsshrzslt ) & = & \log_{10} \hrzvrtprpfct \nonumber \\
& = & 0.134 \mssfrcclypct - 6.0 \quad : \quad
\mbox{$\mssfrcclypct < 20$, CGS} 
\label{eqn:hrz_vrt_prp_fct_MaB95_1}
\end{eqnarray}
where $\mssfrcclypct$ is the mass fraction (in percent) of clay
particles in the parent soil.
Thus $\hrzvrtprpfct$ is extremely sensitive to $\mssfrccly$,
increasing by nearly three orders of magnitude as $\mssfrccly$
the soil texture changes from $\mssfrccly = 0.0$ (sand) to $0.20$
(sandy loam). 
In SI units, (\ref{eqn:hrz_vrt_prp_fct_MaB95_1}) becomes
\begin{eqnarray}
% MaB95 p. 16423 (47)
\hrzvrtprpfct \equiv \flxmssvrtdst / \flxmsshrzslt & = & 
100 \exp [ ( 13.4 \mssfrccly - 6.0 ) \ln 10 ] \quad : \quad 
\mbox{$\mssfrccly < 0.20$, MKS} 
\label{eqn:hrz_vrt_prp_fct_MaB95_2}
\end{eqnarray}
Thus, $\hrzvrtprpfct$ is observed to increase exponentially with
$\mssfrccly$ for $\mssfrccly < 0.20$. 
For clayier soils the few available observations suggest
$\hrzvrtprpfct$ eventually begins to decrease with $\mssfrccly$. 
This reduced deflation efficiency for soils very rich in clay is
consistent with interparticle cohesive forces increasing the
mobilization inhibition as clay particles begin to dominate the soil 
\cite[]{Gil79,Gil81,MaB95}.
The physical arguments for mobilization inhibition due to
interparticle cohesive forces is discussed more quantitatively in 
\S\,\ref{sxn:mst} below.
Unfortunately the soils in Gillette's dataset with 
$\mssfrccly > 0.20$ had significantly higher soil water content than
did the soils with $\mssfrccly < 0.20$.
Thus it is difficult to use these data to form conclusions regarding
the behavior of $\hrzvrtprpfct$ for dry soils when $\mssfrccly > 0.20$.

In their regional dust model, \cite{MBA97} used a blended average
approach rather than using (\ref{eqn:hrz_vrt_prp_fct_MaB95_1})
directly. 
First they defined four distinct characteristic populations 
of coarse particles found in arid and semiarid regions based on
\cite{CMG96}.  
The mineralogical and geometric features of these populations are
described in Table~\ref{tbl:psd_MBA97}.
\begin{table}
\begin{minipage}{\hsize} % Minipage necessary for footnotes KoD95 p. 110 (4.10.4)
\renewcommand{\footnoterule}{\rule{\hsize}{0.0cm}\vspace{-0.0cm}} % KoD95 p. 111
\begin{center}
\caption[Characteristic Arid Soil Populations]{\textbf{Characteristic
Arid Soil Populations}% 
\footnote{\emph{Source:} \cite{CMG96} Table~8, \cite{MBA97} Tables~1 and~3. 
In this table Clay refers to particles with $\dmt < 5.8$\,\um.}%
\label{tbl:psd_MBA97}}   
\vspace{\cpthdrhlnskp}
\begin{tabular}{r l l >{$}r<{$} >{$}l<{$} >{$}r<{$} >{$}l<{$} }
\hline \rule{0.0ex}{\hlntblhdrskp}% 
Type & Code & Minerological & \mbox{Size} & \mbox{GSD} & \mbox{Clay} & \mbox{Efficiency} \\[0.0ex]
% fxm: Doublecheck that MBA97 used number-median diameters not mass
% median diameters here
& & Features & \dmtnma & \gsd & \mssfrccly & \hrzvrtprpfct \\[0.0ex]
& & & \mbox{\um} & & \% & \mbox{\xcm} \\[0.0ex]
\hline \rule{0.0ex}{\hlntblntrskp}%
% CMG96 p. 906 Table 8
% MBA97 p. 4390 Table 1 (typos!) p. 4395 Table 7
% AlG01 p. 18077 Table 3
% NB: MBA97 Table 1 contains apparent typo, where gsd of ASS and FS are switched
% CMG96 and MBA97 Table 7 and AlG01 agree
Aluminosilicated silt & ASS & Clay minerals dominant & 125 & 1.6 & 9.7 & 1.0 \times 10^{-5} \\[0.5ex] 
Fine sand & FS & Quartz dominant & 210 & 1.8 & 3.6 & 1.0 \times 10^{-6} \\[0.5ex] 
Coarse sand & CS & Quartz only & 690 & 1.6 & 0 & 1.0 \times 10^{-7} \\[0.5ex] 
Salts & Sa & Salt and clay minerals & 520 & 1.6 & 3.2 & 3.3 \times 10^{-6} \\[0.5ex]
\hline
\end{tabular}
\end{center}
\end{minipage}
\end{table}
The value of $\hrzvrtprpfct$ assigned to each population was arrived
at by blending the observed $\hrzvrtprpfct$ from the datasest of 
\cite{Gil79,Gil81}.
The first soil type, Aluminosilicated Silt, refers to many compounds
such as \AldOt, \SiOd, \ldots
The second soil type, Fine sand, includes \ldots
The third soil type, Coarse sand, includes \ldots
The fourth soil type, Salts, includes \ldots fxm

% Paris: Alfaro's talk
Alfaro and co-workers find that soil texture and composition affect
soil crusting and aggregation in wind-tunnel experiments on soils of
widely varying composition and texture. 
In particular, \CaCOt\ affects aggregation. % fxm: in what sense?
Interestingly, they found no detectable influence of texture and
composition on sandblasting efficiency per~se.
Thus, composition and texture may influence the threshold velocity 
of saltation, but have no direct affect on sandblasting.
This surprising result de-couples composition from sandblasting
and helps to simplify models.

% Paris Shao's talk
Shao represents the size distribution of saltators in air,
$\dstmsssltofdmt$ as a weighted combination of the size distribution
of the source soil in a ``natural'' (undisturbed) state,
$\dstmssntrofdmt$, and in a ``fully disturbed'' state,
$\dstmssfllofdmt$. 
\begin{eqnarray}
% Shao
\dstmsssltofdmt & = & \aaa \dstmssntrofdmt + (1-\aaa) \dstmssfllofdmt
\label{eqn:dst_mss_slt_dfn}
\end{eqnarray}
where $\aaa$ is fractional area of the fully disturbed soil.
The relationship between $\dstmssntr$ and $\dstmssfll$ must be
determined empirically. 

Aeolian soils are those formed with significant contributions from
upwind dust sources.
These soils are often called \trmdfn{loess}, although the specific
definition of loess is context-sensitive (fxm: add note loess
commission in INQUA).
Soils accumulate and form downwind of dust sources when bioclimatic
constraints such as vegetation and incisions trap near-surface clay
and silt particles.  
\cite{Muh83} shows that regional winds called \trmdfn{Santa Anas}
deposit silty loam soils on the \trmidx{Channel Islands} in
California. 
\cite{SGB05} and \cite{SBG05} describe the evidence for
loess-production in the the \trmdfn{Eureka Flat} flat region of the
Palouse in the northwest U.S.A\@. 

The composition of many of the most common minerals comprising dust is
given in Table~\ref{tbl:mnr_cmp}.
\begin{table}
\begin{minipage}{\hsize} % Minipage necessary for footnotes KoD95 p. 110 (4.10.4)
\renewcommand{\footnoterule}{\rule{\hsize}{0.0cm}\vspace{-0.0cm}} % KoD95 p. 111
\begin{center}
\caption[Mineral Composition of Windblown Dust]{\textbf{Mineral
Composition of Windblown Dust}
\label{tbl:mnr_cmp}}   
\vspace{\cpthdrhlnskp}
\begin{tabular}{>{\raggedright}p{8.0em}<{} >{\raggedright$\ch}p{9.0em}<{$} >{$}l<{$} r >{\raggedright}p{7.0em}<{} r }
\hline \rule{0.0ex}{\hlntblhdrskp}% 
Name & Formula & \mbox{Density\footnote{Source:
\url{http://www.ssc.on.ca/mandm}}} & Size\footnote{Clay defined as
$\dmt < 2$\,\um, Silt as $2 < \dmt < 50$\,\um. From \cite{CSB99}.} &
Properties\csznote{USGS has online reflectance spectroscopy by mineral type \url{http://speclab.cr.usgs.gov}}%
\footnote{\csznote{\cite{CSB99}, p. 22523}Aluminosilicates have similar radiative
properties both in the visible and the infrared.}
\setcounter{mltctt}{\value{footnote}}
& Ref.%  
\footnote{\emph{References:} 
\setcounter{enmrfr}{0} % Reset reference counter for this table
\enmrfrstpprn, \cite{PrK78}\label{idx_mnr_cmp_PrK78};
\enmrfrstpprn, \cite{Pye87}\label{idx_mnr_cmp_Pye87};
\enmrfrstpprn, \cite{DCZ96}\label{idx_mnr_cmp_DCZ96};
\enmrfrstpprn, Adkins et~al.\ (1999)\label{idx_mnr_cmp_Adk99};
\enmrfrstpprn, \cite{CSB99}\label{idx_mnr_cmp_CSB99};}%
 \\[0.0ex]
& & \dnsprt & & & \\[0.0ex]
& & \mbox{\gxcmC} & & & \\[0.0ex]
\hline \rule{0.0ex}{\hlntblntrskp}%
\footnote{\csznote{Calcite} Calcite. A very common mineral, principle
constituent of limestone. Neutralizes particle acidity, 
limits formation of \SOqdm, 
\NOtm.}% 
Calcite & \CaCOt & 2.95 & All & Weak IR bands, Limits acid uptake & \ref{idx_mnr_cmp_Pye87}, \ref{idx_mnr_cmp_DCZ96} \\[0.5ex] % http://www.ssc.on.ca/mandm/calcit.htm
%\footnote{\csznote{Carbonate}Any compound containing the radical \COtdp, e.g., calcite}%
%Carbonate & \COtdp & & & & \ref{idx_mnr_cmp_Pye87} \\[0.5ex] 
%\footnotemark[\value{mltctt}]\csznote{Feldspar}%
\footnote{\csznote{Feldspar}Feldspars form about 60\% of Earth's crust. Aluminosilicate. \M\ can be \K, \Na, \Ca, \Ba, \Rb, \Sr, \Fe.}%
Feldspar & \feldspar & \mbox{Various} & Silt & Weak IR & \ref{idx_mnr_cmp_Pye87} \\[0.5ex] 
\footnote{\csznote{Goethite}Goethite, aka hydrated iron oxide, may be the dominate optically active form of iron in mineral dust.}%
Goethite & \goethite & 3.8 & & Optically dominant? & \ref{idx_mnr_cmp_CSB99} \\[0.5ex] % http://webmineral.com/data/Goethite.shtml
\footnote{\csznote{Gypsum}Gypsum is a possible acid buffer?}%
Gypsum & \gypsum & 2.3 & & Limits acid uptake? & \ref{idx_mnr_cmp_CSB99} \\[0.5ex] % http://www.ssc.on.ca/mandm/gypsum.htm
\footnote{\csznote{Hematite}Hematite is a principle source of free
(soluble) iron.  
Strong visible and IR absorption bands. Mass fraction correlated to
soil ``redness'' index}%
Hematite & \hematite & 5.26 & Silt & Strong Visible \& IR & \ref{idx_mnr_cmp_CSB99} \\[0.5ex] % http://www.ssc.on.ca/mandm/hematit.htm, http://webmineral.com/data/Hematite.shtml
\footnote{\csznote{Illite}Illite are clay minerals, dominant in many
windblown dust samples, abundant in sedimentary rocks,
hygrophobic. Aluminosilicate.}%  
Illite & \illite & 2.75 & Clay & $\tpt, \RH$ tracer & \ref{idx_mnr_cmp_Pye87}, \ref{idx_mnr_cmp_CSB99} \\[0.5ex] % http://webmineral.com/data/Illite.shtml
\footnote{\csznote{Kaolinite}Kaolinite is an Aluminosilicate.}%
Kaolinite & \kaolinite & 2.6 & Clay & $\tpt, \RH$ tracer & \ref{idx_mnr_cmp_CSB99} \\[0.5ex] % http://www.ssc.on.ca/mandm/hematit.htm
\footnote{\csznote{Mica}Micas are phyllosilicates, dominant in many windblown dust samples}% 
Mica & \mica & & Silt & Weak IR & \ref{idx_mnr_cmp_Pye87} \\[0.5ex] % http://www.ssc.on.ca/mandm/montmor.htm
\footnote{\csznote{Opal}Opal is extractable from deep sea sediments.}%
Opal & \opal & 2.1 & All & Hygroscopic? & \ref{idx_mnr_cmp_Adk99} \\[0.5ex] % http://www.ssc.on.ca/mandm/opal.htm
\footnote{\csznote{Quartz}Quartz is nearly always a dominant constituent of windblown
dust.}%
Quartz & \quartz & 2.62 & All & Hygrophobic & \ref{idx_mnr_cmp_Pye87}, \ref{idx_mnr_cmp_CSB99} \\[0.5ex] % http://webmineral.com/data/Quartz.shtml
\footnote{\csznote{Smectite}Smectite is a group, particular values are for
Montmorillonite. Smectites are characterized by swelling in
water. Aluminosilicate.}% 
Smectite & \montmorillonite & 2.7 & Clay & Hygroscopic, $\tpt, \RH$ tracer & \ref{idx_mnr_cmp_PrK78}, \ref{idx_mnr_cmp_Pye87}, \ref{idx_mnr_cmp_CSB99} \\[0.5ex] 
\hline
\end{tabular}
\end{center}
\end{minipage}
\end{table}
Quartz (\quartz) is most commonly associated with sand.
Opal (\opal) content is often measured in oceanographic cores (e.g., Adkins).
\trmidx{Goethite} (\goethite) may be the optically dominant
constituent of mineral dust (Irina Sokolik, private communication,
2006). 
Unfortunately, its optical properties have not yet been measured.

For lack of other information, many studies assume mineral dust
comprises the same elemental distribution as is found, on average, in
Earth's \trmidx{mantle}.
Table~\ref{tbl:mnr_cmp} list the elemental composition of Earth's
crust according according to \cite{CRC95}. 
\begin{table}
\begin{minipage}{\hsize} % Minipage necessary for footnotes KoD95 p. 110 (4.10.4)
\renewcommand{\footnoterule}{\rule{\hsize}{0.0cm}\vspace{-0.0cm}} % KoD95 p. 111
\begin{center}
\caption[Elemental Composition of Earth's Crust and Surface Oceans]{\textbf{Elemental Composition of Earth's Crust and Surface Oceans}
\label{tbl:rth_cmp}}   
\vspace{\cpthdrhlnskp}
\begin{tabular}{>{\raggedright}p{8.0em}<{} l l}
\hline \rule{0.0ex}{\hlntblhdrskp}% 
Name & Mass fraction & \mgxl \\[0.0ex]
\Al & 0.0823 & 2e-3 \\[0.0ex] % CRC95 p. 14-11
\Ca & 0.0415 & 4.12e2 \\[0.0ex] % CRC95 p. 14-11
\Fe & 0.0563 & 2e-3 \\[0.0ex] % CRC95 p. 14-11
\K  & 0.0209 & 3.99e2 \\[0.0ex] % CRC95 p. 14-11
\Mg & 0.0233 & 1.29e3 \\[0.0ex] % CRC95 p. 14-11
\Na & 0.0236 & 1.08e4 \\[0.0ex] % CRC95 p. 14-11
\Ou & 0.461 & 8.57e5 \\[0.0ex] % CRC95 p. 14-11
\Si & 0.282 & 2.2 \\[0.0ex] % CRC95 p. 14-11
\hline
\end{tabular}
\end{center}
\end{minipage}
\end{table} % end tbl:rth_cmp

\cite{MBA97} found that the Western Sahara desert could be adequately 
described by eight different soil types.
Each of the eight possible soil types is a predefined blend of the
four soil populations described in Table~\ref{tbl:psd_MBA97}. 
Table~\ref{tbl:soi_typ_MBA97} defines the populations composing each
of the eight soil types.
\begin{landscape}
\begin{table}
\begin{minipage}{\hsize} % Minipage necessary for footnotes KoD95 p. 110 (4.10.4)
\renewcommand{\footnoterule}{\rule{\hsize}{0.0cm}\vspace{-0.0cm}} % KoD95 p. 111
\begin{center}
\caption[Western Sahara Soil Types]{\textbf{Western Sahara Soil Types}% 
\footnote{\emph{Source:} \cite{MBA97}, Table~2.
Soil types comprising the erodible surfaces of the Western Saharan
region. 
The populations are arranged in decreasing order of mass fraction. 
Each population corresponds to a characteristic soil described in
Table~\ref{tbl:psd_MBA97}.}% 
\label{tbl:soi_typ_MBA97}}   
\vspace{\cpthdrhlnskp}
\begin{tabular}{r l *{10}{>{$}r<{$}} } % KoD95 p. 94 describes '*' notation
\hline \rule{0.0ex}{\hlntblhdrskp}% 
Soil Type & Code & \multicolumn{3}{c}{Population 1} &
\multicolumn{3}{c}{Population 2} & \multicolumn{3}{c}{Population 3} &
\mbox{Flux Ratio} \\[0.0ex]
& & \dmtnma & \gsd & \mssfrc & \dmtnma & \gsd & \mssfrc & \dmtnma &
\gsd & \mssfrc & \hrzvrtprpfct \\[0.0ex]
& & \mbox{\um} & & \% & \mbox{\um} & & \% & \mbox{\um} & & \% & \mbox{\xcm} \\[0.0ex]
\hline \rule{0.0ex}{\hlntblntrskp}%
% MBA97 p. 4395 Table 7
Silty fine sand & SFS & 210 & 1.8 & 62.5 & 125 & 1.6 & 37.5 & - & - & - & 4.5 \times 10^{-6} \\[0.5ex]
Medium sand & MS & 690 & 1.6 & 80 & 210 & 1.8 & 20 & - & - & - & 5.5 \times 10^{-7} \\[0.5ex]
Coarse sand & CS & 690 & 1.6 & 100 & - & - & - & - & - & - & 1.0 \times 10^{-7} \\[0.5ex]
Coarse medium sand & CMS & 690 & 1.6 & 90 & 210 & 1.8 & 10 & - & - & - & 3.3 \times 10^{-7} \\[0.5ex]
Fine sand & FS & 210 & 1.8 & 100 & - & - & - & - & - & - & 1.0 \times 10^{-6} \\[0.5ex]
Silty medium sand & SMS & 125 & 1.6 & 37.5 & 210 & 1.8 & 31.25 & 690 & 1.6 & 31.25 & 4.2 \times 10^{-6} \\[0.5ex]
Moderately salty sand & SEM & 125 & 1.6 & 50 & 520 & 1.5 & 50 & - & - & - & 4.1 \times 10^{-6} \\[0.5ex]
Highly salty sand & SEF & 520 & 1.5 & 75 & 125 & 1.6 & 25 & - & - & - & 3.1 \times 10^{-6} \\[0.5ex]
\hline
\end{tabular}
\end{center}
\end{minipage}
\end{table}
\end{landscape}
The characteristics of the complete soil size distribution of a given
soil type are a weighted average of the chacteristics of each of the
soil sub-populations.
Thus the mean flux ratio for a given soil type with $\NNN$ distinct
soil subpopulations, shown in the last column of
Table~\ref{tbl:soi_typ_MBA97}, is computed as 
\begin{eqnarray}
% MBA97 p. 4390 (8)
\hrzvrtprpfct & = & \sum_{\iii = 1}^{\iii = \NNN} 
\hrzvrtprpfct_{\iii} \mssfrc_{\iii}
\label{eqn:hrz_vrt_prp_fct_MBA97}
\end{eqnarray}
where $\mssfrc_{\iii}$, given in Table~\ref{tbl:soi_typ_MBA97}, 
is the relative mass fraction of the $\iii$th subpopulation.
Each soil type is comprised of up to three log-normal size
distributions in the saltation range $125 \le \dmtnma \le 690$\,\um.

The dynamical model of \cite{MBA97} has $1\dgr \times 1\dgr$
horizontal resolution.
Using a GIS database of surface features, they subdivided each of
these dynamical gridpoints in the Western Sahara into at most five
subgrid soil types.
The total dust emission flux into each dynamical gridpoint was
then composed of the areal weighted fraction of its constituent soil
types. 

\subsection[My Theory]{My Theory}\label{sxn:vrt_csz}
One of the aspects of wind tunnel observations which has not been
explained by (\ref{eqn:flx_vrt_SRF93_4}) is the apparent increase
in bombardment efficiency, as measured by $\flxmsshrzdst/\flxmsshrzslt$
\cite[c.f.][Figure~5]{SRF93}.  

Relaxing the conservative bombardment assumption so that more than 
one saltator ejection per impact is allowed results in 
\begin{eqnarray}
\nrgprd & = & \sum_{\iii = 0}^{\iii = \dstprdfsh}
\frac{\mssslt \uuu_{\iii}^{2}}{2}
\label{eqn:nrg_prd_dfn}
\end{eqnarray}
where $\uuu_{\iii}$ is the initial streamwise velocity of the $\iii$th
product saltator.

\begin{eqnarray}
\flxmssvrtdst & = & \frac{\mssdst \fshfct \dltnrg}{\nrgbnd } \times
\dnsatm \wndfrc^{2} \left( 1 - \frac{\wndfrcthr^{2}}{\wndfrc^{2} } \right)
[ \mssslt ( \uuuend - \uuusrt ) ]^{-1} \\
& = & \mssdst \frac{\fshfct \dltnrg}{\nrgbnd } \times
\frac{\grv \flxmsshrzslt}{2 \wwwsrt \uuubar \mssslt } \\
& = & \mssdst \frac{\fshfct \dltnrg}{\nrgbnd } \times
\flxmsshrzslt [ \mssslt ( \uuuend - \uuusrt ) ]^{-1}
\label{eqn:flx_vrt_csz}
\end{eqnarray}

\section{Sub-Gridscale Properties}
Equation~(\ref{eqn:flx_hrz_dfn}) is valid only in an instantaneous or
equilibrium sense.
Prediction of wind erosion in large scale atmospheric models is
complicated by the non-linearity of (\ref{eqn:flx_hrz_dfn}).
To obtain the total horizontal flux in a given period of time we must
include the effects of spatial and temporal wind speed variation.

Consider a time-series of wind speeds $\wndspd(\tm)$ with
time mean wind speed $\wndspdbar$.
In practice $\wndspdbar$ is known either from observations or model
predictions, and represents the mean of the ``actual'' wind speed
$\wndspd(\xxx,\yyy,\tm)$ over given region and period of time.
In a typical large scale atmospheric model $\wndspdbar$ may be a
10--30~minute average wind speed over an area of $10^{4}$\,\kmS.
The non-linear, cubic relation between $\flxmsshrz$ and $\wndspdslt$
causes $\flxmsshrz(\wndspdbar)$ to differ significantly from
$\overline{\flxmsshrz(\wndspd)}$ on such scales. 
To more accurately estimate of $\flxmsshrzbar$ we must make
assumptions about the temporal and spatial distributions of the wind 
whose mean value is $\wndspdbar$.  

The importance of the sub-gridscale wind PDF cannot be overstated.
% Paris: Nickling talk
Both Aeolian and fluvial shear stress experiments show that the
tail-end of the fluid velocity distribution causes much, if not most, 
of the entrainment.
Since wind variability is related to the mean wind speed
\cite[]{JHM78} and saltation is driven by high wind speeds,
it follows that dust events are characterized by highly variable
winds. 
In turbulence studies, these are called \trmdfn{burst and sweep cycles}.
A ``burst'' is a high-intensity shear event which creates a micro-low
pressure zone.
The ``sweep'' occurs as the fluid fills in this low, often excavating
(eroding) the surface at the same time.
These events may remove the \trmidx{quasi-laminar layer}
(Section~\ref{sxn:qll}) \cite[]{SeH78}. 

We assume that an analytic \trmdfn{probability distribution function}
(PDF) governs the spatial and temporal distribution of surface wind
speeds.  
The PDF differential $\pdfwnd \,\dfr\wndspd$ is the probability that the
wind speed at a given time or location is between $\wndspd$ and 
$\wndspd + \dfr\wndspd$.
This is equivalent to the relative fraction of time or space\footnote{
In fact the spatial and temporal variability of wind are not
interchangeable.
The dust emission flux depends differently on the spatial and temporal
coherence of the wind.
The spatial coherence determines the \trmdfn{fetch}, or distance over
which the saltation flux develops to an equilibrium value.
Likewise, the temporal coherence determines whether the saltation has
enough time to reach an equilibrium value.
\cite{SRL96} discuss a method of separating these two processes.
However, we shall neglect this distinction in the present study.}
occupied by wind speeds between $\wndspd$ and $\wndspd + \dfr\wndspd$.
Since the probability of having a finite wind speed is exactly one, the
PDF must be normalized such that 
$\int_{0}^{\infty} \pdfwnd \,\dfr\wndspd = 1$.
A~suitable PDF thus acts as a weighting function to the instantaneous 
horizontal flux.
The temporal and spatial average horizontal flux is found by
convolving $\pdfwnd$ with (\ref{eqn:flx_hrz_dfn_3})
\begin{equation}
\flxmsshrz = \cst \int_{\wndspdthr}^{+\infty} \pdfwnd \,
\wndspdslt^{3} \left(1-\frac{\wndspdsltthr^{2}}{\wndspdslt^{2}} \right) 
\,\dfr\wndspd
\label{eqn:flx_hrz_pdf}
\end{equation}

\subsection[Weibull Distribution]{Weibull Distribution}
\cite{JHM78} showed that an analytic form well suited for describing 
the probability density function of the surface wind field is the
\trmdfn{Weibull distribution}.
More recent studies have justified the use of the Weibull distribution
for dust emissions in particular \cite[e.g.,][]{GiP88,SRL96,CMT04}.
We define and derive the analytic properties of the Weibull
distribution of wind speeds $\pdfwbl(\wndspd)$ (\ref{eqn:wbl_dst_dfn})
in \S\ref{sxn:wbl} below.    
In a Weibull distribution the frequency of occurrence of winds
exceeding~$\wndspdthr$ decreases exponentially with~$\wndspdthr$,
i.e., $\pdfwbl(\wndspd > \wndspdthr) =
\exp(-\wndspdthr/\wblscl)^{\wblshp}$ where $\wblscl$ and~$\wblshp$ are,
respectively, the \trmidx{scale parameter} and 
\trmidx{shape parameter} of the Weibull distribution
(\ref{eqn:prb_dfn}).   
Employing $\pdfwbl(\wndspd)$ in (\ref{eqn:flx_hrz_pdf}) we obtain 
\begin{equation}
\flxmsshrz = \cst \int_{\wndspdthr}^{+\infty} 
\frac{\wblshp}{\wblscl} 
\left( \frac{\wndspd}{\wblscl } \right)^{\wblshp - 1} 
\exp \left[ - \left( \frac{\wndspd}{\wblscl } \right)^{\wblshp} \right]
\wndspdslt^{3} \left( 1 - \frac{\wndspdsltthr^{2}}{\wndspdslt^{2}} \right) 
\,\dfr\wndspd
\label{eqn:flx_hrz_wbl}
\end{equation}

As shown in \S\ref{sxn:wbl}, integrals of the form 
$\wndspd^{\nnn} \pdfwbl(\wndspd)$ may be expressed analytically in
terms of the incomplete gamma function. 
Applying (\ref{eqn:wbl_dst_mmn}) to (\ref{eqn:flx_hrz_wbl}) we obtain
\begin{equation}
\flxmsshrz = 
\label{eqn:flx_hrz_gmm}
\end{equation}

\subsection[Dust Devils]{Dust Devils}\label{sxn:dst_dvl}
\cite{GiS90} estimated that the injection of mineral dust into the
atmosphere over the United States by \trmidx{dust devils} is
comparable to that injected by large scale dust mobilization.
Renno (fxm: find this reference) also suspect dust devils are
important.

\cite{GGA03} investigated weak dust emission processes, i.e.,
processes important in weak wind conditions when saltation is minimal
and dust fluxes are small.
They found emissions in the Aral Sea region are consistent with a 
dust devils and, possibly, with \trmidx{electrophoresis}.
\cite{YaA97} explain the conditions under which electrostatic effects
may be import for dust liftoff.
\cite{KoR05} develop a theory for dust entrainment by dry convective
plumes and vortices, and estimate such dust devils are responsible for
about one-third of total dust mobilization on Earth.

\section{Drag Partitioning}\label{sxn:drg}
In a series of papers M.~Raupach and colleagues have developed a 
theory of the \trmidx{drag partition} between erodible and
non-erodible surfaces \cite[][]{Rau91,Rau92,RGL93,Rau94}.
Part of this theory recognizes that \trmidx{saltation} alters
the roughness characteristics of the boundary layer.
In neutral conditions the relation between \trmidx{wind speed}
$\wndspd$ and \trmidx{wind friction speed} $\wndfrc$ is a logarithmic 
profile (\ref{eqn:wnd_ntr_dfn}). 
The relationship depends on height $\hgtatm$ above the surface
and the \trmidx{roughness length}.
In a non-saltating environment, (\ref{eqn:wnd_ntr_dfn}) may be
re-expressed as 
\begin{eqnarray}
\wndntr ( \hgtatm ) & = & 
\frac{\wndfrc}{\vonkrmcst} \ln \left( \frac{\hgtatm}{\rghmmnnonslt} \right)
\label{eqn:wnd_ntr_dfn_2}
\end{eqnarray}
In (\ref{eqn:wnd_ntr_dfn_2}), the symbol $\rghmmnnonslt$ is
the \trmidx{non-saltating roughness length}.
The \trmdfn{Owen effect} is the name given to the positive feedback of
saltation on wind erosion via roughness length increases.
\cite{Owe64} showed that the saltation layer which develops during
wind erosion events acts as an additional sink of atmospheric
momentum, causing an increase in momentum deposition to the surface
and thus an increase in roughness length.
Saltating particles mediate this effect by removing momentum from the
atmosphere and transferring it to the surface during each
(non-ballistic) impact. 

\cite{Rau91} showed that the Owen effect entails an implicit 
relationship between the \trmdfn{saltating roughness length}
$\rghmmnslt$ in terms of the \trmdfn{non-saltating roughness length}
$\rghmmnnonslt$
\begin{eqnarray}
% GMB98 p. 6204 (2)
\rghmmnslt & = & \left( \AAA 
\frac{\wndfrc^{2}}{2 \grv} \right)^{1-\wndfrcrat} \rghmmnnonslt^{\wndfrcrat}
\label{eqn:rgh_mmn_slt}
\end{eqnarray}
where $\AAA$ is a constant of order unity, $\grv$ is the acceleration
of gravity, and $\wndfrcrat$ is the ratio of $\wndfrcthr$ to $\wndfrc$
(\ref{eqn:flx_mss_hrz_slt_MaB95_1}).
Equation~(\ref{eqn:rgh_mmn_slt}) must be solved iteratively
because the friction speed $\wndfrc$ itself depends upon $\rghmmnslt$ 
(cf. \S\ref{sxn:rgh}).
Studies have verified that roughening of the surface due to the Owen
effect could be quite dramatic \cite[]{GMB98}, although difficult to
predict using (\ref{eqn:rgh_mmn_slt}).

\cite{GMB98} tested Raupach's theory \cite[]{Rau91} of the Owen
effect \cite[]{Owe64} against data from Owens Lake, California.  
Using measurements of saltation events taken at three stations
over the course of a year, \citeauthor{GMB98} not only verified
(\ref{eqn:rgh_mmn_slt}), but parameterized it into a form easier
to evaluate.
First, they defined the \trmdfn{saltating friction velocity}
$\wndfrcslt$ as the friction velocity measured during fully developed 
saltation, when the roughness length $\rghmmn \equiv \rghmmnslt$, the
saltating roughness length. 
Under neutral conditions, and with reference to
(\ref{eqn:wnd_ntr_dfn_2}), the relation between $\wndntr$,
$\wndfrcslt$, and $\rghmmnslt$ is 
\begin{eqnarray}
% GMB98 p. 6205 (4)
\wndntr ( \hgtatm ) & = & 
\frac{\wndfrcslt}{\vonkrmcst} \ln \left( \frac{\hgtatm}{\rghmmnslt} \right)
\label{eqn:wnd_ntr_dfn_3}
\end{eqnarray}

\cite{GMB98} reasoned that the saltating friction speed $\wndfrcslt$ 
is composed of two parts, the non-salting friction speed $\wndfrc$ and 
the contribution due to the Owen effect.
Defining the difference between the saltating and non-saltating
friction speeds as
\begin{eqnarray}
% GMB98 p. 6208 (10)
\wndfrcdlt & = & \wndfrcslt - \wndfrc
\label{eqn:wnd_frc_dlt_dfn}
\end{eqnarray}
they used the Owen's Lake data to parameterize $\wndfrcdlt$ in terms
of the 10\,m wind speed $\wndten$ and the 10\,m threshold wind speed
$\wndtenthr$.
\begin{eqnarray}
% GMB98 p. 6209 (12)
\wndfrcdlt & = & 0.003 (\wndten - \wndtenthr)^{2}
\label{eqn:wnd_frc_dlt_pzn}
\end{eqnarray}
According to (\ref{eqn:wnd_frc_slt_dfn}), the saltating friction
speed $\wndfrcslt$ increases quadratically with windspeed in excess of
the the friction speed\footnote{The relationship in \cite{GMB98}
equation (12) contains a units conversion error and the coefficient is
$0.003$, not $0.3$ \cite[]{Gil99}.}
Since $\wndfrcslt$ determines the saltation mass flux 
[e.g., (\ref{eqn:flx_hrz_dfn_2}), (\ref{eqn:flx_hrz_bgn}),
(\ref{eqn:flx_hrz_bgn_2}), (\ref{eqn:flx_mss_hrz_slt_Whi79})]
neglecting the Owen effect (\ref{eqn:wnd_frc_dlt_pzn}) would cause
significant underprediction of saltation\footnote{% Paris: Zobeck talk
The Owen effect has not been conclusively measured at locations
other than Owen's Lake (the names are coincidentally the same).
Zobeck could not discern the effect in data from Texas.}.
Rearranging (\ref{eqn:wnd_frc_dlt_dfn}) and using
(\ref{eqn:wnd_frc_dlt_pzn}), the saltating friction speed is
\begin{eqnarray}
% GMB98 p. 6209 (12)
\wndfrcslt & = & \wndfrc + 0.003 (\wndten - \wndtenthr)^{2}
\label{eqn:wnd_frc_slt_dfn}
\end{eqnarray}

The proliferation of terminology regarding friction speeds can lead to
confusion.
With saltating and non-saltating friction speeds, smooth and
non-erodible roughness lengths, drag partition and wind friction
efficiency, it may be difficult to understand how to predict wind
erosion given output of a large scale atmospheric model.
It is important to remember that the wind friction speed $\wndfrc$ 
predicted in such models is generally \textit{not} the relevant
friction speed to employ in dust emissions schemes.
First, large scale models use a roughness length $\rghmmn$ typifying 
momentum exchange on a scale of order 100\,km to predict $\wndfrc$. 
\begin{eqnarray}
\wndfrc, \rghmmn & = & \left\{
\begin{array}{>{\displaystyle}r<{} @{\quad:\quad}l}
\wndfrcnonslt, \rghmmnnonslt
& \wndfrc < \wndfrcthr \\
\wndfrcslt, \rghmmnslt & \wndfrc \ge \wndfrcthr
\end{array} \right.
\label{eqn:wnd_frc_rng}
\end{eqnarray}
The wind friction speed $\wndfrc$ and roughness length $\rghmmn$
measured in wind tunnels changes qualitatively once saltation
develops. 

The ideas behind drag partitioning are quite general.
\cite{Sul02} applied the drag partitioning theory of \cite{Rau92} and
\cite{RGL93} to \trmidx{Mars}.
\citeauthor{Sul02} estimates that Martian threshhold friction velocity
and surface wind speed (at 1.6\,\m\ height) must be near 
$\wndfrcthr \approx 4.5$\,\mxs\ and $\wndspd \approx 45$\,\mxs\ near
the Pathfinder landing site.
\cite{MCT04} generalize Rauchpach's drag partition theory to apply to 
an arbitrary number of roughness surfaces.
They applied this generalization to measured and modeled surfaces in
the Mojave Desert.
\cite{Oki08} develop a shear-stress partitioning model arguably
superior to Raupach's model.

As the preceding section shows, the susceptibility of soils to wind
erosion is highly sensitive to particle size. 
In nature, soils are composed not of a discrete, monodisperse size of
particles, but of a continuous distribution of particles and
compositions which define the \trmdfn{soil texture}.

\section{Dependence On Soil Moisture}\label{sxn:mst}
% Gil88, GiP88 p. 14236 Tbl. 3, SRL96 p. 322, MaB95 p. 16417, 
As is clear from many experiments, interparticle cohesive forces
rapidly begins to dominate the moment balance on dust particles as 
the size of the particles falls beneath about 40\,\um\
\cite{Whi79,IvW82,GrI85}.
Most of the aforementioned experiments, however, were performed upon
dry soil beds.
Thus these experiments suit completely dry environments such as Mars,
but idealize field conditions in partially wet environments such as
the semi-arid regions of Earth.
The notion that a wet soil bed will not detrain any dust particles
agrees with experience, and the quantification of this phenomena for 
the entire soil spectrum from dry to saturated is the topic of this
section. 
These effects are especially important when considering the
persistance of soil moisture perturbations or \trmdfn{soil moisture
  memory} \cite[e.g.,][]{KoS01}. 

See the discussion in \cite{Pye87}, p.~31.

\subsection[Soil Water Content]{Soil Water Content}\label{sxn:swc}
In this section we summarize the traditional measures of soil-water
interaction. 
The following sections will build upon these definitions to present 
parameterization of the influence of soil moisture on $\wndfrcthr$.

The fundamental determinant of soil-water interactions is the amount
of water contained in the soil.
The most easily (and frequently) measured descriptor of soil water
content is the \trmdfn{gravimetric water content} $\gwc$ on a dry-mass 
basis
\begin{eqnarray}
% Hil82 p. 10 (2.5)
\gwc & = & \mssHdO / \msssoidry
\label{eqn:gwc_dfn}
\end{eqnarray}
where $\mssHdO$ is the mass of liquid water and $\msssoidry$ is the mass of  
\trmdfn{dry soil}.
The dryest condition attained by a soil in nature is called \trmdfn{air
dry}. 
The dryest condition to which a soil can be brought in the laboratory
is called \trmdfn{oven dry}.
By convention, $\msssoidry$ is defined as the mass of soil in equilibrium
with an oven at a temperature of $105\dgr$C \cite[]{Hil82}.
Even at this temperature, clay soils may retain appreciable moisture,
i.e., up to a few percent.

The wettest possible condition of soil, in which all pores are filled
with water, is called \trmdfn{saturated soil}.
We shall affix the subscript $\satsbs$ to quantities to indicate the
saturation condition, e.g., $\gwcsat$ is the gravimetric water content
at saturation.
The value of $\gwcsat$, depends on the soil type, but typically $0.25
< \gwcsat < 0.60$. 
Soils rich in organic matter, e.g., peat, may have $\gwcsat > 1$.
Occasionally in the literature the gravimetric water content is
expressed on a wet-mass basis, which is simply defined by replacing
$\msssoidry$ with $\msssoimst$, the mass of moist soil, in
(\ref{eqn:gwc_dfn}). 
In this text we shall only use $\gwc$ defined on a dry-mass basis, as
in (\ref{eqn:gwc_dfn}).

The other frequently used measure of soil moisture is the
\trmdfn{volumetric water content} $\vwc$ 
\begin{eqnarray}
% Hil82 p. 10 (2.6)
\vwc & = & \frac{\vlmHdO}{\vlmHdO + \vlmprt + \vlmair} \\
& = & \vlmHdO / \vlmblk \nonumber
\label{eqn:vwc_dfn}
\end{eqnarray}
$\vlmHdO$, $\vlmprt$, and $\vlmair$ are the volumes of water, (moist)
soil particles, and interparticle gases (i.e., air) in the sample.
The total volume or \trmdfn{bulk volume} $\vlmblk$ of the soil sample 
is defined as 
\begin{eqnarray}
\vlmblk & = & \vlmHdO + \vlmprt + \vlmair
\label{eqn:vlm_blk_dfn}
\end{eqnarray}
$\vlmprt$ measures only the volume of the soil particles themselves, 
whereas $\vlmblk$ includes volume occupied by porous interparticle
spaces and by water.
Thus $\vwc$ is the fractional volume of liquid water relative to the
bulk volume (water plus soil plus air) of the soil.
Typically it is assumed that $\vlmHdO \ll \vlmprt$ and 
$\vlmair \ll \vlmprt$ so that (\ref{eqn:vlm_blk_dfn}) may be
approximated as 
\begin{eqnarray}
\vlmblk & \approx & \vlmprt
\label{eqn:vlm_blk_apx}
\end{eqnarray}
This approximation, which appears to be standard in the literature,
may be inappropriate for loose, wet soils.
As discussed above, $\vwcsat$ will denote the volumetric water content
at saturation.  

A further distinction may be made between the volume of ambient soil
particles $\vlmprt$ and the volume of the dry soil particles
$\vlmprtdry$. 
The two quantities are related by the swelling of the soil under moist
conditions which depend on the amount of swelling.
We define the dimensionless soil \trmdfn{swelling function} $\swlfnc$  
to depend on gravimetric water content and on soil mineralogy
$\mnrfnc$ as follows  
\begin{eqnarray}
\vlmprt & = & \vlmprtdry \swlfncofgwcmnr
\label{eqn:swl_fnc_dfn}
\end{eqnarray}
For minerals which do not absorb water, e.g., quartz,
$\swlfncofgwcmnr = 1$, while $\swlfncofgwcmnr > 1$ for minerals 
which swell with water, e.g., smectites (Table~\ref{tbl:mnr_cmp}).
Note that $\vwc$ (\ref{eqn:vwc_dfn}) is defined relative to the bulk
soil volume $\vlmblk$ while $\gwc$ (\ref{eqn:gwc_dfn}) is defined
relative to the dry soil mass $\msssoidry$.
This can lead to confusion since ambient soil volume $\vlmprt$ depends
on water content for some soils (\ref{eqn:swl_fnc_dfn}).
Although measurement of $\gwc$ is more straightforward than $\vwc$,  
the literature appears to favor use of $\vwc$ hydrologic theory. 
\begin{table}
\begin{minipage}{\hsize} % Minipage necessary for footnotes KoD95 p. 110 (4.10.4)
\renewcommand{\footnoterule}{\rule{\hsize}{0.0cm}\vspace{-0.0cm}} % KoD95 p. 111
\begin{center}
\caption[Soil Volume, Mass, Density and Moisture]{\textbf{Soil
Volume, Mass, Density and Moisture}%
\footnote{\emph{Sources:}}%
\label{tbl:soi_mtr}}
\vspace{\cpthdrhlnskp}
\begin{tabular}{ >{$}r<{$} l >{}p{22.0em}<{} }
\hline \rule{0.0ex}{\hlntblhdrskp}% 
\mathrm{Symbol} & Name & Meaning \\[0.0ex]
\hline \rule{0.0ex}{\hlntblntrskp}%
\vlmprt & Soil Volume & Ambient soil particle volume (moist,
non-porous) \\[0.5ex] 
\vlmblk & Bulk Soil Volume & Ambient soil volume (moist,
porous) \\[0.5ex] 
\vlmblkdry & Bulk Dry Soil Volume & Dry soil volume (porous) \\[0.5ex] 
\vlmprtdry & Dry Soil Volume & Dry soil particle volume (non-porous) \\[0.5ex] 
\vlmwtr & Water Volume & Interparticle water volume \\[0.5ex] 
\vlmair & Air Volume & Porous interparticle space \\[0.5ex]
\hline
\end{tabular}
\end{center}
\end{minipage}
\end{table}

Converting between $\vwc$ and $\gwc$ simply requires knowledge of the
densities of water $\dnsHdO$ and of dry soil particles $\dnsprtdry$.
If we assume $\vlmblk = \vlmblkdry$
\begin{eqnarray}
% http://www.sowacs.com/feature/IMKO/211_def.htm
\gwc & \equiv & \frac{\mssHdO}{\msssoidry} = 
\frac{\vlmHdO}{\vlmblkdry} \times 
\frac{\dnsHdO}{\dnsblkdry} \nonumber \\
& = & 
\frac{\vlmHdO}{\vlmblk} \times 
\frac{\dnsHdO}{\dnsblkdry} \nonumber \\
& = & 
\vwc \dnsHdO / \dnsblkdry
\end{eqnarray}
The so-called \trmidx{dry density} of the soil, $\dnsblkdry$, is
defined by the dry mass of the soil $\msssoidry$ and the bulk volume
$\vlmblk$ (\ref{eqn:vlm_blk_dfn}) of the soil
\begin{eqnarray}
\dnsblkdry \equiv 
\frac{\msssoidry + \mssair}{\vlmHdO + \vlmprt + \vlmair} \nonumber \\
\approx \frac{\msssoidry}{\vlmblk}
\label{eqn:dns_blk_dry_dfn}
\end{eqnarray}
where we have made the approximation that $\mssair \ll \msssoidry$
in the second equation.
The operational determination of $\dnsblkdry$
(\ref{eqn:dns_blk_dry_dfn}) must be made in two stages. 
First the ambient (moist) bulk soil sample volume $\vlmblk$ must be
determined. 
Next, the sample is dried so that the soil dry mass $\msssoidry$ may
be determined.
Therefore natural soil ``in the field'' never has $\dns = \dnsblkdry$.
Soil models employ a variety of strategies to determine $\dnsblkdry$ 
(\ref{eqn:dns_blk_dry_dfn}).
One such strategy is to assume knowledge of the soil particle density
$\dnsprt$ \ldots

If we do not assume $\vlmblk = \vlmblkdry$, then
\begin{eqnarray}
\gwc & \equiv & \frac{\mssHdO}{\msssoidry} = 
\frac{\vlmHdO}{\vlmprtdry} \times 
\frac{\dnsHdO}{\dnsprtdry} \times
\frac{\vlmblk}{\vlmblk} \nonumber \\
& = & 
\frac{\vlmHdO}{\vlmblk} \times 
\frac{\dnsHdO}{\dnsprtdry} \times
\frac{\vlmblk}{\vlmprtdry} =
\vwc \times \frac{\dnsHdO}{\dnsprtdry} \times \frac{\vlmblk}{\vlmprtdry}
\nonumber \\
& \approx & \vwc \dnsHdO /\dnsprtdry
\label{eqn:gwc_vwc}
\end{eqnarray}
where the approximation (\ref{eqn:vlm_blk_apx}) has been made in the
final step.
Alternatively, we can define $\vwc$ in terms of $\gwc$ as
\begin{eqnarray}
% http://www.sowacs.com/feature/IMKO/211_def.htm
\vwc & \equiv & \frac{\vlmHdO}{\vlmblk} = 
\frac{\mssHdO}{\dnsHdO} \times \frac{\dnsblk}{\mssblk}  =
\frac{\mssHdO}{\mssHdO + \msssoidry + \mssair} \times
\frac{\dnsblk}{\dnsHdO} \nonumber \\
& \approx & 
\frac{\mssHdO}{\mssHdO + \msssoidry} \times
\frac{\msssoidry}{\msssoidry} \times
\frac{\dnsblk}{\dnsHdO} \nonumber \\
& = & 
\frac{\mssHdO}{\msssoidry} \times 
\frac{\msssoidry}{\mssHdO + \msssoidry} \times 
\frac{\dnsblk}{\dnsHdO} \nonumber \\ 
& = & 
\gwc \times \frac{\dnsblk}{\dnsHdO} \times
\frac{\msssoidry}{\mssHdO + \msssoidry} \nonumber \\
& \approx & \gwc \dnsblk / \dnsHdO
\label{eqn:vwc_gwc}
\end{eqnarray}
where the last step has used an approximation, analogous to
(\ref{eqn:vlm_blk_apx}) that
\begin{eqnarray}
\mssblk = \mssHdO + \msssoidry + \mssair \approx \msssoidry 
\label{eqn:mss_blk_apx}
\end{eqnarray}
Like (\ref{eqn:vlm_blk_apx}), approximation (\ref{eqn:mss_blk_apx}) 
is only valid if $\mssHdO \ll \msssoidry$ and so breaks down for very
wet soils. 
Conversely, approximations (\ref{eqn:vlm_blk_apx}) and
(\ref{eqn:mss_blk_apx}) are both well-suited for use determining
moisture inhibition of dust deflation in arid soils.
As described in \S\ref{sxn:Bel64}--\ref{sxn:FMB99}, moisture
inhibition of deflation is virtually assured for 
$\gwc \gtrsim 0.2$\,\kgxkg. 

By convention, both $\gwc$ and $\vwc$ are reported as percentages
(i.e., multiplied by 100) rather than as fractions. 
When a soil moisture or mass value is to be expressed in percent,
we superscript it with the \% symbol, i.e., 
$\vwcpct \equiv 100 \vwc$ and $\gwcpct \equiv 100 \gwc$.
Otherwise the value is assumed to be a fraction.

\subsection[Characterization of Soil Water Energy]{Characterization of Soil Water Energy}\label{sxn:soil_wtr_nrg}
The energy of a water parcel in soil determines its hydraulic behavior.
The \trmdfn{total potential energy} of a parcel of soil water,
$\nrgptnsoi$, is the sum of the forces acting on the water which cause
it to differ from a surface of pure, bulk water.
These forces include gravitation, pressure, and osmotic properties,
respectively:
\begin{eqnarray}
\nrgptnsoi & = & \nrgptnsoigrv + \nrgptnsoiprs + \nrgptnsoiosm
\label{eqn:nrg_ptn_ttl_dfn}
\end{eqnarray}
Both the gravitational potential $\nrgptnsoigrv$ and the pressure
potential $\nrgptnsoiprs$ may be converted to mechanical (kinetic) energy
of flow in open soils.
$\nrgptnsoigrv$ and $\nrgptnsoiprs$ are discussed in more detail below.
The osmotic potential $\nrgptnsoiosm$ is due to the presence of salts in 
the water, and will not be discussed further.
\trmdfn{Darcy's law} states that the mass flux of water flowing
through a surface is proportional to the gradient of $\nrgptnsoi$
across the surface. 
Of course water tends to flow downgradient, i.e., from high to low
$\nrgptnsoi$. 

The gravitational potential of a parcel of soil water is defined in
exact analogy to the potential energy of an air parcel.
Thus the gravitational potential $\nrgptnsoigrv$ of a mass $\mss$
at a height $\hgt$ is
\begin{eqnarray}
\nrgptnsoigrv & = & \int_{\hgtrfr}^{\hgt} \mss \grv(\hgt) \,\dfr\hgt
\label{eqn:nrg_ptn_grv_dfn}
\end{eqnarray}
where $\hgtrfr$ is the height of the reference level.
In soil energy studies it is common to define the reference level
as the level of the soil surface, i.e., $\hgtrfr = 0$.
For the remainder of this work we take $\hgtrfr = 0$ and assume that 
$\grv(\hgt)$ equals the mean acceleration of gravity at the Earth's
surface, $\grv$.
Then (\ref{eqn:nrg_ptn_grv_dfn}) reduces to
\begin{eqnarray}
% Hil82 p. 68 (5.8)
\nrgptnsoigrv & = & \mss \grv \hgt \\
& = & \dnslqd \vlm \grv \hgt
\label{eqn:nrg_grv_Hil82}
\end{eqnarray}
where $\dnslqd$ is the density of liquid water and $\vlm$ is its
volume. 
Thus $\nrgptnsoigrv < 0$ for water beneath the soil surface.

The \trmdfn{pressure potential} of a water parcel is due to the
displacement of the parcel relative to a free surface of water.
In other words, a water parcel at the surface of a free body of water
feels no pressure potential.
A parcel residing at a depth $\ddd$ beneath the free surface of water 
feels a total pressure $\prsttl$ which is the sum of the atmospheric
pressure $\prsatm$ and the hydrostatic pressure $\prslqd$ due to the
column of liquid water above it
\begin{eqnarray}
% Hil82 p. 68 (5.8)
\prsttl & = & \prsatm + \prslqd \nonumber \\
& = & \prsatm + \dnslqd \grv \ddd
\label{eqn:prs_prc}
\end{eqnarray}
If the volume of the parcel is $\vlm$, then the pressure potential
energy of the parcel is 
\begin{eqnarray}
% Hil82 p. 68 (5.12)
\nrgptnsoiprs = \prslqd \vlm
\label{eqn:nrg_ptn_prs_dfn}
\end{eqnarray}
Thus the pressure potential per unit volume is 
$\nrgptnsoiprsvlm / \vlm = \prslqd$.

In soil energy studies it is very common to normalize the total
potential energy of the water by the amount of water.
When applied to (\ref{eqn:nrg_grv_Hil82}), this results in the
gravitational potential energy of the water per unit mass,
$\nrgptnsoispcgrv$, and per unit volume, $\nrgptnsoigrvvlm$.
Dividing (\ref{eqn:nrg_grv_Hil82}) by $\mss$ and by $\vlm$,
respectively, we obtain
\begin{eqnarray}
% Hil82 p. 68 (5.9)
\label{eqn:nrg_ptn_grv_mss_dfn}
\nrgptnsoispcgrv & = & \grv \hgt \\
% Hil82 p. 68 (5.10)
\label{eqn:nrg_ptn_grv_vlm_dfn}
\nrgptnsoigrvvlm & = & \dnslqd \grv \hgt
\end{eqnarray}
At STP, $\dnslqd \approx 1000$\,\kgxmC, so that 
$\nrgptnsoigrvvlm \approx 1000 \nrgptnsoispcgrv$.

Work by \cite{ClH78} and \cite{CHC84} provided widely used approximate
empirical relationships between the soil matric potential, water
content, and soil texture.
First, the saturated volumetric water content $\vwcsat$ depends weakly
on the mass fraction of sand in the soil  
\begin{eqnarray}
% Bon96 p. 98
\vwcsat & \approx & 0.489 - 0.126 \mssfrcsnd
\label{eqn:vwc_sat_Bon96}
\end{eqnarray}
Knowledge of $\vwcsat$ is important since, under a simplifying
assumption, be used to obtained the bulk density of the soil.
If the process of soil saturation is assumed is due to filling of
pores (rather than soil swelling), then $\vwcsat$ equals the volume
concentration of air pores which have been displaced.
Assuming soils reach saturation once all air pores have been displaced  
then 
\begin{eqnarray}
% Bon96 p. 98
\vwcsat & = & \vlmair \bigg|_{\vwc = 0}
\label{eqn:vwc_sat_air}
\end{eqnarray}
where LHS and RHS are evaluated at complete saturation and at oven-dry 
conditions, respectively.
For values of $\mssfrcsnd$ in $[0.0,1.0]$ (\ref{eqn:vwc_sat_Bon96})
predicts $\vlmair(\vwc = 0) \in [0.489,0.363]$.

For soils with a swelling function $\swlfnc$ equal to unity
(\ref{eqn:swl_fnc_dfn}), the volumetric water content at saturation  
and the (poreless) volume $\vlmprt$ occupied by soil particles are
simply related
\begin{eqnarray}
\vwcsat + \vlmprt & = & 1 \nonumber \\
\vlmprt & = & 1 - \vwcsat
\label{eqn:vlm_soi_dfn}
\end{eqnarray}
The dry \trmdfn{bulk density} or dry density of the soil $\dnsprtdry$ is
this poreless soil volume $\vlmprt$ times the mean mass density
$\dnsprt$ of individual soil particles 
\begin{eqnarray}
% my own
\dnsprtdry & = & \dnsprt \vlmprt \nonumber \\
& = & \dnsprt (1 - \vwcsat)
\label{eqn:dns_dry_dfn}
\end{eqnarray}
The contribution of air has been neglected in (\ref{eqn:dns_dry_dfn})
since $\dnsatm \ll \dnsprt$. 
It follows from (\ref{eqn:vwc_sat_Bon96}) that (\ref{eqn:dns_dry_dfn})
is also an approximation which is best suited to sandy soils. 
In particular, (\ref{eqn:vwc_sat_Bon96}) and (\ref{eqn:dns_dry_dfn})
together imply that soil bulk density have no dependence on clay
or silt content.  

Since $\dnsprt$ measures the mass density of the particle solids only,
it does not depend on the structure or texture of the soil.
Therefore $\dnsprt$ does not depend on water content (except for
swelling soils), but is sensitive to particle coatings and aggregates,
such as organics.
Soil Organic Matter (\trmidx{SOM}) is less dense than most minerals,
so soils high in organic matter (such as those near the surface), 
are less dense than mineral-dominated soils of the same texture.
Values of $\dnsprt$ for important mineral types are presented in 
Table~\ref{tbl:mnr_cmp}.
A standard value of $\dnsprt = 2.65$\,\gxcmC\ (valid for pure quartz)  
is often employed when the specific soil mineralogy is unknown.
% http://weather.nmsu.edu/teaching_Material/soil252/Chapt5.htm
This value is especially appropriate in mineral dust source regions,
which are typically very low in SOM and high in quartz, feldspars, and
colloidal silicates.
Dust from volcanic and meteoric ash is less dense, closer to 
$\dnsprt = 2.65$\,\gxcmC. 

The relative saturation of the soil $\vwcrel$ is somewhat analogous to
relative humidity:
\begin{eqnarray}
\vwcrel & = & \vwc / \vwcsat 
\label{eqn:vwc_rel_dfn}
\end{eqnarray}
Thus $\vwcrel$ expresses the degree of saturation on a normalized
scale $0 \le \vwcrel \le 1$.

The soil matric potential at saturation, $\smpsat$, is, to a first
approximation, a function only of the mass fraction of the soil that
is sand, $\mssfrcsnd$
\begin{eqnarray}
% Bon96 p. 98
\smpsat & = & -10.0 \times 10^{1.88 - 0.000131 \mssfrcsnd }
\label{eqn:smp_sat_Bon96}
\end{eqnarray}
where $\smpsat$ is in mm\,\HdO.

Finally, \cite{ClH78} and \cite{CHC84}
showed that the measured soil matric potential fits a power law
dependence on the relative saturation
\begin{eqnarray}
% Bon96 p. 98
\label{eqn:smp_Bon96}
\smp & = & \smpsat \vwcrel^{-\smpxpn}
\qquad \mbox{where} \\
\label{eqn:smp_xpn_Bon96}
\smpxpn & = & 2.91 + 0.159 \times \mssfrccly 
\end{eqnarray}
where the dimensions of $\smp$ are mm\,\HdO.
Thus $\smpxpn$ depends upon both the sand mass fraction through
\vwcrel (\ref{eqn:vwc_rel_dfn}) and the clay mass fraction through
$\smpxpn$. 
Clearly the omission of any dependence on the silt mass fraction in
(\ref{eqn:smp_Bon96}) is an approximation.

\subsection[Capillary and Adsorptive Forces]{Capillary and Adsorptive Forces}\label{sxn:cpl}
The influence of moisture on the threshold friction velocity arises
from two forces caused by the presence of liquid water between grains
of soil.
The first force is the \trmdfn{capillary force}, which arises from the
\trmdfn{surface tension of water} and the size and geometry of grains
and pores.
As described in \S\ref{sxn:tns_sfc}, capillary forces cause a
pressure differential to develop between the curved miniscus of a
liquid phase water wedge and the air with which it is in equilbrium. 
The second force is the \trmdfn{adsorptive force}, which arises from
the attraction of polarized water molecules to charged surfaces of
soil grains.
Adsorptive forces cause liquid water (and water vapor) to coat the
surfaces of soil grains.

The combined forces of capillarity and adsorption are called the 
\trmdfn{matric suction} or \trmdfn{matric potential}. 
The soil matric potential $\smp$ is usually expressed as the
potential energy per unit volume of soil in \jxkg.
However, the dimensions and definition of $\smp$ are somewhat
confusing. 
For example, (\ref{eqn:smp_Bon96}) below is an expression for $\smp$
in mm\,\HdO. 
Thus we describe how to convert between various descriptions and units
of the energy levels of soil water. 

\subsection[Empirical Adjustments to Threshold Speed]{Empirical Adjustments to $\wndfrcthr$}\label{sxn:Bel64}
Based on the above discussion, it is clear that the friction threshold
of a soil is sensitive to the water content of the soil, i.e., 
$\wndfrcthr = \wndfrcthr(\gwc)$.
Generally, the influence of moisture on $\wndfrcthr$ is described by
the threshold inhibition function $\frcthrncrwtr$, defined as the ratio
between the wet and dry friction thresholds, $\wndfrcthrwet$ and
$\wndfrcthrdry$, respectively 
\begin{eqnarray}
\frcthrncrwtr (\vwc) & = & \frac{\wndfrcthr(\vwc)}{\wndfrcthr(\vwc = 0) }
= \frac{\wndfrcthrwet}{\wndfrcthrdry }
\label{eqn:frc_thr_wet_dfn}
\end{eqnarray}
Thus $\frcthrncrwtr > 1$.
In practice, the functional dependence on $\vwc$ may be replaced by
any other measure of water content, e.g., $\gwc$. 
In addition to water content, $\frcthrncrwtr$ can depend on soil
texture (i.e., size), salt content, and time since precipitation.
For the time being we neglect these dependencies and refer the
interested reader to the discussions in \cite{Pye87} and
\cite{Gil88}. 

A number of investigators have created simple parameterizations which  
account for the increase of $\wndfrcthr$ with $\vwc$
\cite[][]{Bel64,Pye87,Gil88,SeF95,SRL96,FMB99}.
\cite{Bel64} measured $\wndfrcthr$ for various soil moisture contents.  
His results fit the logarithmic parameterization
\begin{eqnarray}
\frcthrwet(\vwc) & = & 1.8 + 0.6 \log_{10} \vwcpct
\label{eqn:frc_thr_wet_Bel64}
\end{eqnarray}
where $\vwcpct$ is the percent volumetric water content
% fxm: Units of this equation are suspect
Unlike most authors, \cite{Bel64} measured the threshold friction
velocity required to develop a sustained saltation, rather than
the fluid threshold friction velocity.   
As a result, (\ref{eqn:frc_thr_wet_Bel64}) tends to predict higher
values of $\wndfrcthr$ than other parameterizations
\cite[][]{SRL96,FMB99}. 

The wind tunnel experiments of \cite{SRL96} fitted the exponential
relationship
\begin{eqnarray}
\frcthrncrwtr(\vwc) & = & \me^{ 22.7 \vwc }
\label{eqn:frc_thr_wet_SRL96}
\end{eqnarray}
In practice (\ref{eqn:frc_thr_wet_SRL96}) predicts that $\frcthrncrwtr$
becomes so large for $\vwc > 0.04$\,\mCxmC\ that dust mobilization
effectively ceases. 

The differences between (\ref{eqn:frc_thr_wet_Bel64}) and
(\ref{eqn:frc_thr_wet_SRL96}) are significant.
The logarithmic form of (\ref{eqn:frc_thr_wet_Bel64}) curves concave
downwards starting from $\frcthrncrwtr(0.01) = 1.2$.
The exponential form of (\ref{eqn:frc_thr_wet_SRL96}) curves concave
upwards starting from $\frcthrncrwtr(0) = 1$.

\cite{Gil88} studied the dependence of $\wndfrcthr$ on soil type and
condition. 
He characterized the \trmidx{modulus of rupture} of the soil.

\cite{SeF95} studied the behavior of erosive thresholds on soil
moisture in carefully controlled \trmidx{wind tunnel} experiments.
Two soil moisture reading were made for each of five different
soil textures considered. 
The first is the water holding content, $\gwcsat$, which is simply the 
gravimetric water content at saturation. 
\cite{SeF95} found that $\gwcsat$ is tightly correlated with
soil clay fraction $\mssfrccly$
\begin{eqnarray}
% SeF95 p. 306 Fig. 1
\gwcsat & = & 0.433 \mssfrccly
\label{eqn:gwc_sat_dfn}
\end{eqnarray}
This relationship (\ref{eqn:gwc_sat_dfn}) is in contrast to previous
studies which found $\vwcsat$ depends most on sand fraction
$\mssfrcsnd$ (\ref{eqn:vwc_sat_Bon96}) \cite[][]{ClH78,CHC84,Bon96}.  
Second is the maximum gravimetric water content which permits
erosion $\gwcthrmax$, i.e., wind erosion ceases for $\gwc >
\gwcthrmax$.\footnote{The terminology \trmidx{threshold gravimetric 
water content} is somewhat ambiguous.
The quantity $\gwcthrmax$ defined by \cite{SeF95} is not to be
confused with the related $\gwcthrmin$ defined by \cite{FMB99}. 
The maximum gravimetric water content $\gwc$ which permits erosion
is $\gwcthrmax$, i.e., erosion ceases for $\gwc > \gwcthrmax$.
The maximum $\gwc$ which does \textit{not} impede erosion. 
is $\gwcthrmin$, i.e., soil water reduces erosion for 
$\gwc > \gwcthrmin$.} % end footnote
The experimental design allowed separate retrieval of $\gwcthrmax$ in
both abrading and non-abrading environments where the difference was
the presence or absence, respectively, of upstream saltation (which
assists triggering local saltation). 
Regressions of their measurement results showed high-correlation when  
defined in terms the non-dimensional \trmdfn{equivalent threshold
moisture} $\gwceqvthr$
\begin{eqnarray}
\gwceqvthr \equiv \gwcthrmax/\gwcsat
\label{eqn:eqv_thr_mst_dfn}
\end{eqnarray}
Clearly $\gwceqvthr < 1$ for standard soils.

It may be helpful to recap the various moisture thresholds related to
wind erosion.
The soil water content $\gwc$ falls into four possible ranges
described in Table~\ref{tbl:gwc}.
\begin{table}
\begin{minipage}{\hsize}
\begin{center}
\caption[Moisture Constraints]{\textbf{Regimes of Soil Moisture Inhibition of Dust Emissions}
\label{tbl:gwc}}
\vspace{\cpthdrhlnskp}
\begin{tabular}{ >{$}l<{$} l}
\hline \rule{0.0ex}{\hlntblhdrskp}% 
\mbox{Regime} & Constraint \\[0.0ex]
\hline \rule{0.0ex}{\hlntblntrskp}%
0 < \gwc < \gwcthrmin < \gwcthrmax < \gwcsat < 1 &
\mbox{Emissions occur without constraint, moisture is adsorptive} \\[0.5ex]
0 < \gwcthrmin < \gwc < \gwcthrmax < \gwcsat < 1 &
\mbox{Emissions occur with constraint, some moisture is capillary} \\[0.5ex]
0 < \gwcthrmin < \gwcthrmax < \gwc < \gwcsat < 1 &
\mbox{No emissions occur, moisture inhibition too high} \\[0.5ex]
0 < \gwcthrmin < \gwcthrmax < \gwcsat < \gwc < 1 &
\mbox{No emissions occur, soil is waterlogged} \\[0.5ex]
\hline
\end{tabular}
\end{center}
\end{minipage}
\end{table}

The following relations between $\wndfrcthr$, and $\gwceqvthr$ were
obtained:   
\begin{eqnarray}
\wndfrcthr(\gwceqvthr) & = & \left\{
\label{eqn:frc_thr_wet_SeF95}
\begin{array}{>{\displaystyle}l<{} @{\quad:\quad} l }
% SeF95 p. 3307 (4) Non-abraded (non-saltating conditions)
\label{eqn:frc_thr_wet_non_slt_SeF95}
0.305 + 0.022 \gwceqvthr + 0.506 \gwceqvthr^{2} & \wndfrc < \wndfrcthr \quad\mbox{(non-saltating)} \\
% SeF95 p. 3307 (5) Abraded (saltating conditions)
\label{eqn:frc_thr_wet_slt_SeF95}
0.205 + 0.182 \gwceqvthr + 0.375 \gwceqvthr^{2} & \wndfrc \ge \wndfrcthr \quad\mbox{(saltating)}
\end{array} \right.
\end{eqnarray}
The first of the two domains of (\ref{eqn:frc_thr_wet_SeF95}) 
applies to the non-saltating (non-abraded) conditions which exist
prior to the development of a mature saltation layer, or upwind of the
layer. 
The second domain applies to the saltating friction velocity which is
the parameter of merit in saltating conditions.
The threshold friction velocity is seen to increase approximately 20\%
per 10\% increase in $\gwceqvthr$ in both cases.

\subsection[McKenna-Neuman's Theory]{McKenna-Neuman's Theory}\label{sxn:McN89}
\cite{McN89} developed a theoretical expression for 
the increase of $\wndfrcthr$ due to soil moisture valid for sandy
soils. 
Adsorptive forces are thought to be insignificant in soils containing
only coarse sand particles ($\dmt > 100$\,\um) \cite[]{Hil82}.
This is because the molecular forces which lead to adsorption are
largely caused by clay sized particles.

\cite{McN89} assume that the only remaining force able to cause
significant interparticle cohesion in sandy texture soils is the
capillary force $\frccpl$. 
Assuming a disymmetrical conical geometry represented the grain-pore
relationship, they found that $\frccpl$ could be expressed solely in
terms of the surface tension of liquid water $\sfctnswtrlqd$, a factor
$\GGG$ determined by the grain-pore geometry, and the pressure deficit
$\prsxcs$ as
\begin{eqnarray}
% FMB99 p. 151 (3)
\frccpl & = & \frac{\mpi \sfctnswtrlqd^{2} \GGG}{\prsxcs }
\label{eqn:frc_cpl_McN89}
\end{eqnarray}
The pressure deficit $\prsxcs$ is the only factor in
(\ref{eqn:frc_cpl_McN89}) which directly depends on the moisture
content. 
Note that ss $\prsxcs \rightarrow 0$, $\frccpl$ becomes ill-defined. 
Thus (\ref{eqn:frc_cpl_McN89}) is not valid for planar water surfaces
where the capillary forces become very weak.

The work of \cite{McN89} showed that, for coarse sandy soils,
$\frcthrncrwtr$ depended on the square root of the capillary force.
Two geometrically distinct arrangements of particles and pores were
analyzed. 
The first, corresponding to a particle resting angle of $\beta =
30\dgr$, is called an \trmdfn{open packed system}.
The second, corresponding to a particle resting angle of $\beta =
45\dgr$, is called an \trmdfn{close packed system}.
For each of these systems, they found 
$\frcthrncrwtr \propto \sqrt{\frccpl}$.
As indicated in (\ref{eqn:frc_cpl_McN89}), $\frccpl$ depends on the
grain geometry through both $\GGG$ and $\prsxcs$.  
The resulting expressions for $\frcthrncrwtr$ are
\begin{eqnarray}
% FMB99 p. 151 (4)
\frcthrncrwtr & = & \left\{
\begin{array}{>{\displaystyle}r<{} @{\quad:\quad}l}
\left( 1 + \frac{6 \frccpl \sin ( 2 \beta ) \cos ( 2 \beta )}{
\mpi \dmt^{3} ( \dnsprt - \dnsatm ) \grv \sin ( \beta ) } \right)^{1/2}
& \mbox{Open packed system} \\
\left( 1 + \frac{6 \frccpl \sin ( 2 \beta ) ( 1 + 2 \cos \beta )}{
\mpi \dmt^{3} ( \dnsprt - \dnsatm ) \grv \sin ( \beta ) } \right)^{1/2}
 & \mbox{Close packed system}
\end{array} \right.
\label{eqn:frc_thr_wet_McN89}
\end{eqnarray}

Wind tunnel data for three sizes of sand ($\dmt = 190$, $270$, and
$510$\,\um) fit (\ref{eqn:frc_thr_wet_McN89}) reasonably well over the
experimental range of moisture contents tested, $0 < \vwc < 0.02$.
The predictive skill of this theory in the absence of any treatment of
adsorptive forces substantiates the hypothesis that adsorptive forces
are insignificant at increasing $\wndfrcthr$ for large particles.

The dominance of capillary over adhesive forces in coarse sand soil
supports a hypothesis about the partitioning of water between
capillary pores and surface coatings.
If the interparticle forces due to adsorption are negligible for
coarse sandy soil, then it is reasonable to assume that the fraction
of liquid water which is adsorbed to surfaces is negligible compared
to the fraction bound into capillary wedges.
Thus \cite{McN89} assume that virtually all of the soil water in
coarse soils is found in capillary wedges and pores, and very little
is found in surface coatings.
Since the interparticle cohesive forces for coarse sandy soils are
dominated by $\frccpl$ (\ref{eqn:frc_cpl_McN89}), the soil matric
potential, which is the sum of the gravitational and pressure
potentials (\ref{eqn:nrg_ptn_ttl_dfn}), and must be closely related to  
$\frccpl$ as well.
In fact, \cite{McN89} assumed that the soil matric potential is
equivalent to the capillary potential, i.e., 
\begin{eqnarray}
\smp & \approx & \frccpl
\label{eqn:smp_McN89}
\end{eqnarray}
For clayey soils a significant fraction of the water is bound into
adsorptive coatings, so that a simple relationship between $\smp$ and 
$\frccpl$ (\ref{eqn:smp_McN89}) is not justified.
However, a similar relationship may by posited and its results tested
against experiments, as is done by \cite{FMB99}.

\subsection[F\'{e}can's Theory]{F\'{e}can's Theory}\label{sxn:FMB99}
\cite{FMB99} developed a semi-empirical parameterization for 
$\frcthrncrwtr(\gwc)$ suitable for use in large scale atmospheric
models. 
Their parameterization extends the theory of \cite{McN89} for coarse
sandy soils to realistic soils containing arbitrary amounts of clay
and silt.
The major limitation of \cite{McN89} is its neglect of these smaller
particles.
As already discussed, the stronger electrostatic surface forces of
clays, together with their higher surface area to volume ratio, causes
adsorptive forces to become significant in the presence of clays.
As noted by \cite{Hil82}, the adsorptive and capillary forces are
difficult to disentangle when small particles are present because the
distinction between the coatings and wedges at the contact points is
ill-defined. 
In fact, the adsorptive coatings and capillary wedges are essentially
distinct thermodynamic phases of water whose equilibrium is very
difficult to predict for an arbitrary geometry.

Thus, in the absence of a theoretical description of the simultaneous
effects of both capillary and adsorptive forces, \cite{FMB99} chose to
to fit an assumed functional form to experimental data.
First they noted that the combined cohesive forces due to water
(capillary and absortive) are what determines the soil matric
potential of soil with arbitrary clay and silt content
\begin{eqnarray}
\smp & \approx & \frccpl + \frcads
\label{eqn:smp_FMB99}
\end{eqnarray}
Unlike (\ref{eqn:smp_McN89}), (\ref{eqn:smp_FMB99}) applies to any
soil type, include clayey soils.
According to the work of \cite{ClH78} and \cite{CHC84} presented
earlier (\ref{eqn:vwc_sat_Bon96})--(\ref{eqn:smp_xpn_Bon96}), $\smp$
may be empirically expressed in terms of coefficients which explicitly
depend only on the soil sand and clay contents.

The total water content of an arbitrary soil may be considered as the
sum of two components, water at contact points which contributes to
capillary forces, $\vwccpl$, and water in surface coatings which
contributes to adhesive forces, $\vwcads$,
\begin{eqnarray}
% FMB99 p. 152 (9)
\vwc & = & \vwccpl + \vwcads
\label{eqn:vwc_FMB99}
\end{eqnarray}

As noted earlier, $\frcthrncrwtr$ is proportional to the square root of
the capillary force (\ref{eqn:frc_thr_wet_McN89}) for coarse
sand-textured soils. 

\cite{FMB99} parameterized $\frcthrncrwtr$ in terms of gravimetric water
content $\gwc$ rather then volumetric water content $\vwc$ because
most of the field measurements were reported in terms of $\gwc$.
\cite{FMB99} posit that $\frcthrncrwtr(\gwc)$ is unity for water contents
less than the \trmdfn{threshold gravimetric water content} $\gwcthrmin$.
For $\gwc < \gwcthrmin$, the water is mostly distributed in adsorptive
coatings which do not affect $\wndfrcthr$.
When $\gwc = \gwcthrmin$, the adsorptive capacity of the soil has been
reached and additional water increases the capillary forces in the
soil, which do increase $\wndfrcthr$.
Thus, for soils wetter than $\gwcthrmin$, inhibition of particle
deflation through increasing $\wndfrcthr$ is to be expected.
\cite{FMB99} show that most variation in $\gwcthrmin$ is an explicit
function of the soil clay content. 
Using multiple datasets \cite[][, and others]{Bel64,McN89} spanning
the domain $0.0 < \mssfrccly < 0.5$, they found
\begin{eqnarray}
% FMB99 p. 154 (14)
\gwcthrminpct & = & 0.17 \mssfrcclypct + 0.0014 \times (\mssfrcclypct)^{2}
\nonumber \\
\gwcthrmin & = & 0.17 \mssfrccly + 0.14 \times \mssfrccly^{2}
\label{eqn:gwc_thr_FMB99}
\end{eqnarray}
Note that all gravimetric water contents and soil textures in
\cite{FMB99} are assumed to be in percent.
According to (\ref{eqn:gwc_thr_FMB99}), 
as $\mssfrccly$ increases from $0.0$ to $1.0$, 
$\gwcthrmin$ increases quadratically from $0.0$ to $0.31$\,\kgxkg.
Typical Saharan desert clay contents of $0.1 < \mssfrccly < 0.2$ 
lead to $0.018 < \gwcthrmin < 0.04$\,\kgxkg.

\cite{FMB99} parameterized the observed relationship between
$\frcthrncrwtr$, $\gwc$, and $\gwcthrmin$ from multiple datasets.
\begin{eqnarray}
% FMB99 p. 155 (15)
\frcthrncrwtr & = & \left\{
\begin{array}{r@{\quad:\quad}l}
1 & \gwc \le \gwcthrmin \\
\sqrt{ 1 + 1.21 ( \gwcpct - \gwcthrminpct )^{0.68} } & \gwc > \gwcthrmin \\
\sqrt{ 1 + 1.21 [ 100 ( \gwc - \gwcthrmin ) ]^{0.68} } & \gwc > \gwcthrmin
\end{array} \right.
\label{eqn:frc_thr_wet_FMB99}
\end{eqnarray}
By construction, moisture does not affect $\wndfrcthr$ until $\gwc >
\gwcthrmin$, after which increased $\gwc$ quickly quenches dust
production by increasing $\wndfrcthr$.
Note that (\ref{eqn:frc_thr_wet_FMB99}) maintains the quadratic
dependence of $\frcthrncrwtr$ on $\vwc$ explained by \cite{McN89}
(\ref{eqn:frc_thr_wet_McN89}), but also allows for the influence of
clay on soil.
These results may be expressed in terms of $\vwc$ by employing
(\ref{eqn:gwc_vwc}).

\section[Geomorphology]{Geomorphology}\label{sxn:gmr_ltr}
\cite{SGS02}
\cite{TSG02}
\cite{BOP05} use spectral unmixing techniques to produce landform
distributions for North Africa. 
\cite{THK02}
\cite{ZNT03}
\cite{GNH01} describe three phases of saltation activity related to 
crusting on the saline playa of Owens Lake.
\cite{Gil96} reviews evidence for anthropogenic and
dessication-induced dust emission from playas and dry lakes
worldwide. 
\cite{EJK04} describe the provenance of dust from the 
\trmidx{\Bodele\ Depression}. 
\cite{KoF08} investigate the sensitivity of dust erodibility to
geomorphological parameters in a CTM evaluated with MISR optical
depths. 

\section[Dust Source Regions]{Dust Source Regions}\label{sxn:rgn}
\cite{PGT02} characterize dust source regions on the basis of
satellite retrieved indices from the Total Ozone Mapping Spectrometer
(\trmidx{TOMS}).
\cite{WTM03} use TOMS and in~situ observations to characterize source
regions.
\cite{SZL01} use in~situ dust storm reports to analyze the climatology
and distribution of dust sources in the \trmidx{Gobi Desert},
\trmidx{Takla Makan Desert}, and other regions of \trmidx{China}.
Table~\ref{tbl:rgn} characterizes the most important dust source
regions. 
\cite{KoK04} use satellite measurements to show funneling effects of
topography appear to significantly enhance wind speed in the \Bodele\
Depression relative to surrounding regions.
\cite{ZeK05} analyze TOMS aerosol and satellite-derived climate
timeseries to infer the dominant erodibility mechanisms in source
regions. 
\cite{BGB04} and \cite{GBB04} characterize soil crusting and rainfall
infiltration in terms of near infrared soil spectral reflectance. 
\begin{table*}
\begin{minipage}{\hsize} % Minipage necessary for footnotes KoD95 p. 110 (4.10.4)
\renewcommand{\footnoterule}{\rule{\hsize}{0.0cm}\vspace{-0.0cm}} % KoD95 p. 111
\begin{center}
\caption[Erodibility Responses of Major Dust Source Regions]{\textbf{Erodibility Responses of Major Dust Source Regions}% 
\footnote{Highly significant ($\cnfstt < 0.01$) cross-correlations
  $\crrcff$ between autoregression-corrected erodibility indicators
  (dust \AOD\ $\tau$) and climate constraints (precipitation
  $\flxpcp$, \NDVI\ $\NNN$, and wind speed $\wndspd$) from 1979--1994. 
  Lag in months of indicated cross-correlation is shown in
  parentheses.}%
\footnote{\emph{Sources:} Dust source regions identified by
  \cite{PGT02} and subsequent analyses of \cite{TBH02} data.}%
\label{tbl:rgn}}
\vspace{\cpthdrhlnskp}
\begin{tabular}{>{\raggedright}p{1.85in}<{} *{5}{>{\raggedright}p{0.6in}<{}} l } % KoD95 p. 94 describes '*' notation
\hline \rule{0.0ex}{\hlntblhdrskp}% 
Region & $\flxpcp, \tau$ & $\flxpcp, \NNN$ & $\NNN, \tau$ & $\flxpcp, \wndspd$ &
$\wndspd, \tau$ & Cat.\footnote{Erodibility Category Assigned} \\[0.0ex]
\hline \rule{0.0ex}{\hlntblntrskp}%
Eastern Sahel \newrightline 10\dgrn\mbox{--}15\dgrn, 10\dgrw\mbox{--}20\dgre & $-0.27(9)$ & $+0.33(1)$ & $-0.31(0)$ & & & I \\[0.5ex] % 
\Bodele\ Depression \newrightline 15\dgrn\mbox{--}20\dgrn, 10\dgre\mbox{--}20\dgre & $-0.28(9)$ & $+0.26(9)$ & $-0.31(0)$ & & & I \\[0.5ex] % 
Western US \newrightline 25\dgrn\mbox{--}35\dgrn, 110\dgrw\mbox{--}100\dgrw & $-0.22(0)$ & $+0.47(1)$ & $-0.35(0)$ & & & I \\[0.5ex] % 
Lake Eyre Basin \newrightline 30\dgrs\mbox{--}25\dgrs, 136\dgre\mbox{--}145\dgre & $-0.36(1)$ & $+0.61(1)$ & $-0.29(1)$ & & & I \\[0.5ex] % 
Botswana \newrightline 25\dgrs\mbox{--}20\dgrs, 20\dgre\mbox{--}30\dgre & $-0.39(1)$, $-0.23(0)$ & $+0.56(2)$, $+0.31(0)$ & $-0.28(9)$ & & & I \\[0.5ex] % 
Gobi Desert \newrightline 42.5\dgrn\mbox{--}45\dgrn, 105\dgre\mbox{--}110\dgre & & & $-0.28(2)$ & & & I \\[0.5ex] % 
China Loess Plateau \newrightline 32.5\dgrn\mbox{--}37.5\dgrn, 105\dgre\mbox{--}110\dgre & $-0.27(0)$ & & & & & II \\[0.5ex] % 
Great Salt Lake \newrightline 40\dgrn\mbox{--}42.5\dgrn, 115\dgrw\mbox{--}112.5\dgrw & $-0.37(0)$ & $-0.27(0)$ & & $+0.26(0)$ & & II \\[0.5ex] % 
Zone of Chotts \newrightline 32.5\dgrn\mbox{--}35\dgrn, 5\dgre\mbox{--}10\dgre & $+0.21(44)$ & $+0.42(26)$ & & $+0.26(0)$ & & III \\[0.5ex] % 
Tigris/Euphrates \newrightline 27.5\dgrn\mbox{--}32.5\dgrn, 45\dgre\mbox{--}57.5\dgre & $+0.21(14)$ & $-0.26(8)$ & & & & III \\[0.5ex] % 
Saudi Arabia \newrightline 20\dgrn\mbox{--}25\dgrn, 47.5\dgre\mbox{--}52.5\dgre & $+0.36(0)$ & $-0.27(0)$ & & & & IV \\[0.5ex] % 
Oman \newrightline 17.5\dgrn\mbox{--}20\dgrn, 52.5\mbox{--}57.5\dgre & $+0.40(0)$ & & & & & IV \\[0.5ex] % 
Tarim Basin \newrightline 35\dgrn\mbox{--}40\dgrn, 75\dgre\mbox{--}90\dgre & & & $+0.28(21)$ & & $+0.23(0)$, $-0.24(2)$ & IV \\[0.5ex] % 
Thar Desert \newrightline 25\dgrn\mbox{--}30\dgrn, 70\mbox{--}75\dgre & $+0.25(0)$, $-0.24(1)$, $-0.21(2)$ & $+0.57(1)$ & $-0.3(0)$, $-0.33(10)$ & $-0.35(0)$ & $+0.3(1)$ & I, IV \\[0.5ex] % 
\hline
\end{tabular}
\end{center}
\end{minipage}
\end{table*} % end tbl:rgn
While dust emissions often peak in playas and dried lake-beds,
they are strongly-structured, not uniformly distributed, throughout
these sources. 
% Paris: Richard Washington's talk
Emissions seem to peak at the windward edge of playas.

\subsection[\Bodele\ Depression]{\Bodele\ Depression}\label{sxn:bodele}
The \nmidx{\Bodele\ Depression} is presently Earth's strongest dust source.
Half of its emissions cross 15\dgrw\ and enter the Subtropical
Atlantic Region, where they contribute to the trans-Atlantic dust
plume seen clearly in satellite measurements
\cite[][]{HPS97,HBT97,PGT02}. 
Conversely, much of the \Bodele\ dust remains over North Africa,
where it contributes to the background dust resevoir.
Indirect evidence for this reservoir is seen in the insensitivity of 
trans-Atlantic dust events to the timing of wind events in African
source regions \cite[][]{CTH03}.

One research group which visited the region found that white
\trmidx{diatomite} from \trmidx{Paleolake Megachad} covers a large
fraction of the region \cite[][]{Gil05}. 
Diatomite is the ultimate source of much of the \Bodele's dust.
The dust appears to form by splintering and erosion of large diatomite
particles saltating in sand dunes, rather than by lofting of fine
diatomite powders seen in other regions of the \trmidx{Sahara}.

\cite{WTL06} summarize the physical, meteorological, geomorphological,
and anthropogenic reasons why the \Bodele\ Depression emits dust so
efficiently. 
The depression is situated between two mountain ranges, the Tibesti
and the Ennedi.
These topographic features funnel a low-level easterly jet over the
depression. 

\subsection[Takla Makan Desert]{Takla Makan Desert}\label{sxn:takla}
The \nmidx{Takla Makan Desert} in the \nmidx{Tarim Basin} is the most
active dust source in Asia \cite[][]{HPS97,HBT97,PGT02}. 

\subsection[Lake Eyre Basin]{Lake Eyre Basin}\label{sxn:ayre}
The Lake~Eyre Basin is the most important dust source in Australia.
It is an inland-drainage basin and vestigial wetland.

\chapter[Sea Salt]{Sea Salt}\label{sxn:ssl}
The natural aerosols with the greatest mass burden and optical
depth are mineral dust and \trmdfn{sea salt}.
As we have seen, the entrainment of mineral dust into the atmosphere
is a very complicated process.
Sea salt aerosol is also produced by Aeolian erosion but the source
material, the ocean, is much more homogeneous than land.
While a complete understanding of the generation of all sizes of sea
salt has not been achieved, the fundamental relationships between
wind speed and mass fluxes are known empirically and are less
controversial than dust fluxes.

Sea salt and desert dust mix more often than their names imply.
% Paris: Levin talk
\nmidx[Levin, Zev]{Zev Levin} finds that about 60\% of all coarse
particles 500\,m over the Mediterranean are internal mixtures of
dust and sea salt.

\section[Literature Review]{Literature Review}\label{sxn:ssl_ltr}
\cite{Mon71} has a very concise title.
\cite{MDS82} used closed tanks experiments to empirically determine
the spectral mass flux of sea salt generated by bubble-bursting.
\cite{MSD86} presented the first model with distinct parameterizations
for the direct and indirect formation mechanisms.
Testing of this model showed it overpredicted direct droplet
production at high wind speeds, a deficiency which has been addressed
by more recent studies.
\cite{SPC93} developed a generation function based on two-lognormal
modes.
\cite{GBB97} and \cite{GBP97} present a global model of the generation
and distribution of sea salt aerosol.
\cite{AEM95} reviewed and summarized extent production mechanisms.
\cite{And98} contrasted various spume productions parameterizations
with theory and modified the best of these into an improved
parameterization. 
\cite{VDB01} present a tri-modal lognormal generation function based
on open ocean, ship-borne observations.

\section[Sea Salt Generation]{Sea Salt Generation}\label{sxn:ssl_mbl}
Sea salt aerosol is generated by two fundamentally distinct
mechanisms \cite[e.g.,][]{MSD86}.
Wind drag in rough seas detaches droplets directly from the surface.
Surface tension prevents wind from directly separating droplets
smaller than $\dmtprt \sim 40$\,\um\ from waves.
These relatively large droplets are called \trmdfn{spume} and
the generation of spume by wind drag is known as the \trmdfn{direct
mechanism} of sea salt generation.
Spume generation occurs most readily at the crests of breaking waves 
where internal forces binding liquid together are overcome by the
wind drag acting on the cross-sectional area of the crest water.
Thus the direct mechanism is a source exactly at the ocean-atmosphere
interface that occurs only during strong wind events, say $\wndspd
\gtrsim 10$\,\mxs.

The \trmidx{indirect mechanism} of sea salt generation is mediated
by air bubbles.
Whitecaps form when ocean surface waves break and generate small scale
turbulent motions in the surface ocean. 
The white appearance is caused by air trapped in small bubbles on the
surface called \trmdfn{foam} or by bubbles entirely within the near
surface water. 
When bubbles buoyantly reach the ocean surface and burst their liquid
shells explode into \trmdfn{film drops}. 
Film drops are typically $0.1 < \dmt < 5$\,\um\ with a modal size of
about 1--2\,\um. 
During the subsequent collapse of the bubble, a plume of about 1--10
tiny drops known as \trmdfn{jet drops} may be released into the
atmosphere.
The surface tension of the bubble contributes to the generation of the
jet. 
Jet drops range in size from $3 < \dmtprt < 100$\,\um\ with a modal
size of about~10\,\um. 

A sea salt generation function which represents the indirect mechanism
must estimate the fraction of ocean surface which is foam-covered
under specified conditions.  
Whitecap coverage depends on wind speed, stability, temperature,
salinity, wind duration, fetch, and surfactant amount.

\cite{MDS82} performed tank experiments to deduce the spectral
mass flux of sea salt by the indirect mechanism (bubble-bursting).
Based on these experiments, \cite{MSD86} reported the expression for
$\flxnbrvrtssl$ the vertical number flux of sea salt particles
entering the atmosphere as a function of 10\,m wind speed $\wndrfr$
\begin{subequations}
% MSD82 MSD86 SPC93 p. 818 (8) VDB01 p. 20227 (2) fxm:reconcile GMS02 and jwl's versions
\label{eqn:flx_nbr_vrt_ssl_ndr_MSD86}
\begin{align}
\label{eqn:flx_nbr_vrt_ssl_ndr_mcr}
% U is in m s-1 but R is in um
\frac{\partial \flxnbrvrtssl}{\partial \rdsprt} \qquad [\mbox{\nbrxmSsum}] & =  
1.373 \wndrfr^{3.41} \rdsprt^{-3} 
(1 + 0.057 \rdsprt^{1.05}) 10^{1.19 \me^{-\BBB^{2}}} \\
\label{eqn:flx_nbr_vrt_ssl_ndr_SI}
% U is in m s-1 and R is in m
\frac{\partial \flxnbrvrtssl}{\partial \rdsprt} \qquad [\mbox{\nbrxmSsm}] & = 
1.373 \wndrfr^{3.41} \mbox{\ldots}
\end{align}
\end{subequations}
where $\BBB = (0.380 - \log \rdsprt)/0.650$.
Note that the particle radius $\rdsprt$ in
(\ref{eqn:flx_nbr_vrt_ssl_ndr_mcr}) is expressed in microns not
meters.  
The identical expression is re-written in terms of particle diameter
$\dmtprt$ in meters in (\ref{eqn:flx_nbr_vrt_ssl_ndr_SI}). 

\cite{MSD86} also parameterized the size-dependent vertical number flux
of sea salt aerosol due to the direct mechanism $\flxmssvrtssl$
\begin{subequations}
\label{eqn:flx_nbr_vrt_ssl_drc_MSD86}
\begin{align}
\label{eqn:flx_nbr_vrt_ssl_drc_MSD86_GMS02}
% MSD86 as quoted in GMS02 p. 2 (2)
\frac{\partial \flxnbrvrtssl}{\partial \rdsprt} \qquad [\mbox{\nbrxmSsum}] & =
8.60 \times 10^{-6} \me^{2.08 \wndrfr} \rdsprt^{-2} \\
\label{eqn:flx_nbr_vrt_ssl_drc_MSD86_SPC93}
% MSD86 as quoted in SPC93 p. 819 (9) (which I trust more than GMS02)
\frac{\partial \flxnbrvrtssl}{\partial \rdsprt} \qquad [\mbox{\nbrxmSsum}] & =
6.45 \times 10^{-4} \me^{2.08 \wndrfr} \rdsprt^{-3} \me^{-\DDD^{2}}
\end{align}
\end{subequations}
where $\DDD = 2.18 (1.88 - \ln \rdsprt)$.
The version of \cite{MSD86} shown in \cite{SPC93} is shown in
(\ref{eqn:flx_nbr_vrt_ssl_drc_MSD86_SPC93}). 
The version of \cite{MSD86} shown in \cite{GMS02} is shown in
(\ref{eqn:flx_nbr_vrt_ssl_drc_MSD86_GMS02}). 
As mentioned above, (\ref{eqn:flx_nbr_vrt_ssl_drc_MSD86})
overestimates spume production \cite[][]{And98} and should not be used.

\cite{SPC93} developed a more accurate (and complicated)
parameterization of sea salt formation based on measurements taken on
the coast of South Uist Island in the Outer Hebrides, about 100~km
in the North Sea northwest of the Scottish mainland.
The measurements comprehensively sampled aerosols in wind speeds of
$\wndspd \in [0.0, 34.0]$\,\mxs\ at a 
It was found that the number fluxes at a height of about 14~m over the
surface were comprised of two log-normal distributions whose mean and
geometric standard deviation were insensitive to wind speed.
The measurements were made in a range of environmental conditions
and the parameterization (\ref{eqn:flx_nbr_vrt_ssl_SPC93}) was
normalized to predict fluxes of particles whose radii are in
equilibrium with a relative humidity of 80\%, $\rdsght$.
\begin{eqnarray}
% SPC93 p. 816 (6), VDB01 p. 20227 (3)
\frac{\partial \flxnbrvrtssl}{\partial \rdsght} \qquad [\mbox{\nbrxmSsum}] & = &
\sum_{\iii = 1}^{2} \AAA_{\iii} 
\exp \left[ -\fff_{\iii} \left( \ln \frac{\rdsght}{\rds_{\iii}}
\right)^{2} \right]
\label{eqn:flx_nbr_vrt_ssl_SPC93}
\end{eqnarray}
where the empirically determined parameters $\fff_{1}$, $\fff_{2}$, 
$\rds_{1}$, and $\rds_{2}$ are $3.1$, $3.3$, $2.1$\,\um, and
$9.2$\,\um, respectively.
The $\fff_{1}$ and $\rds_{\iii}$ parameters do not depend on
wind speed.
These field measurements could not distinguish between differing
production mechanisms, so (\ref{eqn:flx_nbr_vrt_ssl_SPC93})
implicitly accounts for film, jet, and spume mechanisms.

The relative contribution of the two modes, represented by the
$\AAA_{\iii}$ parameters, strongly depends on wind speed.
\begin{subequations}
% SPC93 p. 816 (7), VDB01 p. 20227 (4)
\label{eqn:cff_A_SPC93}
\begin{align}
\label{eqn:cff_A1_SPC93}
\AAA_{1} & = \exp( 0.0676\wndspd_{14} + 2.43 ) \\
\label{eqn:cff_A2_SPC93}
\AAA_{2} & = \exp( 0.959\sqrt{\wndspd_{14}} - 1.476 )
\end{align}
\end{subequations}
These expressions for $\AAA_{\iii}$ were developed for use with
wind speeds evaluated at 14\,m, $\wndspd_{14}$.
They may be re-expressed in terms of the more standard 10\,m wind
speed $\wndten$ by using boundary layer theory.
Neglecting stability corrections yields \cite[e.g.,][]{And98}
\begin{eqnarray}
% And98 p. 2183 (A4), (A5)
\wndspd_{14} & = & \wndten + \frac{\wndfrc}{\vonkrmcst} 
\ln \left( \frac{14}{10} \right) \nonumber \\
& = & \wndten \left[ 1 + \frac{\sqrt{\xchcffmmnntr}}{\vonkrmcst}
\ln \left( \frac{14}{10} \right) \right]
\end{eqnarray}
where $\xchcffmmnntr$ is the 
\trmidx{neutral exchange coefficient for momentum} at 10\,m
(\ref{eqn:cff_xch_mmn_ntr_dfn}).

\cite{And98} presents a thorough comparison of existing spume
production parameterizations \cite[][]{MSD86,SPC93} and contrasts
these to theoretical models.
He identifies important characteristics which accurate
parameterizations should contain, and then modifies the spume 
generation parameterization of \cite{SPC93}
(\ref{eqn:flx_nbr_vrt_ssl_SPC93}) to reflect this.

\cite{And98} argues convincingly that
(\ref{eqn:flx_nbr_vrt_ssl_SPC93}) underestimates the number of
both film drops and spume drops for two reasons.
The measurements resulting in (\ref{eqn:flx_nbr_vrt_ssl_SPC93}) were
probably biased by a number of factors \cite[]{And98}.
First, the measurements were taken from a 10\,m tower located on the
high-water mark of a gently sloping beach. 
During high tide the water reached the foot of the tower, but during
low tide the water was approximately 300\,m away.
While the distance from the instruments to the water isolated the
measurements from the immediate surf zone, it also meant that
particles were subject to several seconds of atmospheric transport 
before reaching the tower.
During this transport, evaporation and gravitational sedimentation 
remove particles from the marine air and this, argues
\citeauthor{And98}, causes (\ref{eqn:flx_nbr_vrt_ssl_SPC93}) to
underestimate number fluxes by a factor of about 3.5, an argument not
disputed by \citeauthor{SPC93}. 

Therefore \cite{And98} recommend using
(\ref{eqn:flx_nbr_vrt_ssl_SPC93}) for $\rdsght \in [1,10]$\,\um 
($\rdsnot \in [2,21]$\,\um), but multiplying it by 3.5.
Presented in terms of radius at point of droplet formation $\rdsnot$,
the spectral vertical number flux of the modified \citeauthor{SPC93}
generation function is
\begin{eqnarray}
% And98 p. 2181 (3.8)
\frac{\partial \flxnbrvrtssl}{\partial \rdsnot} \qquad [\mbox{\nbrxmSsum}] & = &
3.5 \frac{\partial \flxnbrvrtssl}{\partial \rdsght}
\frac{\dfr \rdsght}{\dfr \rdsnot}
\label{eqn:flx_nbr_vrt_ssl_And98}
\end{eqnarray}
The first term on the RHS is evaluated using \cite{SPC93}
(\ref{eqn:flx_nbr_vrt_ssl_SPC93}) for $\rdsght \in [1,10]$, 
and using simple power laws described below
(\ref{eqn:flx_nbr_vrt_ssl_And98_2}) for large spume. 
The relation between sea spray droplet size at formation, $\rdsnot$,
and the droplet size at relative humidity of 80\%, $\rdsght$, is 
determined by the hygroscopic growth properties of sea salt.
\cite{SPC93} adopted an equilibrium relative humidity $\RH = 98.3$\% for 
the near surface ocean.
% LaP81 p. 328 CCM:dom/flxoce()
This is close to the $\RH = 98$\% used in \cite{LaP81}. 
Assuming $\rdsnot$ is in equilibrium with $\RH = 98.3$\%, \cite{And98}
suggest 
\begin{subequations}
% And98 p. 2180 (3.4)
\label{eqn:rds_ght_dfn}
\begin{align}
\rdsght & = 0.518 \rdsnot^{0.976} \\
\rdsnot & = 1.963 \rdsght^{1.0246}
\end{align}
\end{subequations}
Taking the derivative of (\ref{eqn:rds_ght_dfn}) we obtain
\begin{subequations}
% And98 p. 2180 (3.7)
\label{eqn:dfr_rds_ght_dfn}
\begin{align}
\frac{\dfr \rdsght}{\dfr \rdsnot} & = 0.506 \rdsnot^{-0.024} \\
\frac{\dfr \rdsnot}{\dfr \rdsght} & = 2.011 \rdsght^{0.0246}
\end{align}
\end{subequations}
which may be used in (\ref{eqn:flx_nbr_vrt_ssl_And98}).

For larger spume droplets, \cite{And98} parameterizes droplet
concentration data measured within 20\,\cm\ of the ocean surface.
The parameterization simply scales the number flux with a power 
of the droplet radius, and there are three different regimes
\begin{eqnarray}
% And98 p. 2180 (3.5)
\frac{\partial \flxnbrvrtssl}{\partial \rdsght} & = & \left\{
\begin{array}{ >{\displaystyle}l<{} @{\quad:\quad} r }
\CCC_{1} \wndten \rdsght^{-1} & \rdsght \in [10,37.5]\,\mbox{\um} \\
\multicolumn{2}{c}{} \\[-0.5ex]
\CCC_{2} \wndten \rdsght^{-2.8} & \rdsght \in [37.5,100]\,\mbox{\um} \\
\multicolumn{2}{c}{} \\[-0.5ex]
\CCC_{3} \wndten \rdsght^{-8.0} & \rdsght \in [100.0,250]\,\mbox{\um}
\end{array} \right.
\label{eqn:flx_nbr_vrt_ssl_And98_2}
\end{eqnarray}
The parameters $\CCC_{1}$--$\CCC_{3}$ depend only on wind speed.
They are evaluated by requiring the modified \citeauthor{SPC93} 
distribution (\ref{eqn:flx_nbr_vrt_ssl_And98}) be continuous with
(\ref{eqn:flx_nbr_vrt_ssl_And98_2}) at $\rdsght = 10$\,\um.

It may be more convenient to forecast $\flxnbrvrtssl(\rdsnot)$
directly rather than $\flxnbrvrtssl(\rdsght)$.
Substituting (\ref{eqn:rds_ght_dfn}) and (\ref{eqn:dfr_rds_ght_dfn})
into (\ref{eqn:flx_nbr_vrt_ssl_And98_2}) we obtain $\flxnbrvrtssl$
in terms of $\rdsnot$, and remove the $\RH = 80\%$ assumption
\begin{eqnarray}
% And98 p. 2180 (3.5)
\frac{\partial \flxnbrvrtssl}{\partial \rdsnot} & = &
3.5 \frac{\partial \flxnbrvrtssl}{\partial \rdsght}
\frac{\dfr \rdsght}{\dfr \rdsnot} \nonumber \\
& = & \left\{
\begin{array}{ >{\displaystyle}l<{} @{\quad:\quad} r }
3.5 \CCC_{1} \wndten (1.931 \rdsnot^{-0.976}) (0.506 \rdsnot^{-0.024}) & \rdsnot \in [20.8,80.5]\,\mbox{\um} \\
\multicolumn{2}{c}{} \\[-0.5ex]
3.5 \CCC_{2} \wndten (6.308 \rdsnot^{-2.7328}) (0.506 \rdsnot^{-0.024}) & \rdsnot \in [80.5,219.8]\,\mbox{\um} \\
\multicolumn{2}{c}{} \\[-0.5ex]
3.5 \CCC_{3} \wndten (192.9 \rdsnot^{-7.808}) (0.506 \rdsnot^{-0.024}) & \rdsnot \in [219.8,562.1]\,\mbox{\um}
\end{array} \right. \nonumber \\
& = & \left\{
\begin{array}{ >{\displaystyle}l<{} @{\quad:\quad} r }
3.5 \CCC_{1} \wndten (0.977 \rdsnot^{-1}) & \rdsnot \in [20.8,80.5]\,\mbox{\um} \\
\multicolumn{2}{c}{} \\[-0.5ex]
3.5 \CCC_{2} \wndten (3.192 \rdsnot^{-2.757}) & \rdsnot \in [80.5,219.8]\,\mbox{\um} \\
\multicolumn{2}{c}{} \\[-0.5ex]
3.5 \CCC_{3} \wndten (97.61 \rdsnot^{-7.832}) & \rdsnot \in [219.8,562.1]\,\mbox{\um}
\end{array} \right. \nonumber \\
& = & \left\{
\begin{array}{ >{\displaystyle}l<{} @{\quad:\quad} r }
3.4195 \CCC_{1} \wndten \rdsnot^{-1} & \rdsnot \in [20.8,80.5]\,\mbox{\um} \\
\multicolumn{2}{c}{} \\[-0.5ex]
11.172 \CCC_{2} \wndten \rdsnot^{-2.757} & \rdsnot \in [80.5,219.8]\,\mbox{\um} \\
\multicolumn{2}{c}{} \\[-0.5ex]
341.64 \CCC_{3} \wndten \rdsnot^{-7.832} & \rdsnot \in [219.8,562.1]\,\mbox{\um}
\end{array} \right.
\label{eqn:flx_nbr_vrt_ssl_And98_3}
\end{eqnarray}

For studies which work in terms of particle diameter $\dmtnot$ rather
than radius, we may simplify (\ref{eqn:flx_nbr_vrt_ssl_SPC93}) and
(\ref{eqn:flx_nbr_vrt_ssl_And98_3}). 
Using $\dfr\rdsnot = \frac{1}{2}\dfr\dmtnot$, and rewriting
$\rdsnot^{\xxx}$ as $2^{-\xxx}\dmtnot^{\xxx}$ we obtain
\begin{eqnarray}
% And98 p. 2180 (3.5)
\frac{\partial \flxnbrvrtssl}{\partial \dmtnot} & = &
\frac{\partial \flxnbrvrtssl}{\partial \rdsnot} 
\frac{\dfr\rdsnot}{\dfr\dmtnot} 
= \frac{1}{2} \frac{\partial \flxnbrvrtssl}{\partial \rdsnot} \nonumber \\
& = & \left\{
\begin{array}{ >{\displaystyle}l<{} @{\quad:\quad} r }
& \dmtnot \in [,41.6]\,\mbox{\um} \\
\multicolumn{2}{c}{} \\[-0.5ex]
3.4195 \CCC_{1} \wndten \dmtnot^{-1} & \dmtnot \in [41.6,161.0]\,\mbox{\um} \\
\multicolumn{2}{c}{} \\[-0.5ex]
11.172 \CCC_{2} \wndten \dmtnot^{-2.757} & \dmtnot \in [161.0,439.6]\,\mbox{\um} \\
\multicolumn{2}{c}{} \\[-0.5ex]
341.64 \CCC_{3} \wndten \dmtnot^{-7.832} & \dmtnot \in [439.6,1124.2]\,\mbox{\um}
\end{array} \right.
\label{eqn:flx_nbr_vrt_ssl_And98_4}
\end{eqnarray}

According to \cite{GMS02}, the experiments of
(\ref{eqn:flx_nbr_vrt_ssl_ndr_MSD86}) were also conducted at 80\%
relative humidity. 
The particle size in these parameterizations is, therefore, the
ambient size of deliquescent sea salt aerosol in equilibrium with $\RH
= 80\%$ environment. 
Consequently the mass flux associated with
(\ref{eqn:flx_nbr_vrt_ssl_ndr_MSD86}) should represent particles
whose density is the correct average of the salts and the water.

\cite{VDB01} present a tri-modal lognormal generation function based
on open ocean, ship-borne observations.
These observations show that a fine particle mode ($\dmtnma =
0.4$\,\um) is present in the surf zone.
Table~\ref{tbl:VDB01} shows the parameters of the tri-modal log-normal 
number distribution generation function which gives the best fit to
the observations.
\begin{table}
\begin{minipage}{\hsize} % Minipage necessary for footnotes KoD95 p. 110 (4.10.4)
\renewcommand{\footnoterule}{\rule{\hsize}{0.0cm}\vspace{-0.0cm}} % KoD95 p. 111
\begin{center}
\caption[Tri-modal Sea Salt Parameters]{\textbf{Tri-modal Sea Salt Parameters}% 
\footnote{\emph{Source:} \cite{VDB01}, p.~20228.}%
\footnote{All parameters are for 80\% relative humidity.}%
\label{tbl:VDB01}}   
\vspace{\cpthdrhlnskp}
\begin{tabular}{*{4}{>{$}l<{$}}} % KoD95 p. 94 describes '*' notation
\hline \rule{0.0ex}{\hlntblhdrskp}% 
\iii & \cncttlidx & \rdsnmaidx\footnote{Modal radius $\rdsnmaidx$ for
$\RH = 80$\% and 10~m wind speed $\wndten$.} & \gsdi \\[0.0ex]
& \mbox{\nbrxcmC} & \mbox{\um} & \\[0.0ex]
\hline \rule{0.0ex}{\hlntblntrskp}%
1 & 10^{0.095 \wndten + 0.283} & 0.2 & 1.9 \\[0.5ex]
2 & 10^{0.0422 \wndten + 0.288} & 2.0 & 2.0 \\[0.5ex]
3 & 10^{0.069 \wndten - 3.5} & 12.0 & 3.0 \\[0.5ex]
\hline
\end{tabular}
\end{center}
\end{minipage}
\end{table}
For the three modes the median radii $\rdsnmaidx$\,\um\ at 80\%
relative humidity are $[0.2,2.0,12.0]$ and the geometric standard
deviations $\gsdi$ are 
$[1.9,2.0,3.0]$.
The total number flux $\flxnbrvrtssl$\,\nbrxmSsum\ of the tri-modal
distribution is 
\begin{eqnarray}
% VDB01 p. 20228 Tbl. 1
\frac{\dfr \flxnbrvrtssl}{\dfr \rdsght} & = & 
\sum_{\iii = 1}^{3} 
\frac{\cncttlidx}{\sqrt{2\mpi} \, \rdsght \lngsdi} \exp
\left[ -\frac{1}{2} \left( \frac{\ln(\rdsght/\rdsnmaidx)}{\lngsdi} \right)^{2} \right] 
\label{eqn:flx_nbr_vrt_ssl_VDB01}
\end{eqnarray}

\chapter{Dry Deposition}\label{sxn:dry_dps}
% SeP97 p. 972, Gio86, Gio88

The processes by which aerosols are removed from the atmosphere in the
absence of precipitation are collectively termed \trmdfn{dry
deposition}.
The dry deposition processes which remove aerosols from the atmosphere
to the surface (or canopy) include gravitational sedimentation, 
inertial impaction, and Brownian diffusion.
Dry deposition processes determine the atmospheric residence time of
large aerosols. 
Dry deposition processes are analogous to electrical resistances in
that each process is like a path to ground, and the efficiency of each
path may be construed as a \trmdfn{resistance} and included in
parallel or series with all other possible routes by which an aerosol
may reach the surface.

\section{Dry Deposition Literature}\label{sxn:ddp_ltr}

\cite{SHH78} is a review paper on all aspects of air-sea transfer,
including wet and dry deposition.
\cite{Seh80,Seh84} form a comprehensive review of particle and gaseous
dry deposition.
\cite{Wes89} presents a regional scale dry deposition parameterization
for gases.
\cite{SlS80} introduced a widely-used two layer model of particulate
deposition to water surfaces.
\cite{Wil82} presents a deposition model for water surfaces that
explicitly accounts for atmospheric stability and for hygroscopic
particle growth in the deposition (quasi-laminar) layer.
\cite{RVL93} perform an inter-model comparison of \cite{SlS80},
\cite{Wil82}, and their own model, a generalization of \cite{SlS80}
that accounts for arbitrary reference heights and stability.
\cite{PeE92} derive from first principles a complex model of particle
depostion to vegetated surfaces.
\cite{ZGP01} present a size-segregated particle dry deposition scheme 
with partially addresses underestimates of sum-micron aerosol
deposition velocities prevalent in models.
\cite{LPG95} examined dust deposition and mineralogy in Florida from
1992--1994. 
\cite{RBF01} and \cite{RBA01} derive and present a detailed dry
deposition model for particles and apply it to pesticide spray.
The articles are notable for their clear statement of physical and
mathematical assumptions. 

\section[Deposition Velocity]{Deposition Velocity}\label{sxn:vlc_dps}
We make the steady state assumption that the net flux $\flxdps$ of
a species from a reference height $\hgtrfr$ to a given surface is
related to the concentration of the species $\cnc(\hgtrfr)$ by a
parameter having units of velocity,
\begin{eqnarray}
\flxdps & = & - \vlcdps \cnc % SeP97 p. 958 (19.1)
\label{eqn:flx_dps_dfn}
\end{eqnarray}
where $\vlcdps$ (\mxs) is known as the \trmdfn{deposition velocity} of
the species. 

The physics of dry deposition are entirely contained in $\vlcdps$,
the rate of removal of the species due to all dry deposition
processes.  
Note that $\vlcdps$ is positive when the net flux is $\flxdps$
is from the atmosphere to the surface. 
In the following we assume that there is no upward flux of particles
due to dry deposition.
This is equivalent to stating that the \trmidx{sticking efficiency}
(\ref{eqn:stc_fsh_dfn}) is zero so that particles which strike the
surface remain there.
Since the upward or \trmdfn{resuspension flux} is zero, $\flxmssdps$
is both the net particle flux and the downward particle flux.
For gas phase species, $\cnc$ may be expressed as either a number
concentration in \nbrxmC\ or a mass concentration in \kgxmC.
The deposition flux $\flxdps$ is then either a number flux or a mass
flux of molucules in \nbrxmSs\ or \kgxmSs, respectively. 

For aerosol, the deposition velocity is a function of particle
diameter so that $\vlcdps = \vlcdps(\dmt)$.
The fundamental assumption which defines deposition velocities 
for aerosols is that turbulent and diffusive processes maintain
a constant proportion between the flux of particles of a
given size to the surface and the concentration of these particles at 
a reference height above the surface.
The fluxes of interest are the \trmdfn{spectral deposition number
flux} $\flxnbrdpsofdmt$ (\nbrxmSsm) and the \trmdfn{spectral
deposition mass flux} $\flxmssdpsofdmt$ (\kgxmSsm).
\begin{eqnarray}
\label{eqn:flx_nbr_dps_spc_dfn}
\flxnbrdpsofdmt & = & - \vlcdpsofdmt \dstnbrofdmt \\
\flxmssdpsofdmt & = & - \frac{\mpi \dnsprt}{6} \dmtpcp^{3} \vlcdpsofdmt \dstnbrofdmt
\nonumber \\
\label{eqn:flx_mss_dps_spc_dfn}
& = & - \vlcdpsofdmt \dstmssofdmt
\end{eqnarray}
The deposition velocity is the same for both types of spectral fluxes.
Note the similarity between these definitions and the definition of
precipitation fluxes in
(\ref{eqn:flx_mss_pcp_spc_dfn})--(\ref{eqn:flx_mss_pcp_vlm_dfn}).
Integrating (\ref{eqn:flx_nbr_dps_spc_dfn}) over particle size 
we obtain expression which defines the number-mean deposition
velocity $\vlcdpsnbravg$
\begin{eqnarray}
\int_{0}^{\infty} \flxnbrdpsofdmt \,\dfr\dmt & = & 
- \int_{0}^{\infty} \vlcdpsofdmt \dstnbrofdmt \,\dfr\dmt \nonumber \\
\flxnbrdps & = & 
- \frac{\cncttl}{\cncttl }
\int_{0}^{\infty} \vlcdpsofdmt \dstnbrofdmt \,\dfr\dmt \nonumber \\
& = & - \vlcdpsnbravg \cncttl
\label{eqn:flx_nbr_dps_dfn}
\end{eqnarray}
The same integration procedure applied to
(\ref{eqn:flx_mss_dps_spc_dfn}) leads to 
\begin{eqnarray}
\flxmssdps & = & - \vlcdpsmssavg \mssttl
\label{eqn:flx_mss_dps_dfn}
\end{eqnarray}
The explicit definitions for the number-mean and mass-mean deposition
velocities in (\ref{eqn:flx_nbr_dps_dfn}) and
(\ref{eqn:flx_mss_dps_dfn}) are
\begin{eqnarray}
\label{eqn:vlc_dps_nbr_dfn}
\vlcdpsnbravg & = & \frac{1}{\cncttl}
\int_{0}^{\infty} \vlcdpsofdmt \dstnbrofdmt \,\dfr\dmt \\
\label{eqn:vlc_dps_mss_dfn}
\vlcdpsmssavg & = & \frac{1}{\mssttl}
\int_{0}^{\infty} \vlcdpsofdmt \dstmssofdmt \,\dfr\dmt
\end{eqnarray}

Two experimental methods to determine $\vlcdpsofdmt$ are wind tunnel
and field experiments.
Wind tunnel experiments can select particle sizes and thus control 
$\dstnbrofdmt \dmtdlt$ (or $\dstmssofdmt \dmtdlt$).
Usually $\flxnbrdpsofdmt \dmtdlt$ (or $\flxmssdpsofdmt \dmtdlt$) 
are measured in steady state conditions and $\vlcdpsofdmt$ is then
inferred from a discretized version of (\ref{eqn:vlc_dps_mss_dfn}).
In field experiments it is often impossible to control particle size
so independent measurements of the integrated quantities $\flxnbrdps$
or $\flxmssdpsofdmt$ and $\cncttl$ or $\mssttl$ are made instead. 
From such measurements only $\vlcdpsnbravg$ or $\vlcdpsmssavg$ may be 
inferred. 
Additional assumptions or separate measurements of $\dstnbrofdmt$ or
$\dstmssofdmt$ are required to obtain information about the size
dependence $\vlcdpsofdmt$ from (\ref{eqn:vlc_dps_mss_dfn}). 

\section[Dry Deposition Theory]{Dry Deposition Theory}\label{sxn:ddp_thr}
It is reasonable to assume the particle number flux $\flxnbrdpsofdmt$
is the result of diffusive and gravitational processes acting in
parallel. 
The diffusive transport may be further decomposed into the sum of
downgradient turbulent transport and Brownian diffusion. 
Let the turbulent transport be characterized by an eddy diffusion
coefficient $\dffedd$.
According to \trmidx{Fick's first law} (\ref{eqn:fck_1st_law}), the
Brownian diffusion is characterized by $\dffbrn$
(\ref{eqn:dff_brn_dfn}). 
The spectral number flux (defined to be positive downwards) may then 
be written as the sum of the diffusive and gravitational components as
\begin{eqnarray}
% SeH78 p. 7 (3)
\label{eqn:flx_nbr_dps_spc_dfn_2}
\flxnbrdpsofdmt & = & - ( \dffedd + \dffbrn ) 
\frac{d \dstnbrofdmt}{\dfr\hgt } - \vlcgrvofdmt \dstnbrofdmt \\
\flxnbrdpsofdmt + \vlcgrvofdmt \dstnbrofdmt & = & 
- ( \dffedd + \dffbrn ) \frac{d \dstnbrofdmt}{\dfr\hgt } \nonumber \\
\end{eqnarray}
One may define the non-dimensional length $\hgttld$
\begin{eqnarray}
% SeH78 p. 7
\hgttld & = & \wndfrc \hgt / \vscknm \nonumber \\
\hgt & = & \vscknm \hgttld / \wndfrc
\label{eqn:hgt_tld_dfn}
\end{eqnarray}
Since the units of $\vscknm$ are \mSxs\ and the dimensions of
$\wndfrc$ are \mxs, $\hgttld$ is dimensionless.
Non-dimensionalizing (\ref{eqn:flx_nbr_dps_spc_dfn_2}) by
(\ref{eqn:hgt_tld_dfn}) we obtain
\begin{eqnarray}
% fxm: work through this derivation
\label{eqn:non_dmn_dfn}
- \frac{\wndfrc}{\flxnbrdpsofdmt + \vlcgrvofdmt \dstnbrofdmt } & = & 
\frac{\vscknm}{\dffedd + \dffbrn }
\end{eqnarray}
Integrating over the particle concentration from the reference
height $\hgt$ down to the height of zero particle concentration 
\begin{eqnarray}
% SeH78 p. 7 (4)
- \int_{\cnc(\hgt)}^{\cnc(\dmttld)}
\frac{\wndfrc}{\flxnbrdpsofdmt + \vlcgrvofdmt \dstnbrofdmt } \,\dfr\cnc
& = & 
\int_{\hgttld}^{\dmttld}
\frac{\vscknm}{\dffedd + \dffbrn } \,\dfr\hgttld \\
& \equiv & 
\mbox{Int}_{1} + \mbox{Int}_{2} + \mbox{Int}_{3} \nonumber
\end{eqnarray}
The RHS is a negative-valued resistance integral which quantifies the
diffusional resistance between the integration limits.

\subsection[Sehmel's Experiments]{Sehmel's Experiments}\label{sxn:SeH78}
\cite{SeH78,SeH782} performed wind tunnel experiments over a
range of particle sizes $0.03 < \dmt < 29$\,\um, surface types (grass,
gravel, and water) friction velocities $11$--$144$\,\cmxs, and
roughness lengths $0.001 < \rghmmn < 0.6$\,\cm.
The results were used to parameterize the quasi-laminar layer
resistance to particle deposition
\begin{eqnarray}
% SeH78 p. 10 (13), SeH782 p. 2.12
% fxm: make braces displaystyle
\label{eqn:int3_dfn}
\mbox{Int}_{3} & = & \mbox{Int}_{1977W} = 
\dpysty - \exp \left\{ 
-378.051 + 16.498 \ln \shmnbr 
\right.
\nonumber \\
& & 
% {} so ``+'' gets binary not unary operator spacing KoD95 p. 136
{} + \ln \taurlxtld \left[
-11.818 - 0.2863 \ln \taurlxtld + 0.3226 
\ln \left( \frac{\dmt}{\rghmmn } \right)
- 0.3385 \ln \left( \frac{\dffbrn}{\rghmmn \wndfrc } \right) \right] 
\nonumber \\
& & 
% {} so ``+'' gets binary not unary operator spacing KoD95 p. 136
{} \left. -12.804 \ln \dmt \displaystyle \right\}
\end{eqnarray}
where $\taurlxtld$ is the dimensionless relaxation timescale
\begin{eqnarray}
% SeH782 p. 2.12
\taurlxtld & = & \frac{\dnsprt \dmt^{2} \wndfrc^{2}}{18 \vscdyn \vscknm }
\label{eqn:tau_rlx_tld_dfn}
\end{eqnarray}
For particles $\dmt \lesssim 0.01$\,\um, (\ref{eqn:int3_dfn}) results
in surface resistances that are too large, and \cite{SeH78} recommend 
% See SeH78 p. 20
using pure Brownian diffusion in the quasi-laminar layer.

\section[Resistance Method]{Resistance Method}\label{sxn:rss}
The three most significant barriers to aerosol dry deposition are
aerosol mass, boundary layer stability, and quasi-laminar layer
resistance.
\begin{eqnarray}
\vlctrb & = & \frac{1}{\rssaer + \rsslmn + \rssaer \rsslmn \vlcgrv }
\label{eqn:vlc_trb_dfn}
\end{eqnarray}
There is debate whether the transfer coefficient for aerosol
deposition through the turbulent layer follows the aerodynamic
resistance to heat transfer or to momentum transfer.
\cite{Wil82} use $\rssaer = \rssheat$.
\cite{SeH78}, \cite{SlS80} and derivatives use $\rssaer = \rssmmn$.

Settling by aerosol due to gravity occurs independently of turbulent 
mechanisms of dry deposition.
Thus $\vlcgrv$ is added in parallel to deposition velocity due to
turbulent mix-out.
\begin{eqnarray}
% SeP97 p. 961 (19.7)
\vlcdps & = & \vlctrb + \vlcgrv \nonumber \\
\label{eqn:vlc_dps_dfn}
& = & \frac{1}{\rssaer + \rsslmn + \rssaer \rsslmn \vlcgrv } + \vlcgrv \\
\label{eqn:rss_dps_dfn}
& \equiv & \rssdps^{-1}
\end{eqnarray}

Figure~\ref{fgr:vlc_dps_aer} shows the simulated dry deposition velocity
as a function of aerosol size and surface roughness length.
\begin{figure*}
\begin{center}
% Compare to SeP97 p. 970 Figure 19.3, p. 971 Figure 19.4
% Seh84 p. 553 Figure 12.3, Seh80 p. 999 Figure 5
% Figure created by mie.pro:scv_gph()
\includegraphics[width=0.8\hsize]{/Users/zender/data/fgr/aer/vlc_dps_aer}\vfill
\end{center}
\caption[Dry Deposition Velocity]{
Dry deposition velocity $\vlcdps$ (\cmxs) as a function of aerosol
size $\dmtprt$ (\um), and surface roughness length $\rghmmn$ (cm).
Shown are the total dry deposition velocity $\vlcdps$ (solid), 
the gravitational settling velocity $\vlcgrv$ (dashed), and the
turbulent velocity $\vlctrb$ (dotted).
\label{fgr:vlc_dps_aer}}
\end{figure*}
\cite{SeH78} and \cite{Seh80} predict a stronger dependence of
$\vlcdps$ on $\rghmmn$ than is shown in Figure~\ref{fgr:vlc_dps_aer}.
The reason for the discrepancy is not yet understood.

Dry deposition of gaseous species requires consideration of an
additional term accounting for the so-called \trmdfn{canopy
resistance} $\rsscnp$. 
The canopy resistance accounts for the process of gases becoming
irreversibly absorbed by plant stomata.

\subsection[Gravitational Settling]{Gravitational
Settling}\label{sxn:grv} 
Gravity provides a direct force for moving particles through the
turbulent boundary layer, through small scale eddies near the surface,
and through the quasi-laminar layer immediately adjacent to the
surface. 
However, gravity is an inefficient removal mechanism for particles
smaller than 1\,\um.
Our treatment of gravitational settling follows the method of
\cite{SeP97}. 

First, we assume particles are always falling at the their steady
state gravitational settling velocity $\vlcgrv$.
This is the speed at which aerodynamic drag balances gravitational
acceleration so that net acceleration is zero.
For this reason $\vlcgrv$ is called the \trmdfn{terminal velocity}.

The \trmdfn{Stokes' settling velocity} $\vlcstk$ is the
terminal fall speed of particles satisfying two aerodynamic criteria
first identified by Stokes. 
First, the Reynolds number associated with the particle must be less
than~0.1.
Second, the particle must be large enough so that its slip correction
factor (defined below) is near unity.
Under these conditions 
\begin{eqnarray}
% SeP97 p. 466 (8.42)
\vlcstk & = & \frac{\dmt^{2} \dnsprt \grv \slpcrc}{18 \vscdyn}
\label{eqn:vlc_stk_dfn}
\end{eqnarray}
As a rule of thumb, $\vlcstk = \vlcgrv$ for particles in the size range  
$0.01 < \dmt < 1$\,\um.

Very small particles undergo fewer collisions than large particles,
and are susceptible to non-continuum or kinetic effects of fluids.
These properties combine to define the \trmdfn{slip correction factor} 
\begin{equation}
\slpcrc = 1.0 + \frac{2.0 \mfpatm}{\dmt}
\left[ 1.257 + 0.4 
\exp \left( \frac{-1.1 \dmt}{2 \mfpatm} \right) 
\right]
\label{eqn:slp_crc_dfn}
\end{equation}
where $\mfpatm$ is the mean free path of air, defined in
\S\ref{sxn:mfp}. 
Corrections to (\ref{eqn:vlc_stk_dfn}) due to $\slpcrc$ exceed 10\%
for mineral particles smaller than about $1.5$\,\um.  

The \trmdfn{drag coefficient} $\cffdrggrv$ of particles is usually
expressed in terms of the \trmidx{Reynolds number} $\rynnbrgrv$ of the 
particle motion relative to the medium.
Analytic solutions for drag coefficients exist for spherical particles 
for low $\rynnbrgrv$.
For high $\rynnbrgrv$, and for most aspherical shapes, $\cffdrggrv$
are determined from computer simulations of the unsteady flow fields
past the particle \cite[]{Wan02}.
For example, \cite{Wan02} define $\cffdrggrv$ for cylinders of
diameter $\dmt$ as 
\begin{eqnarray}
% Wan02 p. 80 (3.16)
\cffdrggrv & = & \frac{\dmt}{\dnsaer \vlcvec^{2} \alpha}
\label{eqn:cff_drg_dfn}
\end{eqnarray}
where $\alpha$ is one-half of the cross-sectional area of the particle
normal to the flow direction, and $\vlcvec$ is the velocity field.

Exact relationships exist for the Stokes regime where $\rynnbrgrv < 0.1$.
For $\rynnbrgrv > 0.1$, theory and experiment provide
parameterizations which match the available data, but do not always  
agree with one another.
To begin with, \cite{SeP97}, p.~463, suggest
\begin{eqnarray}
% Sep97 p. 463 (8.32)
\cffdrggrv & = & \left\{
\begin{array}{ >{\displaystyle}l<{} @{\quad:\quad} r}
\frac{24}{\rynnbrgrv} & \rynnbrgrv < 0.1 \\
\multicolumn{2}{c}{} \\[-0.5ex]
\frac{24}{\rynnbrgrv } \left[ 1 + \frac{3 \rynnbrgrv}{16} 
+ \frac{9}{160} \rynnbrgrv^{2} \ln ( 2 \rynnbrgrv ) \right]
& 0.1 \le \rynnbrgrv < 2 \\
\multicolumn{2}{c}{} \\[-0.5ex]
\frac{24}{\rynnbrgrv } ( 1 + 0.15 \rynnbrgrv^{0.687} ) &
2 \le \rynnbrgrv < 500 \\
\multicolumn{2}{c}{} \\[-0.5ex]
0.44 & 500 \le \rynnbrgrv < 2 \times 10^{5}
\end{array} \right.
\label{eqn:cff_drg_sph_SeP97}
\end{eqnarray}
It can be seen that there are three regimes of $\rynnbrgrv$ beyond
Stokes flow. 
In the \trmdfn{transition regime}, 
$0.1 < \rynnbrgrv \lesssim 5$, the method of asymptotic expansions
provides experimentally confirmed values. 
In the \trmdfn{Turbulent regime}, $5 \lesssim \rynnbrgrv \lesssim
500$, a pure parameterization is used.
In the \trmdfn{Limiting regime}, $\rynnbrgrv \gtrsim 500$, 
$\cffdrggrv$ approaches a limiting value of $0.44$.

In reality, objects encounter a turbulence transition in the limiting
regime that causes $\cffdrggrv$ to decrease with increasing
$\rynnbrgrv$. 
Based on \url{http://aerodyn.org/Drag/speed-drag.html}, we changed
the limiting regime cutoff from $\rynnbrgrv = 2 \times 10^{5}$ 
(\ref{eqn:cff_drg_sph_SeP97}) to $1 \times 10^{5}$.
Moreover, we splice a new \trmidx{turbulence transition} regime onto
(\ref{eqn:cff_drg_sph_SeP97}) where $\cffdrggrv$ decreases linearly
with the logarithmic increase in $\rynnbrgrv$ according to 
\begin{eqnarray}
% Sep97 p. 463 (8.32)
\cffdrggrv & = & \left\{
\begin{array}{ >{\displaystyle}l<{} @{\quad:\quad} r}
\frac{24}{\rynnbrgrv} & \rynnbrgrv < 0.1 \\
\multicolumn{2}{c}{} \\[-0.5ex]
\frac{24}{\rynnbrgrv } \left[ 1 + \frac{3 \rynnbrgrv}{16} 
+ \frac{9}{160} \rynnbrgrv^{2} \ln ( 2 \rynnbrgrv ) \right]
& 0.1 \le \rynnbrgrv < 2 \\
\multicolumn{2}{c}{} \\[-0.5ex]
\frac{24}{\rynnbrgrv } ( 1 + 0.15 \rynnbrgrv^{0.687} ) &
2 \le \rynnbrgrv < 500 \\
\multicolumn{2}{c}{} \\[-0.5ex]
0.44 & 500 \le \rynnbrgrv < 1 \times 10^{5} \\
\multicolumn{2}{c}{} \\[-0.5ex]
0.44-0.34[\log_{10}(\rynnbrgrv)-5] & 1 \times 10^{5} \le \rynnbrgrv < 1 \times 10^{6}
\end{array} \right.
\label{eqn:cff_drg_sph_new}
\end{eqnarray}

There are two problems with (\ref{eqn:cff_drg_sph_SeP97}).
First, the expressions are not smoothly matched at the boundaries.
This causes finite $\cffdrggrv$ jumps between continuously varying
particle sizes.
For example, when $\dnsprt \sim 2.65$\,\gxcmC\ (mineral dust), a jump
occurs near $\dmt = 80$\,\um\ because the motion $\rynnbrgrv$ suddenly
changes from $\rynnbrgrv < 2$ to $\rynnbrgrv > 2$.  
This jump nearly doubles sedimentation speed and is very unrealistic.
We attempt to solve this problem by blending solutions over limited
ranges of $\rynnbrgrv$. 
The second problem is the accuracy of the expressions in the
transition regime.
Asymptotic expansions in this regime \cite[]{PrP57,ChB69,PrK78,PrK98}
show that 
\begin{eqnarray}
% PrK78 p. 294 (10-52) PrK98 p. 373 (10-53)
\cffdrggrv & = &
\frac{24}{\rynnbrgrv} \left[ 1 + \frac{3 \rynnbrgrv}{16} 
+ \frac{9}{160} \rynnbrgrv^{2} \ln (\rynnbrgrv / 2) 
+ \frac{9}{160} \rynnbrgrv^{2} \left( 
\elrcst + \frac{5}{3} \ln 2 - \frac{323}{360} \right)
+ \frac{27}{640} \rynnbrgrv^{3} \ln (\rynnbrgrv / 2) 
\right]
\label{eqn:cff_drg_trn_dfn}
\end{eqnarray}
where $\elrcst = 0.577215664$ is \trmdfn{Euler's constant}.
The final term in the \cite{SeP97} expression in the transition
regime (\ref{eqn:cff_drg_sph_SeP97}) is $\ln(2\rynnbrgrv)$ rather than
$\ln(\rynnbrgrv/2)$ as in (\ref{eqn:cff_drg_trn_dfn}). 
This appears to be a misprint in \citeauthor{SeP97} and the
(\ref{eqn:cff_drg_trn_dfn}) form is preferred, regardless of whether
the final two terms in (\ref{eqn:cff_drg_trn_dfn}) are employed. 
Despite its apparent complexity, the fourth term in
(\ref{eqn:cff_drg_trn_dfn}) has been verified independently
\cite[]{PrK98}.

The complete solution of the equation of motion for a falling particle 
does not neglect the non-linear dependence of the drag coefficient on
$\rynnbr$. 
The vertical equation of motion applicable to particles of any size
falling in still air is
\begin{equation}
% SeP97 p. 467 (8.43)
\mss \frac{d \vlczzz}{d \tm } = \mss \grv - 
\frac{\mpi \cffdrggrv \dnsatm \dmt^{2} \vlczzz^{2}}{8 \slpcrc }
\label{eqn:vlc_zzz}
\end{equation}
The particle reaches its \trmidx{terminal velocity} when the LHS of
(\ref{eqn:vlc_zzz}) is zero, i.e., when the acceleration vanishes.
Solving for $\vlcgrv$ in this case yields
\begin{equation}
% SeP97 p. 467 (8.44)
\vlcgrv = \left( \frac{4 \grv \dmt \slpcrc \dnsprt
}{ 3 \cffdrggrv \dnsatm } \right)^{1/2}
\label{eqn:vlc_grv_dfn}
\end{equation}
We use (\ref{eqn:vlc_grv_dfn}) to characterize the size range of 
particles which are susceptible to long term atmospheric suspension
and transport, and the size range of saltators which fall too fast to
become suspended.

Note that $\cffdrggrv$ (\ref{eqn:vlc_grv_dfn}) is a function of
$\vlcgrv$ through (\ref{eqn:ryn_nbr_dfn_2}) and
(\ref{eqn:cff_drg_sph_SeP97}).   
Thus (\ref{eqn:vlc_grv_dfn}) is an implicit equation for $\vlcgrv$.
An iterative solution to (\ref{eqn:vlc_grv_dfn}) is straightforward
but too time consuming for large scale atmospheric models.
A reasonable approximation is to define the \trmdfn{Stokes correction 
factor} $\stkcrc$ as  
\begin{equation}
\stkcrc = \vlcgrv / \vlcstk
\label{eqn:stk_crc_dfn}
\end{equation}
The $\stkcrc$ correction to (\ref{eqn:vlc_stk_dfn}) exceeds~10\%
for mineral particles larger than about $45$\,\um.  

In the case where the particles are very large, such as falling
raindrops, (\ref{eqn:vlc_grv_dfn}) simplifies considerably. 
For $\dmt > 1$\,\mm\ we have $\rynnbrgrv > 500$ and thus 
$\cffdrg = 0.44$ 
(\ref{eqn:cff_drg_sph_SeP97}) and $\slpcrc \approx 1$.
Inserting these values in~(\ref{eqn:vlc_grv_dfn}) we obtain
\begin{equation}
\vlcgrv \quad\mbox{(\mxs)} = 5.45 \left( \frac{\dmt \dnsprt}{\dnsatm } \right)^{1/2}
\label{eqn:vlc_grv_pcp_dfn}
\end{equation}
for very large particles and raindrops.
Finally, we may re-express the Reynolds number
(\ref{eqn:ryn_nbr_dfn}), (\ref{eqn:ryn_nbr_dfn_2})
of flow around a falling particle explicitly in terms of its terminal
velocity as
\begin{equation}
\rynnbr = \vlcgrv \dmt / \vscknm 
\label{eqn:ryn_nbr_dfn_3}
\end{equation}

\subsection[Aerodynamics of Aspherical Particles]{Aerodynamics of Aspherical Particles}\label{sxn:aer_asp}
Most mineral particles are not perfect spheres \cite[]{Gin03}.
Aspherical particles require modifications to the preceding
formulation of kinematic properties, such as the settling speed.
First, asphericity alters the drag coefficient which must be known
over same range of $\rynnbr$ as for spherical particles
(\ref{eqn:cff_drg_sph_SeP97}) in order to determine fall speeds.
Theoretical treatments of the drag coefficient for aspherical
particles are few \cite[][]{Abr70,Boo71,Mit96,Gin03}.
The microphysical treatise by \cite{Wan02} describes the types of
analyses brought to bear on these problems with ice crystals.
Many of these analyses hold for mineral dust particles as well.
\cite{IMA01} present a method for removing number and mass of
aspherical ice crystals of various shapes that should be applicable to  
dust as well. 
\cite{VaC02} describe the trajectory of Asian dust events which reach 
North America.

\cite{Wan02} performed \textit{ab initio} simulations of flow past
finite cylinders with aspect ratios satisfying the relations of
\cite{AuV70}. 
His results show that differences between the infinite and finite
cylinder approximations are small for $\rynnbrgrv > 100$.
The following parameterization fits his data within a few percent
\begin{eqnarray}
% Wan02 p. 81 (3.17)
\log_{10} \cffdrggrv & = & 2.44389 - 4.21639\AAA -
0.20098\AAA^{2}+2.32216\AAA^{3} 
\quad : \quad 0.2 < \rynnbrgrv < 100 \\
\mbox{where} \quad \AAA & = & \frac{\log_{10}\rynnbrgrv+1.0}{3.60206}
\label{eqn:cff_drg_cyl_Wan02}
\end{eqnarray}
Unfortunately, (\ref{eqn:cff_drg_cyl_Wan02}) was constructed
from simulations of cylinders with prescribed aspect ratios 
typical of ice crystals \cite[][p.~53]{Wan02}.
It is desirable to find an analogue to (\ref{eqn:cff_drg_cyl_Wan02})  
which is a function of aspect ratio.

\cite{PPH73} computed the drag coefficient of flow past very thin
oblate spheroids for low and intermediate Reynolds numbers.
\cite{Wan02} computed the drag coefficient of flow past hexagonal
plates and broad-branched crystals and fit his results to the same
functional form as \citeauthor{PPH73}.
For hexagonal plates, the results are
\begin{eqnarray}
% Wan02 p. 101 (3.19)
\cffdrggrv & = & \left( \frac{64}{\mpi \rynnbrgrv} \right)
(1 + 0.078 \rynnbrgrv^{0.945}) 
\quad : \quad 0.2 < \rynnbrgrv < 150 
\label{eqn:cff_drg_plt_hxg_Wan02}
\end{eqnarray}
For broad branch crystals, the results are
\begin{eqnarray}
% Wan02 p. 101 (3.20)
\cffdrggrv & = & \left( \frac{64}{\mpi \rynnbrgrv} \right)
(1 + 0.142 \rynnbrgrv^{0.887}) 
\quad : \quad 0.2 < \rynnbrgrv < 150 
\label{eqn:cff_drg_plt_bbc_Wan02}
\end{eqnarray}

\subsubsection[Radiative Effects of Asphericity]{Radiative Effects of Asphericity}\label{sxn:asp_rdn}
\cite{KaS04} present a comprehensive analysis of the radiative
properties of aspherical mineral dust using a
\trmidx{Composition-Shape-Size} (\trmidx{CSS}) approach based on the
\trmidx{Discrete Dipole Approximation} (\trmidx{DDA}).
\cite{KDK05} apply these modeling techniques to dust aerosol
retrievals from \trmidx{MISR}.
\cite{KKS05} document MISR dust retrievals in optically thick dust
plumes.
\cite{KKL05} show their CSS approach allows MISR retrievals to
distinguish natural dust particle shapes and composition.

A number of recent papers are devoted to the effects of mineral
dusts (and other aerosols) in determining surface reflectance in
deserts and snowpack.
\cite{PDT01} demonstrate the retrieval of snow algal concentrations
(based on properties \trmidx{Chlamydomonas nivalis}) from
\trmidx{AVIRIS} instrument data.
They conclude the visible albedo change due to \trmidx{C.~nivalis} is
not significant.
\cite{MAH05} demonstrate the presence of soot in fresh snow by
minimizing model fits to measured reflectances.
\cite{KAH05} compare aspherical snow grain models to measured
reflectances at high spectral resolution, and demonstrate the
feasibility of retrieving snow grain size from such measurements. 
\cite{AMY05} report high resolution spectral surface albedos over
multiple desert surface types in western China including dunes and
crusted soils.
\cite{ATU05} simulated the aerosol radiative forcing of
multi-component dust aerosol over many surface types, and explored the
sensitivity of the results to a number of assumptions such as
underlying surface albedo, presence of other aerosols, and snowpack.
% \cite{AHK06} \ldots % fxm: what was this reference?
\cite{KBB06} predicted that dust-induced snow melt would free Northern
Asia of snow during the LGM.

Exact radiative treatments of particle asphericity are often hampered 
by the large number of computations required (e.g., DDA) the
incomplete range of size parameters $\szprm$ applicable (e.g.,
Geometric Optics for large $\szprm$, and Finite-Difference-Time-Domain 
(\trmidx{FDTD}) for small $\szprm$), or the absence of crucial 
information such as the phase function (\trmidx{ADT}).

\subsection[Volume-to-Surface Techniques]{Volume-to-Surface Techniques}\label{sxn:vts}
Equal volume-to-surface area (\trmidx{V/S}) (sometimes called
volume-to-area or \trmidx{V/A}) techniques simulate aspherical optical
properties using a collection of spheres which preserve desired
properties of the aspherical particles \cite[]{GrW99,NGW03}.
\cite{GrW99} note that 
\begin{quote}
The use of ``equivalent'' spheres to represent
the scattering and absorption properties of nonspherical particles has
been unsatisfactory in the past because the sphere of equal volume has
too little surface area and thus too little scattering, whereas the
sphere of equal area has too much volume giving too much absorption.
\end{quote}
They suggest representing aspherical particles by distributions of
spheres with the same volume-to-surface-area ($\vlm/\sfc$) ratio as
the aspherical particle.
They suggest the V/S technique works best for highly complex shapes,
such as natural dust particles and ice crystals.
The V/S technique requires as many, but no more, computations than
the standard Mie theory. 

The V/S ratio of a sphere is
\begin{eqnarray}
\frac{\vlmsph}{\sfcsph} & = & 
\frac{\frac{4}{3}\mpi\rds^{3}}{4\mpi\rds^{2}} \nonumber \\
& = & \rds/3
\label{eqn:vts_sph_dfn}
\end{eqnarray}
Based on (\ref{eqn:vts_sph_dfn}), the sphere with the same
volume-to-surface ratio as an arbitrarily shaped aspherical particle
with volume and surface area $\sfc$\,[\mS] and $\vlm$\,[\mC],
respectively, must have radius $\rdsvts$\,[\m] where
\begin{eqnarray}
\rdsvts & = & 3\frac{\vlm}{\sfc}
\label{eqn:rds_vts_dfn}
\end{eqnarray}

The final ingredient necessary to implement the V/S technique is 
the number $\cncvts$\,[\nbrxmC] of V/S-spheres with the same volume
(or area) as the aspherical particles.
We establish $\cncvts$ by imposing the condition that
$\cncvts$ spheres with equal V/S ratios as the aspherical particles  
have the same total volume as $\cnc$ aspherical particles.
\begin{eqnarray}
\cncvts \vlmvts & = & \cnc \vlm \\
\cncvts \frac{4\mpi\rdsvts^{3}}{3} & = & \cnc \vlm \nonumber \\
\frac{\cncvts}{\cnc} & = & \frac{3\vlm}{4\mpi\rdsvts^{3}}
\label{eqn:cnc_vts_dfn}
\end{eqnarray}
We could have have imposed the condition of conserving the aspherical
surface area instead of volume in (\ref{eqn:rds_vts_dfn}).
By construction V/S-spheres will have the correct area because
V/S-spheres have, by definition, the correct V/S ratio.  
Once $\cncvts/\cnc$ is known (\ref{eqn:cnc_vts_dfn}), we can construct
a model population of spheres with the same total volume and surface
area as the aspherical particles.  
What differs beteween the model population of spheres and the actual
population of aspherical particles is the number concentration.

The special cases of $\rdsvts$ and $\cncvts$ for hexagonal prisms are 
given in Section~\ref{sxn:hxg} in
Equations~(\ref{eqn:rds_vts_hxg_dfn}) and~(\ref{eqn:cnc_vts_hxg_dfn}),
respectively. 

\cite{NGW03} evaluate the V/S approximation for hexagonal prisms over
the range of densities from thin cirrus clouds to thick snowpack. 
One difficulty in applying the V/S approximation in both clouds and
snowpacks is determining the correct V/S ratio.
Knowledge of particle shape (e.g., hexagonal crystals) and aspect
ratio for all particle sizes is necessary to apply the V/S
approximation. 
While this is helpful in theory, in practice this shape and aspect
ratio information is unavailable and difficult to measure.

One alternative is to estimate the bulk V/S ratio (e.g., of the entire
cloud or snowpack) rather than the size-specific ratios and work
backwards from there.
For example, bulk snow mass concentration $\mssttl$\,[\kgxmC] 
(i.e., density) may be available from model or observations. 
This may be combined with theoretical or empirical estimates of 
specific snow surface area $\sfcspc \equiv \sfcttl/\mssttl$\,[\mSxkg]
to yield the snow surface area concentration 
$\sfcttl = \sfcspc\mssttl$\,[\mSxmC]. 
\cite{LTD04} describe a theory for the time evolution of~$\sfcspc$,
and \cite{CLD03} give examples of this function for particular
snowpacks. 

\subsection[Large Dust Particles]{Large Dust Particles}\label{sxn:ldp}
\cite{BCD88} present evidence for particles as large as 75\,\um\
reaching Hawaii from Asia.
\cite{Gin03} studied the processes contributing to large particle
transport. 

\section{Quasi-Laminar Layer Resistance}\label{sxn:qll}

\subsection[Stokes Number]{Stokes Number}\label{sxn:stk}
Large particles less able to change direction with a boundary
layer flow that must veer sharply to avoid obstructions such as  
roughness elements (plants, trees) and the surface.
Instead, the particle's inertia carries it through the quasi-laminar
layer, allowing the particle to be deposited on the surface.
This dry depositional process is called \trmdfn{inertial impaction}.
The \trmdfn{Stokes number} $\stknbr$ determines the particle
susceptibility to inertial impaction.
The Stokes number relevant to boundary layer flow depends on the
particle size $\dmt$, particle density $\dnsatm$, slip correction
factor $\slpcrc$, fluid velocity $\vlcinf$, viscosity $\vscdyn$, and
characteristic length scale $\lngscl$.   
\begin{equation}
% SeP97 p. 487
\stknbr = \frac{\dmt^{2} \dnsprt \slpcrc \vlcinf}{18 \vscdyn \lngscl }
\label{eqn:stk_nbr_dfn_1}
\end{equation}
$\stknbr$ is the ratio of the particle stopping distance to the
characteristic length of the flow.  

It is difficult to apply (\ref{eqn:stk_nbr_dfn_1}) directly in
the boundary layer since $\lngscl$ and $\vlcinf$ have not been 
characterized yet.
Instead, we make the assumptions that \ldots
% fxm: Really? I thought we used a sophisticated Stokes correction
Then $\stknbr$ may be computed as 
\begin{equation}
% SeP97 p. 965
\stknbr = \frac{\vlcgrv \wndfrc^{2}}{\grv \vscknm}
\label{eqn:stk_nbr_dfn_2}
\end{equation}
Note that (\ref{eqn:stk_nbr_dfn_2}) employs $\vlcgrv$
(\ref{eqn:vlc_grv_dfn}) rather than $\vlcstk$
(\ref{eqn:vlc_stk_dfn}). 

\subsection[Brownian Diffusion]{Brownian Diffusion}\label{sxn:brn}
Transport through the quasi-laminar layer also depends on the
\trmdfn{Brownian diffusivity} $\dffbrn$ of the particles.
Brownian diffusivity measures the efficiency of particle displacements
due to random motion between collisions.
This thermally driven motion is isotropic and depends on the
temperature of the fluid and the mass of the particle.
\begin{equation}
% SeP97 p. 474 (8.73), PeE92 p. 2558 (21)
% Wil82 p. 1935 provide alternate pzn.
\dffbrn = \frac{\bltcst \tpt \slpcrc}{3 \mpi \vscdyn \dmt } 
\label{eqn:dff_brn_dfn}
\end{equation}
Without the slip correction factor (\ref{eqn:dff_brn_dfn}) is known as
the \trmdfn{Stokes-Einstein relation}.
The displacement of a particle due to Brownian motion may carry the
particle across the quasi-laminar layer and deposit it to the surface.
It is instructive to compare the diffusivity of aerosols
(\ref{eqn:dff_brn_dfn}) to the diffusivity of gases
(\ref{eqn:dff_gas_air_dfn}).

The \trmdfn{Schmidt number} $\shmnbr$ is the ratio of the kinematic
viscosity of the fluid to the Brownian diffusivity of the particle
\begin{equation}
% SeP97 p. 972, p. 964
\shmnbr = \vscknm / \dffbrn
\label{eqn:shm_nbr_dfn}
\end{equation}
Thus $\shmnbr$ is the ratio of two diffusions: the diffusivity of
momentum and vorticity $\vscknm$ to the Brownian diffusivity of the
particle $\dffbrn$ \cite[]{Sli82}.

The two most important processes (besides gravity) for aerosol
transport through the quasi-laminar layer are inertial impaction
(\ref{eqn:stk_nbr_dfn_2}) and Brownian diffusion
(\ref{eqn:shm_nbr_dfn}). 
The total resistance of the quasi-laminar layer to aerosol dry
deposition may be approximated as the resistance to these processes
acting in parallel 
\begin{eqnarray}
% SeP97 p. 965 (19.18), p. 972 (19.18)
\rsslmn & = & \left\{
\begin{array}{ >{\displaystyle}l<{} @{\quad:\quad}l}
\frac{1}{\wndfrc (\shmnbr^{-2/3} + 10^{-3/\stknbr})} & \mbox{Solid surfaces} \\
\frac{1}{\wndfrc (\shmnbr^{-1/2} + 10^{-3/\stknbr})} & \mbox{Liquid surfaces}
\end{array} \right.
\label{eqn:rss_lmn_dfn}
\end{eqnarray}
The Schmidt number term in the denominator accounts for Brownian
diffusion and is dominant for $\dmt \lesssim 0.7$\,\um.
The Stokes number term accounts for inertial impaction and becomes
important for $\dmt \gtrsim 5$\,\um.
There is some confusion concerning the dependence of $\rsslmn$ on
$\shmnbr$. 
The resistance to particle or gaseous diffusion across the
quasi-laminar layer to a solid surface is proportional to
$\shmnbr^{-2/3}$ \cite[]{SHH78}.
This result can be derived by considering viscous flow at high
Reynolds number over a fixed, smooth surface.
A free surface such as liquid water, however, will tend to slip in the
direction of the mean wind so that the characteristic air velocity in
the diffusion layer is somewhat larger.
According to \cite{SlS80}, the resulting transfer coefficient for
particles across the quasi-laminar layer to a free surface (e.g.,
ocean) is proportional to $\shmnbr^{-1/2}$.  

Equation~(\ref{eqn:rss_lmn_dfn}) neglects the following processes
which may contribute to aerosol transport across the quasi-laminar
layer: \trmdfn{thermophoresis}, \trmdfn{electrophoresis},
and \trmdfn{diffusiophoresis}\footnote{diffusiophoresis is also known
as \trmdfn{Stefan flow}} (``phoresis'' means ``force'')
\cite[e.g.,][]{SHH78,SeP97}.  

\chapter{Wet Deposition}\label{sxn:wet_dps} % Sep97 p. 1021
\cszepigraph{%
Just a box of rain---\\
Wind and water---\\
Believe it if you need it,\\
if you don't just pass it on}
{http://arts.ucsc.edu/gdead/agdl/box.html}{Box of Rain\label{BoR}}{Robert Hunter} 
\csznote{
\cszepigraph{I heard a voice inside my head\\in the desert wind so dry}{http://www3.clearlight.com/~acsa/introjs.htm?/~acsa/songfile/GOMORRAH.HTM}{Gomorrah}{Robert Hunter}
\cszepigraph{I'm going where those chilly winds don't blow}{http://www3.clearlight.com/~acsa/introjs.htm?/~acsa/songfile/MEANDMYU.HTM}{Going down the road}{Traditional}
\cszepigraph{unbroken chain of the western wind}{http://arts.ucsc.edu/gdead/agdl/chain.html}{Unbroken Chain}{Robert M. Petersen}
\cszepigraph{like the morning sun you come\\and like the wind you go}{http://arts.ucsc.edu/gdead/agdl/uncle.html}{Uncle John's Band}{Robert Hunter}
\cszepigraph{Sunlight splatters dawn with answers\\Darkness shrugs and bids the day goodbye}{http://arts.ucsc.edu/gdead/agdl/stephen.html}{Saint Stephen}{Robert Hunter}
\cszepigraph{Wildflower seed in the sand and wind\\May the four winds blow you home again}{http://arts.ucsc.edu/gdead/agdl/franklin.html}{Franklin's Tower}{Robert Hunter}
\cszepigraph{What fatal flowers of\\darkness spring from\\seeds of light}{http://arts.ucsc.edu/gdead/agdl/allah.html}{Blues for Allah}{Robert Hunter}
\cszepigraph{Rainbows end down that highway where ocean breezes blow}{http://www.eff.org/Publications/John_Perry_Barlow/HTML/barlows_lyrics.html#estimated}{Estimated Prophet}{John Perry Barlow}
\cszepigraph{Come wash the nighttime clean,\\Come grow this scorched ground green}{http://www.eff.org/Publications/John_Perry_Barlow/HTML/barlows_lyrics.html#cassidy}{Cassidy}{John Perry Barlow}
\cszepigraph{I went down to those dusty streets, blood was on my mind}{http://www.eff.org/Publications/John_Perry_Barlow/HTML/barlows_lyrics.html#mexicali}{Mexicali Blues}{John Perry Barlow}
\cszepigraph{It bein' summer, I took off my shirt,\\And I tried to wash off some of that dusty dirt.}{http://www3.clearlight.com/~acsa/introjs.htm?/~acsa/songfile/MEANDMYU.HTM}{Me and my uncle}{John Phillips}
\cszepigraph{the bottle was dusty but the liquor was clean}{http://arts.ucsc.edu/gdead/agdl/brown.html}{Brown Eyed Women}{Robert Hunter}
} % end csznote

This section describes the sink processes which occur in the presence
of clouds and precipitation.
Due to the problem of unresolved scales and uncertain collision and
collection coefficients, representation of wet deposition in large
scale atmospheric models is as much art as science.
Wet depositional processes have accumulated much redundant
terminology.
For our purposes, \trmdfn{wet deposition} refers to (the sum of) all
depositional processes by which aerosol are removed from the
atmosphere due to physical uptake (collection, precipitation) by
cloud or precipitation particles.

The efficacy of various aerosol wet deposition schemes has been tested
for the case of \PbCCX\ deposition.
\RnCCXXII\ is continuously emitted from the Earth's continental surfaces
at a rate which is fairly well known on large scales,
$0.72$--$1.2$,\axcmSs\ \cite[]{GBS98}.  
\RnCCXXII\ has a half-life of 3.8\,days and decays to \PbCCX\ with a
half-life of 22\,years.
The \PbCCX\ gas attaches rapidly to ambient aerosol in the
accumulation mode.

\section{Wet Deposition Literature}\label{sxn:wdp_ltr}

Removal of accumulation mode \PbCCX\ by dry deposition is inefficient
so that removal of atmospheric \PbCCX\ is due to wet deposition to
first approximation.  
\cite{GiC86} introduced a first-order loss rate method.
\cite{BJG93} apply the method of \cite{GiC86} to~\PbCCX.
\cite{GBS98} generalize \cite{BJG93} to include size-dependence and
evaluate the scheme with \PbCCX.
\cite{LaC982} show that the gravitational settling of smaller
non-precipitating cloud particles may cause a significant downward
redistribution of soluble trace gases.
Collision or coalescence scavenging mechanisms are thought to be
responsible for the incorporation of the dust particles into raindrops.
\cite{LPG90} reported \pH\ as high as $8.2$ in raindrops containing
mineral dust.

\cite{Sli82,Sli84} reviews the measurements and theory of wet and dry
deposition and presents a summary of the various dimensionless fluid
mechanical quantities which define particle flow.
The \trmdfn{Peclet number} $\pclnbr$ is the ratio of the transport
velocity of particles (i.e., $\vlcgrv$ for gravitational
sedimentation) to their diffusion velocity ($\dffbrn/\rds$).
\begin{eqnarray}
% SeP97 p. 972, p. 964
\pclnbr & = & \vscknm \rds / \dffbrn
\label{eqn:pcl_nbr_dfn_1}
\end{eqnarray}
Inserting (\ref{eqn:shm_nbr_dfn}) into (\ref{eqn:pcl_nbr_dfn_1}) we
obtain 
\begin{eqnarray}
\pclnbr & = & \rynnbr \shmnbr
\label{eqn:pcl_nbr_dfn_2}
\end{eqnarray}
Although we shall not, many authors use $\pclnbr$ in place of the
product of the $\rynnbr \shmnbr$ in aerosol kinematics such as
(\ref{eqn:cll_fsh_brn_dff_dfn}) below. 

\cite{DaH76} synthesized then available parameterizations of
diffusion, impaction, and interception into a numerically tractable
definition of the scavenging coefficient.
\cite{NGD94} performed a sensitivity study of the scavenging
coefficient to the formulation of the particle and droplet size
distribution and showed that a lognormal distribution of raindrop
sizes produce negligible bias compared to other size distributions.

% \cite{FHL04} might be ..., Horowitz, Leovy?
FHL04 use \SOd\ concentration as a proxy for chemical aging and
hygroscopicity of mineral aerosol.
Such aging exacerbates nucleation scavenging of dust downwind of
polluted regions, and may help explain model underprediction of dust
deposition in the northwest Pacific relative to the tropical North
Atlantic.

\section[Collision Processes]{Collision Processes}\label{sxn:cll_prc}
The \trmdfn{collision efficiency} between collector particles (i.e.,
raindrops) of size $\dmtpcp$ and collected particles (i.e., aerosols)
of size $\dmtprt$\footnote{In order to reduce confusion, this chapter
uses lowercase $\dmtprt$ to denote the smaller, collected (i.e.,
aerosol) particle size and uppercase $\dmtpcp$ to denote the larger,
collector (i.e., raindrop) size. 
In the rest of this monograph, aerosol particle size is usually
indicated simply by $\dmt$.} is denoted $\cllfshofpcpprt$. 
If aerodynamic interactions between the particles did not occur then
$\cllfshofpcpprt$ would be unity.
Thus $\cllfshofpcpprt$ corrects geometric collection volumes for the
effects of streamline deformation due to particle-fluid interactions.
In practice the processes accounted for by $\cllfshofpcpprt$ include
Brownian diffusion, interception, and inertial impaction and so we 
assume
\begin{equation}
\cllfshofpcpprt = \cllfshbrndff + \cllfshntc + \cllfshmpc
\label{eqn:cll_fsh_dfn_1}
\end{equation}
where $\cllfshbrndff$, $\cllfshmpc$, and $\cllfshntc$ are the
individual collision efficiencies for the processes of Brownian
diffusion, interception, and inertial impaction.
Processes not typically accounted for include thermophoresis and
electrophoresis. 
We next consider the physical forms of the individual processes.

\subsection[Brownian Diffusion]{Brownian Diffusion}\label{sxn:dff_brn}
The collision efficiency due to Brownian Diffusion, $\cllfshbrndff$,
accounts for particles ``random walking'' across streamlines due to
thermal motion.   
Brownian diffusion is primarily important for very small particles,
thus $\cllfshbrndff$ dominates $\cllfshofpcpprt$ for 
$\dmtprt < 0.2$\,\um\ (cf. Figure~\ref{fgr:cll_fsh}, below).
According to \cite{Sli84},
\begin{eqnarray}
% SeP97 p. 1019 (20.56), Sli82 p. 324 (45) and p. 331 caption of Figure 11, NGD94 p. 2336 (10)
\cllfshbrndff & = & \frac{4}{\rynnbr \shmnbr } 
[ 1 + 0.4 \rynnbr^{1/2} \shmnbr^{1/3} + 0.16 \rynnbr^{1/2} \shmnbr^{1/2} ]
\label{eqn:cll_fsh_brn_dff_dfn}
\end{eqnarray}
The first two terms inside the brackets constitute the particle
Sherwood number. 
The $0.4$ coefficient was determined experimentally. 
Note that $\rynnbr$ and $\shmnbr$ in (\ref{eqn:cll_fsh_brn_dff_dfn})
refer to different particles: $\rynnbr$ is the \trmidx{Reynolds
number} of the raindrop, but $\shmnbr$ is the \trmidx{Schmidt number} 
of the aerosol.

\subsection[Interception]{Interception}\label{sxn:ntc}
The collision efficiency due to interception, $\cllfshntc$,
accounts for particles which follow the streamlines of fluid flow
around an approaching raindrop, but whose radius is greater than
the distance of the streamline to the raindrop.
Thus interception is strictly due to particle size, not mass, 
and is primarily important for large particles.
According to \cite{Sli84},
\begin{eqnarray}
\cllfshntc & = & 4 \dmtrat [ \vscrat + 
\dmtrat ( 1 + 2 \sqrt{\rynnbr} ) ]
\label{eqn:cll_fsh_ntc_dfn}
\end{eqnarray}
where $\dmtrat \equiv \dmtprt / \dmtpcp $ is the ratio of the
collectee to collector sizes.
As in (\ref{eqn:cll_fsh_brn_dff_dfn}), the $\rynnbr$ refers to the
raindrop, not to the aerosol.
The \trmdfn{viscosity ratio}, 
$\vscrat \equiv \vscdynatm / \vscdynHdO$,
is the ratio of the \trmidx{dynamic viscosity} of the atmosphere to
the dynamic viscosity of the fluid in the collector (i.e., liquid
water in the case of raindrops). 
\cite{Sli82} uses $\fff$, the ratio of the maximum internal
circulation speed within the raindrop to the to raindrop fall speed,
in place of $\vscrat$ in (\ref{eqn:cll_fsh_ntc_dfn}).
Section 10.3.1 of \cite{PrK98} contains an extensive discussion of
both of these quantities.
They suggest an appropriate viscosity ratio for falling raindrops of
$\vscrat \approx 1/55$, and an appropriate velocity ratio $\fff
\approx 1/25$.
Interception $\cllfshntc$ dominates $\cllfshofpcpprt$ for 
$1 < \dmtprt < 3$\,\um\ \cite[]{NGD94} 
(cf. Figure~\ref{fgr:cll_fsh}, below).

\subsection[Inertial Impaction]{Inertial Impaction}\label{sxn:mpc}
The collision efficiency due to impaction, $\cllfshmpc$,
accounts for particles whose inertia prevents them from following the
streamlines of fluid flow around an approaching raindrop.
Due to their large size (and Stokes number) and the sharpness
of the streamlines, inertia prevents these particles from being pushed
out of the trajectory falling raindrops so collisions occur. 
A more apt description of this process is \trmdfn{inertial impaction}.
Impaction is primarily important for very large particles.

The derivation of $\cllfshmpc$ depends on the fluid mechanical concept
of \trmdfn{potential flow}.
The relative flow between a raindrop and an aerosol is characterized
by the difference in fall velocities, $\vlcgrvpcp - \vlcgrvprt$.
The Stokes number of the aerosol in this relative flow is
\begin{eqnarray}
% Sli82 p. 329 and p. 331, SeP97 p. 1019, NGD94 p. 2336 (4)
\stknbrrlt & = & \frac{2 \taurlx ( \vlcgrvpcp - \vlcgrvprt )}{\dmtpcp }
\label{eqn:stk_nbr_rlt_dfn}
\end{eqnarray}
where $\taurlx$ is the characteristic relaxation timescale of the
aerosol 
\begin{eqnarray}
% SeP97 p. 465 (8.38)
\taurlx & = & \frac{\mss \slpcrc}{3 \mpi \vscdynatm \dmtprt } \\
\taurlxtld & = & \frac{\dnsprt \dmt^{2} \wndfrc^{2}}{18 \vscdynatm \vscknm } \nonumber \\
& = & \frac{\dnsprt \dmt^{2} \wndfrc^{2}}{18 \vscdynatm
( \vscdynatm / \dnsprt ) } \nonumber \\
& = & \frac{\dnsprt^{2} \dmt^{2} \wndfrc^{2}}{18 \vscdynatm^{2} } \nonumber \\
& = & \frac{\mpi \dnsprt \dmt^{3}}{6} \frac{\dnsprt}{\mpi \dmt } 
\frac{\wndfrc^{2}}{3 \vscdynatm^{2} } \nonumber \\
& = & \frac{\mss \dnsprt \wndfrc^{2}}{3 \mpi \dmt \vscdynatm^{2} } \nonumber \\
& = & \frac{\mss}{3 \mpi \vscdynatm \dmtprt } 
\frac{\dnsprt \wndfrc^{2}}{\vscdynatm } \nonumber \\
& = & \taurlx \frac{\dnsprt \wndfrc^{2}}{\vscdynatm } \\
& = & \mbox{s} \ \mbox{\kgxmC} \mbox{\mSxsS} \ ( \mbox{\kgxms} )^{-1}\nonumber
\label{eqn:tau_rlx_dfn}
\end{eqnarray}
Since $\dnsprt \wndfrc^{2} / \vscdynatm$ has units of \xs, $\taurlxtld$
is indeed dimensionless.
The relaxation timescale is the $\eee$-folding time for a particle to
approach its terminal velocity starting from a motionless state.
$\taurlx$ is purely a function of the aerosol particle and the
environment. 
The raindrop properties enter the definition of $\stknbrrlt$ by
determining the speed of the relative flow, and by determining the
characteristic length of the flow.
This characteristic length, $\dmtpcp$, which appears in the
denominator of (\ref{eqn:stk_nbr_rlt_dfn}), is the distance
the aerosol must be displaced in order to avoid impaction.

A~critical Stokes number $\stknbrcrt$ for this relative flow
may be derived for particles directly in the path of the upstream
stagnation point of the approaching sphere (raindrop). 
$\stknbrcrt$ is the maximum Stokes number of the relative flow a
particle can have and not be impacted when it is directly in the path
of the approaching sphere. 
Particles with $\stknbrrlt < \stknbrcrt$ are even less likely to impact
in slower, more viscous flows \cite[]{Sli82}.
A~combination of numerical simulation and analytic expansion yields
\begin{eqnarray}
% Sli82 p. 329 and p. 331, SeP97 p. 1019 (20.57), NGD94 p. 2336 (8)
\stknbrcrt & = & \frac{\frac{12}{10} + \frac{1}{12} \ln ( 1 + \rynnbr) }{
1 + \ln \rynnbr }
\label{eqn:stk_nbr_crt_dfn}
\end{eqnarray}
As before, $\rynnbr$ (\ref{eqn:ryn_nbr_dfn}) refers to the raindrop,
not the aerosol. 

The critical Stokes numbers for flow near the stagnation point past a
sphere, circular cylinder, and disk have been derived analytically 
as $1/12$, $1/8$, and $\mpi/16$, respectively \cite[]{Sli82}.
Thus (\ref{eqn:stk_nbr_crt_dfn}) asymptotes to the correct
$\stknbrcrt$ for large $\rynnbr$ past a sphere. 

According to \cite{Sli82}, $\cllfshmpc$ may be expressed solely in
terms of $\stknbrrlt$ and $\stknbrcrt$ as 
\begin{eqnarray}
\cllfshmpc & = & \left\{ 
\begin{array}{l@{\quad:\quad}r}
0 & \stknbrrlt < \stknbrcrt \\
\dpysty \left( \frac{\dnspcp}{\dnsprt}\right)^{1/2}
\left( \frac{\stknbrrlt - \stknbrcrt}
{\stknbrrlt - \stknbrcrt + \frac{2}{3}} \right)^{3/2} & 
\stknbrrlt \ge \stknbrcrt 
\end{array} \right.
\label{eqn:cll_fsh_mpc_dfn}
\end{eqnarray}
The threshold in (\ref{eqn:cll_fsh_mpc_dfn}) results in a
discontinuous transition from a regime of no impaction for 
$\dmtprt \lesssim 3$\,\um\ to impaction-dominated for
$\dmtprt \gtrsim 3$\,\um.
$\cllfshmpc$ dominates $\cllfshofpcpprt$ for $\dmtprt \gtrsim 3$\,\um\ 
(cf. Figure~\ref{fgr:cll_fsh}, below). 
Particle density $\dnsprt$ plays a role in the impaction efficiency 
as $\cllfshmpc \propto \dnsprt^{-1/2}$ (\ref{eqn:cll_fsh_mpc_dfn}).
Raindrop density $\dnspcp$ plays two roles in the impaction
efficiency.
First, $\cllfshmpc \propto \sqrt{\dnspcp}$.
Second, the Reynolds number deterimines $\stknbrcrt$
(\ref{eqn:stk_nbr_crt_dfn}) and $\rynnbr \propto \vlcgrvpcp \propto
\sqrt{\dnspcp}$ (\ref{eqn:vlc_grv_dfn}). 

\section[Collision Efficiency]{Collision Efficiency}\label{sxn:cll_fsh}
Summing contributions from these three processes, Brownian diffussion,
interception, and impaction, (\ref{eqn:cll_fsh_dfn_1}) becomes
\begin{eqnarray}
% fxm: use multiline equation method instead of new equations here
\cllfshofpcpprt & = & 
% Brownian Diffusion
\frac{4}{\rynnbr \shmnbr } 
[ 1 + 0.4 \rynnbr^{1/2} \shmnbr^{1/3} + 0.16 \rynnbr^{1/2} \shmnbr^{1/2} ]
\nonumber \\ 
& & % Interception
% {} so ``+'' gets binary not unary operator spacing KoD95 p. 136
{} + 4 \dmtrat [ \vscrat +
\dmtrat ( 1 + 2 \sqrt{\rynnbr} ) ]
\nonumber \\
& & % Impaction
% {} so ``+'' gets binary not unary operator spacing KoD95 p. 136
{} + \hvyfnc ( \stknbrrlt - \stknbrcrt )
\left( \frac{\dnspcp}{\dnsprt } \right)^{1/2}
\left( \frac{\stknbrrlt - \stknbrcrt }{
\stknbrrlt - \stknbrcrt + \frac{2}{3} } \right)^{3/2}
\label{eqn:cll_fsh_dfn_2}
\end{eqnarray}
where $\hvyfnc ( \xxx )$ is the \trmdfn{Heaviside step function}, 
$\dnspcp$ is the density of the precipitation, and $\dnsprt$ is the
aerosol density.  
\cite{Sli82} cautions that although the processes discussed so far in
(\ref{eqn:cll_fsh_dfn_2}) may be reasonably well known (i.e., within a
factor of two), other processes (particle growth,
\trmidx{thermophoresis} and \trmidx{electrophoresis}) may cause the
actual $\cllfshofpcpprt$ to differ from (\ref{eqn:cll_fsh_dfn_2}) by
\textit{orders of magnitude}. 

Figure~\ref{fgr:cll_fsh} shows the simulated collision efficiency
$\cllfshofpcpprt$ between monodisperse raindrops
and an aerosol population. 
\begin{figure*}
\begin{center}
% Compare to DaH76 p. 46 Fgr 1, SeP97 p. 1020 Fgr 20.10
% Figure created by mie.pro:psd_bch()
\includegraphics[width=0.8\hsize]{/Users/zender/data/fgr/aer/psd_cll_fsh}\vfill
\end{center}
\caption[Collision Efficiency]{
Collision efficiency $\cllfsh$ as a function of aerosol size
$\dmtprt$ for monodisperse raindrops.
\label{fgr:cll_fsh}}
\end{figure*}
The transition between diffusion and interception regimes occurs near
$\dmtprt = 0.6$\,\um, and the transition between interception and
impaction regimes occurs near $\dmtprt = 5.0$\,\um.
\cite{DaH76} present simpler, analytically integrable forms of
$\cllfshofpcpprt$. 

The collision efficiency (\ref{eqn:cll_fsh_dfn_2}) embodies an
impressive amount of theoretical and empirical physical studies.
We shall note some interesting features of the solution.
Particle density $\dnsprt$ appears directly only in the impaction
term. 
Diffusion and interception are functions of aerosol size (not mass), 
and of environmental and raindrop properties.
Thus aerosols of different compositions are expected to have similar
collision efficiencies except in the inertial impaction regime where
$\cllfshofpcpprt \propto (\dnspcp/\dnsprt)^{1/2}$.

\section[Sticking Efficiency]{Sticking Efficiency}\label{sxn:stc_fsh}
The preceding discussion covers only the geometric and fluid dynamical
aspects of particle motion that contribute to particle collision.
To compute the actual mass removal rate we must know how often 
collisions result in collection or retention of the the aerosol.
Thus we are interested in the \trmdfn{sticking efficiency} or
\trmdfn{retention efficiency} of interparticle collisions.
Sticking efficiency is defined as the fraction of collisions that 
result in immediate accretion of the collected particle, as opposed to
collisions where the particle simply ``bounces off'' the collector.
The \trmdfn{collection efficiency} $\clcfshofpcpprt$ is defined as the  
sticking efficiency times the collision efficiency 
\begin{eqnarray}
\stcfsh & = & \cllfsh / \clcfsh \\
\label{eqn:stc_fsh_dfn}
\clcfsh & = & \stcfsh \cllfsh
\label{eqn:clc_fsh_dfn}
\end{eqnarray}
Of course $\clcfsh$ may depend on many factors besides size, such as
composition, charge, and relative humidity. 

Submicron particles have surface areas very large relative to their
volume (mass). 
In these conditions molecular forces such as dipole-dipole
interactions and van~der~Waals forces are strong enough that $\stcfsh
= 1$ is well-justified particles are always retained after collisions
\cite[]{Sli82}. 
There are no data suggesting the sticking (retention) efficiency 
$\stcfsh$ is other than unity for $\dmt \lesssim 20$\,\um\
Larger particles are less susceptible to Van~der~Waals forces, and
there is evidence that $\stcfsh < 1$ \cite[][p.~330]{Sli82}, although
$\stcfsh$ is still highly species-dependent.
Lacking any comprehensive data we shall assume $\stcfsh = 1$ unless
otherwise noted. 

\section{Scavenging Efficiency}\label{sxn:scv_fsh}
Having discussed the aerodynamic properties of particle-particle
interactions we are now ready to formulate the problem of scavenging
of aerosols by raindrops in a form suitable for numerical models.
The problem is to determine the rate of removal of aerosols of a given
size distribution by a precipitation rate of a given intensity and
size distribution.
We shall adopt the convention that uppercase symbols refer to the
collector species (called ``raindrops'') and lowercase symbols refer
to the collected species (called ``aerosols'').

It will prove useful to derive expressions for the precipitation
fluxes in terms of the raindrop size distribution.
The rainfall rate $\flxpcp$ may be expressed in terms of the 
microphysical distribution of raindrops of size $\dmtpcp$ and their
fallspeeds, $\vlcgrvpcp$.  
The \trmdfn{spectral precipitation intensity} $\flxvlmpcpofdmt$ 
(\mCxmSsm\ or \mxsm) is the convolution of the raindrop volume
distribution with the terminal velocity of the raindrops
\begin{eqnarray}
\flxvlmpcpofdmt & = & \frac{\mpi}{6} \vlcgrvpcp \dmtpcp^{3}
\dstnbrpcpofdmt 
\label{eqn:flx_vlm_pcp_spc_dfn}
\end{eqnarray}
The physical dimensions precipitation intensity can be confusing and 
require clarification.
The spectral volume flux $\flxvlmpcpofdmt$ (\mCxmSsm) is the volume of
rain (\mC) falling per unit horizontal area (\xmS) per unit time (\xs)
per unit raindrop size (\xm).  
Cancelling spatial dimensions in numerator and denominator leaves
the dimensions \xs, which lack any physically intuitive meaning.
Canceling like dimensions less assiduously results in \mxsm, which
is more easily interpreted as the rate of change of depth of water per
unit surface area per unit raindrop size.

Usually $\flxpcp$ is measured or quoted as a \trmdfn{precipitation
intensity} or \trmdfn{precipitation volume flux} $\flxvlmpcp$ (\mxs)
which is the total rate of increase of liquid water depth in a unit
area due to precipitation of all sizes.
The total precipitation intensity $\flxvlmpcp$ (\mCxmSs\ or \mxs) is
obtained by $\flxvlmpcpofdmt$ integrating over all raindrop sizes
\begin{eqnarray}
% NGD94 p. 2336 (11), SeP97 p. 1018 (20.55)
\flxvlmpcp & = & \frac{\mpi}{6}
\int_{0}^{\infty} \vlcgrvpcp \dmtpcp^{3} \dstnbrpcpofdmt \,\dfr\dmtpcp
\label{eqn:flx_vlm_pcp_dfn}
\end{eqnarray}
To obtain $\flxvlmpcp$ in the more commonly used units of \mmxhr,
multiply $\flxvlmpcp$ by $1000 \times 3600 = \mbox{3,600,000}$.
For comparison, typical values of precipitation intensity $\flxvlmpcp$
during drizzle and heavy rain are $0.5$ and $25$\,\mmxhr, respectively.

Precipitation intensity may also be be expressed as a mass flux.
The \trmdfn{spectral precipitation mass flux} $\flxmsspcpofdmt$ 
(\kgxmSsm) and \trmdfn{precipitation mass flux} $\flxmsspcp$ are
defined analogously to (\ref{eqn:flx_vlm_pcp_spc_dfn}) and
(\ref{eqn:flx_vlm_pcp_dfn}), respectively, but with raindrop volume
replaced by raindrop mass 
\begin{eqnarray}
\label{eqn:flx_mss_pcp_spc_dfn}
\flxmsspcpofdmt & = & \frac{\mpi \dnslqd}{6} \vlcgrvpcp \dmtpcp^{3}
\dstnbrpcpofdmt \\
\label{eqn:flx_mss_pcp_dfn}
\flxmsspcp & = & \frac{\mpi \dnslqd}{6}
\int_{0}^{\infty} \vlcgrvpcp \dmtpcp^{3} \dstnbrpcpofdmt \,\dfr\dmtpcp \\
\label{eqn:flx_mss_pcp_vlm_dfn}
& = & \flxvlmpcp \dnslqd
\end{eqnarray}
Here we have assumed that the precipitation is liquid water so that
the density of the collectors $\dnspcp = \dnslqd$.
In environments where precipitation composed of ice, snow, or other
chemical constituents, $\dnslqd$ should be replace by the appropriate
density in the definition of $\flxmsspcpofdmt$.
For the simple case of rain, however, 
$\flxmsspcp \approx 1000 \flxvlmpcp$.  
Thus to convert $\flxmsspcp$ to $\flxvlmpcp$ in \mmxhr, multiply
$\flxmsspcp$ by $3600$.  
Although $\flxpcp$ is relatively easy to measure at the surface,
its vertical distribution is not.

Precipitation scavenging of aerosol takes place in the atmosphere
below clouds when falling raindrops collide with and collect aerosols.
Consider the physics of precipitation scavenging from the point of
view of a single rain droplet of diameter $\dmtpcp$ falling with speed
$\vlcgrvpcp$. 
During its descent this raindrop sweeps out a volume of 
$\mpi \dmtpcp^{2} \vlcgrvpcp / 4$ 
per unit time.
Any aerosol partially located in this volume will also be collected,
so the geometric collection volume of the raindrop is actually
$\mpi ( \dmtpcp + \dmtprt )^{2} \vlcgrvpcp / 4$ 
per unit time.
Finally, the motion of the aerosol reduces the \trmdfn{collection
volume} $\vlmclc$ (\mCxs) of a raindrop to a cylinder expanding at the
speed of the relative motion, 
\begin{eqnarray}
% SeP97 p. 1016
\vlmclc = \mpi ( \dmtpcp + \dmtprt )^{2} (\vlcgrvpcp - \vlcgrvprt) / 4
\label{eqn:clc_vlm_gmt}
\end{eqnarray}
The number of aerosols of size between $\dmtprt$ and $\dmtprt +
\dfr\dmtprt$ in a given volume of space is, by definition, 
$\dstnbr(\dmtprt) \,\dfr\dmtprt$.
The number of aerosol particles a raindrop of size $\dmtpcp$
encounters per unit time in its geometric path, $\nbrscvgmt$, is
\begin{equation}
% SeP97 p. 1016
\nbrscvgmtofpcpprt = \frac{\mpi}{4} ( \dmtpcp + \dmtprt )^{2} (\vlcgrvpcp -
\vlcgrvprt) \dstnbr(\dmtprt) \,\dfr\dmtprt
\label{eqn:nbr_scv_gmt_dfn}
\end{equation}
Assuming a perfect sticking efficiency $\stcfsh = 1$, $\nbrscvgmt$ is
the rate of aerosol collection (\nxmCs) in the geometric limit.
The rate of mass collection (\kgxmCs) in the geometric limit would be
obtained by simply replacing $\dstnbr$ with $\dstmss$ in
(\ref{eqn:nbr_scv_gmt_dfn}).  

The great complication to the geometric formulation of aerosol number
and mass scavenging is due to the aerodynamic interaction of the
raindrop and aerosol described above.
These interactions were combined into an aerodynamic correction factor
called the \trmidx{collision efficiency}, $\cllfshofpcpprt$.
We may now understand $\cllfshofpcpprt$ to be the fraction of
particles of size $\dmtprt$ contained within the geometric collision
volume of raindrops of size $\dmtpcp$ (\ref{eqn:nbr_scv_gmt_dfn}) that
are actually encountered.  
With this aerodynamic correction, the definition of the actual rate of
collection of aerosol by a raindrop, $\nbrscvaer$, is completed
\begin{equation}
% SeP97 p. 1016
\nbrscvaerofpcpprt = \frac{\mpi}{4} ( \dmtpcp + \dmtprt )^{2} (\vlcgrvpcp -
\vlcgrvprt) \cllfshofpcpprt \dstnbr(\dmtprt) \,\dfr\dmtprt
\label{eqn:nbr_scv_aer_dfn}
\end{equation}
Note that in the following we use $\cllfshofpcpprt$ rather than
$\clcfshofpcpprt$ in conformance with convention. 
However, if the sticking efficiency $\stcfsh \ne 1$ then
$\cllfsh$ must be replaced with $\clcfsh$ (\ref{eqn:clc_fsh_dfn}).

$\nbrscvaerofpcpprt$ applies to a single raindrop and aerosol size.
The total rate of collection of aerosol particles of size $\dmtprt$ is
obtained by integrating (\ref{eqn:nbr_scv_aer_dfn}) over the number
distribution of the raindrops $\dstnbrpcpofdmt$,
\begin{equation}
% SeP97 p. 1017 (20.50)
\nbrscvaerofprt = \dstnbr(\dmtprt) \,\dfr\dmtprt
\int_{0}^{\infty} \frac{\mpi}{4} 
( \dmtpcp + \dmtprt )^{2} (\vlcgrvpcp - \vlcgrvprt) 
\cllfshofpcpprt \dstnbrpcpofdmt \,\dfr\dmtpcp
\label{eqn:nbr_scv_aer_ntg_dfn}
\end{equation}
Fortunately the complexity of (\ref{eqn:nbr_scv_aer_ntg_dfn}) can be 
reduced for typical atmospheric atmospheric conditions.
The fall speed of rain ($\dmtpcp > 100$\,\um) greatly exceeds 
the fall speed of any long-lived aerosol ($\dmtprt < 20$\,\um), so
$\vlcgrvpcp \gg \vlcgrvprt$ (\ref{eqn:vlc_grv_dfn}) and
$( \dmtpcp + \dmtprt )^{2} \approx \dmtpcp^{2}$.
These approximations are not necessarily valid for drizzle.

Using these approximations, in conditions of steady precipitation the
rates of number and mass removal of aerosol particles of size
$\dmtprt$ due to below cloud precipitation scavenging are
\begin{eqnarray}
% SeP97 p. 1017 (20.50)
\nbrscvaerofprt & \approx & \dstnbr(\dmtprt) \,\dfr\dmtprt
\int_{0}^{\infty} \frac{\mpi}{4} 
\dmtpcp^{2} \vlcgrvpcp
\cllfshofpcpprt \dstnbrpcpofdmt \,\dfr\dmtpcp \nonumber \\
\mssscvaerofprt & \approx & \dstmss(\dmtprt) \,\dfr\dmtprt
\int_{0}^{\infty} \frac{\mpi}{4} 
\dmtpcp^{2} \vlcgrvpcp
\cllfshofpcpprt \dstnbrpcpofdmt \,\dfr\dmtpcp
\label{eqn:mss_scv_aer_dfn}
\end{eqnarray}

With the great uncertainties presented above firmly in mind, we are
ready to define the \trmdfn{scavenging coefficient}, 
$\scvcff$ as the first order removal rate of aerosol mass and number
concentration.
Noting that $\nbrscvaer = \dfr\dstnbr/\dfr\tm$ and
$\mssscvaer = \dfr\dstmss/\dfr\tm$, (\ref{eqn:mss_scv_aer_dfn})
may be rewritten as first order loss equations in $\dstnbrofprt$ and
$\dstmssofprt$, respectively 
\begin{eqnarray}
% SeP97 p. 1017, (20.53)
\label{eqn:dst_scv_nbr_dfn}
\frac{\dfr\dstnbrofprt}{\dfr\tm} & = & 
-\scvcffofprt \dstnbrofprt \,\dfr\dmtprt \\
\label{eqn:dst_scv_mss_dfn}
\frac{\dfr\dstmssofprt}{\dfr\tm} & = & 
-\scvcffofprt \dstmssofprt \,\dfr\dmtprt
\end{eqnarray}
where the negative sign indicates removal of aerosol number and mass
from the environment and the positive definite scavenging coefficient
$\scvcffofprt$ (\xs) is  
\begin{eqnarray}
% NGD94 p. 2338 (19), SeP97 p. 1017 (20.53)
\scvcffofprt & = &
\int_{0}^{\infty} \frac{\mpi}{4} 
( \dmtpcp + \dmtprt)^{2} ( \vlcgrvpcp - \vlcgrvprt )
\cllfshofpcpprt \dstnbrpcpofdmt \,\dfr\dmtpcp \\
& \approx &
\int_{0}^{\infty} \frac{\mpi}{4} 
\dmtpcp^{2} \vlcgrvpcp
\cllfshofpcpprt \dstnbrpcpofdmt \,\dfr\dmtpcp
\label{eqn:scv_cff_dfn}
\end{eqnarray}
Interestingly, $\scvcffofprt$ applies to both  
number and mass concentrations for any aerosol size.
One difficulty in parameterizing (\ref{eqn:scv_cff_dfn}) for large
scale atmospheric models is that $\scvcffofprt$ is very sensitive to
the raindrop distribution $\dstnbrpcpofdmt$.

All the information about the raindrop size and rainfall intensity
in $\scvcffofprt$ is contained in $\vlcgrvpcp$ and $\dstnbrpcpofdmt$ 
(\ref{eqn:flx_mss_pcp_dfn}). 
It is useful to recast $\scvcffofprt$ explicitly in terms of
$\flxpcp$, which is often measured (or predicted).
\begin{eqnarray}
% NGD94 p. 2338 (19), SeP97 p. 1017 (20.53)
\scvcffofprt & \approx &
\int_{0}^{\infty} \frac{\mpi}{4} 
\dmtpcp^{2} \vlcgrvpcp
\cllfshofpcpprt \dstnbrpcpofdmt \,\dfr\dmtpcp
\label{eqn:scv_cff_pcp_dfn}
\end{eqnarray}

Suitable approximations for  may considerably simplify
the problem of parameterizing $\scvcffofprt$ in large scale models.
Using the approximation for raindrop fall speed $\vlcgrvpcp$
(\ref{eqn:vlc_grv_pcp_dfn}), yields
\begin{eqnarray}
% NGD94 p. 2338 (19), SeP97 p. 1017 (20.53)
\scvcffofprt 
& \approx &
\int_{0}^{\infty} \frac{\mpi}{4} 
5.45 \left( \frac{\dmt \dnslqd}{\dnsatm } \right)^{1/2}
\dmtpcp^{2} \cllfshofpcpprt \dstnbrpcpofdmt \,\dfr\dmtpcp \nonumber \\
& \approx &
4.28 \left( \frac{\dnslqd}{\dnsatm } \right)^{1/2}
\int_{0}^{\infty} \dmtpcp^{5/2} \cllfshofpcpprt \dstnbrpcpofdmt \,\dfr\dmtpcp
\label{eqn:scv_cff_apx_dfn}
\end{eqnarray}
Assuming analytically integrable distributions (e.g., lognormal) for 
$\dstnbrpcpofdmt$, the principal difficulty in applying
(\ref{eqn:scv_cff_apx_dfn}) is the complicated form of
$\cllfshofpcpprt$ (\ref{eqn:cll_fsh_dfn_2}).

Solving (\ref{eqn:dst_scv_nbr_dfn})--(\ref{eqn:dst_scv_mss_dfn})
for the time dependent number or mass concentration yields
\begin{eqnarray}
\label{eqn:nbr_tm_sln}
\dstnbr( \dmtprt , \tm + \dlttm ) & = & 
\dstnbr( \dmtprt , \tm ) \me^{- \scvcffofprt \dlttm} \\
\label{eqn:mss_tm_sln}
\dstmss( \dmtprt , \tm + \dlttm ) & = & 
\dstmss( \dmtprt , \tm ) \me^{- \scvcffofprt \dlttm}
\end{eqnarray}
The $\eee$-folding timescale for wet scavenging of particle number and
mass is $[\scvcffofprt]^{-1}$.
Figure~\ref{fgr:scv_cff} shows the simulated $\scvcffofprt$ for
monodisperse raindrops ($\dmtpcp = 400$\,\um) collecting aerosol
during a mild precipitation event with intensity 
$\flxvlmpcp = 1.0$\,\mmxhr.
\begin{figure*}
\begin{center}
% Figure created by mie.pro:psd_bch()
\includegraphics[width=0.8\hsize]{/Users/zender/data/fgr/aer/psd_scv_cff}\vfill
\end{center}
\caption[Scavenging coefficient]{
Scavenging coefficient $\scvcff$\,\xs\ as a function of aerosol size  
$\dmtprt$\,\um\ for a monodisperse raindrop size.
\label{fgr:scv_cff}}
\end{figure*}
The discontinuity between the interception and impaction regimes is 
located at about $\dmtprt = 2$\,\um.
On either side of this discontinuity removal timescales change from
about 1\,\hr\ to a few days.

To determine the total removal of aerosol number and size by
raindrops, we must integrate the scavenging rate of each aerosol size 
(\ref{eqn:dst_scv_nbr_dfn})--(\ref{eqn:dst_scv_mss_dfn})
over the aerosol size distribution $\dstnbrofprt$
\begin{eqnarray}
% SeP97 p. 1018, (20.54)
\begin{array}{ >{\displaystyle}r<{} >{\displaystyle}l<{} >{\displaystyle}l<{} >{\displaystyle}r<{} >{\displaystyle}l<{} }
\label{eqn:scv_nbr_dfn}
\frac{\dfr\nbrprt}{\dfr\tm} & = & 
\frac{d}{\dfr\tm} \int_{0}^{\infty} \dstnbrofprt \,\dfr\dmtprt & = & 
- \int_{0}^{\infty} \scvcffofprt \dstnbrofprt \,\dfr\dmtprt \\[1.5ex]
\label{eqn:scv_mss_dfn}
\frac{\dfr\mssprt}{\dfr\tm} & = & 
\frac{d}{\dfr\tm} \int_{0}^{\infty} \dstmssofprt \,\dfr\dmtprt & = & 
- \int_{0}^{\infty} \scvcffofprt \dstmssofprt \,\dfr\dmtprt
\end{array}
\end{eqnarray}

Inserting 
$\dstmssofprt = \frac{\mpi}{6} \dmtprt^{3} \dnsprt \dstnbrofprt$
into (\ref{eqn:scv_mss_dfn}) we obtain
\begin{eqnarray}
\frac{\dfr\mssprt}{\dfr\tm}
& = & - \int_{0}^{\infty} 
\scvcffofprt \frac{\mpi}{6} \dmtprt^{3} \dnsprt \dstnbrofprt \,\dfr\dmtprt 
\nonumber \\
& = & - \frac{\mpi \dnsprt}{6} \int_{0}^{\infty} 
\scvcffofprt \dmtprt^{3} \dstnbrofprt \,\dfr\dmtprt \nonumber \\
& = & - \scvcffmssavg \frac{\mpi \dnsprt}{6} \int_{0}^{\infty} 
\dmtprt^{3} \dstnbrofprt \,\dfr\dmtprt \nonumber \\
& = & - \scvcffmssavg \mssprt
\label{eqn:dst_scv_mss_ttl_dfn}
\end{eqnarray}
where in the third step we removed the explicit dependence on
$\scvcffofprt$ in the integrand by defining a new, integrated
quantity, $\scvcffmssavg$, the \trmdfn{mass-mean scavenging
coefficient}: 
\begin{eqnarray}
\scvcffmssavg & = & \frac{
\frac{\mpi \dnsprt}{6}
\int_{0}^{\infty} \scvcffofprt \dmtprt^{3} \dstnbrofprt \,\dfr\dmtprt 
}{
\frac{\mpi \dnsprt}{6}
\int_{0}^{\infty} \dmtprt^{3} \dstnbrofprt \,\dfr\dmtprt }
\nonumber \\
& = & \frac{
\int_{0}^{\infty} \scvcffofprt \dmtprt^{3} \dstnbrofprt \,\dfr\dmtprt 
}{ \int_{0}^{\infty} \dmtprt^{3} \dstnbrofprt \,\dfr\dmtprt }
\label{eqn:scv_cff_mss_avg_dfn}
\end{eqnarray}
The term \trmdfn{mass-weighted scavenging coefficient} is a better
technical description of $\scvcffmssavg$, but is not widely used.
$\scvcffmssavg$, which does not explicitly depend on particle size,
is the first order rate coefficient for aerosol mass removal by
precipitation. 
The solution of (\ref{eqn:dst_scv_mss_ttl_dfn}) for time dependent
aerosol mass concentration is
\begin{eqnarray}
\mssprt ( \tm ) & = & \mssprt(\tm = 0) \me^{- \scvcffmssavg \tm}
\label{eqn:mss_ttl_sln}
\end{eqnarray}
The timescale for mass depletion of the entire aerosol distribution by
wet scavenging is thus $\scvcffmssavg^{-1}$. 

The \trmdfn{number-mean scavenging coefficient} $\scvcffnbravg$
(\xs) is defined by an analogous procedure.
Beginning with (\ref{eqn:scv_nbr_dfn}) we have
\begin{eqnarray}
\frac{\dfr\nbrprt}{\dfr\tm}
& = & - \int_{0}^{\infty} \scvcffofprt \dstnbrofprt \,\dfr\dmtprt \nonumber \\
& = & - \scvcffnbravg \nbrprt
\label{eqn:dst_scv_nbr_ttl_dfn}
\end{eqnarray}
where
\begin{eqnarray}
\label{eqn:scv_cff_nbr_avg_dfn}
\scvcffnbravg & = & \frac{
\int_{0}^{\infty} \scvcffofprt \dstnbrofprt \,\dfr\dmtprt 
}{
\int_{0}^{\infty} \dstnbrofprt \,\dfr\dmtprt } \\
& = & \frac{
\int_{0}^{\infty} 
[ \int_{0}^{\infty} \frac{\mpi}{4} \dmtpcp^{2} \vlcgrvpcp \dstnbrpcpofdmt 
\cllfshofpcpprt \,\dfr\dmtpcp ] \, \dstnbrofprt \,\dfr\dmtprt 
}{
\int_{0}^{\infty} \dstnbrofprt \,\dfr\dmtprt } \nonumber \\
& = & \frac{
\frac{2 \dmtpcp}{3} \frac{\mpi}{4} \int_{0}^{\infty} 
[ \int_{0}^{\infty} \dmtpcp^{2} \vlcgrvpcp \dstnbrpcpofdmt 
\cllfshofpcpprt \,\dfr\dmtpcp ] \, \dstnbrofprt \,\dfr\dmtprt 
}{
\frac{2 \dmtpcp}{3} 
\int_{0}^{\infty} \dstnbrofprt \,\dfr\dmtprt } \nonumber \\
& = & \frac{
\frac{\mpi}{6} \int_{0}^{\infty} \dmtpcp^{3} \vlcgrvpcp \dstnbrpcpofdmt \,\dfr\dmtpcp
\int_{0}^{\infty} \cllfshofpcpprt \dstnbrofprt \,\dfr\dmtprt 
}{
\frac{2 \dmtpcp}{3} 
\int_{0}^{\infty} \dstnbrofprt \,\dfr\dmtprt } \nonumber \\
\label{eqn:scv_cff_nbr_avg_dfn_ilg}
& = & \frac{
3 \flxvlmpcp \int_{0}^{\infty} \cllfshofpcpprt \dstnbrofprt \,\dfr\dmtprt 
}{
2 \dmtpcp \int_{0}^{\infty} \dstnbrofprt \,\dfr\dmtprt }
\end{eqnarray}
where we have followed \cite{SeP97} in changing the order of
integration to introduce $\flxvlmpcp$ (\ref{eqn:flx_vlm_pcp_dfn}) into 
the final expression.   
This order change is only legal when raindrops are monodisperse
because $\dmtpcp$ is present in $\cllfshofpcpprt$, see, e.g.,
(\ref{eqn:cll_fsh_ntc_dfn}).  
The solution of (\ref{eqn:dst_scv_nbr_ttl_dfn}) for time dependent
aerosol mass concentration is
\begin{eqnarray}
\nbrprt ( \tm ) & = & \nbrprt(\tm = 0) \me^{- \scvcffnbravg \tm}
\label{eqn:nbr_ttl_sln}
\end{eqnarray}
The timescale for number depletion of the entire aerosol distribution
by wet scavenging is thus $\scvcffnbravg^{-1}$. 

The \trmdfn{scavenging efficiency} $\scvfshofprt$
(dimensionless) is the fractional change in aerosol number or mass
concentration  during a time increment $\dlttm$.
Applying this definition to either time-dependent concentration
(\ref{eqn:nbr_tm_sln}) or (\ref{eqn:mss_tm_sln}) yields the same
result, 
\begin{eqnarray}
% NGD94 p. 2338 (22), GBS98 p. 11443 (A2)
\scvfshofprt & = & 
\frac{\dstmss ( \dmtprt, \tm + \dlttm ) - \dstmssofprttm
}{ \dstmssofprttm } \nonumber \\
& = & 
\frac{\dstmssofprttm \me^{- \scvcffofprt \dlttm } - \dstmssofprttm
}{ \dstmssofprttm } \nonumber \\
& = & 
\frac{\dstmssofprttm ( \me^{- \scvcffofprt \dlttm } - 1 )
}{ \dstmssofprttm } \nonumber \\
& = & \me^{- \scvcffofprt \dlttm } - 1
\label{eqn:scv_fsh_dfn}
\end{eqnarray}
Thus $\scvfshofprt$ depends explicitly only on particle size.
The sign convention used in (\ref{eqn:scv_fsh_dfn}) ensures that
$\scvfshofprt$ is negative definite. 

The \trmdfn{mass-mean scavenging efficiency} $\scvfshmssavg$
applies to the entire mass distribution. 
The derivation of $\scvfshmssavg$ is analogous to
(\ref{eqn:scv_fsh_dfn}), but begins with the time-dependent 
solution to total aerosol mass concentration (\ref{eqn:mss_ttl_sln}), 
\begin{eqnarray}
% NGD94 p. 2338 (22), GBS98 p. 11443 (A2)
\scvfshmssavg & = & 
\frac{\mssprt ( \tm ) - \mssprt ( \tm + \dlttm )}{\mssprt ( \tm ) }
\nonumber \\
& = & 
\frac{\mssprt( \tm ) - \mssprt( \tm ) \me^{- \scvcffmssavg \dlttm } 
}{ \mssprt ( \tm ) } \nonumber \\
& = & 
\frac{\mssprt( \tm ) ( 1 - \me^{- \scvcffmssavg \dlttm } )
}{ \mssprt ( \tm ) } \nonumber \\
& = & 1 - \me^{- \scvcffmssavg \dlttm }
\label{eqn:scv_fsh_mss_avg_dfn}
\end{eqnarray}
The derivation of the \trmdfn{number-mean scavenging efficiency}
$\scvfshnbravg$ proceeds analogously from (\ref{eqn:nbr_ttl_sln}) and
yields 
\begin{eqnarray}
\scvfshnbravg & = & 1 - \me^{- \scvcffnbravg \dlttm }
\label{eqn:scv_fsh_nbr_avg_dfn}
\end{eqnarray}

Assuming precipitation falls at a uniform rate $\flxpcp$ during a time
period $\dlttm$, we may redefine the scavenging efficiency in terms of
rainfall depth rather than elapsed time.
Let $\flxvlmpcpdlt$ (m) be the depth of precipitation accumulated in
time $\dlttm$, i.e., 
\begin{eqnarray}
\flxvlmpcpdlt & = & \flxvlmpcp \dlttm
\label{eqn:flx_vlm_pcp_dlt_dfn}
\end{eqnarray}
To complement $\flxvlmpcpdlt$ we define
precipitation-volume-normalized scavenging coefficients whose
dimensions are \xm:  
% fxm: need convention to distinguish between volume-normalized and
% mass-normalized scavenging coefficients
\begin{eqnarray}
% GBS98 p. 11443 (A2), NGD94 p. 
\scvcffbarofprt & = & \scvcffofprt / \flxvlmpcp \\
\scvcffmssavgbar & = & \scvcffmssavg / \flxvlmpcp \\
\scvcffnbravgbar & = & \scvcffnbravg / \flxvlmpcp
\label{eqn:scv_cff_pcp_nrm_vlm_dfn}
\end{eqnarray}
and precipitation-mass-normalized scavenging coefficients whose
dimensions are \mSxkg:  
\begin{eqnarray}
% GBS98 p. 11443 (A2), NGD94 p. 
\scvcffbarofprt & = & \scvcffofprt / \flxmsspcp \\
\scvcffmssavgbar & = & \scvcffmssavg / \flxmsspcp \\
\scvcffnbravgbar & = & \scvcffnbravg / \flxmsspcp
\label{eqn:scv_cff_pcp_nrm_mss_dfn}
\end{eqnarray}
Note that $\scvcffbar$ expressed in \xmm\
(\ref{eqn:scv_cff_pcp_nrm_vlm_dfn}) is numerically equal to
$\scvcffbar$ expressed in \mSxkg\ (\ref{eqn:scv_cff_pcp_nrm_mss_dfn}) 
as long as the hydrometeor density is taken to be 1000\,\kgxmC, which
is accurate for rainwater.

Figure~\ref{fgr:scv_cff_mss_nrm} shows the simulated $\scvcffbarofprt$ 
for various lognormal size distributions of raindrops.
The number-median raindrop size and geometric standard deviation of 
lognormal raindrop distributions used to simulate convective and
stratiform rain events are $\dmtpcpnma = 1000$\,\um, $\gsdpcp = 1.86$
and $\dmtpcpnma = 400$\,\um, $\gsdpcp = 1.86$, respectively
\cite[]{NGD94}. 
\begin{figure*}
\begin{center}
% Figure created by mie.pro:scv_gph
% cp -f ${DATA}/ps/scv_gph.eps ${DATA}/fgr/aer
\includegraphics[width=0.8\hsize]{/Users/zender/data/fgr/aer/scv_gph}\vfill
\end{center}
\caption[Scavenging coefficient for polydisperse raindrops]{
Precipitation-normalized scavenging coefficient 
$\scvcffbar$\,\xmm\ (same as \mSxkg) as a function of
aerosol size $\dmtprt$ (\um) for polydisperse lognormal raindrop size  
distributions.  
Filled points are mass-mean, precipitation-normalized scavenging
coefficients computed for a lognormal size distribution typical of
far-traveled mineral dust aerosol.
\label{fgr:scv_cff_mss_nrm}}
\end{figure*}
These normalized scavenging coefficients can be scaled to any
precipitation flux simply by multiplying by $\flxmsspcp$
(\ref{eqn:scv_cff_pcp_nrm_mss_dfn}). 
Polydisperse raindrops do not greatly reduce the scavenging
discontinuity between interception and impaction regimes seen in
monodisperse simulations (cf. Figure~\ref{fgr:cll_fsh}).
As discussed in \S\ref{sxn:cll_fsh}, aerosol density does not play 
a role in the scavenging efficiency except in the impaction process. 
Thus Figure~\ref{fgr:scv_cff_mss_nrm} is virtually identical for all
aerosol compositions (i.e., sulfate, dust, carbon, and sea salt) for
sizes $\dmtprt \lesssim 2$\,\um.  

Figure~\ref{fgr:scv_cff_mss_nrm} also shows that great care must be
taken in discretizing the continuous equations of aerosol evolution.
Particles susceptible to inertial impaction ($\dmtprt \gtrsim 2$\,\um) 
should not be placed in the same bin as smaller particles.
Given the uncertainties involved in predicting or measuring the
raindrop size distribution $\dstnbrpcpofdmt$, \cite{Sli84} recommends
using the mass-weighted mean raindrop size $\dmtvwrpcp$ to determine
the scavenging rates. 
\cite{Mas71} found that the fourth power of $\dmtvwrpcp$ varied
linearly with the precipitation intensity:
\begin{eqnarray}
% Sli84 p. 476 (11.33) as suggested by Mason (1971)
\dmtvwrpcp\ \mbox{mm} & = &  
0.70\ \mbox{mm} 
\left( \frac{\flxvlmpcp}{1\ \mbox{\mmxhr} } \right)^{1/4} 
\nonumber \\
\dmtvwrpcp & = & 
700 \times 10^{-6} ( 3.6 \times 10^{6} \flxvlmpcp )^{1/4} \nonumber \\
& = & 0.0305 ( \flxvlmpcp )^{1/4} \nonumber \\
\dmtvwrpcp & = & 
700 \times 10^{-6} ( 3.6 \times 10^{3} \flxmsspcp )^{1/4} \nonumber \\
& = & 5.42 \times 10^{-3} ( \flxmsspcp )^{1/4}
\label{eqn:dmt_vwr_pcp_Mas71}
\end{eqnarray}
where all equations except for the first are in SI units.
According to (\ref{eqn:dmt_vwr_pcp_Mas71}), $\dmtvwrpcp = 700$\,\um\ 
when $\flxvlmpcp = 1$\,\mmxhr.

\section[Precipitation-Aerosol Interactions]{Precipitation-Aerosol Interactions}\label{sxn:pcp_aer}\label{sxn:ndr}
The microphysics of precipitation are simultaneously one of the most
important and least understood aspects of the climate system.
Precipitation prediction (and, eventually, modification) is the 
``holy grail'' of weather forecasting.
Aerosols interact with precipitation formation by serving as
additional nucleation sites, and thus adjusting the overall
competition for water vapor in the cloud.
Hygroscopic dust may serve as \trmidx{CCN}, and even hygrophobic dust
with appropriate crystal structure may serve as~\trmidx{IN}.
Relatively coarse-sized dust ($\dmt \gtrsim 5$\,\um) which nucleate
immediately create large cloud droplets.
These nuclei are called \trmidx{giant cloud condensation nuclei}, or
\trmidx{GCCN}. 
% Paris: Levin talk
\nmidx[Levin, Zev]{Zev Levin} finds that about 30\,\GCCNxcmC\ are
enough to double integrated precipitation in warm clouds.
This holds unless IN are present.
% San Francisco: Mohler/Schnaiter/DeMott poster p. 172
\nmidx[M\"{o}hler, O.]{O.~M\"{o}hler} found that deposition freezing
on Saharan dust commences at $1.1 \lesssim \RHice \lesssim 1.2$.
This is far lower than the $\RHice$ required for deposition freezing
on carbonaceous or liquid aerosol.
\cite{DSP03} found in that dust aerosols act as
Ice Nuclei more efficiently than any other particles above
homogeneous freezing temperatures. 
They found IN concentrations exceeded $1$\,\xcmC\ in aerosol layers
containing sub-micron dust particles.
\cite{SDP03} show that dust particles glaciated a supercooled
($-8.8 < \tpt < -5.2$\dgrc) altocumulus cloud during
\trmidx{CRYSTAL-FACE}.
This causes an indirect effect on climate through cloud structure.
\cite{YTT04} found mineral dust aerosol was included in more than 50\%
of droplets artificially formed in a mineshaft.
\cite{DuV04} show that the \trmidx{Saharan Air Layer} (\trmidx{SAL}),
and the dust it often carries, sometimes plays an important role in
suppression tropical cyclone activity in the North Atlantic. 

\section{Hygroscopic Growth}\label{sxn:scv_hyg}
Hygroscopic growth of aerosol must be considered in the formulation of
precipitation scavenging.
Swelling and coating of aerosol can increase the number-median
diameter $\dmtnma$ of an aerosol distribution by a factor of two or
more. 
This increase in size will change the mean scavenging properties of
the aerosol distribution $\scvcffmssavg$. 
If the aerosol mass is largely contained within the accumulation mode
then hygroscopic growth will increase $\scvcffmssavg$.
If the aerosol mass is largely contained in sizes smaller than the
accumulation mode then hygroscopic growth could decrease
$\scvcffmssavg$. 

\symdfn{\scvcffofprt}{\xs}{Scavenging coefficient}{First order rate of
mass and number removal of aerosol of size $\dmtprt$}
\symdfn{\scvcffmssavg}{\xs}{Mass-mean scavenging coefficient}{First
order rate coefficient for mass removal by wet scavenging}

\chapter[Paleoclimate Aerosols]{Paleoclimate Aerosols}\label{sxn:pcl}

\cite{AdF95} and \cite{Cro95} present vegetation reconstructions of
the LGM climate.
\cite{PeM95} present an energy-balance/ice-sheet model which accounts
for dust triggered ice-albedo feedbacks that may play a role in
deglaciations. 
\cite{ORL96} use a modeled dust distribution to show that 
the longwave radiative effects of dust may have induced significant
regional warming during the \trmidx{LGM}.
\cite{CrB97} study the effects of vegetation on LGM climate (without
aerosols). 
\cite{RSK97} describe the eleven year cycle of dust found in the
Greenland GISP2 ice core.
\cite{AAG98} describe the behavior of atmospheric dust based on
glacial and interglacial records.
\cite{HKR01} review the role of dust in present day and LGM climate
changes. 
Using extensive records \cite[][]{KoH01} together with models, they
predict the impact of dust on future climate.
\cite{ROR01} assembled hypothesized aerosol and solar forcing time
series, including mineral dust, from 1500--present.
\cite{AWL00} 
\cite{BKQ03} use a global coupled climate ocean biogeochemistry model
constrained by \trmidx{DIRTMAP} data to estimate the
glacial-interglacial effect of dust-supplied \Fe\ on atmospheric  
$\prsprtCOd$ is less than 30\,\ppm.
\cite{CRK03} model the radiative forcing of dust during the last
glacial conditions.
\cite{ACA05} summarize oceanic and terrestrial sediment records which 
demonstrate Global Iron Connections (GICs) on multiple timescales over
the last 130\,\kya.
The evidence they summarize is consistent with the Iron Hypotheses
in all ocean basins except the South Pacific.
Using a simple technique they estimate Asian dust is responsible for
$\Delta\prsprtCOd \sim 4\mbox{--}9$\,\ppm, or one-tenth to
one-third of the total dust effect estimated by \cite{BKQ03}.
\cite{WFF06} show 750,000\m,\yr\ timeseries of natural aerosols from
the EPICA ice core of Dome~C in Antarctica.
\cite{BRL06} apportion aerosol in Dome~C into continental and sea-salt
contributions. 
\cite{MAB07} show that dust deposition to Antarctica from Patagonia
over the past few hundred years is highly correlated with Patagonian
climate, which includes the signal of global warming.

\chapter{Thermodynamics of Gases}\label{sxn:tdy}

\section{Temperature}\label{sxn:tpt}
The kinetic theory of temperature links relates the macroscopic
temperature to statistical properties of the molecular ensemble
comprising the parcel.
Statistical mechanics \cite[e.g.,][p.~8]{Tso02} tells us that the mean 
kinetic energy of molecules with three spatial degrees of freedom is
\begin{eqnarray}
% Tso02 p. 8 (2.9)
\langle \frac{\bar{\mssmlc\mlcspd^{2}}}{2} \rangle & = & \frac{3}{2} \bltcst\tpt
\label{eqn:nrg_knt_dfn}
\end{eqnarray}
where $\bltcst$ is \trmidx{Boltzmann's constant}.
Inverting this definition for temperature yields
\begin{eqnarray}
% Tso02 p. 8 (2.9)
\tpt & = & \frac{2}{3\bltcst\tpt} \langle \frac{\bar{\mssmlc\mlcspd^{2}}}{2} \rangle \\
& = & \frac{1}{3\bltcst\tpt} \langle \bar{\mssmlc\mlcspd^{2}} \rangle 
\label{eqn:tpt_knt_dfn}
\end{eqnarray}

\section{Ideal Gas Law}\label{sxn:igl}
$\prsprtwtrsatinf(\tpt)$ is given in Table~\ref{tbl:sat_vap_prs_wtr}.
The \trmidx{Ideal Gas Law} (IGL) has many forms.
For purposes of illustration, we take the fundamental IGL form to be
in terms of
\trmidx{pressure} $\prs$\,[\Pa], 
\trmidx{volume} $\vlm$\,[\mC],
\trmidx{mole number} $\molnbr$\,[\mol],
\trmidx{temperature} $\tpt$\,[\K],
and \trmidx{universal gas constant} $\gascstunv$\,[\jxmolK]:
\begin{eqnarray}
\prs \vlm & = & \molnbr \gascstunv \tpt
\label{eqn:igl_dfn}
\end{eqnarray}
The \trmidx{specific gas constant} $\gascst$\,[\jxkgK] is the ratio of
the universal gas constant to the \trmidx{mean molecular weight}
$\mmw$\,[\kgxmol] of the gas 
\begin{eqnarray}
\gascst & \equiv & \gascstunv / \mmw
\label{eqn:gas_cst_spc_dfn}
\end{eqnarray}
We may re-cast the IGL (\ref{eqn:igl_dfn}) in terms of gas-specific 
properties using (\ref{eqn:gas_cst_spc_dfn}) 
\begin{eqnarray}
\prs \vlm & = & \molnbr \mmw \gascst \tpt \nonumber \\
& = & \mss \gascst \tpt
\label{eqn:igl_2_dfn}
\end{eqnarray}
where we have introduced the \trmidx{mass} $\mss$\,[\kg] of gas 
\begin{eqnarray}
\mss & \equiv & \mmw \molnbr
\label{eqn:mss_dfn}
\end{eqnarray}

Introducing the \trmidx{specific volume} $\vlmspc$\,[\mCxkg] of gas
\begin{eqnarray}
\vlmspc & \equiv & \vlm / \mss
\label{eqn:vlm_spc_dfn}
\end{eqnarray}
leads to
\begin{eqnarray}
\prs \vlmspc & = & \gascst \tpt
\label{eqn:igl_3_dfn}
\end{eqnarray}
Since the \trmidx{density} $\dns$\,[\kgxmC] is $\vlmspc^{-1}$
\begin{eqnarray}
\prs & = & \dns \gascst \tpt \nonumber \\
\label{eqn:igl_4_dfn}
\end{eqnarray}
The explicit presence of density in the equation of hydrostatic
equilibrium (\ref{eqn:hyd_eqm}) makes Equation~(\ref{eqn:igl_4_dfn}) 
a particularly useful form of the IGL.

\subsection{Change in Saturation with Temperature}\label{sxn:dedT}
Applying (\ref{eqn:igl_dfn}) to water vapor leads to
\begin{eqnarray}
\prsprtwtr & = & \dnsvpr \gascstvpr \tpt \\
\label{eqn:igl_vpr_dfn}
\dnsvpr & = & \prsprtwtr / (\gascstvpr \tpt) \\
\dnsvprsat(\tpt) & = & \prsprtwtrsat(\tpt) / (\gascstvpr \tpt) 
\label{eqn:dns_vpr_sat_dfn}
\end{eqnarray}
Thus $\dnsvprsat$, like $\prsprtwtrsat$, is a function of $\tpt$ only.  
The rate of change of saturated properties (vapor pressure, density, 
mixing ratio) with respect to temperature often appears in cloud and
aerosol microphysics.
Measurements of $\dfr\prsprtwtrsat/\dfr\tpt$ are available in a
similar form to $\prsprtwtrsat$ (Table~\ref{tbl:sat_vap_prs_wtr}).  
These independent measurements result in a parameterization that is
slightly different from a direct algebraic derivative of
Table~\ref{tbl:sat_vap_prs_wtr}. 
Most commonly gradients of other saturated properties are expressed in
terms of $\prsprtwtrsat$ and $\dfr\prsprtwtrsat/\dfr\tpt$. 
Differentiating the \trmidx{Ideal Gas Law} for vapor
(\ref{eqn:igl_vpr_dfn}) shows that
\begin{eqnarray}
\frac{\dfr\dnsvprsat}{\dfr\tpt} & = & \frac{d}{\dfr\tpt}
\frac{\prsprtwtrsat}{\gascstvpr \tpt} \nonumber \\
& = & -\gascstvpr \prsprtwtrsat + \frac{1}{\gascstvpr \tpt} 
\frac{\dfr\prsprtwtrsat}{\dfr\tpt}
\label{eqn:ddnsvprsatdt_dfn}
\end{eqnarray}

\chapter{Radiative Properties}\label{sxn:rdn}
The interaction of particles and radiation determines the
photochemical and radiative forcing of the particles.

\section[Radiation Literature Review]{Radiation Literature Review}\label{sxn:rdn_ltr}
\cite{Geo01} modeled radiative forcing by dust in the Arabian Sea 
region and evaluated the predictions against pre-INDOEX measurements. 
\cite{MGH03}, \cite{HHS03}, and \cite{HFO03} modeled radiative forcing
by North African mineral dust during the Saharan Dust Experiment
(\trmidx{SHADE}) campaign \cite[]{THP03}. 
\cite{CWJ03} PRIDE fxm.
\cite{MTP04} examine the surface radiative forcing by dust and its
influence on the hydrologic cycle.
\cite{CSK04} measured the size distribution and analyzed the optical
properties of soot-dust mixtures in Asian outflow.
\cite{Rou05} presents the optical constants of montmorillonite, an
important constituent of mineral dust.
\cite{RPO91} measure optical constants of kaolinite and
montmorillonite.
\cite{QOL78} measure optical constants of \trmidx{limestone}, aka
\trmidx{amorphous calcite} from 0.2--32.8\,\um. 
\cite{EgH79} report optical optical constants of many minerals including
\trmidx{montmorillonite}, \trmidx{illite}, \trmidx{kaolinite}, mica,
and feldspar. 
\cite{LQB93} measure optical constants of crystalline and powdered
\trmidx{calcite} and \trmidx{gypsum} from 2.5--300\,\um.

\section[Refractive Indices]{Refractive Indices}\label{sxn:idx_rfr}
The interaction of radiation with matter is characterized by the
\trmdfn{index of refraction} of the material at a give wavelength,
$\idxrfrwvl$. 
For non-absorbing media, such as air, the index of refraction is
real-valued and represents the ratio of the speed of light through a
vacuum to the speed of light through the material. 
For instance, the index of refraction of water is approximately 1.33, 
thus light propogates through a vacuum about one-third more quickly
than through the ocean.
In cloud and aerosol physics, however, absorption may not be neglected
and the refractive index is best represented by a complex number,
where the real part represents scattering and the imaginary part
absorption. 
The nomenclature of refractive indices is quite intricate, since 
many related properties are also represented by complex numbers and 
this is not always made clear.
Our discussion of refractive indices adopts the notation of
\cite{BoH83}, whose notation is concise yet is not ambiguous.

\trmdfn{Refractive indices} describe the absorbing and scattering
properties of the medium.
As such, the refractive index is connected to the dielectric
properties of the medium.
Plane electromagnetic waves of the form 
\begin{eqnarray}
% BoH83 p. 25 (2.39)
\lctvctcpx & = & \lctvctnot \exp( \mi \wvnbrvct \cdot \xxx - \mi \frqngl \tm )
\nonumber \\
\mgnvctcpx & = & \mgnvctnot \exp( \mi \wvnbrvct \cdot \xxx - \mi \frqngl \tm )
\label{eqn:pln_wv_dfn}
\end{eqnarray}
satisfy \trmidx{Maxwell's equations} under certain conditions.
Here $\lctvctcpx$ and $\mgnvctcpx$ are complex representation of the
electric and magnetic fields, respectively.
The actual electric and magnetic fields are the real components of the
complex representations, i.e., 
$\lctvct = \mbox{Re}(\lctvctcpx)$ and 
$\mgnvct = \mbox{Re}(\mgnvctcpx)$, respectively.
The \trmdfn{wavenumber vector} $\wvnbrvct$ (\xm) and the
\trmdfn{angular frequency} $\frqngl$ (\xs) define the spatial and
temporal scales of the waves.
The constant vectors $\lctvctnot$ and $\mgnvctnot$ determine the
amplitude and direction of the field.

When the medium is absorbing, the wavenumber vector is complex 
\begin{eqnarray}
\wvnbrvct & = & \wvnbrvctrl + \mi \wvnbrvctimg
\label{eqn:wv_nbr_vct_dfn}
\end{eqnarray}
where $\wvnbrvctrl$ and $\wvnbrvctimg$ are
positive, real vectors.
The wave vector of a homogeneous wave may be simplified as
\begin{eqnarray}
\wvnbrvct & = & (\wvnbrrl + \mi \wvnbrimg) \prphat
\label{eqn:wv_nbr_scl_dfn}
\end{eqnarray}
When (\ref{eqn:wv_nbr_vct_dfn}) is substituted into the constituitive
relations \cite[e.g.,][]{BoH83}, we obtain
\begin{eqnarray}
% BoH83 p. 27 (2.47)
\wvnbr & = & \wvnbrrl + \mi \wvnbrimg = 
\frac{\frqngl \idxrfr}{\cstspdlgt } \nonumber \\
\idxrfr & = & \frac{\cstspdlgt \wvnbr}{\frqngl } \nonumber \\
& = & \frac{\cstspdlgt ( \wvnbrrl + \mi \wvnbrimg )}{\frqngl } \nonumber \\
& = & \cstspdlgt \sqrt{ \lctmu \lctprmt } \nonumber \\
& = & \sqrt{ \frac{\lctmu \lctprmt}{\lctmuvcm \lctprmtvcm }} \nonumber \\
& = & \idxrfrrl + \mi \idxrfrimg
\label{eqn:idx_rfr_dfn}
\end{eqnarray}
where $\idxrfr$ is the \trmdfn{index of refraction}.
The real and imaginary components of $\idxrfr$, $\idxrfrrl$ and
$\idxrfrimg$, are positive definite due to the sign conventions chosen 
in (\ref{eqn:pln_wv_dfn}) and (\ref{eqn:idx_rfr_dfn}).

For example, the irradiance of a plane wave passing through an
electromagnetic medium is attenuated with the distance $\zzz$
within the medium that the wave has traversed.
For a wave of initial irradiance $\flx_{0}$
\begin{eqnarray}
% BoH83 p. 29 (2.52)
\flx(\zzz) & = & \flx_{0} \me^{-\alpha \zzz}
\label{eqn:irr_dfn}
\end{eqnarray}
where $\alpha$ is the \trmidx{absorption coefficient} and is related
to the index of refraction
\begin{eqnarray}
% BoH83 p. 29 (2.52)
\alpha & = & \frac{4 \mpi \idxrfrimg}{\wvl }
\label{eqn:abs_cff_dfn_dfn}
\end{eqnarray}

Complex refractive indices (\ref{eqn:idx_rfr_dfn}) are not the only 
quantities which concisely describe the optical properties of a
material.
The permittivity $\lctprmt$ is usually expressed in terms of
the \trmdfn{relative permittivity} $\lctdlc$ and
the \trmdfn{vacuum permittivity} $\lctprmtvcm$, also called the 
\trmdfn{permittivity of free space}.
\begin{eqnarray}
% http://en.wikipedia.org/wiki/Permittivity
\lctprmt & \equiv & \lctdlc \lctprmtvcm \\
\lctprmtvcm & \equiv & \frac{1}{\cstspdlgt^{2}\lctmuvcm} 
\approx 8.8541878176 \times 10^{-12} \qquad [\mbox{\fxm}]
\end{eqnarray}
% fxm: Straighten out permittivity notation.
% Main problem is that most of aer.tex say permittivity where it means
% relative permittivity
The vacuum permittivity $\lctprmtvcm$ is a fundamental constant with
SI units [\fxm]. 
The relative permittivity $\lctdlc$ is also called the 
\trmdfn{dielectric function} and \trmdfn{dielectric constant} and is
dimensionless.  
\begin{eqnarray}
% BoH83 p. 227 (9.1)
\label{eqn:lct_dlc_dfn}
\lctdlc & = & \lctdlcrl + \mi \lctdlcimg \\
& = & \idxrfr^{2} \nonumber \\
& = & (\idxrfrrl + \mi \idxrfrimg)^{2} \nonumber \\
& = & \idxrfrrl^{2} - \idxrfrimg^{2} + 2 \idxrfrrl \idxrfrimg \mi \nonumber \\
\lctdlcrl & = & \frac{\lctprmtrl}{\lctprmtvcm} =
\idxrfrrl^{2} - \idxrfrimg^{2} \nonumber \\
\lctdlcimg & = & \frac{\lctprmtimg}{\lctprmtvcm} =
2 \idxrfrrl \idxrfrimg \nonumber \\
\idxrfr & = & \sqrt{\lctdlc} \nonumber \\
& = & \sqrt{ \lctdlcrl + \mi \lctdlcimg} \nonumber \\
\idxrfrrl & = & \sqrt{ 
\frac{\sqrt{\lctdlcrlsqr + \lctdlcimgsqr} + \lctdlcrl}{2}}
\nonumber \\
\idxrfrimg & = & \sqrt{ 
\frac{\sqrt{\lctdlcrlsqr + \lctdlcimgsqr} - \lctdlcrl}{2}}
\nonumber
\end{eqnarray}
The square root in (\ref{eqn:lct_dlc_dfn}) with positive $\idxrfrimg$  
conforms with the sign convention made in (\ref{eqn:pln_wv_dfn}) and
(\ref{eqn:idx_rfr_dfn}).  

\subsection[Reflectance-Based Refractive Indices]{Reflectance-Based Refractive Indices}\label{sxn:rfl}
Refractive indices may be measured by many techniques, including 
\trmidx{Kramers-Kronig}\footnote{Named in honor of Hendrik Kramers and
  Ralph Kronig.} analysis and dispersive analysis.
The \trmdfn{dispersive analysis} measures the zenith (normal)
spectral reflectance of the material, and fits this to a series of
independent damped oscillator resonances known as 
\trmdfn{Lorentz lines} \cite[][]{QOL78,LQB93}
\begin{eqnarray}
% LQB93 p. 192
% NB: Defining equation has ambiguous typo in frequency subscript
% Comparison to QOL78 shows that first wavenumber in denominator is resonance frequency
\lctdlc & = & \lctdlchgh + \sum_{\lnidx=1}^{\lnidx=\lnnbr} 
\frac{\lnstridx}{\rsnwvnidx^{2} - \wvn^{2} - \mi\cstdmpidx\wvn}
\label{eqn:dsp_anl_LQB93}
\end{eqnarray}
where the three parameters for each Lorentz line are the line strength
$\lnstridx$, the resonance position $\rsnwvnidx$, and the damping
constant $\cstdmpidx$\,[\xcm].
We write (\ref{eqn:dsp_anl_LQB93}) in terms of \trmidx{wavenumber} rather
than frequency for consistency with \cite{QOL78} and \cite{LQB93} who
report line parameters in \xcm\ units.

\cite{LQB93} tabulate a modified line strength $\lnstrmod$ defined by
\begin{eqnarray}
% LQB93 p. 192
\lnstridx & = & \lnstrmodidx \rsnwvnidx^{2} 
\label{eqn:ln_str_mod_dfn}
\end{eqnarray}
The modified line strengths $\lnstrmod$ are smaller than $\lnstr$ and
this increases the convergence speed of data-fitting to
(\ref{eqn:dsp_anl_LQB93}). 

\cite{QOL78} report resonance parameters from a dispersive analysis 
which constructs the dielectric constant as
\begin{eqnarray}
% LQB93 p. 192
% NB: Defining equation has ambiguous typo in frequency subscript
% Comparison to QOL78 shows that first wavenumber in denominator is resonance frequency
\lctdlc & = & \lctdlchgh + \sum_{\lnidx=1}^{\lnidx=\lnnbr} 
\frac{4\mpi\lnstrQOLidx\rsnwvnidx^{2}[\rsnwvnidx^{2}-\wvn^{2}+\mi\cstdmpQOLidx\wvn]}
{(\rsnwvnidx^{2} - \wvn^{2})^{2} + (\cstdmpQOLidx\rsnwvnidx\wvn)^{2}}
\label{eqn:dsp_anl_QOL78}
\end{eqnarray}
where $\lnstrQOLidx$ is a line strength and $\cstdmpQOLidx$ is a
(dimensionless) damping constant.
It is straightforward to verify that (\ref{eqn:dsp_anl_QOL78}) is
equivalent to (\ref{eqn:dsp_anl_LQB93}) when the following equivalence
between the line strength and damping parameter definitions of
\cite{QOL78} and \cite{LQB93}
\begin{subequations}
\begin{align}
% LQB93 p. 192
\lnstr &= \lnstrmod \rsnwvn^{2} = 4\mpi\lnstrQOL\rsnwvn^{2} \\
\lnstrQOL &= \lnstr/(4\mpi\rsnwvn^{2}) = \lnstrmod/(4\mpi) \\
\cstdmp &= \rsnwvn\cstdmpQOL \\
\cstdmpQOL &=  \cstdmp/\rsnwvn
\label{eqn:QOL78_LQB93}
\end{align}
\end{subequations}

\trmdfn{Fresnel's equation} determines the normal spectral
reflectance~$\rfl$ in terms of the refractive index~$\idxrfr$ 
\begin{eqnarray}
% LQB93 p. 193
\rfl & = & \left| \frac{1-\idxrfr}{1+\idxrfr} \right|^{2} \\
& = & \frac{(\idxrfrrl-1)^{2} + \idxrfrimg^{2}}{(\idxrfrrl+1)^{2} + \idxrfrimg^{2}}
\label{eqn:rfl_lrn_dfn}
\end{eqnarray}

\section[Effective Medium Approximations]{Effective Medium Approximations}\label{sxn:ema}
Heterogeneous aerosols are very common in nature.
Analytic solutions exist for certain geometries of mixtures, such as
concentric spheres \cite[]{BoH83}.
One way to treat the optics of these complex particles is to determine
an effective refractive index which implicitly accounts for the
fraction and optical properties of each individual component. 
The search for the best effective refractive index has fostered a
variety of approximations known as 
\trmdfn{effective medium approximations} (\trmidx{EMA}s).
The web site
\href{http://www.mpia-hd.mpg.de/homes/henning/Dust_opacities/Opacities/Ralf/Eff/rules.html}{Effective Medium Theories}
is devoted to EMAs, and is remarkably useful.
EMAs include volume-weighting, 
the \trmidx{Maxwell Garnett approximation} \cite[]{MaG04}, 
the \trmidx{Bruggeman approximation} \cite[]{Bru35}, 
the \trmidx{Hollow Sphere Equivalent} \cite[]{BoH83},
and the \trmidx{extended effective medium approximation} \cite[]{ViC98}.   
Effective medium approximations all have the virtue of not requiring
exact knowledge of the geometries of the multi-component aerosols.

The use of inclusion mass fraction $\mssfrcnclttl$ and inclusion volume
fraction $\vlmfrcnclttl$ in effective medium approximations varies.
If the density of the medium and the inclusion are equal, then
$\mssfrcnclttl$ and $\vlmfrcnclttl$ may be interchanged in
(\ref{eqn:mxw_grn_dfn}) and (\ref{eqn:mxw_grn_idx_rfr_dfn}).
In general, $\vlmfrcnclttl$ is more appropriate because it is a spatial
measure and effective medium approximations are based on
approximations to the spatial distribution of radiation.
Most authors use $\vlmfrcnclttl$ \cite[e.g.,][]{ViC98}.
However, some authors \cite[e.g.,][]{BoH83} use $\mssfrcnclttl$.
We consistently use $\vlmfrcnclttl$ in favor of $\mssfrcnclttl$.

\subsection[General Considerations]{General Considerations}\label{sxn:ema_gnr}
A two component aerosol is the simplest example of a 
\trmidx{multi-component aerosol} (\trmidx{MCA}).
An idealized geometry would be a spherical \trmidx{matrix} 
containing a spherical \trmidx{inclusion}. 
The mass $\mssmtx$, and volume $\vlmmtx$ of the matrix component do
not depend on the inclusion, though the matrix radius $\rdsmtx$ does. 
The inclusion radius $\rdsncl$, mass $\mssncl$, and volume $\vlmncl$
are completely independent of the matrix.
Relations among the 
\trmidx{radius fraction} $\rdsfrcnclmtx$, 
\trmidx{mass fraction} $\mssfrcnclmtx$, and
\trmidx{volume fraction} $\vlmfrcnclmtx$ of the inclusion relative to
the matrix are
\begin{eqnarray}
\vlmfrcnclmtx & \equiv & \frac{\vlmncl}{\vlmmtx} = \mssfrcnclmtx \frac{\dnsmtx}{\dnsncl} \\
\mssfrcnclmtx & \equiv & \frac{\mssncl}{\mssmtx} = \vlmfrcnclmtx \frac{\dnsncl}{\dnsmtx} \\
\vlmfrcnclmtx & \equiv & \frac{\vlmncl}{\vlmmtx} = \frac{\rdsfrcnclmtx^{3}}{1-\rdsfrcnclmtx^{3}} \\
\rdsfrcnclmtx & \equiv & \frac{\rdsncl}{\rdsmtx} = \left(\frac{\vlmfrcnclmtx}{\vlmfrcnclmtx+1}\right)^{1/3}
\label{eqn:rds_frc_ncl_mtx_dfn}
\end{eqnarray}
Note that (\ref{eqn:rds_frc_ncl_mtx_dfn}) applies to the ratios of
inclusion to matrix properties (analogous to dry mixing ratios), not
inclusions to total properties (analogous to moist mixing ratios). 

\subsection[Hollow Sphere Equivalent]{Hollow Sphere Equivalent}\label{sxn:hse_apx}
The \trmdfn{Hollow Sphere Equivalent approximation}
\cite[][p.~149]{BoH83} defines\ldots % fxm

\subsection[Volume-Weighted]{Volume-Weighted}\label{sxn:vlw_apx}
The simplest estimate for the multi-component index of refraction is
to weight the optical properties of each component by the volume
occupied by that component.
\begin{eqnarray}
\label{eqn:vlm_wgt_dfna}
\idxrfravg & = & \sum_{\spcidx = 1}^{\spcnbr} \vlmfrcspc \idxrfrspc \\
\label{eqn:vlm_wgt_dfnb}
\lctprmtavg & = & \sum_{\spcidx = 1}^{\spcnbr} \vlmfrcspc \lctprmtspc
\end{eqnarray}
Volume-weighting the refractive indices $\idxrfr$
(\ref{eqn:vlm_wgt_dfna}) rather than the electric permittivity
$\lctprmt$ (\ref{eqn:vlm_wgt_dfnb}) leads to different answers since  
$\lctprmtavg = \idxrfravg^{2}$ (\ref{eqn:idx_rfr_dfn}).
Whether (\ref{eqn:vlm_wgt_dfna}) or (\ref{eqn:vlm_wgt_dfnb}) makes
more physical sense or performs better should be investigated.

The volume-weighted approximation is often made for liquid mixtures.
However, Section~\ref{sxn:apx_pmr} describes a more accurate
treatment, the theory of partial molar refraction.

\subsection[Maxwell Garnett Approximation]{Maxwell Garnett Approximation}\label{sxn:mxw_grn_apx}
The \trmdfn{Maxwell Garnett approximation}\footnote{The name comes
from J.~C. Maxwell Garnett (unhyphenated), the theory's originator.}
\cite[]{MaG04} defines an effective dielectric function $\lctprmtavg$
for a medium formed of a background matrix containing random sizes,
shapes, orientations, and positions of inclusions of different
compositions.  
The approximation takes a relatively simple form for homogeneous
spherical inclusions of any size and position and with the same
composition embedded in a matrix of a distinct composition.
Let the (complex) dielectric constants of the inclusions and the
matrix be $\lctprmtncl$ and $\lctprmtmtx$, respectively, and let 
the total volume fraction of the inclusions be~$\vlmfrcnclttl$.
Then the Maxwell Garnett approximation for the effective dielectric
constant of the two component mixture is 
\begin{eqnarray}
% BoH83 p. 217 (8.50), ViC98 p. 3 (5), SoT99 p. 9435 
\lctprmtavg & = & \dpysty \lctprmtmtx \left[ 1+ 
\frac{3\vlmfrcnclttl \left( \frac{\lctprmtncl-\lctprmtmtx}{\lctprmtncl+2\lctprmtmtx} \right) 
}{
1-\vlmfrcnclttl \left( \frac{\lctprmtncl-\lctprmtmtx}{\lctprmtncl+2\lctprmtmtx} \right) 
} \right] \\
& = & \lctprmtmtx \left[ 1+ 
\frac{3\vlmfrcnclttl (\lctprmtncl-\lctprmtmtx)}
{\lctprmtncl+2\lctprmtmtx-\vlmfrcnclttl(\lctprmtncl-\lctprmtmtx)} \right]
\nonumber \\
& = & \lctprmtmtx \left[ 
\frac{\lctprmtncl+2\lctprmtmtx-\vlmfrcnclttl(\lctprmtncl-\lctprmtmtx)+3\vlmfrcnclttl(\lctprmtncl-\lctprmtmtx)}
{\lctprmtncl+2\lctprmtmtx-\vlmfrcnclttl(\lctprmtncl-\lctprmtmtx)} \right]
\nonumber \\
& = & \lctprmtmtx \left[ 
\frac{\lctprmtncl+2\lctprmtmtx-\vlmfrcnclttl\lctprmtncl+\vlmfrcnclttl\lctprmtmtx+3\vlmfrcnclttl\lctprmtncl-3\vlmfrcnclttl\lctprmtmtx}
{\lctprmtncl-\vlmfrcnclttl\lctprmtncl+2\lctprmtmtx+\vlmfrcnclttl\lctprmtmtx} \right]
\nonumber \\
& = & \lctprmtmtx \left[ 
\frac{\lctprmtncl(1+2\vlmfrcnclttl)+2\lctprmtmtx(1-\vlmfrcnclttl)}
{\lctprmtncl(1-\vlmfrcnclttl)+\lctprmtmtx(2+\vlmfrcnclttl)} \right]
\nonumber
\label{eqn:mxw_grn_dfn}
\end{eqnarray}
The original definition, Equation~(\ref{eqn:mxw_grn_dfn}), although
cumbersome, contains the fewest complex arithmetic operations, and is
thus most efficiently evaluated.

We use (\ref{eqn:lct_dlc_dfn}) to rewrite the Maxwell Garnett
approximation (\ref{eqn:mxw_grn_dfn}) in terms of refractive indices
\begin{eqnarray}
\idxrfravg^{2} & = & \idxrfrmtx^{2} \left[ 
\frac{\idxrfrncl^{2}(1+2\vlmfrcnclttl)+2\idxrfrmtx^{2}(1-\vlmfrcnclttl)}
{\idxrfrncl^{2}(1-\vlmfrcnclttl)+\idxrfrmtx^{2}(2+\vlmfrcnclttl)} \right] 
\label{eqn:mxw_grn_idx_rfr_dfn}
\end{eqnarray}
where $\vlmfrcnclttl$ is the volume fraction of the inclusion, and
$\idxrfrmtx$ and $\idxrfrncl$ are the matrix (medium) and inclusion
(particle) refractive indices, respectively \cite[]{ViC98}.  
Equation~(\ref{eqn:mxw_grn_idx_rfr_dfn}) is suitable for evaluation
when refractive indices, rather than electrical permittivity, is the  
independent variable of choice.
Equivalently,
\begin{eqnarray}
% ViC98 p. 3 (5) 
\idxrfravg^{2} & = & \idxrfrmtx^{2}
\frac{\idxrfrncl^{2}+2\idxrfrmtx^{2}+2\vlmfrcnclttl(\idxrfrncl^{2}-\idxrfrmtx^{2})}
{\idxrfrncl^{2}+2\idxrfrmtx^{2}-\vlmfrcnclttl(\idxrfrncl^{2}-\idxrfrmtx^{2})}
\label{eqn:mxw_grn_idx_rfr_dfn2}
\end{eqnarray}
This form may yield greater insight into the relative roles of the  
the matrix and inclusion in the optical properties, at the expense of 
a few more complex arithmetic operations.

Although (\ref{eqn:mxw_grn_dfn}) has the correct limit as
$\lctprmtmtx \rightarrow \lctprmtncl$, it is not symmetric under the
exchange of the matrix and the inclusion.
This is problematic when it is difficult to determine which component
is the matrix and which is the inclusion.
Nonetheless, no approximation has been more successful or widely used
than (\ref{eqn:mxw_grn_dfn}). 

The Maxwell Garnett approximation extends to $\spcnbr$-component
aerosol mixtures where $\spcnbr > 2$ \cite[][p.~216]{BoH83}.
One of the $\spcnbr$~components must be identified as the matrix;
the remaining $\spcnbr-1$~components are considered inclusions.
Inclusions contribute to the effective permittivity $\lctprmtavg$
according to their volume, permittivity, and geometry: 
\begin{eqnarray}
% BoH83 p. 216
\lctprmtavg & = & 
\frac{(1-\vlmfrcnclttl) \lctprmtmtx + 
\sum_{\spcidx = 1}^{\spcnbr-1} \vlmfrcspc \emabetaspc \lctprmtspc}
{1-\vlmfrcnclttl+\sum_{\spcidx = 1}^{\spcnbr-1} \vlmfrcspc \emabetaspc}
\label{eqn:mxw_grn_ncl_mlt}
\end{eqnarray}
where the total volume fraction of inclusions is
\begin{eqnarray}
\vlmfrcnclttl & \equiv & \sum_{\spcidx = 1}^{\spcnbr-1} \vlmfrcspc
\label{eqn:vlm_frc_ncl_ttl_dfn}
\end{eqnarray}
The geometrical factor $\emabetaspc$ is related to the probability
distribution function of the path lengths through the inclusions. 
The optical average computed by (\ref{eqn:mxw_grn_ncl_mlt}) is 
a non-linearly weighted average (rather than a pure volume-weighted
average) of the inclusions and the matrix because the matrix occupies
a preferred (and asymmetric) position in the geometrical
factor~$\emabetaspc$. 

The geometric factor $\emabeta$ reduces to a simple analytic expression
for idealized spherical inclusions 
\begin{eqnarray}
% BoH83 p. 217 near (8.50)
\emabetaspc & = & \frac{3\lctprmtmtx}{\lctprmtspc+2\lctprmtmtx}
\label{eqn:mxw_grn_beta}
\end{eqnarray}
Substitution of (\ref{eqn:mxw_grn_beta}) into
(\ref{eqn:mxw_grn_ncl_mlt}) with $\spcnbr = 2$ leads directly to
(\ref{eqn:mxw_grn_dfn}). 

\subsection[Bruggeman Approximation]{Bruggeman Approximation}\label{sxn:brg_apx}
The \trmdfn{Bruggeman approximation} \cite[]{Bru35} for the mean
dielectric function of a two component mixture is \cite[]{BoH83}
\begin{eqnarray}
% BoH83 p. 217 (8.51), ViC98 p. 3 (4)
\vlmfrcnclttl \frac{\lctprmtncl - \lctprmtavg}{\lctprmtncl + 2 \lctprmtavg }
+ ( 1 - \vlmfrcnclttl )
\frac{\lctprmtmtx - \lctprmtavg}{\lctprmtmtx + 2 \lctprmtavg }
& = & 0
\label{eqn:brg_dfn}
\end{eqnarray}
Certain limits of (\ref{eqn:brg_dfn}) must be carefully treated
numerically 
First, (\ref{eqn:brg_dfn}) is indeterminate in the limit as
$\lctprmtmtx \rightarrow \lctprmtncl$.
In this case, set $\lctprmtavg = \lctprmtncl = \lctprmtmtx$.
Second, the limit as $\vlmfrcnclttl \rightarrow 0$ may cause difficulty
with the quadratic equation numerical solution presented below
(cf.\ (\ref{eqn:brg_dfn_qdr})).
In this case, direct substitution of $\vlmfrcnclttl = 0$ into
(\ref{eqn:brg_dfn}) yields the correct result that
$\lctprmtavg \rightarrow \lctprmtmtx$.

A prime advantage of (\ref{eqn:brg_dfn}) is its symmetry: 
$\lctprmtavg$ is invariant under exchange of $\lctprmtncl$ and
$\lctprmtmtx$. 
The terminology of ``\trmidx{matrix}'' and ``\trmidx{inclusion}''
is therefore ill-suited for the Bruggeman approximation, since
conceptually, a matrix occupies more volume than an inclusion
and therefore has a preferred status.
It is more consistent to simply label components by numerical
subscripts (e.g., $\idxrfr_{1}, \idxrfr_{2}$) in the Bruggeman
approximation.
However, the very symmetry of the components allows us to continue
to use our old terminology with the Bruggeman approximation without
fear of applying the matrix or inclusion properties in the wrong slot
(since their identities are irrelevant).

This component symmetry of the Bruggeman approximation
(\ref{eqn:brg_dfn}) has conceptual advantages over other effective
medium approxiations (e.g., (\ref{eqn:mxw_grn_dfn})) whenever it is
difficult to assign one component the role of the matrix and the other
the role of inclusion.   
However, (\ref{eqn:brg_dfn}) has not been quite as successful
at predicting experimental results as the Maxwell Garnett
approximation \cite[][p. 217]{BoH83}. 

Application of the Bruggeman approximation is more difficult than
the Maxwell Garnett theory because of the implicit definition of 
$\lctprmtavg$ in (\ref{eqn:brg_dfn}).
Fortunately, straightforward algebraic manipulation reduces
the definition to a quadratic polynomial in $\lctprmtavg$.
Multiplying (\ref{eqn:brg_dfn}) by $\lctprmtncl + 2 \lctprmtavg$ and then 
by $\lctprmtmtx + 2 \lctprmtavg$ we obtain
\begin{eqnarray}
\vlmfrcnclttl(\lctprmtncl-\lctprmtavg) 
+(1-\vlmfrcnclttl)
\frac{(\lctprmtmtx-\lctprmtavg)(\lctprmtncl+2\lctprmtavg)
}{ \lctprmtmtx+2\lctprmtavg }
& = & 0 \nonumber \\ 
\vlmfrcnclttl (\lctprmtncl-\lctprmtavg)(\lctprmtmtx+2\lctprmtavg)
+ (1-\vlmfrcnclttl) 
(\lctprmtmtx-\lctprmtavg)(\lctprmtncl+2\lctprmtavg)
& = & 0 \nonumber \\ 
\vlmfrcnclttl 
[\lctprmtncl\lctprmtmtx+(2\lctprmtncl-\lctprmtmtx)\lctprmtavg-2\lctprmtavg^{2}]
+ (1-\vlmfrcnclttl) 
[\lctprmtmtx \lctprmtncl+(2\lctprmtmtx-\lctprmtncl)\lctprmtavg-2\lctprmtavg^{2}]
& = & 0 \nonumber \\ 
(-2\vlmfrcnclttl+2\vlmfrcnclttl-2)\lctprmtavg^{2}+
[\vlmfrcnclttl(2\lctprmtncl-\lctprmtmtx)+(1-\vlmfrcnclttl)(2\lctprmtmtx-\lctprmtncl)]\lctprmtavg +
\vlmfrcnclttl\lctprmtncl\lctprmtmtx-\vlmfrcnclttl\lctprmtncl\lctprmtmtx+
\lctprmtncl\lctprmtmtx & = & 0 \nonumber \\ 
-2\lctprmtavg^{2}+
[2\vlmfrcnclttl\lctprmtncl-\vlmfrcnclttl\lctprmtmtx+2\lctprmtmtx-\lctprmtncl-2\vlmfrcnclttl\lctprmtmtx+\vlmfrcnclttl\lctprmtncl]\lctprmtavg+
\lctprmtncl\lctprmtmtx & = & 0 \nonumber \\ 
-2\lctprmtavg^{2}+
[(3\vlmfrcnclttl-1)\lctprmtncl+(2-3\vlmfrcnclttl)\lctprmtmtx]\lctprmtavg+
\lctprmtncl\lctprmtmtx & = & 0 \nonumber \\
2\lctprmtavg^{2}+
[(1-3\vlmfrcnclttl)\lctprmtncl+(3\vlmfrcnclttl-2)\lctprmtmtx]\lctprmtavg-
\lctprmtncl \lctprmtmtx & = & 0
\label{eqn:brg_dfn_qdr}
\end{eqnarray}
This quadratic equation is analytically solvable for $\lctprmtavg$.
In deciding which of the roots is physical, one must discard
roots which make (\ref{eqn:brg_dfn}) indeterminate, i.e., 
$\lctprmtavg \ne -\lctprmtncl/2$ and $\lctprmtavg \ne -\lctprmtmtx/2$.
Finally, the square root of $\lctprmtavg$ with the positive imaginary
component, if any, must be taken to conform with the sign conventions
(\ref{eqn:pln_wv_dfn}) and (\ref{eqn:idx_rfr_dfn}).
In practice, root finding methods (e.g., bracketing) may be preferable 
when a large number of smoothly varying spectral properties must be
solved for. 

In terms of refractive indices, the Bruggeman approximation
(\ref{eqn:brg_dfn}) is equivalent to
\begin{eqnarray}
% ViC98 p. 3 (4) 
\vlmfrcnclttl\frac{\idxrfrncl^{2}-\idxrfravg^{2}}{\idxrfrncl^{2}+2\idxrfravg^{2}}+
(1-\vlmfrcnclttl)\frac{\idxrfrmtx^{2}-\idxrfravg^{2}}{\idxrfrmtx^{2}+2\idxrfravg^{2}}
& = & 0
\label{eqn:brg_idx_rfr_dfn}
\end{eqnarray}
where
$\vlmfrcnclttl$ is the volume fraction of the inclusion,
and $\idxrfrncl$ and $\idxrfrmtx$ are the inclusion and matrix
refractive indices, respectively.
Recall that the Bruggeman approximation is symmetric under interchange
of the components, so either component may be labeled the matrix and
the other the inclusion.
The solution to (\ref{eqn:brg_idx_rfr_dfn}) is obtained by
substituting (\ref{eqn:lct_dlc_dfn}) into (\ref{eqn:brg_dfn_qdr}):
\begin{eqnarray}
% ViC98 p. 3 (4) has conflicting conventions on matrix and inclusion
% ViC98 p. 3 first paragraph says that m1=ncl, m2=mtx
% ViC98 p. 3 figure captions use m1=mtx and m2=ncl
2\idxrfravg^{4}+ 
[(1-3\vlmfrcnclttl)\idxrfrncl^{2}+(3\vlmfrcnclttl-2)\idxrfrmtx^{2}]\idxrfravg^{2}-
\idxrfrncl^{2}\idxrfrmtx^{2} & = & 0
\label{eqn:brg_idx_rfr_dfn_qdr}
\end{eqnarray}
which is a quadratic equation in~$\idxrfravg^{2}$. 
The preceding discussions on physically realistic solutions to
(\ref{eqn:brg_dfn_qdr}) apply equally
to~(\ref{eqn:brg_idx_rfr_dfn_qdr}). 

The Bruggeman approximation extends to $\spcnbr$-component aerosol
mixtures where $\spcnbr > 2$.
% fxm: describe this  

\subsection[Extended Effective Medium Aproximation]{Extended Effective Medium Approximation}\label{sxn:xem_apx}
The \trmdfn{extended effective medium approximation} (\trmidx{EEMA}) 
\cite[]{ViC98} relaxes the assumption (implicit in other effective
medium approximations) that the grain sizes are small compared to the 
wavelength of light.
The resulting extended effective refractive index $\idxrfravg$ is
obtained by iterative solution to
\begin{eqnarray}
% ViC98 p. 2 (1)
\idxrfravg^{2}_{\kkk+1} & = & \idxrfrncl^{2}
\frac{\AAA_{\kkk}(1-\vlmfrcnclttl)+\vlmfrcnclttl\BBB_{\kkk}}{\AAA_{\kkk}(1-\vlmfrcnclttl)-2\vlmfrcnclttl\BBB_{\kkk}}
\label{eqn:xem_dfn}
\end{eqnarray}
where 
$\idxrfravg$ is the effective refractive index,
$\idxrfrmtx$ is the matrix (medium) refractive index,
$\idxrfrncl$ is the inclusion refractive index,
$\kkk$ denotes the iteration,
$\vlmfrcnclttl$ is the volume fraction of the inclusion,
$\AAA_{\kkk}$ and $\BBB_{\kkk}$ are given by 
\begin{subequations}
\label{eqn:xem_cff_dfn}
\begin{align}
% ViC98 p. 3 (2,3)
\label{eqn:xem_A_dfn}
\AAA_{\kkk} & = -\frac{12\mi\mpi^{2}\idxrfravg^{3}}{\wvl^{3}} \\
\label{eqn:xem_B_dfn}
\BBB_{\kkk} & = \frac{3}{4\mpi\rds^{3}} 
\sum_{\nnn}(2\nnn+1)[\aaa_{\nnn}(\rds,\idxrfrmtx/\idxrfravg_{\kkk})+
\bbb_{\nnn}(\rds,\idxrfrmtx/\idxrfravg_{\kkk})]
\end{align}
\end{subequations}
where
$\rds$ is the inclusion radius,
$\wvl$ is the incident wavelength,
and $\aaa_{\nnn}(\rds,\idxrfr)$ and $\bbb_{\nnn}(\rds,\idxrfr)$ are
the traditional Mie scattering coefficients.

The most general definition of the size parameter $\szprm$ is
is the ratio of particle circumference times real refractive index 
of the surrounding medium to wavelength 
\begin{eqnarray}
\szprm & = & 2 \mpi \rds \idxrfrmtxrl / \wvl
\label{eqn:sz_prm_gnr_dfn}
\end{eqnarray}
For particles in air, $\idxrfrmtxrl \approx 1$ so 
$\szprm \approx 2\mpi\rds/\wvl$.
The size parameter $\szprmncl$ of inclusions in cloud droplets
is approximately $\idxrfrmtxrl \approx 1.33$ times the size parameter
$\szprm$ of the same inclusion in air or in vacuum.
\cite{ViC98} show that the extended effective medium approximation
is more accurate thatn Maxwell Garnett and Bruggeman appoximations for
inclusion size parameters $\szprmncl > 0.5$. 

\section[Refractive Indices of Dust]{Refractive Indices of Dust}\label{sxn:idx_rfr_dst}
Laboratory and field measurements of dust refractive properties are 
reported in
\cite{PGS77,Vol73,Pat81,BoH83,TDH88,SAJ93,HKS98}. 
Refractive properties of salts (e.g., \NaCl, \NHqdSOq, \NHqNOt)
are in \cite{Vol73,TaM94,Tan97}, 
Refractive properties of pure water are reported in
\cite{Ray72,Seg81,WWQ89,PoF97} (liquid phase) and
\cite{Ray72,War84,PeG91} (ice phase).  
Sophisticated models may be used in conjunction with field
measurements to constrain optical properties.
Multiple studies have used this technique to estimate $\idxrfrimg$ for
dust \cite[][]{DuK00,Dub01,DHE02,DHL02,CTT02,STD03,Tor03}. 
Uncertainty in dust optical properties straddles the boundary between 
dust causing net cooling and net heating of the climate system.
This boundary occurs at a \trmidx{single scattering albedo} 
$\ssa \sim 0.91$ \cite[]{LiS982}.

The Aerosol Robotic Network, \trmdfn{AERONET}, for example, 
retrieves size distributions from multi-spectral solar
\trmidx{almucantar} radiances \cite[][]{DuK00}.
The almucantar radiances are radiance measurements in a circle of 
equal scattering angle centered in a plane about the Sun, i.e., 
radiance measurements at known forward scattering phase function 
angles.
\cite{DSH00} quantify the accuracy of the retrievals.
\cite{DHL02} show that spherical particle assumptions bias retrieved
dust properties. 
In particular, Mie theory over-estimates the concentration of small
($\dmt < 0.2$\,\um) particles and under-estimates the real refractive
index at shorter wavelengths.

At 0.64\,\um, \cite{KTD01} found $\ssa = 0.97 \pm 0.02$ from a
combination of satellite measurements and \trmidx{in situ} remote
sensing from \trmidx{AERONET}.
This is in contrast to $\ssa = 0.87 \pm 0.04$ typical of earlier
studies \cite[][]{SoT96}.
\cite{CTT02} determined the UV imaginary index of refraction
$\idxrfrimg$ of Saharan dust particles from \trmidx{TOMS} data using a 
three-dimensional model of dust transport.
At Sal and Tenerife, $\idxrfrimg = 0.0048 (0.0024$--$0.0060)$ and 
$\idxrfrimg = 0.004 (0.002$--$0.005)$ at $\wvl = 331$ and $360$~nm,
respectively. 
At Dakar, $\idxrfrimg = 0.006 (0.0024$--$0.0207)$ and 
$\idxrfrimg = 0.005 (0.0020$--$0.0175)$ at $\wvl = 331$ and $360$~nm, 
respectively. 
After integrating over the measured size distributions, they obtained
single scattering albedos at $\wvl = 331$~nm of 
$\ssa = 0.81 (0.65$--$0.90)$, $0.84 (0.82$--$0.91)$, and 
$0.86 (0.83$--$0.89)$ 
at Dakar, Sal, and Tenerife, respectively.

\section[Liquid or solid mantle coatings]{Liquid or solid mantle coatings}\label{sxn:idx_rfr_mntl}
The problem of determining the optical properties of a coated sphere
(or cylinder) has been solved analytically.
Naturally, the solutions bear a strong resemblance to \trmidx{Mie theory}.
Additional features are caused by interference patterns between the
reflections of the core and the mantle.

\section[Homogeneously Mixed Liquids]{Homogeneously Mixed Liquids}\label{sxn:idx_rfr_lqd}

\subsection[Partial Molar Refraction]{Partial Molar Refraction}\label{sxn:apx_pmr}
The \trmdfn{partial molar refraction} approximation assumes that all
components are homogeneously mixed in the aerosol.
Thus the approach is best-suited to liquid aerosols.
The parameterization technique, described and evaluated by
\cite{Ste90}, depends on the observed additivity of a quantity 
known as the \trmdfn{molar refraction} $\rfrmlr$ [\mCxmol] of
condensed phase species.
The molar refraction may be simply defined in terms of the
molar volume and the refractive index $\idxrfr$ as 
\begin{eqnarray}
% Ste90 p. 1676 (1)
\rfrmlr & = & \left( \frac{\idxrfr^{2} -1}{\idxrfr^{2} + 2} \right)
\vlmmlr 
\label{eqn:rfr_mlr_dfn}
\end{eqnarray}
where the \trmdfn{molar volume} $\vlmmlr$ [\mCxmol] is the physical
volume occupied by the solution per mole of solution.
\begin{eqnarray}
\vlmmlr & = & \mmwavg / \dns
\label{eqn:vlm_mlr_dfn}
\end{eqnarray}

If the ratio of the two quadratic functions of $\idxrfr$ in
(\ref{eqn:rfr_mlr_dfn}) appears non-intuitive, it is worthwhile to
trace its origins because it appears in many optical approximations. 
The factor $(\idxrfr^{2} -1)/(\idxrfr^{2} + 2)$ appears in the 
$\aaa_{\onesbs}$ coefficient of \trmidx{Mie theory}.
The $\aaa$ coefficients are defined in terms of spherical Bessel
functions, and when the Bessel functions are expanded into power
series \cite[e.g.,][p. 131]{BoH83}, the dominant term in the
$\aaa_{\onesbs}$ coefficient contains the quadratic factor.

The mean molecular weight of a solution $\mmwavg$ [\kgxmol] is the sum
of the dimensionless volume mixing ratio (i.e., molar fraction)
$\vmr_{\spcidx}$ of each species times its mean molecular weight
$\mmw_{\spcidx}$.
\begin{eqnarray}
% Ste90 p. 1676 (2)
\mmwavg & = & 
\sum_{\spcidx = 1}^{\spcnbr} \vmr_{\spcidx} \mmw_{\spcidx}
\label{eqn:mmw_avg_dfn}
\end{eqnarray}
Dividing $\mmwavg$ by the density of the bulk solution $\dns$ we
obtain the bulk molar volume $\vlmmlr$ in \mCxmol
\begin{eqnarray}
% Ste90 p. 1676 (2)
\vlmmlr
& = & 
\mmwavg / \dns
=
\sum_{\spcidx = 1}^{\spcnbr} \vmr_{\spcidx} \mmw_{\spcidx} / \dns
\label{eqn:vlm_mlr_dfn2}
\end{eqnarray}

The molar refraction $\rfrmlr$ (\ref{eqn:rfr_mlr_dfn}) of a solution 
is the sum of the partial molar refractions of all the constituent
species $\rfrmlr_{\spcidx}$, weighted by the respective volume mixing
ratios $\vmr_{\spcidx}$
\begin{eqnarray}
% Ste90 p. 1676 (3)
\rfrmlr & = & 
\sum_{\spcidx = 1}^{\spcnbr} \vmr_{\spcidx} \rfrmlr_{\spcidx}
\label{eqn:rfr_mlr_dfn2}
\end{eqnarray}

To illustrate the usefulness of the partial molar refraction
approximation we consider a two component solution.
For concreteness, imagine that the first component is liquid water and 
the second component is \NaCl, so that the solution is similar to
deliquescent sea salt aerosol (which would also include \MgSOq).
The physical properties of the components and of the bulk solution are  
labeled with and without subscripts, respectively, (e.g.,
$\vlmmlr_{\onesbs}$, $\vlmmlr_{\twosbs}$, and $\vlmmlr$).
Using (\ref{eqn:rfr_mlr_dfn2}) and then (\ref{eqn:rfr_mlr_dfn}) we 
see that
\begin{eqnarray}
% Ste90 p. 1677 (4)
\rfrmlr & = & \vmr_{\onesbs} \rfrmlr_{\onesbs}
+ \vmr_{\twosbs} \rfrmlr_{\twosbs} \nonumber
\label{eqn:rfr_mlr_dfn3}
\end{eqnarray}
If we consider the properties of the first component (e.g., pure
water) as known, and the properties of the solution are measured,
then the molar refraction of the second component (e.g., \NaCl) may be 
inferred as
\begin{eqnarray}
% Ste90 p. 1677 (4)
\rfrmlr_{\twosbs} & = & \frac{\rfrmlr}{\vmr_{\twosbs}} - 
\frac{\vmr_{\onesbs}\rfrmlr_{\onesbs}}{\vmr_{\twosbs}} \nonumber \\
& = & \frac{\vlmmlr}{\vmr_{\twosbs}}
\left( \frac{\idxrfr^{2} -1}{\idxrfr^{2} + 2} \right)
- \frac{\vmr_{\onesbs}\vlmmlr_{\onesbs}}{\vmr_{\twosbs}} 
\left( \frac{\idxrfr_{\onesbs}^{2} -1}{\idxrfr_{\onesbs}^{2} + 2} \right)
\label{eqn:rfr_mlr_two_dfn}
\end{eqnarray}
Since all the quantities on the RHS are known, $\rfrmlr_{\twosbs}$ is
determined. 
\cite{Ste90} show that the $\rfrmlr_{\twosbs}$ determined from 
(\ref{eqn:rfr_mlr_two_dfn}) is generally independent of ionic
strength. 

In modeling studies, often the mass fractions of the various species
are known (or predicted) and it is the mean refractive index of the
bulk solution that is of interest. 
Inverting (\ref{eqn:rfr_mlr_dfn}) to obtain $\idxrfr$ in terms of
$\rfrmlr$ and $\vlmmlr$ we obtain
\begin{eqnarray}
% Ste90 p. 1677 (5)
\idxrfr & = & \left(
\frac{1 + 2 \rfrmlr / \vlmmlr}{1 - \rfrmlr / \vlmmlr} \right)^{1/2}
\label{eqn:idx_rfr_pmr_dfn}
\end{eqnarray}
Thus the quantity $\rfrmlr / \vlmmlr$ is seen to determine the
optical properties of the bulk solution.
We may rewrite $\rfrmlr / \vlmmlr$ in terms of the physical volume
$\vlm$ [\kgxmC] of the bulk solution, and the mass fraction
$\mssfrc_{\spcidx}$ and density $\dns_{\spcidx}$ of each component.
Combining (\ref{eqn:vlm_mlr_dfn2}) and (\ref{eqn:rfr_mlr_dfn2}) we
obtain
\begin{eqnarray}
% Ste90 p. 1677 (6)
\frac{\rfrmlr}{\vlmmlr} 
& = & 
\frac{\sum_{\spcidx = 1}^{\spcnbr} \vmr_{\spcidx} \rfrmlr_{\spcidx} }
 { \sum_{\spcidx = 1}^{\spcnbr} \vmr_{\spcidx} \mmw_{\spcidx} / \dns }
\nonumber \\
& = & 
\label{eqn:rfr_ovr_vlm_dfn}
\end{eqnarray}

\section[Sulfate Aerosols]{Sulfate Aerosols}\label{sxn:idx_rfr_SO4}
According to \cite[]{KSR00}
\begin{eqnarray}
% KSR00
\frac{\extspcwet}{\extspc} & = & 
\exp \left( \ccc_{1} + \frac{\ccc_{2}}{\ccc_{3} + \RH} + \frac{\ccc_{4}}{
\ccc_{5} + \RH} \right)
\label{eqn:hyg_grw_SO4_pzn}
\end{eqnarray}
For \HdSOq\ with $\dmtnma = 0.1$\,\um\ and $\ln \gsd = 0.7$, 
\cite{KSR00} showed $[\ccc_{1}, \ldots, \ccc_{5}] =
[11.24,-0.304,-1.088,-177.6,15.37]$. 

\section[Radiative Heating of Particles]{Radiative Heating of Particles}\label{sxn:prt_htg}
Due to human influence the fraction of Earth's aerosol composed of
absorbing substances, especially carbon, is constantly increasing.
It is therefore of interest to quantify the aerosol-induced heating.
In highly absorptive atmospheric conditions, it may be possible
to use sophisticated instruements to determine aerosol properties such
as the \trmidx{single scattering albedo} from the measured temperature
change resulting from radiant heating. 

Net particle heating $\htgprt$ is the result of latent, sensible, and 
radiant heating, each a complex microphysical process.
Condensation and deposition of vapor to the surface of a particle
cause \trmdfn{latent heating}.
Evaporation and sublimation of vapor from the surface of a particle
cause \trmdfn{latent cooling}.
Thus latent heating, denoted $\htgltn$, requires mass transfer to or
from the particle surface.
Thermal conductance, that is, the transfer of heat to or from the
surface of the particle by molecular diffusion, is called
\trmdfn{sensible heating} and denoted by $\htgsns$.
Absorption and emission of radiant energy by a particle is called
\trmdfn{radiant heating}, denoted $\htgrdn$.
\begin{eqnarray}
\htgprt & = & \htgltn + \htgsns + \htgrdn
\label{eqn:htg_prt_dfn}
\end{eqnarray}

We now briefly digress to discuss the physical units employed in this
section. 
To understand aerosol heating it is convenient to work in terms of
\trmdfn{power}, or energy per unit time.
Thus we shall express $\htgprt$, $\htgltn$, $\htgsns$, and $\htgrdn$
in \jxs\ or Watts.
It is technically correct to say that these $\htg_{\xxx}$ measure a
\trmdfn{heating rate}, i.e., the rate at which heat, in any of its
forms, is transferred.
However, in the literature the terminology ``heating rate'' is
generally used for quantities measured in temperature change per unit
time, e.g., \kxs\ or \kxday.

Meteorological models often predict heating rates related, but not
identical to $\htg_{\xxx}$.
Most often it is the heatings per unit volume, $\htgvlm_{\xxx}$, 
\jxmCs\ = \wxmC, that are available.
$\htgvlm$ is the integral of the particle heating weighted
by the particle distribution in a unit volume,
\begin{eqnarray}
\htgvlm = \int \dstnbrofdmt \htg (\dmt) \,\dfr\dmt
\end{eqnarray}
Heatings per unit mass are also used.
These are denoted by $\htgmss_{\xxx}$ in units of \jxkgs\ or \wxkg.
The conversion between volumetric and specific heating is 
$\htgvlm = \dns \htgmss$ where $\dns$ is the density.

\subsection{Latent Heating}\label{sxn:ltn_htg}
The heat budget (\ref{eqn:htg_prt_dfn}) of wetted particles is driven
by the latent heating which occurs as particles constantly adjust to
changing humidity in their environment by seeking a thermodynamically
stable size.
During the continual processes of condensation and evaporation any net
change of mass changes the latent heat stored by the particle so that
\begin{eqnarray}
\htgltn & = & \ltnspc \frac{\dfr\mss}{\dfr\tm}
\label{eqn:htg_ltn_dfn1}
\end{eqnarray}
The rate of diffusional growth of a particle depends on $\dnsvprdlt$,
the difference between the vapor density at the surface of the
particle, $\dnsvprrds$ and the vapor density far from the surface, in
the surrounding medium, $\dnsvprinf$. 
\begin{eqnarray}
\dnsvprdlt & = & \dnsvprinf - \dnsvprrds
\label{eqn:dns_vpr_dlt_dfn}
\end{eqnarray}
The solution to the vapor diffusion equation (see \S\ref{sxn:cnt}) for
a spherical particle yields 
\begin{eqnarray}
\frac{\dfr\mss}{\dfr\tm} & = & 4 \mpi \rds \vntmss \dffvpr 
( \dnsvprinf - \dnsvprrds ) 
\label{eqn:mss_dff_sln}
\end{eqnarray}
where $\vntmss$ is the \trmdfn{mass ventilation coefficient}.
The mass ventilation coefficient $\vntmss$ is of order unity and
accounts for the alteration of vapor convection and diffusion to the
particle due to particle motion.
It can be shown that the limiting behavior of $\vntmss$ is
characterized by the dimensionless product 
$\shmnbrvpr^{1/3} \rynnbr^{1/2}$, 
where $\rynnbr$ is the particle \trmidx{Reynolds number}
(\ref{eqn:ryn_nbr_dfn}) and $\shmnbrvpr$ is the \trmidx{Schmidt
number} of vapor in air. 
Empirical parameterizations \cite[see][p.~541]{PrK98} show
\begin{eqnarray}
% PrK98 p. 541 (13-60)
\vntmss & = & \left\{
\begin{array}{l@{\quad:\quad}r}
1
& \rynnbr \rightarrow 0 \\
1.0 + 0.108 (\shmnbrvpr^{1/3} \rynnbr^{1/2})^{2} 
& \shmnbrvpr^{1/3} \rynnbr^{1/2} < 1.4 \\
0.78 + 0.38 \shmnbrvpr^{1/3} \rynnbr^{1/2}
& 1.4 \le \shmnbrvpr^{1/3} \rynnbr^{1/2} \le 51.4
\end{array} \right.
\label{eqn:vnt_mss_dfn}
\end{eqnarray}
Thus whether the dependence on $\shmnbrvpr^{1/3} \rynnbr^{1/2}$ is
quadratic or linear depends on $\shmnbrvpr^{1/3} \rynnbr^{1/2}$
itself. 

By analogy to (\ref{eqn:shm_nbr_dfn}) we have
\begin{eqnarray}
\shmnbrvpr & = & \vscknm / \dffHdOair
\label{eqn:shm_nbr_vpr_dfn}
\end{eqnarray}
where $\vscknm$ is the \trmidx{kinematic viscosity of air} and 
$\dffHdOair$, the \trmidx{diffusivity of vapor in air}, is given by
(\ref{eqn:dff_H2O_air}).

Combining (\ref{eqn:mss_dff_sln}) with (\ref{eqn:htg_ltn_dfn1}) 
yields 
\begin{eqnarray}
\htgltn & = & 4 \mpi \rds \vntmss \dffvpr \ltnspc \dnsvprdlt
\label{eqn:htg_ltn_dfn2}
\end{eqnarray}
When curvature effects are negligible and the particle is
(at least coated with) liquid water, the surface vapor pressure  
is the \trmidx{saturated vapor pressure} over planar surfaces,
$\dnsvprrds = \dnsvprsat(\tptprt)$. 

The \trmidx{Ideal Gas Law} (\ref{eqn:igl_vpr_dfn}) states that 
$\dnsvprsat(\tptprt) = \prsprtwtrsat / ( \gascstvpr \tptprt )$ 
so that $\htgltn$ depends on $\tptprt$ directly, and implicitly
through $\prsprtwtrsatsfc$ (\ref{eqn:klv_law_dfn}).
The temperature dependence can be factored out of
(\ref{eqn:htg_ltn_dfn2}) by expressing the vapor density gradient in
terms of $\tptdlt$
\begin{eqnarray}
\dnsvprinf - \dnsvprrds 
& = & 
\dnsvprinf - \dnsvprsat ( \tptprt )
\nonumber \\ 
& = & 
\dnsvprsat ( \tptinf ) - \dnsvprsat ( \tptprt ) +
\dnsvprinf - \dnsvprsat ( \tptinf )
\nonumber \\ 
& = & 
\frac{\dnsvprsat ( \tptinf ) - \dnsvprsat ( \tptprt )}{\tptinf - \tptprt} 
\times ( \tptinf - \tptprt ) +
\dnsvprinf - \dnsvprsat ( \tptinf )
\nonumber \\ 
& \approx & 
\left( \overline{\frac{\dfr\dnsvpr}{\dfr\tpt}} \right)_{\mathrm{sat}} 
\tptdlt +
\dnsvprsat ( \tptinf ) 
\left( \frac{\dnsvprinf}{\dnsvprsat ( \tptinf )} - 1 \right)
\nonumber \\ 
& \approx & 
\left( \overline{\frac{\dfr\dnsvpr}{\dfr\tpt}} \right)_{\mathrm{sat}} 
\tptdlt -
( 1 - \RHinf ) \dnsvprsat ( \tptinf ) 
\label{eqn:dns_vpr_apx}
\end{eqnarray}
where $\RHinf$ is the relative humidity far from the particle.
In the penultimate step we follow \cite{PrK98} and assume the vapor
density gradient may be approximated by the mean slope of the
saturation vapor density curve evaluated somewhere in 
$\tpt \in [\tptprt, \tptinf]$.
This approximation is valid for the small $\tptdlt$ typically found in
atmospheric particles and it is exact in the limit
$\tptdlt \rightarrow 0$.  
The approximation (\ref{eqn:dns_vpr_apx}) allows us to rewrite 
(\ref{eqn:htg_ltn_dfn2}) as 
\begin{eqnarray}
\htgltn & = & 4 \mpi \rds \vntmss \dffvpr \left[
\left( \overline{\frac{\dfr\dnsvpr}{\dfr\tpt}} \right)_{\mathrm{sat}} 
\ltnspc \tptdlt
 - ( 1 - \RHinf ) \dnsvprsat ( \tptinf ) \right]
\label{eqn:htg_ltn_dfn3}
\end{eqnarray}
Casting $\htgltn$ into a form with a linear dependence on $\tptdlt$ 
(\ref{eqn:htg_ltn_dfn3}) plus a constant will prove helpful in solving
for $\tptdlt$ in \S\ref{sxn:tpt_prt}.

It is important to bear in mind that diffusional growth
(\ref{eqn:mss_dff_sln}) is only one of several limiting growth
processes, others being surface- and volume-dependent chemical
reactions \cite[][p.~685]{SeP97}.
However, we shall neglect the effects of chemistry on 
particle mass and on \trmdfn{particle temperature}, $\tptprt$.

% LGGE
\subsubsection{Thermal Conductivity of Snow}\label{sxn:cnd_trm_snw}
The \trmidx{thermal conducitivity of snow} controls the efficiency of 
snowpack sensible heat flux.
\citet{Jor91} uses
\begin{eqnarray}
% ODB04 p. 94 (6.65)
\cndtrmsnw & = & \cndtrmair +
(7.75 \times 10^{-5} \dnssnw + 1.105 \times 10^{-6} \dnssnw^{2})
(\cndtrmice - \cndtrmair)
\label{eqn:cnd_trm_snw}
\end{eqnarray}
where $\cndtrmsnw$, $\cndtrmice$, and $\cndtrmair$ are the thermal
conductivity of snow, ice, and air, respectively.
% ODB04 p. 95 (6.69)
The heat capacity of snow is the sume of the heat capacities of its
liquid and ice components.

\subsection{Sensible Heating}\label{sxn:sns_htg}
Heat diffusion and mass diffusion (\S\ref{sxn:ltn_htg}) from a
particle are exact mathematical analogues of eachother.
Thus \trmidx{Fick's First Law} (\ref{eqn:fck_1st_law}) suggests that
the rate of sensible heat transfer is proportional to the  
the difference between the surface temperature of the particle,
$\tptprt$, and the temperature of the surrounding medium, $\tptinf$.
\begin{eqnarray}
\tptdlt & = & \tptinf - \tptprt
\label{eqn:tpt_dlt_dfn}
\end{eqnarray}
The sign convention for $\tptdlt$ is arbitrary and we chose
(\ref{eqn:tpt_dlt_dfn}) to follow \cite{PrK98}.  
Heat flows to the particle when $\tptdlt > 0$, from the particle when
$\tptdlt < 0$.
In complete analogy to (\ref{eqn:mss_dff_sln}), the solution to the heat
diffusion equation for a spherical particle is
\begin{eqnarray}
% PrK98 p. 544 (13-65)
\htgsns & = & 4 \mpi \rds \vnttrm \cndtrmair \tptdlt
\label{eqn:htg_sns_dfn}
\end{eqnarray}
where $\cndtrmair$~(\wxmk) is the \trmdfn{thermal conductivity of dry air}
\cite[][p.~508]{PrK98} 
\begin{eqnarray}
% PrK98 p. 508 (13-18a)
\cndtrmair 
\csznote{& = & 1.0 \time 10^{-3} \times 
4.1855 \times [ 5.69 + 0.017 ( \tpt - 273.15 ) ] \nonumber \\
& = & 0.023815 + 7.12 \times 10^{-5} ( \tpt - 273.15 ) \nonumber \\
& = & 0.023815 + 7.12 \times 10^{-5} \tpt - 0.0194356 \nonumber \\}
& = & 0.0043794 + 7.12 \times 10^{-5} \tpt
\label{eqn:cnd_trm_air_dfn}
\end{eqnarray}
Related to $\cndtrmair$ is the \trmdfn{thermal diffusivity of air}
$\dfftrmair$~(\mSxs) \cite[][p.~507]{PrK98} 
\begin{eqnarray}
% PrK98 p. 507 (13-17)
\dfftrmair & = & \cndtrmair / ( \dnsatm \heatcpcspcprs )
\label{eqn:dff_trm_air_dfn}
\end{eqnarray}

The sign convention in (\ref{eqn:htg_sns_dfn}) ensures that particles
get warmer ($\htgsns > 0$) when $\tptinf > \tptprt$ and heat is 
conducted to the particle.
Thus $\htgsns$ (\ref{eqn:htg_sns_dfn}) and $\htgltn$
(\ref{eqn:htg_ltn_dfn3}) both depend linearly on $\tptdlt$.

The \trmdfn{thermal ventilation coefficient} $\vnttrm$ is a
factor of order unity which accounts for the effects of particle
motion on heat conduction. 
Since heat diffusion and mass diffusion are mathematically analogous,  
$\vnttrm$ is usually expressed in terms of $\vntmss$.
To obtain $\vnttrm$ from (\ref{eqn:vnt_mss_dfn}), we simply 
replace the mechanical diffusivity $\dffHdOair$ by the thermal
diffusivity of air $\dfftrmair$ (\ref{eqn:dff_trm_air_dfn}) in
(\ref{eqn:shm_nbr_vpr_dfn}) to obtain a Schmidt number for thermal
diffusion $\shmnbrtrm$
\begin{eqnarray}
% PrK98 p. 541
\shmnbrtrm & = & \vscknm / \dfftrmair
\label{eqn:shm_nbr_trm_dfn}
\end{eqnarray}
Using $\shmnbrtrm$ in place of $\shmnbrvpr$ (\ref{eqn:vnt_mss_dfn}),
we obtain
\begin{eqnarray}
% PrK98 p. 541 (13-60)
\vnttrm & = & \left\{
\begin{array}{l@{\quad:\quad}r}
1
& \rynnbr \rightarrow 0 \\
1.0 + 0.108 (\shmnbrtrm^{1/3} \rynnbr^{1/2})^{2} 
& \shmnbrtrm^{1/3} \rynnbr^{1/2} < 1.4 \\
0.78 + 0.38 \shmnbrtrm^{1/3} \rynnbr^{1/2}
& 1.4 \le \shmnbrtrm^{1/3} \rynnbr^{1/2} \le 51.4
\end{array} \right.
\label{eqn:vnt_trm_dfn}
\end{eqnarray}

Radiant heating $\htgrdn$ is generally of secondary importance for
$\tptaer$, because $\htgrdn \ll \htgltn$.
When this is the case, the net particle heating is a balance 
of latent and sensible heat transfer.
A growing cloud droplet, for example, condenses more water vapor 
than it evaporates. 
The excess latent heat of condensation is ultimate conducted to the
atmosphere by thermal diffusion, i.e., sensible heat transfer.
The balance of latent and sensible heating is described in many texts  
\cite[][p.~447]{PrK78},\cite[][p.~542]{PrK98},\cite[][p.~103]{RoY94}.

\subsection{Radiant Heating}\label{sxn:rdn_htg}
Radiant heating $\htgrdn$ may be very important in the heat balance of 
dry particles since $\htgltn \approx 0$, and, as shown below, 
for particles not in \trmidx{thermal equilibrium}.
An experimental apparatus may employ powerful radiant heating
techniques such as lasers to probe aerosols. 

There are many approaches to the problem of determining $\htgrdn$.
Under the assumptions of the Mie approximation, $\htgrdn$ depends on
the mean intensity $\ntnmnfrq$ or actinic flux intercepted by the
particle and on the surface temperature of the particle $\tptprt$ but
not on the orientation of the particle in space.
A particle with equivalent-sphere radius $\rdssph$ will radiate
isotropically and absorb incident radiation as a perfect blackbody
modulated by its \trmdfn{absorption efficiency} $\fshabs$
\begin{eqnarray}
\htgrdn(\rdssph) & = & 4\mpi \int_{0}^{\infty} 
\mpi \rdssph^{2} \fshabs( \rdssph, \wvl ) 
[ \ntnwvlbar(\wvl) - \plkwvl(\tptprt,\wvl) ] \,\dfr\wvl
\label{eqn:htg_rdn_dfn}
\end{eqnarray}
where $\plkwvl$ is the \trmdfn{Planck function}.
Thus $\htgrdn$ depends explicitly on $\tptprt$ through $\plkwvl$.
If we decompose $\ntnwvlbar(\wvl)$ into the sum of a collimated
light source of strength $\flx(\wvl)$ (e.g., the sun) and a mean
intensity from the diffuse field (which does not include the
collimated source) of strength $\ntnwvldffbar$.
Then we have
\begin{eqnarray}
\ntnwvlbar( \wvl ) & = & \flx( \wvl ) + \ntnwvldffbar( \wvl ) \\
\htgrdn(\rdssph) & = & \int_{0}^{\infty} 
\mpi \rdssph^{2} \fshabs ( \rdssph, \wvl ) 
[ \flx( \wvl ) + 
4 \mpi \ntnwvldffbar ( \wvl ) - 4\mpi \plkwvl ( \tptprt, \wvl) ] \,\dfr\wvl
\label{eqn:htg_rdn_dfn2}
\end{eqnarray}
Thus $\htgrdn$ is itself the net result of radiant heating, caused by
$\flx$ and $\ntnwvldffbar$, compensated by radiant cooling, caused
by $\plkwvl$.
Neglecting $\flx$ for the time being, (\ref{eqn:htg_rdn_dfn2})
makes clear that the largest radiative contributions to $\htgrdn$
occur where the radiation field is out of thermal equilibrium, i.e.,
where $\ntnwvldffbar \ne \plkwvl$.
At cloud top, for instance, 
$\ntnwvldffbar \approx \frac{1}{2} \plkwvl$ \cite[][]{Bar78}.

We may further simplify (\ref{eqn:htg_rdn_dfn2}) by considering the
case of a spectrally uniform, collimated source, e.g., a laser.
In this case $\flx( \wvl ) = \flx(\wvlnot) \equiv \flxnot$.
We shall assume that the diffuse radiation field is in thermal
equilibrium at all wavelengths. 
Finally we shall drop $\rdssph$ in favor of $\rds$ for simplicity, 
with the understanding $\rdssph$ should be used in practice, where
appropriate.
With $\ntnwvlbar = \plkwvl$, (\ref{eqn:htg_rdn_dfn2}) reduces to
\begin{eqnarray}
\htgrdn(\rds) & = & \mpi \rds^{2} \flxnot \fshabs ( \rds, \wvlnot )
\label{eqn:htg_rdn_dfn3}
\end{eqnarray}
The validity of the assumption of thermal equilibrium depends on the
application, but we note that radiant cooling must become significant
when $\tptprt \gg \tptinf$. 

\subsection{Particle Temperature Evolution}\label{sxn:tpt_prt}
Substituting (\ref{eqn:htg_ltn_dfn2}), (\ref{eqn:htg_sns_dfn}), and
(\ref{eqn:htg_rdn_dfn}) into (\ref{eqn:htg_prt_dfn}) we obtain
\begin{eqnarray}
\htgprt & = & 
4 \mpi \rds \vntmss \dffvpr \ltnspc \dnsvprdlt +
4 \mpi \rds \vnttrm \cndtrmair \tptdlt + 
\htgrdn
\label{eqn:htg_prt_dfn2}
\end{eqnarray}

The heat gained or lost by the particle when $\htgprt \ne 0$
(\ref{eqn:htg_prt_dfn}) causes $\tptprt$ to change.
The heating will thus cause $\tptdlt$ to change at a rate
proportional the particle mass $\mss$ and the 
\trmidx{specific heat at constant pressure} $\heatcpcspcprs$ 
of the particle
\begin{eqnarray}
\htgprt & = & \mss \heatcpcspcprs \frac{\dfr\tptprt}{\dfr\tm} \nonumber \\ 
& = & - \mss \heatcpcspcprs \frac{\dfr\tptdlt}{\dfr\tm}
\label{eqn:htg_prt_dfn3}
\end{eqnarray}
where $\dfr\tptdlt = - \dfr\tptprt$ has been substituted from
(\ref{eqn:tpt_dlt_dfn}) under the assumption that $\tptinf$ is fixed. 

Setting (\ref{eqn:htg_prt_dfn2}) equal to (\ref{eqn:htg_prt_dfn3}) 
we obtain
\begin{eqnarray}
- \mss \heatcpcspcprs \frac{\dfr\tptdlt}{\dfr\tm} & = & 
4 \mpi \rds \vntmss \dffvpr \ltnspc \dnsvprdlt +
4 \mpi \rds \vnttrm \cndtrmair \tptdlt + \htgrdn \nonumber \\
\frac{\dfr\tptdlt}{\dfr\tm} 
& = & 
- \frac{4 \mpi \rds \vntmss \dffvpr \ltnspc}{\mss \heatcpcspcprs} \dnsvprdlt
- \frac{4 \mpi \rds \vnttrm \cndtrmair}{\mss \heatcpcspcprs} \tptdlt
- \frac{\htgrdn}{\mss \heatcpcspcprs} \nonumber \\
& = & 
- \frac{4 \mpi \rds (\vntmss \dffvpr \ltnspc \dnsvprdlt + 
\vnttrm \cndtrmair \tptdlt)}{\mss \heatcpcspcprs} 
- \frac{\htgrdn}{\mss \heatcpcspcprs} \nonumber \\
& = & 
- \frac{3 (\vntmss \dffvpr \ltnspc \dnsvprdlt + 
\vnttrm \cndtrmair \tptdlt)}{\rds^{2} \dnsprt \heatcpcspcprs} 
- \frac{\htgrdn}{\mss \heatcpcspcprs}
\label{eqn:tpt_dlt_dfr}
\end{eqnarray}
The RHS of (\ref{eqn:tpt_dlt_dfr}) implicitly depends on
$\tptdlt$ both through $\dnsvprdlt$ (\ref{eqn:dns_vpr_apx})
and through $\htgrdn$ (\ref{eqn:htg_rdn_dfn}).
To solve (\ref{eqn:tpt_dlt_dfr}) analytically, we introduce the
approximation for $\dnsvprdlt$ (\ref{eqn:dns_vpr_apx}) into
(\ref{eqn:tpt_dlt_dfr}) 
\begin{eqnarray}
\frac{\dfr\tptdlt}{\dfr\tm} 
& = & 
- \frac{3 [\vntmss \dffvpr \ltnspc 
\left( \overline{\frac{\dfr\dnsvpr}{\dfr\tpt}} \right)_{\mathrm{sat}} +
\vnttrm \cndtrmair ]}{\rds^{2} \dnsprt \heatcpcspcprs} \tptdlt
+ \frac{3 \vntmss \dffvpr \ltnspc ( 1 - \RHinf ) \dnsvprsat ( \tptinf )}
{\rds^{2} \dnsprt \heatcpcspcprs} 
- \frac{\htgrdn}{\mss \heatcpcspcprs}
\label{eqn:tpt_dlt_dfr_apx}
\end{eqnarray}

For dry particles, which cannot participate in latent heating, 
all factors containing $\ltnspc$ are zero and we obtain the much simpler
relation
\begin{eqnarray}
\frac{\dfr\tptdlt}{\dfr\tm} 
& = & 
- \frac{3 \vntmss \dffvpr \ltnspc \vnttrm \cndtrmair}
{\rds^{2} \dnsprt \heatcpcspcprs} \tptdlt
- \frac{\htgrdn}{\mss \heatcpcspcprs}
\label{eqn:tpt_dlt_dfr_dry}
\end{eqnarray}
The two terms on the RHS represent sensible and radiative heat
transfer, respectively. 

For the time being we shall neglect the dependence of $\htgrdn$ on
$\tptdlt$ and treat (\ref{eqn:tpt_dlt_dfr_apx}) as a 
linear, first order differential equation for $\tptdlt$ 
\begin{eqnarray}
\frac{\dfr\tptdlt}{\dfr\tm} + \AAA \tptdlt & = & \BBB \qquad \mbox{where} \\
\AAA 
& = & 
\frac{3 [\vntmss \dffvpr \ltnspc 
\left( \overline{\frac{\dfr\dnsvpr}{\dfr\tpt}} \right)_{\mathrm{sat}} +
\vnttrm \cndtrmair ]}{\rds^{2} \dnsprt \heatcpcspcprs}
\nonumber \\
& = & 
\frac{4 \mpi \rds [\vntmss \dffvpr \ltnspc 
\left(\overline{\frac{\dfr\dnsvpr}{\dfr\tpt}} \right)_{\mathrm{sat}} +
\vnttrm \cndtrmair ]}{\mss \heatcpcspcprs}
\nonumber \\
\BBB & = & 
\frac{3 \vntmss \dffvpr \ltnspc ( 1 - \RHinf ) \dnsvprsat ( \tptinf )}
{\rds^{2} \dnsprt \heatcpcspcprs} 
- \frac{\htgrdn}{\mss \heatcpcspcprs} \nonumber
\label{eqn:tpt_dlt_dfr_apx_2}
\end{eqnarray}
The \AAA\ term and the \BBB\ term are independent, but $\BBB - \AAA$
is positive when the particle is losing mass, and negative when the
particle is gaining mass. 

The solution to (\ref{eqn:tpt_dlt_dfr_apx_2}) may be obtained by
multiplying each side by the \trmdfn{integrating factor} 
$\me^{\AAA \tm}$. 
Then the LHS side becomes is the perfect differential of $\me^{\AAA \tm}
\tptdlt$ and the RHS may be integrated by standard techniques.
The constant of integration is identified by imposing the
temperature gradient initial condition
$\tptdlt(\tm = 0) = \tptinf - \tptprt(\tm = 0) \equiv \tptdltnot$.
The result is 
\begin{eqnarray}
\tptdlt( \tm ) & = & \frac{\BBB}{\AAA} + 
\left( \tptdltnot - \frac{\BBB}{\AAA} \right) \me^{-\AAA \tm} 
\nonumber \\ 
& = & \tptdltnot \me^{-\AAA \tm} + 
\frac{\BBB}{\AAA} \left( 1 - \me^{-\AAA \tm} \right)
\label{eqn:tpt_dlt_dfr_sln_1}
\end{eqnarray}
It is instructive to examine the limiting cases of
(\ref{eqn:tpt_dlt_dfr_sln_1}). 
As $\tm \rightarrow \infty$, $\tptdlt \rightarrow \BBB/\AAA$.
Thus the particle asymptotes to a steady state temperature
difference with the environment given by $\BBB/\AAA$.

Inserting (\ref{eqn:tpt_dlt_dfr_apx_2}) into
(\ref{eqn:tpt_dlt_dfr_sln_1}) yields
\begin{eqnarray}
\frac{\BBB}{\AAA} & = & 
\frac{\vntmss \dffvpr \ltnspc ( 1 - \RHinf ) \dnsvprsat ( \tptinf )}
{\vntmss \dffvpr \ltnspc 
\left( \overline{\frac{\dfr\dnsvpr}{\dfr\tpt}} \right)_{\mathrm{sat}} +
\vnttrm \cndtrmair} -
\frac{\htgrdn}
{4 \mpi \rds [\vntmss \dffvpr \ltnspc 
\left( \overline{\frac{\dfr\dnsvpr}{\dfr\tpt}} \right)_{\mathrm{sat}} +
\vnttrm \cndtrmair ]}
\label{eqn:htg_std_stt_dfn}
\end{eqnarray}

The relaxation ($\me$-folding) time for the particle-environment heating
gradient to reach steady state in this (simplified) scenario is
$\tauhtg = \AAA^{-1}$. 
\begin{eqnarray}
\tptdlt( \tm ) & = & \frac{\BBB}{\AAA} + 
\left( \tptdltnot - \frac{\BBB}{\AAA} \right) \me^{- \tm / \tauhtg} 
\nonumber \\
\tauhtg & = & 
\frac{\rds^{2} \dnsprt \heatcpcspcprs}
{3 [\vntmss \dffvpr \ltnspc 
\left( \overline{\frac{\dfr\dnsvpr}{\dfr\tpt}} \right)_{\mathrm{sat}} +
\vnttrm \cndtrmair ]} \nonumber \\
& = & 
\frac{\mss \heatcpcspcprs}
{4 \mpi \rds [\vntmss \dffvpr \ltnspc 
\left(\overline{\frac{\dfr\dnsvpr}{\dfr\tpt}} \right)_{\mathrm{sat}} +
\vnttrm \cndtrmair ]}
\label{eqn:tpt_dlt_sln}
\end{eqnarray}
$\tauhtg$ is also called the \trmdfn{adaptation timescale}.
In the absence of latent heating processes, $\tptdlt$ and $\tauhtg$
depend only on conduction and radiant heating.
Thus the heating balance for dry particles (\ref{eqn:tpt_dlt_sln})
reduces to  
\begin{eqnarray}
\tptdlt( \tm ) & = & \frac{\BBB}{\AAA} + 
\left( \tptdltnot - \frac{\BBB}{\AAA} \right) \me^{- \tm / \tauhtg} 
\nonumber \\
\tauhtg & = & 
\frac{\rds^{2} \dnsprt \heatcpcspcprs}{3 \vnttrm \cndtrmair } \nonumber \\
& = & 
\frac{\mss \heatcpcspcprs}{4 \mpi \rds \vnttrm \cndtrmair }
\label{eqn:tpt_dlt_sln_dry}
\end{eqnarray}

\begin{eqnarray}
\fshsct & = & \fshabs / ( 1 - \ssa )
\label{eqn:fsh_sct_dfn}
\end{eqnarray}

\subsection{Temperature Adjustment}\label{sxn:tpt_adj}

\chapter{Gas Phase Chemistry}\label{sxn:chm_gas}

Mineral dust aerosol affects atmospheric chemistry through both
\trmdfn{heterogeneous} and \trmdfn{photochemical} processes.
Heterogeneous chemistry includes reactions occurring on the surface of
mineral aerosol.
Mineral dust affects atmospheric photochemistry by scattering and
absorbing sunlight and thus altering the actinic flux field.
Through these mechanisms, mineral dust may significantly alter
tropospheric concentrations of \SIV, \NOy, \Ot, and \OH.

\section[Literature Review]{Literature Review}\label{sxn:ltr_rvw_chm}
\cite{Bri962} is a good introductory treatment of air composition and
chemistry. 
\cite{CLG85} examined mechanisms of \HNOt\ and \HdOd\ formation in the
troposphere. 
\cite{ChD82} examined the effect of free radicals on the composition
of cloud water and rain.
\cite{ChS92} showed that the atmospheric cycles of sulfur and sea salt
aerosol are coupled.
\cite{GMW86} showed that transition metals can alter cloud droplet
chemical compositions.
\cite{GrG88} present a detailed model or raindrop chemistry.
\cite{HeL96} discussed the role of atmospheric dust in the acid rain
problem. 
The absorption cross sections and quantum yields of the most important
photochemical paths in atmospheric chemistry are summarized every few
years in \cite{JPL97}.
\cite{KlW96} discuss the formaldehyde content of atmospheric aerosol.
\cite{LLL96} develop and validate a three dimensional cloud chemistry
model. 
Chemistry in a rainband has been simulated by \cite{BHH92,Bar94,LYM90}
\cite{PWS95} present a thorough overview of the dynamics of
tropospheric aerosols.
\cite{PiL98} discuss the effects of particle shape and internal
inhomogeneity on radiative forcing.
\cite{SPG96} examined the role of natural aerosol variations in
anthropogenic ozone depletion.
\cite{MuB95} describe the development and validation of the global
chemical transport model IMAGES.
\cite{BHW98} describe the development and validation of the global
chemical transport model MOZART.
Column, in situ, and episodic aerosol radiative forcing has been
studied by many authors 
\cite[e.g.,][]{JLR98,LiS982,PTM98,SKR98}.
Aerosol forcing of climate has been studied by many authors
\cite[e.g.,][]{KiB93,VRN98,TLF96,MiT98a,MiT98b,SAK98,AKS98,Bou95,ChW94,DKS91,HaR98,LiS981,SFC97,THC97,WKB98}.
Global sulfur aerosol distributions have been studied by many authors
\cite[e.g.,][]{RBK00,BRK00,DKS91,THC97}.

\section[Oxidation States]{Oxidation States}
The \trmdfn{oxidation state} of an atom measures its valence, and thus
its reactivity, in a substance.
The oxidation state of an atom in a pure element is zero (e.g., \Ar).
An \trmdfn{ion} is a compound whose net charge is not zero.
In a monatomic ion the charge and the oxidation state are the same
For example, both the oxidation state and charge of \Hp\ are $+1$.
For homonuclear compounds formed by covalent bonds (e.g., \Nd, \Od),
the electrons are split evenly between the atoms.
For heteronuclear compounds formed by covalent bonds, the electrons
are captured by the elements with the stronger attraction.
For example, oxygen has a greater attraction for electrons than
hydrogen, and strips the electrons from the hydrogen.
In \HdO\, therefore, the oxidation state of hydrogen is $+1$ and that
of the oxygen is $-2$ (each hydrogen lost its electron to the oxygen).  
In non-metallic covalent compounds, hydrogen always loses its
electrons to other atoms and is assigned an oxidation state of $+1$.

Oxygen is nearly always assigned an oxidation state of $-2$ in
covalent compounds, e.g., \SOd, \NOd, \COd. 
One exception is for peroxides, e.g., \HdOd, where oxygen has an
oxidation state of $-1$.
The total number of electrons is conserved so that, for an
electrically neutral structure, the sum of the oxidation states equals
zero. 
For example, \HNOt\ has oxidation states of hydrogen, nitrogen, and
oxygen of $+1$, $+5$, and $-2$, respectively.
The sum of the oxidation states is 
$1 \times (1) + 1 \times (5) + 3 \times (-2) = 0$.
For an ion, the sum of the oxidation states equals the total charge of
the ion.
For example, the sulfate ion \SOqdm\ has oxidation states of
sulfur and oxygen of $+6$ and $-2$, respectively.
The sum of the oxidation states is $1 \times (+6) + 4 \times (-2) = -2$.

In the atmospheric literature, oxidation states are usually expressed
as Roman numerals.
Thus the oxidation states of sulfur and oxygen in \SOqdm\ are written
as $+$VI and $-$II, respectively. 
Often, the oxidation state of a compound determines its reactivity,
and thus is a good predictor of its atmospheric residence time.
For example, most anthropogenic sulfur is emitted to the atmosphere as
\SOd, which is an \SIV\ compound.
Through various \trmdfn{oxidation} reactions, \SIV\ is usually
transformed transformed to \SVI, which is highly stable.
For sulfur, the solubility of the compound increases with oxidation,
so that \SVI\ compounds are highly soluble. 
As a result, much of the \SVI\ in the atmosphere is in the form of
particles or droplets and its residence time is determined by dry and
wet deposition processes.

Table~\ref{tbl:chm_abb} lists the members of important families of
atmospheric chemicals.
\begin{table}
\begin{minipage}{\hsize} % Minipage necessary for footnotes KoD95 p. 110 (4.10.4)
\renewcommand{\footnoterule}{\rule{\hsize}{0.0cm}\vspace{-0.0cm}} % KoD95 p. 111
\begin{center}
\caption[Constituents of Important Chemical
Families]{\textbf{Constituents of Important Families of Atmospheric
Compounds}%
\footnote{\emph{Sources:}}%
\label{tbl:chm_abb}}   
\vspace{\cpthdrhlnskp}
\begin{tabular}{ >{$\ch}r<{$} l >{$\ch}p{22.0em}<{$} }
\hline \rule{0.0ex}{\hlntblhdrskp}% 
Family & Name & Members \\[0.0ex]
\hline \rule{0.0ex}{\hlntblntrskp}%
\CCly & Organic chlorine & \CFClt\ \mbox{(CFC-11)} + \CFdCld\
\mbox{(CFC-12)} \, + \\
& & \CFCldCFdCl\ + \mbox{(CFC-113)} + \CClq + \CFdHCl +
\CHtCl \\[0.5ex] % SeP97 p. 179  
CFCs & Chlorofluorocarbons & \mbox{CFC-11}, \mbox{CFC-12},
\mbox{CFC-113}, \mbox{CFC-114}, \mbox{CFC-115}
\\[0.5ex] % SeP97 p. 88
\ClOx & Oxides of chlorine & \ClO \\[0.5ex] % SeP97 p. 177
\Cly & Inorganic chlorine & \Cl + 2\Cld + \ClO + \OClO + 2\CldOd +
\HOCl + \HCl + \BrCl + \ClONOd \\[0.5ex] % SeP97 p. 177
HCFCs & Hydrochlorofluorocarbons & \mbox{HCFC-22, HCFC-123,
HCFC-124} \\[0.5ex] % SeP97 p. 88
HFCs & Hydrofluorocarbons & \\[0.5ex] 
\HOx & Hydrogen radicals & \Hu, \OH, \HOd \\[0.5ex] % SeP97 p. 57, p. 171 
\NOx & Oxides of nitrogen & \NO, \NOd \\[0.5ex] % SeP97 p. 71
\NOx & Odd nitrogen family & \N + \NO + \NOd + \NOt + 2 \times \NdOc +
\HOdNOd \\[0.5ex] % BrS86  p. ???
\NOy & Reactive odd nitrogen & \NOx + \NdOc, \HNOt, \NOt, \HNOq,
\HNOd, PAN \\[0.5ex] % SeP97 p. 71
\Ox & Odd oxygen & \OtP, \OsD, \Ot \\[0.5ex] % SeP97 p. 165, BrS86 p. 204
\SOx & Oxides of sulfur & \SOd, \SOq \\[0.5ex] % DCZ96 p. 22869
\hline
\end{tabular}
\end{center}
\end{minipage}
\end{table}

Table~\ref{tbl:oxd_S} lists the oxidation states of important
atmospheric sulfur compounds.
\begin{table}
\begin{minipage}{\hsize} % Minipage necessary for footnotes KoD95 p. 110 (4.10.4)
\renewcommand{\footnoterule}{\rule{\hsize}{0.0cm}\vspace{-0.0cm}} % KoD95 p. 111
\begin{center}
\caption[Oxidation States Of Sulfur Species]{\textbf{Oxidation States of
Atmospheric Sulfur Species}%
\footnote{\emph{Sources:}}%
 \label{tbl:oxd_S}}
\vspace{\cpthdrhlnskp}
\begin{tabular}{ r l >{$}r<{$} }
\hline \rule{0.0ex}{\hlntblhdrskp}% 
Formula & Name & \mbox{Oxidation} \\[0.0ex]
& & \mbox{state} \\[0.0ex]
\hline \rule{0.0ex}{\hlntblntrskp}%
% SeP97 p. 57
\HdS & Hydrogen sulfide & -2 \\[0.0ex]
\DMS & Dimethyl sulfide (DMS) &  \\[0.0ex]
\CSd & Carbon disulfide &  \\[0.0ex]
\OCS & Carbonyl sulfide &  \\[0.5ex]
\SOd & Sulfur dioxide & 4 \\[0.0ex]
\HSOtm & Bisulfite ion &  \\[0.0ex]
\SOtdm & Sulfite ion &  \\[0.5ex]
\HdSOq & Sulfuric acid & 6 \\[0.0ex]
\HSOqm & Bisulfate ion &  \\[0.0ex]
\SOqdm & Sulfate ion & \\[0.0ex]
\MSA & Methane sulfonic acid (MSA) &  \\[0.5ex]
\hline
\end{tabular}
\end{center}
\end{minipage}
\end{table}

Table~\ref{tbl:oxd_N} lists the oxidation states of important
atmospheric nitrogen compounds.
\begin{table}
\begin{minipage}{\hsize} % Minipage necessary for footnotes KoD95 p. 110 (4.10.4)
\renewcommand{\footnoterule}{\rule{\hsize}{0.0cm}\vspace{-0.0cm}} % KoD95 p. 111
\begin{center}
\caption[Oxidation States of Nitrogen Species]{\textbf{Oxidation
States of Atmospheric Nitrogen Species}%
\footnote{\emph{Sources:}}% 
\label{tbl:oxd_N}}   
\vspace{\cpthdrhlnskp}
\begin{tabular}{ r l >{$}r<{$} }
\hline \rule{0.0ex}{\hlntblhdrskp}% 
Formula & Name & \mbox{Oxidation} \\[0.0ex]
& & \mbox{state} \\[0.0ex]
\hline \rule{0.0ex}{\hlntblntrskp}%
% SeP97 p. 56
\NHt & Ammonia & -3 \\[0.5ex]
\Nd & Nitrogen & 0 \\[0.5ex]
\NdO & Nitrous oxide & 1 \\[0.5ex]
\NO & Nitric oxide & 2 \\[0.5ex]
\HNOd & Nitrous acid & 3 \\[0.5ex]
\NOd & Nitrogen dioxide & 4 \\[0.5ex]
\HNOt & Nitric acid & 5 \\[0.0ex]
\NOtm & Nitrate radical &  \\[0.0ex]
\NdOc & Dinitrogen pentoxide &  \\[0.5ex]
\NOt & Nitrate & 6 \\[0.5ex]
\hline
\end{tabular}
\end{center}
\end{minipage}
\end{table}

\section{Stratospheric Chemistry}

Ozone is a very important molecule.
\begin{rxnarray}
\label{rxn:O2_OpO}
\Od + h\nu & \yields & \Ou + \Ou \\ % SeP97 p. 164 (4.1)
\label{rxn:OpO2_O3}
\Ou + \Od + \M & \yields & \Ot + \M \\ % SeP97 p. 164 (4.2)
\label{rxn:O3_OpO2}
\Ot + h\nu & \yields & \Ou + \Od \\ % SeP97 p. 164 (4.3)
\label{rxn:O3pO_O2pO2}
\Ot + \Ou & \yields & \Od + \Od % SeP97 p. 164 (4.4)
\end{rxnarray}
Reactions (\ref{rxn:O2_OpO})-(\ref{rxn:O3pO_O2pO2}) were first
proposed as the mechanism for ozone production in the strosphere by
Sidney Chapman in 1930; they are now called the \trmdfn{Chapman
mechanism}.  
\M\ is a third body which mediates reaction (\ref{rxn:OpO2_O3}).
Usually \M\ is an \Od\ or \Nd\ which gives some of its thermal energy
to facilitate the reaction.

The Chapman mechanism is a useful too for demonstrating many of the
concepts of atmospheric chemical cycles, including photochemistry,
second-order reactions, rate limiting steps, diurnal cycles, and
chemical families.  

By the 1960s it became clear that the Chapman mechanism alone
overpredicts the abundance of ozone in the stratosphere. 
Additional ozone destruction mechanisms are necessary to bring
observations into agreement with the predictions of 
(\ref{rxn:O2_OpO})-(\ref{rxn:O3pO_O2pO2}). 
In the 1970s a number of catalytic mechanisms for ozone destruction
were proposed 
\begin{rxnarray}
\label{rxn:XpO3_XOpO2}
\X + \Ot & \yields & \XO + \Od \\ % SeP97 p. 170 (4.7)
\label{rxn:XOpO_XpO2}
\XO + \Ou & \yields & \X + \Od \\ % SeP97 p. 170 (4.8)
\mbox{Net: } \Ot + \Ou & \yields & \Od + \Od % SeP97 p. 170
\end{rxnarray}

\section{Sulfur Chemistry}
The anthropogenic contribution to the total sulfur burden of the
atmosphere is greater than XXX\%.
Sulfur is important because \ldots.

\subsection[Sulfur Emissions]{Sulfur Emissions}
\cite{BSP96} provides a global data of sulfur emissions.

\subsection[Gas Phase Sulfur Chemistry]{Gas Phase Sulfur Chemistry}
Gas phase oxidation of \SOd\ occurs in the presence of the \OH\
radical and water vapor
\begin{rxnarray}
\SOd + \OH(+\Od) & \yields & \SOt + \HOd \\ % DCZ96 p. 22872 (13a)
\SOt + \HdO & \yields & \HdSOq % DCZ96 p. 22872 (13b)
\label{rxn:oxd_SO2_gas}
\end{rxnarray}

\section{Nitrogen Chemistry}
Approximately 80\% of Earth's atmosphere by weight is nitrogen in the
form of \Nd. 
Nitrogen is crucial source of energy to the biosphere, but most
organisms cannot utilize \Nd\ directly.
Instead, these biochemical processes require nitrogen that has been
\trmdfn{fixed}, or converted to a chemically useful state.
The next most abundant compounds of nitrogen in the atmosphere are
nitrous oxide (\NOd) and ammonia (\NHt).

\subsection[Nitrogen Emissions]{Nitrogen Emissions}
\cite{BSP96} provides a global data of nitrogen emissions.

\subsection[Gas Phase Nitrogen Chemistry]{Gas Phase Nitrogen Chemistry}
In the troposphere, the photochemistry of nitrogen oxides provides the
main source of atomic oxygen.
Since atomic oxygen is the rate limiting constituent in ozone
formation, nitrogen photochemistry is crucial to determining ozone
levels in the troposphere.

\chapter{Heterogeneous Chemistry}\label{sxn:chm_htr}
Chemistry which occurs at the interface between thermodynamic 
phases is called \trmdfn{heterogeneous chemistry}. 
Atmospheric aerosols provide surfaces which demarcate the vapor and
solid or liquid phases.
Usually the chemistry occuring at the surface of liquid aerosols is
insignificant compared to the chemistry occuring inside the liquid.
This is because the vapor phase constituents may diffuse through the 
surface boundary into the liquid where there is a much larger volume
of reactants.
Atmospheric chemistry in the liquid phase is called \trmdfn{aqueous
chemistry} because water is by far the dominant solvent in the
(lower) atmosphere. 
We discuss aqueous chemistry more fully in \S\ref{sxn:chm_aqs}.
In this chapter, we shall concentrate on the chemistry occuring at the
interface between solid and liquid phases.
We henceforth restrict the term heterogeneous chemistry to refer to
chemistry solid and vapor phases.
For this reason, heterogeneous chemistry is also called
\trmdfn{surface chemistry}.

Reactions on aerosol surfaces were recognized as fundamental to the
Earth system only in the recent past. 
The annual springtime catalytic destruction of Antarctic stratospheric
ozone, i.e., the ``ozone hole'', was attributed to surface
heterogeneous chemistry only in 1986 \cite[][]{SGR86,MSW86}.
Since then, a multitude of studies have examined the role of surface
heterogeneous chemistry on volcanic aerosols, \trmdfn{polar
stratospheric clouds} (PSCs), and cirrus clouds.

Heterogeneous chemistry requires additional parameters not necessary in
pure gas phase chemistry.
The most significant single parameter required, of course, is
knowledge of the abundances of species on the surface. 
For concreteness, we shall assume that the solid surface facilitating
reactions is an aerosol, such as mineral dust.
Such surfaces are often composed of reactive compounds.
For example, mineral dust particles are approximately five percent
calcium carbonate (\CaCOt) by weight \cite[]{Pye87}.
Thus neutralizing reactions between the surface and acids adsorbed
from the gas phase are possible.

\section[Mass Transfer Rates]{Mass Transfer Rates}
The loss rate of an atmospheric species to an aerosol surface may be
expressed as a pseudo first order reaction rate, $\rxrsfcttl$, which
is a function of aerosol radius and ambient conditions.
The gas-phase diffusion limited surface uptake rate $\rxrsfc$ is
\begin{eqnarray}
% DCZ96 p. 22871 (2), SeP97 p. 636 (11.128), SeP97 p. 606 (11.43)
\rxrsfc (\rds) & = & 4 \mpi \rds \dffA \vntcff \times
\left(
\frac{\frac{3}{4} \mssacmcff ( 1 + \kndnbr ) }{
\kndnbr^{2} + \kndnbr + 0.283 \kndnbr \mssacmcff + \frac{3}{4} \mssacmcff }
\right)^{-1}
\label{rxn:rxr_sfc_dfn}
\end{eqnarray}
The total pseudo-first-order rate coefficient $\rxrsfcttl$ is
\begin{eqnarray}
% DCZ96 p. 22871 (3)
\rxrsfcttl & = & \int_{0}^{\infty} \rxrsfc( \rds ) \dstfnc ( \rds ) \,\dfr\rds
\label{rxn:rxr_sfc_ttl_dfn}
\end{eqnarray}
where $\rxrsfcttl$ is in \xs\ and $\rxrsfc$ is in \mCxs.
\begin{rxnarray}
\Avpr & \eqbm & \Asld % BOT99 p. 22872 (13a)
\end{rxnarray}
Thus the rate of depletion of $\Avpr$ is $\rxrsfc \Avpr$\,\mlcxs. 

\subsection[Mean Free Path]{Mean Free Path}\label{sxn:mfp}
\begin{eqnarray}
% [m] SeP97 p. 455 (8.5)
\mfpBB & = & \frac{1}{( \mpi \sqrt{2} \cncB \dmtcllB^{2} ) }
\label{eqn:mfp_BB_dfn}
\end{eqnarray}

\begin{eqnarray}
% [m] SeP97 p. 455 (8.6)
\mfpBB & = & \frac{2 \vscdyn}{\prs ( 8 \mss / \mpi \gascstunv \tpt )^{1/2} }
\label{eqn:mfp_BB_dfn_2}
\end{eqnarray}

\begin{eqnarray}
% [frc] SeP97 p. 456 (8.8)
\kndnbrAB & = & \frac{2 \mfpAB}{\dmt } = \frac{\mfpAB}{\rds }
\label{eqn:knd_nbr_AB_dfn}
\end{eqnarray}

\begin{eqnarray}
% [m] SeP97 p. 457 (8.9)
\mfpAB & = & ( \mpi \sqrt{2} \cncA
\dmtcllA^{2} + \mpi \sqrt{ 1 + \mssratAB } 
\cncdryair \dmtcllavgAB^{2} )^{-1}
\label{eqn:mfp_AB_dfn}
\end{eqnarray}

\begin{eqnarray}
% [m] SeP97 p. 457 (8.11)
\mfpAB & = & ( \mpi \sqrt{ 1 + \mssratAB } 
\cncdryair \dmtcllavgAB^{2} )^{-1}
\label{eqn:mfp_AB_dfn_2}
\end{eqnarray}

Chapman-Enskog theory Hirschfelder et al (1954)
\begin{eqnarray}
% [m] SeP97 p. 457 (8.12)
\mfpAB & = & \frac{3}{8 \mpi } 
\frac{[ \mpi \bltcst^{3} \tpt^{3} ( 1 + \mssratAB ) / ( 2 \mmwA ) ]^{1/2} }{
\dnsatm \dmtcllavgAB \Omega_{\A \B}^{(1,1)} }
\label{eqn:mfp_AB_dfn_3}
\end{eqnarray}
This expression is very unwieldy.
However, it contains useful insights into limiting cases for
$\mfpAB$. 
If the molecules are considered as hard spheres then the collision
integral $\Omega_{\A \B}^{(1,1)} = 1$ and (\ref{eqn:mfp_AB_dfn_3})
reduces to
\begin{eqnarray}
% [m] SeP97 p. 457 (8.13)
\mfpAB & = & \frac{32 \dffAB}{3 \mpi ( 1 + \mssratAB ) \vlcmlcavgA }
\label{eqn:mfp_AB_dfn_4}
\end{eqnarray}
Taking the limits of (\ref{eqn:mfp_AB_dfn_4}) we find
\begin{eqnarray}
\lim_{\mssratAB \rightarrow x} \mfpAB & = & \left\{
\begin{array}{ >{\displaystyle}r<{} @{\ =\ } >{\displaystyle}l<{} @{\quad:\quad} r}
\frac{32 \dffAB}{3 \mpi \vlcmlcA } & 3.395 \frac{\dffAB}{\vlcmlcA } & \mssratAB \ll 1 \\
\multicolumn{3}{c}{} \\[0.0ex]
\frac{16 \dffAB}{3 \mpi \vlcmlcA } & 1.700 \frac{\dffAB}{\vlcmlcA } & \mssratAB = 1 \\
\multicolumn{3}{c}{} \\[0.0ex]
\frac{32 \dffAB}{3 \mpi \mssratAB \vlcmlcA } & 3.395 \frac{\dffAB }{
\mssratAB \vlcmlcA } & \mssratAB \gg 1
\end{array} \right.
\label{eqn:mfpAB_lmt}
\end{eqnarray}
Thus for all cases of interest in the atmosphere, we may say that
$\mfpAB$ is of order $\dffAB / \vlcmlcavgA$.
A number of simple expressions for $\mfpAB$ in terms of $\dffAB$ have
been derived from the kinetic theory of gases. 
These include the expression of Fuchs and Sutugin (1971, MFC)
\begin{eqnarray}
% [m] SeP97 p. 457 (8.11)
\mfpAB & = & \frac{3 \dffAB}{\vlcmlcA }
\label{eqn:mfp_AB_dfn_5}
\end{eqnarray}
The expression used by Loyalka (1973)
\begin{eqnarray}
% [m] SeP97 p. 457 (8.16)
\mfpAB & = & \frac{4 \dffAB}{\sqrt{\mpi} \vlcmlcA} = 2.257 \frac{\dffAB}{\vlcmlcA}
\label{eqn:mfp_AB_dfn_6}
\end{eqnarray}
and an expression derived from applying the kinetic gas theory to
Fick's First Law (\ref{eqn:fck_1st_law})
\begin{eqnarray}
% [m] SeP97 p. 457 (8.19)
\mfp & = & \frac{2 \dff}{\vlcmlc } 
\label{eqn:mfp_AB_dfn_7}
\end{eqnarray}
The four alternative definitions of $\mfpAB$, 
(\ref{eqn:mfp_AB_dfn_4}), (\ref{eqn:mfp_AB_dfn_5}),
(\ref{eqn:mfp_AB_dfn_6}), and (\ref{eqn:mfp_AB_dfn_7}),  
differ from one another by constants of order unity.
These definitions are employed to determine mass transfer
coefficients in \S\ref{sxn:trn}. 
Which expression to use may be determined by which theory of mass
transfer is employed. 
The ambiguity in $\mfpAB$ reflects a corresponding ambiguity in $\dffAB$.
Thus the validity of the expression chosen for $\mfpAB$ or $\dffAB$
may depend on the application.

\subsection[Binary Gaseous Diffusion Coefficient]{Binary Gaseous Diffusion Coefficient}\label{sxn:dff}
Consider the binary diffusivity of a trace gas \A\ in an atmosphere
of gas \B.
For concreteness, imagine \B\ is dry air.
The diffusivities of the various species are computed as in
\cite{SeP97}.
% NB: A formula for binary diffusivity is given in Perry's 
% ``Chemical Engineers Handbook'' (1984) Chap. 3, Table/page 3-285 
% NCAR Library Reference book TP151.C52
\begin{eqnarray}
\mssratAB & = & \mmwA / \mmwB \\
\label{eqn:dmt_cll_avg_dfn}
\dmtcllavgAB & = & \frac{\dmtcllA + \dmtcllB}{2} \\ % [m] SeP97 p. 457 (8.10)
\label{eqn:dff_gas_air_dfn}
\dffAB & = & \frac{3 \mpi \vlcmlcA (1 + \mssratAB) \mfpAB}{32} % [mlc m2 s-1] SeP97 p. 457 (8.13)
\end{eqnarray}
Thus values for \B\ in this procedure should be weighted averages of
the properties of \Nd\ and \Od. 
When the diffusivity appears with only a single subscript, e.g.,
$\dffA$, it implies the diffusivity of \A\ in air.
The least well-known of the parameters required in 
(\ref{eqn:dmt_cll_avg_dfn})--(\ref{eqn:dff_gas_air_dfn}) is
$\dmtcllA$. 
Table~\ref{tbl:dmt_cll} gives $\sigma$ for some important atmospheric
gases.
\begin{table}
\begin{minipage}{\hsize} % Minipage necessary for footnotes KoD95 p. 110 (4.10.4)
\renewcommand{\footnoterule}{\rule{\hsize}{0.0cm}\vspace{-0.0cm}} % KoD95 p. 111
\begin{center}
\caption[Molecular Collision Diameters]{\textbf{Molecular Collision
Diameters $\dmtcllA$~(m) for Important Atmospheric Gases}% 
\footnote{\emph{Sources:} CRC95 = \cite{CRC95}, p.~6-244; SeP97 =
\cite{SeP97}.}% 
\label{tbl:dmt_cll}}   
\vspace{\cpthdrhlnskp}
\begin{tabular}{ r l >{$}l<{$} l }
\hline \rule{0.0ex}{\hlntblhdrskp}% 
Formula & Name & \dmtcllA & \\[0.0ex]
& & \mbox{m} & \\[0.0ex]
\hline \rule{0.0ex}{\hlntblntrskp}%
% CRC95 p. 6-244, SeP97 p. 1292 Table A.7
\COd & Carbon Dioxide & 3.34 \times 10^{-10} & CRC95 \\[0.5ex]
\Od & Oxygen & 2.98 \times 10^{-10} & CRC95 \\[0.5ex]
\Nd & Nitrogen & 3.15 \times 10^{-10} & CRC95 \\[0.5ex]
Air & Dry air & 3.46 \times 10^{-10} & SeP97 \\[0.5ex]
\HNOt & Nitric Acid & ??? \times 10^{-10} & \\[0.5ex]
& Everything else & ??? \times 10^{-10} & \\[0.5ex]
\hline
\end{tabular}
\end{center}
\end{minipage}
\end{table}
The collision diameter of dry air, $\dmtcllair = 3.46 \times 10^{-10}$,
is apparently a weighted average of $\dmtcllNd$ and $\dmtcllOd$.
Since $\dmtcllA$ is poorly known for many important reactants, it is
difficult to compute $\dffA$ for every species of interest.
It is often preferable to use measurements of $\dffA$ if they are
available. 
Table~\ref{tbl:dff} lists the modeled and measured diffusion
coefficients of many important atmospheric gases in air.
\begin{table}
\begin{minipage}{\hsize} % Minipage necessary for footnotes KoD95 p. 110 (4.10.4)
\renewcommand{\footnoterule}{\rule{\hsize}{0.0cm}\vspace{-0.0cm}} % KoD95 p. 111
\begin{center}
\caption[Binary Gasesous Diffusion Coefficients]{\textbf{Modeled And
Measured Binary Gaseous Diffusion Coefficients}% 
\footnote{\emph{Sources:} \cite{Sch86}, p.~435, \cite{PrK98}, p.~746.}%
\footnote{\emph{Sources:} Measured and modeled $\dffA$ of many
important atmospheric species in air at STP.  
Shown is the product $\prs_{0} \dff_{\XXX \mathrm{air}}$ in \atmmSxs\
where $\prs_{0} = 1$~atm.  
Dividing this number by the ambient pressure $\prs$~atm yield the
binary diffusivity in \mSxs\ at the ambient pressure.
The model results are produced using 
(\ref{eqn:dmt_cll_avg_dfn})--(\ref{eqn:dff_gas_air_dfn}).}%
\label{tbl:dff}}   
\vspace{\cpthdrhlnskp}
\begin{tabular}{r >{$}l<{$} >{$}l<{$} r}
\hline \rule{0.0ex}{\hlntblhdrskp}% 
& \multicolumn{2}{c}{\mbox{Diffusivity}} 
& \\[0.0ex]
Species & \multicolumn{2}{c}{$\prs_{0} \dff_{\XXX \mathrm{air}}
\mbox{\,\atmmSxs}$} & Reference \\[0.0ex]
& \mbox{Model} & \mbox{Measurement} & \\[0.0ex]
\hline \rule{0.0ex}{\hlntblntrskp}%
\CHq & & 2.28 \times 10^{-5} & Sch86 \\[0.5ex] % Sch86 p. 435 Table 2
\COd & & \mbox{1.59--1.70} \times 10^{-5} & Sch86 \\[0.5ex] % Sch86 p. 435 Table 2
\CSd & & 1.05 \times 10^{-5} & Sch86 \\[0.5ex] % Sch86 p. 435 Table 2
\HdO & & 2.64 \times 10^{-5} & Sch86 \\[0.5ex] % Sch86 p. 435 Table 2
\csznote{\HdO} & & 2.11 \times 10^{-5} & PrK98 \\[0.5ex] % PrK98 p. 503 (13-3)
\footnote{See (\ref{eqn:dff_H2O_air}) for temperature and pressure dependence}%
\HNOt & & 1.27 \times 10^{-5} & PrK98 \\[0.5ex] % PrK98 p. 746 Table 17.13
\NHt & & 2.34 \times 10^{-5} & Sch86 \\[0.5ex] % Sch86 p. 435 Table 2
\csznote{\NHt} & & 2.18 \times 10^{-5} & PrK98 \\[0.5ex] % PrK98 p. 746 Table 17.13
\NdOc & & 1.22 \times 10^{-5} & PrK98 \\[0.5ex] % PrK98 p. 746 Table 17.13
\Od & & 2.07 \times 10^{-5} & Sch86 \\[0.5ex] % Sch86 p. 435 Table 2
\Ot & & 1.57 \times 10^{-5} & PrK98 \\[0.5ex] % PrK98 p. 746 Table 17.13
\SOd & & 1.26 \times 10^{-5} & Sch86 \\[0.5ex] % Sch86 p. 435 Table 2
\csznote{\SOd} & & 1.30 \times 10^{-5} & PrK98 \\[0.5ex] % PrK98 p. 746 Table 17.13
\hline
\end{tabular}
\end{center}
\end{minipage}
\end{table}
The modeled values are computed using
(\ref{eqn:dmt_cll_avg_dfn})--(\ref{eqn:dff_gas_air_dfn}) and the
values of $\dmtcllA$ given in Table~\ref{tbl:dmt_cll}.
Some studies assume $\dffA = 1.0 \times 10^{-5}$\,\mSxs\ for all
atmospheric species. 
This is not a good idea for chemical reactions which are gas-phase
diffusion limited.   
For such reactions, the error in $d[\A]/\dfr\tm$ scales linearly with the
error in $\dffA$.
Table~\ref{tbl:dff} shows $\dffA$ varies by a factor of about two for
important atmospheric reactants. 

The diffusivity of water vapor is better measured than other
atmospheric gases.
\cite{PrK78} recommend
\begin{equation}
% PrK98 p. 503 (13-3)
\dffHdOair = 2.11 \times 10^{-5} \times \frac{\prsSTP}{\prs} 
\left( \frac{\tpt}{\tptfrzpnt} \right)^{1.94} 
\label{eqn:dff_H2O_air}
\end{equation}
where $\prsSTP = 101325$~Pa, and $\tptfrzpnt = 273.15$\,K, and
$\dffHdOair$ is in \mSxs.

\section[Surface Reaction Rates]{Surface Reaction Rates}
Heterogeneous chemical reactions on aerosol surfaces may proceed no
faster than the reactants can diffuse to the surface of the aerosol. 
We now focus on determining the rate of diffusion of a molecular
species to an aerosol surface.

Consider an aerosol distributed such that $\dst(\dmt) \,\dfr\dmt$ is the
number concentration of aerosol particles with size between $\dmt$ and
$\dmt + \Delta \dmt$.
We wish to calculate the rate of adsorption of the gaseous species \A\
from the atmosphere onto the surface of the aerosol distribution.
As shown in \S\ref{sxn:cnt}, this rate will not vary linearly with
the surface area of the aerosol except in extremely rarified
conditions.

\section[Uptake coefficients]{Uptake coefficients}\label{sxn:upt}
Two parameters are used to describe the uptake of gas phase species by
a surface.
The first is the \trmdfn{mass accomodation coefficient} $\mssacmcff$.
The mass accomodation coefficient is the probability that a molecule
which makes contact which a condensed surface, e.g., through a
collision, adheres to the surface. 
For a liquid surface, the molecule must enter the liquid phase, i.e.,
become hydrolyzed.
For a solid surface, the molecule adheres to the surface, a process
called \trmdfn{adsorption}.
By definition, $\mssacmcff$ only counts reversible uptake, i.e., 
molecules which are free to desorb and rejoin the atmosphere. 
The term \trmdfn{sticking coefficient} may be used for $\mssacmcff$
when the surface is solid.
Table~\ref{tbl:acm_cff} lists the accomodation coefficients measured
for many important atmospheric gases and aerosols.
\begin{table}
\begin{minipage}{\hsize} % Minipage necessary for footnotes KoD95 p. 110 (4.10.4)
\renewcommand{\footnoterule}{\rule{\hsize}{0.0cm}\vspace{-0.0cm}} % KoD95 p. 111
\begin{center}
\caption[Measured Mass Accomodation Coefficients]{\textbf{Measured
Mass Accomodation Coefficients}% 
\footnote{\emph{Sources:} DCZ96 = \cite{DCZ96}; JPL97 = \cite{JPL97}, 
pp.~224--227; Sep97 = \cite{SeP97}, p.~634.; PrK98 = \cite{PrK98},
p.~164 Table~5.4., \url{http://www.mi.uni-hamburg.de/technische_meteorologie/Meso/homepages/fmueller/mass_acc.html}}%
\label{tbl:acm_cff}}   
\vspace{\cpthdrhlnskp}
\begin{tabular}{ l >{\raggedright}p{1.0in}<{} >{\raggedright}p{1.1in}<{} >{$}l<{$} >{$}l<{$} >{$}l<{$} l }
\hline \rule{0.0ex}{\hlntblhdrskp}% 
Gas & Surface & Surface & \tpt & \mssacmcff & \mbox{Uncert-} & Reference \\[0.0ex]
& Type & Composition & \mbox{K} & & \mbox{ainty} & \\[0.0ex]
\hline \rule{0.0ex}{\hlntblntrskp}%
\HdO & Water ice & \HdO (s) & 200 & 0.5 & 2 & JPL97 \\[0.5ex]
\csznote{\HdO} & Water ice & \HdO (s) & 173\mbox{--}273 & 0.01\mbox{--}1.0 & & PrK98 \\[0.5ex]
\csznote{\HdO} & Liquid water & \HdO (l) & & 0.01\mbox{--}1.0 & & PrK98 \\[0.5ex]
\csznote{\HdO} & Nitric acid liquid & \HNOt$\cdot$n\HdO (l) & 278 & > 0.3 & & JPL97 \\[0.5ex]
\csznote{\HdO} & Nitric acid ice & \HNOt$\cdot$3\HdO (s) & 197 & & & JPL97 \\[0.5ex]
\csznote{\HdO} & Sulfuric acid & \HdSOq$\cdot$n\HdO (l) & 298 & > 0.002 & & JPL97 \\[0.5ex]
\csznote{\HdO} & Sodium Chloride & \NaCl (s) & 299 & > 0.5 & & JPL97 \\[0.5ex]
\csznote{\HdO} & Sodium Chloride & \NaCl (l) & 298 & > 0.0004 & & JPL97 \\[0.5ex]
\csznote{\HdO} & Carbon/Soot & C (l) & 298 & > 0.0004 & & JPL97 \\[0.5ex]
\HOd & Liquid water & \HdO (l) & 275 & > 0.02 & & JPL97 \\[0.5ex]
\csznote{\HOd} & Liquid water & \HdO (l) & & > 0.01\mbox{--}0.2 & & DCZ96 \\[0.5ex]
\csznote{\HOd} & Aqueous salts & \NHqHSOq (aq) \mbox{and} \LiNOt (aq) & 293 & > 0.2 & & JPL97 \\[0.5ex]
\csznote{\HOd} & Sodium chloride & \NaCl (s) & 295 & 0.02 & 5 & JPL97 \\[0.5ex]
\csznote{\HOd} & Potassium chloride & \KCl (s) & 295 & 0.02 & 5 & JPL97 \\[0.5ex]
\csznote{\HOd} & Sulfuric acid & \HdSOq (l) & 295 & > 0.05 & 5 & DCZ96 \\[0.5ex]
\Ot & Water ice & \HdO (s) & 195\mbox{--}262 & > 0.04 & & JPL97 \\[0.5ex]
\csznote{\Ot} & Liquid water & \HdO (l) & 292 & > 0.002 & & JPL97 \\[0.5ex]
\csznote{\Ot} & Liquid water & \HdO (l) & & 0.0005 & & SeP97 \\[0.5ex]
\csznote{\Ot} & Nitric acid ice & \HNOt$\cdot$n\HdO(s) & 195 & > 0.002 & & JPL97 \\[0.5ex]
\csznote{\Ot} & Sulfuric acid & \HdSOq$\cdot$n\HdO (l) & \approx 195 & > 0.00025 & & JPL97 \\[0.5ex]
\hline
\end{tabular}
\end{center}
\end{minipage}
\end{table}
Frank M\"{u}ller maintains an online database of mass accomodation
coefficients at
\url{http://www.mi.uni-hamburg.de/technische_meteorologie/Meso/homepages/fmueller/mass_acc.html}. 

For solid surfaces, mass accomodation is dominated by
\trmdfn{physisorption} or \trmdfn{chemisorption} rather than mass
transport across the interface. 
Physisorption, or physical absorption, is due to attractive physical
forces, such the attraction due to permanent dipoles.
Chemisorption, or chemical absorption, is due to exchange of electrons
between the surface and the molecules.

%For example, the flux of water vapor molecules to and from an aerosol
%surface in equilibrium are, by definition, equal (bad/wrong example???).

If a molecule attaches to a surface and then undergoes a reaction and
chemical transformation, it is no longer able to desorb to the
atmosphere with its original identity.
In this case the molecule is permanently lost to the atmosphere.
The \trmdfn{uptake coefficient} $\mssuptcff$ measures the total
probability that a vapor species molecule enters the surface phase and
is irreversibly consumed by a chemical reaction or adsorption.
\begin{eqnarray}
% [frc] Gra02 p. 864 (2)
\mssuptcff & = & 
\frac{\mbox{Number of molecules taken up on the surface per second}}
{\mbox{Total number of gas-surface collisions per second}}
\label{eqn:mss_upt_cff_dfn}
\end{eqnarray}
\cite{ULA01} present a detailed, layer-by-layer model of heterogeneous
uptake. 
\cite{Gra02} reviews measurements and models of uptake of nitrogen
oxides on tropospheric aerosols.

\cite{USP01} explain why the measured initial true uptake coefficient
$\mssuptcffnot$ may not be suitable for use in models.
Most diffusion and kinetic-limited uptake models (e.g,
\S\ref{sxn:dff_vpr_sfc}) assume that reactant molecules collide with a
smooth spherical surface once per approach.
This is true for smooth, spherical particles.
\citeauthor{USP01} note that non-smooth, aspherical particles such as
mineral dust present opportunities for more than one collision per
approach.
For such particles, the uptake coefficient should be adjusted by the
number of collisions per approach, $\sfccrcaspsph$, so that
\begin{eqnarray}
% [frc] USP01 p. 18062 HaC01a p. 3104 (ix)
\mssuptcffcrc & = & \sfccrcaspsph \mssuptcffnot
\label{eqn:mss_upt_cff_crc_dfn}
\end{eqnarray}
They found that $1.0 \le \sfccrcaspsph \le 10$ for a variety of
laboratory powders.
Nature produces more complex aspherical shapes than laboratory
powders.
Authentic samples of China Loess had $\sfccrcaspsph \approx 22$.
Applying this factor using (\ref{eqn:mss_upt_cff_crc_dfn}), nitric
acid uptake on dust increases from 
$\mssuptcffnotHNOt = 5.2 \times 10^{-5}$ to 
$\mssuptcffcrcHNOt = 1.1 \times 10^{-3}$ \cite[]{USP01}.
Thus the $sfccrcaspsph$ correction leads to an enormous change in the
simulated gaseous uptake of natural dust particles.

\cite{USP01} define $\sfccrcaspsph$ as the ratio of the
\trmidx{Brunauer-Emmett-Teller} (\trmidx{BET}) \trmidx{surface area}
to the surface area of a smooth particle of the same diameter. 
If the particles are ellipsoidal with major and minor axes of length
of $\dmtmjr$ and $\dmtmnr$, respectively, then $\sfccrcaspsph$ may be
defined analytically.
Using the properties of ellipsoids described in \S\ref{sxn:lps}, 
we find that 
\begin{eqnarray}
% fxm: include BET area in definition
\sfccrcaspsph & = & \frac{\sfclps}{\mpi \dmt^{2}}
\label{eqn:sfc_crc_asp_sph_dfn}
\end{eqnarray}

\cite{HaC01a} define the BET-corrected uptake coefficient
$\mssuptcffBET$ as the product of the initial uptake coefficient
$\mssuptcffnot$ and the ratio of the aperature-to-BET areas:
\begin{eqnarray}
% HaC01a p. 3104 (ix)
\mssuptcffBET & = & \frac{\xsx}{\sfcBET} \mssuptcffnot
\label{eqn:mss_upt_cff_BET_dfn}
\end{eqnarray}
where $\xsx$\,[\mS] is the cross-sectional area of the Knudsen cell
aperture and $\sfcBET$\,[\mS] is the BET area.
This \trmidx{BET correction} reduces values of $\mssuptcffnot$ by
2--3 order of magnitude, but may not be applicable to mineral dust.
\cite{HaC01a} give two strong reasons.
First, $\sfcBET$ is proportional to sample mass while $\xsx$ remains
constant with sample mass.
Thus, if applicable, (\ref{eqn:mss_upt_cff_BET_dfn}) predicts that
$\mssuptcffBET$ measurements will exhibit a sensitivity to total
sample mass for particles of a given size. 
This sensitivity was not observed in measurements of \HNOt\ uptake on 
sand-sized \AldOt\ when sample mass was varied between 30--100\,\mg.
Second, (\ref{eqn:mss_upt_cff_BET_dfn}) predicts that $\mssuptcffBET$
measurements will exhibit a sensitivity to sample particle size for a
given total sample mass.
A series of measurements of \HNOt\ uptake on \AldOt\ particles ranging
from $\dmt < 10$\,\um\ to $74 < \dmt < 149$\,\um\ showed no such
size-sensitivity.

The two corrections to $\mssuptcffnot$ presented thus far,
(\ref{eqn:mss_upt_cff_crc_dfn}) and~(\ref{eqn:mss_upt_cff_BET_dfn}),
are independent.
The shape correction (\ref{eqn:mss_upt_cff_crc_dfn}) increases the
effective uptake of atmospheric particles by acccounting for the
multiple gas-particle collisions per approach the occur on realistic
particles 
The \trmidx{BET correction} (\ref{eqn:mss_upt_cff_BET_dfn}) decreases
the measured initial uptake rate $\mssuptcffnot$ by accounting for the  
BET area of uptake (which is much larger than the Knudsen cell
aperture) during the experiment.
A third correction, the \trmidx{pore diffusion correction} is
discussed in \cite{USP01} and \cite{HaC01a}.

Our description of heterogeneous uptake follows the nomenclature of
the review paper by \cite{KWZ95}.
Gaseous uptake by a liquid surface is the most complicated form of 
removal, so we shall treat this first.
We shall then consider uptake to solid surfaces as a limiting case of
the liquid phase uptake.

The \trmdfn{resistance-in-series} method considers gaseous uptake to
condensed surfaces as an analog of an electrical circuit.
The flux of molecules (current) across a physical barrier (resistance)
causes a drop in gas phase density (voltage) of the molecule.  
Implicit in this model is the assumption that the various resistances
may be decoupled and treated as independent steps in a rate limited
problem. 
This is almost always a valid assumption in the atmosphere.
The total reactive uptake of the gas phase species is denoted by
$\mssuptcffttl$, which is analogous to a conductance.
The corresponding resistance to uptake, $\mssuptcffttl^{-1}$, is
obtained as the series sum of the resistance due to gas phase
diffusion, $\mssuptcffdff^{-1}$, the resistance due to surface
accomodation, $\mssacmcff$, and the resistance due to solubility and
reaction $(\mssuptcffsol + \mssuptcffrxn)^{-1}$ 
\begin{eqnarray}
% [frc] KWZ95 p. 790 (10)
\frac{1}{\mssuptcffttl } & = & \frac{1}{\mssuptcffdff }
+ \frac{1}{\mssacmcff } + \frac{1}{\mssuptcffsol + \mssuptcffrxn }
\label{eqn:upt_rss_dfn}
\end{eqnarray}
The nomenclature capitalizes the individual uptake rates
$\mssuptcffdff$, $\mssuptcffsol$, and $\mssuptcffrxn$, to indicate
that they represent normalized, process-limited rates rather than
absolute probabilities. 
In other words $\mssuptcffdff$, $\mssuptcffsol$, and $\mssuptcffrxn$, 
unlike $\mssacmcff$, may exceed unity.
When $\mssuptcffdff$, $\mssuptcffsol$, and $\mssuptcffrxn$ are all
large, i.e., offer little resistance, then 
$\mssuptcffttl \rightarrow \mssacmcff$. 
Note that allowance for time dependence in (\ref{eqn:upt_rss_dfn}) may
be made in any of the uptake coefficients. 

With the definition of $\mssuptcffttl$ in (\ref{eqn:upt_rss_dfn}), the
total net flux of molecules from the gaseous species \A\ to the
condensed phase is 
\begin{eqnarray}
% [frc] KWZ95 p. 790 (11)
\mlcflxtldA & = & \frac{\cncA \mssuptcffttl \vlcmlcavgA}{4}
\label{eqn:mlc_flx_tld_rss}
\end{eqnarray}
This expression provides the basic link between the chemical
reactivity of an aerosol, $\mssuptcffttl$, which is usually obtained
from measurements, and the quantity of most interest to chemical
models, $\mlcflxtldA$.  
In modeling applications, $\mlcflxtldA$ (\ref{eqn:mlc_flx_tld_rss}) is
multiplied by the total surface area of the aerosol to yield the rate
of change of gas phase \A.
Note that (\ref{eqn:mlc_flx_tld_rss}) is in the form of a kinetic
collision rate (\ref{eqn:mlc_flx_tld_knt_dfn}) times an adjustment
factor, $\mssuptcffttl$.
Thus $\mssuptcffttl$ accounts for all rate limiting processes except
kinetic transport to the surface.
In other words, $\mssuptcffttl$ is the mass transport rate normalized
by the Boltzmann gas kinetic flux \cite[]{KWZ95}.

\subsection[Laboratory Measurements of Mass Uptake]{Laboratory Measurements of $\mssuptcffttl$}
In practice, $\mssuptcff$ may be measured under controlled conditions
for a particular vapor species $\AAA$ and type of aerosol.
Table~\ref{tbl:mss_upt_cff} lists the uptake coefficients measured for
common reactions between many important atmospheric gases and aerosol
surfaces. 
% 19990316: NB: longtable does not work inside a table environment
%\setlength{\LTcapwidth}{\hsize} % Longtable caption width, default is 4in
%\begin{landscape}
\begin{longtable}[t]{ l >{\raggedright}p{7.0em}<{} >{\raggedright}p{8.0em}<{} >{$}l<{$} >{$}l<{$} >{$}r<{$} p{5.0em} l }
& & & & & & & \kill % NB: longtable requires caption as table entry
\caption[Measured Uptake Coefficients]{\textbf{Measured Uptake Coefficients}%
\footnote{\emph{Sources:} DCZ96 = \cite{DCZ96}; 
JPL97 = \cite{JPL97}, pp.~233--238; 
Sep97 = \cite{SeP97}, p.~634; 
HaC01a = \cite{HaC01a}, p.~3103, Table~5; % HaC01a p. 3103 Tbl. 5
HaC01b = \cite{HaC01b}, p.~18060, Table~7; % HaC01b p. 18060 Tbl. 7
HaC03a = \cite{HaC03a}, p.~886; % HaC03a p. 886
HaC03b = \cite{HaC03b}, p.~124, Table~3; % HaC03b p. 124 Table 3 
USP01 = \cite{USP01}, p.~18060, Table~4; % USP01 p. 18060 Tbl. 4
ULA01 = \cite{ULA01}, p.~6615, Table~5; % ULA01 p. 6615 Tbl. 5
Gra02 = \cite{Gra02}, p.~867, Table~2; % Gra02 p. 867 Tbl. 2
MUG02 = \cite{MUG02}, p.~10-3, Table~1; % MUG02 Table 1 p. 10-3
MUG03 = \cite{MUG03}, p.~3206, Table~1. % MUG03 Table 1 p. 3206
All values are initial true uptake rates, unless otherwise specified.}% 
\footnote{(s), (l), (aq) denote solid phase, pure liquid, and
(dilute) aqueous solution, respectively}%
\footnote{$\Delta$ is uncertainty factor}%
\footnote{\emph{Notes:} 
\setcounter{enmnot}{0} % Reset reference counter for this table
\enmnotstpprn, Strong water dependence observed\label{idx_sns_H2O}; 
\enmnotstpprn, Measurement for bulk dust sample\label{idx_blk};
\enmnotstpprn, Measurement for single unpolished crystal\label{idx_sgl_rgh};
\enmnotstpprn, Measurement for single polished crystal\label{idx_sgl_smt};
\enmnotstpprn, Apparently sample called Gobi dust in \cite{ULA01} is China Loess in \cite{Gra02} and these are both distinct from the China Loess in \cite{USP01}\label{idx_gobi}; 
\enmnotstpprn, Size distribution was too large to measure $\sfccrcaspsph$ (\ref{eqn:sfc_crc_asp_sph_dfn})\label{idx_sahara};
\enmnotstpprn, Average uptake coefficient for 120\,s is factor of six lower than $\mssuptcffnot$\label{idx_avg_120s}; 
\enmnotstpprn, Specified initial uptake rate for 30\,ppbv \Ot\ at STP. 
Range is $3.5 \times 10^{-4}$--$5.5 \times 10^{-6}$.\label{idx_O3};}% end footnote
\label{tbl:mss_upt_cff}} \\ % end caption
\hline\hline \rule{0.0ex}{\hlntblhdrskp}% 
Gas & Surface & Surface & \tpt & \mssuptcff & \Delta & Ref. & Notes \\[0.0ex]
& Type & Composition & \mbox{K} & & & & \\[0.0ex]
\hline\hline \rule{0.0ex}{\hlntblntrskp}%
\endfirsthead % Lines between and \endfirsthead appear at top of table
\caption[]{(continued)} \\ % Set label for following pages
Gas & Surface & Surface & \tpt & \mssuptcff & \Delta & Ref. & Notes \\[0.0ex]
& Type & Composition & \mbox{K} & & & & \\[0.0ex]
\hline\hline \rule{0.0ex}{\hlntblntrskp}%
\endhead % Previous block appears at top of every page
\endlastfoot % Previous block appears at end of table
\multicolumn{8}{c}{$\HNOt + \mbox{Surface} \yields \mbox{Products}$\rule[-0.5ex]{0ex}{1.5ex}} \\[0.0ex]
\HNOt & Liquid water & \HdO (l) & 268 & 0.07 & & SeP97 & \\[0.5ex]
\csznote{\HNOt} & Liquid water & \HdO (l) & 293 & 0.19 & & SeP97 & \\[0.5ex]
\multicolumn{8}{c}{$\HNOt + \C \sldprn \yields \mbox{Products}$\rule[-0.5ex]{0ex}{1.5ex}} \\[0.0ex]
\HNOt & Carbon/Soot & C(s) & 190\mbox{--}440 & 0.04 & 5 & JPL97 & \\[0.5ex]
\csznote{\HNOt} & Hexane Soot & & & 1 \times 10^{-5} & & Gra02 & \ref{idx_avg_120s} \\[0.5ex] % Gra02 p. 875
\csznote{\HNOt} & Hexane Soot & & & 7 \times 10^{-5} & & Gra02 & \\[0.5ex] % Gra02 p. 875 Fig. 17
\multicolumn{8}{c}{$\HNOt + \mbox{Dust} \yields \mbox{Products}$\rule[-0.5ex]{0ex}{1.5ex}} \\[0.0ex]
\csznote{\HNOt} & \AldOt & & & (13 \pm 3.3) \times 10^{-2} & & HaC01a & \ref{idx_blk} \\[0.5ex] % HaC01a p. 3101 Tbl. 2
\csznote{\HNOt} & \AldOt & & & (1.6 \pm 1.4) \times 10^{-3} & & HaC01a & \ref{idx_sgl_rgh} \\[0.5ex] % HaC01a p. 3103 Tbl. 2
\csznote{\HNOt} & \AldOt & & & (6.7 \pm 1.9) \times 10^{-4} & & HaC01a & \ref{idx_sgl_smt} \\[0.5ex] % HaC01a p. 3103 Tbl. 2
\HNOt & Alumina & $\alpha$-\AldOt & & (9.7 \pm 0.5) \times 10^{-5} & & ULA01, Gra02 & \\[0.5ex] % ULA01 p. 6615 Tbl. 5, Gra02 p. 867 Tbl. 2
\HNOt & Alumina & $\alpha$-\AldOt & & 2.0 \times 10^{-3} & & USP01 & \\[0.5ex] % USP01 p. 18060 Tbl. 4
\csznote{\HNOt} & Arizona Dust & & & (5.7 \pm 1.5) \times 10^{-2} & & HaC01a & \\[0.5ex] % HaC01a p. 3103 Tbl. 4
\csznote{\HNOt} & Calcium Carbonate & \CaCOt & & 7.1 \times 10^{-2} & & Fenter & \\[0.5ex] % Hin04, Fenter et al. (1995) Atmos. Environ. (29) 3365-3372
\csznote{\HNOt} & Calcium Carbonate & \CaCOt & & 2.4 \times 10^{-4} & & Gra02 & \\[0.5ex] % Gra02 p. 867 Tbl. 2
\csznote{\HNOt} & Calcium Carbonate & \CaCOt & & (9.7 \pm 2.4) \times 10^{-2} & & HaC01a & \ref{idx_blk} \\[0.5ex] % HaC01a p. 3101 Tbl. 3
\csznote{\HNOt} & Calcium Carbonate & \CaCOt & & (1.75 \pm 0.39) \times 10^{-3} & & HaC01a & \ref{idx_sgl_rgh} \\[0.5ex] % HaC01a p. 3103 Tbl. 3
\csznote{\HNOt} & Calcium Carbonate & \CaCOt & & (9.6 \pm 1.2) \times 10^{-4} & & HaC01a & \ref{idx_sgl_smt} \\[0.5ex] % HaC01a p. 3103 Tbl. 3
\csznote{\HNOt} & \Ca-montmorillonite & \Ca\montmorilloniteb & & 11.4 \times 10^{-2} & & HaC01b & \\[0.5ex] % HaC01b p. 18060 Tbl. 5
\csznote{\HNOt} & Calcium Oxide & \CaO & & (6.1 \pm 0.3) \times 10^{-3} & & ULA01, Gra02 & \\[0.5ex] % ULA01 p. 6615 Tbl. 5, Gra02 p. 867 Tbl. 2
\csznote{\HNOt} & Calcium Oxide & \CaO & & 1.6 \times 10^{-2} & & USP01 & \ref{idx_sns_H2O} \\[0.5ex] % USP01 p. 18060 Tbl. 4
\csznote{\HNOt} & Chinese Dust (Taklamakan) & & & 17.1 \times 10^{-2} & & HaC01b & \\[0.5ex] % HaC01b p. 18060 Tbl. 5
\csznote{\HNOt} & Dolomite & \dolomite & & 14.0 \times 10^{-2} & & HaC01b & \\[0.5ex] % HaC01b p. 18060 Tbl. 5
\csznote{\HNOt} & Gobi Dust & & & (5.2 \pm 0.3) \times 10^{-5} & & ULA01, Gra02 & \ref{idx_gobi} \\[0.5ex] % ULA01 p. 6615 Tbl. 5, Gra02 p. 867 Tbl. 2
\csznote{\HNOt} & Gobi Dust & & & 1.1 \times 10^{-3} & & USP01 & \\[0.5ex] % USP01 p. 18060 Tbl. 4
\csznote{\HNOt} & Illite & \illiteb & & 10.8 \times 10^{-2} & & HaC01b & \\[0.5ex] % HaC01b p. 18060 Tbl. 5
\csznote{\HNOt} & Illite/smectite (70:30) & \illiteb/\smectite & & 8.9 \times 10^{-2} & & HaC01b & \\[0.5ex] % HaC01b p. 18060 Tbl. 5
\csznote{\HNOt} & Iron Oxide & $\alpha$-\FedOt & & (5.3 \pm 0.3) \times 10^{-5} & & ULA01, Gra02 & \\[0.5ex] % ULA01 p. 6615 Tbl. 5, Gra02 p. 867 Tbl. 2
\csznote{\HNOt} & Kaolinite & \kaolinite & & 11.4 \times 10^{-2} & & HaC01b & \\[0.5ex] % HaC01b p. 18060 Tbl. 5
\csznote{\HNOt} & Magnesium Oxide & \MgO & & (3.7 \pm 0.2) \times 10^{-4} & & ULA01 & \\[0.5ex] % ULA01 p. 6615 Tbl. 5
\csznote{\HNOt} & Magnesium Oxide & \MgO & & 1.4 \times 10^{-3} & & USP01 & \ref{idx_sns_H2O}\\[0.5ex] % USP01 p. 18060, Tbl. 4
\csznote{\HNOt} & Magnesium Oxide & \MgO & & 4.0 \times 10^{-4} & & Gra02 & \\[0.5ex] % Gra02 p. 867 Tbl. 2
\csznote{\HNOt} & \Na-montmorillonite & \Na\montmorilloniteb & & 8.1 \times 10^{-2} & & HaC01b & \\[0.5ex] % HaC01b p. 18060 Tbl. 5
\csznote{\HNOt} & Orthoclase & \orthoclase & & 8.4 \times 10^{-2} & & HaC01b & \\[0.5ex] % HaC01b p. 18060 Tbl. 5
\csznote{\HNOt} & Palygorskite & \palygorskite & & 19.6 \times 10^{-2} & & HaC01b & \\[0.5ex] % HaC01b p. 18060 Tbl. 5
\csznote{\HNOt} & Quartz & \SiOd & & (2.9 \pm 0.2) \times 10^{-5} & & ULA01, Gra02 & \\[0.5ex] % ULA01 p. 6615 Tbl. 5, Gra02 p. 867 Tbl. 2
\csznote{\HNOt} & Ripidolite (chlorite) & \chlorite & & 10.2 \times 10^{-2} & & HaC01b & \\[0.5ex] % HaC01b p. 18060 Tbl. 5
\csznote{\HNOt} & Saharan Dust (Cape Verde) & & & (11 \pm 3) \times 10^{-2} & & HaC01a & \ref{idx_blk} \\[0.5ex] % HaC01a p. 3103 Tbl. 5
\csznote{\HNOt} & Saharan Dust (Cape Verde) & & & 13.6 \times 10^{-2} & & HaC01b & \\[0.5ex] % HaC01b p. 18060 Tbl. 5
\csznote{\HNOt} & Saharan Dust (Cape Verde) & 296 & & 0.1 & & HaC03a & \\[0.5ex] % HaC03a p. 886
\csznote{\HNOt} & Saharan Sand & & & (2.0 \pm 0.1) \times 10^{-5} & & ULA01 & \\[0.5ex] % ULA01 p. 6615 Tbl. 5
\csznote{\HNOt} & Saharan Sand & & & \sfccrcaspsph \times 2.0 \times 10^{-5} & & USP01 & \ref{idx_sahara} \\[0.5ex] % USP01 p. 18060, Tbl. 4
\multicolumn{8}{c}{$\HNOt + \Na \X \sldprn \yields \Hu \X + \NaNOt$\rule[-0.5ex]{0ex}{1.5ex}} \\[0.0ex]
\HNOt & Sodium Chloride & \NaCl (s) & 295\mbox{--}298 & 0.02 & 3 & JPL97 & \\[0.5ex]
\csznote{\HNOt} & Sodium Bromide & \NaBr (s) & \approx 290 & 0.02 & 10 & JPL97 & \\[0.5ex]
\csznote{\HNOt} & Potassium Bromide & \KBr (s) & \approx 290 & 0.02 & 10 & JPL97 & \\[0.5ex]
\csznote{\HNOt} & Potassium Chloride & \KCl (s) & \approx 290 & 0.02 & 10 & JPL97 & \\[0.5ex]
\multicolumn{8}{c}{$\NdOc + \HdO \yields 2\HNOt$\rule[-0.5ex]{0ex}{1.5ex}} \\[0.0ex]
\NdOc & Water ice & \HdO (s) & 195\mbox{--}200 & 0.01 & 3 & JPL97 & \\[0.5ex]
\csznote{\NdOc} & Liquid water & \HdO (l) & 260\mbox{--}295 & 0.05 & 2 & JPL97 & \\[0.5ex]
\csznote{\NdOc} & Liquid water & \HdO (l) & 283 & 0.04 & & SeP97 & \\[0.5ex]
\csznote{\NdOc} & Nitric acid ice & \HNOt$\cdot$3\HdO (s) & 200 & 0.0003 & 3 & JPL97 & \\[0.5ex]
\csznote{\NdOc} & Sulfuric acid & \HdSOq$\cdot$\discretionary{}{}{}n\HdO (aq) & 195\mbox{--}300 & \approx 0.1 & & JPL97 & \\[0.5ex]
\csznote{\NdOc} & Sulfuric acid monohydrate & \HdSOq$\cdot$\HdO (s) & 200\mbox{--}300 & & 3 & JPL97 & \\[0.5ex]
\csznote{\NdOc} & Sulfuric acid tetrahydrate & \HdSOq$\cdot$4\HdO (s) & 195\mbox{--}207 & 0.006 & 2 & JPL97 & \\[0.5ex]
\csznote{\NdOc} & Ternary acid & \HdSOq$\cdot$\discretionary{}{}{}n\HNOt$\cdot$\discretionary{}{}{}n\HdO (aq) & 195\mbox{--}218 & & & JPL97 & \\[0.5ex]
\multicolumn{8}{c}{$\NdOc + \mbox{Surface} \yields \mbox{Products}$\rule[-0.5ex]{0ex}{1.5ex}} \\[0.0ex]
\NdOc & Ammonium sulfate & \NHqdSOq (aq) & & 0.06\mbox{--}0.12 & & DCZ96 & \\[0.5ex] % XXX: not sure about phase 
\csznote{\NdOc} & Sulfuric acid & \HdSOq$\cdot$\discretionary{}{}{}n\HdO (aq) & & 0.06\mbox{--}0.12 & & DCZ96 & \\[0.5ex]
\multicolumn{8}{c}{$\NdOc + \NaCl \sldprn \yields \ClNOd + \NaNOt \sldprn$\rule[-0.5ex]{0ex}{1.5ex}} \\[0.0ex]
\NdOc & Sodium chloride & \NaCl (s) & 300 & 0.0005 & 20 & JPL97 & \\[0.5ex]
\csznote{\NdOc} & Sodium chloride & \NaCl (aq) & 300 & > 0.02 & 20 & JPL97 & \\[0.5ex]
\multicolumn{8}{c}{$\NdOc + \M \Br \sldprn \yields \mbox{Products}$\rule[-0.5ex]{0ex}{1.5ex}} \\[0.0ex]
\NdOc & Sodium bromide & \NaBr (s) & \approx 300 & & & JPL97 & \\[0.5ex]
\csznote{\NdOc} & Potassium bromide & \KBr (s) & \approx 300 & 0.004 & 10 & JPL97 & \\[0.5ex]
\multicolumn{8}{c}{$\NHt + \HdSOq \yields \NHqHSOq$\rule[-0.5ex]{0ex}{1.5ex}} \\[0.0ex]
\NHt & Sulfuric acid & \HdSOq$\cdot$n\HdO (l) & 288\mbox{--}300 & 0.4 & 2.5 & JPL97 & \\[0.5ex]
\multicolumn{8}{c}{$\NOt + \HdO \yields \HNOt + \OH$\rule[-0.5ex]{0ex}{1.5ex}} \\[0.0ex]
\NOt & Dust & & & 0.1 & & ZhC99 & \\[0.5ex]
\NOt & Liquid water & \HdO (l) & 273 & 0.0002 & 20 & JPL97 & \\[0.5ex]
\csznote{\NOt} & Liquid water & \HdO (l) & & 0.0002 & & SeP97 & \\[0.5ex]
\multicolumn{8}{c}{$\Ot + \mbox{Surface} \yields \mbox{Products}$\rule[-0.5ex]{0ex}{1.5ex}} \\[0.0ex]
\Ot & Alumina & \chmphz{\AldOt}{s} & 210\mbox{--}300 & & & JPL97 & \\[0.5ex]
\csznote{\Ot} & Alumina & $\alpha$-\AldOt & & (8 \pm 5) \times 10^{-5} & & MUG02 & \\[0.5ex] % MUG02 Table 1 p. 10-3
\csznote{\Ot} & Alumina ($\dmt \sim 25$\,\um) & $\alpha$-\AldOt & & (1.4 \pm 0.3) \times 10^{-4} & & MUG03 & \\[0.5ex] % MUG03 Table 1 p. 3206
\csznote{\Ot} & Alumina ($\dmt \sim 25$\,\um) & $\alpha$-\AldOt & & 7.6 \times 10^{-6} & & MUG03 (steady-state at 4.5\,h) & \\[0.5ex] % MUG03 Table 1 p. 3206
\csznote{\Ot} & Alumina ($\dmt \sim 1$\,\um) & $\alpha$-\AldOt & & (9 \pm 3) \times 10^{-5} & & MUG03 & \\[0.5ex] % MUG03 Table 1 p. 3206
\csznote{\Ot} & Carbon/Soot & \chmphz{C}{s} & \approx 300 & 0.003 & 20 & JPL97 & \\[0.5ex]
\csznote{\Ot} & Carbon & \chmphz{C}{s} & & 0.0004 & & DCZ96 & \\[0.5ex] % Fendel et al. (1995)
\csznote{\Ot} & Iron & \chmphz{Fe}{s} & & 0.0004 & & DCZ96 & \\[0.5ex] % Fendel et al. (1995)
\csznote{\Ot} & Iron Oxide & $\alpha$-\FedOt & & (1.8 \pm 0.7) \times 10^{-4} & & MUG02 & \\[0.5ex] % MUG02 Table 1 p. 10-3
\csznote{\Ot} & Quartz & \SiOd & & (5 \pm 3) \times 10^{-5} & & MUG02 & \\[0.5ex] % MUG02 Table 1 p. 10-3
\csznote{\Ot} & China Loess & & & (2.7 \pm 0.9) \times 10^{-5} & & MUG02 & \\[0.5ex] % MUG02 Table 1 p. 10-3
\csznote{\Ot} & Saharan Dust (Cape Verde) & 296 & & 3.0 \times 10^{-5} & & HaC03b & \ref{idx_O3} \\[0.5ex] % HaC03b Table 3 p. 124
\csznote{\Ot} & Saharan Sand & & & (6 \pm 3) \times 10^{-5} & & MUG02 & \\[0.5ex] % MUG02 Table 1 p. 10-3
\csznote{\Ot} & Ground & & & (6 \pm 3) \times 10^{-5} & & MUG02 & \\[0.5ex] % MUG02 Table 1 p. 10-3
\csznote{\Ot} & Sieved ($\dmt < 50$\,\um) & & & (4 \pm 2) \times 10^{-6} & & MUG02 & \\[0.5ex] % MUG02 Table 1 p. 10-3
\csznote{\Ot} & Sodium chloride & \NaCl (s) & \approx 300 & > 2 \times 10^{-10} & 20 & JPL97 & \\[0.5ex]
\hline
\end{longtable}
%\end{landscape}
The total uptake coefficient $\mssuptcff$ depends on which reactions
are involved.
In the laboratory, the allowed reactions are controlled by using pure
gases and aerosol surfaces.
This allows the $\mssuptcff$ for each reaction to be isolated and
measure separately.
The relationship developed in \ldots may be used to infer $\mssacmcff$
from $\mssuptcff$ for the surface types presented in
Table~\ref{tbl:mss_upt_cff}. 

The total areal molecular flux $\mssuptcffttl$ into the condensed
phase is often represented in terms of the gas phase
concentration and a total \trmdfn{deposition resistance} $\rssdps$ or
its inverse the \trmdfn{deposition velocity} $\vlcdps$
\begin{eqnarray}
% KWZ95 p. 791 (12,13), SeP97 p. 958 (19.1)
\label{eqn:vlc_dps_dfn_2}
\vlcdps & = & \rssdps^{-1} \\
\mlcflxtldA & = & \cncA / \rssdps \nonumber \\
& = & \cncA \vlcdps
\label{eqn:flx_dps_dfn_2}
\end{eqnarray}
The units of $\rssdps$ and $\vlcdps$ are \sxm\ and \mxs, respectively. 
The definitions of $\rssdps$ and $\vlcdps$ for gaseous deposition to
aerosol surfaces are completely analogous to the definitions of
aerosol deposition to land surfaces in \S\ref{sxn:rss}.
Combining (\ref{eqn:flx_dps_dfn_2}) with (\ref{eqn:mlc_flx_tld_rss})
we obtain
\begin{eqnarray}
% KWZ95 p. 792 (14)
\label{eqn:vlc_dps_dfn_3}
\vlcdps & = & \frac{\mssuptcffttl \vlcmlcavgA}{4} \\
\label{eqn:mss_upt_cff_prx}
\mssuptcffttl & = & \frac{4 \vlcdps}{\vlcmlcavgA }
\end{eqnarray}
Since no reference (\ref{eqn:mss_upt_cff_prx}) does not depend on the
geometry of the surface, it applies equally well to flat surfaces and
curved particles.
Table~\ref{tbl:dps_vlc} below takes advantage of this to convert
field deposition resistance measurements to uptake coefficients.
We now examine the individual resistances 
$\mssuptcffdff$, $\mssuptcffsol$, and $\mssuptcffrxn$ which comprise
$\mssuptcffttl$ (\ref{eqn:upt_rss_dfn}).

\subsection[Gas Transport Limitation]{Gas Transport Limitation}\label{sxn:dff_lmt}

\begin{eqnarray}
% [frc] JPL97 p. 222 
\mssuptcffsol & = & \frac{4 \hnr \gascstunv \tpt}{\sqrt{\mpi} \vlcmlcavg
} \left( \frac{\dfflqd}{\tm } \right)^{1/2} \\ 
\mssuptcffrxn & = & \frac{4 \hnr \gascstunv \tpt}{\vlcmlcavg } 
\sqrt{ \dfflqd \rxrsfc } \\ 
\mssuptcffttl & = & \frac{1}{\mssacmcff } + \frac{1}{\mssuptcffsol +
\mssuptcffrxn }
\label{eqn:mss_upt_dfn}
\end{eqnarray}
where $\dfflqd$ is the liquid phase diffusion coefficient.
In the limit of low solubility or long exposure time, $\mssuptcffsol
\rightarrow 0$ so that 
\begin{eqnarray}
% [frc] JPL97 p. 223 
\mssuptcffttl & = & \frac{1}{\mssacmcff } + \frac{1}{\mssuptcffrxn }
\label{eqn:mss_upt_dfn_2}
\end{eqnarray}
However, when no gas phase diffusion limitation exists, \cite{KWZ95}
show 

\subsection[Uptake on Solids]{Uptake on Solids}\label{sxn:sld}
Uptake by solid surfaces is limited by the ability of the species to
diffuse to the interior of the aerosol.
\cite{KWZ95} cite typical condensed phase diffusion coefficients of
$10^{-5}$ and $10^{-12}$\,\cmSxs\ for diffusion on liquid and solid
surfaces, respectively. 
Thus, the condensed phase diffusion coefficient for solids virtually
zero. 
As a result, net resistance to solid phase uptake may be interpreted
as the limit of (\ref{eqn:upt_rss_dfn}) as $\mssuptcffsol,
\mssuptcffrxn \rightarrow 0$.
Under these conditions, the chemistry is restricted to occur only at
the solid surface itself.
Thus the solid uptake rate is linearly proportional to both the
reaction rate and the species concentration. 

Due to heterogeneous chemistry, the number of surface sites hospitable
to a given species is not constant with time.
On a solid surface, these sites may become occupied by the products of
chemical reactions as the aerosol ages.
The surface of a liquid aerosol may also become saturated with a given  
reactant.
Site coverage or surface saturation cause $\mssacmcff \rightarrow 0$.

\subsection[Field Measurements of Mass Uptake]{Field Measurements of $\mssuptcffttl$}
Uptake coefficients may also be inferred from field measurements.
For example, bare soil surfaces are comprised of the parent materials
of atmospheric dust.
Many field observations of trace gas uptake by bare surfaces exist.
In most such measurements, the overall dry deposition resistance is
inferred by measuring the vertical gradient of the trace gas
concentration (\ref{eqn:rss_dps_dfn}). 
Using these measurements as proxies for trace gas uptake by
atmospheric dust requires the assumption that the trace gas uptake 
is not significantly affected by sub-soil microbes, plant matter, or
bulk properties of the soil that are not duplicated by the airborne
dust. 
Table~\ref{tbl:dps_vlc} lists the deposition velocities of \Ot\
measured over various surfaces.
\begin{table}
\begin{minipage}{\hsize} % Minipage necessary for footnotes KoD95 p. 110 (4.10.4)
\renewcommand{\footnoterule}{\rule{\hsize}{0.0cm}\vspace{-0.0cm}} % KoD95 p. 111
\begin{center}
\caption[Deposition Velocity of O$_{3}$]{\textbf{Deposition Velocity of
O$_\mathbf{3}$ Over Various Surfaces}% 
\footnote{\emph{Sources:} \cite{DCZ96}}%
\footnote{$\mssuptcff$ is inferred from $\vlcdps$ using $\mssuptcff =
4 \vlcdps/\vlcmlcavg$ (\ref{eqn:mss_upt_cff_prx}) for a temperature
$\tpt = 300$\,K\@.}% 
\label{tbl:dps_vlc}}   
\vspace{\cpthdrhlnskp}
\begin{tabular}{ l >{\raggedright}p{5.0em}<{} >{\raggedright}p{5.0em}<{} >{$}l<{$} >{$}l<{$} >{$}r<{$} l }
\hline \rule{0.0ex}{\hlntblhdrskp}% 
Gas & Texture & Condition & \vlcdps & \rssdps & \mssuptcff & Ref. \\[0.0ex]
& & & \mbox{\mxs} & \mbox{\sxm} & & \\[0.0ex]
\hline \rule{0.0ex}{\hlntblntrskp}%
\Ot & Sand & Moist & 0.0025 & 400 & 2.7 \times 10^{-5} & Ald69 \\[0.0ex]
\Ot & Sand & Dry & 0.005 & 200 & 5.5 \times 10^{-5} & Ald69 \\[0.0ex]
\Ot & Bare soil & Loamy & 0.025\mbox{--}0.05 & 20\mbox{--}40 &
2.7\mbox{--}5.4 \times 10^{-4} & TRW73 \\[0.0ex]
\Ot & Bare soil & High 
\footnote{$\vwc$ is volumetric water content.}%
$\vwc$ & 0.0077\mbox{--}0.014 & 70\mbox{--}130
& 8.5\mbox{--}17 \times 10^{-5} & TRW73 \\[0.0ex]
\Ot & Bare soil & Cold & 0.10 & 10 & 1.1 \times 10^{-3} & WCW81 \\[0.0ex]
\Ot & Sand & Day & 0.025 & 40 & 2.7 \times 10^{-4} & GaR80 \\[0.0ex]
\Ot & Sand & Night & 0.017 & 60 & 1.9 \times 10^{-4} & GaR80 \\[0.0ex]
\Ot & Sand & Loamy, dry & 0.033 & 30 & 3.6 \times 10^{-4} & GaR80 \\[0.0ex]
\Ot & Sand & & 0.0014 & 714 & 1.5 \times 10^{-5} & Gar76 \\[0.0ex]
\Ot & \CaCOt & & 0.0022 & 455 & 2.4 \times 10^{-5} & Gar76 \\[0.0ex]
\Ot & Soil & $^{c}\vwc = 27\%$ & 0.0084 & 119 & 9.2 \times 10^{-5} & Gar76 \\[0.0ex]
\Ot & Soil & $\vwc = 4\%$ & 0.0176 & 57 & 1.9 \times 10^{-4} & Gar76 \\[0.0ex]
\Ot & Sand, rock & Day & 0.0011 & 909 & 1.2 \times 10^{-5} & SBP87 \\[0.0ex]
\Ot & Sand, rock & Night & 0.0003 & 3333 & 3.3 \times 10^{-6} & SBP87 \\[0.0ex]
\Ot & Sahara & Day & 0.0015 & 667 & 1.6 \times 10^{-5} & GHM95 \\[0.0ex]
\Ot & Sahara & Night & 0.00065 & 1538 & 7.1 \times 10^{-6} & GHM95 \\[0.0ex]
\hline
\end{tabular}
\end{center}
\end{minipage}
\end{table}
An important advantage of field measurements of reactive uptake is
that the measured uptake coefficient includes all realistic paths of
uptake.
Unfortunately, the exact reactions defining these all the paths is
unknown. 
Nevertheless, inferences of $\mssuptcff$ from field measurements are
important since they place independent bounds on heterogeneous
transfer rates.

\section[Vapor Diffusion to Spherical Particles]{Vapor Diffusion to Spherical Particles}\label{sxn:dff_vpr_sfc}
In order to relate $\mssacmcff$ or $\mssuptcff$ to $\rxrsfc$, we must
develop a more exact theory of gas phase diffusion to aerosol
surfaces. 
Molecules come into contact with aerosol surfaces through gas phase
diffusion modified by the motion of the aerosol.
The characteristics of gaseous transfer to solid surfaces depend on
the density of the gaseous medium.
The two extremes of media are called the \trmdfn{continuum regime} and
the \trmdfn{kinetic regime}.
The continuum regime describes gaseous diffusion of vapors to an
aerosol when the mean free path between molecular collisions is 
much smaller than the size of the aerosol.
Gaseous diffusion in the troposphere is described by the continuum
regime. 

The kinetic regime describes the diffusion of vapors to an aerosol
in atmospheres where the mean free path between molecular collisions
is comparable to or larger than the size of the aerosol. 
In this case, vapor deposition to surface is described by ``billiard
ball physics'', where the gaseous molecules do not interfere with one
another, nor do they possess the properties of a continuum. 
Gaseous diffusion processes above the tropopause may need to account
for kinetic effects. 

\subsection[Knudsen Number]{Knudsen Number}\label{sxn:knd_nbr}
In order to quantify the deposition of gases to aerosol surfaces it is
important to distinguish between two important limits, or regimes, of
molecule-aerosol interactions.
Particles which are much smaller than the mean free path between
molecular collisions do not experience the molecular vapor as a
continuous fluid. 
Vapor molecules do not ``see'' the particle surface as a significant
obstacle, and collide with it relatively infrequently. 
From the point of view of the particle, the gas molecules collide with
it in discrete intervals, rather than in a continuous barrage.
This limit of molecule-aerosol interaction is called the
\trmdfn{non-continuum regime} or the \trmidx{kinetic regime}.

At the other extreme, molecules may be so numerous, or the particles
so large, that the molecules can not travel very far without striking
the particle or colliding with another molecule. 
In this case the particle is subject to a continuous barrage of
molecular collisions. 
From the point of view of the particle, the molecules form a continous
fluid which unceasingly exert pressure on the particle.
This limit of molecule-aerosol interaction is called the
\trmidx{continuum regime}.

The \trmdfn{Knudsen number} $\kndnbr$ is the dimensionles parameter
used to locate molecule-aerosol interactions along the spectrum from
the kinetic to the continuum regime.
Thus $\kndnbr$ depends upon both the size of the particle and the 
mean free path $\mfp$ of the medium
\begin{eqnarray}
% [frc] SeP97 p. 452 (8.1)
\kndnbr & = & \frac{2 \mfp}{\dmt } = \frac{\mfp}{\rds }
\label{eqn:knd_nbr_dfn}
\end{eqnarray}

\subsection[Continuum Regime]{Continuum Regime}\label{sxn:cnt}
In this section we examine the diffusion of a gaseous species \A\ to
the surface of a spherical particle in the continuum regime, i.e.,
$\kndnbrAB \ll 1$.
We shall assume that the particle is motionless and the atmosphere is
stagnant, subject only to Brownian diffusion.
The effects of particle motion on these results will be discussed in 
\S\ref{sxn:mtn}. 
Since the particle geometry is assumed spherical, we shall cast our
equations in spherical coordinates where $\rds$ is the distance from
the center of the particle.

The physical scenario to consider is that of a spherical aerosol of
radius $\rdsprt$ in contact with gas phase species \A\ whose ambient
concentration (many particle diameters away) is $\cncinf$ and whose
concentration at any radial position $\rds$ is $\cnc(\rds)$.
There is no angular dependence in this system, only radial.
Imagine a sphere concentric to this particle.
Mass conservation requires that the rate of change of [\A] inside the 
hypothetical spherical shell must be due to the flux of \A\ through
the surface of the shell.  
We denote this \trmdfn{radial flux} at distance $\rds$ by
$\mlcflxtldA(\rds)$\,\mlcxmSs.      
Since $\mlcflxtldA(\rds)$ has dimensions of per unit area, it may also
be called the \trmdfn{areal flux}.
The governing differential equation of the system may be obtained by
considering the mass balance of an infinitesimally thin spherical
shell passing through $\rds$.
The rate of change of the number of molecules of \A\ within the shell
is simply the difference between the number flux of molecules entering
at $\rds$ and the flux leaving at $\rds + \dltrds$. 
The total number flux of molecules is the area times the radial number
flux, i.e., $4 \mpi \rds^{2} \mlcflxtldA(\rds)$.
The volume of the shell is $4 \mpi \rds^{2} \dltrds$.
The rate of change of concentration is the quotient of these two
quantities 
\begin{eqnarray}
\frac{\partial \cnc}{\partial \tm } & = & \lim_{\dltrds \rightarrow 0}
\frac{4 \mpi \rds^{2} \mlcflxtldA(\rds) - 
4 \mpi ( \rds + \dltrds)^{2} \mlcflxtldA(\rds + \dltrds)
}{ 4 \mpi \rds^{2} \dltrds } \nonumber \\
& = & \lim_{\dltrds \rightarrow 0}
\frac{\rds^{2} \mlcflxtldA(\rds) - \rds^{2} \mlcflxtldA(\rds + \dltrds)
- 2 \rds \dltrds \mlcflxtldA(\rds + \dltrds) - (\dltrds)^{2} \mlcflxtldA(\rds + \dltrds)
}{ \rds^{2} \dltrds } \nonumber \\
& = & \lim_{\dltrds \rightarrow 0}
- \left( \frac{\mlcflxtldA (\rds + \dltrds) - \mlcflxtldA(\rds)}{\dltrds } +
\frac{2}{\rds } \mlcflxtldA(\rds + \dltrds) +
\frac{\dltrds}{\rds^{2} } \mlcflxtldA(\rds + \dltrds) \right) \nonumber \\
& = & - \left( \frac{\partial \mlcflxtldA}{\partial \rds } +
\frac{2 \mlcflxtldA}{\rds } \right)
\label{eqn:cnc_dff_dvl}
\end{eqnarray}
The negative sign on the RHS indicates that $\cnc$ decreases for
positive $\mlcflxtldA$, i.e., when the net flux of \A\ is outwards.
Equation (\ref{eqn:cnc_dff_dvl}) is mathematically equivalent to 
\begin{eqnarray}
% [mlc m-2 s-1] SeP97 p. 596 (11.1)
\frac{\partial \cnc}{\partial \tm } & = & - \frac{1}{\rds^{2} } 
\frac{\partial}{\partial \rds } \rds^{2} \mlcflxtldA( \rds ) 
\label{eqn:cnc_dff}
\end{eqnarray}
To proceed further, we turn to the results of physical experiments.
\trmdfn{Fick's First Law} states that the flux of a vapor is
proportional to its concentration gradient, and that the constant of
proportionality is, by definition, the bulk diffusivity of the vapor 
\begin{eqnarray}
% SeP97 p. 596 (11.2)
\mlcflxtldA & = & - \dffA \frac{\partial \cncA}{\partial \rds }
\label{eqn:fck_1st_law}
\end{eqnarray}
where $\dffA$ is the binary diffusion coefficient of \A\ in air.
In stating Fick's law (\ref{eqn:fck_1st_law}), we have maintained the 
usual convention that the origin of coordinates ($\rds = 0$) is at the
center of the particles, so that $\mlcflxtldA$ is positive for
a net flux of molecules from the particle to the atmosphere. 
A mnemonic for this convention is to remember that $\mlcflxtldA$ is
positive when net molecular flow is toward larger $\rds$.
However, we are usually more interested in particle absorption than
desorption so the convention expressed in (\ref{eqn:fck_1st_law}) may 
prove confusing.
One must bear in mind that negative radial fluxes indicate net
uptake by the particle.

The evaluation of $\dffA$ is discussed in \S\ref{sxn:dff}.
Applying Fick's Law (\ref{eqn:fck_1st_law}) to (\ref{eqn:cnc_dff_dvl}) 
results in
\begin{eqnarray}
% SeP97 p. 596 (11.2)
\frac{\partial \cncA}{\partial \tm }  & = & \dffA \left(
\frac{\partial^{2} \cncA}{\partial \rds^{2} } + \frac{2}{\rds }
\frac{\partial \cncA}{\partial \rds } \right)
\label{eqn:cnc_dff_2}
\end{eqnarray}
Also known as \trmdfn{Fick's Second Law}, this is the standard, second
order, time dependent diffusion equation in radial coordinates.
The second term on the RHS of (\ref{eqn:cnc_dff_2}) arises from the
spherical geometry of the problem.
This term becomes significant as $\rds \rightarrow 0$ and vanishes
$\rds \rightarrow \infty$.
In the latter case, (\ref{eqn:cnc_dff_2}) reduces to the one
dimensional diffusion equation in Cartesian coordinates.

The solution to (\ref{eqn:cnc_dff_2}) requires two conditions in
space and one condition in time.
Let us examine the evolution of the system in which a particle is
placed in a uniform atmosphere of concentration $\cncinf$ at $\tm 
= 0$.
Let $\cncsfc$ denote the equilibrium gas phase concentration at the
surface of the particle. 
Then the boundary and initial conditions of the system are
\begin{eqnarray}
% SeP97 p. 597 (11.5)
\cncA ( \rds, 0 ) & = & \cncinf, \qquad \rds > \rdsprt \\
\label{eqn:cnc_dff_bnd_cnd}
\cncA ( \rdsprt, \tm ) & = & \cncsfc \\
\cncA ( \infty, \tm ) & = & \cncinf
\end{eqnarray}
The solution to (\ref{eqn:cnc_dff_2}) subject to
(\ref{eqn:cnc_dff_bnd_cnd}) may be obtained by the method of 
\trmidx{Laplace transforms} or similarity functions, see
\cite{SeP97} or \cite{PrK98} for details. 

The full, time-dependent solution of (\ref{eqn:cnc_dff_2}) is
\begin{eqnarray}
% [mlc m-3] SeP97 p. 597 (11.8) PrK98 p. 503 (13-4)
\label{eqn:cnc_dfn}
\cnc(\rds,\tm) & = & 
\cncinf - \frac{\rdsprt}{\rds } ( \cncinf - \cncsfc ) +
\frac{2 \rdsprt}{\rds \sqrt{ \mpi } } ( \cncinf - \cncsfc ) 
\int_{0}^{ (\rds - \rdsprt ) / ( 2 \sqrt { \dffA \ttt } ) }
\me^{ - \xxx^{2} } \,\dfr\xxx \\
& = & 
\cncinf - \frac{\rdsprt}{\rds } ( \cncinf - \cncsfc )
\left[ 1 - \erffnc \left( \frac{\rds - \rdsprt}{2 \sqrt { \dffA
\ttt } } \right) \right] \nonumber \\
\label{eqn:cnc_cerf_dfn}
& = & 
\cncinf - \frac{\rdsprt}{\rds } ( \cncinf - \cncsfc ) \,
\cerffnc \left( \frac{\rds - \rdsprt}{2 \sqrt { \dffA \ttt } } \right)
\end{eqnarray}
where the last two steps simply re-express the third term on the RHS
of (\ref{eqn:cnc_dfn}) in terms of the \trmdfn{error function} and of
the \trmdfn{complementary error function} (Section~\ref{sxn:erf}).
This time-dependent term vanishes as $\tm \rightarrow \infty$ and the
upper limit of integration approaches zero. 
The first two terms on the RHS of (\ref{eqn:cnc_dfn}) are
time-independent, and define the steady state concentration at any
distance $\rds$
\begin{eqnarray}
% [mlc m-3] SeP97 p. 597 (11.9) PrK98 p. 504 (13-6)
\cnc(\rds) & = & 
\cncinf - \frac{\rdsprt}{\rds } ( \cncinf - \cncsfc )
\label{eqn:cnc_ss_dfn}
\end{eqnarray}

If we apply Fick's Law (\ref{eqn:fck_1st_law}) to (\ref{eqn:cnc_dfn})
and (\ref{eqn:cnc_ss_dfn}), we obtain time-varying and steady state
radial fluxes at any distance from the particle
\begin{eqnarray}
% [mlc m-2 s-1] SeP97 p. 597 (11.10) PrK98 p. 504 (13-5) 
\mlcflxtld(\rds,\tm) & = & 
- \dffA \frac{\partial \cnc(\rds,\tm)}{\partial \rds } \nonumber \\
& = & 
% This step uses chain rule TaM83 p. 553
- \dffA ( \cncinf - \cncsfc ) \frac{\rdsprt}{\rds }
\left\{ \frac{1}{\rds} + \frac{2}{\rds \sqrt { \mpi } } 
\int_{0}^{ (\rds - \rdsprt ) / ( 2 \sqrt { \dffA \ttt } ) }
\me^{ - \xxx^{2} } \,\dfr\xxx \right. \nonumber \\
& & \left. {} + \frac{2}{\sqrt{ \mpi } } 
\exp \left[ - \left( \frac{\rds - \rdsprt}{2 \sqrt { \dffA \ttt } }
\right) \right]^{2} 
\left( \frac{1}{2 \sqrt { \dffA \ttt } } \right)
\right\} \nonumber \\
& = & 
\label{eqn:mlc_flx_tld_cnt_dfn}
- \dffA ( \cncinf - \cncsfc ) \frac{\rdsprt}{\rds }
\left\{ \frac{1}{\rds} + \frac{2}{\rds \sqrt { \mpi } } 
\int_{0}^{ (\rds - \rdsprt ) / ( 2 \sqrt { \dffA \ttt } ) }
\me^{ - \xxx^{2} } \,\dfr\xxx \right. \nonumber \\
& & \left. {} + \frac{1}{\sqrt{ \mpi \dffA \tm } } 
\exp \left[ - \left( \frac{\rds - \rdsprt}{2 \sqrt { \dffA \ttt } }
\right) \right]^{2} 
\right\} \nonumber \\
& = & 
\label{eqn:mlc_flx_tld_cnt_cerf_dfn}
- \dffA ( \cncinf - \cncsfc ) \frac{\rdsprt}{\rds } 
\left\{
\frac{1}{\rds } \,
\cerffnc \left( \frac{\rds - \rdsprt}{2 \sqrt { \dffA \ttt } } \right)
+
\frac{\partial}{\partial \rds } \left[ \cerffnc \left( \frac{\rds
- \rdsprt}{2 \sqrt { \dffA \ttt } } \right) \right]
\right\}
\end{eqnarray}
Note that we have used the chain rule in differentiating the upper
limit of integration.
We shall examine three aspects of (\ref{eqn:mlc_flx_tld_cnt_dfn})
in closer detail: the steady state behavior ($\tm \rightarrow
\infty$), the time varying behavior at the particle surface ($\rds =
\rdsprt$), and the steady state behavior at the particle surface.

Substituting $\tm \rightarrow \infty$ into
(\ref{eqn:mlc_flx_tld_cnt_dfn}) we find that the steady state radial
flux at any distance $\rds > \rdsprt$ is 
\begin{eqnarray}
% [mlc m-2 s-1] SeP97 p. 597
\mlcflxtld(\rds) & = & - \frac{\dffA ( \cncinf - \cncsfc ) \rdsprt}{\rds^{2} }
\label{eqn:mlc_flx_tld_cnt_ss_dfn}
\end{eqnarray}
The dependence on $\rdsprt$ stems from the boundary condition
(\ref{eqn:cnc_dff_bnd_cnd}) that the particle is a perfect source or
sink of radius $\rdsprt$. 
The fact that $\mlcflxtld$ depends linearly rather than quadratically
on $\rdsprt$ is a characteristic of diffusive, as opposed to kinetic,
transport.
The inverse dependence of $\mlcflxtld$ on $\rds^{2}$ arises from mass
continuity. 

Substituting $\rds = \rdsprt$ into (\ref{eqn:mlc_flx_tld_cnt_dfn})
we obtain we obtain the time-varying radial flux to the surface of the
particle
\begin{eqnarray}
% [mlc m-2 s-1] SeP97 p. 597 (11.10) PrK98 p. 504 (13-5)
\mlcflxtld(\rdsprt,\tm) & = & 
- \frac{\dffA ( \cncinf - \cncsfc )}{\rdsprt } 
\left( 1 + \frac{\rdsprt}{\sqrt { \mpi \dffA \tm } } \right)
\label{eqn:mlc_flx_tld_cnt_sfc_dfn}
\end{eqnarray}
The flux at the droplet surface will reach steady state conditions
when the time-dependent term vanishes, i.e., when $\tm \gg \taucnt$
where 
\begin{eqnarray}
% [s] SeP97 p. 597
\taucnt & = & \frac{\rdsprt^{2}}{ \mpi \dffA }
\label{eqn:tau_cnt_dfn}
\end{eqnarray}
$\taucnt$ is the timescale for continuum diffusion to the surface of a
particle. 
At STP, $\taucnt < 1.7 \times 10^{-6}$~s for $\rdsprt < 100$\,\um.

Setting $\rds = \rdsprt$ in (\ref{eqn:mlc_flx_tld_cnt_ss_dfn}) or 
$\tm \rightarrow \infty$ in (\ref{eqn:mlc_flx_tld_cnt_sfc_dfn}) we obtain
\begin{eqnarray}
% [mlc m-2 s-1] SeP97 p. 597
\mlcflxtld(\rdsprt) & = & - \frac{\dffA ( \cncinf - \cncsfc )}{\rdsprt }
\label{eqn:mlc_flx_tld_cnt_sfc_ss_dfn}
\end{eqnarray}
Thus continuum diffusion theory shows the steady state radial flux of
\A\ at the particle's surface decreases linearly with $\rdsprt$.
The total steady state flux of \A\ from the atmosphere to the
particle's surface $\mlcflxcnt$ is the radial flux at $\rds = \rdsprt$  
(\ref{eqn:mlc_flx_tld_cnt_sfc_ss_dfn}) times the particle area. 
$\mlcflxcnt$ is the \trmdfn{spherical flux}.
The subscript $\cntsbs$ denotes that $\mlcflxcnt$ was obtained from
continuum diffusion theory, i.e., for $\kndnbrAB \rightarrow 0$.  
\begin{eqnarray}
% [mlc s-1] SeP97 p. 600 (11.25)
\mlcflxcnt & \equiv & - 4 \mpi \rdsprt^{2} \mlcflxtld(\rdsprt) \nonumber \\
& = & 4 \mpi \rdsprt \dffA ( \cncinf - \cncsfc )
\label{eqn:mlc_flx_cnt_dfn}
\end{eqnarray}
Note that the spherical flux $\mlcflxcnt$ and the radial flux
$\mlcflxtld$ have the opposite sign convention. 
The factor of negative one was imposed so that $\mlcflxcnt$ is
positive during radial inflow (net uptake by the particle), and
negative during radial outflow.  
This result (\ref{eqn:mlc_flx_cnt_dfn}) is due to Maxwell and so is
called the \trmdfn{Maxwellian flux}.

Experiments and intuition, however, suggest that continuum diffusion
does not apply to gas molecules within about one mean free path of the
particle surface. 
The spherical flux $\mlcflxcnt$ varies linearly with particle size
$\rdsprt$ rather than with the particle area $\rdsprt^{2}$.
This result is somewhat surprising.
As we show in \S\ref{sxn:knt}, the spherical flux predicted by kinetic 
theory (valid when $\kndnbrAB \gg 1$) does vary with $\rdsprt^{2}$.
The problem of accounting for both continuum and kinetic diffusion is
non-trivial, and will be discussed in \S\ref{sxn:trn}.  
We shall show that useful results for $\kndnbrAB \approx 0$ and for
$\kndnbrAB \gg 1$ may be expressed in terms of $\mlcflxcnt$
(\ref{eqn:mlc_flx_cnt_dfn}). 
 
Corrections to the spherical flux $\mlcflxcnt$
(\ref{eqn:mlc_flx_cnt_dfn}) for the effects of kinetic transfer to the 
aerosol surface, mass accomodation, reactive uptake, and Henry's Law 
equilibrium, may all need to be evaluated before determining the
expected molecular flux to a particle.
Let us denote the actual molecular uptake by the particle as $\mlcflx$ 
and define generic properties $\mlcflx$ which are independent of
exactly how $\mlcflx$ is defined.

We may easily manipulate $\mlcflxcnt$ to obtain the rate of
growth/shrinkage of droplet mass $\mss$, and the rate of change of 
concentration of \A\
\begin{eqnarray}
% [kg s-1] SeP97 p. 600 (11.25) PrK98 p. 504 (13-9)
\left( \frac{\dfr\mss}{\dfr\tm } \right)_{0} & = & \mlcflxcnt \mmwA \nonumber \\
& = & 4 \mpi \rdsprt \dffA \mmwA ( \cncinf - \cncsfc )
\label{eqn:mss_flx_cnt_dfn}
\end{eqnarray}
where the zero subscript indicates the result applies only to
stationary particles. 
For $\A = \HdO$, (\ref{eqn:mss_flx_cnt_dfn}) is the classical solution
to diffusion equation for condensational growth of cloud drops. 

The maximum diffusive flux of \A\ to the particle
(\ref{eqn:mlc_flx_cnt_dfn}) occurs when $\cncsfc = 0$ 
\begin{eqnarray}
% SeP97 p. 618 (11.80)
\mlcflxcntmax & = & 4 \mpi \rdsprt \dffA \cncinf
\label{eqn:mlc_flx_cnt_max_dfn}
\end{eqnarray}
If the rate of change of ambient [\A] is due solely to mass uptake
following gaseous diffusion, then (\ref{eqn:mlc_flx_cnt_max_dfn})
implies 
\begin{eqnarray}
% BOT99 p. 118 (3.40)
\frac{d [\A]}{d \tm } & = & - 4 \mpi \rdsprt \dffA [\A] \\
& = & - \mlcflxcntmax
\label{eqn:snk_cnt_A}
\end{eqnarray}

One subtlety to note is that we have not modified the flux due
bulk diffusion $\mlcflxcnt$ (\ref{eqn:mlc_flx_cnt_dfn}) by the
accomodation coefficient $\mssacmcff$. 
The reason for this is that we wish to avoid double counting 
$\mssacmcff$. 
The continuum fluxes predicted in this section should not be applied
in isolation.
These fluxes describe the first resistance a molecule must overcome in
order to meet a surface.
Bulk phase diffusion merely describes the rate at which molecules flow
by a stationary particle.
When molecules are close enough to the particle, they may collide with
the particle, but only by making a kinetic jump (Brownian diffusion)
from the near surface to the surface of the particle.
The mass accomodation coefficient is defined to modify the efficiency
of this final step, the kinetic collision described in the next
section. 

\subsection[Kinetic Regime]{Kinetic Regime}\label{sxn:knt}
In this section we examine the diffusion of a gaseous species \A\ to 
the surface of a spherical particle in the kinetic regime, i.e.,
$\kndnbrAB \gg 1$.
Note that the kinetic regime describes the actions of any molecule in
isolation, including the final jump of a molecule in a continuum to
the particle interface.
This is called \trmdfn{Interfacial transport}.
Thus we shall apply the kinetic regime results to describe the final
step of diffusive transfer of bulk gas molecules to a particle.

The \trmdfn{equipartition theorem} states that the thermal energy
available is equally divided among translational and rotational
degrees of freedom % fxm: verify this in stat mech text
The thermal energy of a molecule varies linearly with the temperature,
i.e., $\nrg \sim \bltcst \tpt$.
When the thermal energy is expressed as kinetic energy,
\begin{eqnarray}
% [m s-1] SeP97 p. 453 (8.2)
\bltcst \tpt & \propto & \mss \vlc^{2} \nonumber \\
\vlc & \propto & \sqrt{ \frac{\tpt}{\mss } }
\label{eqn:vlc_tpt_dfn}
\end{eqnarray}
Thus, the square-root dependence of $\vlcmlc$ on temperature and mass
stems from equipartition theorem.

The mean speed of molecules in random thermal motion may be computed
from \trmdfn{Maxwell-Boltzmann statistics}. 
The mean molecular speed $\vlcmlcavg$ is obtained by averaging the 
Maxwell-Boltzmann distribution of speeds.
For the gaseous species \A\ with molecular mass $\mmwA$, 
\begin{eqnarray}
% [m s-1] SeP97 p. 453 (8.2)
\vlcmlcavgA & = & \sqrt{\frac{8 \gascstunv \tpt}{\mpi \mmwA}}
\label{eqn:vlc_mlc_avg_dfn}
\end{eqnarray}
where $\gascstunv$ is the universal gas constant and $\tpt$ is the
ambient temperature. 
$\vlcmlcavgA$ is call the \trmdfn{thermal speed} of the molecule.
For oxygen molecules at STP, $\vlcmlcavgOd = 425$\,\mxs.
For ozone molecules at 300\,K\, $\vlcmlcavgOt = 364$\,\mxs.

The kinetic theory of gases can be used to compute the number flux of
\A, i.e., the number of molecules of \A\ crossing a unit surface in a
unit time.
Denoting the number concentration of species \A\ by $\cncA = [\A]$,
\begin{eqnarray}
% [mlc s-1] SeP97 p. 600 (11.23)
\mlcflxtldA & = & \frac{1}{4} \cncA \vlcmlcavgA
\label{eqn:mlc_flx_tld_knt_dfn}
\end{eqnarray}
where the subscript $\kntsbs$ denotes that the flux applies only in
the kinetic regime.
As expected, $\mlcflxtldA$ varies linearly with $\cncA$. 
Since Brownian diffusion is isotropic, the orientation of the surface
does not appear in (\ref{eqn:mlc_flx_tld_knt_dfn}).
Thus the flux to a concave surface is simply $\mlcflxtldA$ times the
surface area.
A sphere of radius $\rdsprt$ has a surface area of $4\mpi \rdsprt^{2}$ so
that the number $\mlcflxstrknt$ of kinetic collisions of \A\ with the 
surface is 
\begin{eqnarray}
\mlcflxstrknt & = & \mpi \rdsprt^{2} \cncA \vlcmlcavgA
\label{eqn:mlc_flx_str_knt_dfn}
\end{eqnarray}
where the superscript $^{\strsbs}$ denotes that the flux does not
account for uptake or reaction probabilities.

For heterogeneous chemistry, we are interested in the net uptake of
\A\ by the surface.
As mentioned earlier, molecules of \A\ may be desorbed from the
surface as well as adsorbed by the surface.
Thus the net flux of \A\ to the surface is determined by the gradient
in the ambient vapor and the surface concentrations, $\cncinf$ and
$\cncsfc$, respectively.  
Also we may impose the condition that only a fraction $\mssacmcff$ of
the surface collisions result in surface capture of the molecule.
With these two conditions, (\ref{eqn:mlc_flx_str_knt_dfn}) becomes
\begin{eqnarray}
% [mlc s-1] SeP97 p. 600 (11.25)
\mlcflxknt & = & \mpi \rdsprt^{2} \vlcmlcavgA \mssacmcff ( \cncinf - \cncsfc ) 
\label{eqn:mlc_flx_knt_dfn}
\end{eqnarray}
Laboratory measurements of $\mssacmcff$ are possible, though
difficult. 
Table~\ref{tbl:acm_cff} lists measurements of $\mssacmcff$ relevant to
mineral dust chemistry.

A useful metric of the sensitivity of $\mlcflx$ to the difference
between the kinetic regime and the continuum regimes is obtained by
taking the ratio of (\ref{eqn:mlc_flx_knt_dfn}) to
(\ref{eqn:mlc_flx_cnt_dfn}). 
\begin{eqnarray}
\frac{\mlcflxknt}{\mlcflxcnt } & = & 
\frac{\mssacmcff \vlcmlcavgA \rdsprt}{4 \dffAB }
\label{eqn:knt_cnt_rat}
\end{eqnarray}
If we assume that $\dffAB = \mfpAB \vlcmlcavgA / 3$
(\ref{eqn:mfp_AB_dfn_5}), then (\ref{eqn:knt_cnt_rat}) becomes
\begin{eqnarray}
\frac{\mlcflxknt}{\mlcflxcnt } & = & 
\frac{3 \mssacmcff \rdsprt}{4 \kndnbrAB }
\label{eqn:knt_cnt_rat_2}
\end{eqnarray}

The maximum flux of \A\ to the particle through kinetic collisions
occurs when $\cncsfc = 0$ 
\begin{eqnarray}
\mlcflxkntmax & = & \mpi \rdsprt^{2} \vlcmlcavgA \mssacmcff \cncinf
\label{eqn:mlc_flx_knt_max_dfn}
\end{eqnarray}
If the rate of change of ambient [\A] is due solely to mass uptake of
kinetic collisions, then (\ref{eqn:mlc_flx_knt_max_dfn}) implies
\begin{eqnarray}
% BOT99 p. 118 (3.39)
\frac{d [\A]}{d \tm } & = & - \mpi \rdsprt^{2} \vlcmlcavgA \mssacmcff [\A] \\
& = & - \mlcflxkntmax
\label{eqn:snk_knt_A}
\end{eqnarray}

\subsection[Transition Regime]{Transition Regime}\label{sxn:trn}
As shown in the previous two sections, the rate of uptake of gas phase
molecules by a surface depends on the ``regime'' which characterizes
the mean free path of the molecules relative to the size of the
particle.  
When the Knudsen number $\kndnbr \gg 1$, only the kinetic regime is
important.  
Bulk phase diffusion is too slow to compete with kinetic impacts from
the far field, and (\ref{eqn:mlc_flx_knt_dfn}) describes the loss of
\A\ to the surface.
When $\kndnbr$ is of order unity or less, both kinetic impacts and
continuum, bulk phase diffusion effects are important.  
Bulk phase diffusion serves to transport [\A] within order one mean
free path of the particle, and the final step to the particle surface
is a reversible kinetic jump.
In practice there is no clear distinction between the two regimes, but 
the rate of uptake should smoothly transition between the two.

In the troposphere, continuum diffusion is significant to atmospheric
aerosols of interest.
We shall first estimate the total surface uptake by assuming that
continuum that kinetic transport acts in series to bulk gas phase
diffusion.  
This assumption neglects matching the boundary conditions between the
continuum and kinetic regimes, but serves to illustrate the overall
physics of the problem.
A solution which accounts for the boundary conditions will be shown
below. 
First we define the characteristic time scales for diffusive and
interfacial transport, using (\ref{eqn:snk_cnt_A}) and
(\ref{eqn:snk_knt_A}), respectively
\begin{eqnarray}
% BOT99 p. 118 (3.42)
\taudff & = & 4 \mpi \rdsprt \dffA \\
\tauntf & = & \mpi \rdsprt^{2} \vlcmlcavgA \mssacmcff
\label{eqn:tau_dff_ntf}
\end{eqnarray}

Adding (\ref{eqn:snk_cnt_A}) in series to (\ref{eqn:snk_knt_A}) to
obtain the total uptake 
\begin{eqnarray}
% BOT99 p. 118 (3.42)
\frac{1}{\tauttl } & = & \frac{1}{\taudff } + \frac{1}{\tauknt }
\nonumber \\
& = & \frac{1}{4 \mpi \rdsprt \dffA } + \frac{1}{\mpi \rdsprt^{2}
\vlcmlcavgA \mssacmcff } \nonumber \\
& = & \frac{1}{4 \mpi \rdsprt \dffA } \times
\frac{\mssacmcff \vlcmlcavgA \rdsprt}{\mssacmcff \vlcmlcavgA \rdsprt } +
\frac{1}{\mpi \rdsprt^{2} \vlcmlcavgA \mssacmcff } \times
\frac{4 \dffA}{4 \dffA } \nonumber \\
& = & \frac{4 \dffA + \mssacmcff \vlcmlcavgA \rdsprt}{
4 \mpi \mssacmcff \vlcmlcavgA \rdsprt^{2} \dffA } \nonumber \\
\tauttl & = & \frac{4 \mpi \mssacmcff \vlcmlcavgA \rdsprt^{2} \dffA}{
4 \dffA + \mssacmcff \vlcmlcavgA \rdsprt } 
\label{eqn:snk_ttl_A}
\end{eqnarray}

The most common assumption which rigorously meshes the kinetic theory
with the continuum theory of gaseous deposition is due to Fuchs (1964).
His technique is to match the flux prediction by continuum transport
theory to the flux predicted by kinetic transport theory at a distance
$\Delta$ from the surface of the particle.
Thus flux matching assumes that continuum processes control the
transport of the gas from the ambient background to a radial distance
$\Delta$ from the surface of the particle.
For $\rdsprt < \rds < \rdsprt + \Delta$, the gas transfer is assumed
to obey kinetic theory.
This is a reasonable hypothesis, since, at the microscopic level, the
last jump of a vapor phase molecule before colliding with the surface
must appear to the molecule as a kinetic jump.
Thus the flux matching technique recognizes that the fluid continuum
description does not apply to individual molecules on the microscopic
level.
Implicit in the flux matching assumption is that $\Delta \sim \mfp$.
The exact value of $\Delta$ depends on the form of the assumptions
assumptions employed.
We now present the original formulation of Fuchs (1964), which makes
no assumption on the relative masses of the trace gas \A\ and the
medium \B.
We shall assume that the medium \B\ is air.

The solution of this problem leads to an expression for 
$\mlcflx$ in terms of $\dffAB$.
As noted previously, there are many possible expressions for $\dffAB$
in terms of $\mfpAB$.
Substituting $\dffAB = \mfpAB \vlcmlcavgA / 3$ and expressing the
result in terms of $\mlcflxcnt$ (\ref{eqn:mlc_flx_cnt_dfn}) leads to 
\begin{eqnarray}
% [mlc s-1] SeP97 p. 602 (11.31)
\frac{\mlcflx}{\mlcflxcnt } & = & 
\frac{3}{4} \times \frac{1 + \kndnbr \Delta / \mfp}{
\frac{3}{4} + \kndnbr + ( \Delta / \mfp ) \kndnbr^{2} }
\label{eqn:mlc_flx_Fuc64_1}
\end{eqnarray}
where the subscript $_{\A \B}$ is implied for all quantities.
Using the assumptions of (\ref{eqn:knt_cnt_rat_2}),
(\ref{eqn:mlc_flx_Fuc64_1}) implies
\begin{eqnarray}
% [mlc s-1] SeP97 p. 602 (11.31)
\frac{\mlcflx}{\mlcflxknt } & = & 
\frac{1 + \kndnbr \Delta / \mfp}{
1 + ( \kndnbr \Delta / \mfp ) + \frac{3}{4} \kndnbr^{-1} }
\label{eqn:mlc_flx_dfn_Fuc64_2}
\end{eqnarray}
Fuchs and Sutugin (1971, MFC) proposed, for $\mssacmcff = 1$, \cite[]{SeP97}
\begin{eqnarray}
% [mlc s-1] SeP97 p. 602 (11.34)
\frac{\mlcflx}{\mlcflxcnt } & = & 
\frac{1 + \kndnbr}{1 + 1.71 \kndnbr + \frac{4}{3} \kndnbr^{2} }
\label{eqn:mlc_flx_dfn_Fuc71_1}
\end{eqnarray}
which, for $\mssacmcff \ne 1$, becomes
\begin{eqnarray}
% [mlc s-1] SeP97 p. 606 (11.43)
\frac{\mlcflx}{\mlcflxcnt } & = & 
\frac{\frac{3}{4} \mssacmcff ( 1 + \kndnbr ) }{
\kndnbr^{2} + \kndnbr + 0.283 \kndnbr \mssacmcff + \frac{3}{4} \mssacmcff }
\label{eqn:mlc_flx_dfn_Fuc71_2}
\end{eqnarray}

Many variations of (\ref{eqn:mlc_flx_Fuc64_1}) have been proposed.
Fuchs and Sutugin (1970, MFC) proposed \cite[]{BOT99}
\begin{eqnarray}
% [mlc s-1] BOT99 p. 118 (3.43)
- \frac{d [\A]}{d \tm } & = & 
\mpi \mssacmcff \vlcmlcavgA \rdsprt^{2} [\A]
\left( 1 + \frac{3 \mssacmcff ( 1 + 0.47 \kndnbr )}{
4 \kndnbr ( 1 + \kndnbr ) } \right)^{-1}
\label{eqn:mlc_flx_dfn_Fuc70_1}
\end{eqnarray}
It is unclear whether (\ref{eqn:mlc_flx_dfn_Fuc70_1}) is or should be
identical to (\ref{eqn:mlc_flx_dfn_Fuc71_2})

\subsection[Aerosol Motion]{Aerosol Motion}\label{sxn:mtn}

\section[Diffusion Limited Rates]{Diffusion Limited Rates}

\chapter{Aqueous Chemistry}\label{sxn:chm_aqs}
Many different physical quantities are used to measure the abundance
of a dissolved species in liquid or particulate matter.
When the phase of the species and particle is liquid, we refer to the
trace species as the \trmdfn{solute} and the bulk substance (usually
water) as the \trmdfn{solvent}. 
The mixture of the solute and the solvent is called the
\trmdfn{solution}. 
Most often in atmospheric chemistry the solvent is considered to be
pure liquid water.
However, a significant fraction of some aerosols may be solute.
For example, a typical stratospheric sulfuric acid aerosol may be
composed of 75\% \HdSOq\ and 25\% \HdO\ by weight.

Aqueous phase chemical reactions are usually expressed in terms of the
solute \trmdfn{molarity} or \trmdfn{molar concentration} $\mlrcnc$.
Molarity expresses the solute concentration in \molxl. 
However, since the volume of a solution depends on its temperature,
molar concentration changes with temperature.
Concentration may also be expressed in terms of \trmdfn{molality}, or
\molxkg.
Molality has the advantage of being analogous to a mass mixing ratio,
i.e., it does not change due to temperature or volume fluctuations.
Because 1\,\kg\ of liquid water is approximately 1\,\ltr\ at STP, dilute
aqueous solutions in water have the property that the molarity is
approximately numerically equal to the molality.

The molality of a solution is proportional to the mass fraction of
solute in the solution.

Another useful measure is the \trmdfn{molar concentration}, measured
in \molxmC.
The molar concentration $\mlrcnc$ and molality $\mll$ may be expressed
in terms of eachother using
\begin{eqnarray}
% SeP97 p. 1295
\label{eqn:mll_dfn}
\mll & = & \frac{\mlrcnc}{\dns - \mlcwgt \mlrcnc} \\
\label{eqn:mlrcnc_dfn}
\mlrcnc & = & \frac{\mll \dnssln}{\mlcwgt \mll + 1}
\end{eqnarray}
where $\dnssln$ is the density of the solution (solute + solvent) and 
$\mlcwgt$ is the molecular weight of the solute.

\section{Henry's Law}
The English chemist William Henry first quantified the linear
dependence of the molar fraction of ideal dilute solutions.
He related the partial pressure $\prsprtA$ of a gaseous species \A\
to the molality~$\mllA$ of~\A\ in an aqueous dilute solution.
Molality is another term for volume mixing ratio, i.e., the number of
molecules of~\A\ per molecule of liquid.
According to \trmdfn{Henry's Law}
\begin{equation}
\mllAbrk = \hnrA \, \prsprtA
\label{eqn:hnr_law_dfn}
\end{equation}
where $\hnrA$ is the Henry's law coefficient of species~\A.
The traditional units of $\hnrA$ are \molxlatm, and these are the
dimensions used by \cite{SeP97}.
In these units, $\prsprtA$ should be specified in atmospheres (atm),
and $\mllAbrk$ will be in \molxl.
The dimensions of \molxl\ are often abbreviated simply as~M.
Thus the dimensions of $\hnrA$ are often written \Mxatm.
However, many other units for $\hnr$ are used in the literature. 
The review article \cite{San991} presents an excellent summary of the
various definitions, dimensions, and conversion factors for Henry's
law constants.

Further complicating matters, some chemists define $\hnrA$ by placing
it on the LHS of (\ref{eqn:hnr_law_dfn}) \cite[e.g.,][]{Atk90}.
Whichever system of units and definition of $\hnr$ are employed, it is
imperative to be sure that the dimensions of $\mllAbrk$, $\hnrA$, and
$\prsprtA$ are commensurate. 
Using \molxlatm\ in (\ref{eqn:hnr_law_dfn}), $\hnrA$ increases with
solubility and very soluble gases have large Henry's Law coefficients 
(e.g., $\hnrHNOt = 2.1 \times 10^{5}$\,\molxlatm). 

Table~\ref{tbl:hnr} shows $\hnr$ at 298\,K\ for many important
atmospheric compounds.
\begin{table}
\begin{minipage}{\hsize} % Minipage necessary for footnotes KoD95 p. 110 (4.10.4)
\renewcommand{\footnoterule}{\rule{\hsize}{0.0cm}\vspace{-0.0cm}} % KoD95 p. 111
\begin{center}
\caption[Henry's Law Coefficients]{\textbf{Henry's Law
coefficients of Gases Dissolving in Liquid Water}%
\footnote{\emph{Sources:} \cite{SeP97}, p.~341; \cite{JPL97}, p.~246,
Table~65, \url{http://www.mpch-mainz.mpg.de/~sander/res/henry.html}}%   
\label{tbl:hnr}}
\vspace{\cpthdrhlnskp}
\begin{tabular}{ r l >{$}l<{$} }
\hline \rule{0.0ex}{2.0ex} 
\footnote{Compounds are listed in order of increasing solubility}%
Formula & Name & 
\mbox{\footnote{Values of $\hnr$ (\molxlatm) are for 298\,K}%
$\hnr_{298}$} \\[0.0ex] 
& & \mbox{\Mxatm} \\[0.0ex]
\hline \rule{0.0ex}{2.0ex} 
\Od & Oxygen & 1.3 \times 10^{-3} \\[0.3ex]
\NO & Nitric oxide & 1.9 \times 10^{-3} \\[0.3ex]
\CdHq & & 4.8 \times 10^{-3} \\[0.3ex]
\NOd & Nitrogen dioxide & 1.0 \times 10^{-2} \\[0.3ex]
\Ot & Ozone & 1.13 \times 10^{-2} \\[0.3ex]
\NdO & Nitrous oxide & 2.5 \times 10^{-2} \\[0.3ex]
\COd & Carbon dioxide & 3.4 \times 10^{-2} \\[0.3ex]
\HdS & Hydrogen sulfide & 0.12 \\[0.3ex]
\DMS & Dimethyl sulfide (DMS) & 0.56 \\[0.3ex]
\SOd & Sulfur dioxide & 1.23 \\[0.3ex]
\OH & Hydroxyl & 25 \\[0.3ex]
\HNOd & Nitrous acid & 49 \\[0.3ex]
\NHt & Ammonia & 62 \\[0.3ex]
\HCl & Hydrogen chloride & 727 \\[0.3ex]
\HOd & Hydroperoxyl & 2.0 \times 10^{3} \\[0.3ex]
\HdOd & Hydrogen peroxide & 7.45 \times 10^{4} \\[0.3ex]
\HNOt & Nitric acid & 2.1 \times 10^{5} \\[0.3ex]
\footnote{\cite{JPL97}, p.~246, Table~65 report $\hnrNOt = 0.6 \pm  
0.3$\,\molxlatm. The discrepancy is unexplained.}%
\NOt & Nitrate & 2.1 \times 10^{5} \\[0.3ex]
\hline
\end{tabular}
\end{center}
\end{minipage}
\end{table}
\cite{San992} maintains an online database of Henry's Law coefficients 
and useful ancillary material at
\url{http://www.mpch-mainz.mpg.de/~sander/res/henry.html}.
The solubility of these compounds spans eight orders of magnitude.
Liquid water is relatively impermeable to gases such as \NO\ and \Ot. 
Most of the mass of these insoluble species is present in the gas
phase in the interstitial air between aerosols and clouds.  
Since fewer molecules of insoluble species must diffuse into the
liquid phase, these species quickly adjust to a Henry's Law
equilibrium in typical clouds.
Highly soluble species, on the other hand, may be \trmdfn{diffusion
limited} and take a long time to reach Henry's Law equilibrium.
Thus more care must be taken before assuming gases such as \HNOt\ and
\HdOd\ are in Henry's Law equilibrium.

We emphasize that Henry's Law (\ref{eqn:hnr_law_dfn}) is valid only
for dilute solutions of \A.   
Solutions comprised mostly of \A\ obey another law, \trmdfn{Raoult's
Law} \cite[e.g.,][]{Atk90,SeP97}. 

The temperature dependence of $\hnrA$ is given by 
\begin{equation}
% SeP97 p. 342 (6.5)
\hnrA(\tpt) = \hnrA( \tpt_{0} ) \exp\! 
\left[ \frac{\dltnthA}{\gascstunv} 
\left( \frac{1}{\tpt_{0}} - \frac{1}{\tpt} \right)
\right]
\label{eqn:hnr_tpt_dfn}
\end{equation}
where $\gascstunv$ is the universal gas constant and $\dltnthA$ is the
change in enthalpy of \A\ due to its dissolution into the aqueous
phase, i.e., the heat released by the phase change of \A\ from the
vapor to the aqueous state.  
$\dltnthA$ is sometimes called the \trmdfn{heat of dissolution}.

\section{Aqueous Equilibria}
The most important liquid in the atmosphere is water.
Solutions for which liquid water is the solvent are called
\trmdfn{aqueous solutions}.
Many trace gases dissolved in an aqueous solution rapidly dissociate
into ions. 
Table~\ref{tbl:aqs_eqm} lists the equilibrium constant and temperature
dependence of many important atmospheric species in equilibrium with
liquid water.
\begin{table}
\begin{center}
\begin{minipage}{\hsize} % Minipage necessary for footnotes KoD95 p. 110 (4.10.4)
\renewcommand{\footnoterule}{\rule{\hsize}{0.0cm}\vspace{-0.0cm}} % KoD95 p. 111
\caption[Aqueous Equilibrium Constants]{\textbf{Aqueous Equilibrium
Constants of Important Atmospheric Trace Gases}%
\footnote{\emph{Sources:} Adapted from \cite{SeP97}}%
\label{tbl:aqs_eqm}}   
\vspace{\cpthdrhlnskp}
\begin{tabular}{ >{$}r<{$} >{$\ch}r<{$} >{$\ch}c<{$} >{$\ch}l<{$} >{$}l<{$} >{$}r<{$} >{$}r<{$} l }
\hline \rule{0.0ex}{\hlntblhdrskp}% 
\mbox{Symbol} & \multicolumn{3}{c}{Equilibrium} & 
\mbox{\footnote{Values for equilibrium constant $\eqmcst$, and reaction
enthalpies $\dltnth$ and $\dltnth/\gascstunv$ are specified at 298\,K}}% 
\eqmcst_{298} & ^{b}\dltnth_{298} & ^{b}\dltnth_{298}/\gascstunv & Ref. \\[0.0ex]
\rule[-0.5ex]{0.0ex}{0.5ex} & & & & \mbox{\molxl} & \mbox{\kcalxmol} & \mbox{K} & \\[0.0ex]
\hline \rule{0.0ex}{\hlntblntrskp}% 
\mbox{\footnote{As shown in (\ref{eqn:eqm_cst_H2O_dfn}), the units of
$\eqmcst_{\mbox{\tiny\ref{rxn:H2O_H++OH}}}$ are \MS.}}%
\eqmcst_{\mbox{\tiny\ref{rxn:H2O_H++OH}}} & \HdO & \eqbm & \Hp + \OHm & 1.0 \times 10^{-14}
& 13.35 & -6710 & SeP97 \\[0.3ex] % SeP97 p. 344 (6.10)
\eqmcst_{\mbox{\tiny\ref{rxn:CO2H2O_H++HCO3-}}} & \COdHdO & \eqbm & \Hp + \HCOtm & 
4.3 \times 10^{-7} & 1.83 & -1000 & SeP97 \\[0.3ex] % SeP97 p. 346 (6.15)
\eqmcst_{\mbox{\tiny\ref{rxn:HCO3-_H++CO3--}}} & \HCOtm & \eqbm & \Hp + \COtdm & 
4.7 \times 10^{-11} & 3.55 & -1760 & SeP97 \\[0.3ex] % SeP97 p. 346 (6.16)
\eqmcst_{\mbox{\tiny\ref{rxn:SO2H2O_H++HSO3-}}} & \SOdHdO & \eqbm & \Hp + \SOtdm & 
1.3 \times 10^{-2} & -4.16 & 1960 & SeP97 \\[0.3ex] % SeP97 p. 348 (6.31)
\eqmcst_{\mbox{\tiny\ref{rxn:HSO3-_H++SO3--}}} & \HSOtm & \eqbm & \Hp + \SOtdm & 
6.6 \times 10^{-8} & -2.23 & 1500 & SeP97 \\[0.3ex] % SeP97 p. 348 (6.32)
\eqmcst_{\mbox{\tiny\ref{rxn:NH3H2O_OH-+NH4+}}} & \NHtHdO & \eqbm & \OHm + \NHqp & 
1.7 \times 10^{-5} & 8.65 & -450 & SeP97 \\[0.3ex] % SeP97 p. 353 (6.47)
\eqmcst_{\mbox{\tiny\ref{rxn:HNO3H2O_H++NO3-}}} & \HNOtHdO & \eqbm & \Hp + \NOtm & 
15.4 & \mbox{???} & 8700 & SeP97 \\[0.3ex] % SeP97 p. 355 (6.54)
\eqmcst_{\mbox{\tiny\ref{rxn:H2O2H2O_H++HO2-}}} & \HdOdHdO & \eqbm & \Hp + \HOdm & 
2.2 \times 10^{-12} & \mbox{???} & -3730 & SeP97 \\[0.3ex] % SeP97 p. 357 (6.63)
\hline
\end{tabular}
\end{minipage}
\end{center}
\end{table}

\subsection[Water]{Water}\label{sxn:aqs_eqm_H2O}
A small fraction of the water solvent dissociates into hydrogen ions
and hydroxide ions 
\begin{rxnarray}
\HdO & \eqbm & \Hp + \OHm % SeP97 p. 344 (6.10)
\label{rxn:H2O_H++OH}
\end{rxnarray}
The equilibrium constant for (\ref{rxn:H2O_H++OH}) $\eqmcstHdOprm$ is
\begin{equation}
\eqmcstHdOprm = \frac{[\Hp] [\OHm]}{[\HdO] } % SeP97 p. 363 (6.11)
\label{eqn:eqm_cst_H2O_prm_dfn}
\end{equation}
where the measured value of $\eqmcstHdOprm$ at 298\,K\ is $1.82 \time
10^{-16}$~M.
Such a small number of ions per \HdO\ molecule means that [\HdO] is
constant to a very good approximation.
Studies of bulk liquid water tell us that the molar concentration of
pure water, [\HdO], is about $55.5$~M.
Since [\HdO] is nearly constant, it is usual to define a new
equilibrium constant for (\ref{rxn:H2O_H++OH}) which subsumes [\HdO]
into its definition  
\begin{eqnarray}
\eqmcstHdO & = & \eqmcstHdOprm [\HdO] \nonumber \\ % SeP97 p. 363
& = & [\Hp] [\OHm] \nonumber \\ % SeP97 p. 345 (6.12)
& = & 1.0 \times 10^{-14} \mbox{\,\MS\ at 298\,K} % SeP97 p. 345
\label{eqn:eqm_cst_H2O_dfn}
\end{eqnarray}
An explicit expression for [\OHm] in terms of [\Hp] is useful in
expressing the charge balance for the solution (discussed below)
\begin{equation}
[\OHm] = \frac{\eqmcstHdO}{[\Hp] }
\label{eqn:cnc_OH-}
\end{equation}

In pure liquid water there can be only one hydrogen ion per hydroxide
ion so that $[\Hp] = [\OHm] = 1.0 \times 10^{-7}$~M
(\ref{rxn:H2O_H++OH}).  
By convention, the concentration of hydrogen ions in a solution
defines the \trmdfn{\pH} of the solution
\begin{eqnarray}
\pH & = & - \log_{10}[\Hp] % SeP97 p. 345 (6.13)
\label{eqn:pH_dfn}
\end{eqnarray}
Thus as [\Hp] increases, \pH\ decreases.
According to (\ref{eqn:eqm_cst_H2O_dfn}) the \pH\ of pure water at
298\,K\ is~7.0.
Note that [\Hp] in (\ref{eqn:pH_dfn}) must be in units of 
$\mathrm{M} \equiv$~\molxl\ to produce standard \pH\ values.

\subsection[Carbon Dioxide]{Carbon Dioxide \COd}\label{sxn:aqs_eqm_CO2}
Carbon dioxide is the most abundant trace gas in the atmosphere
following \HdO. 
Since \COd\ has a very long atmospheric residence time, it is well
mixed in the troposphere and stratosphere.
The anthropogenic contribution of \COd\ to the atmosphere has made the
question of net \COd\ uptake or release by the oceans very important.
Carbonates are also prevalent in the atmosphere on mineral dust
aerosol. 
These carbonates may play a role in buffering acid rain.
To gain understanding of these processes we now examine the aqueous
equilibria of \COd\ and its products.

A \COd\ molecule dissolved in pure water ends in one of
three complexes: \COdHdO, \HCOtm, or \COtdm\
\begin{rxnarray}
\label{rxn:CO2_CO2H2O}
\COd (g) + \HdO & \eqbm & \COdHdO \\ % SeP97 p. 346 (6.14)
\label{rxn:CO2H2O_H++HCO3-}
\COdHdO & \eqbm & \Hp + \HCOtm \\ % SeP97 p. 346 (6.15)
\label{rxn:HCO3-_H++CO3--}
\HCOtm & \eqbm & \Hp + \COtdm % SeP97 p. 346 (6.16)
\end{rxnarray}
When it dissolves in water, \COd\ molecules attach to \HdO\ forming
\COdHdO\ (\ref{rxn:CO2_CO2H2O}). 
This is called \trmdfn{hydrolysis}. 
Hydrolyzed \COd\ is written either \COdHdO\ or \COd~(aq).
The symbols are different but denote the same compound.
The \COdHdO\ may dissociate into a hydrogen ion and a bicarbonate ion 
(\ref{rxn:CO2H2O_H++HCO3-}).
The \HCOtm\ may further dissociate into a hydrogen and a carbonate
ion (\ref{rxn:HCO3-_H++CO3--}).

The equilibrium constants of equilibrium reactions
(\ref{rxn:CO2_CO2H2O})--(\ref{rxn:HCO3-_H++CO3--}) are
\begin{eqnarray}
\label{eqn:eqm_CO2_CO2H2O}
\eqmcst_{\mbox{\scriptsize \ref{rxn:CO2_CO2H2O}}} & = & \frac{[\COdHdO]}{\prsprtCOd } \\ % SeP97 p. 346 (6.17)  
& \equiv & \hnrCOd \nonumber \\
\label{eqn:eqm_CO2H2O_H++HCO3-}
\eqmcst_{\mbox{\scriptsize \ref{rxn:CO2H2O_H++HCO3-}}} & = & \frac{[\Hp] [\HCOtm]}{[\COdHdO] } \\ % SeP97 p. 346 (6.18)
\label{eqn:eqm_HCO3-_H++CO3--}
\eqmcst_{\mbox{\scriptsize \ref{rxn:HCO3-_H++CO3--}}} & = & \frac{[\Hp] [\COtdm] }{
[\HCOtm] } % SeP97 p. 346 (6.18)
\end{eqnarray}
Note that (\ref{eqn:eqm_CO2_CO2H2O}) is the definition of the Henry's
Law coefficient for \COd\ (\ref{eqn:hnr_law_dfn}).
The measured values of $\eqmcst_{\mbox{\scriptsize \ref{rxn:CO2H2O_H++HCO3-}}}$ and
$\eqmcst_{\mbox{\scriptsize \ref{rxn:HCO3-_H++CO3--}}}$ at 298\,K\ are given in
Table~\ref{tbl:aqs_eqm}. 

The concentrations of \COdHdO, \HCOtm, and \COtdm\ may be expressed in
terms of (\ref{eqn:eqm_CO2_CO2H2O})--(\ref{eqn:eqm_HCO3-_H++CO3--}).
Rearranging (\ref{eqn:eqm_CO2_CO2H2O})
\begin{eqnarray}
\label{eqn:cnc_CO2H2O}
[\COdHdO] & = & \hnrCOd \prsprtCOd % SeP97 p. 346 (6.20)  
\end{eqnarray}
Again, this is simply a statement of Henry's Law equilibrium
(\ref{eqn:hnr_law_dfn}). 
Note that [\COdHdO] depends on temperature through $\hnrCOd$
(\ref{eqn:hnr_tpt_dfn}). 
Making use of (\ref{eqn:cnc_CO2H2O}) in (\ref{eqn:eqm_CO2H2O_H++HCO3-})
\begin{eqnarray}
\label{eqn:cnc_HCO3m}
[\HCOtm] & = & \frac{\eqmcst_{\mbox{\scriptsize \ref{rxn:CO2H2O_H++HCO3-}}} [\COdHdO]}{[\Hp] } \nonumber \\
& = & \frac{\prsprtCOd \hnrCOd \eqmcst_{\mbox{\scriptsize \ref{rxn:CO2H2O_H++HCO3-}}}}{[\Hp] } % SeP97 p. 346 (6.21)  
\end{eqnarray}
In addition to temperature, [\HCOtm] also depends on \pH\@.
Using (\ref{eqn:cnc_HCO3m}) in (\ref{eqn:eqm_HCO3-_H++CO3--}) 
\begin{eqnarray}
\label{eqn:cnc_CO3dm}
[\COtdm] & = & \frac{\eqmcst_{\mbox{\scriptsize \ref{rxn:HCO3-_H++CO3--}}} [\HCOtm]}{[\Hp] } \nonumber \\ % SeP97 p. 346 (6.22)
& = & \frac{\prsprtCOd \hnrCOd \eqmcst_{\mbox{\scriptsize \ref{rxn:CO2H2O_H++HCO3-}}} \eqmcst_{\mbox{\scriptsize \ref{rxn:HCO3-_H++CO3--}}}}{[\Hp]^{2} } % SeP97 p. 346 (6.22)
\end{eqnarray}
Thus [\COtdm] increases rapidly with increasing \pH\@.
If this seems confusing, remember that \pH\ increases as [\Hp]
decreases. 

\HCOtm\ and \COtdm\ are inevitable products of \COd\ dissolution.
The \COdHdO\ which dissociates into these products is replaced by
additional dissolved \COd.
This additional \COd\ is not accounted for in the definition of
Henry's Law (\ref{eqn:hnr_law_dfn}).
Thus (\ref{eqn:cnc_CO2H2O}) does not account for the \COd\ which was  
subsequently converted to \HCOtm\ and \COtdm.
In order to compute the total amount of \COd\ dissolved in the
droplets in any form, we define $\COdttl$ as 
\begin{equation}
% SeP97 p. 346 (6.23)
[\COdttl] = [\COdHdO] + [\HCOtm] + [\COtdm]
\label{eqn:CO2_ttl_dfn}
\end{equation}
Then, using (\ref{eqn:cnc_CO2H2O})--(\ref{eqn:cnc_CO3dm}) we find
\begin{eqnarray}
\label{eqn:CO2_ttl_dfn_2}
[\COdttl] & = & \prsprtCOd \hnrCOd \left(
1 + \frac{\eqmcst_{\mbox{\scriptsize \ref{rxn:CO2H2O_H++HCO3-}}}}{[\Hp] } + 
\frac{\eqmcst_{\mbox{\scriptsize \ref{rxn:CO2H2O_H++HCO3-}}} \eqmcst_{\mbox{\scriptsize \ref{rxn:HCO3-_H++CO3--}}}}{[\Hp]^{2} }
\right)
\\ % SeP97 p. 346 (6.24)
\label{eqn:hnr_ffc_CO2_dfn}
& \equiv & \prsprtCOd \hnrffcCOd
\end{eqnarray}
where (\ref{eqn:hnr_ffc_CO2_dfn}) defines the \trmdfn{effective
Henry's Law coefficient} $\hnrffcCOd$.

The effective Henry's Law coefficient $\hnrffc$ of a compound accounts
for the total dissolution of the compound in the aqueous phase
including all dissociation pathways. 
Thus $\hnrffc \ge \hnr$ for all compounds.

Until now, we have not discussed the charge balance in the solution.
In bulk liquids the number of positive ions must equal the number of
negative ions, a condition called \trmdfn{electroneutrality}.
The electroneutrality equation for a system composed of $\NNN^+$
positive ions and $\NNN^-$ negative ions is 
\begin{equation}
\sum_{\iii = 1}^{\NNN^+} \qqq_{\iii} [\AAA_{\iii}] = 
\sum_{\iii = 1}^{\NNN^-} \qqq_{\iii} [\AAA_{\iii}]  
\label{eqn:lct_ntr_dfn}
\end{equation}
where $\qqq_{\iii}$ is the net charge and $[\AAA_{\iii}]$ the concentration
of ion $\AAA_{\iii}$. 
In a closed system containing only \COd\ and liquid water,
electroneutrality (\ref{eqn:lct_ntr_dfn}) requires that
\begin{equation}
1 \times [\Hp] = 1 \times [\OHm] + 1 \times [\HCOtm] + 2 \times [\COtdm]
\label{eqn:lct_ntr_CO2_H2O}
\end{equation}
where we have explicitly weighted each solute by its net charge
$\qqq_{\iii}$. 
Electroneutrality provides the closure condition which allows us to 
determine the concentration of all the ions in a system.

The electroneutrality condition in terms of [\Hp] is obtained by
substituting [\OH] (\ref{eqn:cnc_OH-}), [\HCOtm] (\ref{eqn:cnc_HCO3m}),
and [\COtdm] (\ref{eqn:cnc_CO3dm}) into (\ref{eqn:lct_ntr_CO2_H2O})  
\begin{equation}
[\Hp] = \frac{\eqmcstHdO}{[\Hp] } +
\frac{\prsprtCOd \hnrCOd \eqmcst_{\mbox{\scriptsize \ref{rxn:CO2H2O_H++HCO3-}}}}{[\Hp] } + 
\frac{2 \prsprtCOd \hnrCOd \eqmcst_{\mbox{\scriptsize \ref{rxn:CO2H2O_H++HCO3-}}}
\eqmcst_{\mbox{\scriptsize \ref{rxn:HCO3-_H++CO3--}}}}{[\Hp]^{2} }
\label{eqn:elc_ntr_eqn_CO2}
\end{equation}
This is a cubic equation in one unknown, [\Hp]
\begin{equation}
[\Hp]^{3} - ( \eqmcstHdO + \prsprtCOd \hnrCOd
\eqmcst_{\mbox{\scriptsize \ref{rxn:CO2H2O_H++HCO3-}}} ) [\Hp] -  
2 \prsprtCOd \hnrCOd \eqmcst_{\mbox{\scriptsize \ref{rxn:CO2H2O_H++HCO3-}}}
\eqmcst_{\mbox{\scriptsize \ref{rxn:HCO3-_H++CO3--}}} = 0
\label{eqn:Hp_CO2_H2O}
\end{equation}
The ambient temperature $\tpt$ determines the rate coefficients 
$\eqmcstHdO$, $\hnrCOd$, $\eqmcst_{\mbox{\scriptsize \ref{rxn:CO2H2O_H++HCO3-}}}$, and
$\eqmcst_{\mbox{\scriptsize \ref{rxn:HCO3-_H++CO3--}}}$ through (\ref{eqn:hnr_tpt_dfn}). 
If $\tpt$ and $\prsprtCOd$ are known, then (\ref{eqn:Hp_CO2_H2O}) 
may be solved numerically for [\Hp].

Consider a system of pure liquid water at 298\,K\ in equilibrium
solution with the \COd\ concentration typical of the preindustrial
era, $\prsprtCOd = 270$~ppm. 
For concreteness, imagine this system as a cloud droplet in a pristine 
area. 
We gather the properties describing this system from
Tables~\ref{tbl:hnr} and~\ref{tbl:aqs_eqm}
\begin{eqnarray*}
\prsprtCOd & = & 270 \times 10^{-6} \mbox{~atm} \\ 
\eqmcstHdO & = & 1.0 \times 10^{-14} \mbox{\,\MS} \\ 
\hnrCOd & = & 3.4 \times 10^{-2} \mbox{\,\Mxatm} \\  
\eqmcst_{\mbox{\scriptsize \ref{rxn:CO2H2O_H++HCO3-}}} & = & 4.3 \times 10^{-7} \mbox{~M} \\ 
\eqmcst_{\mbox{\scriptsize \ref{rxn:HCO3-_H++CO3--}}} & = & 4.7 \times 10^{-11} \mbox{~M} 
\end{eqnarray*}
To obtain the \pH, we must solve 
\begin{eqnarray}
\label{eqn:pH_x}
\xxx^{3} + \bbb \xxx + \ccc & = & 0
\end{eqnarray}
where the coefficients are defined by (\ref{eqn:elc_ntr_eqn_CO2})
\begin{eqnarray*}
\bbb & = & - ( \eqmcstHdO + \prsprtCOd \hnrCOd
\eqmcst_{\mbox{\scriptsize \ref{rxn:CO2H2O_H++HCO3-}}} ) \\
& = & -3.96 \times 10^{-12} \\ 
\ccc & = & - 2 \prsprtCOd \hnrCOd \eqmcst_{\mbox{\scriptsize \ref{rxn:CO2H2O_H++HCO3-}}} 
\eqmcst_{\mbox{\scriptsize \ref{rxn:HCO3-_H++CO3--}}} \\
& = & -371 \times 10^{-24}
\end{eqnarray*}
This solution to (\ref{eqn:pH_x}) is obtained by numerical methods
and we find that  
\begin{eqnarray}
\Hp & = & 2.70 \times 10^{-6} \mbox{~M} \nonumber \\   
\OH & = & 2.70 \times 10^{-6} \mbox{~M} \nonumber \\   
\mbox{pH} & = & x.x \nonumber \\   
\COdHdO & = & 2.70 \times 10^{-6} \mbox{~atm} \times \frac{3.4 \times
10^{-2} \mathrm{~M}}{\mathrm{atm} } = 91.8 \mbox{~nM} \nonumber \\   
\HCOtm & = & \times 10^{-6} \mbox{~atm} \times \frac{3.4 \times
10^{-2} \mathrm{~M}}{\mathrm{atm} } = 91.8 \mbox{~nM} \nonumber \\   
\COtdm & = & \times 10^{-6} \mbox{~atm} \times \frac{3.4 \times
10^{-2} \mathrm{~M}}{\mathrm{atm} } = 91.8 \mbox{~nM}
\label{eqn:pH_CO2}
\end{eqnarray}
In the year 2000, the ambient concentration of \COd\ was about
$\prsprtCOd = 355$~ppm, an increase of 31\% relative to preindustrial
levels.  
At these modern levels, the coefficients $\bbb$ and $\ccc$ in
(\ref{eqn:pH_x}) become XXX and YYY, respectively.
The concentrations in (\ref{eqn:pH_CO2}) become \ldots.
These new constants yield a \pH\ for pure rainwater of XXX.
Thus anthropogenic activities have increased the \pH\ of rain by XXX\%.

% PDF bookmarks disallow math formatting so use \texorpdfstring instead
\subsection[Sulfur Dioxide \texorpdfstring{\SOd}{SO2}]{Sulfur Dioxide \SOd}\label{sxn:aqs_eqm_SO2}
Aqueous phase chemistry plays a very important role in the overall
oxidation of \SIV\ to \SVI\ in the atmosphere, producing perhaps as
much as 50\% \cite[]{BRK00}.
\SOd\ emissions center over industrial regions, and understanding
phenomena such as acid rain and ship-tracks depends on understanding
\SOd\ aqueous phase chemistry.
The equilibrium Henry's Law constant of \SOd\ is $1.23$\,\Mxatm\ at
298\,K\@.  
Thus \SOd\ itself is not highly soluble in water.
However, as we shall show below, total \SOd\ dissolution is highly
sensitive to \pH\ and this can result in very high concentrations
of \SOd\ in cloud water. 

\SOd\ equilibrates with water by the types of paths as \COd.
A gaseous \SOd\ molecule dissolved in pure water ends in one of
three complexes: \SOdHdO, \HSOtm, or \SOtdm
\begin{rxnarray}
\label{rxn:SO2_SO2H2O}
\SOd (g) + \HdO & \eqbm & \SOdHdO \\ % SeP97 p. 346 (6.30)
\label{rxn:SO2H2O_H++HSO3-}
\SOdHdO & \eqbm & \Hp + \HSOtm \\ % SeP97 p. 346 (6.31)
\label{rxn:HSO3-_H++SO3--}
\HSOtm & \eqbm & \Hp + \SOtdm % SeP97 p. 346 (6.32)
\end{rxnarray}
After dissolving in liquid water (\ref{rxn:SO2_SO2H2O}), the
hydrolyzed sulfur dioxide may dissociate into a hydrogen ion and a
bisulfite ion (\ref{rxn:SO2H2O_H++HSO3-}).
The \HSOtm\ may further dissociate into a hydrogen and a sulfite ion
(\ref{rxn:HSO3-_H++SO3--}). 
The equilibrium constants of equilibrium reactions
(\ref{rxn:SO2_SO2H2O})--(\ref{rxn:HSO3-_H++SO3--}) are
\begin{eqnarray}
\label{eqn:eqm_SO2_SO2H2O}
\eqmcst_{\mbox{\scriptsize \ref{rxn:SO2_SO2H2O}}} & = & \frac{[\SOdHdO]}{\prsprtSOd } \\ % SeP97 p. 346 (6.33)  
& \equiv & \hnrSOd \nonumber \\
\label{eqn:eqm_SO2H2O_H++HSO3-}
\eqmcst_{\mbox{\scriptsize \ref{rxn:SO2H2O_H++HSO3-}}} & = & \frac{[\Hp] [\HSOtm]}{[\SOdHdO] } \\ % SeP97 p. 346 (6.34)
\label{eqn:eqm_HSO3-_H++SO3--}
\eqmcst_{\mbox{\scriptsize \ref{rxn:HSO3-_H++SO3--}}} & = & \frac{[\Hp] [\SOtdm] }{
[\HSOtm] } % SeP97 p. 346 (6.35)
\end{eqnarray}
Note that (\ref{eqn:eqm_SO2_SO2H2O}) is the definition of the Henry's
Law coefficient for \SOd\ (\ref{eqn:hnr_law_dfn}).
The measured values of $\eqmcst_{\mbox{\scriptsize \ref{rxn:SO2H2O_H++HSO3-}}}$ and
$\eqmcst_{\mbox{\scriptsize \ref{rxn:HSO3-_H++SO3--}}}$ at 298\,K\ are given in
Table~\ref{tbl:aqs_eqm}. 

The concentrations of \SOdHdO, \HSOtm, and \SOtdm\ may be expressed in
terms of (\ref{eqn:eqm_SO2_SO2H2O})--(\ref{eqn:eqm_HSO3-_H++SO3--}).
Rearranging (\ref{eqn:eqm_SO2_SO2H2O}) leads to
\begin{eqnarray}
\label{eqn:cnc_SO2H2O}
[\SOdHdO] & = & \hnrSOd \prsprtSOd % SeP97 p. 346 (6.36)  
\end{eqnarray}
Again, this is simply a statement of Henry's Law equilibrium
(\ref{eqn:hnr_law_dfn}). 
Note that [\SOdHdO] depends on temperature through $\hnrSOd$
(\ref{eqn:hnr_tpt_dfn}). 
Making use of (\ref{eqn:cnc_SO2H2O}) in (\ref{eqn:eqm_SO2H2O_H++HSO3-})
\begin{eqnarray}
\label{eqn:cnc_HSO3m}
[\HSOtm] & = & \frac{\eqmcst_{\mbox{\scriptsize \ref{rxn:SO2H2O_H++HSO3-}}} [\SOdHdO]}{[\Hp] } \nonumber \\
& = & \frac{\prsprtSOd \hnrSOd \eqmcst_{\mbox{\scriptsize \ref{rxn:SO2H2O_H++HSO3-}}}}{[\Hp] } % SeP97 p. 346 (6.37)  
\end{eqnarray}
In addition to temperature, [\HSOtm] also depends on \pH\@.
Using (\ref{eqn:cnc_HSO3m}) in (\ref{eqn:eqm_HSO3-_H++SO3--}) 
\begin{eqnarray}
\label{eqn:cnc_SO3dm}
[\SOtdm] & = & \frac{\eqmcst_{\mbox{\scriptsize \ref{rxn:HSO3-_H++SO3--}}} [\HSOtm]}{[\Hp] } \nonumber \\ % SeP97 p. 346 (6.38)
& = & \frac{\prsprtSOd \hnrSOd \eqmcst_{\mbox{\scriptsize \ref{rxn:SO2H2O_H++HSO3-}}} \eqmcst_{\mbox{\scriptsize \ref{rxn:HSO3-_H++SO3--}}}}{[\Hp]^{2} } % SeP97 p. 346 (6.38)
\end{eqnarray}
Since \pH\ increases as [\Hp] decreases (\ref{eqn:pH_dfn}), [\SOtdm]
increases rapidly with increasing \pH\@. 
The equilibrium concentration of \HSOtm\ increases by nearly one order
of magnitude per unit \pH\ for $1 < \pH < 8$, while [\SOtdm] increases
nearly twice as fast!
The strong \pH\ sensitivity of [\HSOtm] and [\SOtdm] is responsible for
enabling chemical mechanisms in certain \pH\ regimes, and disabling them
in others. 
Many sure features of cloud droplet chemistry are illustrated below in
\S\ref{sxn:chm_aqs_S}.

We define the total dissolved \SOd, \SOdttl, as the sum of the three
reservoirs 
\begin{equation}
% SeP97 p. 346 (6.23)
[\SOdttl] = [\SOdHdO] + [\HSOtm] + [\SOtdm]
\label{eqn:SO2_ttl_dfn}
\end{equation}
As shown in Table~(\ref{tbl:oxd_S}), [\SOdttl] is the total dissolved
sulfur in oxidation state 4. 
Thus, in accordance with common usage, we shall use [\SIV] instead of
[\SOdttl] in the following.
Then, using (\ref{eqn:cnc_SO2H2O})--(\ref{eqn:cnc_SO3dm}) we find
\begin{eqnarray}
\label{eqn:SO2_ttl_dfn_2}
[\SIV] & = & \prsprtSOd \hnrSOd \left(
1 + \frac{\eqmcst_{\mbox{\scriptsize \ref{rxn:SO2H2O_H++HSO3-}}}}{[\Hp] } + 
\frac{\eqmcst_{\mbox{\scriptsize \ref{rxn:SO2H2O_H++HSO3-}}} \eqmcst_{\mbox{\scriptsize \ref{rxn:HSO3-_H++SO3--}}}}{[\Hp]^{2} }
\right)
\\ % SeP97 p. 346 (6.24)
\label{eqn:hnr_ffc_SO2_dfn}
& \equiv & \prsprtSOd \hnrffcSIV
\end{eqnarray}
where (\ref{eqn:hnr_ffc_SO2_dfn}) defines the \trmidx{effective
Henry's Law coefficient} $\hnrffcSIV$.
The \pH\ sensitivity of [\HSOtm] and [\SOtdm] discussed above causes 
$\hnrffcSIV$ to increase by an order of magnitude per unit \pH\ for 
$1 < \pH < 8$.
For $\pH < 2$, most aqueous \SIV\ in the form of \SOdHdO.
For $2 < \pH < 7$, the typical range for urban regions, aqueous \SIV\ 
in mostly in the form of \HSOtm. 
For $\pH > 7$, aqueous \SIV\ is mostly \SOtdm.

\subsection[Ammonia]{Ammonia \NHt}\label{sxn:aqs_eqm_NH3}
Ammonia is the dominant basic gas in the atmosphere.
The equilibrium Henry's Law constant of \NHt\ is  $62$\,\Mxatm\ at
298\,K\@.  
If clouds with $\pH < 5$ are present, nearly all gas phase ammonia
will dissolve into the cloud droplets.
Thus \NHt\ is able to contribute to neutralizing very acidic cloud
droplets. 

\NHt\ dissolved in pure water will end in one of two forms, \NHtHdO\
or \NHqp
\begin{rxnarray}
\label{rxn:NH3_NH3H2O}
\NHt (g) + \HdO & \eqbm & \NHtHdO \\ % SeP97 p. 353 (6.46)
\label{rxn:NH3H2O_OH-+NH4+}
\NHtHdO & \eqbm & \OHm + \NHqp % SeP97 p. 353 (6.47)
\end{rxnarray}
After dissolving in liquid water (\ref{rxn:NH3_NH3H2O}), ammonia
may dissociate into ammonium ions and hydroxyl ions
(\ref{rxn:NH3H2O_OH-+NH4+}). 
Note that the notation \NHqOH\ is often used instead of \NHtHdO.
The equilibrium constants of (\ref{rxn:NH3_NH3H2O}) and
(\ref{rxn:NH3H2O_OH-+NH4+}) are, respectively
\begin{eqnarray}
\label{eqn:eqm_NH3_NH3H2O}
\eqmcst_{\mbox{\scriptsize \ref{rxn:NH3_NH3H2O}}} & = & \frac{[\NHtHdO]}{\prsprtNHt } \\ % SeP97 p. 353 (6.48)
& \equiv & \hnrNHt \nonumber \\
\label{eqn:eqm_NH3H2O_OH-+NH4+}
\eqmcst_{\mbox{\scriptsize \ref{rxn:NH3H2O_OH-+NH4+}}} & = & \frac{[\OHm] [\NHqp]}{[\NHtHdO] } % SeP97 p. 353 (6.49)
\end{eqnarray}

The aqueous phase concentrations of \NHtHdO\ and \NHqp\ may be
expressed in terms of
(\ref{eqn:eqm_NH3_NH3H2O})--(\ref{eqn:eqm_NH3H2O_OH-+NH4+}). 
Rearranging (\ref{eqn:eqm_NH3_NH3H2O}) leads to
\begin{eqnarray}
\label{eqn:cnc_NH3H2O}
[\NHtHdO] & = & \hnrNHt \prsprtNHt % SeP97 p. 353 
\end{eqnarray}
which is simply a statement of Henry's Law equilibrium
(\ref{eqn:hnr_law_dfn}). 
Note that [\NHtHdO] depends on temperature through $\hnrNHt$ 
(\ref{eqn:hnr_tpt_dfn}). 
Making use of (\ref{eqn:cnc_NH3H2O}) in (\ref{eqn:eqm_NH3H2O_OH-+NH4+})
\begin{eqnarray}
\label{eqn:cnc_NH4+}
[\NHqp] & = & \frac{\eqmcst_{\mbox{\scriptsize \ref{rxn:NH3H2O_OH-+NH4+}}} [\NHtHdO]}{[\OHm] } \nonumber \\  % SeP97 p. 353 (6.50)  
& = & \frac{\prsprtNHt \hnrNHt \eqmcst_{\mbox{\scriptsize \ref{rxn:NH3H2O_OH-+NH4+}}}}{[\OHm] } \nonumber \\
& = & \frac{\prsprtNHt \hnrNHt \eqmcst_{\mbox{\scriptsize \ref{rxn:NH3H2O_OH-+NH4+}}} [\Hp]}{\eqmcstHdO } % SeP97 p. 353 (6.50)  
\end{eqnarray}
It is traditional to describe concentrations in terms of [\Hp], so 
we substituted (\ref{eqn:cnc_OH-}) for [\OHm] in the last line.
Since [\Hp] decreases as \pH\ increases (\ref{eqn:pH_dfn}), [\NHqp]  
also decreases with increasing \pH\@.
Thus ammonia tries to neutralize acidic solutions.
This is in contrast to the acidic species we have studied until now.

The total dissolved \NHt\ is the sum of the two reservoirs    
\newline\parbox{6in}{ % KoD95 p. 138
\begin{eqnarray*}
% SeP97 p. 353 (6.51)
[\NHtttl] & = & [\NHqttl] = [\NHtHdO] + [\NHqp]
% 1 & = & 2
% [\NHqttl] & = & [\NHqOH] + [\NHqp] % 19990307: fxm Including this line trips a bug in LaTeX
\end{eqnarray*}
}\hfill % end parbox KoD95 p. 138
\parbox{1cm}{\begin{eqnarray}\label{eqn:NH3_ttl_dfn}\end{eqnarray}}\newline
Thus, $\NHtttl$ and $\NHqttl$ are equivalent, and, following
tradition, we prefer the former.
Then, using (\ref{eqn:cnc_NH3H2O})--(\ref{eqn:cnc_NH4+}) we find
\begin{eqnarray}
\label{eqn:NH3_ttl_dfn_2}
[\NHtttl] & = & \prsprtNHt \hnrNHt \left(
1 + \frac{\eqmcst_{\mbox{\scriptsize \ref{rxn:NH3H2O_OH-+NH4+}}} [\Hp]}{\eqmcstHdO }
\right)
\\ % SeP97 p. 353 (6.59)
\label{eqn:hnr_ffc_NH3_dfn}
& \equiv & \prsprtNHt \hnrffcNHt
\end{eqnarray}
where $\hnrffcNHt$ is the effective Henry's Law coefficient for
hydrogen peroxide.
The fraction of \NHtttl\ which is \NHqp\ is (\ref{eqn:cnc_NH4+}) divided
by (\ref{eqn:NH3_ttl_dfn_2})
\begin{eqnarray}
% SeP97 p. 353 (6.56)
\frac{[\NHqp]}{[\NHtttl] } & = & 
\frac{\prsprtNHt \hnrNHt \eqmcst_{\mbox{\scriptsize \ref{rxn:NH3H2O_OH-+NH4+}}} [\Hp] }{
\eqmcstHdO } 
\times
\frac{1}{\prsprtNHt \hnrNHt } \times
\frac{\eqmcstHdO}{\eqmcstHdO + \eqmcst_{\mbox{\scriptsize \ref{rxn:NH3H2O_OH-+NH4+}}} [\Hp] }
\nonumber \\
& = & 
\frac{\eqmcst_{\mbox{\scriptsize \ref{rxn:NH3H2O_OH-+NH4+}}} [\Hp] }{
\eqmcstHdO + \eqmcst_{\mbox{\scriptsize \ref{rxn:NH3H2O_OH-+NH4+}}} [\Hp] }
\end{eqnarray}
As shown in Table~\ref{tbl:aqs_eqm}, 
$\eqmcst_{\mbox{\scriptsize \ref{rxn:NH3H2O_OH-+NH4+}}} = 1.7 \times 10^{-5}$~M at 298\,K\@.
Thus $\eqmcst_{\mbox{\scriptsize \ref{rxn:NH3H2O_OH-+NH4+}}} \times [\Hp] \gg \eqmcstHdO$
for $\pH < 8$.
Thus [\NHtttl] is dominated by [\NHqp] in most clouds.

\subsection[Nitric Acid]{Nitric Acid \HNOt}\label{sxn:aqs_eqm_HNO3}
Aqueous phase chemistry is an important pathway for the oxidation of
odd nitrogen into nitrates.
The aqueous pathway is very efficient because the large Henry's law
constants for nitric acid and for nitrate exceed those of all other
major atmospheric species.
As shown in Table~\ref{tbl:hnr}, 
$\hnrHNOt = \hnrNOt = 2.1 \times 10^{5}$\,\Mxatm.
in the presence of clouds.
This high solubility means that most gas phase \HNOt\ and \NOt\ is
dissolved in clouds when clouds are present. 

Dissolved \HNOt\ in pure water takes one of two paths
\begin{rxnarray}
\label{rxn:HNO3_HNO3H2O}
\HNOt (g) + \HdO & \eqbm & \HNOtHdO \\ % SeP97 p. 355 (6.53)
\label{rxn:HNO3H2O_H++NO3-}
\HNOtHdO & \eqbm & \Hp + \NOtm % SeP97 p. 355 (6.54)
\end{rxnarray}
After dissolving in liquid water (\ref{rxn:HNO3_HNO3H2O}), nitric acid
may dissociate into nitrate and hydrogen ions (\ref{rxn:HNO3H2O_H++NO3-}).
The equilibrium constants of (\ref{rxn:HNO3_HNO3H2O}) and
(\ref{rxn:HNO3H2O_H++NO3-}) are, respectively
\begin{eqnarray}
\label{eqn:eqm_HNO3_HNO3H2O}
\eqmcst_{\mbox{\scriptsize \ref{rxn:HNO3_HNO3H2O}}} & = & \frac{[\HNOtHdO]}{\prsprtHNOt } \\ % SeP97 p. 355 (6.57)  
& \equiv & \hnrHNOt \nonumber \\
\label{eqn:eqm_HNO3H2O_H++NO3-}
\eqmcst_{\mbox{\scriptsize \ref{rxn:HNO3H2O_H++NO3-}}} & = & \frac{[\Hp] [\NOtm]}{[\HNOtHdO] } % SeP97 p. 355 (6.55)
\end{eqnarray}

The aqueous phase concentrations of \HNOtHdO\ and \NOtm\ may be
expressed in terms of
(\ref{eqn:eqm_HNO3_HNO3H2O})--(\ref{eqn:eqm_HNO3H2O_H++NO3-}). 
Rearranging (\ref{eqn:eqm_HNO3_HNO3H2O}) leads to
\begin{eqnarray}
\label{eqn:cnc_HNO3H2O}
[\HNOtHdO] & = & \hnrHNOt \prsprtHNOt % SeP97 p. 355 (6.57)  
\end{eqnarray}
which is simply a statement of Henry's Law equilibrium
(\ref{eqn:hnr_law_dfn}). 
Note that [\HNOtHdO] depends on temperature through $\hnrHNOt$ 
(\ref{eqn:hnr_tpt_dfn}). 
Making use of (\ref{eqn:cnc_HNO3H2O}) in (\ref{eqn:eqm_HNO3H2O_H++NO3-})
\begin{eqnarray}
\label{eqn:cnc_NO3m}
[\NOtm] & = & \frac{\eqmcst_{\mbox{\scriptsize \ref{rxn:HNO3H2O_H++NO3-}}} [\HNOtHdO]}{[\Hp] } \nonumber \\
& = & \frac{\prsprtHNOt \hnrHNOt \eqmcst_{\mbox{\scriptsize \ref{rxn:HNO3H2O_H++NO3-}}}}{[\Hp] } % SeP97 p. 355 (6.58)  
\end{eqnarray}
Thus [\NOtm] depends upon \pH\ as well as temperature.

We define the total dissolved \HNOt, \HNOtttl, as the sum of the two
reservoirs  
\begin{equation}
% SeP97 p. 355 (6.56)
[\HNOtttl] = [\HNOtHdO] + [\NOtm]
\label{eqn:HNO3_ttl_dfn}
\end{equation}
Then, using (\ref{eqn:cnc_HNO3H2O})--(\ref{eqn:cnc_NO3m}) we find
\begin{eqnarray}
\label{eqn:HNO3_ttl_dfn_2}
[\HNOtttl] & = & \prsprtHNOt \hnrHNOt \left(
1 + \frac{\eqmcst_{\mbox{\scriptsize \ref{rxn:HNO3H2O_H++NO3-}}}}{[\Hp] }
\right)
\\ % SeP97 p. 355 (6.59)
\label{eqn:hnr_ffc_HNO3_dfn}
& \equiv & \prsprtHNOt \hnrffcHNOt
\end{eqnarray}
where $\hnrffcHNOt$ is the effective Henry's Law coefficient for
nitric acid.

Although \HNOtHdO\ and \NOtm\ are both possible reservoirs for
dissolved \HNOt, the extreme solubility of \HNOt\ causes virtually all 
\HNOt\ dissolved in cloud water forms \NOtm, not \HNOtHdO.
To see what precludes \HNOtHdO\ from being an important reservoir, we
must examine the equilibrium coefficients in
(\ref{eqn:HNO3_ttl_dfn_2}). 
As shown in Table~\ref{tbl:aqs_eqm}, 
$\eqmcst_{\mbox{\scriptsize \ref{rxn:HNO3H2O_H++NO3-}}} = 15.4$~M at 298\,K\@.
Since $\eqmcst_{\mbox{\scriptsize \ref{rxn:HNO3H2O_H++NO3-}}} \gg 1 \gg [\Hp]$, the RHS of
(\ref{eqn:HNO3_ttl_dfn_2}) is always dominated by the concentration of
\NOtm, even in the most acidic cloud droplets. 
Thus we may approximate $\hnrffcHNOt$ (\ref{eqn:hnr_ffc_HNO3_dfn}) as 
\begin{eqnarray}
\label{eqn:HNO3_ttl_dfn_3}
\hnrffcHNOt & \approx & \frac{\hnrHNOt \eqmcst_{\mbox{\scriptsize \ref{rxn:HNO3H2O_H++NO3-}}}
}{ [\Hp] } \\ % SeP97 p. 356 (6.61)
& = & \frac{3.23 \times 10^{6} \mbox{\MSxatm}}{[\Hp] } \quad
\mbox{~at 298\,K} \nonumber
\end{eqnarray}

\subsection[Hydrogen Peroxide]{Hydrogen Peroxide \HdOd}\label{sxn:aqs_eqm_H2O2}
Hydrogen peroxide is a very soluble atmospheric gas, and an important
ingredient in aqueous phase reactions involving both sulfur and
nitrogen. 
The equilibrium Henry's Law constant of \HdOd\ is 
$7.45 \times 10^{4}$\,\Mxatm\ at 298\,\K\@. 
Thus \HdOd\ can be found in relatively large concentrations in cloud
water. 
Moreover, \cite{AFA94} describe a photochemical production mechanism
for \HdOd\ in the aqueous phase.
Thus \HdOd\ is likely to participate in most sulfur and nitrogen
reduction. 

Once \HdOd\ dissolves in pure water, it takes one of two paths
\begin{rxnarray}
\label{rxn:H2O2_H2O2H2O}
\HdOd (g) + \HdO & \eqbm & \HdOdHdO \\
\label{rxn:H2O2H2O_H++HO2-}
\HdOdHdO & \eqbm & \Hp + \HOdm % SeP97 p. 356 (6.62)
\end{rxnarray}
After dissolving in liquid water (\ref{rxn:H2O2_H2O2H2O}), hydrogen peroxide
may dissociate into hydroperoxyl and hydrogen ions (\ref{rxn:H2O2H2O_H++HO2-}).
The equilibrium constants of (\ref{rxn:H2O2_H2O2H2O}) and
(\ref{rxn:H2O2H2O_H++HO2-}) are, respectively
\begin{eqnarray}
\label{eqn:eqm_H2O2_H2O2H2O}
\eqmcst_{\mbox{\scriptsize \ref{rxn:H2O2_H2O2H2O}}} & = & \frac{[\HdOdHdO]}{\prsprtHdOd } \\ % SeP97 p. 355 (6.57)  
& \equiv & \hnrHdOd \nonumber \\
\label{eqn:eqm_H2O2H2O_H++HO2-}
\eqmcst_{\mbox{\scriptsize \ref{rxn:H2O2H2O_H++HO2-}}} & = & \frac{[\Hp] [\HOdm]}{[\HdOdHdO] } % SeP97 p. 355 (6.55)
\end{eqnarray}

The aqueous phase concentrations of \HdOdHdO\ and \HOdm\ may be
expressed in terms of
(\ref{eqn:eqm_H2O2_H2O2H2O})--(\ref{eqn:eqm_H2O2H2O_H++HO2-}). 
Rearranging (\ref{eqn:eqm_H2O2_H2O2H2O}) leads to
\begin{eqnarray}
\label{eqn:cnc_H2O2H2O}
[\HdOdHdO] & = & \hnrHdOd \prsprtHdOd % SeP97 p. 355 (6.57)  
\end{eqnarray}
which is simply a statement of Henry's Law equilibrium
(\ref{eqn:hnr_law_dfn}). 
Note that [\HdOdHdO] depends on temperature through $\hnrHdOd$ 
(\ref{eqn:hnr_tpt_dfn}). 
Making use of (\ref{eqn:cnc_H2O2H2O}) in (\ref{eqn:eqm_H2O2H2O_H++HO2-})
\begin{eqnarray}
\label{eqn:cnc_HO2-}
[\HOdm] & = & \frac{\eqmcst_{\mbox{\scriptsize \ref{rxn:H2O2H2O_H++HO2-}}} [\HdOdHdO]}{[\Hp] } \nonumber \\
& = & \frac{\prsprtHdOd \hnrHdOd \eqmcst_{\mbox{\scriptsize \ref{rxn:H2O2H2O_H++HO2-}}}}{[\Hp] } % SeP97 p. 355 (6.58)  
\end{eqnarray}
Thus [\HOdm] depends upon \pH\ as well as temperature.

We define the total dissolved \HdOd, \HdOdttl, as the sum of the two
reservoirs  
\begin{equation}
% SeP97 p. 355 (6.56)
[\HdOdttl] = [\HdOdHdO] + [\HOdm]
\label{eqn:H2O2_ttl_dfn}
\end{equation}
Then, using (\ref{eqn:cnc_H2O2H2O})--(\ref{eqn:cnc_HO2-}) we find
\begin{eqnarray}
\label{eqn:H2O2_ttl_dfn_2}
[\HdOdttl] & = & \prsprtHdOd \hnrHdOd \left(
1 + \frac{\eqmcst_{\mbox{\scriptsize \ref{rxn:H2O2H2O_H++HO2-}}}}{[\Hp] }
\right)
\\ % SeP97 p. 355 (6.59)
\label{eqn:hnr_ffc_H2O2_dfn}
& \equiv & \prsprtHdOd \hnrffcHdOd
\end{eqnarray}
where $\hnrffcHdOd$ is the effective Henry's Law coefficient for
hydrogen peroxide.

Although \HdOdHdO\ and \HOdm\ are both possible reservoirs for
dissolved \HdOd, \HdOdHdO\ is a very weak electrolyte (i.e., does 
not readily ionize in cloud water to form \HOdm).
To see what precludes \HOdm\ from being an important reservoir, we
must examine the equilibrium coefficients in
(\ref{eqn:H2O2_ttl_dfn_2}). 
As shown in Table~\ref{tbl:aqs_eqm}, 
$\eqmcst_{\mbox{\scriptsize \ref{rxn:H2O2H2O_H++HO2-}}} = 2.2 \times 10^{-12}$~M at 298\,\K\@.
Thus $\eqmcst_{\mbox{\scriptsize \ref{rxn:H2O2H2O_H++HO2-}}} / [\Hp] < 10^{-4}$ for 
$\pH < 7.5$.
Thus (\ref{eqn:H2O2_ttl_dfn_2}) is dominated by the concentration of 
\HdOdHdO, even in very basic cloud droplets. 

\section[Aqueous Sulfur Chemistry]{Aqueous Sulfur Chemistry}\label{sxn:chm_aqs_S}
Uptake and subsequent processing of atmospheric \SOd\ by aerosol
and clouds is one of the most important, and most studied aspects of
atmospheric aqueous chemistry.
The solubility of \SIV\ is strongly \pH-dependent, as the effective
Henry's Law constant $\hnrffcSIV$ increases by approximately one order
of magnitude per unit \pH\ for $1 < \pH < 8$ (\S\ref{sxn:chm_aqs_S}).
As the uptake of \SIV\ to droplets depends on \pH, so is the relative
importance of aqueous phase oxidation of \SIV.

In clouds or aerosols, \SOd\ may be oxidized by ozone 
\begin{reaction}
\SIV + \Ot \yields \SVI + \Od % SeP97 p. 363 (6.79)
\end{reaction}
Such reactions are fast in the aqueous phase, but slow enough to be
neglected in the gas phase \cite[][p. 363]{SeP97}.
For example,
\begin{reaction}
\SOtdm + \Ot \yields \SOqdm + \Od % DCZ96 p. 22872 (9)
\end{reaction}

Hydrogen peroxide may form in cloud water \cite{AFA94}.
Hydrogen peroxide oxidizes \SOd\ via
\begin{reaction}
\HSOtmaq + \HdOdaq \yields \HSOqmaq + \HdO % DCZ96 p. 22872 (12)
\label{rxn:HSO3-+H2O2_HSO4-+H2O}
\end{reaction}
Reaction (\ref{rxn:HSO3-+H2O2_HSO4-+H2O}) requires sufficient aqueous
volume to be of importance and may, to a first approximation, be
neglected on mineral aerosol \cite[][]{DCZ96}. 

\section[Aqueous Iron Chemistry]{Aqueous Iron Chemistry}\label{sxn:chm_aqs_Fe}
In this section we discuss the catalysis of \SIV\ oxidation by
\FeIII\ and \MnII. 
We shall assume this chemistry takes place in the liquid phase.
The basic mechanism is known to be
\begin{rxnarray}
\label{rxn:SIV+O2_SVI} 
\HSOtm + \HdOd & \yields & \SOqdm + 2\Hp + \HdO % BRK00 p. 6 (7)
\end{rxnarray}
This reaction mechanism was modeled by \cite{PaS89} and \cite{PSP92}.  

\section[Aqueous Nitrogen Chemistry]{Aqueous Nitrogen Chemistry}\label{sxn:chm_aqs_N}
Nitrates may be formed through heterogeneous chemistry in the
atmosphere and are biologically useful.
Thus we shall now concentrate on determining the effects of aerosol
chemistry on nitrate production.

\section{Mass Transport and Diffusion}
The uptake of gaseous species onto liquid or solid aerosols is subject
to the constraints of \trmdfn{mass transport}.
The equilibrium predicted by Henry's law, for example, will only be
reached if there are no barriers to solute transport from the gaseous
to the aqueous phase. 
Such barriers, or, more properly, resistances to mass transport might
be due to interfacial resistance, aerodynamic resistance, the rate
of diffusion, etc.
For more in depth discussion of these processes see, e.g.,
\cite{Sch86,LeC91,SeP97,JPL97}. 

\chapter{Aerosol Physics}\label{sxn:aer}
Aerosol physics encompass the processes that nucleate, grow, maintain,
and destroy atmospheric particulates.
Before describing these processes quantitatively, we first define 
a set of standard terminology to use.
Most definitions are taken verbatim from \cite{Val85}, and some are
followed with supplementary descriptions.
\trmdfn{aerosol particle}: Solid or liquid particle mostly
consisting of some substance(s) other than water, and without the
stable bulk liquid or solid phases of water on it.
\trmdfn{haze particle}: Partly or wholly water-soluble aerosol
particle at humidities exceeding that necessary for deliquescence and
in stable equilibrium with respect to changes in humidity.
\trmdfn{cloud droplet}: Bulk (thermodynamically stable) liquid-water
droplet, usually formed on a cloud-condensation nucleus, and growing
to a size (mass) determined by the available supply of vapor.
\trmdfn{ice crystal}: Bulk (thermodynamically stable) solid water,
initially formed by the freezing of a cloud droplet or by deposition
nucleation, and growing to a size (mass) determined by the available
supply of water vapor.
\trmdfn{Condensation nuclei} (CN): Those particles that will grow to
visible size when any liquid condenses on them at supersaturations
just below that necessary to activate small ions.
\trmdfn{Aitken nuclei} (AN): Those particles that grow to visible size
when water condenses on them at supersaturations just below those that
will activate small ions.

The definitions of CN and AN refer to methods of detection and
counting of aerosol.
Instruments make the aerosol ``visible'' by condensing some liquid
(to count CN) or water specifically (to count AN).

\trmdfn{Cloud condensation nuclei} (CCN): Particles that can serve as
nuclei of atmospheric cloud droplets, i.e, particles on which liquid
water condenses at supersaturations typical of atmospheric cloud
formation (fraction of a percent to a few percent, depending on cloud
type). 
Concentrations of CCN need to be given in terms of a supersaturation
spectrum covering the range of interest or at some specified
supersaturation value. 

\trmdfn{Primary processes (mechanisms) of ice formation}: Nucleation
of ice from the liquid or vapor phases of water, either homogeneously
or heterogeneously. 
The purpose of this terminology is to emphasize the distinction from
secondary processes.

\trmdfn{Secondary processes (mechanisms) of ice formation}: Creation
of ice particles by processes that require prior existence of some
other macroscopic (thermodynamically stable) ice particles; examples
are splintering of freezing drops, crystal fragmentation, etc.

\trmdfn{Ice nuclei (IN)}: Generic name for substances, usually in the
form of aerosol particles, which under suitable conditions of
supersaturation or supercoooling nucleate ice.
This term is to be used if no distinction is intended or is possible
as to the specific nucleation mechanism involved.

\trmdfn{Ice nucleation modes}: Names or short phrases are needed to
refer to particular nucleation modes.
Since the details of the molecular embryo formation are difficult to
ascertain, it is of great importance that references to nucleation
modes be made with the utmost care for clarity of usage, relying, if
needed, on more complete descriptions of the phenomenon in question.
In terms of ice nucleation, the basic distinction that has to be made
is whether nucleation is from the vapor or from the liquid phase.
There may be some ambiguity when the metastable phase is the vapor,
since the formation of liquid or liquid-like layers might actually be
involved in the creation of ice embryos, but unless there is definite
evidence for this, the pragmatic view is to consider the bulk vapor as
the initial phase.
The following definitions aid in making these distinctions.

\trmdfn{Deposition nucleation}: The formation of ice in a
(supersaturated) vapor environment. 

\trmdfn{Freezing nucleation}: The formation of ice in a (supercooled) 
liquid environment.
\begin{enumerate}
\item \trmdfn{Condensation freezing}: The sequence of events whereby a
cloud condensation nucleus (\trmidx{CCN}) initiates freezing of the
condensate.
\item \trmdfn{Contact freezing}: Nucleation of a supercooled droplet
subsequent to an aerosol particle's coming into contact with it.
(This name is preferable to \trmdfn{contact nucleation} as it focuses
attention on the fact that the basic process is freezing.)
\item \trmdfn{Immersion freezing}: Nucleation of supercooled water by
a nucleus suspended in the body of water.
\end{enumerate}

Instrumentation for aerosol measurements:
\begin{enumerate}
\item \trmdfn{Aethalometer}
\item \trmdfn{Nephelometer}
\item \trmdfn{Spectroradiometer}
\item \trmdfn{Pyranometer}
\item \trmdfn{Condensation Particle Counter} (\trmidx{CPC})
\item \trmdfn{Cloud Condensation Nuclei Counter} (\trmidx{CCNC})
\item \trmdfn{Optical Particle Counter} (\trmidx{OPC})
\item \trmdfn{Cloud Droplet Probe} (\trmidx{CDP})
\end{enumerate}

\chapter{Chemistry and Mineral Dust}\label{sxn:chm_htr_dst}
Mineral dust facilitates both heterogeneous (surface) chemistry and 
aqueous phase chemistry.
One of the greatest challenges in assessing the influence of mineral
dust on atmospheric chemistry is in constraining which mechanisms
operate in a given meteorological environment (e.g., relative
humidity), because the mineralogical composition of dust is very
uncertain. 

It is known that sulfate and nitrate concentrations vary with mineral 
Perhaps the greatest uncertainty is whether or not the mineral dust
aerosol is \trmdfn{wetted}. 
A wetted particle is simply one on which liquid phase solution
exists. 
A slightly wetted particle may have liquid water in only some facial
crevices or nooks.
A completely wetted particle is covered in solution.
Due to the great size of mineral particles, once a particle is wetted
it may act as a giant CCN, immediately growing into a cloud droplet
(or at least competing for the vapor to do so).  
In this case the hygroscopic growth may significantly enhance the
cross-sectional area and the volume of the aerosol.
Of course the dry volume of the dust particle does not change, but the
water coating may significantly alter the particles aerodynamic
properties (e.g., fall speed) and optical properties.
Thus, there is a complete spectrum of plausible aqueous states that a
mineral dust particle may occupy, ranging from completely dry to an
activated cloud droplet.
Table~\ref{tbl:dlq} lists the deliquescence relative humidities of
electrolyte solutions at 298\,\K\@.
\begin{table}
\begin{minipage}{\hsize} % Minipage necessary for footnotes KoD95 p. 110 (4.10.4)
\renewcommand{\footnoterule}{\rule{\hsize}{0.0cm}\vspace{-0.0cm}} % KoD95 p. 111
\begin{center}
\caption[Deliquescence Relative Humiditiy]{\textbf{Deliquescence 
Relative Humidity of Electrolyte Solutions at 298\,\K}% 
\footnote{\emph{Sources:} \emph{Source:} Adapted from \cite{SeP97},
p.~508.}% 
\label{tbl:dlq}}   
\vspace{\cpthdrhlnskp}
\begin{tabular}{ r l >{$}r<{$} }
\hline \rule{0.0ex}{\hlntblhdrskp}% 
Formula & Name & \mbox{RHD} \\[0.0ex]
& & \% \\[0.0ex]
\hline \rule{0.0ex}{\hlntblntrskp}%
% SeP97 p. 508
\KCl & Potassium chloride & 84.2 \pm 0.3 \\[0.5ex]
\NadSOq & Sodium sulfate & 84.2 \pm 0.4 \\[0.5ex]
\NHqCl & Ammonium chloride & 80.0 \\[0.5ex]
\NHqdSOq & Ammonium sulfate & 79.9 \pm 0.5 \\[0.5ex]
\NaCl & Sodium chloride & 75.3 \pm 0.1 \\[0.5ex]
\NaNOt & Sodium nitrate & 74.3 \pm 0.4 \\[0.5ex]
\NHqtHSOqd & Letovicite & 69.0 \\[0.5ex]
\NHqNOt & Ammonium nitrate & 61.8 \\[0.5ex]
\NaHSOq & Sodium bisulfate & 52.0 \\[0.5ex]
\NHqHSOq & Ammonium bisulfate & 40.0 \\[0.5ex]
\hline
\end{tabular}
\end{center}
\end{minipage}
\end{table}

Needless to say, the reaction surface available for chemistry is
crucial in determining the chemical role of the particle.
Surface reactions may occur on a complete dry particle, but it is
often found that the overall rate of reaction decreases as the
particle ages. 
The older the particle, the more of its surface that has been covered
with reaction products.
Once every available surface site has adsorbed the reactants and
facilitated a reaction, there is no more space for reactions involving
the particle's minerals (e.g., iron).
Aqueous reactions, on the other hand, may occur through the entire
volume of the droplet.
Assuming the reactions are not significantly limited by diffusion,
then the volume suitable for reactions greatly exceeds the equivalent
volume of surface chemistry sites.

However, surface reactions may operate in all weather, whereas aqueous
reactions require specific conditions.  
Thus the importance of surface chemistry relative to aqueous chemistry
depends not only on the reaction mechanism, but upon the synoptic
scale environment.
For example, surface reactions may be more important than aqueous
reactions over arid regions such as deserts.

Thus a complete description of chemistry on mineral dust requires
specififications of parameters for both surface and aqueous
chemistry. 

\section[Role of Carbonate]{Role of Carbonate}\label{sxn:chm_dst_CO3}
Wetted alkaline substances such as mineral dust tend to neutralize
acidic solutions and thus increase solution~pH\@.
The calcium carbonate (\CaCOt) content of soils is important in
determining the soil alkalinity.
\CaCOt\ may neutralize acidity through
\begin{reaction}
\CaCOt + 2\Hp \yields \Cadp + \COd + \HdO \\ % DCZ96 p. 22872 (11)
\end{reaction}
The calcium content of arid soils is typically about 5\% by weight
\cite[]{Pye87,DCZ96}.

The global distributions of \CaCOt, as well as other important crustal 
minerals, must be obtained from \textit{in situ} soil samples.
The IGBP global soil dataset \cite[]{CaS98,Sch99} assembles many
thousand soil profiles, or \trmdfn{pedons}, into one
quality-controlled dataset. 
The SOILDATA program included with the dataset is designed to select
statiscally unbiased samples of soil parameters for any soil vertical
level or \trmdfn{horizon} for a specified geographic region.
The resulting global distribution of \CaCOt\ peaks strongly around the
Arabian Pensinsula and Iran, where \CaCOt\ accounts for up to 15\% of
topsoil (defined as soil within 5\,\cm\ of surface) mass.
The global mean continental \CaCOt\ topsoil content is about~2.5\%.
% Paris: Arimoto talk
Water soluble \Cadp\ is highly correlated ($\crrcff \approx 0.8$) with
\Al\ near source regions in China.
% Paris: Perry talk
In contrast, \Si\ correlates very poorly with~\Al\ in Asian and
African dust. 
Thus \Ca\ is a potentially-useful tracer for total dust.

\section[Hygroscopic Growth]{Hygroscopic Growth}\label{sxn:hyg}
Heterogeneous chemistry on mineral aerosol is very sensitive to the
presence of liquid phase water. 
Uptake coefficients on wetted aerosols are often orders of magnitude
higher than on dry solids (Table~\ref{tbl:mss_upt_cff}).
Due to its highly varying composition, dust, unfortunately, has no
single deliquesence point as do pure salts (Table~\ref{tbl:dlq}).
However, observations of the hygroscopic growth factors on mineral
dust allow us to make some generalizations.

The theory of hygroscopic growth is discussed many texts
\cite[e.g.,][]{Han76,PrK78,RuD84,SeP97,WKB98}.
Five sophisticated thermodynamic equilibrium models of inorganic
aerosol growth are intercompared in \cite{ZSS00}.
First, the increase in aerosol mass due the accumulation of liquid
phase water is called \trmdfn{hygroscopic growth}.

\cite{WKB98} provide parameterizations for the hygroscopic growth of
\HdSOq\ and \NHqdSOq\ as a function of \RH\ alone.
Consider a size distribution $\dstfnc (\dmt)$ of dry aerosol with
number-median diameter $\dmtnma$ and geometric standard deviation
$\gsd$. 
Our goal is to predict the corresponding size distribution of the
aerosol in the wetted state, i.e., at any \RH\@.  
Let $\dmtnmawet$ and $\gsdwet$ denote the wet median diameter and
geometric standard deviation, respectively.
\cite{WKB98} showed that expressions of the form 
\begin{eqnarray}
\frac{\dmtnmawet}{\dmtnma } & = & 1.0 +
\exp \left( \aaa_{1} + \frac{\aaa_{2}}{\aaa_{3} + \RH } + \frac{\aaa_{4}}{
\aaa_{5} + \RH } \right) \\
\frac{\gsdwet}{\gsd } & = & 1.0 +
\exp \left( \bbb_{1} + \frac{\bbb_{2}}{\bbb_{3} + \RH } + \frac{\bbb_{4}}{
\bbb_{5} + \RH } \right)
\label{eqn:hyr_grw_WKB98}
\end{eqnarray}
where $\aaa_{\iii}$ and $\bbb_{\iii}$ are fitting coefficients.

\cite{Han76} reviewed theoretical and observational approaches to
determining the hygroscopic growth factor.
He characterized his results in terms of the water activity of a
particle's liquid coating with respect to a plane surface.
This quantity, called the \trmdfn{activity coefficient} $\actcff$ for
short, is an empirical parameter which defines the thermodynamic
properties of the solution relative to the thermodynamic properties of
pure water. 
It may be expressed directly in terms of another measureable
parameter, the \trmdfn{van't Hoff factor} $\vnthff$ of the solute as 
\cite[e.g.,][]{Han76,PrK78}
\begin{equation}
% Han76 p. 80 (2.6a)
\actcff = \frac{\molnbrwtr}{\molnbrwtr + \vnthff \molnbrslt }
\label{eqn:act_cff_dfn}
\end{equation}
where $\molnbrwtr$ is the mole number of water, and $\molnbrslt$ is
the mole number of the dry solute.
Table~\ref{tbl:act} contains parameterizations for the activity
coefficient of many important atmospheric aerosol species. 
\begin{landscape}
\begin{table}
\begin{minipage}{\hsize} % Minipage necessary for footnotes KoD95 p. 110 (4.10.4)
\renewcommand{\footnoterule}{\rule{\hsize}{0.0cm}\vspace{-0.0cm}} % KoD95 p. 111
\begin{center}
\caption[Activity Coefficients]{\textbf{Activity Coefficients}% 
\footnote{\emph{Sources:} \cite{TaM94}, p.~18805, Table~1.
Coefficients $\cst_{\iii}$ for use in the parameterization 
$\actcffwtr = 1.0 + \sum_{\iii=1}^{\iii=4} \cst_{\iii} \xxx^{\iii}$
where $\actcffwtr$ is water activity and $\xxx$ is the solute weight
as a percent of the total solution weight. 
Coefficients $\AAA_{\iii}$ for use in the parameterization 
$\dnssln = 0.9971 + \sum_{\iii=1}^{\iii=3} \AAA_{\iii} \xxx^{\iii}$
where $\dnssln$ is the density of the solution.
}%
\footnote{Read $-2.715e-3$ as $-2.715 \times 10^{-3}$}%
\label{tbl:act}}   
\vspace{\cpthdrhlnskp}
\begin{tabular}{ *{9}{>{$}r<{$}} } % KoD95 p. 94 describes '*' notation
\hline \rule{0.0ex}{\hlntblhdrskp}% 
& \NHqdSOq & \NHqHSOq & \NHqtHSOqd & \multicolumn{3}{c}{\NadSOq} & \NaHSOq & \NaNOt \\[0.0ex]
\hline \rule{0.0ex}{\hlntblntrskp}%
\xxx, \% & 0\mbox{--}78 & 0\mbox{--}97 & 0\mbox{--}78 & 0\mbox{--}40 & & 
\mbox{\footnote{For this concentration range only, $\actcffwtr = 1.557 +
\sum \cst_{\iii} \xxx^{\iii}$}}%
40\mbox{--}67 & 0\mbox{--}95 & 0\mbox{--}98 \\[0.5ex]
\cst_{1} & -2.715e-3 & -3.05e-3 & -2.42e-3 & -3.55e-3 & & -1.99e-2 & -4.98e-3 & -5.52e-3 \\[0.5ex]
\cst_{2} & 3.113e-5 & -2.94e-5 & -4.615e-5 & 9.63e-5 & & -1.92e-5 & 3.77e-6 & 1.286e-4 \\[0.5ex]
\cst_{3} & -2.336e-6 & -4.43e-7 & -2.83e-7 & -2.97e-6 & & 1.47e-6 & -6.32e-7 & -3.496e-6 \\[0.5ex]
\cst_{4} & 1.412e-8 & 
\mbox{\footnote{Ellipsis indicates coefficient is zero (?)}}% fxm: verify this
\ldots & \ldots & \ldots & & \ldots & \ldots & 1.843e-8 \\[0.5ex]
\AAA_{1} & 5.92e-3 & 5.87e-3 & 5.66e-3 & & 8.871e-3 & & 7.56e-3 & 6.512e-3 \\[0.5ex]
\AAA_{2} & -5.036e-6 & -1.89e-6 & 2.96e-6 & & 3.195e-5 & & 2.36e-5 & 3.025e-5 \\[0.5ex]
\AAA_{3} & 1.024e-8 & 1.763e-7 & 6.68e-8 & & 2.28e-7 & & 2.33e-7 & 1.437e-7 \\[0.5ex]
\stddvn(\actcffwtr) & 2.76e-3 & 7.94e-3 & 5.97e-3 & 3.87e-3 & &  3.39e-3 & 3.13e-3 & 6.65e-3 \\[0.5ex] 
\stddvn(\dns) & 8.98e-5 & 4.03e-4 & 2.13e-3 & & 1.71e-3 & & 7.55e-4 & 3.83e-4 \\[0.5ex]
\hline
\end{tabular}
\end{center}
\end{minipage}
\end{table}
\end{landscape}

The relations between the 
total mass of the aerosol $\mssaerttl$, 
the liquid water mass of the aerosol $\mssaerwtr$,
the dry mass of the soluble component $\mssaerslt$,
the dry mass of insoluble component $\mssaernsl$,
and the total dry mass of the aerosol $\mssaerdry$ are
\begin{eqnarray}
\mssaerdry & = & \mssaerslt + \mssaernsl \\
\mssaerttl & = & \mssaerdry + \mssaerwtr
\label{eqn:mss_aer_ttl_dfn}
\end{eqnarray}

We shall now consider consider the relationship between the ambient
relative humidity \RH\ and aerosol mass.
Recall that the equilibrium vapor pressure of a gas in contact with
the condensed phase of the gas is the saturation vapor pressure of the
gas and depends only on temperature.
However, as shown in \S\ldots, the saturation vapor pressure of a
gas over a curved surface, $\prsprtwtrsatsfc$, depends additionally on
the radius of curvature of the surface.
Thus let $\prsprtwtrsatinf(\tpt)$ and $\prsprtwtrsatsfc(\tpt,\rdsprt)$ 
represent the saturated water vapor pressure far from the aerosol
surface, and at the aerosol surface, respectively.
Parameterizations for the saturated vapor pressure of water over
planar liquid and ice surfaces are described in
Table~\ref{tbl:sat_vap_prs_wtr}.  
\begin{table}
\begin{minipage}{\hsize} % Minipage necessary for footnotes KoD95 p. 110 (4.10.4)
\renewcommand{\footnoterule}{\rule{\hsize}{0.0cm}\vspace{-0.0cm}} % KoD95 p. 111
\begin{center}
\caption[Saturation Vapor Pressure of Water]{\textbf{Saturation Vapor
Pressure of Water Over Planar Surfaces}% 
\footnote{\emph{Sources:} \cite{PrK78} p.~625, \cite{PrK98} p. 854}%
\footnote{Range of validity is $-50 < \tptcls < 50$~C}%
\label{tbl:sat_vap_prs_wtr}}   
\vspace{\cpthdrhlnskp}
\begin{tabular}{>{$}r<{$} >{$}l<{$} >{$}l<{$}}
\hline \rule{0.0ex}{\hlntblhdrskp}% 
\mbox{\footnote{Coefficients for use in
$\prsprtwtrsatinf(\tptcls) =
\cff_{0}+\tptcls(\cff_{1}+\tptcls(\cff_{2}+\tptcls(\cff_{3}+\tptcls(\cff_{4}+\tptcls(\cff_{5}+\cff_{6}\tptcls)))))$,
where $\tptcls$ is the temperature in Celsius and
$\prsprtwtrsatinf(\tpt)$ is the saturated vapor pressure in mb.
(\ref{eqn:klv_law_dfn})}}%
\mbox{Coefficient} & \mbox{Liquid} & \mbox{Ice} \\[0.0ex]
\hline \rule{0.0ex}{\hlntblntrskp}%
% PrK78 p.~625
\cff_{0} & 6.107799961 & 6.109177956 \\[0.5ex]
\cff_{1} & 4.436518521 \times 10^{-1} & 5.034698970 \times 10^{-1} \\[0.5ex]
\cff_{2} & 1.428945805 \times 10^{-2} & 1.886013408 \times 10^{-2} \\[0.5ex]
\cff_{3} & 2.650648471 \times 10^{-4} & 4.176223716 \times 10^{-4} \\[0.5ex]
\cff_{4} & 3.031240396 \times 10^{-6} & 5.824720280 \times 10^{-6} \\[0.5ex]
\cff_{5} & 2.034080948 \times 10^{-8} & 4.838803174 \times 10^{-8} \\[0.5ex]
\cff_{6} & 6.136820929 \times 10^{-11} & 1.838826904 \times 10^{-10} \\[0.5ex]
\hline
% PrK98 p. 854 (only values for ice changed, liquid is same as PrK78)
\cff_{0} & 6.107799961 & 6.10690449 \\[0.5ex]
\cff_{1} & 4.436518521 \times 10^{-1} & 5.02660639 \times 10^{-1} \\[0.5ex]
\cff_{2} & 1.428945805 \times 10^{-2} & 1.87743264 \times 10^{-2} \\[0.5ex]
\cff_{3} & 2.650648471 \times 10^{-4} & 4.13476180 \times 10^{-4} \\[0.5ex]
\cff_{4} & 3.031240396 \times 10^{-6} & 5.72333773 \times 10^{-6} \\[0.5ex]
\cff_{5} & 2.034080948 \times 10^{-8} & 4.71651246 \times 10^{-8} \\[0.5ex]
\cff_{6} & 6.136820929 \times 10^{-11} & 1.78086695 \times 10^{-10} \\[0.5ex]
\hline
\end{tabular}
\end{center}
\end{minipage}
\end{table}
The relationship between $\prsprtwtrsatinf$ and $\prsprtwtrsatsfc$
is called \trmdfn{Kelvin's Law}
\begin{eqnarray}
% Han76 p. 94 (), IrG81 p. 89 (5) 
% fxm: should density be for H2O even though dust has insoluble component?
\prsprtwtrsatsfc & = & \prsprtwtrsatinf \exp \left( \frac{2 \mmwHdO
\sfctnswtrlqd}{\dnswtr \gascstunv \tpt \rdsprt } \right)
\label{eqn:klv_law_dfn}
\end{eqnarray}
The difference between $\RHinf$ and $\RHstr$ exceeds $1\%$ for 
$\rdsprt \lesssim 0.1$\,\um\ and exceeds $10\%$ for $\rdsprt \lesssim
0.01$\,\um. % IrG81 p. 90 Tbl. V-I

The \trmdfn{relative humidity} \RH\ is simply the ratio of the ambient
water vapor pressure, $\prsprtwtr$, to the saturated vapor pressure
for the same temperature 
\begin{eqnarray}
\RH & = & \frac{\prsprtwtr}{\prsprtwtrsatinf(\tpt)} 
\label{eqn:RH_dfn}
\end{eqnarray}
Note that $\prsprt$ is used for partial pressures of most atmospheric
gases, e.g., $\prsprtCOd$, but the letter $\prsprtwtr$ is reserved
specifically for the partial pressure of water vapor, and thus an
$\HdO$ subscript for $\prsprt$ would be redundant.
The relative humidity at the surface of the particle is $\RHsfc$.
$\RHsfc$ differs from $\RHinf$ due to the \trmdfn{Kelvin effect}
(\ref{eqn:klv_law_dfn}), i.e., curvature.
Combining (\ref{eqn:klv_law_dfn}) and (\ref{eqn:RH_dfn}) 
\begin{eqnarray}
% Han76 p. 94 () fxm: should density be for H2O even though dust has insoluble component?
\RHsfc & = & \RH \exp \left( \frac{2 \mmwHdO \sfctnswtrlqd}{\dnswtr
\gascstunv \tpt \rdsprt } \right)
\label{eqn:RH_str_dfn}
\end{eqnarray}

We define the \trmdfn{mean linear mass increase coefficient} $\mlmic$ as
\begin{eqnarray}
% Han76 p. 94 (2.41a)
\mlmic & = & \frac{\mssaerwtr ( 1 - \RHsfc )}{\mssaerdry \RHsfc}
\label{eqn:mlmic_dfn}
\end{eqnarray}
The quantities on the RHS of (\ref{eqn:mlmic_dfn}) are observable, so
$\mlmic$ may be measured in field or laboratory experiments. 
For practical reasons $\mlmic$ is usually inferred from macroscopic
aggregations of aerosol samples where curvature effects may be
ignored.
However, $\mlmic$ itself is independent of size. 
Thus it is valid to apply an $\mlmic$ inferred from measurements of
macroscopic, bulk aerosol samples to individual aerosol particles
\cite[]{Han76}.

Inverting (\ref{eqn:mlmic_dfn}) allows us to express $\mssaerwtr$ in
terms of $\mssaerdry$ 
\begin{eqnarray}
% Han76 p. 94 (2.41a)
\frac{\mssaerwtr}{\mssaerdry } & = & \frac{\mlmic \RHsfc}{1 - \RHsfc } 
\label{eqn:mss_aer_rat_dfn}
\end{eqnarray}
Table~\ref{tbl:dlq_dst} shows water activity $\actcffwtr$, mean linear
mass increase coefficient $\mlmic$, and water uptake per unit mass of
dry material $\mssaerwtr / \mssaerdry$ for both increasing and decreasing
$\actcffwtr$ 
\begin{table}
\begin{minipage}{\hsize} % Minipage necessary for footnotes KoD95 p. 110 (4.10.4)
\renewcommand{\footnoterule}{\rule{\hsize}{0.0cm}\vspace{-0.0cm}} % KoD95 p. 111
\begin{center}
\caption[Hygroscopic Growth of Saharan Dust]{\textbf{Observed
Hygroscopic Growth of Saharan Dust}%
\footnote{\emph{Source:} \cite{Han76}, p.~115, Table~IV. 
Measurements taken over the Atlantic from April 16--25, 1969.
The aerosol samples were identified as Saharan dust based on
prevailing meteorology and sample analysis.}%
\label{tbl:dlq_dst}}
\vspace{\cpthdrhlnskp}
\begin{tabular}{ >{$}l<{$} >{$}l<{$} >{$}r<{$} c >{$}l<{$} >{$}l<{$} >{$}r<{$} }
\hline 
\multicolumn{3}{c}{\rule[-0.7ex]{0.0ex}{2.0ex} Increasing $\actcffwtr$} & & 
\multicolumn{3}{c}{\rule[-0.7ex]{0.0ex}{2.0ex} Decreasing $\actcffwtr$} 
\\[0.0ex] \cline{1-3} \cline{5-7}
\actcffwtr & \mlmic & \mssaerwtr / \mssaerdry & & 
\actcffwtr & \mlmic & \mssaerwtr / \mssaerdry \\[0.0ex] 
\hline \rule{0.0ex}{\hlntblntrskp}%
0.204 & 0.032 & 0.008 & & & & \\[0.2ex]
& & & & 0.224 & 0.049 & 0.014 \\[0.2ex]
& & & & 0.294 & 0.045 & 0.019 \\[0.2ex]
0.349 & 0.024 & 0.013 & & & & \\[0.2ex]
0.457 & 0.021 & 0.018 & & & & \\[0.2ex]
& & & & 0.506 & 0.047 & 0.048 \\[0.2ex]
0.590 & 0.024 & 0.034 & & & & \\[0.2ex]
& & & & 0.623 & 0.072 & 0.119 \\[0.2ex]
0.648 & 0.033 & 0.060 & & & & \\[0.2ex]
0.700 & 0.088 & 0.206 & & & & \\[0.2ex]
& & & & 0.701 & 0.175 & 0.410 \\[0.2ex]
0.751 & 0.130 & 0.392 & & & & \\[0.2ex]
& & & & 0.753 & 0.158 & 0.482 \\[0.2ex]
0.789 & 0.133 & 0.497 & & & & \\[0.2ex]
0.843 & 0.120 & 0.644 & & & & \\[0.2ex]
0.896 & 0.107 & 0.922 & & & & \\[0.2ex]
0.900 & 0.105 & 0.945 & & & & \\[0.2ex]
0.955 & 0.075 & 1.604 & & & & \\[0.2ex]
0.971 & 0.072 & 2.411 & & & & \\[0.2ex]
0.976 & 0.071 & 2.887 & & & & \\[0.2ex]
0.986 & 0.067 & 4.719 & & & & \\[0.2ex]
0.990 & 0.065 & 6.435 & & & & \\[0.2ex]
0.997 & 0.065 & 20.90 & & & & \\[0.2ex]
1.000 & 0.081 & \infty & & & & \\[0.2ex]
\hline
\end{tabular}
\end{center}
\end{minipage}
\vfill
\end{table}
$\mlmic$ reaches a maximum value of 0.133 at $\RH \approx 80\%$.
The aerosol contains more water on the decreasing branch of relative
humidity than on the increasing branch.
This is called the \trmdfn{hysteresis effect}.
The water mass of the aerosol exceeds the dry mass when 
$\RH \gtrsim 90\%$.
The representativeness of these results is unknown.
Examining the adsorption of water vapor on clays, sand and soot at
298\,\K, Winkler (1970, MFC), as cited by \cite{PrK78}, p.~119, found 
that $\mssaerwtr \lesssim 0.5 \mssaerdry$ for $\RH < 95\%$.

The wet radius $\rdswet$ may be expressed in terms of the dry radius
$\rdsdry$ according to  
\begin{eqnarray}
% Han76 p. 123 (6.1)
\frac{\rds}{\rdsdry } & = & \left(
1 + \frac{\mlmic \dnsprtdry \RHsfc}{\dnswtr ( 1 - \RHsfc ) }
\right)^{1/3}
\label{eqn:hyg_grw_rds_Han76_1}
\end{eqnarray}
where $\dnsdry$ and $\dnswtr$ are the densities of the dry aerosol and
of liquid water, respectively.
Equation~(\ref{eqn:hyg_grw_rds_Han76_2}) is valid for $\RH <
0.90\mbox{--}0.95$ and for $\rdsdry > 0.1$\,\um.
If these constraints are too restrictive, \cite{Han76} derived an
expression valid for smaller particles at higher \RH
\begin{eqnarray}
% Han76 p. 123 (6.2)
\frac{\rds}{\rdsdry } & = & \left(
1 + \frac{\mlmic \dnsdry \RHsfc}{\dnswtr ( 1 - \RHsfc ) } 
\right)^{1/3} - 
\frac{2 \sfctnswtrlqd \spcvlmwtr}{\gascstwtr \tpt \rdsdry } 
\frac{\RHsfc}{1 - \RHsfc } 
\left(
1 + \frac{\mlmic \dnsdry \RHsfc}{\dnswtr ( 1 - \RHsfc ) } 
\right)^{-1/3}
\label{eqn:hyg_grw_rds_Han76_2}
\end{eqnarray}
Equation~(\ref{eqn:hyg_grw_rds_Han76_2}) is valid for $0.70 < \RH <
0.99$ and for $\rdsdry > 0.04$\,\um. 

\subsection[Zhang's Mechanism]{Zhang's Mechanism}\label{sxn:chm_dst_ZhC99}
\cite{ZhC99} studied the influence of mineral dust on tropospheric
chemistry in East Asia.

\subsection[Dentener's Mechanism]{Dentener's Mechanism}\label{sxn:chm_dst_DCZ96}
\cite{ZSK94} studied the synoptic scale impact of mineral dust on
particulate nitrate and sulfate formation.
Using similar methods on a global scale, \cite{DCZ96} performed the
first global study of heterogeneous chemistry on mineral dust
particles.  
The framework they developed for heterogeneous chemistry on mineral
dust provides an excellent overview of the roles of mineral dust as an
agent for sulfur and nitrogen oxidation.

\cite{DCZ96} assumed \HOd\ transfers to both dry and wetted dust
particles, where it reacts with \Fe\ to produce \HdOd. 
\setcounter{reaction}{4} % Reset counter for chemical reactions (dchem.sty)
\renewcommand{\thereaction}{DCZ96:\arabic{reaction}} % Set style for reaction numbering (dchem.sty)
\begin{rxnarray}
\label{rxn:HO2+FeIII_FeII+O2+H+}
\HOd + \FeIII & \yields & \FeII + \Od + \Hp \\ % DCZ96 p. 22871 (5)
\label{rxn:HO2+FeII+H+_FeIII+H2O2}
\HOd + \FeII + \Hp & \yields & \FeIII + \HdOd % DCZ96 p. 22871 (6) (Typo in JGR?)
\end{rxnarray}
Unfortunately, measurements of $\mssacmcff$ are not available for
\HOd\ to mineral dust transfer itself.
However $\mssacmcff$ has been measured for liquid water and \HdSOq\
surfaces.
These measurements show % DCZ96 p. 22871
$\mssacmcff(\HOd \rightarrow \HdO) > $0.01--0.2
and $\mssacmcff(\HOd \rightarrow \HdSOq) > 0.05$, respectively.
Saharan dust at relative humidity $\RH > 50\%$ takes up significant
amounts of water.
Based on these measurements, \cite{DCZ96} chose 
$\mssuptcff(\HOd \rightarrow \mbox{dust}) = 0.1$ for both wetted and dry
mineral aerosol.

\cite{DCZ96} ignored transfer of gas phase \OH\ and \HdOd\ to mineral
dust surfaces for two reasons.
First, the loss rate of \OH\ to mineral dust surfaces is expected to
be negligible compared to its quick destruction due to gas phase
processes. 
Second, hydrogen peroxide has a large mass accomodation coefficient
for aqueous surfaces, but the small volume of water available on dust
particles seems to imply a very small sink for \HdOd.
There are two reasons why this assumption should be re-examined.
First, dust particles may act as giant CCN, or become incorporated in
existing cloud droplets through collisions.
Sufficient aqueous volume is available in cloud droplets to process
large amounts of \HdOd\ in, e.g., \SOd\ reduction. 
The second reason is that reaction (\ref{rxn:HO2+FeII+H+_FeIII+H2O2})
may cause dust particles to be a source of \HdOd\ rather than a sink.

\cite{DCZ96} paid particular attention to the role of heterogeneous
chemistry on mineral dust as a sink for odd nitrogen.
In particular, they simulated the processing of \NdOc.
\NdOc\ concentration is a maximum at nighttime, when boundary layer
relative humidity also peaks.
Thus mineral dust particles are likely to have enough water to allow
the conversion of \NdOc\ to \HNOt\ via
\begin{reaction}
\NdOc + \HdO \yields 2\HNOt % DCZ96 p. 22871 (7)
\label{rxn:chm_NO3}
\end{reaction}
Unfortunately, the uptake of \NdOc\ on mineral particles has not been
directly measured so that estimates of 
$\mssuptcff(\NdOc \rightarrow \mbox{dust})$ must be based on uptake by
other surfaces.
Various studies, summarized by \cite{DCZ96}, show that
$0.06 < \mssuptcff(\NdOc \rightarrow \HdSOq) < 0.12$, 
$0.05 < \mssuptcff(\NdOc \rightarrow \NHqdSOq) < 0.09$, 
$\mssuptcff(\NdOc \rightarrow \HdO) \sim 0.05$, 
$\mssuptcff(\NdOc \rightarrow \HdSOq) \sim 0.05$, but that
$\mssuptcff(\NdOc \rightarrow \mbox{dry salts}) \sim 10^{-4}$.
Because of the relatively high probability of wetted dust particles at
night, \cite{DCZ96} chose 
$\mssuptcff(\NdOc \rightarrow \mbox{dust}) = 0.1$, a value much closer
to the uptake coefficients observed on liquid surfaces than on dry
surfaces. 

Gas phase \NOt\ may also adsorb onto dust surfaces. 
However, \cite{DCZ96} found \NOt\ adsorption to have a small effect on
overall removal of \NOx\ and therefore neglected it.

\cite{DCZ96} paid very careful attention to uptake of \HNOt\ and \SOd\
on mineral dust.
The abundance of \CaCOt\ in dust allows it to neutralize these strong
acids.
Two of the three main conclusions of \cite{DCZ96} were that production
of \SOqdm\ and \NOtm\ in the atmosphere was strongly affected by the
presence of mineral dust.

Once sequestered on mineral dust nitric acid is assumed to be
neutralized by alkaline material, usually \CaCOt\ \cite[]{DCZ96}
\begin{rxnarray}
\HNOt + \CaCOt & \yields & \Cadp\NOtm + \HCOtm \\ % DCZ96 p. 22872 (8a)
\HNOt + \Cadp\NOtm + \HCOtm & \yields & \Ca(\NOt)_{2} + \HdO + \COd % DCZ96 p. 22872 (8b)
\label{rxn:chm_HNO3}
\end{rxnarray}
This reaction is fast, and is considered diffusion-limited.  
\cite{KGL03} first observed this mechanism in the laboratory.
They showed that the calcium nitrate end-product deliquesces at low \RH\@. 

\begin{rxnarray}
% fxm: These reactions are numbered with DCZ96 prefixes in aer.ps
\CaCOt + 2\HNOt & \yields & \CaNOtd + \HdCOt \\ % KGL03 p. 48-1 (R1)
\label{rxn:CaCO3+2HNO3}
\CaNOtd + \HdCOt & \yields & \CaNOtd + \COd + \HdO \\ % KGL03 p. 48-1 (R1), HaC01a p. 3102 (1)
\label{rxn:CaNO32+H2CO3}
\CaNOtd\sldprn + n\cdot\HdO & \yields & \CaNOtd\lqdprn % KGL03 p. 48-1 (R2)
\label{rxn:CaNO32+nH2O}
\end{rxnarray}
The deliquesence in (\ref{rxn:CaNO32+nH2O}) helps explain the high
reactivity of \HNOt\ and \CaCOt.

The uptake and neutralization of sulfur dioxide on mineral surfaces
may be substantially more complicated.
\SOd\ strongly adsorbs to the oxide surfaces characteristic of mineral
dust \cite[]{UAC02}.
However, \SOd\ adsorption onto pure quartz appears negligible.
For \CaCOt, 
\begin{rxnarray}
% fxm: This reaction is numbered with DCZ96 prefixes in aer.ps
\CaCOt + 2\HNOt & \yields & \CaNOtd + \HdCOt \\ % UAC02 p. 16-6 (6)
\label{rxn:CaCO3+2HNO3}
\end{rxnarray}

Once sequestered on wetted mineral dust, \SOd\ dissociates \HdO\
to form the sulfite ion \SOtdm\ via
(\ref{rxn:SO2H2O_H++HSO3-})--(\ref{rxn:HSO3-_H++SO3--}). 
The sulfite may be oxidized into sulfate by ozone 
\begin{rxnarray}
\SOtdm + \Ot & \yields & \SOqdm + \Od % DCZ96 p. 22872 (9)
\label{rxn:chm_SO2}
\end{rxnarray}
This reaction is only expected to take place in the presence of liquid
water at very high~\pH\@.
For $\pH > 8$, (\ref{rxn:chm_SO2}) proceeds very quickly, becoming
limited only by gas phase diffusion processes.
The range of \pH\ which must be accounted for when considering
heterogeneous chemistry on mineral dust is quite large.
\cite{DCZ96} assume that, on mineral dust aerosol (as opposed to in
cloud droplets), there is not a sufficient volume of water to allow 
\HdOd\ oxidation of \SOd\ (\ref{rxn:HSO3-H2O+H2O2H2O_HSO4-H2O+H2O}) to
compete with (\ref{rxn:chm_SO2}).

Unfortunately, direct measurement of the uptake of \SOd\ and \NOtm\
onto dust are not available.
Their uptake coefficients on liquid water are known to be about~0.1. 
\cite{DCZ96} summarize the uptake coefficients that have been measured
for a variety of ``dry'' aerosols such as \FedOt, fly ash, and soot.
These measurements indicate
$10^{-3} < \mssuptcff(\SOd \rightarrow \mbox{dry aerosol}) < 10^{-6}$. 
The uptake coefficients tended to decrease with prolonged exposure,
but their rate of decrease was much less at high relative humidity.

More recent experiments have measure the uptake of \HNOt\ onto mineral
dust in a laboratory flow tube (V. Grassian, personal communication,
1999, 2003). 
It is found that mineral dust is coated with a surface layer of \HdO\
at very low relative humidities, i.e., $\RH > 30\%$.
This liquid coating uptakes \HNOt\ at rates corresponding to 
$10^{-4} < \mssuptcff(\HNOt \rightarrow \mbox{dust}) < 10^{-3}$
\cite[]{ULA01,Gra02,KGL03}.
These rates are much smaller than the $\mssuptcff = 0.1$ employed by
\cite{DCZ96} and \cite{BBS04} and explain much of the difference in
model results.

The rate of \SOd\ deposition to mineral dust aerosol has also been
inferred from measurements of surface resistance to \SOd\ transfer
over calcarious soils, \FedOt, and other proxies for mineral dust.
A measured value of the resistance to surface transfer by dry
deposition $\rssdps$ is converted to a deposition velocity $\vlcdps$  
using (\ref{eqn:vlc_dps_dfn_2}).
Then, the uptake coefficient $\mssuptcff$ is inferred using the 
approximation \cite[]{Sch92}
\begin{equation}
\mssuptcff = \frac{\vlcdps}{\vlcmlcavg }
\label{eqn:mss_upt_cff_Sch92}
\end{equation}
where $\vlcmlcavg$ is the ambient thermal speed of \SOd\ molecules
(\ref{eqn:vlc_mlc_avg_dfn}). 

\begin{rxnarray}
\CaCOt + 2\Hp & \yields & \Cadp + \COd + \HdO \\ % DCZ96 p. 22872 (11)
\label{rxn:HSO3-H2O+H2O2H2O_HSO4-H2O+H2O}
\HSOtmaq + \HdOdaq & \yields & \HSOqmaq + \HdO \\ % DCZ96 p. 22872 (12)
\SOd + \OH + \Od & \yields & \SOt + \HOd \\ % DCZ96 p. 22872 (13a)
\SOt + \HdO & \yields & \HdSOq % DCZ96 p. 22872 (13b)
\label{rxn:chm_HOx}
\end{rxnarray}

\subsection[Bauer's Mechanism]{Bauer's Mechanism}\label{sxn:chm_dst_BBS04}
\cite{BBS04} performed global model simulations of heterogeneous
chemistry on mineral dust and compared them to observations in Italy
during the MINATROC campaign.

\subsection{My Mechanism}\label{sxn:chm_csz}
\setcounter{reaction}{0} % Reset counter for chemical reactions (dchem.sty)
\renewcommand{\thereaction}{Zen00:\arabic{reaction}} % Set style for reaction numbering (dchem.sty)
We have implemented a representation of heterogeneous chemistry on
mineral dust based on \cite{ZSK94} and \cite{ZhC99}. 
This scheme accounts for uptake of \HdOd, \HNOt, \HOd,
\NdOc, \NOd, \NOt, \Ot, \OH, and \SOd\ on mineral dust particles.
Table~\ref{tbl:mss_upt_cff_mdl} shows the uptake coefficients employed in
the global mineral dust model \cite[]{BiZ03,BiZ04}.
\begin{table}
\begin{minipage}{\hsize} % Minipage necessary for footnotes KoD95 p. 110 (4.10.4)
\renewcommand{\footnoterule}{\rule{\hsize}{0.0cm}\vspace{-0.0cm}} % KoD95 p. 111
\begin{center}
\caption[Uptake Coefficients of Mineral Dust in UCI CTM]{\textbf{Uptake Coefficients of Mineral Dust in UCI CTM}% 
\footnote{\emph{Sources:} $\RHthr = 50\%$}%
\label{tbl:rxr_dst_mdl}}
\vspace{\cpthdrhlnskp}
\begin{tabular}{ >{$\ch}l<{$} >{$}l<{$} r }
\hline \rule{0.0ex}{\hlntblhdrskp}% 
Reaction & \mbox{Uptake}\ \mssuptcff & References% 
\footnote{\emph{References:} 
\setcounter{enmrfr}{0} % Reset reference counter for this table
\enmrfrstpprn, \cite{DCZ96}\label{idx_rxr_dst_mdl_DCZ96}; 
\enmrfrstpprn, \cite{SeP97}\label{idx_rxr_dst_mdl_SeP97};
\enmrfrstpprn, \cite{JPL97}\label{idx_rxr_dst_mdl_JPL97};
\enmrfrstpprn, \cite{ZhC99}\label{idx_rxr_dst_mdl_ZhC99};
\enmrfrstpprn, \cite{USP01}\label{idx_rxr_dst_mdl_USP01};
\enmrfrstpprn, \cite{ULA01}\label{idx_rxr_dst_mdl_ULA01};
\enmrfrstpprn, \cite{MUG02}\label{idx_rxr_dst_mdl_MUG02};
\enmrfrstpprn, \cite{MUG03}\label{idx_rxr_dst_mdl_MUG03};
}%
\\[0.0ex]
\hline \rule{0.0ex}{\hlntblntrskp}%
\HdOd + Dust \yields Products & 1.0 \times 10^{-4} & \ref{idx_rxr_dst_mdl_DCZ96} \\[0.5ex]
\HNOt + Dust \yields Products & 1.1 \times 10^{-3} & \ref{idx_rxr_dst_mdl_DCZ96}, \ref{idx_rxr_dst_mdl_JPL97}, \ref{idx_rxr_dst_mdl_USP01} \\[0.5ex]
\HOd + Dust \yields Products & 0.1 & \ref{idx_rxr_dst_mdl_DCZ96}, \ref{idx_rxr_dst_mdl_ZhC99} \\[0.5ex]
\NdOc + Dust \yields Products & 1.0 \times 10^{-3} & \ref{idx_rxr_dst_mdl_DCZ96}, \ref{idx_rxr_dst_mdl_JPL97} \\[0.5ex]
\NOd + Dust \yields Products & 4.4 \times 10^{-5} & \ref{idx_rxr_dst_mdl_USP01} \\[0.5ex]
\NOt + Dust \yields Products & 0.1 & \ref{idx_rxr_dst_mdl_SeP97}, \ref{idx_rxr_dst_mdl_ZhC99} \\[0.5ex]
\Ot + Dust \yields Products & 5.0 \times 10^{-5} & \ref{idx_rxr_dst_mdl_DCZ96}, \ref{idx_rxr_dst_mdl_JPL97}, \ref{idx_rxr_dst_mdl_ZhC99}, \ref{idx_rxr_dst_mdl_MUG02} \\[0.5ex]
\OH + Dust \yields Products & 1.0 \times 10^{-5} & \ref{idx_rxr_dst_mdl_ZhC99} \\[0.5ex] % Based on Jack Lunsford Texas A&M measurements of OH uptake on dry Fe2O3
%\SOd + Dust \yields Products & 3.0 \times 10^{-4} & \ref{idx_rxr_dst_mdl_DCZ96}, \ref{idx_rxr_dst_mdl_ZhC99} \\[0.5ex]
\hline
\end{tabular}
\end{center}
\end{minipage}
\end{table}

fxm: Redundant?
\begin{table}
\begin{minipage}{\hsize} % Minipage necessary for footnotes KoD95 p. 110 (4.10.4)
\renewcommand{\footnoterule}{\rule{\hsize}{0.0cm}\vspace{-0.0cm}} % KoD95 p. 111
\begin{center}
\caption[Uptake coefficients in Global Mineral Dust Model]{\textbf{Uptake
coefficients in Global Mineral Dust Model}% 
\footnote{\emph{Source:} 
Source of best guess value. 
Most values adopted from \cite{DCZ96},
$\RHthr = 50\%$
}%
\label{tbl:mss_upt_cff_mdl}}
\vspace{\cpthdrhlnskp}
\begin{tabular}{ >{$}r<{$} >{$}r<{$} >{$}l<{$} } % KoD95 p. 94 describes '*' notation
\hline \rule{0.0ex}{\hlntblhdrskp}% 
\mssuptcff(\HOd \rightarrow \mbox{dust}) & = & 0.1 \\ % DCZ96 p. 22871 
\mssuptcff(\Ot \rightarrow \mbox{dust}) & = & 0.00005 \\ % DCZ96 p. 22874 
^{a}\mssuptcff(\NdOc \rightarrow \mbox{dust}) & = & \left\{
\begin{array}{l@{\quad:\quad}r}
0.001 & \RH < \RHthr \\ % DCZ96 p. 22871 
0.1 & \RH \ge \RHthr % DCZ96 p. 22871 
\end{array} \right. \\
^{a}\mssuptcff(\HNOt \rightarrow \mbox{dust}) & = & 0.1 \\ % DCZ96 p. 22872 
^{a}\mssuptcff(\SOd \rightarrow \mbox{dust}) & = &\left\{
\begin{array}{l@{\quad:\quad}r}
3 \times 10^{-4} & \RH < \RHthr \\ % DCZ96 p. 22872 
0.1 & \RH \ge \RHthr % DCZ96 p. 22872 
\end{array} \right. \\
\hline
\end{tabular}
\end{center}
\end{minipage}
\end{table}
\footnote{Uptake only when $[\NOtm] + 2[\SOqdm] < 2[\Cadp]$}
To be consistent with the deliquesence approximation we 
employ separate mass uptake coefficients above and below $\RHthr$.
Furthermore, some reactions take place only when the empirical
alkalinity relationship is obeyed:
\begin{equation}
[\NOtm] + 2[\SOqdm] < 2[\Cadp] % 
\end{equation}

The mineral dust model employs the following heterogeneous pathways.
\begin{rxnarray}
\label{rxn:Zen00_1} 
\HNOt + Dust & \yields & Products % DCZ96 p. 22872 (8a)
\end{rxnarray}

Table~\ref{tbl:rxr_dst} lists the reactions employed in the mineral
dust model of \cite{Zen00}. 
\begin{table}
\begin{minipage}{\hsize} % Minipage necessary for footnotes KoD95 p. 110 (4.10.4)
\renewcommand{\footnoterule}{\rule{\hsize}{0.0cm}\vspace{-0.0cm}} % KoD95 p. 111
\begin{center}
\caption[Reactions in Global Mineral Dust Model]{\textbf{Reactions Included in Global Mineral Dust Model}% 
\footnote{\emph{Sources:}
DCZ96 = \cite{DCZ96}; 
JPL97 = \cite{JPL97};
HeC99 = \cite{HeC99};
BrS86 = \cite{BrS86};
Gra99 = \cite{Gra99};}% 
\label{tbl:rxr_dst}}   
\vspace{\cpthdrhlnskp}
\begin{tabular}{ r >{$\ch}c<{$} >{$}c<{$} >{$}c<{$} r }
\hline \rule{0.0ex}{\hlntblhdrskp}% 
Index & Reaction & \mbox{Uptake coefficient} & \mbox{Humidity} & Reference \\[0.0ex]
\hline \rule{0.0ex}{\hlntblntrskp}%
\ref{rxn:Zen00_1} & \HNOt + Dust \yields Products & 
0.005 & -1.0 & DCZ96 \\[0.5ex]
\hline
\end{tabular}
\end{center}
\end{minipage}
\end{table}

\section{Global Sulfur Cycle}\label{sxn:glb_slf}
\setcounter{reaction}{0} % Reset counter for chemical reactions (dchem.sty)
\renewcommand{\thereaction}{BRK00:\arabic{reaction}} % Set style for reaction numbering (dchem.sty)
\cite{BRK00} developed a global model of the atmospheric sulfur cycle.
Their model provides instructive examples of the use of the 
(computational) time saving approximations which are currently
required in large scale atmospheric models.

Anthropogenic emissions of sulfur are interpolated from the GEIA
emissions inventory \cite[][]{BSP96} representative of the year 1985.  
The GEIA database assumes sulfur emissions (67\,\TgSxyr) are 98\% \SOd\
and 2\% \SOqdm\ 2\% by weight. 
Gas phase \SOd\ is converted to \SOqdm\ via a multistep reaction
\cite[e.g.,][]{SeP97}. 
\cite{BRK00} assume that the rate limiting reaction in this chain is
\begin{rxnarray}
\label{rxn:BRK00_1}
\SOd + \OH + \M & \yields & \SOqdm + \M  % BRK00 p. 6 (1)
\end{rxnarray}
The \OH\ radical concentration required in (\ref{rxn:BRK00_1}), as
well as \NHt, \HOd, and \Ot\ required in subsequent reactions, is
prescribed from the monthly values generated by the IMAGES model
\cite[][]{MuB95}. 

Conversion of DMS to \SOd\ is the main natural path for production of
sulfate. 
Natural emissions of DMS are interpolated from the global monthly
dataset of \cite{BBE94}.
\begin{rxnarray}
\label{rxn:BRK00_2} 
DMS + \OH & \yields & \alpha \SOd + (1 - \alpha) \MSA \\ % BRK00 p. 6 (2)
\label{rxn:BRK00_3} 
DMS + \NOt & \yields & \SOd + \HNOt % BRK00 p. 6 (3)
\end{rxnarray}

Hydrogen peroxide is entirely prognostic within the model.
Gas phase \HdOd\ is produced by \HOd, photolyzed, and destroyed by
\OH\ according to 
\begin{rxnarray}
\label{rxn:BRK00_4} 
\HOd + \HOd & \yields & \HdOd + \Od \\ % BRK00 p. 6 (4)
\label{rxn:BRK00_5} 
\HdOd + h\nu & \yields & 2 \OH \\ % BRK00 p. 6 (5)
\label{rxn:BRK00_6} 
\HdOd + \OH & \yields & \HOd + \HdO % BRK00 p. 6 (6)
\end{rxnarray}
The photolysis rate coefficient $\prc_{\HdOd}$ in
(\ref{rxn:BRK00_5}) is diagnosed from the diurnal-average zenith angle
and the gridbox height.

Aqueous phase hydrogen peroxide oxidizes \HSOtm\
\begin{rxnarray}
\label{rxn:BRK00_7} 
\HSOtm + \HdOd & \yields & \SOqdm + 2\Hp + \HdO % BRK00 p. 6 (7)
\end{rxnarray}
Trace metals in solution can catalyze production of \HdOd\
\cite[][]{AFA94,DCZ96}.
However, \cite{BRK00} assume \HdOd\ is in Henry's law equilibrium.
Wetted mineral dust aerosol, which is rich in trace metals, is likely
to provide an extreme test of this assumption \cite[][]{DCZ96}.

Aqueous phase ozone reacts with sulfite ions to produce sulfate and
oxygen 
\begin{rxnarray}
\label{rxn:BRK00_8} 
\HSOtm + \Ot & \yields & \SOqdm + \Hp + \Od \\ % BRK00 p. 6 (8)
\label{rxn:BRK00_9} 
\SOtdm + \Ot & \yields & \SOqdm + \Od % BRK00 p. 6 (9)
\end{rxnarray}
As mentioned below, \HdOd\ and \Ot\ are assumed to be in Henry's law 
equilibrium.
The rates of oxidation by \Ot\ are highly sensitive to droplet \pH\@.   
Therefore the model evaluates the aqueous phase reactions
(\ref{rxn:BRK00_7})--(\ref{rxn:BRK00_9}), and the droplet \pH, every
two minutes. 

The aqueous and gas phase concentrations of \HdOd, \Ot, and \SOd\
are assumed to be in Henry's Law equilibrium at all times. 
\begin{rxnarray}
\label{rxn:BRK00_10} 
\HdOd (g) & \eqbm & \HdOd (aq) \\ % BRK00 p. 6 (10)
\label{rxn:BRK00_11} 
\Ot (g) & \eqbm & \Ot (aq) \\ % BRK00 p. 6 (11)
\label{rxn:BRK00_12} 
\SOd (g) & \eqbm & \SOd (aq) % BRK00 p. 6 (12)
\end{rxnarray}
The final two gas phase components, \HdSOt\ and \HSOtm, produce
\Hp\ when they hydrolyze.
\begin{rxnarray}
\label{rxn:BRK00_13} 
\HdSOt & \eqbm & \HSOtm + \Hp \\ % BRK00 p. 6 (13)
\label{rxn:BRK00_14} 
\HSOtm & \eqbm & \SOtdm + \Hp % BRK00 p. 6 (14)
\end{rxnarray}

Table~\ref{tbl:rxr_SO4} lists the reaction rates employed in the
global sulfur model of \cite{BRK00}.
\begin{landscape}
\begin{table}
\begin{minipage}{\hsize} % Minipage necessary for footnotes KoD95 p. 110 (4.10.4)
\renewcommand{\footnoterule}{\rule{\hsize}{0.0cm}\vspace{-0.0cm}} % KoD95 p. 111
\begin{center}
\caption[Reaction Rates in Global Sulfur Model]{\textbf{Reactions
Included in Global Sulfur Model of \cite{BRK00}}% 
\footnote{\emph{Sources:} \emph{Sources:} JPL97 = \cite{JPL97}.}%
\label{tbl:rxr_SO4}}   
\vspace{\cpthdrhlnskp}
\begin{tabular}{ r >{$\ch}r<{$} >{$\ch}c<{$} >{$\ch}l<{$} >{$}l<{$} >{$}r<{$} r }
\hline \rule{0.0ex}{\hlntblhdrskp}% 
Index & Reaction & & & \mbox{Rate} & E/R & Reference \\[0.0ex]
\hline \rule{0.0ex}{\hlntblntrskp}%
& & & \mathrm{Gas Chemistry} & & & \\[0.5ex]
\ref{rxn:BRK00_1} & \SOd + \OH + \M & \yields & \SOqdm + \M & k_{0} =
3.0 \times 10^{-31} ( \tpt / 300 )^{-3.3} & & JPL97 \\[0.0ex]
& & & & k_{\infty} = 1.5 \times 10^{-12} & & JPL97 \\[0.5ex]
\ref{rxn:BRK00_2} & DMS + \OH & \yields & \alpha \SOd + (1 -
\alpha) \MSA & & & Y90 \\[0.5ex] % BRK00 p. 6 (2)
\ref{rxn:BRK00_3} & DMS + \NOt & \yields & \SOd + \HNOt & & & JPL97 \\[0.5ex] % BRK00 p. 6 (3)
\ref{rxn:BRK00_4} & \HOd + \HOd & \yields & \HdOd + \Od & 8.6 \times
10^{-12} & -590 & JPL97 \\[0.5ex] % BRK00 p. 6 (4)
\ref{rxn:BRK00_5} & \HdOd + h\nu & \yields & 2 \OH & & & \\[0.5ex] % BRK00 p. 6 (5)
\ref{rxn:BRK00_6} & \HdOd + \OH & \yields & \HOd + \HdO & 1.7 \times
10^{-12} & 160 & JPL97 \\[0.5ex] % BRK00 p. 6 (6)
\hline
& & & \mathrm{Aqueous Chemistry} & & & \\[0.5ex]
\ref{rxn:BRK00_7} & \HSOtm + \HdOd & \yields & \SOqdm + 2\Hp + \HdO
& \mbox{\footnote{hello}}2.7 \times 10^{7} & 4750 & HC85 \\[0.5ex] % BRK00 p. 6 (7)
\ref{rxn:BRK00_8} & \HSOtm + \Ot & \yields & \SOqdm + \Hp + \Od & 3.7
\times 10^{5} & 5300 & HC85 \\[0.5ex] % BRK00 p. 6 (8)
\ref{rxn:BRK00_9} & \SOtdm + \Ot & \yields & \SOqdm + \Od & 1.5 \times
10^{9} & 5280 & HC85 \\[0.5ex] % BRK00 p. 6 (9)
\hline
& & & \mathrm{Equilibrium Reactions} & & & \\[0.5ex]
\ref{rxn:BRK00_10} & \HdOd (g) & \eqbm & \HdOd (aq) & 7.4 \times 10^{4}
& -6621 & LK86 \\[0.5ex] % BRK00 p. 6 (10)
\ref{rxn:BRK00_11} & \Ot (g) & \eqbm & \Ot (aq) & 1.15 \times 10^{-2}
& -2560 & NBS65 \\[0.5ex] % BRK00 p. 6 (11)
\ref{rxn:BRK00_12} & \SOd (g) & \eqbm & \SOd (aq) & 1.23 & -3120 &
NBS65 \\[0.5ex] % BRK00 p. 6 (12)
\ref{rxn:BRK00_13} & \HdSOt & \eqbm & \HSOtm + \Hp & 1.3 \times
10^{-2} & -2015 & M82 \\[0.5ex] % BRK00 p. 6 (13)
\ref{rxn:BRK00_14} & \HSOtm & \eqbm & \SOtdm + \Hp & 6.3 \times
10^{-8} & -1505 & M82 \\[0.5ex] % BRK00 p. 6 (14)
\hline
\end{tabular}
\end{center}
\end{minipage}
\end{table}
\end{landscape}

\chapter{Photochemistry}\label{sxn:pch}

\cite{HeC99} studied the sensitivity of photolysis rates and and ozone
production to mineral aerosol properties.
\setcounter{reaction}{0} % Reset counter for chemical reactions (dchem.sty)
\renewcommand{\thereaction}{HeC99:R\arabic{reaction}} % Set style for reaction numbering (dchem.sty)
\begin{rxnarray}
% HeC99 reactions
\NOd + h\nu & \yields^{\lambda < 420\mathrm{~nm}} & \NO + \Ou \\ 
\Ou + \Od + \M & \yields & \Ot + \M \\
\Ot + \NO & \yields & \Od + \NOd \\
\Ot + h\nu & \yields^{\lambda < 320\mathrm{~nm}} & \Od + \OsD \\
\OsD + \HdO & \yields & 2OH \\
\NO + \HOd & \yields & \NOd + \OH \\
\Ot + \OH & \yields & \Od + \HOd \\
\Ot + \HOd & \yields & 2\Od + \HOd \\
\NOd + \OH & \yields & \HNOt
\label{rxn:pch_HeC99}
\end{rxnarray}

A more complete set of photodissociation reactions is
\setcounter{reaction}{0} % Reset counter for chemical reactions (dchem.sty)
\renewcommand{\thereaction}{PR\arabic{reaction}} % Set style for reaction numbering (dchem.sty)
\begin{rxnarray}
\Ot + h\nu & \Yields^{\lambda < 320\mathrm{~nm}} & \OsD + \OdtSgm \\ % BrS86 p. 154
\Ot + h\nu & \Yields^{\lambda > 320\mathrm{~nm}} & \OtP + \OdsDg \\ % BrS86 p. 154
\NOt + h\nu & \Yields^{\lambda < 580\mathrm{~nm}} & \NOd + \Ou \\ % BrS86 p. 167
\NOt + h\nu & \Yields^{\lambda > 580\mathrm{~nm}} & \NO + \Od \\ % BrS86 p. 154
\NdO + h\nu & \Yields^{\lambda < 240\mathrm{~nm}} & \Nd + \OsD \\
\HdOd + h\nu & \Yields^{\lambda < 350\mathrm{~nm}} & \OH + \OH \\
\HOd + h\nu & \Yields^{\lambda < 250\mathrm{~nm}} & \OH + \Ou \\ 
\NOd + h\nu & \Yields^{\lambda < 420\mathrm{~nm}} & \NO + \Ou
\label{rxn:pch}
\end{rxnarray}

\section{Heterogeneous Photochemistry}\label{sxn:htr_pch}
Absorption bands of adsorbed species are usually red-shifted relative 
to bands of the gas phase species.
Photolysis of the adsorbed species may be significantly more efficient
if the shift crosses the sharp cutoff of atmospheric UV radiation
near $\wvl = 290$\,\um.

\chapter[Biogeochemistry]{Biogeochemistry}\label{sxn:bgc}

\section{Biogeochemistry Literature}\label{sxn:bgc_ltr}

The minerals composing dust may be of significant biogeochemical
relevance to many ecosystems.
The ``Iron Hypothesis'' suggests that dust delivers the crucial
micronutrient \FeIII\ to otherwise iron-limited ecosystems
\cite[]{MaF88,Mar90}. 
\cite{DCM97} describe a method to use the \Al/\Ti\ ratios of ocean
sediments as a proxy for ocean paleoproductivity (specifically
of biogenic opal).
\cite{MAR00} show that increased Aeolian deposition to the Southern
Ocean at the last glacial maximum contributed significantly to an increased
net \COd\ flux.
\cite{FMT00} study the iron supply and demand in the upper ocean.
\cite{GKT01} estimate the deposition of iron to the global ocean from observations.
\cite{MDG02} use an intermediate complexity marine ecosystem model
\cite[]{MDK02} to study the role of dust-borne iron in nutrient
transport. 
\cite{MeV00} describe the use of dissolved \Al\ to estimate dust
deposition fluxes to the ocean.
\cite{Joh01} argues that the interplanetary dust flux to remote oceans
is a significant compared to terrestrial and upwelling Fe fluxes.
\cite{EGM01} assess the trace metal (including \Fe) budgets of the
western Mediterranean Sea.
\cite{RGL99} present estimates of the deposition budgets of \Fe, \Al,
and total dust in the northwestern Mediterranean between 1985--1997. 
\cite{BDS02} show direct observations of enhanced particulate carbon
occur in the North Pacific after passage of a Gobi dust storm.
\cite{JMS02} summarize the current knowledge and uncertainties in
linking oceanic iron inputs to nutrient production, fixation, and
export to the deep ocean.
\cite{GCA02} present evidence for a coupling between \trmidx{SeaWiFS} 
\trmidx{AOD} and upper ocean chlorophyll concentration at daily,
weekly and monthly timescales. 
The AOD is thought to be dust from Australia.
In near coastal zones, riverine mineral fluxes to the ocean margin are
often more important than atmospheric inputs.
\cite{JMT02} describe a model of the riverine fluxes of dissolved
\HCOtm, \Si, and \Ge\ to the ocean based on a global lithological
map. 
\cite{GHM03} implemented \Al\ and \Si\ geochemistry in an ocean
ecosystem model to evaluate dust deposition fields.
\cite{GFS03} estimate \Fe\ deposition from wet scavenging.
\cite{HMC04} estimate soluble iron from observations and a global
mineral aerosol model. 
\cite{JAA05} review the connections betwee \Fe, dust, ocean
biogeochemistry, and climate.
\cite{MBB05} review understanding of transport, solubility, and
deposition of iron to the ocean via dust aerosols.
\cite{LMM05} estimate global iron solubility and Al concentration
from observations and a global aerosol model.
\cite{MDL06} describe dust-mediated impacts of paleo, pre-industrial,
present day, and future climates on ocean biogeochemistry and air-sea 
\COd\ exchange.
\cite{SBD05} analyze the cycling of \Al\ in the Arabian Sea based on
three years of measurements of atmospheric and marine \Al\
concentrations. 
\cite{CPJ07} use measurements of atmospheric and marine \Fe\
concentrations in the Gulf of Aqaba to infer dust deposition and
constrain \Fe\ solubility.

\section{Limiting Nutrients: Iron}\label{sxn:lmt_ntr_Fe}

\section{Limiting Nutrients: Nitrogen}\label{sxn:lmt_ntr_N}

\section{Limiting Nutrients: Phosphorous}\label{sxn:lmt_ntr_P}

\cite{AHC03} argue that \trmidx{phosphorous deficiency} may limit or
co-limit \trmidx{phytoplankton} growth in much of the \trmidx{North
  Atlantic}. 
It is thought that the North Atlantic shows less \trmidx{nitrogen
  deficiency} than the \trmidx{Pacific Ocean} because mineral dust
deposition is much stronger in the North Atlantic.
Nitrogen fixation is often limited by the availability of iron since
the enzyme \trmidx{nitrogenase} (which converts \Nd\ to \NHq) has high
iron requirements. 
Should dust provide this iron, it could explain why the Atlantic may
be less nitrogen- and more phosphorous-limited than the Pacific.

\chapter[Implementation in NCAR models]{Implementation in NCAR models}\label{sxn:mpl}

This section describes how the above physics and chemistry have been 
implemented in the NCAR-Dust model.

\section{Initialization}\label{sxn:ini}
The model requires as input global distributions of soil texture, mass
fractions of mobilized dust, leaf area index, soil moisture and
wind speed. 

At startup, the model computes $\stkcrc(\dmt,\prs_{0},\tpt_{0})$
(\ref{eqn:stk_crc_dfn})for each size at a temperature and pressure
representative of arid erosion regions on Earth (currently $\prs =
1000$~mb and $\tpt = 295$\,\K).  
The model computes $\vlcstk$ (\ref{eqn:vlc_stk_dfn}) every timestep
and applies the time-invariant correction factor
$\stkcrc(\dmt,\prs_{0},\tpt_{0})$ to obtain $\vlcgrv$.
In this manner the iterative solution to (\ref{eqn:vlc_grv_dfn}) is
avoided every timestep.

The model iteratively solves (\ref{eqn:wnd_frc_thr_obs}) until the
fractional difference between successive iteration is less
than~$10^{-5}$.  
Convergence is usually be obtained within five iterations.

\section{Main Loop}\label{sxn:loop}

\subsection{Mobilization}\label{sxn:mdl_mbl}
A more computationally amenable form of (\ref{eqn:wnd_frc_thr_obs}) is
employed.
\begin{eqnarray}
\wndfrcthr & = & \left\{
\begin{array}{l@{\quad:\quad}r}
\left[
\frac{0.1666681 \dnsprt \grv \dmt}{-1 + 1.928 \BBB^{0.0922}}
\left( 1 + \frac{6 \times 10^{-7}}{\dnsprt \grv \dmt^{2.5}} \right)
\right]^{1/2}
\dnsatm^{-1/2}
& 0.03 \le \BBB \le 10 \\
\left[
0.0144 \dnsprt \grv \dmt ( 1 - 0.0858 \me^{ -0.0617 ( \BBB - 10 ) } )^{2}
\left( 1 + \frac{6 \times 10^{-7}}{\dnsprt \grv \dmt^{2.5} } \right)
\right]^{1/2}
\dnsatm^{-1/2}
& \BBB > 10
\end{array} \right.
\label{eqn:wnd_frc_thr_obs_2}
\end{eqnarray}
Expression (\ref{eqn:wnd_frc_thr_obs_2}) isolates all the
microphysical properties are in the first term on the RHS.
This term contains the size and density of the aerosol and thus needs
to be computed only once for a given aerosol size. 
The other term, $\dnsatm^{-1/2}$ depends on the ambient environmental
conditions which are the same for all aerosol sizes.
Hence the (presumably) time-varying environmental term is all that
need be recomputed each timestep.

To convert the streamwise particle flux to a total vertical dust flux,
we use (\ref{eqn:hrz_vrt_prp_fct_MaB95_2}) with the clay mass fraction
$\mssfrcclyprm$ defined by $\mssfrcclyprm = \min(\mssfrccly,0.20)$.

\chapter{Appendix}\label{sxn:app}

\section[Physical Constants]{Physical Constants}\label{sxn:cst}
Table~\ref{tbl:cst} list the values of common physical constants.
These values should be consistent with the 1998 CODATA adjustment
reported in \cite{MoT00}.
\begin{table}
\begin{minipage}{\hsize} % Minipage necessary for footnotes KoD95 p. 110 (4.10.4)
\renewcommand{\footnoterule}{\rule{\hsize}{0.0cm}\vspace{-0.0cm}} % KoD95 p. 111
\begin{center}
\caption[Physical Constants]{\textbf{Physical Constants}% 
\label{tbl:cst}}   
\vspace{\cpthdrhlnskp}
\begin{tabular}{ >{$}r<{$} p{12em} >{$}r<{$} l r}
\hline \rule{0.0ex}{\hlntblhdrskp}% 
\mbox{Symbol} & Name & \mbox{Value} & Units & 
\footnote{\emph{Sources:} Bol80 = \cite{Bol80}, RRG98 = \cite{RRG98}, 
MoT00 = \cite{MoT00}}%
Reference \\[0.0ex]
\hline \rule{0.0ex}{\hlntblntrskp}%
\cstplk & Planck's constant & 6.626196 \times 10^{-34} & \js & \\[0.5ex] % From ThS99 p. 489
\mssrth & Mass of Earth & 5.98 \times 10^{24} & \kg & \\[0.5ex] % 
\gascstunv & Universal gas constant & 8.31441 & \jxmolK & \\[0.5ex]
\cstgrv & Newtonian constant of gravitation & 6.673 \times 10^{-11} &
\mCxkgsS \quad \NmSxkg & MoT00 \\[0.5ex]
\bltcst & Boltzmann's constant & 1.38063 \times 10^{-23} & \jxK & \\[0.5ex]
\cststfblt & Stefan-Boltzmann constant & 5.67032 \times 10^{-8} & \wxmSkQ & \\[0.5ex] % GoY89 p. 462
\tptfrzpnt & Freezing point of water & 273.15 & K & Bol80 \\[0.5ex]
\tpttrppnt & Triple point of water & 273.16 & K & Bol80 \\[0.5ex]
\sfctnswtrlqd & 
\footnote{See (\ref{eqn:tns_sfc_PrK78}) for temperature dependence}%
Surface tension of liquid water & 7.610 \times 10^{-3} &
\Nxm, \jxmS & PrK78 \\[0.5ex]
\mmwHdO & 
\footnote{Based on isotopic composition of species in Earth's atmosphere}%
Mean molecular weight of \HdO & 1.8015259 \times 10^{-2} & \kgxmol &
RRG98 \\[0.5ex]
\dnswtr & Density of liquid water & 1000.0 & \kgxmC & \\
\hline
\end{tabular}
\end{center}
\end{minipage}
\end{table}

\section[Common Chemical Conversions]{Common Chemical Conversions}\label{sxn:ccc}
Implementation of chemistry in numerical models often requires
conversion between various quantities related to the concentration of
chemical species.
Table~\ref{tbl:ccc} summarizes these conversions. 
% \cite{SeP97} p. 20 provides another summary
\begin{table}
\begin{minipage}{\hsize} % Minipage necessary for footnotes KoD95 p. 110 (4.10.4)
\renewcommand{\footnoterule}{\rule{\hsize}{0.0cm}\vspace{-0.0cm}} % KoD95 p. 111
\begin{center}
\caption[Concentration Conversion Table]{\textbf{Conversion Factors
Describing Species Abundances}%
\footnote{
The formulae are written in terms of a generic species \A.
The formulae use conventional notation where $\prs$ is the
ambient pressure, $\tptvrt$ is virtual temperature, $\dnsatm =
\prs/(\gascstdryair \tptvrt)$ is mass density, $\gascstdryair$ is the
gas constant of dry air, $\mmw$ is a mean molecular weight, and
$\cstAvagadro$ is Avagadro's number.}% 
\label{tbl:ccc}}
\vspace{\cpthdrhlnskp}
\begin{tabular}{ >{\raggedright}p{10.0em}<{} >{$\dpysty}c<{$}
>{\raggedright}p{8.0em}<{} *{5}{>{$\dpysty}c<{$}} } % KoD95 p. 94 describes '*' notation
\hline \rule{0.0ex}{\hlntblhdrskp}% 
Name & \mbox{Symbol(s)} & Units & \mmr & \vmr & \ppr & \dns \\[0.0ex]
\hline \rule{0.0ex}{\hlntblntrskp}%
Number concentration & \cncA, [\A] & \mlcxmC & \frac{\mmrA \dnsatm \cstAvagadro \cncA}{\mmwA } & & & & \\[3.0ex]
Mass mixing ratio & \mmrA & \kgxkg & \frac{\cncA \mmwA}{\dnsatm \cstAvagadro } & & & & \\[3.0ex]
Volume mixing ratio, Number mixing ratio, Mole fraction \newline Volume fraction & \vmrA & \molxmol & \frac{\mmrA \mmwdryair}{\mmwA } & & & & \\[3.0ex]
Partial pressure & \pprA & Pa & \dnsA \gascstA \tpt & & & & \\[4.0ex]
Mass density & \dnsA & \kgxmC & & & & & \\[3.0ex]
\hline
\end{tabular}
\end{center}
\end{minipage}
\end{table}

\section[Surface Tension of Water]{Surface Tension of Water}\label{sxn:tns_sfc}
The boundary between two adjoining thermodynamic phases a transitional
region only a few molecules thick.
The distinct properties of the substance on either side of the
boundary may cause discrete jumps in thermodynamic properties across
the boundary, even when the phases are in equilibrium.
In fact, the boundary surface itself behaves as a thermodynamically
distinct phase of the substance with its own internal energy.
The internal energy of this \trmdfn{surface phase} depends on the 
the area of the boundary surface, $\sfcbnd$.
Because the surface phase is infinitesimally thin, the change in
internal energy due to pressure-volume work which applies to other
phases (i.e., liquid, solid, or vapor) of the substance does not apply
to the surface phase.
Instead, the conjugate pair of variables describing the isentropic
change in energy as the boundary surface changes are the
\trmdfn{surface tension}, $\sfctns$, and $\sfcbnd$.
These conjugate variables are defined such that the change in internal
energy of the system at constant entropy is the product of the surface
tension and the boundary area
\begin{eqnarray}
\dfr\nrgtrn & = & \sfctns \,\dfr\sfcbnd
\label{eqn:nrg_trn_dfn}
\end{eqnarray}
where $\sfctns$ is a force per unit length (\pxm) or energy per unit
area (\jxmS).
Although (\ref{eqn:nrg_trn_dfn}) is valid for any surface phase
species, we shall now restrict our attention to liquid water.
The surface tension of liquid water $\sfctnswtrlqd$ is positive
definite. 
Thus work must be performed to increase $\sfcbnd$.

The underlying cause of surface tension effects is the polar nature of
the water molecule.
Molecules internal to a liquid phase droplet experience a symmetric,
attractive force field generated by the dipole moments of the
surrounding molecules.
Molecules in the surface phase, on the other hand, have no neighbors
across the boundary (in the vapor phase) and so experience a net
attraction towards the center of the droplet. 
This inward attraction maintains the surface tension of the droplet.

A system consisting of the liquid, vapor, and surface phases of water
obeys the standard thermodynamic rules for equilibrium.
In other words, the Gibbs free energy must be a minimum

The pressure difference between the liquid water in the droplet and
the vapor phases outside the droplet is thus 
\begin{eqnarray}
% PrK78 104 
\prsxcs & = & \frac{2 \sfctnswtrlqd}{\rds}
\label{eqn:prs_xcs_dfn}
\end{eqnarray}
Thus the \trmdfn{pressure excess} varies inversely with the size of
the droplet. 
As the droplets become so large that $\rds \rightarrow \infty$,
$\prsxcs \rightarrow 0$.
Thus the pressure excess over a bulk planar surface of water (e.g.,
the ocean) is zero.
For a droplet of size $\rds = 1$\,\um, the pressure excess is about
\ldots. 

The surface tension of water has a moderate temperature dependence,
which appears to be linear to within the uncertainty of the
observations \cite[][p. 104]{PrK78}.
\begin{eqnarray}
% PrK78 p. 104 (5-12)
\sfctnswtrlqd & = & 7.610 \times 10^{-3} - 1.55 \times 10^{-5} ( \tpt - 273.15 ) \qquad 243 < \tpt < 313 \mbox{\,\K}
\label{eqn:tns_sfc_PrK78}
\end{eqnarray}
where $\sfctnswtrlqd$ is in \Nxm\ or \jxmS.

\section{Atmospheric Viscosity}\label{sxn:vsc}
Viscosity is a key quantity in all fluid mechanics because it
determines the efficiency of diffusive relative to advective
processes. 
The \trmdfn{dynamic viscosity} $\vscdyn$ appears directly in the
momentum equations (\ref{eqn:elr_eom_vec}).
The observed linear relationship between the shear stress in a fluid
and its velocity gradient is known as \trmdfn{Newton's law of
friction}. 
\begin{equation}
% Kun90 p. 7
\wndstr = \vscdyn \frac{\dfr\uuu}{\dfr\hgt}
\label{eqn:nwt_lof}
\end{equation}
Thus $\vscdyn$ is the ratio of the shear stress to the velocity 
gradient of the fluid \cite[][p.~7]{Kun90}.

For ideal gases $\vscdyn$ is largely determined by the momentum
transfer due to molecular collisions.
This momentum is proportional to the molecular velocity
(\ref{eqn:vlc_mlc_avg_dfn}). 
For cloud and aerosol processes we need to specify $\mu$ in regions
where collisional processes ensure rapid equipartition of energy.
In such regions the mean molecular velocity varies quadratically with
temperature $m\vvv^{2} \propto \wblshp \tpt$.
Thus $\mu$ depends only on $\tpt$, being roughly proportional to
$\sqrt{\tpt}$. 
For Earth's atmosphere an approximate expression for $\vscdyn$ is 
\cite[][p. 102]{RoY94}.
\begin{equation}
% [kg m-1 s-1] RoY94 p. 102, see also PrK78 p. 323 (10-107), and FTV89
\vscdyn = 1.72 \times 10^{-5} \left( \frac{\tpt}{273 } \right)^{3/2}
\frac{393}{\tpt + 120 } 
\label{eqn:vsc_dyn_dfn}
\end{equation}
The units of $\vscdyn$ are \kgxms.
The scaled equations of motion often make it more convenient to work
with the \trmdfn{kinematic viscosity} $\vscknm$, which is the ratio
of the dynamic viscosity to the density.
Thus $\vscknm$ depends on both $\tpt$ and $\prs$ and is simply defined  
as 
\begin{equation}
\vscknm \equiv \vscdyn / \dnsatm
\label{eqn:vsc_knm_dfn}
\end{equation}
The units of $\vscknm$ are \mSxs.
Near the surface in arid regions, typical values of $\vscdyn$ and 
$\vscknm$ are $1.7 \times 10^{-5}$\,\kgxms\ and 
$1.3 \times 10^{-5}$\,\mSxs, respectively. 

\section[Error Function]{Error Function}\label{sxn:erf}
% NB: This section should be merged in from psd
The \trmdfn{error function} $\erfxxx$ may be defined as the partial
integral of a Gaussian curve
\begin{equation}
% PTV96 p. 213 (6.2.8)
\erfzzz = \frac{2}{\sqrt{\mpi}} \int_{0}^{\zzz} \me^{-\xxx^{2}} \,\dfr\xxx
\label{eqn:erf}
\end{equation}
The error function is bounded by the limits $\erffnc(0) = 0$ and
$\erffnc(\infty) = 1$. 
The \trmdfn{complementary error function} is defined eponymously as 
$\cerfxxx = 1 - \erfxxx$.

\section[Gamma Function]{Gamma Function}\label{sxn:gmm}
The \trmdfn{gamma function} $\gmmfnc(\xxx)$ is defined by
\begin{equation}
% PTV03 p. 218 (6.1.1)
\gmmfnc(\xxx) = \int_{0}^{\infty} \ttt^{\xxx-1} \me^{-\ttt} \,\dfr\ttt
\label{eqn:gmm_fnc_dfn}
\end{equation}
Useful special values of the gamma function include
\begin{eqnarray}
% PTV03 p. 218 (6.1.1)
\gmmfnc({\textstyle{\frac{1}{2}}}) & = & \sqrt{\mpi} \\
\gmmfnc(\nnn+1) & = & \nnn! \qquad\mbox{where $\nnn \in \cal{I}$}
\label{eqn:gmm_fnc_dfn}
\end{eqnarray}
where $\nnn$ is an integer.
The recurrence properties of gamma functions are
\begin{eqnarray} 
% Gre78 p. 57 (3.20), p. 59 (3.26)
\gmmfnc(\zzz) & = & (\zzz - 1)\gmmfnc(\zzz - 1) \qquad \zzz > 1
\nonumber \\
\gmmfnc(\zzz) & = & \frac{\gmmfnc(\zzz + 1)}{\zzz} \qquad \zzz < 0
\nonumber
\label{eqn:gmm_rcr_dfn}
\end{eqnarray}
Thus we see that 
\begin{eqnarray} 
\gmmfnc({\textstyle{-\frac{1}{2}}}) & = & 
\frac{\gmmfnc({\textstyle{\frac{1}{2}}})}{{\textstyle{-\frac{1}{2}}}} 
\nonumber \\
& = & - 2 \gmmfnc({\textstyle{\frac{1}{2}}}) \nonumber \\
& = & - 2 \sqrt{\mpi} 
\label{eqn:gmm_hlf_dfn}
\end{eqnarray}

\section[Incomplete Gamma Functions]{Incomplete Gamma Functions}\label{sxn:gmm_inc}
The four members of the Incomplete Gamm Function family are
$\gmmfnc(\alpha,\xxx)$, $\gamma(\alpha,\xxx)$, $\PPP(\alpha,\xxx)$,
and $\QQQ(\alpha,\xxx)$ 
\cite[][p.~209]{AbS64,PTV96}.
\begin{eqnarray}
\PPP(\alpha,\xxx) & \equiv & 
1-\QQQ(\alpha,\xxx) \equiv \frac{\gamma(\alpha,\xxx)}{\gmmfnc(\alpha)} \equiv
\frac{1}{\gmmfnc(\alpha)} \int_{0}^{\xxx} 
\ttt^{\alpha - 1} \me^{-\ttt} \,\dfr\ttt \nonumber \\
\QQQ(\alpha,\xxx) & \equiv & 1-\PPP(\alpha,\xxx) \equiv
\frac{\gmmfnc(\alpha,\xxx)}{\gmmfnc(\alpha)} \equiv
\frac{1}{\gmmfnc(\alpha)} \int_{\xxx}^{\infty} 
\ttt^{\alpha - 1} \me^{-\ttt} \,\dfr\ttt \nonumber \\
\gamma(\alpha,\xxx) & \equiv & \gmmfnc(\alpha) \PPP(\alpha,\xxx) \equiv
\gmmfnc(\alpha) [1-\QQQ(\alpha,\xxx)] \equiv
\gmmfnc(\alpha) - \gmmfnc(\alpha,\xxx) \equiv
\int_{0}^{\xxx} \ttt^{\alpha - 1} \me^{-\ttt} \,\dfr\ttt \nonumber \\
\gmmfnc(\alpha,\xxx) & \equiv & \gmmfnc(\alpha) \QQQ(\alpha,\xxx) \equiv
\gmmfnc(\alpha) [1-\PPP(\alpha,\xxx)] \equiv 
\gmmfnc(\alpha) - \gamma(\alpha,\xxx) \equiv 
\int_{\xxx}^{\infty} \ttt^{\alpha - 1} \me^{-\ttt} \,\dfr\ttt \nonumber \\
\label{eqn:gmm_fnc_inc_dfn}
\end{eqnarray}
The nomenclature for incomplete gamma functions is confusing.
The most precise name for $\gmmfnc(\alpha,\xxx)$ is the
``upper incomplete gamma function''.
Here ``upper'' refers to both ``upper''-case gamma and to the bounds
of integration, $\xxx \rightarrow \infty$.
One important reference, MathWorld, calls $\gmmfnc(\alpha,\xxx)$ the
``incomplete gamma function'' but we avoid that terminology.
The upper incomplete gamma function $\gmmfnc(\alpha,\xxx)$ reduces to
the regular (or complete) gamma function $\gmmfnc(\alpha)$ when 
$\xxx = 0$ \cite[e.g.,][]{AbS64}.

The most precise name for $\gamma(\alpha,\xxx)$ is the
``lower incomplete gamma function'' for self-evident reasons.
The upper and lower incomplete gamma functions functions satisfy 
\begin{eqnarray}
\gmmfnc(\alpha,\xxx) + \gamma(\alpha,\xxx) & = & \gmmfnc(\alpha)
\label{eqn:Gmm_gmm_fnc_dfn}
\end{eqnarray}
Equation (\ref{eqn:Gmm_gmm_fnc_dfn}) demonstrates the property for
which the ``incomplete gamma functions'' are named.

The \trmdfn{regularized gamma functions}
$\PPP(\alpha,\xxx)$ and $\QQQ(\alpha,\xxx)$ satisfy
\begin{eqnarray}
\PPP(\alpha,\xxx) + \QQQ(\alpha,\xxx) & = & 1
\label{eqn:gmm_fnc_rgl_dfn}
\end{eqnarray}
The regularized functions $\PPP(\alpha,\xxx)$ and $\QQQ(\alpha,\xxx)$
functions are the lower and upper incomplete gamma functions
$\gamma(\alpha,\xxx)$ and $\gmmfnc(\alpha,\xxx)$, respectively,
normalized by the complete gamma function~$\gmmfnc(\alpha)$.
Computing $\PPP(\alpha,\xxx)$ and $\QQQ(\alpha,\xxx)$ may be more
numerically tractable than computing $\gamma(\alpha,\xxx)$ and
$\gmmfnc(\alpha,\xxx)$ when $\gmmfnc(\alpha)$ is large.
Most references call $\PPP(\alpha,\xxx)$ the
``the incomplete gamma function'' (which is somewhat ambiguous) or, 
more precisely, ``the regularized incomplete gamma function''
\cite[e.g.,][]{AbS64,PTV96}.
Most references call $\QQQ(\alpha,\xxx)$ 
``the complemented regularized incomplete gamma function''.
For these reasons, we try to avoid the generic and ambiguous
terminology ``the incomplete gamma function''.
One must be very careful to check which form of ``the incomplete
gamma function'' numerical libraries implement against the defining
relations (\ref{eqn:gmm_fnc_inc_dfn}).

\section{Marshall-Palmer Distribution}\label{sxn:mrsplm}
The Marshall-Palmer distribution function is described in terms of the 
the mean diameter $\dmtnwa$\,\mm, the total number distribution
$\cncttl$\,\mCxmm, and the precipitation rate $\RRR$\,\mmxhr\ as follows 
\begin{equation}
% PrK98 p. 34 (2-15)
\pdfmrsplm(\dmt) = \cncttl \exp (-\Lambda \dmtnwa)
\label{eqn:mrsplm_dst_dfn}
\end{equation}
where $\cncttl = 8 \times 10^{3}$\,\xmm\ and $\Lambda
= 4.1\RRR^{-0.21}$\,\xmm\ were determined empirically.  

\section{Rayleigh Distributions}\label{sxn:ryl}
A single parameter distribution which has been used to approximate
wind speed observations in nature is the \trmdfn{Rayleigh distribution}.
The Rayleigh distribution function is 
\begin{equation}
% GSL 
\pdfryl(\wndspd) = \frac{\wndspd}{\sigma^{2}} 
\exp \left( \frac{\aaa^{2} - \wndspd^{2}}{2 \sigma^{2} } \right)
\label{eqn:ryl_dst_dfn}
\end{equation}
where $\sigma$ is the \trmdfn{scale parameter}.

The cumulative distribution function of (\ref{eqn:ryl_dst_dfn}) is 
\begin{equation}
% GSL 
\int_{0}^{\wndspd} \pdfryl(\wndspd) \,\dfr\wndspd = 
1 - \exp \left( - \frac{\wndspd^{2}}{2 \sigma^{2}} \right)
\label{eqn:cdf_ryl_dfn}
\end{equation}

Related to (\ref{eqn:ryl_dst_dfn}) is the \trmdfn{Rayleigh tail
distribution}. 
The tail obeys (\ref{eqn:ryl_dst_dfn}) for $\wndspd > \wndspdthr$, i.e., 
\begin{eqnarray}
% GSL 
\pdfryltail(\wndspd) & = & \left\{
\begin{array}{l@{\quad:\quad}r}
0 & \wndspd < \wndspdthr \\ 
\displaystyle 
\frac{\wndspd}{\sigma^{2}} \exp \left( 
\frac{\wndspdthr^{2} - \wndspd^{2}}{2\sigma^{2}} 
\right) & \wndspd \ge \wndspdthr
\end{array} \right.
\label{eqn:ryltail_dst_dfn}
\end{eqnarray}
where $\wndspdthr$ is the lower limit of the distribution.
Thus (\ref{eqn:ryltail_dst_dfn}) is a truncated Rayleigh distribution
which is normalized for wind speeds $\wndspd$ greater than a threshold
$\wndspdthr$.

The cumulative distribution function of (\ref{eqn:ryltail_dst_dfn}) is 
\begin{equation}
% GSL 
\int_{0}^{\wndspd} \pdfryltail(\wndspd) \,\dfr\wndspd = 
1 - \exp \left( \frac{\wndspdthr^{2} - \wndspd^{2}}{2 \sigma^{2} } \right)
\label{eqn:cdf_ryltail_dfn}
\end{equation}

\section{Weibull Distribution}\label{sxn:wbl}
The Weibull distribution function is 
\begin{equation}
\pdfwbl(\wndspd) = \frac{\wblshp}{\wblscl} 
\left( \frac{\wndspd}{\wblscl } \right)^{\wblshp - 1} 
\exp \left[ - \left( \frac{\wndspd}{\wblscl } \right)^{\wblshp} \right]
\label{eqn:wbl_dst_dfn}
\end{equation}
where $\wblscl$ and $\wblshp$ are known as the \trmdfn{scale parameter}
and the \trmdfn{shape parameter}, respectively.
The scale parameter determines the maximum value of $\pdfwbl$, while
the shape parameter determines the variance of the distribution.
As will be shown below, (\ref{eqn:wbl_dst_dfn}) is normalized on the
interval $\wndspd \in [0, \infty)$.

For applications to wind erosion, we are interested in higher moments
of the Weibull distribution.
For generality, consider the $\mmnnbr$th moment of the Weibull
distribution, defined by 
\begin{eqnarray}
\pdfwbl(\wndspd,\mmnnbr) & \equiv & \wndspd^{\mmnnbr} \pdfwbl(\wndspd) \\
& = & \wndspd^{\mmnnbr} \left( \frac{\wblshp}{\wblscl} \right)
\left( \frac{\wndspd}{\wblscl } \right)^{\wblshp - 1} 
\exp \left[ - \left( \frac{\wndspd}{\wblscl } \right)^{\wblshp} \right]
\nonumber \\
& = & \wblshp \wndspd^{\mmnnbr+\wblshp-1} \wblscl^{\wblshp - 2} 
\exp \left[ - \left( \frac{\wndspd}{\wblscl } \right)^{\wblshp} \right]
\label{eqn:wbl_mmn_dfn}
\end{eqnarray}
Hence $\pdfwbl(\wndspd,\mmnnbr)$ reduces to $\pdfwbl(\wndspd)$ 
for $\mmnnbr = 0$.

We define the cumulative PDF of the $\mmnnbr$th moment of the Weibull 
distribution as
\begin{eqnarray}
\wblfnc(\wndspdthr,\mmnnbr) & \equiv &
\int_{\wndspdthr}^{\infty} \pdfwbl(\wndspd,\mmnnbr) \,\dfr\wndspd = 
\int_{\wndspdthr}^{\infty} \wndspd^{\mmnnbr} \pdfwbl(\wndspd) \,\dfr\wndspd \\
& = &
\int_{\wndspdthr}^{\infty} 
\wndspd^{\mmnnbr} \left( \frac{\wblshp}{\wblscl} \right)
\left( \frac{\wndspd}{\wblscl } \right)^{\wblshp - 1} 
\exp \left[ - \left( \frac{\wndspd}{\wblscl } \right)^{\wblshp} \right]
\,\dfr\wndspd
\label{eqn:wbl_mmn_cml_dfn}
\end{eqnarray}
The limits of integration in (\ref{eqn:wbl_mmn_cml_dfn}) ensure that 
$\pdfwbl(\wndspd,\mmnnbr)$ contributes to
$\wblfnc(\wndspdthr,\mmnnbr)$ only for $\wndspd > \wndspdthr$.
Hence $\wndspdthr$ is called the threshold wind speed.
For $\mmnnbr = 0$, (\ref{eqn:wbl_mmn_dfn}) gives the probability that 
wind speed $\wndspd$ exceeds the threshold windspeed $\wndspdthr$.
For $\mmnnbr > 0$, $\wblfnc(\wndspdthr,\mmnnbr)$ defines statisical
moments of the wind speeds greater (faster) than~$\wndspdthr$.
For example, $\mmnnbr = 1$ defines the mean wind speed faster
than~$\wndspdthr$.
The mean wind speed is proportional to the mean momentum transport, a
useful quantity to know because it is conserved. 
The cumulative PDF of the second moment ($\mmnnbr = 2$) defines the
mean square wind speed exceeding~$\wndspdthr$.
The mean square wind speed is proportional to the kinetic energy,
another conserved quantity.

Defining $\wblfnc(\wndspdthr,\mmnnbr)$ (\ref{eqn:wbl_mmn_dfn}) to
include $\wndspd > \wndspdthr$ rather than the complementary 
convention $\wndspd < \wndspdthr$ is convenient for threshold
phenomena.   
The complementary (opposite) convention ($\wndspd < \wndspdthr$) is
appropriate for other processes and is often employed.
One must be careful to understand the sense (i.e., the convention) of
a cumulative PDF function before employing it.  

To simplify (\ref{eqn:wbl_mmn_cml_dfn}) we make the change of
variables  
\begin{eqnarray}
\xxx & = & (\wndspd/\wblscl)^{\wblshp} \nonumber \\
(\wndspd / \wblscl )^{(\wblshp - 1)} & = & \xxx^{(\wblshp - 1)/\wblshp} \nonumber \\
\wndspd^{\mmnnbr} & = & \wblscl^{\mmnnbr} \xxx^{\mmnnbr / \wblshp} \nonumber \\
\dfr\wndspd & = & \wblscl \wblshp^{-1} \xxx^{-(\wblshp - 1)/\wblshp} \nonumber \\
\dfr\xxx & = & \wblshp \wblscl^{-1} \left( \frac{\wndspd}{\wblscl}
\right)^{\wblshp - 1} 
\label{eqn:cov_x}
\end{eqnarray}
This change of variables maps 
$\wndspd \in [\wndspdthr,\infty)$ to
$\xxx \in [\xxxthr,\infty)$ where we have defined
$\xxxthr = (\wndspdthr/\wblscl)^{\wblshp}$.
Substituting this into (\ref{eqn:wbl_mmn_cml_dfn}) we obtain
\begin{eqnarray}
\wblfnc(\wndspdthr,\mmnnbr) & = &
\int_{\xxxthr}^{\infty} 
\wblscl^{\mmnnbr} \xxx^{\mmnnbr / \wblshp} \wblshp \wblscl^{-1} \xxx^{(\wblshp - 1)/\wblshp}
\me^{-\xxx} \wblscl \wblshp^{-1} \xxx^{-(\wblshp - 1)/\wblshp} \,\dfr\xxx
\nonumber \\
& = &
\wblscl^{\mmnnbr} \int_{\xxxthr}^{\infty} 
\xxx^{\mmnnbr / \wblshp} \me^{-\xxx} \,\dfr\xxx
\nonumber \\
& = &
\wblscl^{\mmnnbr} \int_{\xxxthr}^{\infty} 
\xxx^{\alpha - 1} \me^{-\xxx} \,\dfr\xxx
\label{eqn:wbl_dst_mmn}
\end{eqnarray}
where we defined $\alpha = 1 + \mmnnbr/\wblshp$ in the last step.
The integral expression on the RHS of (\ref{eqn:wbl_dst_mmn}) is
the \trmdfn{upper incomplete gamma function} $\gmmfnc(\alpha,\xxxthr)$.
Section~\ref{sxn:gmm_inc} describes incomplete gamma function
properties. 

Rewriting (\ref{eqn:wbl_dst_mmn}) in terms of
(\ref{eqn:gmm_fnc_inc_dfn}) we obtain
\newline\parbox{6in}{ % KoD95 p. 138
\begin{eqnarray*}
\wblfnc(\wndspdthr,\mmnnbr) & = &
\wblscl^{\mmnnbr} \gmmfnc(\alpha,\xxxthr) \\
& = &
\wblscl^{\mmnnbr} \gmmfnc \left( \frac{\wblshp + \mmnnbr}{\wblshp}, \left( \frac{\wndspdthr
}{\wblscl} \right)^{\wblshp} \right) 
\label{eqn:wbl_gmm_fnc_inc_dfn}
\end{eqnarray*}
}\hfill % end parbox KoD95 p. 138
\parbox{1cm}{\begin{eqnarray}\label{eqn:wbl_gmm_dfn}\end{eqnarray}}\newline
Thus the moments of the Weibull distribution are given by the
gamma function when $\wndspdthr = 0$\,\mxs\ and by the (upper)
incomplete gamma function when $\wndspdthr > 0$\,\mxs.  

The cumulative distribution function of the full Weibull PDF
(\ref{eqn:wbl_dst_dfn}) is solved by setting $\wndspdthr = 0$\,\mxs\
and $\mmnnbr = 0$ in (\ref{eqn:wbl_gmm_fnc_inc_dfn})  
\begin{eqnarray}
\wblfnc(\wndspdthr = 0,\mmnnbr = 0) & = &
\wblscl^{0} \gmmfnc (1,0) = \gmmfnc(1) = 1
\label{eqn:wbl_dst_nrm}
\end{eqnarray}
Hence (\ref{eqn:wbl_dst_nrm}) verifies that the Weibull PDF
(\ref{eqn:wbl_dst_dfn}) correctly normalizes to one when integrated
over all wind speeds.

Equation~(\ref{eqn:wbl_gmm_dfn}) makes a number of useful wind speed
statistics readily available.
First, the probability of the wind speed exceeding the threshold wind
speed is
\begin{eqnarray}
% JHM78 p. 350 (2)
\prbfnc(\wndspd > \wndspdthr) = \wblfnc(\wndspdthr,0) 
& = & \gmmfnc (1, \xxxthr) \nonumber \\
& = & \int_{\xxxthr}^{\infty} \me^{-\xxx} \,\dfr \xxx \nonumber \\
& = & \me^{-\xxxthr} \nonumber \\
& = & \exp \left[- \left( \frac{\wndspdthr}{\wblscl} \right)^{\wblshp} \right]
\label{eqn:prb_dfn}
\end{eqnarray}
Of course 
$\prbfnc(\wndspd < \wndspdthr) = 1 - \prbfnc(\wndspd > \wndspdthr)$.
In a Weibull distribution the frequency of occurrence of winds
exceeding $\wndspdthr$ decreases exponentially with $\wndspdthr$.
Thus the Weibull distribution produces wind speed statistics analogous
to the Boltzmann statistics of energy levels.

The windspeed $\wndspdidx$ such that a given fraction $\prbidx$ of
the Weibull PDF exceeds $\wndspdidx$ is obtained by inverting
(\ref{eqn:prb_dfn})
\begin{eqnarray}
\exp \left[- \left( \frac{\wndspdidx}{\wblscl} \right)^{\wblshp} \right] & = & \prbidx \nonumber \\
- \left( \frac{\wndspdidx}{\wblscl} \right)^{\wblshp} & = & \ln \prbidx \nonumber \\
- \left( \frac{\wndspdidx}{\wblscl} \right) & = & (\ln \prbidx)^{1/\wblshp} \nonumber \\
\wndspdidx & = & \wblscl (-\ln \prbidx)^{1/\wblshp}
\label{eqn:wnd_prb_wbl_dfn}
\end{eqnarray}
According to (\ref{eqn:wnd_prb_wbl_dfn}) the median wind speed~\wndspdmdn\ is 
\begin{eqnarray}
\wndspdmdn & \equiv & \wndspd(\prbidx = 0.5) = \wblscl (\ln 2)^{1/\wblshp}
\label{eqn:wnd_mdn_wbl_dfn}
\end{eqnarray}

The optimal sampling and weighting of the Weibull PDF
(\ref{eqn:wbl_dst_dfn}) (or any PDF) depends on the purpose.
Relation (\ref{eqn:wnd_prb_wbl_dfn}) can be used to sample (and weight)
the Weibull PDF itself (\ref{eqn:wbl_dst_dfn}), not higher order
moments (\ref{eqn:wbl_mmn_dfn}).
Wind erosion integrals (\ref{eqn:flx_hrz_pdf}) depend on quadratic,
cubic, and quartic, wind speed moments, e.g.,
(\ref{eqn:flx_hrz_dfn_2}), (\ref{eqn:flx_hrz_dfn_3}),
(\ref{eqn:flx_hrz_bgn_3}), and (\ref{eqn:flx_mss_hrz_slt_Whi79}).  
Accurate evaluation of such integrals should optimally sample the
probability distributions of the higher order moments. 

The mean wind speed~$\wndspdbar$ is, by definition, the first moment
($\mmnnbr = 1$) of the entire Weibull distribution
(\ref{eqn:wbl_gmm_dfn}) 
\begin{equation}
\wndspdbar \equiv \wblfnc(\wndspdthr = 0,\mmnnbr = 1) = \wblscl [\gmmfnc (1 + 1/\wblshp)]
\label{eqn:wnd_spd_avg_wbl_dfn}
\end{equation}

\subsection{Truncated Weibull Distributions}\label{sxn:wbl_trc}
If a PDF and cumulative distribution are both expressible in closed
form, then it is straightforward to determine statistical properties
of the \trmidx{truncated PDF}, i.e., arbitrarily bounded segments of
the PDF. 
For the Weibull PDF (\ref{eqn:wbl_dst_dfn}) and cumulative PDF
(\ref{eqn:wbl_mmn_cml_dfn}), it is often useful to know the
statistical properties between two wind speeds defined such that
$\wndspdone < \wndspdtwo$. 
The linearity of integrals applied to (\ref{eqn:wbl_mmn_cml_dfn})
allows us to immediately express the probability that 
$\wndspdone < \wndspd < \wndspdtwo$
\begin{eqnarray}
\label{eqn:prb_wbl_trc_dfn}
\wblfnc(\wndspdone,\wndspdtwo;0) \equiv 
\prbfnc(\wndspdone < \wndspd < \wndspdtwo) 
& = & 
\wblfnc(\wndspdone,0) - \wblfnc(\wndspdtwo,0) \\
& = & 
\exp \left[- \left( \frac{\wndspdone}{\wblscl} \right)^{\wblshp} \right] -
\exp \left[- \left( \frac{\wndspdtwo}{\wblscl} \right)^{\wblshp} \right]
\label{eqn:prb_wbl_trc_dfn_cls}
\end{eqnarray}
where we substituted (\ref{eqn:prb_dfn}) to obtain the closed form in 
the final step.
Similarly, the normalization property (\ref{eqn:wbl_dst_nrm}) allows
us to determine the mean wind speed $\wndspdbar$ of any Weibull PDF
truncated to $[\wndspdone,\wndspdtwo]$ 
\begin{eqnarray}
\label{eqn:wnd_spd_avg_wbl_trc_dfn}
\wndspdbar = \wblfnc(\wndspdone,\wndspdtwo;1) 
& = & 
\frac{\wblfnc(\wndspdone,1) - \wblfnc(\wndspdtwo,1)}
{\wblfnc(\wndspdone,0) - \wblfnc(\wndspdtwo,0)} \\
& = & 
\wblscl \times 
\left[
\gmmfnc \left( \frac{\wblshp + 1}{\wblshp}, \left( \frac{\wndspdone}{\wblscl} \right)^{\wblshp} \right) 
-
\gmmfnc \left( \frac{\wblshp + 1}{\wblshp}, \left( \frac{\wndspdtwo}{\wblscl} \right)^{\wblshp} \right) 
\right]
\nonumber \\ & & {} % KoD99 p. 138 for spacing info
\times
\left\{ \exp \left[- \left( \frac{\wndspdone}{\wblscl} \right)^{\wblshp} \right] -
\exp \left[- \left( \frac{\wndspdtwo}{\wblscl} \right)^{\wblshp} \right] \right\}^{-1}
\label{eqn:wnd_spd_avg_wbl_trc_dfn_cls}
\end{eqnarray}
where we substituted $\mmnnbr = 1$ into (\ref{eqn:wbl_gmm_fnc_inc_dfn})
to obtain the closed form in (\ref{eqn:wnd_spd_avg_wbl_trc_dfn_cls}). 
The final term is the inverse of (\ref{eqn:prb_wbl_trc_dfn_cls}) and 
normalizes the mean to the probability contained in the truncated
region.  

\subsection{Wind Speed Observations}\label{sxn:wbl_obs}
In atmospheric studies $\wndspdbar$ is often a predicted or observed
quantity from which we obtain the \trmidx{scale parameter}
\begin{equation}
% JHM78 p. 351 (16), Zen01c p. 169 (17.20)
\wblscl = \wndspdbar [\gmmfnc (1 + 1/\wblshp)]^{-1}
\label{eqn:wbl_scl_dfn}
\end{equation}
\cite{GiP88} noted that $\gmmfnc (1 + 1/\wblshp) \approx 0.9$ over
the usual range of $\wblshp$ values.
Thus typical values of the scale parameter are 
$\wblscl \approx 1.1 \wndspdbar$\,\mxs.

\cite{JHM78} examined hourly surface wind speeds at a site in the
continental United States.
They found an empirical relationship between the shape parameter 
and the mean wind speed
\begin{equation}
% JHM78 p. 352 (20), Zen01c p. 169 (17.21)
\wblshp = \wblshpvarprm \sqrt{\wndspdbar}
\label{eqn:wbl_shp_dfn}
\end{equation}
where $\wndspdbar$ is in \mxs.
% fxm: I don't know exactly what this means
They used the normalized standard deviation of the measured wind speed
distribution, ($\sigma/\wndspdbar$) to classify sites as low
(10~percentile), average, or high (90~percentile) variability.  
For low, average, and high variability winds $\wblshpvarprm$ is
approximately 1.05, 0.94, and~0.83, respectively. 
We assume the surface winds are highly variable during dust events
(i.e., when $\wndspd > \wndspdthr$) and employ $\wblshpvarprm = 0.83$.

The shape parameter $\wblshpmdp$ may be transferred from reference
height $\hgtrfr$ (e.g., 10\,\m) to some other height $\hgtmdp$ using
\cite[]{JHM78}  
\begin{equation}
% JHM78 p. 351 (8) GrZ04 
\wblshpmdp=\wblshprfr[1-0.088\ln(\hgtrfr/10)]/[1-0.088\ln(\hgtmdp/10)] 
\label{eqn:wbl_shp_mdp_dfn}
\end{equation}

\section[Aspherical Shapes]{Aspherical Shapes}\label{sxn:asp}
Natural aerosols come in a variety of shapes.
This morphology plays a significant role in thermodynamics, dry
deposition, optics, and chemical uptake.
A fundamental understanding of aerosol physics is best grounded in the
spherical particle assumption.
Practical applications, however, must account for aspherical effects.
% fxm: Write section on effects of non-smooth surfaces
For now we defer a related problem, non-smooth surfaces.

\subsection[Cylinders]{Cylinders}\label{sxn:cyl}
Few aspherical shapes are analytically tractible.
As a result, cylinders are the most studied variant.

\subsection[Ellipsoids]{Ellipsoids}\label{sxn:lps}
The most convenient, aspherical shape without corners is the 
\trmdfn{ellipsoid}.
An ellipsoid is a solid of revolution generated by rotating an
ellipse about one its \trmdfn{major axis}~$\dmtmjr$ to form an
\trmdfn{oblate ellipsoid}, or about its minor axis~$\dmtmnr$
to form a \trmdfn{prolate ellipsoid}.

Our discussion of ellipse properties is based on the following
assumptions:
\begin{enumerate}
\item The major axis of an ellipse is denoted the $\mjrsbs$-axis.
  The length of the major axis is $\dmtmjr = 2\rdsmjr = 2\mjrsbs$.
  Rotation about the $\mjrsbs$-axis forms an oblate ellipsoid.
  Then the third axis $\CCC = \mjrsbs$.
\item The minor axis of an ellipsoid is denoted the $\mnrsbs$-axis.
  The length of the minor axis is $\dmtmnr = 2\rdsmnr = 2\mnrsbs$.
  Rotation about the $\mnrsbs$-axis forms a prolate ellipsoid.
  Then the third axis $\CCC = \mnrsbs$.
\end{enumerate}
With these conventions, the eccentricity~$\xcnlps$ surface
area~$\sfclps$ of the ellipsoid are  
\begin{eqnarray}
% [frc] Gin03 p. 2 (7), http://mathworld.wolfram.com/ProlateSpheroid.html
\label{eqn:xcn_lps_dfn}
\xcnlps & = & \frac{\sqrt{\dmtmjr^{2}-\dmtmnr^{2}}}{\dmtmjr} \\
\label{eqn:sfc_lps_dfn}
% fxm: Re-write in terms of diameter
% fxm: is 2e factor in right place?
\sfclps & = & \frac{2 \mpi}{2 \xcnlps}
\left\{\rdsmjr^{2} + \rdsmnr^{2}
\left[ \log \left( \frac{1 + \xcnlps}{1 - \xcnlps} \right) \right]
\right\} \nonumber \\
& = & 2 \mpi \rdsmnr^{2} + 2 \mpi \frac{\dmtmnr \dmtmjr}{\xcnlps} \sin^{-1} \xcnlps
\end{eqnarray}

\subsection[Hexagonal Prisms]{Hexagonal Prisms}\label{sxn:hxg}
The most common aspherical shape with edges encountered in atmospheric
particles are \trmidx{hexagonal prisms}, a shape found in natural
snow, ice, and mineral dust crystals 
\cite[][]{AuV70,Hey72,Kno72,HeP84,Gin03,NGW03}.
Hexagonal prisms often form from \trmidx{Ice Nuclei} under the right 
temperature and humidity conditions.
We now develop in detail the geometric properties and the V/S
technique (Section~\ref{sxn:vts}) for hexagonal prisms.

Hexagonal prisms prisms are characterized by two lengths.
The prism length $\dmthxg$ is the distance between the opposite
hexagonal faces, also known as the \trmidx{$\ccc$-axis}.
The half-width $\rdshxg$ of basal face of a hexagonal prism is also
the width of its six facets.
The tranverse or basal face of a prism is likewise known as the
\trmidx{$\aaa$-axis}. 

Together $\rdshxg$ and $\dmthxg$ completely characterize the geometry
of the prism.
It is convenient to define the \trmidx{aspect ratio} $\asphxg$ as the 
ratio of the $\ccc$-axis length to the $\aaa$-axis width (the full
width $2\rdshxg$, not the half-width $\rdshxg$):
\begin{eqnarray}
% NGW03 p. 3 (5)
\asphxg & \equiv & \frac{\dmthxg}{2\rdshxg}
\label{eqn:asp_hxg_dfn}
\end{eqnarray}
With this definition, prisms with $\asphxg < 1$ are called 
\trmidx{hexagonal plates} and prisms with $\asphxg > 1$ are called  
\trmidx{hexagonal columns}, respectively.

% This paragraph is copied in appendix sxn:fct in psd.tex:
The surface area $\sfchxg$\,[\mS] and volume $\vlmhxg$\,[\mC] of a 
hexagonal prism are
\begin{eqnarray}
% GrW99 p. 31699 (3)
\sfchxg & = & 6\rdshxg\dmthxg+3\sqrt{3}\rdshxg^{2} = 
              3\rdshxg(2\dmthxg+\sqrt{3}\rdshxg) \\
\label{eqn:sfc_hxg_dfn}
% GrW99 p. 31699 (3) NGW03 p. 3 (4)
\vlmhxg & = & \frac{3\sqrt{3}\rdshxg^{2}\dmthxg}{2}
\label{eqn:vlm_hxg_dfn}
\end{eqnarray}
These definitions may be re-cast in terms of $\rdshxg$ and $\asphxg$
by using $\dmthxg = 2\rdshxg\asphxg$ from (\ref{eqn:asp_hxg_dfn})  
\begin{eqnarray}
\sfchxg & = & 12\rdshxg^{2}\asphxg+3\sqrt{3}\rdshxg^{2} = 
              3\rdshxg^{2}(4\asphxg+\sqrt{3}) \\
\label{eqn:sfc_hxg_asp_dfn}
% NGW03 p. 3 (6c)
\vlmhxg & = & 3\sqrt{3}\rdshxg^{3}\asphxg
\label{eqn:vlm_hxg_asp_dfn}
\end{eqnarray}
It is more convenient to describe hexagons in terms
of ($\rdshxg$,$\asphxg$) than ($\rdshxg$,$\dmthxg$).
This is because natural hexagonal prisms may share similar aspect
ratios over a large range of sizes.

The radius $\rdssfc$ of a sphere of equivalent surface area 
to hexagonal prism is obtained by setting $\sfcsph=\sfchxg$ 
\begin{eqnarray}
\sfcsph & = & \sfchxg \nonumber \\
4\mpi\rdssfc^{2} & = & 3\rdshxg^{2}(4\asphxg+\sqrt{3}) \nonumber \\
\rdssfc & = & \left[ 
\frac{3\rdshxg^{2}(4\asphxg+\sqrt{3})}{4\mpi} \right]^{1/2} \\
& = & \left[ 
\frac{6\rdshxg\dmthxg + 3\sqrt{3}\rdshxg^{2}}{4\mpi} \right]^{1/2} \nonumber
\label{eqn:rds_sfc_hxg_dfn}
\end{eqnarray}
Similarly, we set $\vlmsph=\vlmhxg$ to find the radius $\rdsvlm$ of
spheres of equivalent volume: 
\begin{eqnarray}
\vlmsph & = & \vlmhxg \nonumber \\
\frac{4\mpi\rdsvlm^{3}}{3} & = & 3\sqrt{3}\rdshxg^{3}\asphxg \nonumber \\
\rdsvlm & = & 3\sqrt{3}\rdshxg^{3}\asphxg \nonumber \\
\rdsvlm & = & \left[ 
\frac{9\sqrt{3}\rdshxg^{3}\asphxg}{4\mpi} \right]^{1/3} \\
& = & \left[ 
\frac{9\sqrt{3}\rdshxg^{2}\dmthxg}{8\mpi} \right]^{1/3} \nonumber
\label{eqn:rds_vlm_hxg_dfn}
\end{eqnarray}

The geometry of equal volume-to-area (in particular, surface area)
spheres is described by \cite{GrW99} and \cite{NGW03}.
The radius $\rdsvts$ of the sphere with the same volume-to-total
surface area ratio as a hexagonal prism is determined by
\begin{eqnarray}
% GrW99 p. 31699 (1-4) NGW03 p. 3 (1-8)
\frac{\vlmsph}{\sfcsph} & = & \frac{\vlmhxg}{\sfchxg} \nonumber \\
\left(\frac{4\mpi\rdsvts^{3}}{3}\right)\bigg/\left(4\mpi\rdsvts^{2}\right) & = & 
\frac{3\sqrt{3}\rdshxg^{3}\asphxg}{3\rdshxg^{2}(4\asphxg+\sqrt{3})} \nonumber \\
\frac{\rdsvts}{3} & = & 
\frac{\sqrt{3}\rdshxg\asphxg}{4\asphxg+\sqrt{3}} \nonumber \\
% NGW03 p. 3 (6a)
\rdsvts & = & 
\frac{3\sqrt{3}\rdshxg\asphxg}{4\asphxg+\sqrt{3}} \\
% GrW99 p. 31699 (3)
& = & \frac{3\sqrt{3}\rdshxg\dmthxg}{4\dmthxg+2\sqrt{3}\rdshxg} \nonumber
\label{eqn:rds_vts_hxg_dfn}
\end{eqnarray}

\cite{GrW99} point out that
(\ref{eqn:rds_sfc_hxg_dfn})--(\ref{eqn:rds_vts_hxg_dfn}) apply equally if
cross-sectional area~$\xsa$ is used instead of surface area~$\sfc$.
This is because the ratio $\sfc/\xsa = 4$ for all randomly oriented
convex shapes. 
Since randomly oriented spheres and hexagonal crystals have the same
ratio $\sfc/\xsa=4$, this ratio would factor out of both sides of the 
above derivation yielding $\rdssfc = \rdsxsa$ and $\rdsvts = \rdsvtx$
where $\rdsxsa$ and $\rdsvtx$ are the radii of spheres of equivalent 
cross-sectional area and volume-to-cross-sectional area, respectively.

Following the derivation of (\ref{eqn:cnc_vts_dfn}), we find the
number $\cncvts$ of spheres with equal V/S ratios to hexagonal prisms  
that have the same total volume as $\cnchxg$ hexagonal prisms  
\begin{eqnarray}
\cncvts \vlmvts & = & \cnchxg \vlmhxg \\
\cncvts \frac{4\mpi\rdsvts^{3}}{3} & = & 
\cnchxg 3\sqrt{3}\rdshxg^{3}\asphxg \nonumber \\
\frac{\cncvts}{\cnchxg} & = & 
\frac{9\sqrt{3}\rdshxg^{3}\asphxg}{4\mpi\rdsvts^{3}} \nonumber \\
& = & 
\frac{9\sqrt{3}\rdshxg^{3}\asphxg}{4\mpi} 
\left( \frac{4\asphxg+\sqrt{3}}{3\sqrt{3}\rdshxg\asphxg} \right)^{3} \nonumber \\
& = & 
\frac{9\sqrt{3}\rdshxg^{3}\asphxg}{4\mpi} \times
\frac{(4\asphxg+\sqrt{3})^{3}}{81\sqrt{3}\rdshxg^{3}\asphxg^{3}} \nonumber \\
& = & 
\frac{1}{4\mpi} \times
\frac{(4\asphxg+\sqrt{3})^{3}}{9\asphxg^{2}} \nonumber \\
% NGW03 p. 3 (6b)
& = & 
\frac{(4\asphxg+\sqrt{3})^{3}}{36\mpi\asphxg^{2}}
\label{eqn:cnc_vts_hxg_dfn}
\end{eqnarray}
\cite{NGW03} plot the behavior of (\ref{eqn:cnc_vts_hxg_dfn}) and show
that the minimum number of V/S-spheres per hexagonal crystal is
$\cncvts = 1.65$ which occurs at aspect ratio $\asphxg = 0.866$.

It is useful to define the \trmidx{size parameter} $\szprmhxg$ of
hexagonal prisms when discussing their optical properties. 
Following \cite{NGW03},
\begin{eqnarray}
\szprmhxg & = & \frac{2\mpi}{\wvl} \sqrt{\frac{\rdshxg\dmthxg}{2}} \nonumber \\
\szprmhxg & = & \frac{2\mpi\rdshxg}{\wvl} \sqrt{\asphxg}
\label{eqn:sz_prm_hxg_dfn}
\end{eqnarray}
This can be compared to the size parameter for spherical particles
(\ref{eqn:sz_prm_gnr_dfn}) .

\section{Vector Mathematics}\label{sxn:vec_mth}
\subsection[Del Operator]{Del Operator}\label{sxn:del_crt}
We define the \trmdfn{del operator} in Cartesian coordinates as 
\begin{equation}
\nabla = \ihat \frac{\partial}{\partial \xxx} + 
\jhat \frac{\partial}{\partial \yyy} + 
\khat \frac{\partial}{\partial \zzz}
\label{eqn:del_crt}
\end{equation}
where $\ihat$, $\jhat$, and $\khat$ are the unit vectors in the
$\xxx$, $\yyy$, and $\zzz$ directions, respectively.
Thus $\nabla$ is a vector differential operator.
Another name for $\nabla$ is \trmdfn{nabla}.
The del operator can act on both scalar and vector fields.
These operations are given distinct names to make clear the type of
field $\nabla$ is operating on, and exactly how $\nabla$ operates on
the field.
In all cases, however, the operations employ the del operator.

\subsection[Gradient]{Gradient}\label{sxn:grd_crt}
Every continuous, differentiable scalar field $\sclfld$ defines a vector 
field called the \trmdfn{gradient} of $\sclfld$ or $\mathrm{grad}\ \sclfld$.
The $\mathrm{grad}$ operator is written $\nabla$.
In Cartesian coordinates, the gradient of $\sclfld$ is 
\begin{equation}
\nabla \sclfld = 
\frac{\partial \sclfld}{\partial \xxx} \ihat + 
\frac{\partial \sclfld}{\partial \yyy} \jhat + 
\frac{\partial \sclfld}{\partial \zzz} \khat
\label{eqn:grd_crt}
\end{equation}
Note that the unit vectors appear after the differential operators, 
differentiation of the unit vectors is \textit{not} implied. 
The gradient of $\sclfld$ points in the direction of most rapidly
changing $\sclfld$, and the magnitude of $\nabla \sclfld$ is the rate
of change of $\sclfld$ along that direction.

\subsection[Divergence]{Divergence}\label{sxn:dvr_crt}
Associated with every continuous and differentiable vector field
$\vctbld$ is a scalar field called the \trmdfn{divergence} of
$\vctbld$, or $\mathrm{div}\ \vctbld$.
The $\mathrm{div}$ operator is written $\nabla \cdot$.
In Cartesian coordinates, the divergence of $\vctbld$ is 
\begin{equation}
\nabla \cdot \vctbld = 
\frac{\partial \vctxxx}{\partial \xxx} + 
\frac{\partial \vctyyy}{\partial \yyy} + 
\frac{\partial \vctzzz}{\partial \zzz} 
\label{eqn:dvr_crt}
\end{equation}
The ``$\cdot$'' operator denotes the inner product and is read ``dot''. 
Hence the divergence operator equals ``del dot''.
For definiteness, let $\vctbld$ represent the fluid velocity field and
consider an infinitesimal control volume $\dfr\vlm$ with surface area
$\dfr\sfc$ centered at the point $\pnt$. 
Then value of $\nabla \cdot \vctbld$ at $\pnt$ is the fluid outflow
through $\sfc$ per unit volume. 
\subsection[Curl]{Curl}\label{sxn:crl_crt}
Associated with every continuous and differentiable vector field
$\vctbld$ is another vector field called the \trmdfn{curl} of
$\vctbld$, or $\mathrm{curl}\ \vctbld$.
The $\mathrm{curl}$ operator is written $\nabla \times$.
In Cartesian coordinates, the curl of $\vctbld$ is 
\begin{equation}
\nabla \times \vctbld = 
\left( \frac{\partial \vctzzz}{\partial \yyy} - 
\frac{\partial \vctyyy}{\partial \zzz} \right) \ihat + 
\left( \frac{\partial \vctxxx}{\partial \zzz} - 
\frac{\partial \vctzzz}{\partial \xxx} \right) \jhat + 
\left( \frac{\partial \vctyyy}{\partial \xxx} - 
\frac{\partial \vctxxx}{\partial \yyy} \right) \khat
\label{eqn:crl_crt}
\end{equation}
Hence the curl operator equals ``del cross''.
When $\vctbld$ represents the fluid velocity, then 
$\nabla \times \vctbld$ is called the \trmdfn{vorticity}.
In this case, the vorticity $\nabla \times \vctbld$ at the point
$\pnt$ is twice the angular velocity of the fluid at $\pnt$.

\subsection[Laplacian]{Laplacian}\label{sxn:lpl_crt}
Finally, we define an operator called the \trmdfn{Laplacian} which,
like del (\ref{eqn:del_crt}), can act on both scalar and vector
fields. 
The Laplacian operator is written $\nabla^{2}$, which is often read
as \trmdfn{del square}. 
The Laplacian is not del times del, but rather
del \textit{dot} del. 
In Cartesian coordinates the Laplacian operator is
\begin{eqnarray}
\nabla^{2} = \nabla \cdot \nabla & = &
\left( \ihat \frac{\partial}{\partial \xxx} + 
\jhat \frac{\partial}{\partial \yyy} + 
\khat \frac{\partial}{\partial \zzz} \right)
\cdot
\left( \ihat \frac{\partial}{\partial \xxx} + 
\jhat \frac{\partial}{\partial \yyy} + 
\khat \frac{\partial}{\partial \zzz} \right)
\nonumber \\
& = &
\frac{\partial^{2}}{\partial \xxx^{2}} + 
\frac{\partial^{2}}{\partial \yyy^{2}} + 
\frac{\partial^{2}}{\partial \zzz^{2}}
\label{eqn:lpl_crt}
\end{eqnarray}
The similarity between (\ref{eqn:lpl_crt}) and (\ref{eqn:del_crt})
explains why the Laplacian is often called ``del square''.
In Cartesian coordinates the Laplacian of $\sclfld$ and the Laplacian
of $\vctbld$ are 
\begin{eqnarray}
\nabla^{2} \sclfld & = &
\frac{\partial^{2} \sclfld}{\partial \xxx^{2}} + 
\frac{\partial^{2} \sclfld}{\partial \yyy^{2}} + 
\frac{\partial^{2} \sclfld}{\partial \zzz^{2}} \\
\nabla^{2} \vctbld & = & 
\left( 
\frac{\partial^{2} \vctxxx}{\partial \xxx^{2}} +
\frac{\partial^{2} \vctxxx}{\partial \yyy^{2}} +
\frac{\partial^{2} \vctxxx}{\partial \zzz^{2}} 
\right) \ihat + 
\left( 
\frac{\partial^{2} \vctyyy}{\partial \xxx^{2}} +
\frac{\partial^{2} \vctyyy}{\partial \yyy^{2}} +
\frac{\partial^{2} \vctyyy}{\partial \zzz^{2}} 
\right) \jhat \nonumber \\
& & + \left( 
\frac{\partial^{2} \vctzzz}{\partial \xxx^{2}} +
\frac{\partial^{2} \vctzzz}{\partial \yyy^{2}} +
\frac{\partial^{2} \vctzzz}{\partial \zzz^{2}} 
\right) \khat
\label{eqn:lpl_crt_2}
\end{eqnarray}
Hence the Laplacian of a scalar field is a scalar field and 
the Laplacian of a vector field is a vector field.
Comparing (\ref{eqn:lpl_crt_2}) to (\ref{eqn:grd_crt}) and
(\ref{eqn:dvr_crt}), we see that the Laplacian of a scalar function is
the divergence of the gradient of the function. % Definitely true
The Laplacian of a vector field is \textit{not} the gradient 
of the divergence of the field. 

\section[Spherical Coordinates]{Spherical Coordinates}\label{sxn:sph}
The spherical coordinate system is the primary coordinate system used
in large scale meteorology.
Virtually all meteorological applications use latitude $\mrd$ as the 
meridional coordinate and longitude $\znl$ as the azimuthal
coordinate. 
The latitude angle $\mrd$ projects onto the $\zzz$-axis and is defined
in the domain $-\mpi/2 \le \mrd \le \mpi/2$ so that $\mrd = 0$ is at the
equator.
The longitude angle $\znl$ is defined in the domain $0 \le \znl <
2\mpi$ so that $\znl = 0$ is on the $\xxx$-axis (at Greenwich).
We shall refer to such coordinates as \trmdfn{spherical coordinates}
or \trmdfn{spherical meteorological coordinates}.

There is, however, a potential source of confusion in using spherical  
coordinates. 
The confusion arises when sciences besides large scale meteorology
are involved.
In many (most?) sciences besides large scale meteorology, it is
traditional to work with \trmdfn{spherical polar coordinates}.
The additional term ``polar'' indicates that the angle~$\plr$
which projects onto the $\zzz$-axis is defined in the domain 
$0 \le \plr \le \mpi$ so that $\plr = 0$ is at the North pole.
Thus many physicists, astronomers, geodysists, and radiative transfer 
theoreticians are used to working with the co-latitude~$\plr$, rather
than with the latitude,~$\mrd$.
To compound the confusion, meteorologists use $\mrd$ for the
meridional angle while astronomers and radiative transfer theorists
use $\azi$ for the azimuthal angle.

\subsection[Cartesian-Spherical Transormations]{Cartesian-Spherical Transformations}\label{sxn:crt_sph}
Spherical polar coordinates ($\rdl$,$\plr$,$\azi$) transform into
\trmidx{Cartesian coordinates} ($\xxx$,$\yyy$,$\zzz$) as
\begin{subequations}
\begin{align}
\xxx & = \rdl\sin\plr\cos\azi \\
\yyy & = \rdl\sin\plr\sin\azi \\
\zzz & = \rdl\cos\plr
\label{eqn:sph_crt_trn}
\end{align}
\end{subequations}
Cartesian coordinates ($\xxx$,$\yyy$,$\zzz$) transform into
spherical polar coordinates ($\rdl$,$\plr$,$\azi$) as
\begin{subequations}
\begin{align}
\rdl & = \sqrt{\xxx^{2}+\yyy^{2}+\zzz^{2}} \\
\plr & = \arccos(\zzz/\rdl) \\
\azi & = \arctan(\yyy/\xxx)
\label{eqn:crt_sph_trn}
\end{align}
\end{subequations}
Cartesian unit vectors ($\ihat$,$\jhat$,$\khat$) transform into
spherical polar unit vectors ($\rdlhat$,$\plrhat$,$\azihat$)
\begin{subequations}
\begin{align}
\rdlhat & = \sin\plr\cos\azi\ihat + \sin\plr\sin\azi\jhat + \cos\plr\khat \\
\plrhat & = \cos\plr\cos\azi\ihat + \cos\plr\sin\azi\jhat - \sin\plr\khat \\
\azihat & = -\sin\azi\ihat + \cos\azi\jhat
\label{eqn:crt_sph_unit_vct_trn}
\end{align}
\end{subequations}
It can be verified the basis vectors are orthonormal.

Spherical meteorological coordinates ($\rdl$,$\mrd$,$\znl$) transform into
\trmidx{Cartesian coordinates} ($\xxx$,$\yyy$,$\zzz$) as
\begin{subequations}
\begin{align}
\xxx & = \rdl\cos\mrd\cos\znl \\
\yyy & = \rdl\cos\mrd\sin\znl \\
\zzz & = \rdl\sin\mrd
\label{eqn:mtr_crt_trn}
\end{align}
\end{subequations}
Cartesian coordinates ($\xxx$,$\yyy$,$\zzz$) transform into
spherical meteorological coordinates ($\rdl$,$\mrd$,$\znl$) as
\begin{subequations}
\begin{align}
\rdl & = \sqrt{\xxx^{2}+\yyy^{2}+\zzz^{2}} \\
\mrd & = \arcsin(\zzz/\rdl) \\
\znl & = \arctan(\yyy/\xxx)
\label{eqn:crt_sph_trn}
\end{align}
\end{subequations}
Cartesian unit vectors ($\ihat$,$\jhat$,$\khat$) transform into
spherical meteorological unit vectors ($\rdlhat$,$\mrdhat$,$\znlhat$)
(20160117 fxm: verify this)
\begin{subequations}
\begin{align}
\rdlhat & = \cos\mrd\cos\znl\ihat + \cos\mrd\sin\znl\jhat + \sin\mrd\khat \\
\mrdhat & = \sin\mrd\cos\znl\ihat + \sin\mrd\sin\znl\jhat - \cos\mrd\khat \\
\znlhat & = -\sin\znl\ihat + \cos\znl\jhat
\label{eqn:crt_mtr_unit_vct_trn}
\end{align}
\end{subequations}

\subsection[Gradient]{Gradient}\label{sxn:grd_sph}
In spherical meteorological coordinates, the \trmidx{gradient} of $\sclfld$ is 
\begin{eqnarray}
\nabla \sclfld & = & 
\frac{\partial \sclfld}{\partial \rdl} \rdlhat + 
\frac{1}{\rdl} \frac{\partial \sclfld}{\partial \mrd} \mrdhat + 
\frac{1}{\rdl \cos \mrd} \frac{\partial \sclfld}{\partial \znl} \znlhat
\label{eqn:grd_sph_mtr}
\end{eqnarray}
In spherical polar coordinates, the gradient of $\sclfld$ is 
\begin{eqnarray}
\nabla \sclfld & = & 
\frac{\partial \sclfld}{\partial \rdl} \rdlhat + 
\frac{1}{\rdl} \frac{\partial \sclfld}{\partial \plr} \plrhat + 
\frac{1}{\rdl \sin \plr} \frac{\partial \sclfld}{\partial \azi} \azihat
\label{eqn:grd_sph_plr}
\end{eqnarray}

\subsection[Divergence]{Divergence}\label{sxn:dvr_sph}
In spherical meteorological coordinates, the \trmidx{divergence} of $\vctbld$ is 
\begin{eqnarray}
\nabla \cdot \vctbld & = & 
\frac{1}{\rdl^{2}} \frac{\partial}{\partial \rdl } ( \rdl^{2} \vctrdl ) + 
\frac{1}{\rdl \sin \mrd} \frac{\partial}{\partial \mrd} ( \cos \mrd \vctmrd ) + 
\frac{1}{\rdl \sin \mrd} \frac{\partial \vctznl}{\partial \znl} 
\nonumber \\
& = &
\frac{1}{\rdl^{2} \cos \mrd} \left(
\cos \mrd \frac{\partial}{\partial \rdl } ( \rdl \vctrdl ) + 
\rdl \frac{\partial}{\partial \mrd} ( \cos \mrd \vctmrd ) + 
\rdl \frac{\partial \vctznl}{\partial \znl} 
\right)
\label{eqn:dvr_sph_mtr}
\end{eqnarray}
In spherical polar coordinates, the divergence of $\vctbld$ is 
\begin{eqnarray}
\nabla \cdot \vctbld & = &
\frac{1}{\rdl^{2}} \frac{\partial}{\partial \rdl } ( \rdl^{2} \vctrdl ) + 
\frac{1}{\rdl \sin \plr} \frac{\partial}{\partial \plr} ( \sin \plr \vctplr ) + 
\frac{1}{\rdl \sin \plr} \frac{\partial \vctazi}{\partial \azi} 
\nonumber \\
& = &
\frac{1}{\rdl^{2} \sin \plr} \left(
\sin \plr \frac{\partial}{\partial \rdl } ( \rdl \vctrdl ) + 
\rdl \frac{\partial}{\partial \plr} ( \sin \plr \vctplr ) + 
\rdl \frac{\partial \vctazi}{\partial \azi} 
\right)
\label{eqn:dvr_sph_plr}
\end{eqnarray}

\subsection[Curl]{Curl}\label{sxn:crl_sph}
In spherical meteorological coordinates, the \trmidx{curl} of $\vctbld$ is 
\begin{eqnarray}
\nabla \times \vctbld & = & 
\frac{1}{\rdl \cos \mrd} \left( 
\frac{\partial \vctmrd}{\partial \znl} - 
\frac{\partial}{\partial \mrd} ( \cos \mrd \vctznl ) 
\right) \rdlhat +
\frac{1}{\rdl} \left( 
\frac{\partial}{\partial \rdl} ( \rdl \vctznl ) - 
\frac{1}{\cos \mrd} \frac{\partial \vctrdl}{\partial \znl} 
\right) \mrdhat \nonumber \\
& &  {} + \frac{1}{\rdl} \left( 
\frac{\partial \vctrdl}{\partial \mrd} - 
\frac{\partial}{\partial \rdl} ( \rdl \vctmrd )
\right) \znlhat
\label{eqn:crl_sph_mtr}
\end{eqnarray}
In spherical polar coordinates, the curl of $\vctbld$ is 
\begin{eqnarray}
\nabla \times \vctbld & = & 
\frac{1}{\rdl \sin \plr} \left( 
\frac{\partial \vctplr}{\partial \azi} - 
\frac{\partial}{\partial \plr} ( \sin \plr \vctazi ) 
\right) \rdlhat +
\frac{1}{\rdl} \left( 
\frac{\partial}{\partial \rdl} ( \rdl \vctazi ) - 
\frac{1}{\sin \plr} \frac{\partial \vctrdl}{\partial \azi} 
\right) \plrhat \nonumber \\
& & {} + \frac{1}{\rdl} \left( 
\frac{\partial \vctrdl}{\partial \plr} - 
\frac{\partial}{\partial \rdl} ( \rdl \vctplr )
\right) \azihat
\label{eqn:crl_sph_plr}
\end{eqnarray}

\subsection[Laplacian]{Laplacian}\label{sxn:lpl_sph}
In spherical polar coordinates, the \trmidx{Laplacian} of $\sclfld$ is 
\begin{eqnarray}
\nabla^{2} \sclfld & = & 
\frac{1}{\rdl^{2}} \frac{\partial}{\partial \rdl} 
\left( \rdl^{2} \frac{\partial \sclfld}{\partial \rdl} \right) +
\frac{1}{\rdl^{2} \sin \plr} \frac{\partial}{\partial \plr}
\left( \sin \plr \frac{\partial \sclfld}{\partial \plr} \right) + 
\frac{1}{\rdl^{2} \sin^{2} \plr} \frac{\partial^{2} \sclfld}{\partial \azi^{2}}
\nonumber \\
 & = & 
\frac{1}{\rdl} \frac{\partial^{2}}{\partial \rdl^{2}} ( \rdl \sclfld ) +
\frac{1}{\rdl^{2} \sin \plr} \frac{\partial}{\partial \plr}
\left( \sin \plr \frac{\partial \sclfld}{\partial \plr} \right) + 
\frac{1}{\rdl^{2} \sin^{2} \plr} \frac{\partial^{2} \sclfld}{\partial \azi^{2}}
\label{eqn:lpl_sph_plr_scl}
\end{eqnarray}

The vector Laplacian consists of the scalar Laplacian applied to each 
individual component of the vector.
In spherical polar coordinates, the Laplacian of $\vctbld$ is quite
complex.
It is most easily derived by taking the curl of (\ref{eqn:crl_sph_plr})
and applying (\ref{eqn:vct_idn_crl_crl}) to the result. % fxm: eqn is in rt.tex
The radial ($\rdlhat$) component of the result is 
\begin{eqnarray}
% Arf85 p. 105 (2.48)
\nabla^{2} \vctbld \bigg|_{\rdl} & = & \left(
-\frac{2}{\rdl^{2}} + \frac{2}{\rdl} \frac{\partial}{\partial \rdl} +
\frac{\partial^{2}}{\partial \rdl^{2}} + 
\frac{\cos \plr}{\rdl^{2} \sin \plr} \frac{\partial}{\partial \plr} +
\frac{1}{\rdl^{2}} \frac{\partial^{2}}{\partial \plr^{2}} +
\frac{1}{\rdl^{2} \sin^{2} \plr} \frac{\partial^{2}}{\partial \azi^{2}}
\right) \vctrdl  
\nonumber \\ & & {} + % KoD99 p. 138 for spacing info
\left( -\frac{2}{\rdl^{2}} \frac{\partial}{\partial \plr} -
\frac{2 \cos \plr}{\rdl^{2} \sin \plr} \right) \vctplr +
\left( - \frac{2}{\rdl^{2} \sin \plr} \frac{\partial}{\partial \azi} 
\right) \vctazi
\label{eqn:lpl_sph_plr_vct_rdl}
\end{eqnarray}
Using (\ref{eqn:lpl_sph_plr_scl}) to simplify the operator on the
radial component of $\vctrdl$, we obtain
\begin{eqnarray}
% Arf85 p. 105 (2.48)
\nabla^{2} \vctbld & = & 
\left( \nabla^{2} \vctrdl - \frac{2}{\rdl^{2}} \vctrdl -
\frac{2}{\rdl^{2}} \frac{\partial \vctplr}{\partial \plr} -
\frac{2 \cos \plr}{\rdl^{2} \sin \plr} \vctplr -
\frac{2}{\rdl^{2} \sin \plr} \frac{\partial \vctazi}{\partial \azi}
\right) \rdlhat
\nonumber \\ & & {} + % KoD99 p. 138 for spacing info
\left( \nabla^{2} \vctplr - \frac{1}{\rdl^{2} \sin^{2} \plr} \vctplr +
\frac{2}{\rdl^{2}} \frac{\partial \vctrdl}{\partial \plr} - 
\frac{2 \cos \plr}{\rdl^{2} \sin^{2} \plr} 
\frac{\partial \vctazi}{\partial \azi}
\right) \plrhat
\nonumber \\ & & {} + % KoD99 p. 138 for spacing info
\left( \nabla^{2} \vctazi - \frac{1}{\rdl^{2} \sin^{2} \plr} \vctazi +
\frac{2}{\rdl^{2} \sin \plr} \frac{\partial \vctrdl}{\partial \azi} +
\frac{2 \cos \plr}{\rdl^{2} \sin^{2} \plr} 
\frac{\partial \vctplr}{\partial \azi} 
\right) \azihat
\label{eqn:lpl_sph_plr_vct}
\end{eqnarray}
Quantities satisfying the \trmidx{vector wave equation} on the sphere
will require (\ref{eqn:lpl_sph_plr_vct}).

\section{Fluid Mechanics}\label{sxn:fld_mch}
Let the vector field $\vlcbld$ represent the fluid velocity.
The velocity is a function of position $\psnbld$ and time $\tm$. 
In Cartesian coordinates, the components of $\vlcbld$ are 
\begin{equation}
\vlcbld = \vlcxxx \ihat + \vlcyyy \jhat + \vlczzz \khat
\label{eqn:vlc_dfn}
\end{equation}
where $\ihat$, $\jhat$, and $\khat$ are the unit vectors in the
$\xxx$, $\yyy$, and $\zzz$ directions, respectively.
The magnitude of $\vlcbld$, or \trmdfn{wind speed}, is
\begin{equation}
|\vlcbld| = \wndspd = (\vlcbld \cdot \vlcbld)^{1/2} = 
(\vlcxxx^{2} + \vlcyyy^{2} + \vlczzz^{2})^{1/2}
\label{eqn:spd_dfn}
\end{equation}
The wind speed $\wndspd$ is measured in \mxs.

Consider a scalar field $\tpt$ that is a function of space and time,
i.e., $\tpt = \tpt(\xxx,\yyy,\zzz,\tm)$.
For concreteness, say $\tpt_{0}$ represents the temperature at a given 
point $(\xxx_{0},\yyy_{0},\zzz_{0})$. 
The partial derivative of $\tpt$ with respect to time, $\partial_{\tm}
\tpt$ represents the change in $\tpt$ due to internal sources and
sinks. 
Condensation or evaporation of water vapor, chemical reactions, and
radiative processes are examples of internal sources and sinks of heat.  
Such internal processes can alter $\tpt_{0}$ in the absence of any
transport, i.e., when $\vlcbld = 0$.
However, $\tpt_{0}$ may also change due to \trmdfn{advective} processes.   
For example, wind may bring warmer air to a cool region, increasing
the local temperature.

The \trmdfn{material derivative} embodies these processes.
The material derivative is the sum of the rates of change of a
property due to internal processes and to advection
\begin{equation}
\frac{\mtrdrv}{\mtrdrv \tm} = 
\frac{\partial}{\partial \tm} + \vlcbld \cdot \nabla
\label{eqn:mtr_drv_dfn}
\end{equation}
Since internal and advective processes are the only physical processes
which can alter a conserved quantity, the material derivative in
(\ref{eqn:mtr_drv_dfn}) is also called the \trmdfn{total derivative}.
The material derivative in (\ref{eqn:mtr_drv_dfn}) is also called the
\trmdfn{Lagrangian derivative} or the 

To continue our example with the temperature field $\tpt$
\begin{equation}
\frac{\mtrdrv \tpt}{\mtrdrv \tm} = 
\frac{\partial \tpt}{\partial \tm} + \vlcbld \cdot \nabla \tpt
\label{eqn:mtr_drv_tpt}
\end{equation}
The LHS is the rate of change of $\tpt$ following a fluid element.
The RHS has two contributions.
The first, $\frac{\partial \tpt}{\partial \tm}$, is any temporal
change in $\tpt$ not due to motion.
For example, a fixed amount of condensation increases $\tpt$ in a
parcel regardless of how fast the parcel moves.
However, the Lagrangian description of the parcel.

Consider a fixed parcel of fluid in motion with its environment.
To develop a mental picture of this parcel let us imagine that every 
molecule in it is colored green, and that the parcel moves within a
medium of blue molecules.
Regardless of how the fluid motion deforms the shape of the green
parcel, its mass is unchanged.
Since the mass of the parcel is conserved, its density must change if
its volume changes.
Conservation of mass requires that changes in local fluid density 
occur only with a corresponding change in mass convergence or
divergence. 
\begin{equation}
\frac{\partial \dnsatm}{\partial \tm} + 
\nabla \cdot ( \dnsatm \vlcbld ) = 0
\label{eqn:eoc_flx}
\end{equation}
This is the \trmdfn{equation of continuity in flux form}.  
Using the product rule, (\ref{eqn:eoc_flx}) becomes
\begin{eqnarray}
\frac{\partial \dnsatm}{\partial \tm } + 
\dnsatm \nabla \cdot \vlcbld + 
\vlcbld \cdot \nabla \dnsatm & = & 0 \nonumber \\
\frac{\mtrdrv \dnsatm}{\mtrdrv \tm } +
\dnsatm \nabla \cdot \vlcbld & = & 0 \nonumber \\
\frac{1}{\dnsatm } \frac{\mtrdrv \dnsatm}{\mtrdrv \tm } +
\nabla \cdot \vlcbld & = & 0
\label{eqn:eoc_dvr}
\end{eqnarray}
The latter is the \trmdfn{equation of continuity in velocity
divergence form}. 
The first term in (\ref{eqn:eoc_flx}) is the fractional rate of change
of density.
The divergence theorem tells us that the second term, $\nabla \cdot
\vlcbld$, equals the fractional rate of change of volume. 
Thus the conservation of mass implies that the fractional rate of
change of density must compensate the fractional rate of change of
volume. 

In steady state flow the density of a fluid does not change with time
so that $\partial_{\tm} \dnsatm = 0$ and 
\begin{equation}
\dnsatm \nabla \cdot \vlcbld + \vlcbld \cdot \nabla \dnsatm = 0
\label{eqn:ss_dfn}
\end{equation}
If the fluid is \trmdfn{incompressible}, (\ref{eqn:ss_dfn})
simplifies further.
Incompressibility does not mean that the fluid flow is everywhere
uniform. 
Rather, it means that when the flow in one direction changes, due to
an impediment, say, that there is a corresponding change in the flow
in the other directions such that the density of the fluid does not
change. 
In this sense a balloon filled with water is incompressible.
Squeezing it in one direction does not change the density of the water
because the balloon expands in other directions allowing water to
remain at a constant density.
The mathematical expression of the property of incompressibility can
be shown to be \cite[e.g.,][]{Dut86}
\begin{equation}
\nabla \cdot \vlcbld = 0
\label{eqn:ncm_dfn_vec}
\end{equation}
In Cartesian coordinates (\ref{eqn:ncm_dfn_vec}) becomes
\begin{equation}
% SeP97 p. 460
\frac{\partial \uuu_{\xxx}}{\partial \xxx} + 
\frac{\partial \uuu_{\yyy}}{\partial \yyy} + 
\frac{\partial \uuu_{\zzz}}{\partial \zzz} = 0 
\label{eqn:ncm_dfn_crt}
\end{equation}

All of classical mechanics are based on \trmdfn{Newton's laws of
motion}.  
The first law states that
\trmdfn{Newton's Second Law} of motion states that there is a conserved
quantity called \trmdfn{momentum} which is the mass $\mss$ of an
object times its velocity $\vvvbld$
\begin{eqnarray}
\mmnbld & = & \mss \vvvbld
\label{eqn:2nd_law_1}
\end{eqnarray}
Since momentum $\mmnbld$ is conserved and mass $\mss$ is fixed,
(\ref{eqn:2nd_law_1}) implies that the velocity $\vvvbld$ of an object 
remains constant until an external force is applied.
Hence, a body in motion remains in motion until an external force is
applied. 
The second law is often quoted in a form which may be obtained by
differentiating (\ref{eqn:2nd_law_1}) with respect to time
\begin{eqnarray}
\frcbld & = & \mss \frac{\dfr\vvvbld}{\dfr\tm} \nonumber \\
\frcbld & = & \mss \xclbld
\label{eqn:2nd_law_2}
\end{eqnarray}
where $\frcbld$ denotes \trmdfn{force} and $\xclbld$
\trmdfn{acceleration}. 
Thus force $\frcbld$ is the time rate of change of momentum
$\mmnbld$. 
Equation~(\ref{eqn:2nd_law_2}) states that an applied force $\frcbld$
alters the velocity of an object by inducing an acceleration
$\xclbld$.  
Although (\ref{eqn:2nd_law_1}) and (\ref{eqn:2nd_law_2}) may appear to 
be solely definitions at this point, their richness and predictive
capacity quickly become apparent when physical laws are substituted
for $\frcbld$.
For example, the force of gravitation $\frcgrvbld$ between two objects of
masses $\mss_{1}$ and $\mss_{2}$ is 
\begin{eqnarray}
\frcgrvbld & = & - \frac{\cstgrv \mss_{1} \mss_{2}}{ \rdl^{2} } \rdlhat
\label{eqn:law_grv}
\end{eqnarray}
where the objects are separated by a displacement $\rdsbld$ and 
$\cstgrv = 6.672 \times 10^{-11}$\,\NmSxkg\ is the 
\trmdfn{universal gravitational constant}.
All of celestial mechanics follows from substituting
(\ref{eqn:law_grv}) into~(\ref{eqn:2nd_law_2}).

The fluid dynamical momentum equations are obtained by rewriting 
Newton's second law of motion (\ref{eqn:2nd_law_2}) to take account of 
forces and geometries.
First we express (\ref{eqn:2nd_law_2}) in terms of a fluid density
$\dns$  
\begin{eqnarray}
\dns \frac{\dfr\vvvbld}{\dfr\tm} & = & \frcbld 
\label{eqn:2nd_law_3}
\end{eqnarray}
We are now following the fluid dynamical convention of placing the 
acceleration terms on the LHS and the applied forces on the RHS.
The material derivative (\ref{eqn:mtr_drv_dfn}) describes the time
rate of change of a fluid parcel's momentum due to the net force
$\frcbld$ on the parcel 
\begin{eqnarray}
\dns \frac{\mtrdrv \vvvbld}{\dfr\tm} & = & \frcbld \\
\frac{\partial \vvvbld}{\partial \tm} + \vlcbld \cdot \nabla \vvvbld 
& = & \frac{\frcbld}{\dns } 
\label{eqn:2nd_law_4}
\end{eqnarray}
The net force on a fluid parcel is the vector sum of the
\trmdfn{external forces} and \trmdfn{internal forces} on the parcel.
External forces act on all the fluid independent of its state of
motion. 
External forces are also called \trmdfn{body forces}.
The only body force of relevance to large scale fluid mechanics is
gravity. 
Coulomb attraction is another important body force which may be
important for some aerosol motion.

Internal forces, on the other hand, may depend on the state of motion
of the fluid.
Internal forces depend on the position and orientation of the parcel.
These forces are called \trmdfn{stresses} and expressed as a force per
unit area.
Since the stress depends on both position and orientation, the stress
must be expressed as a tensor $\strtns$, called the \trmdfn{stress
tensor}.  
The stress tensor represents all permutations of
directionally-dependent forces. 
For example, $\strtns_{\xxx\zzz}$ is the stress acting in the direction
of the $\xxx$ axis (i.e., along $\ihat$) on a surface normal to the
$\zzz$ coordinate axis (i.e., normal to $\khat$).
The stress at any point of a parcel is the vector sum of stresses
acting normal to the surface and forces acting tangential to the
surface. 
It can be shown that the stress normal to a parcel is, in fact, the 
pressure $\prs$.
The surface tangential stresses to a parcel are also called
\trmdfn{shear stress}, \trmdfn{frictional stress}, \trmdfn{viscous
stress} or simply \trmdfn{drag}. 
The viscous stress is also a tensor, and denoted $\strtnsvsc$.
Thus $\frcbld$ (\ref{eqn:2nd_law_4}) is more commonly written as 
\begin{eqnarray}
\frcbld & = & \dnsatm \grvbld + \nabla \strtns \nonumber \\
& = & \dnsatm \grvbld - \nabla \prs + \nabla \cdot \strtnsvsc
\label{eqn:frc_bld_dfn}
\end{eqnarray}
where $\grvbld = - \grv \khat$.
The pressure gradient is preceded by a negative sign so that
acceleration is positive towards the direction of decreasing
pressure. 
We may proceed further if we assume the shear stress term $\nabla
\cdot \strtnsvsc$ obeys Newton's law of friction (\ref{eqn:nwt_lof}).  
This is an excellent assumption for most fluids.
\begin{eqnarray}
\frcbld & = & \dnsatm \grvbld - \nabla \prs + \vscdyn \nabla^{2} \vvvbld
\label{eqn:frc_bld_dfn_2}
\end{eqnarray}

Inserting (\ref{eqn:frc_bld_dfn_2}) in (\ref{eqn:2nd_law_4}), the
momentum equations become 
\begin{eqnarray}
\frac{\partial \vvvbld}{\partial \tm} + \vvvbld \cdot \nabla \vvvbld 
& = & \grvbld + 
\frac{1}{\dnsatm } \nabla \prs + 
\frac{\vscdyn}{\dnsatm } \nabla^{2} \vvvbld
\label{eqn:2nd_law_5}
\end{eqnarray}

In \trmdfn{Euler's equations of motion}, conservation of momentum
(Newton's second law of motion) appears as 
\begin{eqnarray}
\frac{\mtrdrv \vvvbld}{\mtrdrv \tm} 
& = &
-\frac{1}{\dnsatm} \nabla \prsatm + \grvbld + 
\frac{\vscdyn}{\dnsatm} \nabla^{2} \vvvbld 
\nonumber \\
\frac{\partial \vvvbld}{\partial \ttt} + \vvvbld \cdot \nabla \vvvbld 
& = &
-\frac{1}{\dnsatm} \nabla \prsatm + \grvbld + 
\frac{\vscdyn}{\dnsatm} \nabla^{2} \vvvbld
\label{eqn:elr_eom_vec}
\end{eqnarray}
where $\prsatm$ is the atmospheric pressure, $\grvbld$ is the
gravitational acceleration, and $\vscdyn$ is the \trmdfn{dynamic
viscosity of air}.
The advective term $\AAAbld = \vvvbld \cdot \nabla \vvvbld$ on the
LHS can be confusing.
The unique interpretation is $\AAAbld = ( \vvvbld \cdot \nabla )
\vvvbld$, i.e., $\AAAbld$ is the result of $\vvvbld \cdot \nabla$ (a
scalar operator) operating on the vector $\vvvbld$.
\begin{eqnarray}
\AAAbld & = &
\left( 
\vctxxx \frac{\partial \vctxxx}{\partial \xxx} +
\vctyyy \frac{\partial \vctxxx}{\partial \yyy} +
\vctzzz \frac{\partial \vctxxx}{\partial \zzz} 
\right) \ihat + 
\left( 
\vctxxx \frac{\partial \vctyyy}{\partial \xxx} +
\vctyyy \frac{\partial \vctyyy}{\partial \yyy} +
\vctzzz \frac{\partial \vctyyy}{\partial \zzz} 
\right) \jhat \nonumber \\
& & + \left( 
\vctxxx \frac{\partial \vctzzz}{\partial \xxx} +
\vctyyy \frac{\partial \vctzzz}{\partial \yyy} +
\vctzzz \frac{\partial \vctzzz}{\partial \zzz} 
\right) \khat
\label{eqn:adv_dfn}
\end{eqnarray}
One common misinterpretation of $\AAA$ is $\vvvbld \cdot ( \nabla
\vvvbld )$ which is meaningless since $\nabla \vvvbld$ is undefined.
Another misinterpretation of $\AAA$ is 
$\vvvbld \cdot \nabla \cdot \vvvbld$.
This is also fallacious since, no matter which inner product is
computed first ($\vvvbld \cdot \nabla$ or $\nabla \cdot \vvvbld$), the
final operation would have to be the inner product of a scalar and a
vector. 

In Cartesian coordinates, the $\xxx$ component of 
(\ref{eqn:elr_eom_vec}) is
\begin{equation}
\frac{\partial \vctxxx}{\partial \ttt} + 
\left( 
\vctxxx \frac{\partial \vctxxx}{\partial \xxx} +
\vctyyy \frac{\partial \vctxxx}{\partial \yyy} +
\vctzzz \frac{\partial \vctxxx}{\partial \zzz} 
\right) =  
-\frac{1}{\dnsatm} \frac{\partial \prs}{\partial \xxx} +
\frac{\vscdyn}{\dnsatm} 
\left( 
\frac{\partial^{2} \vctxxx}{\partial \xxx^{2}} +
\frac{\partial^{2} \vctxxx}{\partial \yyy^{2}} +
\frac{\partial^{2} \vctxxx}{\partial \zzz^{2}} 
\right) 
\label{eqn:mmn_cns_scl}
\end{equation}

% Bibliography
%\renewcommand\refname{\normalsize Publications}
\bibliographystyle{agu04}
\bibliography{bib}
\printindex % Requires makeidx KoD95 p. 221

\csznote{ 
% Usage: Place usage here at end of file so comment character % not needed
cd ~/sw/crr;make -W aer.tex aer.dvi aer.ps aer.pdf aer.txt;cd -

# dvips -Ppdf -G0 -o ${DATA}/ps/aer.ps ~/sw/crr/aer.dvi;ps2pdf ${DATA}/ps/aer.ps ${DATA}/ps/aer.pdf
# cd ~/sw/crr;texcln aer;make aer.pdf;bibtex aer;makeindex aer;make aer.pdf;bibtex aer;makeindex aer;make aer.pdf;cd -
scp -p ~/sw/crr/aer.tex ~/sw/crr/aer.txt ~/sw/crr/aer.dvi ${DATA}/ps/aer.ps ${DATA}/ps/aer.pdf dust.ess.uci.edu:Sites/facts/aer

# 20030625: latex2html chokes on math in tables in \csznote{}s
# cd ~/sw/crr;latex2html -dir Sites/facts/aer aer.tex
# cd ~/sw/crr;latex2html -nolatex -dir Sites/facts/aer aer.tex
# NB: tth works well on aer.tex
# cd ~/sw/crr;tth -a -Laer -p./:${TEXINPUTS}:${BIBINPUTS} < aer.tex > aer.html
# scp aer.html dust.ess.uci.edu:Sites/facts/aer
# NB: tex4ht works poorly on aer.tex
# cd ~/sw/crr;htlatex aer.tex # Works OK, index not linked
cd ~/sw/crr;htlatex aer.tex;makeindex -o aer.ind aer.4dx
cd ~/sw/crr;scp aer*.css aer*.html dust.ess.uci.edu:Sites/facts/aer
# NB: tex4moz works poorly on aer.tex
cd ~/sw/crr;/usr/share/tex4ht/mzlatex aer.tex
cd ~/sw/crr;scp aer*.css aer*.html aer*.xml dust.ess.uci.edu:Sites/facts/aer
} % end csznote on usage

\end{document}

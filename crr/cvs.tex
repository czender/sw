% $Id$

% Purpose: Summarize frequently used CVS commands

% Copyright (c) 1998--2014, Charles S. Zender
% Permission is granted to copy, distribute and/or modify this document
% under the terms of the GNU Free Documentation License (GFDL), Version 1.3
% or any later version published by the Free Software Foundation;
% with no Invariant Sections, no Front-Cover Texts, and no Back-Cover Texts.
% GFDL: http://www.gnu.org/copyleft/fdl.html

% The original author of this software, Charlie Zender, seeks to improve
% it with your suggestions, contributions, bug-reports, and patches.
% Charlie Zender <zender at uci dot edu>
% Department of Earth System Science
% University of California, Irvine
% Irvine, CA 92697-3100

% Usage: 
% cd ${HOME}/crr;make -W cvs.tex cvs.txt cvs.dvi cvs.ps cvs.pdf;cd -
% scp ${HOME}/crr/cvs.tex ${HOME}/crr/cvs.txt ${DATA}/ps/cvs.ps ${DATA}/ps/cvs.pdf ${HOME}/crr/cvs.dvi dust.ess.uci.edu:/var/www/html/doc/cvs

% # NB: latex2html chokes on cvs.tex
% latex2html -dir /var/www/html/doc/cvs cvs.tex
% # NB: tth works well on cvs.tex
% cd ${HOME}/crr;tth -a -Lcvs -p${TEXINPUTS} < cvs.tex > cvs.html
% scp cvs.html dust.ess.uci.edu:/var/www/html/doc/cvs
% # NB: tex4ht works well on cvs.tex
% cd ${HOME}/crr;htlatex cvs.tex
% scp cvs*.css cvs*.html dust.ess.uci.edu:/var/www/html/doc/cvs
% # NB: tex4moz works well on cvs.tex
% cd ${HOME}/crr;/usr/share/tex4ht/mzlatex cvs.tex
% scp cvs*.css cvs*.html cvs*.xml dust.ess.uci.edu:/var/www/html/doc/cvs

\documentclass[12pt,twoside]{article}

% Standard packages
\usepackage{ifpdf} % Define \ifpdf
\ifpdf % PDFLaTeX
\usepackage[pdftex]{graphicx} % Defines \includegraphics*
\pdfcompresslevel=9
\usepackage{thumbpdf} % Generate thumbnails
\usepackage{epstopdf} % Convert .eps, if found, to .pdf when required
\else % !PDFLaTeX
\usepackage{graphicx} % Defines \includegraphics*
\fi % !PDFLaTeX
\usepackage{array} % Table and array extensions, e.g., column formatting
\usepackage{lscape} % Landscape environment
\usepackage{makeidx} % Index keyword processor: \printindex and \see
\usepackage{natbib} % \cite commands from aguplus
\usepackage{times} % Postscript Times-Roman font KoD99 p. 375
\usepackage{url} % Typeset URLs and e-mail addresses

% hyperref is last package since it redefines other packages' commands
% hyperref options, assumed true unless =false is specified:
% backref       List citing sections after bibliography entries
% baseurl       Make all URLs in document relative to this
% bookmarksopen Unknown
% breaklinks    Wrap links onto newlines
% colorlinks    Use colored text for links, not boxes
% hyperindex    Link index to text
% plainpages=false Suppress warnings caused by duplicate page numbers
% pdftex        Conform to pdftex conventions
% Colors used when colorlinks=true:
% linkcolor     Color for normal internal links
% anchorcolor   Color for anchor text
% citecolor     Color for bibliographic citations in text
% filecolor     Color for URLs which open local files
% menucolor     Color for Acrobat menu items
% pagecolor     Color for links to other pages
% urlcolor      Color for linked URLs
\ifpdf % PDFLaTeX
\usepackage[backref,breaklinks,colorlinks,citecolor=blue,linkcolor=blue,urlcolor=blue,hyperindex,plainpages=false,pdftex]{hyperref} % Hyper-references
\pdfcompresslevel=9
\else % !PDFLaTeX
\usepackage[backref=false,breaklinks,colorlinks=false,hyperindex,plainpages=false]{hyperref} % Hyper-references
\fi % !PDFLaTeX

% Personal packages
\usepackage{csz} % Library of personal definitions
\usepackage{jrn_agu} % AGU-sanctioned journal title abbreviations

% Commands which must be executed in preamble
\makeindex % Index described on KoD95 p. 220

% Commands specific to this file
\newcommand{\CVScolon}{:}

% Margins
\topmargin -24pt \headheight 12pt \headsep 12pt
\textheight 9in \textwidth 6.5in
\oddsidemargin 0in \evensidemargin 0in
\marginparwidth 0pt \marginparsep 0pt
\footskip 24pt
\footnotesep=0pt

\setcounter{secnumdepth}{6}

\begin{document}

\begin{center}
Online version: \url{http://dust.ess.uci.edu/doc/cvs} \hfill \today\\
\bigskip
\bigskip
{\Large \textbf{CVS Cheat Sheet}}\\
\bigskip
by Charlie Zender\\
University of California, Irvine\\
\end{center}
Department of Earth System Science \hfill zender@uci.edu\\
University of California \hfill Voice: (949)\thinspace 891-2429\\
Irvine, CA~~92697-3100 \hfill Fax: (949)\thinspace 824-3256

\pagenumbering{roman}
\setcounter{page}{1}
\pagestyle{headings}
\thispagestyle{empty}
%\onecolumn
\tableofcontents
\pagenumbering{arabic}
\setcounter{page}{1}
%\markleft{CVS Cheat Sheet}
%\markright{}
\thispagestyle{empty}

\section{Software Configuration Management}
The process of orderly tracking, versioning, and serving, and
administration of project source code is called
\trmdfn{Software Configuration Management} (\trmdfn{SCM}).
This document describes two different software tools which aid in SCM:
\href{http://www.cvshome.org}{\trmidx{Concurrent Versioning System}}
(\trmidx{CVS}) and 
\href{http://subversion.tigris.org}{\trmidx{Subversion}} (\trmidx{SVN}).
These tools are intimately related since SVN is designed by some of
the original authors of CVS to replace CVS.
Currently, all of Section~\ref{sxn:cvs} (99\% of this document)
is devoted to CVS, while the newer portions, Section~\ref{sxn:svn},
is devoted to SVN.

\section{CVS}\label{sxn:cvs}

Section~\ref{sxn:cvs_gnr} presents the generic CVS commands.
The commands proceed in order of increasing complexity. 
After that, there are working examples from a number of specific
modules.  

\subsection{CVS Preliminaries}\label{sxn:cvs_prl}

\subsubsection{Abbreviations}
The following abbreviations are used:
\trmidx{CWD} means the working copy of the CVS module in and beneath
the \trmidx{Current Working Directory}; \cmdidx{mdl} = module name;
\cmdidx{fl} = file name; \cmdidx{brnch} = branch tag; 
\cmdidx{vrs} = version name (which can be a branch tag).

\subsubsection{CVS Keywords}
Only ``\texttt{cvs commit}'', ``\texttt{cvs tag}'', and ``\texttt{cvs
rtag}'' change the repository. All other CVS commands change, if
anything, the contents of the CWD only, and so are recoverable. 
CVS commands should be executed from the top level of the CWD.
In this case, the module name \cmdprn{mdl} may be optional at the
end of the command (because CVS finds the module name by looking in
the ``CVS'' subdirectory \verb'mdl/CVS'). 
Executing CVS commands from one level above the module requires
specification of the module name so that CVS knows where to find
\verb'mdl/CVS'. 

In addition to the usual RCS keywords (\texttt{Header}, \texttt{Date},
\ldots), CVS defines the keyword \texttt{Name}.  
The string ``\texttt{\$Name\$}'' expands to include the tag specified 
when the module was checked out, e.g., ``\texttt{cvs co -r ccm3\_5\_22
ccm}'' produces ``\texttt{\$Name\CVScolon ccm3\_5\_22 \$}'' but
``\texttt{cvs co ccm}'' produces ``\texttt{\$Name\CVScolon \$}'' for
two reasons. 
First, no tag was specified with the checkout so the default
\verb'ccm' module, i.e., the main trunk, is retrieved.
The main trunk has no default tag name, so the keyword value is an
empty string. 
Second, checking out the main trunk never expands
``\texttt{\$Name\$}'', but checking out a branch, any branch, does.
Branch tags are sticky, so a branch always has an associated tag
name. 
The rest of the RCS keywords expand to their usual meanings regardless
of whether the module is a tagged version or branch.
To explicitly turn off CVS keyword expansion when checking out a
module (recommended to avoid unnecessary conflicts due to CVS metadata
changes), use  the ``\texttt{-kk}'' option instead of the default,
``\texttt{-kkv}''.   
The \verb'-k' options are \textit{sticky}, meaning they apply to any
derived files as well.

\subsubsection{CVS Errors}
Occasionally CVS does not perform as expected.
Two errors are particularly frequent, and easy to fix.
The first is that, when asked to commit files to the repository, CVS
complains that some subset of the files is out-of-date, and asks you
to fix this before it proceeds. 
This is, in fact, not an error, but very helpful behavior by CVS
telling you that somehow your working directory is out of date with
respect to the repository. 
Some versions of CVS will commit those files that are modified but not
commit those that are out of date.
Other versions of CVS will not commit any file until all files in the
CWD are current.
In this case, simply perform a ``\texttt{cvs update}'' before
attempting to ``\texttt{cvs commit}'' again.
This should solve the problem.

The second error occurs when CVS attempts to modify the repository,
e.g., during a ``\texttt{cvs commit}'' operation, and it complains
that it has never heard of any of the files, e.g., ``\texttt{cvsbin
commit: nothing known about `to'}''. 
If this occurs on the CGD Sun network, it might be because you have
used simply \verb'cvs' rather than \verb'cvsbin' as the CVS command.
Check this and try your command again.

\subsubsection{CVS Debugging}
When attempting to solve problems with CVS, try using the
``\texttt{-t}'' switch so CVS will trace its execution.

\subsubsection{CVS Environment Variables}
CVS makes use of several \trmidx{environment variables}.
\begin{verbatim}
export CVS_RSH='ssh'    # Needed for ssh access to NCAR CGD CVS
export CVSROOT=':ext:dust.ess.uci.edu:/data/home/${USER}/cvs'
export CVSUMASK='002'   # Default file permissions for CVS
\end{verbatim}

\subsubsection{CVS Administration}
In order to execute the \cmdidx{cvs admin} command, the user must be a
member of the \flprn{cvsadmin} group, if it exists.
\begin{verbatim}
cvs admin -oREV::
cvs admin -mREV:MSG Replace the log message of revision REV with MSG.
\end{verbatim}

It took five years since I began using CVS in 1998 before I needed to
learn how to use CVS administrative commands to repair repositories.
On July~30, 2003 the laptop I was using ran out of power and somehow
wrote binary garbage into every file that \cmdidx{emacs} was
visiting. 
I only lost a few hours of work, and it seemed like time to backup 
everything I was working on.
That made things a little worse because I incidentally committed the
binary garbage to CVS. 
How did I remove the binary garbage from the repository?
\begin{verbatim}
# Replace corrupt file with latest working backup
cvs update -r 1.65 -p pnp.tex > pnp.tex
# Delete garbage file from repository
cvs admin -o 1.65:1.67 pnp.tex # Method 1: Collapse everything between 
cvs admin -o 1.66 pnp.tex # Method 2: Delete a specific version
# Delete garbage file from repository
cvs update -r 1.53 -p prp_mri.tex > prp_mri.tex
cvs admin -o 1.54 prp_mri.tex
\end{verbatim}

\subsubsection{Repository Relocation}\label{sxn:rlc}
Occassionally a CVS repository moves and one must adjust the
 repository location in all modules that were checked out from that
 repository. 
One method of doing this is simply deleting the old modules and then
 checking them out from the new location.
This works fine unless the module is not synchronized with the
 repository, i.e., if there is new work that has not been checked in. 
In this case it may be wiser to edit the repository location in the 
checked out modules, rather than re-checking out the modules.
\begin{verbatim}
# Modules in repository on esmf.ess.uci.edu 
mdl_lst='aca aeroce aeronet afgl anl anv arese avhrr c caco3 ccm ck cld clm crr
dmr dot dst elisp erbe ess ess_acc ess_atm ess_bnd ess_ccc ess_lsp
ess_phz ess_prc ess_rdn facts frc fsf grd hdf hire idea idl idx_rfr
igbp igpp improve ipcc job jrn linux lsm ltr map match mk
mny ncep ncl perl phd poetry pr prp rt rvw sdn slr_spc specfun tex
time toms uci www' 
mdl_lst='bxm c++ esmf f mie mie_ww ppr_BiZ03 ppr_BiZ04 ppr_FlZ05
ppr_GrZ04 ppr_ZMT04 ppr_ZeK05 ppr_ZeT06b ppr_ZeT06 prp_gtcp
prp_itr sltsbl'
mdl_lst='nco'
for mdl in ${mdl_lst} ; do
  cd ${HOME}/${mdl}
  printf "Updating module ${mdl}...\n"
  for fl in `find . -name Root -print` ; do
    printf "Updating Root name in ${fl}...\n"
#   echo 'zender@goldhill.cgd.ucar.edu:/home/zender/cvs' > ${fl}
#   echo 'zender@nco.cvs.sf.net:/cvsroot/nco' > ${fl}
#   echo 'zender@esmf.ess.uci.edu:/u/zender/cvs' > ${fl}
    echo ':ext:zender@nco.cvs.sf.net:/cvsroot/nco' > ${fl}
  done # end loop over fl
done # end loop over mdl
for mdl in ${mdl_lst} ; do
  cd ${HOME}/${mdl}
  printf "Updating module ${mdl}...\n"
  for fl in `find . -name Repository -print` ; do
    printf "Updating Repository name in ${fl}...\n"
    echo ${mdl} > ${fl}
  done # end loop over fl
done # end loop over mdl

# Modules in repository on goldhill.cgd.ucar.edu
aer -> /home/zender/match_dst/dst
cam2_0_2_dev41_brnchT_dust2
cam_dev
ccm_dst
ccsm2
ccsm2_2_beta08
clm2_deva_52
clm_dev
crm
match_dst

# Modules in repository on dust.ess.uci.edu
bxm
c++
esmf
f
mie
mie_ww
ppr_BiZ03
ppr_BiZ04
ppr_FlZ05
ppr_GrZ04
ppr_ZMT04
ppr_ZeK05
ppr_ZeT06b
ppr_ZeT06
prp_gtcp
prp_itr
sltsbl

# Modules in repository on sf.net
nco
\end{verbatim}

\subsection{Modules}\label{sxn:cvs_mdl}
To create a CVS repository, use
\begin{verbatim}
cvs -d ${HOME}/cvs init
\end{verbatim}
The following subsections present specific examples for both generic
and CGD-related modules. 
Many of the same examples are used for different modules, so that the
examples might differ only in the name of the target module, and
sometimes not at all (e.g., ``\texttt{cvs co}'').
The goal has been to keep the examples self-contained for each
module, at the expense of document length and redundancy. 
The examples are drawn from modules in active use in CGD.
Table~\ref{tbl:rps}\ summarizes the three repositories used in the
examples. 
\begin{landscape}
\begin{table}
\begin{center}
\begin{tabular}{l l >{\ttfamily}l<{} >{\ttfamily}l<{}}
\hline \rule{0.0ex}{\hlntblhdrskp}% 
Software & Machine & Repository Location & Module(s) \\[0.0ex]
\hline
\hline \rule{0.0ex}{\hlntblntrskp}%
\cxx, mie\footnote{\texttt{cvs -d :ext:dust.ess.uci.edu:/home/zender/cvs co -kk mie}} & \mchidx{dust.ess.uci.edu} & /home/zender/cvs & c++ mie \\[0.5ex]
NCO & \mchidx{cvs.nco.sourceforge.net} & /cvsroot/nco & nco \\[0.5ex]
CCM, CCM-Dust, CRM & \mchidx{goldhill.cgd.ucar.edu} & /fs/cgd/csm/models/CVS.REPOS & ccm ccm\_dst crm \\[0.5ex]
MATCH, MATCH-Dust & \mchidx{goldhill.cgd.ucar.edu} & /fs/cgd/csm/people/eaton/CVS & match match\_dst dst \\[0.5ex]
CAM & \mchidx{goldhill.cgd.ucar.edu} & /fs/cgd/csm/models/CVS.REPOS & cam2\_0\_2\_dev41\_brnchT\_dust2 \\[0.5ex]
\hline
\end{tabular}
\caption{CVS Repositories Used in Examples\label{tbl:rps}}   
\end{center}
\end{table}
\end{landscape}
Most of the examples do not explicitly specify the repository (with
the \verb'-d' argument).
This is because most commands are executed within, rather than above,
the CWD.
Table~\ref{tbl:rps}\  should be consulted when the repository to use
is in doubt. 

\subsection{CVS Server Issues}\label{sxn:cvs_srv}
A precise discussion of setting up a CVS server is given at
\url{http://www.korayguclu.de/index.php?&file=linux.cvs.pserver.xml}.
The CVS documentation describes the necessary modifications to the
internet \trmidx{daemon} configuration file \flidx{/etc/inetd.conf}.
\trmidx{RedHat Linux} uses a more powerful and complex
(some call this sophisticated) daemon, \cmdidx{xinetd} (pronounced 
``zy-net-d''), configured in \flidx{/etc/xinetd.conf}.
On RedHat systems, CVS password server services are controlled by a
file called \flidx{cvspserver} located in \flidx{/etc/xinetd.Dachau's}.
\begin{verbatim}
# Others
sudo scp ~/linux/etc/inetd.conf.dust /etc/inetd.conf
# RedHat
sudo scp ~/linux/etc/xinetd.d/cvspserver /etc/xinetd.d
sudo /etc/rc.d/init.d/xinetd restart
\end{verbatim}
The CVS documentation specifies how to configure the \cmdidx{port
  number} (2401), \cmdidx{socket\_type}, and \cmdidx{server\_args}
  flags.
Internet daemon services (i.e., \cmdprn{inetd} or \cmdprn{xinetd})
should be started automatically, in, say, system \trmidx{runlevel}~3. 

\subsubsection[Read-only Access]{Read-only Access}\label{sxn:cvs_read}
Place the \flidx{readers}, \flidx{writers}, and \flidx{passwd} file
in the directory \flprn{${CVSROOT}/CVSROOT}.
\begin{verbatim}
cd ${HOME}/cvs/CVSROOT
chmod 644 readers writers
chmod 444 readers writers
% cat > readers
sci_prg
jtalaman
mflanner
% cat > writers
zender
mflanner
% cat > passwd
sci_prg::cvspub
esmf::cvspub
\end{verbatim}
Add user \flidx{cvspub} to the system.
Do not give the user a home directory or interactive access.

Once read-only services work, outside users may check out 
modules by logging in as the \trmidx{anonymous user}.
\begin{verbatim}
cvs -d :pserver:sci_prg@dust.ess.uci.edu:/home/zender/cvs login
cvs -d :pserver:sci_prg@dust.ess.uci.edu:/home/zender/cvs co -kk c++
\end{verbatim}

\subsection{Generic Modules}\label{sxn:cvs_gnr}

\subsubsection[Import]{Import an existing directory}\label{sxn:cvs_mpr}
Import files in directory \verb'mdl' to create new CVS module
\verb'mdl'.    
The keywords ``zender'' (vendor tag) and ``mdl-0\_1'' (release tag)
are used for initial module tags. 
\begin{verbatim}
cd mdl
cvs import -m "Imported sources" mdl zender mdl-0_1
cvs -d :ext:dust.ess.uci.edu:/home/zender/cvs import -m "Imported sources" mdl zender mdl-0_1
cd ..
mv mdl mdl.bck
cvs co -kk mdl
ls -R mdl
/bin/rm -r mdl.bck
\end{verbatim}

\subsubsection[Import RCS]{Import RCS files}
Edit the CVS repository to create the appropriate source directories. 
Make sure all RCS files are unlocked, then copy them into the CVS
repository. 
\begin{verbatim}
cd mdl/RCS
rcs -u *
mkdir ${CVSROOT}/mdl
cp .*,v *,v ${CVSROOT}/mdl
cd ../..
mv mdl mdl.bck
cvs co mdl
ls -R mdl
/bin/rm -r mdl.bck
\end{verbatim}

\subsubsection[Checkout]{Checkout}
Checkout module \verb'mdl'. A \verb'-d' argument before the
verb specifies the CVS repository to use (instead of
\texttt{\$CVSROOT}). %$ NB: Placed here to satisfy Emacs Highlighting regex. 
By default, a module \verb'mdl' is placed in the \verb'mdl' directory
of the CWD. 
A \verb'-d' argument after the verb specifies an arbitrary directory
for the module.
The \verb'-kk' option suppresses RCS keyword expansion (e.g., of
``\texttt{\$Header\CVScolon: \$}''), thereby minimizing the number of
conflicts during a future \texttt{cvs update}. 
Note that \verb'co' is a synonym for \verb'checkout'---the two are
interchangeable. 
\begin{verbatim}
cvs co mdl     # Expands keywords
cvs co -kk mdl # Does not expand keywords. Use prior to cvs update.
cvs checkout -d drc mdl
cvs -d /fs/cgd/csm/models/CVS.REPOS co mdl
cvs -d /fs/cgd/csm/models/CVS.REPOS co -d drc mdl
\end{verbatim}

\subsubsection[Add]{Add}
Schedule file \verb'fl' for addition to the repository.  
Actual addition takes effect with the next ``\texttt{cvs commit}''
command.\\  
\verb'cvs add fl'

\subsubsection[Remove]{Remove}
Schedule file \verb'fl' for removal from the repository.  
Actual removal takes effect with the next ``\texttt{cvs commit}''
command.\\ 
\verb'cvs remove fl'

\subsubsection[Query]{Query}
Show changes of CWD relative to repository. 
Option \verb'-n' specifies that no changes to the repository will
occur.\\  
\verb'cvs -n update'

\subsubsection[Diff]{Diff}
Show changes relative to particular versions, tags, or times.
\begin{verbatim}
cvs diff -kk main.c
cvs diff -r ccm3_6 -kk main.c
cvs diff -r 1.1 -kk main.c
cvs diff -D "last week" -kk main.c
cvs diff -D "4 days ago" -kk main.c
cvs diff -D "3/12/98" -kk main.c
\end{verbatim}
Revert repository to state it was in 24~hours before.
Use \verb'-p' switch to overwrite version in current directory.
This prevents CVS from applying sticky flag to new versions.
\begin{verbatim}
cvs update -D "24 hours ago" -p ncap.h > ncap.h
cvs update -D "24 hours ago" -p ncap_lex.l > ncap_lex.l
cvs update -D "24 hours ago" -p ncap_yacc.y > ncap_yacc.y
cvs update -D "24 hours ago" -p ncap_utl.c > ncap_utl.c
\end{verbatim}
Checkout a date-stamped version of a module:
\begin{verbatim}
cvs -z3 -d zender@nco.cvs.sf.net:/cvsroot/nco co -D 20050524 -d nco-20050524 nco
\end{verbatim}
Compare module files from two different dates:
\begin{verbatim}
cvs diff -c --ignore-all-space -kk -D 20050519 -D 20050520 
\end{verbatim}

\subsubsection[Commit]{Commit}
Commit changes beneath CWD. 
The \verb'commit' verb accepts optional filename arguments for
file-by-file (rather than entire module) commits.
\begin{verbatim}
cvs commit                # Invokes editor for log message
cvs commit -m "fixed bug" # Uses "fixed bug" for log message
cvs commit README         # Only commits README file
\end{verbatim}

\subsubsection[Tag]{Tag}
Tag CWD with \verb'mdl-1_0'. 
Option \verb'-c' causes \verb'tag' to verify that files beneath CWD
are not modified relative to the repository. 
This ensures the repository has all the information needed to exactly
reproduce the CWD from the tag name in the future (with, e.g.,
``\texttt{cvs co -r mdl1\_0 mdl}''.
\begin{verbatim}
cvs tag -c mdl-1_0     # From top-level of CWD
cvs tag -c mdl-1_0 mdl # Above top-level of CWD
\end{verbatim}

\subsubsection[Rtag]{Rtag}
``Repository'' tag (\verb'rtag') the module with \verb'tag_nm'.
Verb \verb'rtag', as opposed to \verb'tag', operates on the
repository, not the CWD.  
Does this only tag those portions of the repository in and beneath the
CWD? (fxm).
By default, \verb'rtag' tags the most recent version of the
module. 
This can be overridden with \verb'-D date' option.
Use (\verb'rtag') with caution on branches, because it will update the
main repository as well (fxm).\\
\verb'cvs rtag tag_nm'

\subsubsection[Release]{Release}
Release module \verb'mdl' and delete its working
directories. 
CVS prompts the user whether to actually delete the directory:\\   
\verb'cvs release -d mdl # From level above CWD' 

\subsubsection[Update]{Update}
Update the CWD for \verb'mdl' so that it reflects the latest 
version of the repository. 
One, and only one, \verb'-j' option, ``\texttt{-j vrs}'', updates
the CWD of \verb'mdl' to the latest revision of the \textit{ancestor}
of \verb'vrs'.  
The ancestor is the model from which the \verb'vrs' branch split.
Given two \verb'-j' options, ``\texttt{-j vrs1 -j vrs2}'', 
the \verb'update' command takes the differences (uses \verb'diff')
between version \verb'vrs1' and version \verb'vrs2' and
applies them (uses \verb'patch') to the CWD.
The \verb'-kk' option suppresses RCS keyword expansion, thereby
minimizing the number of conflicts during an update.
For this reason, it is wise to use \texttt{cvs co -kk mdl}
before using \texttt{cvs update}.
\begin{verbatim}
cvs update      # Update all files in module
cvs update *    # Update currently checked-out files only
cvs update mdl  # Above top-level of CWD
cvs update -kk
cvs update -kk -d # Add directories not in working copy
cvs update -j vrs -kk
cvs update -j vrs1 -j vrs2 -kk
cvs update -kk -p params.h > foo        # Redirect file to stdout
# Replace modified file with repository file
cvs update -kk -p params.h > params.h   
\end{verbatim}

\subsection{C++ Module}\label{sxn:cvs_c++}
The C++ module \verb'c++' was imported using ``\texttt{cvs import -m
"Imported sources" c++ zender c++-1\_0}''. 

\subsubsection[Checkout]{Checkout}
Checkout the latest version of the C++ module. 
\begin{verbatim}
# From NCAR, from ~
cvsbin co c++
# From home, from ~
cvs -d :pserver:zender@bearmtn.cgd.ucar.edu:/home/zender/cvs login
cvs -d :pserver:zender@bearmtn.cgd.ucar.edu:/home/zender/cvs co -kk c++
# From UCI, from ~
cvs -d :ext:zender@goldhill.cgd.ucar.edu:/home/zender/cvs co -kk c++
\end{verbatim}

\subsubsection[Commit]{Commit}
Commit C++ CWD to remote repository.
One of the nicest feature about CVS is that it knows about remote
repositories, so commands like \verb'commit' and \verb'update' need no
extra arguments.
\begin{verbatim}
# Update CWD at home and continue working
cvs update c++ # From home ~/
# Commit CWD at home to repository at NCAR
cvs commit c++ # From home ~/
\end{verbatim}

\subsection{NCO Module}\label{sxn:cvs_nco}

The NCO module \verb'nco' was imported using ``\texttt{cvs import -m
"Imported sources" nco zender nco-1\_0}''. 

\subsubsection[Checkout]{Checkout}
Checkout the latest version of the NCO module. 
\begin{verbatim}
# From home, from ~/nc
cvs -d :pserver:zender@bearmtn.cgd.ucar.edu:/home/zender/cvs login
cvs -d :pserver:zender@bearmtn.cgd.ucar.edu:/home/zender/cvs co nco
\end{verbatim}

\subsubsection[Update]{Update}
Update a remote module (i.e., at home) to the current NCAR NCO
repository. 
\begin{verbatim}
cvs status
cvs -n update
cvs update      # Update all files in module
cvs update *    # Update currently checked-out files only
\end{verbatim}

\subsection{CRM Module}\label{sxn:cvs_ccm_crm}

The CRM module \verb'crm' was created as a branch of CCM version
3.5.22.
First, we defined module \verb'crm' by editing the
\verb'/fs/cgd/csm/models/CVS.REPOS/CVSROOT/modules' file to contain
the following lines: 
\begin{verbatim}
...
# CCM Column models
crm atm/ccm_crm &ccm_crm_src
ccm_crm_src -d src atm/ccm_crm_src &eul &physics &ccmlsm_share &dom &csm_share &
srchutil &control
...
\end{verbatim}
Finally, we executed the following commands to create the CRM branch
of the CCM:
\begin{verbatim}
cvs co -r ccm3_5_22 crm
cd src/crm
cvs tag -b ccm_brnch_crm 
cd src/control
cvs tag -b ccm_brnch_crm [list of files in control needed by CRM]
cd src/physics
cvs tag -b ccm_brnch_crm [list of files in physics needed by CRM]
...
\end{verbatim}
Note that we used \verb'tag' rather than \verb'rtag' in the above
because only \verb'tag' allows one to tag a specified subset of the
files in a given module.
In this case, the desired subset was the CCM files which are also
required by the CRM, but no others (this excluded superfluous files,
e.g., \verb'vdiff.F').
All work on the CRM module is actually performed on this
\verb'ccm_brnch_crm' branch of the CRM, rather than the CRM main trunk
This methodology, i.e., working on the branch rather than the main
trunk, is also used for the Dust module component of CCM, MATCH, and
MOZART described in Sections~\ref{sxn:cvs_ccm_dst}, \ref{sxn:cvs_match_dst},
and \ref{sxn:cvs_mozart_dst} respectively.  

\subsubsection[Create Distribution]{Create Distribution}
Release a new CRM.
This process uses many CVS commands.
First, make sure the \verb'crm' module in the repository is
up-to-date, so that the CWD can be exactly reproduced.
Second, release the module. 
Third, check out the CRM with the \verb'-kk' option. 
Fourth, difference the CWD with the most recent branch tag.
This difference file shows the differences between the releases.
Do these differences make sense?
Fifth, tag the module.
Sixth, execute the distribution script.
\begin{verbatim}
cd ..                   # Move above top-level of CWD
cvs commit crm          # Above top-level of CWD
cvs release -d crm      # Above top-level of CWD
cvs -d /fs/cgd/csm/models/CVS.REPOS co -r ccm_brnch_crm -kk crm
cvs diff -kk -r ccm3_6 crm > pre.diff
cvs tag -c ccm3_6_brnchT_crm2_0 crm
crm_dst.pl --dbg=1 ccm3_6_brnchT_crm2_0
\end{verbatim}

\subsubsection[Update]{Update to newer CCM version}
Update the CWD to a newer CCM version.
This process uses many CVS commands.
First, checkout a clean copy of the CRM branch into the CWD.
Second, difference the pre-merged CRM from its current CCM base.
This difference file shows all the CRM-specific modifications to the
current CCM base code.
Third, update the CWD to the desired CCM version.
Given two \verb'-j' options, (``\texttt{-j ccm3\_5\_22 -j
ccm3\_6\_0}''), the \verb'update' command takes the differences (uses
\verb'diff') from CCM version \verb'ccm3_5_22' to CCM version
\verb'ccm3_6_0' and applies them (uses \verb'patch') to the CWD.
Fourth, examine and fix the conflicts caused by this merge.
Use, e.g., ``\texttt{cvs status}'' to locate conflicts.
Make sure to compile, test, and generate new *.out files for the
updated model.
This ensures that any new source files will be included in
\verb'Srcfiles' and \verb'Depends'.
Also, it is important to compile with
``\texttt{-DSINGLE\_SOURCE\_FILE}'' option.
This verifies that any new files merged into, and needed by, the CRM
are accounted for.
Unfortunately, there is no simple way to ensure that superfluous
source files have not crept into the CRM.
Fifth, difference the post-merged CRM from the new CCM base.
This difference file shows all the CRM-specific modifications to the
new CCM base code.
Sixth, difference the two difference files.
This file highlights conflicts caused by the merge process and,
hopefully, any mistakes CVS made in performing the update.
Seventh, commit the updated code to the head of the branch.
Eighth, if desired, tag the branch as a new release of CRM.
\begin{verbatim}
# Above top-level of CWD
cvs -d /fs/cgd/csm/models/CVS.REPOS co -r ccm_brnch_crm -kk crm
cvs diff -kk -r ccm3_5_22 crm > pre.diff
cvs update -j ccm3_5_22 -j ccm3_6 crm
cvs status crm | grep -i conflict
cvs diff -kk -r ccm3_6 crm > post.diff
diff pre.diff post.diff > diff.diff
cvs commit crm
cvs tag -c ccm3_6_brnchT_crm2_0 crm
\end{verbatim}

\subsubsection[Checkout]{Checkout}
Checkout the latest version of the CRM. 
By default, the \verb'crm' module is placed in the \verb'crm'
directory of the CWD. 
A \verb'-d' argument after the verb specifies an arbitrary
directory.
\begin{verbatim}
cvs co -r ccm_brnch_crm -kk crm
cvs -d /fs/cgd/csm/models/CVS.REPOS co -r ccm_brnch_crm -kk crm
cvs -d /fs/cgd/csm/models/CVS.REPOS co -r ccm_brnch_crm -d drc crm
cvs -d :pserver:zender@goldhill.cgd.ucar.edu:/fs/cgd/csm/models/CVS.REPOS \
co -r ccm_brnch_crm -kk crm
\end{verbatim}
% cvs -d :pserver:zender@goldhill.cgd.ucar.edu:/fs/cgd/csm/models/CVS.REPOS co -r ccm_brnch_crm -kk crm
% From UCI
% cvs -d :ext:zender@goldhill.cgd.ucar.edu:/fs/cgd/csm/models/CVS.REPOS co -r ccm_brnch_crm -kk crm

\subsubsection[Export]{Export}
Export the latest version of the module \verb'crm' from CCM
branch \verb'ccm_brnch_crm' into directory
\verb'/data/zender/crm-1.1' and prepare a compressed tarfile
distribution.  
The \verb'export' command strips all the CVS directories from the
output.  
This is most useful for creating distributions for public release. 
Using \verb'-kkv' instead of \verb'-kv' would expand CVS keywords into
keywords plus values. 
\begin{verbatim}
/bin/rm -r -f /data/zender/crm-1.1
cvs export -kv -r ccm_brnch_crm -d /data/zender/crm-1.1 crm
cd /data/zender; gtar -cvzf crm-1.1.tar.gz ./crm-1.1
\end{verbatim}

\subsubsection[Tag]{Tag}
Tag CWD with \verb'ccm3_6_brnchT_crm2_0'. 
Option \verb'-c' causes \verb'tag' to verify that files beneath CWD
are not modified relative to the repository. 
This ensures the repository has all the information needed to exactly 
reproduce the CWD from the tag name in the future. 
Note it is wise to tag only modules which have been checked out  
with the \verb'-kk' option.
This prevents large numbers of trivial differences between tagged
versions.\\  
\verb'cvs tag -c ccm3_6_brnchT_crm2_0'

\subsubsection[Remote Checkout]{Remote Checkout}
Checkout the latest version of the CRM from home. 
The first \verb'-d' option specifies the remote repository.
The second \verb'-d' option specifies the name of the directory to place the
module in.
The final argument, \verb'crm', is the name of the module to checkout.
For unknown reasons this command erroneously places the src directories one
level too high in most cases.
This is apparently a CVS bug fixed in later versions of CVS.
\begin{verbatim}
cvs -d :ext:zender@goldhill.cgd.ucar.edu:/fs/cgd/csm/models/CVS.REPOS \
co -r ccm_brnch_crm -N -d crm crm
\end{verbatim}
Here is a workaround for the above problem:
\begin{verbatim}
mv ccmlsm_share dom physics srchutil control csm_share eul src
mv src crm
\end{verbatim}

\subsection{CCM-Dust Module}\label{sxn:cvs_ccm_dst}
The module \verb'ccm_dst' comprises the sub-modules \verb'ccm' and
\verb'dst'. 
Thus \verb'ccm_dst' is the complete CCM with the dust modifications. 
Code modifications related to dust appear in the usual files in
\verb'src' (e.g., file \verb'physics/aphys.F'). 
Most of the dust physics are isolated in the new directory
\verb'src/dust'.  
The branch \verb'ccm_brnch_dst' was created as a branch of CCM version
3.5.22.
First, we defined module \verb'ccm_dst' by editing the
\verb'/fs/cgd/csm/models/CVS.REPOS/CVSROOT/modules' file to contain
the following lines: 
\begin{verbatim}
...
# Atmospheric models
ccm             atm/ccm &ccm_src &ccm_tools
ccm_dst         atm/ccm &ccm_dust_src &ccm_tools
...
#  Sub-modules to ccm
ccm_src -d src atm/ccm_src &dynamics &physics &control &spmd &dom &som \
                &lsm &ccmlsm_share &csm_share &mathutil &srchutil
ccm_dust_src -d src atm/ccm_src &dst &dynamics &physics &control &spmd &dom &som \
                &lsm &ccmlsm_share &csm_share &mathutil &srchutil
...
physics         atm/ccm_src_dirs/physics
dst             atm/ccm_src_dirs/dust
...
\end{verbatim}
Finally, we executed the following commands to create the Dust branch
of the CCM:
\begin{verbatim}
cvs co ccm_dst
cd ccm_dst
cvs tag -b ccm_brnch_dst
\end{verbatim}

\subsubsection[Checkout]{Checkout}
Checkout the latest version of the CCM-Dust module. 
A \verb'-d' argument after the verb specifies an arbitrary directory
for the module.
\begin{verbatim}
cd /fs/cgd/data0/zender
cvsbin -d /fs/cgd/csm/models/CVS.REPOS co -r ccm_brnch_dst -kk ccm_dst
# From home, from ~/fsh/dst 
cvs -d :ext:zender@goldhill.cgd.ucar.edu:/fs/cgd/csm/models/CVS.REPOS \
co -r ccm_brnch_dst -kk ccm_dst
cvs -d :pserver:zender@goldhill.cgd.ucar.edu:/fs/cgd/csm/models/CVS.REPOS \ 
co -r ccm_brnch_dst -kk ccm_dst 
# From UCI
cvs -d :ext:zender@goldhill.cgd.ucar.edu:/fs/cgd/csm/models/CVS.REPOS co -r ccm_brnch_dst -kk ccm_dst 
# CCM Box model only
cvs -d :ext:zender@goldhill.cgd.ucar.edu:/fs/cgd/csm/models/CVS.REPOS co -r ccm_brnch_dst -kk -d aer dst
# Standard CCM, from UCI
cvs -d :ext:zender@goldhill.cgd.ucar.edu:/fs/cgd/csm/models/CVS.REPOS co -kk -d ccm_o2x ccm
cvs -d :ext:zender@goldhill.cgd.ucar.edu:/fs/cgd/csm/models/CVS.REPOS co -r ccm3_10 -d ccm_o2x ccm
# Source directory only 
cvs -d :ext:zender@goldhill.cgd.ucar.edu:/fs/cgd/csm/models/CVS.REPOS co -r ccm3_10 -d src ccm_src
cvs -d :ext:zender@goldhill.cgd.ucar.edu:/fs/cgd/csm/models/CVS.REPOS co -kk -d src ccm_src
# Then when updating parent, use cvs update -l -kk for local update only
# Marianna's dust CLM
cvs -d :ext:zender@goldhill.cgd.ucar.edu:/fs/cgd/csm/models/CVS.REPOS co -r clm2_deva_52 -d clm2_deva_52 clm2
# Natalie's dust CAM/CLM
cvs -d :ext:zender@goldhill.cgd.ucar.edu:/fs/cgd/csm/models/CVS.REPOS co -r cam2_0_2_dev41_brnchT_dust2 -d cam2_0_2_dev41_brnchT_dust2 cam1
# Head of CAM 
cvs -d :ext:zender@goldhill.cgd.ucar.edu:/fs/cgd/csm/models/CVS.REPOS co -r cam_dev -d cam_dev cam1
# SNICAR
cvs -d :ext:zender@goldhill.cgd.ucar.edu:/fs/cgd/csm/models/CVS.REPOS rtag -b -r cam3_0_25 cam3_0_25_brnch_snicar cam1
cvs -d :ext:zender@goldhill.cgd.ucar.edu:/fs/cgd/csm/models/CVS.REPOS co -kk -r cam3_0_25_brnch_snicar -d cam_snicar cam1
\end{verbatim}
% cvs -d :pserver:zender@goldhill.cgd.ucar.edu:/fs/cgd/csm/models/CVS.REPOS co -r ccm_brnch_dst -kk ccm_dst

\subsubsection[Query]{Query}
Show which files in CWD have changed relative to the most recent
committed version in the Dust repository.
Then show precisely what has changed for file \verb'dstmbl.F'.
\begin{verbatim}
cvs -nq update -kk
\end{verbatim}
The result of \verb'cvs -n update' is a list of files which differ
from the repository. 
Each file is prefixed by a code.
The most common codes are \verb'C' (for conflicts), \verb'M'
(local modifications newer than repository), and \verb'U' (local file
needs updating).
Conflicts 
\begin{verbatim}
/home/zender/f: cvs -n update
cvsbin update: Updating .
C Makefile
M csz_f77.F
U sng.F
\end{verbatim}
The precise differences relative to the repository are shown with
\begin{verbatim}
cvs diff -kk
\end{verbatim}
Show which files in CWD have changed relative to version
\verb'mdl-1_0' in the CCM repository.
Then show precisely what has changed for file \verb'Makefile'.
\begin{verbatim}
cvs -nq update -r mdl-1_0 -kk
cvs diff -r mdl-1_0 -kk Makefile
\end{verbatim}

\subsubsection[Release]{Release}
Release the current Dust module and check out a new working copy. 
\begin{verbatim}
# Above top-level of CWD
cvs -d /fs/cgd/csm/models/CVS.REPOS release -d ccm_dst
cvs -d /fs/cgd/csm/models/CVS.REPOS co -r ccm_brnch_dst -kk ccm_dst
\end{verbatim}

\subsubsection[Tag]{Tag}
Note that tagging in the CCM repository invokes error checking modules
and, by default, launches windows which request one line summaries
on the local machine.
Attempting to tag from behind a firewall or proxy can thus result in 
failure because the windows are unable to open on the requested
console.
In this case, the tagging must be done from
\flprn{goldhill.cgd.ucar.edu}, and the user's \cmdidx{DISPLAY} must be
set to \verb'NONE'
\begin{verbatim}
export DISPLAY=NONE
\end{verbatim}
This problem does not arise when tagging from an unfiltered IP address.

Tag CWD with \verb'ccm3_6_brnchT_dst1_2'. 
Option \verb'-c' causes \verb'tag' to verify that files beneath CWD
are not modified relative to the repository. 
This ensures the repository has all the information needed to exactly
reproduce the CWD from the tag name in the future.
\begin{verbatim}
cd ccm_dst              # Move to top-level of CWD
cvs tag -c ccm3_6_brnchT_dst1_2
\end{verbatim}

\subsubsection[Update]{Update}
Update the CWD to a newer CCM version.
\begin{verbatim}
# Above top-level of CWD
cvs -d /fs/cgd/csm/models/CVS.REPOS co -r ccm_brnch_dst -kk ccm_dst
cvs -d /fs/cgd/csm/models/CVS.REPOS diff -kk -r ccm3_5_22 ccm_dst > pre.diff
cvs -d /fs/cgd/csm/models/CVS.REPOS update -j ccm3_5_22 -j ccm3_6 ccm_dst
cvs -d /fs/cgd/csm/models/CVS.REPOS status ccm_dst | grep -i conflict
cvs -d /fs/cgd/csm/models/CVS.REPOS diff -kk -r ccm3_6 ccm_dst > post.diff
diff pre.diff post.diff > diff.diff
cvs commit ccm_dst
cvs tag -c ccm3_6_brnchT_dst1_0 ccm_dst
\end{verbatim}

\subsection{MATCH-Dust Module}\label{sxn:cvs_match_dst}
The module \verb'match_dst' comprises the sub-modules \verb'match' and
\verb'dst'. 
Thus \verb'match_dst' is the complete MATCH with the dust modifications. 
Code modifications related to dust appear in the usual files in
\verb'src' (e.g., file \verb'physlic.F'). 
Most of the dust physics are isolated in new files in the \verb'dst'
directory, e.g., \verb'dstmbl.F'.
The branch \verb'match_dst-0' was created as a branch of MATCH
Spitfire version 3.2.9.
First, we defined module \verb'match_dst' by editing the
\verb'/fs/cgd/csm/people/eaton/CVS/CVSROOT/modules' file to contain
the following lines: 
\begin{verbatim}
match           match
matchSrc        -d src match/src
matchReaders    -d readers match/readers
...
dst             dust
match_dst       match &dst
\end{verbatim}
The \verb'dust' source directory in the MATCH repository is actually a
symbolic link to the dust source in the CSM repository: 
\begin{verbatim}
cd /fs/cgd/csm/people/eaton/CVS
ln -s /fs/cgd/csm/models/CVS.REPOS/atm/ccm_src_dirs/dust .
\end{verbatim}
Next, we added a MATCH tag to the CCM branch of the Dust module.
\begin{verbatim}
cvs -d /fs/cgd/csm/models/CVS.REPOS co -kk -r ccm_brnch_dst dst
cd dst
cvs tag -c MATCH_SPITFIRE-3_2_9
\end{verbatim}
This enabled us to check out both MATCH and Dust at the same time in
preparation for the next step, creating the MATCH-Dust branch.
Note that in order to apply MATCH tags to files in the CCM repository,
we had to disconnect the CCM-Dust module from the default CCM tag
syntax checking routines.
Finally, we created and tagged the MATCH-Dust branch.
\begin{verbatim}
cvs -d /fs/cgd/csm/people/eaton/CVS rtag -b -r MATCH_SPITFIRE-3_2_9 \
match_brnch_dst match_dst
\end{verbatim}
In the preceding command, \verb'MATCH_SPITFIRE-3_2_9' is the
pre-existing tag associated with all files that were placed in the 
new \verb'match_brnch_dst' branch of the \verb'match_dst' module.
Originally, we left the \verb'MATCH_SPITFIRE-3_2_9' tag on the files
in the \verb'dst' directory, but this caused problems when merging
changes to the main trunk of the MATCH into MATCH-Dust.
The tag caused CVS to think that the Dust-specific files were parts of
\verb'MATCH_SPITFIRE-3_2_9', and thus CVS tried to remove them when
updating to a later version on the main trunk.
Thus, we later removed the \verb'MATCH_SPITFIRE-3_2_9' tag from all
the files in the \verb'dst' directory.

\subsubsection[Repository changes]{Changes to standard MATCH repository}
Once these commands were completed, we slightly altered the
architecture of the MATCH-Dust repository to facilitate CCM-style
compiling methods (i.e., compile from list of directories rather than
list of individual files plus directories).
\verb'match_brnch_dst' was created from the MATCH 3.2 repository,
and these commands were used to modify the standard MATCH repository:
\begin{verbatim}
# Routine to read CCM dynamics data conflicts with NCEP reader routine.
# Store it in parallel, non-conflicting, directory:
cd /fs/cgd/data0/zender/match_dst/readers
mkdir ccm
cvs add ccm
cd ccm
mv /fs/cgd/data0/zender/match_dst/src/dyninp.F .
cvs remove /fs/cgd/data0/zender/match_dst/src/dyninp.F
cvs add dyninp.F

/bin/rm extoro.F setzen.F # Remove relics from Mark Lawrence
cvs remove extoro.F setzen.F
/bin/rm cloud_dum.F # Routines superceded by prognostic cloud water
cvs remove cloud_dum.F

cd ..
cvs commit -m "Rearranged files to work with MATCH-Dust Makefile"
\end{verbatim}
To create a dust branch from the standard MATCH 4.X repository, the
following changes to the standard directory structure would be
required: 
\begin{verbatim}
# Routine to read CCM dynamics data conflicts with NCEP reader routine.
# Store it in parallel, non-conflicting, directory:
cd /fs/cgd/data0/zender/match_dst/readers
mkdir ccm
cvs add ccm
cd ccm
mv /fs/cgd/data0/zender/match_dst/src/dyninp.F .
cvs remove /fs/cgd/data0/zender/match_dst/src/dyninp.F
cvs add dyninp.F

# Remove routines in libncaru.a and libmss.a on NCAR computers
/bin/rm setzen.F # Remove relics from Mark Lawrence
cvs remove setzen.F

cd ..
cvs commit -m "Rearranged files to work with MATCH-Dust Makefile"
\end{verbatim}

\subsubsection[Checkout]{Checkout}
Checkout the latest version of the MATCH-Dust module. 
A \verb'-d' argument after the verb specifies an arbitrary directory
for the module.
\begin{verbatim}
# Get MATCH-Dust on Dataproc
cd /fs/cgd/data0/zender
cvs -d /fs/cgd/csm/people/eaton/CVS co -r match_brnch_dst -kk match_dst
# Get MATCH-Dust in CGD
cvs -d :pserver:zender@goldhill.cgd.ucar.edu:/fs/cgd/csm/people/eaton/CVS \
co -r match_brnch_dst -kk match_dst 
# Get MATCH-Dust at UCI
cvs -d :ext:zender@goldhill.cgd.ucar.edu:/fs/cgd/csm/people/eaton/CVS co -r match_brnch_dst -kk match_dst 
# Get Dust box model at UCI (i.e., checkout the dst directory, name it aer)
cvs -d :ext:zender@goldhill.cgd.ucar.edu:/fs/cgd/csm/people/eaton/CVS co -r match_brnch_dst -kk -d aer dst
# Get Assimilation model at UCI
cvs -d :ext:zender@goldhill.cgd.ucar.edu:/fs/cgd/csm/people/eaton/CVS co -r match3_3_24_ind1_19 -kk -d ${HOME}/match_asm match
cvs -d :ext:zender@goldhill.cgd.ucar.edu:/fs/cgd/csm/people/eaton/CVS co -r match3_3_24_ind1_19 -kk -d ${HOME}/match_asm/dst dst
# Create Pat's branch
cvs -d :ext:zender@goldhill.cgd.ucar.edu:/fs/cgd/csm/people/eaton/CVS rtag -b -r match-3_3_23-dst-0_9_8 match_brnch_dst_pjones match_dst
# Checkout Pat's branch (``old'' model)
cvs -d :ext:zender@goldhill.cgd.ucar.edu:/fs/cgd/csm/people/eaton/CVS co -r match_brnch_dst_pjones -kk -d match_dst_old match_dst 
# Create Alf's sandblasting branch
cvs tag -b -r match-4_0_beta2-dst-1_1_1 match_brnch_dst_sbl match_dst
# Checkout Alf's sandblasting branch
cvs -d :ext:zender@goldhill.cgd.ucar.edu:/fs/cgd/csm/people/eaton/CVS co -r match_brnch_dst_sbl -kk -d match_dst_sbl match_dst 
# Get Dust box model from ESMF (i.e., model directory is named dead)
cvs -d :ext:zender@esmf.ess.uci.edu:/u/zender/cvs co -kk -r match_brnch_dst dead
\end{verbatim}

\subsubsection[Tag]{Tag}
Tag CWD with \verb'match-3_2_10-dst-1_3'.
Option \verb'-c' causes \verb'tag' to verify that files beneath CWD
are not modified relative to the repository. 
This ensures the repository has all the information needed to exactly
reproduce the CWD from the tag name in the future.
Be sure this version compiles before tagging it.
\begin{verbatim}
cd match_dst            # Move to top-level of CWD
cvs tag -c match-3_2_10-dst-1_3
\end{verbatim}

\subsubsection[Merge CCM into CCM-Dust]{Merge CCM changes into CCM-Dust}
Merge changes to main CCM trunk into the CCM-Dust branch. 
Merging main trunk changes into CCM-Dust should never change the
\verb'dst' directory itself, but should alter the contents of
\verb'src/physics', \verb'src/control', etc. 
This example merges the changes from CCM version
\verb'ccm3_6' to CCM version
\verb'ccm3_6_6' into CCM-Dust version
\verb'ccm3_6_brnchT_dst-0_9_6', which is then tagged as
\verb'ccm3_6_6_brnchT_dst-0_9_6'. 
\begin{verbatim}
cd /fs/cgd/data0/zender/ccm_dst
cvs diff -kk -r ccm3_6 > pre.diff
cvs update -kk -j ccm3_6 -j ccm3_6_6
cvs status | grep -i conflict
cvs diff -kk -r ccm3_6_6 > post.diff
diff pre.diff post.diff > diff.diff
make ccm
cvs commit 
cvs tag -c ccm3_6_6_brnchT_dst-0_9_6
\end{verbatim}

\subsubsection[Merge MATCH into MATCH-Dust]{Merge MATCH changes into MATCH-Dust}
Merge changes to main MATCH trunk into the MATCH-Dust branch. 
Merging main trunk changes into MATCH-Dust should never change the
\verb'dst' directory itself, but should alter the contents of
\verb'src', \verb'readers', etc. 
This example merges the changes from MATCH version
\verb'match3_2_10_scyc1_0' to MATCH version
\verb'match3_3_20' into MATCH-Dust version
\verb'match-3_2_10-dst-1_4a', which is then tagged as
\verb'match-3_3_20-dst-1_4a'. 
\begin{verbatim}
cd /fs/cgd/data0/zender/match_dst
cvs diff -kk -r match3_2_10_scyc1_0 > pre.diff
cvs update -kk -j match3_2_10_scyc1_0 -j match3_3_20
cvs status | grep -i conflict
cvs diff -kk -r match3_3_20 > post.diff
diff pre.diff post.diff > diff.diff
make match
cvs commit 
cvs tag -c match-3_3_20-dst-1_4a
\end{verbatim}

\subsubsection[Merge MATCH-Dust into CCM-Dust]{Merge MATCH-Dust changes into CCM-Dust}
Merge changes to the MATCH-Dust branch into the CCM-Dust branch. 
A simple ``\texttt{cvs update}'' does not work because the changes are
on different branches.
This example merges the changes from MATCH-Dust version
\verb'match-3_2_9-dst-1_2' to MATCH-Dust version
\verb'match-3_2_10-dst-1_4' into CCM-Dust version
\verb'ccm3_6_brnchT_dst1_2', which is then tagged as
\verb'ccm3_6_brnchT_dst1_4'.  
\begin{verbatim}
cd /fs/cgd/data0/zender/ccm_dst/src/dst
cvs diff -kk -r ccm3_6_brnchT_dst1_2 > pre.diff
cvs update -kk -j match-3_2_9-dst-1_2 -j match-3_2_10-dst-1_4
cvs status | grep -i conflict
cvs diff -kk -r match-3_2_10-dst-1_4 > post.diff
diff pre.diff post.diff > diff.diff
make ccm
cvs commit 
cvs tag -c ccm3_6_brnchT_dst1_4
\end{verbatim}

\subsubsection[Merge CCM-Dust into MATCH-Dust]{Merge CCM-Dust changes into MATCH-Dust}
Merge changes to the CCM-Dust branch into the MATCH-Dust branch.  
A simple ``\texttt{cvs update}'' does not work because the changes are
on different branches.
This example merges the changes from CCM-Dust version
\verb'match-3_2_9-dst-1_0' to CCM-Dust version
\verb'ccm3_6_brnchT_dst1_2' into MATCH-Dust version
\verb'match-3_2_9-dst-1_0', which should then be tagged as
\verb'match-3_2_9-dst-1_2'. 
\begin{verbatim}
cd /fs/cgd/data0/zender/match_dst/dst
cvs diff -kk -r match-3_2_9-dst-1_0 > pre.diff
cvs update -kk -j match-3_2_9-dst-1_0 -j ccm3_6_brnchT_dst1_2
cvs update -p -j match-3_2_9-dst-1_0 params.h > params.h
cvs status | grep -i conflict
cvs diff -kk -r ccm3_6_brnchT_dst1_2 > post.diff
diff pre.diff post.diff > diff.diff
cvs commit
make match
cvs tag -c match-3_2_9-dst-1_2
\end{verbatim}

\subsection{MOZART-Dust Module}\label{sxn:cvs_mozart_dst}
The module \verb'mozart_dst' comprises the sub-modules \verb'mozart' and
\verb'dst'. 
Thus \verb'mozart_dst' is the complete MOZART with the dust modifications. 
The MOZART Dust implementation was created well after MATCH-Dust, and
uses most of the same directory structure.
Thus, the CVS commands for manipulating the MOZART-Dust model are
directly analogous to those described in Section~\ref{sxn:cvs_match_dst},
with the following exceptions.

\subsubsection[Field names]{Field names}
By default, MOZART archives both instantaneous values of constituents
mixing ratios. 
These are given the names assigned in the code, e.g., \verb'DSTQ01'.
MOZART can also be told to archive time-averaged values of these
constituent fields.
Time-averaged constituent fields are assigned names based on there
constituent index in the model, e.g., \verb'TRA01'.

\section{Subversion Overview}\label{sxn:svn}

\subsection{Modules}\label{sxn:svn_mdl}
To create a \trmidx{Subversion} repository, use
\begin{verbatim}
svnadmin create /home/zender/svn
export SVNROOT="svn+ssh://dust.ess.uci.edu/data/home/${USER}/svn/trunk"
\end{verbatim}

\subsection{Generic Modules}\label{sxn:svn_gnr}

\subsubsection[Import]{Import an existing directory}\label{sxn:cvs_mpr}
The \cmdidx{cvs2svn} command converts CVS repositories to
\trmidx{Subversion} repositories.
\begin{verbatim}
# Dry run do not actually commit anything.
cvs2svn -v --dry-run -s /home/zender/nco esmf.ess.uci.edu:/u/zender/cvs
cvs2svn -v --dry-run -s /home/zender/svn /home/zender/cvs
cvs2svn -v --dry-run -s /var/www/html/svn esmf.ess.uci.edu:/u/zender/cvs
# Commands actually used
cvs2svn -v --existing-svnrepos --force-tag=map-2_1_2 \
        -s /home/zender/svn /home/zender/cvs
# Valid Subversion keywords are:
Date Revision Author HeadURL Id
By default cvs2svn expands CVS keywords Name, Revision, Header 
prior to importing to Subversion.
cvs2svn correctly transitions CVS keywords Date, Id to Subversion.
# Keyword expansion appears to be automatic: How to turn off?
cvs2svn --keywords-off
# Change mode (chmod) of local file to executable
cd ~/aca;svn propset svn:executable ON htrn2nc.pl aca.pl
cd ~/aca;svn propset svn:executable ON htrn2nb.sh htrn2nc.sh aca.sh etags.sh 
# Find all files changed since 20081118
cvs diff -N -c -D 20081118 | grep "Index:" > ~/foo.new
# Backup ESMF CVS
cd /home/zender
tar cvzf ./cvs.tar.gz ./cvs
scp cvs.tar.gz dust.ess.uci.edu:
# Production conversion on PBS
/bin/rm -r -f ~/svn
cvs2svn -v -s /home/zender/svn /home/zender/cvs > ~/cvs2svn.txt
# keywords-off does not help me email from DLW 20081125
# cvs2svn --keywords-off -v -s /home/zender/svn /home/zender/cvs > ~/cvs2svn.txt
# http://www.cvsnt.org/manual/html/Substitution-modes.html
# To switch off keyword substitution for all files in a subtree: 
cvs update -k +o
cvs commit -m "Change substitution mode"
\end{verbatim}
Import new source directories:
\begin{verbatim}
svn help import
cd mdl
svn import -m "Imported sources" ~/mdl ${SVNROOT}/mdl
cd ..
mv mdl mdl.bck
svn co ${SVNROOT}/mdl
ls -R mdl
/bin/rm -r mdl.bck
\end{verbatim}

Convert my CVS repositories from ESMF to PBS 20090205:
\begin{verbatim}
rps_usr='zender/cvs'
#rps_srv_old='dust'
#rps_srv_new='esmf'
rps_srv_old='esmf'
rps_srv_new='dust'
export CVS_RSH='ssh'    # Needed for ssh access to NCAR CGD CVS
export CVSROOT=":ext:${USER}@${rps_srv_new}.ess.uci.edu:${HOME}/cvs"
echo ":ext:${rps_srv_new}.ess.uci.edu:/home/zender/cvs" > ~/cvs_zender.txt
locate CVS/Root | grep zender > ~/cvs_Root.txt
for fl in `cat cvs_Root.txt`; do
rps_crr=`cat ${fl}`
case "${rps_crr}" in 
    *${rps_srv}*${rps_usr}* ) 
    printf "${fl} points to a ${rps_usr} repository on ${rps_srv_old}, will redirect to ${rps_srv_new}\n"
    /bin/cp -f ~/cvs_zender.txt ${fl}
    ;; # endif match*
    * )
    printf "${fl} does not point to a ${rps_usr} repository on ${rps_srv_old}, will not change\n"
    ;; # endif default
esac # endcase ${rps_crr}
done
# Server instructions here were not helpful:
# http://sanatio.blogspot.com/2005/12/cvs-server-on-ubuntu.html
# Instead, had to manually create lock directories on server with
sudo mkdir /var/locks
sudo mkdir /var/locks/cvs
sudo chmod 777 /var/locks
sudo chmod 777 /var/locks/cvs
# 20150216: Sourceforge reported lock errors
# https://sourceforge.net/p/forge/site-support/9688
# Took a week to fix, no thanks to SF, finally figured this out:
rsync -av nco.cvs.sourceforge.net::cvsroot/nco/* . # Backup CVS repo
ssh -t zender,nco@shell.sourceforge.net create # Create login shell
sf-help --web # Print location of hidden directories
cd /home/project-web/nco # Apparently this is my repo
adminrepo --unlock cvs # This gets rid of the dreaded locks
adminrepo --save # This did not actually seem to help
\end{verbatim}

Check out modules:
\begin{verbatim}
svn checkout file:///home/zender/svn/trunk/f ~/foo
svn checkout file:///home/zender/svn/trunk/mdl ~/mdl
\end{verbatim}
Check out date-stamped or version-stamped module into new directory foo
\begin{verbatim}
svn checkout --revision {2009-11-11} $SVNROOT/crr ~/foo
svn checkout --revision {2013-12-02} file:///home/zender/svn/trunk/nco ~/foo
\end{verbatim}
Export date-stamped or version-stamped file into new file foo
\begin{verbatim}
svn export --revision {2009-11-11} $SVNROOT/crr/crr.txt ~/foo
\end{verbatim}
Examine status or working directory
\begin{verbatim}
svn status
\end{verbatim}
Commit changes in working directory
\begin{verbatim}
svn 
\end{verbatim}
List modules contained in repository
\begin{verbatim}
svn ls svn+ssh://zender@dust.ess.uci.edu/home/zender/svn/trunk
svn checkout file:///home/zender/svn/trunk/dot ~/dot
\end{verbatim}
Tagging
\begin{verbatim}
svn copy svn+ssh://zender@dust.ess.uci.edu/home/zender/svn/trunk/f \
svn+ssh://zender@dust.ess.uci.edu/home/zender/svn/tags/f/2.0.3 \
-m "Tagging the 2.0.3 release of the project" 
\end{verbatim}
Equivalence between \texttt{svn} and \texttt{svn+ssh} methods:
\begin{verbatim}
# Using Subversion only full path not necessary, but svnserve must run on server:
svn checkout svn://dust.ess.uci.edu/svn/trunk/f
# Using Subversion + SSH, full path is necessary, but svnserve not used:
svn checkout svn+ssh://dust.ess.uci.edu/home/zender/svn/trunk/f
\end{verbatim}

Converting CVS keywords to Subversion properties:
First, find the files with keywords, e.g., 
\begin{verbatim}
egrep -rl '\$(Author|Date|Header|Name|Revision|Source|State|Id)' *
\end{verbatim}
Determine the appropriate properties to set by searching
Subversion's hidden \flidx{.svn} directory, e.g.,
\begin{verbatim}
# Convert $Id$ to $Id$ keyword
perl -pi -e 's/\$Header\$/\$Id\$/g;' *
perl -pi -e 's/\$Id$/\$Id\$/g;' *
egrep -rl '\$Id: ' * | grep -v /.svn/ | xargs svn propset svn:keywords Id
svn commit
\end{verbatim}

Resolving conflicts:
After editing to resolve a conflict, one may still receive errors
on commits attempts like
\begin{verbatim}
zender@greenplanet:~/ess_gcm$ svn commit -m ""
svn: Commit failed (details follow):
svn: Aborting commit: '/state/partition1/home/zender/ess_gcm/cam_gcm.sh' remains in conflict
\end{verbatim}
The solution is to explicitly tell Subversion that the conflict has
been resolved
\begin{verbatim}
zender@greenplanet:~/ess_gcm$ svn resolved cam_gcm.sh
Resolved conflicted state of 'cam_gcm.sh'
zender@greenplanet:~/ess_gcm$ svn commit -m "Resolved funky conflict"
Sending        cam_gcm.sh
Transmitting file data .
Committed revision 14055.
\end{verbatim}

Renaming files:
\begin{verbatim}
svn mv foo bar
\end{verbatim}

Create modules:
\begin{verbatim}
cd
svn import mdl ${SVNROOT}/mdl -m "Imported Sources"
mv mdl mdl.bck
svn co ${SVNROOT}/mdl
ls -R mdl
/bin/rm -r mdl.bck
\end{verbatim}

\section{Git Overview}\label{sxn:git}

ACME favors Git, and ACME-flavored development procedures for Git are
described
\href{https://acme-climate.atlassian.net/wiki/display/Docs/Development+Quick+Guide}{here}. 

\subsection{Cloning}\label{sxn:git_cln}

It is much more convenient to clone repositories with SSH 
than with HTTPS because SSH replaces the cumbersome
key-management/wallet apps that vary between OS's. 
To clone a remote repository using SSH protocol, first install
an SSH key from the development machine to the remote server (e.g.,
GitHub), by following these instructions
\href{https://help.github.com/articles/generating-ssh-keys}.
\begin{verbatim}
git clone git@github.com:ACME-Climate/DiagnosticsWorkflow
\end{verbatim}

To clone a remote repository using HTTPS protocol, use
\begin{verbatim}
git clone https://github.com/ACME-Climate/DiagnosticsWorkflow
\end{verbatim}
Note that OLCF does not support HTTPS protocol.
On OLCF machines, one must use clone Git repositories with the 
SSH protocol instead.
To change the access protocol for a checked-out repository from
HTTPS to SSH use
\begin{verbatim}
git remote set-url origin git@github.com:username/repository.git
git remote set-url origin git@github.com:czender/prv.git
\end{verbatim}

\subsection{Modules}\label{sxn:git_mdl}

\subsection{Creating}\label{sxn:git_crt}

\begin{verbatim}
cd ~/diwg
# Transform current directory into Git repository. Create .git subdirectory.
git init
# Push new local Git repository to GitHub
git remote add origin https://github.com/czender/diwg.git
git push origin --all
git push origin --tags
git push --set-upstream origin master
\end{verbatim}

\subsection{Merging}\label{sxn:git_mrg}

A Git \cmdprn{pull} is basically a merge.
If one pulls the remote master onto a local repository which contains
changes, Git complains that the pull will overwrite the changes, then
quits. 
To force the overwrite of any local changes, one uses \cmdprn{fetch}
to update the local metadata, then \cmdprn{reset} to point the local
repository to the remote head.
\begin{verbatim}
# Fetch from default remote, origin
git fetch
# Reset current branch (master) to origin's master
git reset --hard origin/master
\end{verbatim}
This procedure is often necessary with NCO on older machines that
automatically generate different .cpp and .hpp files from newer
machines.  

Merging changes in branches to the master
\begin{verbatim}
# Merging NCO branches
# This method pushes everthing from devel to master
# Works as intended when devel and master have same files
cd ~/nco
git checkout master
git pull origin master
git merge devel
git push origin master

# Merging ACME branches
# Above method does not work as intended because devel and master have (some) different files
# Assume we are working on devel and make changes we want to incorporate into master
# First, commit the changes to devel (so they are not lost when checking out master)
git commit -am "Changes to devel"
# The follow this recipe
git fetch
git checkout master
git pull
git checkout devel climo_nco.sh climo_ann.sh
git status
git diff --cached
git commit -am "message"
git push origin master
git checkout devel
\end{verbatim}

Merging changes in master out to the branches
\begin{verbatim}
# Merging occurs locally (no pull request necessary)
# Push to remote when satisfied with results
# Graphical explanation at:
# http://stackoverflow.com/questions/3876977/update-git-branches-from-master
git checkout rgr
git merge master # Retains true history, best solution
# Any CONFLICTS are identified by <<<< HEAD ==== master >>>> syntax and must be hand-edited
# Branch will be called HEAD
# Unclear what to do with ChangeLog (see graphic) above:
# Edit to keep monotonic in time or let-alone preserve true sequence of merge?
git commit -a -m ``Merged changes in master into rgr branch''
# ...or...
git checkout rgr
git rebase master # Retains false history, inferior solution
git commit -a -m ``Rebased rgr branch to master''
\end{verbatim}

\subsection{Renaming}\label{sxn:git_rnm}

Rename local and remote branches
\begin{verbatim}
git branch -m old_branch new_branch       # Rename branch locally    
git push origin :old_branch               # Delete old branch
git push --set-upstream origin new_branch # Push new branch, set local branch to track new remote
\end{verbatim}

\subsection{Forensics}\label{sxn:git_rnm}

Obtain dated or tagged versions of code
\begin{verbatim}
git 
\end{verbatim}

% Bibliography
%\renewcommand\refname{\normalsize Publications}
%\nocite{ZeK971}
%\bibliographystyle{agu}
%\bibliography{/home/zender/tex/bib}

\end{document}

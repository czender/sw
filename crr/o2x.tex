% $Id$

% Purpose: Paper on global abundance and forcing of Oxygen collision complexes 
% URL: http://dust.ess.uci.edu/ppr/o2x.ps.gz
% URL: http://dust.ess.uci.edu/ppr/o2x.pdf

% NB: Solomon highly recommends reading Burkholder and McKellar

% Usage:
% dvips -Ppdf -G0 -o ${DATA}/ps/o2x.ps ~/crr/o2x.dvi;ps2pdf ${DATA}/ps/o2x.ps ${DATA}/ps/o2x.pdf
% cd ~/crr;make -W o2x.tex o2x.dvi o2x.ps;cd -
% cd ~/crr;texcln o2x;make o2x.pdf;bibtex o2x;makeindex o2x;make o2x.pdf;bibtex o2x;makeindex o2x;make o2x.pdf;cd -
% scp -p ${DATA}/ps/o2x.pdf dust.ess.uci.edu:/var/www/html/ppr/ppr_Zen99.pdf

% Create *.pdf versions of every graphics file for use with pdftex: 
% cd ${DATA}/ps; foreach fl ( `ls *.eps` ); epstopdf $fl; end
% PDF and Postscript versions have different margins, fonts, and font sizes:
% PDF seems to always enforce horizontal truncation of graphics
% PDF version occupies significantly less (~0.8 cm) vertical space
% PDF version occupies significantly less (~0.8 cm) horizontal space
% PDF version fonts 
% Despite these differences, line breaks, page breaks, and thus page numbers are identical
% Overall, I prefer the Postscript version

\documentclass[agupp,twoside]{aguplus} % Concise preprint style
%\documentclass[agupp,twoside,draft]{aguplus} % No figures
%\documentclass[12pt,agums]{aguplus} % Double-spaced manuscript for submittal
%\documentclass[jgrga]{aguplus} % Galleys to save money

% AGU++ Options
%\printfigures % Adds figures at end
\doublecaption{35pc} % Captions printed twice
\sectionnumbers % Turns on section numbers
\extraabstract % Adds supplemental abstract
%\afour % European A4 paper size
%\figmarkoff % Supress marginal markings

% AGUTeX options and entries
%\tighten % Turns off double spacing, has no effect with agupp
%\singlespace % Restores single spacing
%\doublespace % Restores double spacing

% Standard packages
\usepackage{ifpdf} % Define \ifpdf
\ifpdf % PDFLaTeX
\usepackage[pdftex]{graphicx} % Defines \includegraphics*
\pdfcompresslevel=9
\usepackage{thumbpdf} % Generate thumbnails
\usepackage{epstopdf} % Convert .eps, if found, to .pdf when required
\else % !PDFLaTeX
\usepackage{graphicx} % Defines \includegraphics*
\fi % !PDFLaTeX

\usepackage{times} % Times/Roman font, aguplus STRONGLY recommends this for the camera-ready version!!

% hyperref is last package since it redefines other packages' commands
% hyperref options, assumed true unless =false is specified:
% backref       List citing sections after bibliography entries
% breaklinks    Wrap links onto newlines
% colorlinks    Use colored text for links, not boxes
% hyperindex    Link index to text
% plainpages=false Suppress warnings caused by duplicate page numbers
% pdftex        Conform to pdftex conventions
\ifpdf % PDFLaTeX
\usepackage[backref,breaklinks,colorlinks,hyperindex,plainpages=false,pdftex]{hyperref} % Hyper-references
\pdfcompresslevel=9
\else % !PDFLaTeX
\usepackage[backref,breaklinks,colorlinks,hyperindex,plainpages=false]{hyperref} % Hyper-references
\fi % !PDFLaTeX

% Personal packages
\usepackage{csz} % Library of personal definitions
\usepackage{dmn} % Dimensional units
\usepackage{chm} % Commands generic to chemistry
\usepackage{hyp} % Hyphenation exception list
\input{jgr_abb} % AGU-sanctioned journal title abbreviations

% Commands specific to this file
\usepackage{o2x} % Commands specific to O2*X work
\newlength{\tblwdt} % Horizontal size of current table
\newlength{\fltwdtsngclm}\setlength{\fltwdtsngclm}{20.0pc} % Float width to fill single column
\newlength{\fltwdtdblclm}\setlength{\fltwdtdblclm}{41.0pc} % Float width to fill whole page

% \lefthead and \righthead are automatically uppercased by jgrga documentstyle 
% Short title: ZENDER: CLIMATOLOGY OF \OdX\ ABUNDANCE AND SOLAR ABSORPTION
% \OdX does not look good when italicized, so expand instead
\lefthead{ZENDER}
\righthead{CLIMATOLOGY OF O$_2 {\cdot} $X ABUNDANCE AND SOLAR ABSORPTION}
\received{January~28, 1999}
\revised{July~14, 1999}
\accepted{July~22, 1999}
\journalid{JGRD}{October~27, 1999}
\articleid{24471}{24484}
\paperid{1999JD900797}
% The $ in the following line screws up the hilit19 highlighting
%\ccc{0148-0227/99/1999JD900797\ 09.00}
\ccc{0148-0227/99/1999JD900797\$09.00}
\cpright{AGU}{1999}
% \cpright{Crown}{1994}
% (No \ccc{} for Crown copyrights.)

\authoraddr{C. S. Zender,
Department of Earth System Science, University of California, Irvine,
CA~~92697-3100. (zender@uci.edu)}
%National Center for Atmospheric Research, P.O. Box 3000, Boulder,
%CO~~80307. (zender@ncar.ucar.edu)}    

\slugcomment{JOURNAL OF GEOPHYSICAL RESEARCH, VOL.~104, NO.~D20, PAGES
24,471--24,484, OCTOBER~27,~1999}
%\slugcomment{Submitted to \textit{J. Geophys. Res. Atm.},
%January~28, 1999. Revised July~14, 1999. Accepted July~22, 1999.}
%\slugcomment{\today}

% NB: Placement of figures is very sensitive to \textfraction
\renewcommand\textfraction{0.3} % Minimum fraction of page that is text
\setcounter{totalnumber}{10} % Maximum number of floats per page
\setcounter{topnumber}{10} % Maximum number of floats at top of page
\setcounter{dbltopnumber}{10} % Maximum number of floats at top of two-column page
\setcounter{bottomnumber}{10} % Maximum number of floats at bottom of page
\renewcommand\topfraction{1.0} % Maximum fraction of top of page occupied by floats
\renewcommand\dbltopfraction{1.0} % Maximum fraction of top of two-column page occupied by floats
\renewcommand\bottomfraction{1.0} % Maximum fraction of bottom of page occupied by floats
\renewcommand\floatpagefraction{1.0} % Fraction of float page filled with floats
\renewcommand\dblfloatpagefraction{1.0} % Fraction of double column float page filled with floats

\begin{document}

% NB: title must be lowercased by hand. The documentstyle does not enforce it.
\def\paperchaptertitle{Global climatology of abundance and solar
absorption of oxygen collision complexes} 
\title{\paperchaptertitle}

\author{Charles S. Zender\altaffilmark{1}}
\affil{National Center for Atmospheric Research, Boulder, Colorado} 
\altaffiltext{1}{Now at Department of Earth System Science, University
of California, Irvine.\\
\smallskip
\noindent Copyright 1999 by the American Geophysical Union.\\
Paper number 1999JD900797.\\
0148-0227/99/1999JD900797\$09.00}

\begin{abstract}
To improve our understanding of the absorption of solar radiation in 
the atmosphere we have characterized the spectral absorption, the
spatial and temporal abundance, and the radiative forcing of the
oxygen collision pairs \OdOd\ and \OdNd\ ($\OdX \equiv \OdOd +
\OdNd$). 
The regional, vertical, seasonal, and annual patterns of \OdX\
abundance and radiative forcing are obtained from a general
circulation model. 
We estimate the mean absorption by \OdX, heretofore neglected in large
scale atmospheric models, is 0.75--1.2\,\wxmS, or 1--2\% of total
atmospheric solar absorption.   
%For comparison, solar absorption by \Od\ and \COd\ is roughly 2.0 and
%0.5\,\wxmS, respectively.   
\OdX\ absorption reduces surface insolation by 0.48--0.78\,\wxmS\ and  
increases the net radiative flux at the tropopause by
0.32--0.52\,\wxmS.   
These ranges bracket the uncertainties due to spectral absorption
cross sections, \OdNd\ efficiency, \OdX\ abundance, and cloud
distribution. 
Globally averaged, \OdX\ enhances absorption equally in clear and
in cloudy skies.

We create a global climatology of well-mixed collision complex
abundances by scaling \OdX\ abundance to other complexes such as
\NdNd\ and \OdAr.  
Collision complex abundance depends quadratically on the
concentrations of the constituents.  
This dependence causes a 20\% increase in \OdX\ abundance in the 
Arctic relative to the Tropics for the same sea level pressure.  
The variations in zonal mean \OdX\ abundance due to surface elevation,
the annual mean meridional temperature gradient, and to seasonal
temperature variations are 40, 15, and 10\%, respectively.    
\OdX\ heating obeys the weakly absorbing, linear limit so it peaks at
the surface in clear skies, but clouds shift this peak up by
200--300\,\mb\ on seasonal timescales. 
Surface albedo and clouds strongly modulate the solar forcing
efficiency of \OdX\ (forcing per unit abundance) by altering the mean
photon path length. 
These factors produce annual mean forcing maxima in the subtropics
over bright deserts and regions of marine stratus.    
Inclusion of \OdX\ in models is likely to reduce cold biases in the
summertime polar atmosphere, where \OdX\ contributes 2--4\% of total
solar heating. 
%The seasonal forcing peaks in summertime polar regions.
%Clouds, surface reflectance, and reduced water vapor cause the forcing
%efficiency in the Antarctic to exceed the Arctic by 30--75\%.

% ncks -H -u -F -C -v NPCO2O2 ${DATA}/dmr04/dmr04_8589_1202_x.nc
% ncks -H -u -F -C -v NPCO2O2 ${DATA}/dmr04/dmr04_8589_0608_x.nc

\end{abstract}

\section{Introduction}\label{sxn:ntr}

% Original studies: Get McKellar, Ketellar (this has the O2-N2 shape),
% and Cho

\nocite{EPP95,GOB90,MCB98,PEP97,SPS98}

% Get rid of horizontal rule separating this footnote from text
%\renewcommand{\footnoterule}{} 
\setlength{\footnotesep}{12pt} 
%\footnote{\noindent Copyright 1999 by the American Geophysical Union.\\
%Paper number 1999JD900797.\\
%0148-0227/99/1999JD900797\$09.00}%

Our ability to balance the solar radiation budget of the atmosphere is
uncertain \cite[e.g.,][and references
therein]{StT902,KAC97,RaV97,Col98}.  
Until we can perform this fundamental task, our skill at
predicting the all sky radiation budget, and hence at predicting
climate and climate change, will continue to be highly uncertain
\cite[e.g.,][]{CPB89}. 
The atmospheric collision complexes of oxygen \OdOd\ and \OdNd\
(henceforth $\OdX \equiv \OdOd + \OdNd$) have recently been
identified as contributors to discrepancies between modeled and
observed surface insolation \cite[]{PEP97,SPS98}.  
\OdX\ presents a timely and interesting case for a radiative budget
study because its solar absorption cross section is relatively well
known, but its abundance and absorption have not been characterized at
the global scale necessary to fully describe its impact on climate. 

The absorption features of oxygen collision complexes have been
studied in the laboratory and in the atmosphere for many decades
\cite[][and references therein]{PeP80,SPS98}.   
Collisions relax the selection rules for dipole-forbidden transitions
in one or both \Od\ molecules, allowing for absorption during the
collision. 
The detection of discrepancies between modeled and observed solar
radiation in cloudy skies \cite[]{CZM95,RaV97} provoked interest in
whether \OdX\ contributes to this discrepancy \cite[]{EPP95,MGL97}
and in what its global radiative forcing characteristics are.

\cite{PEP97} first estimated the global annual mean solar
absorption by \OdOd\ in clear sky and in all sky (i.e., including
clouds) conditions as 0.53 and 0.57\,\wxmS, respectively.   
Two newer estimates include the effects of the powerful 1.27\,\um\ band 
and of nitrogen induced absorption (\OdNd) in this band.
Estimates from these studies suggest that globally and annually
averaged, \OdX\ absorbs 0.9--1.3\,\wxmS\ \cite[]{SPS98} or 0.84\,\wxmS\
\cite[]{MCB98}.
Thus \OdX\ is thought to contribute 1--2\% of the total solar
atmospheric absorption of 67\,\wxmS\ (out of 342\,\wxmS\ incoming)
\cite[]{KiT97}.  
These estimates of \OdX\ absorption are based on single column
line-by-line, multiple-scattering radiative transfer models which have
been extensively validated against atmospheric observations by high
spectral resolution instruments.   
The disparity in the latter two estimates is mostly due to
uncertainties in absorption cross sections for \OdX\ bands in the near
infrared and, to a lesser extent, from extrapolating single-column
results to global annual means.

We use the term ``radiative forcing'' to denote various measures of
the influence of atmospheric heating by \OdX\ on the climate system.
Line-by-line estimates of \OdX\ absorption are too computationally 
expensive to convey the rich regional and seasonal behavior of
\OdX\ radiative forcing caused by variations in temperature, pressure,  
elevation, clouds, surface albedo, and mixing ratios of other
radiatively active atmospheric constituents. 
A general circulation model (GCM) is best suited to provide
self-consistent, time-varying realistic distributions of \OdX\
forcing. 
%The motivations for modeling the global distribution and forcing of
%\OdX\ are at least twofold:  
%First, to refine estimates of \OdX\ forcing.
%Prior estimates were extrapolated from single-column studies.
%We diagnose forcing from within a GCM where the most important 
%boundary conditions vary realistically in space and time.
Employing a GCM allows us to extend prior studies and to provide
additional forcing metrics, including clear versus cloudy sky forcing,
surface forcing, and net forcing at the tropopause.
These metrics allow us to compare the radiative forcing by \OdX\ to
other absorbers, including greenhouse gases. 
These comparisons, in turn, help to prioritize implementation of \OdX\
physics in large-scale atmospheric models. 
It is worthwhile noting that in contrast to anthropogenically
influenced greenhouse gas concentrations, \OdX\ abundance is not
changing.   

%The other goal of this study is to establish the spatial and temporal
%patterns of abundance and forcing due to \OdX.
To our knowledge no prior study has constructed a systematic global
climatology for any collision complex (or dimer). 
Moreover, the regional, seasonal, and vertical patterns of \OdX\
abundance generalize to all complexes formed from well-mixed gases. 
Thus these patterns, which depend mainly on the meridional
temperature gradient and orographic features, apply to all
well-mixed collision complexes, for example, \NdNd, \OdAr, and \OdCOd.   
The patterns of \OdX\ radiative forcing, which depend additionally on 
insolation, surface albedo, and cloud cover, only apply to other
complexes with absorption features similar to those of \OdX.    
Thus, by characterizing \OdX\ abundance and forcing, we automatically 
obtain insight into all well-mixed collision complexes.

Establishing the spatial and temporal patterns of \OdX\ abundance and
forcing will also help future studies ascertain whether unexplained
atmospheric solar absorption is consistent with absorption by
collision complexes (or dimers) of well-mixed molecules. 
The existence of complexes besides \OdX\ which cause significant
solar radiative forcing is highly speculative.
All significant structured atmospheric absorption measured in the
atmosphere for large portions of the solar spectrum has been
convincingly attributed to known absorbers
\cite[]{PEP97,SPS98,MCB98,VRC981}.   
Moreover, a number of recent field studies explains all observed
broadband clear sky solar absorption to within the uncertainties in
the observations and models \cite[]{CRV97,ZBP97,RaV97,JiC98}.  
However, \cite{KAC97} and \cite{HNS98} find inexplicable discrepancies
between models and observations of the clear sky diffuse field, and  
many independent studies show enhanced cloudy sky solar absorption
relative to model predictions on both local and global scales
\cite[]{CZM95,ZBP97,Col98}.  
Since \OdX\ has been proposed as a candidate for explaining some of
these discrepancies \cite[]{EPP95,MGL97}, this study attempts to bound
the role of \OdX\ in explaining enhanced absorption in clear and
cloudy skies. 

The remainder of this study is organized as follows:
Section~\ref{sxn:mth} presents the nomenclature of collision complexes
and summarizes the laboratory measurements of \OdX\ cross sections.
Section~\ref{sxn:spc} illustrates the modeled spectral characteristics 
of \OdX\ absorption during a field campaign and shows its net vertical
heating signature in an idealized atmosphere.
Section~\ref{sxn:res} presents the regional, seasonal, and vertical
climatologies of \OdX\ abundance and forcing and generalizes \OdX\
abundance to other well-mixed collision complexes.
Section~\ref{sxn:dsc} summarizes the results and discusses their
implication for future studies of the atmosphere. 

\section{Characterization of Abundance and Absorption Cross
Section}\label{sxn:mth} 

Quantifying the global radiative forcing of a gaseous molecular
species $i$ requires characterization of both its concentration 
$\cnc_i(x,y,z,t)$\,\mlcxcmC\ and its absorption cross section
$\xsx_i(\wvl)$\,\cmSxmlc.  
The convolution of these two quantities with the instantaneous
radiation field yields the instantaneous forcing due to the gas.
However, careful laboratory measurements provide convincing evidence
that for atmospheric temperatures and solar wavelengths, \OdX\
absorption is due to transient collision complexes rather than bound
dimers (or van~der~Waals complexes) \cite[]{GOB90,OTN91}.
Since \OdX\ is not a bound molecule, we begin by defining properties
analogous to $\cnc_i$ and $\xsx_i$, but that apply to collision
complexes.    

\subsection{Abundance}\label{sxn:bun} 
The ``abundance'' $\cncXY$ of a collision complex of molecules $\X$
and $\Y$ is analogous to a traditional number concentration of a
molecular species.
We define $\cncOdX \equiv \cncOd \cncX$ where $\cncOd$ and $\cncX$
are the number concentrations of \Od\ and of its collision
partner (\Nd\ or \Od, unless otherwise specified) in \mlcxcmC,
respectively.  
Thus the dimensions of $\cncOdX$ are \mlcSxcmS. 
By definition, $\cncOdX$ is linearly proportional to the frequency of  
collisions of \Od\ with \X.
To graphically depict the total column abundance of \OdX, we will
present the vertical column paths $\npcOdX$ defined by 
\newline\parbox{18.0pc}{ % KoD95 p. 138
\begin{eqnarray}
\npcOdOd & \equiv & \int_{0}^{\mathrm{TOA}} \cncOdOd \, d\pth = 
\int_{0}^{\mathrm{TOA}} \cncOd^2 \, d\pth \nonumber \\ 
\npcOdNd & \equiv & \int_{0}^{\mathrm{TOA}} \cncOd \cncNd \, d\pth = 
\gamma \npcOdOd \nonumber
%\label{eqn:N}
\end{eqnarray}
}\hfill % end parbox KoD95 p. 138
\parbox{2.0pc}{\begin{eqnarray}\label{eqn:N}\end{eqnarray}}\newline
where the integration over $\pth$ (height) is from the surface ($\pth =
0$) to the top of atmosphere (TOA $\approx 1.5$\,\mb\ due to GCM
vertical discretization).   
Likewise, the column path of a molecule \X, in \mlcxcmS, will be
denoted by $\npcX$.  
We assume in (\ref{eqn:N}) that the volume mixing ratios of \Nd\ and
\Od\ are constant in the Earth's atmosphere and that their ratio is
the constant $\gamma \equiv \cncNd/\cncOd = 3.72787$
\cite[][p. 9]{GoY89}.  
Since the vertical profile of $\cncOdX$ varies with the square of
atmospheric density, variations in $\gamma$ above the stratosphere
have negligible impact on the column abundance, and thus forcing, of 
\OdX.  

\subsection{Absorption Cross Sections}\label{sxn:xsx} 
The absorption bands of the \OdOd\ and \OdNd\ collision complexes have
been studied for decades.
\cite{SPS98} give an excellent review of laboratory research on
oxygen collision complex absorption.
Here we summarize only the research necessary to understand the  
physical parameterizations and uncertainties present in this
study. 

Laboratory measurements of \OdX\ absorption infer a binary absorption
coefficient $\xsxb(\wvl)$ from the quadratic dependence of measured
absorption on \Od\ concentration, $\cncOd$. 
\cite{NeB98} provide a detailed description of this type of
measurement and its uncertainties.
The dimensions of $\xsxb(\wvl)$ are \cmFxmlcS.
%Thus $\xsxbOdX(\wvl)$ is the absorption optical depth due to \OdX\
%($\odxcOdX$) at wavelength $\wvl$ per unit number concentration 
%(\mlcxcmC) of \Od\ per unit number path (\mlcxcmS) of \X.   
Thus, for a given concentration $\cncOd$\,\mlcxcmC\ and path length
$\pthdlt$ centimeters of collision partner (\Od\ or \Nd), the optical
depths $\odxcOdX$ due to \OdX\ at wavelength $\wvl$ are
\begin{eqnarray}
\odxcOdOd(\wvl) & = & \xsxbOdOd(\wvl) \, \cncOd^2 \pthdlt \\
\odxcOdNd(\wvl) & = & \xsxbOdNd(\wvl) \, \cncOd \cncNd \pthdlt \nonumber \\
& = & \gamma \, \epsNd \xsxbOdOd(\wvl) \, \cncOd^2 \pthdlt \nonumber \\
\odxcOdNd(\wvl) & = & \gamma \, \epsNd \odxcOdOd(\wvl)
\label{eqn:tau}
\end{eqnarray}
where (\ref{eqn:tau}) applies to the 1.27\,\um\ band only
(see Plate~\ref{plt:abs_xsx}) and $\epsNd$, discussed below, 
is the efficiency of \Nd\ (relative to \Od) at inducing \Od\
absorption through collisional processes (i.e., $\xsxbOdNd = \epsNd
\xsxbOdOd$). 

Plate~\ref{plt:abs_xsx} shows the solar absorption continua for
\OdX\ measured by \cite{GOB90} at $\sim 0.15$~nm resolution for $0.335
< \wvl < 1.137$\,\um. 
\begin{plate*}
\begin{center}
% NB: figure produced with ~/arese/arese.pro:abs_xsx_gph()
\includegraphics[width=\fltwdtdblclm]{/data/zender/fgr/o2x/abs_xsx}\vfill
\end{center}
\caption{
Solar absorption continua of \OdX, \Ot, \Od, and \NOd.
Continua of \OdOd\ (red) and \OdNd\ (blue) are plotted as
binary absorption coefficients $\xsxbOdX$ (\cmFxmlcS) on the right-hand
axis. 
Continua of \Ot\ (black), \NOd\ (green), and \Od\ (yellow) are
plotted as traditional cross sections $\xsx$ (\cmSxmlc) on the
left-hand axis.  
\label{plt:abs_xsx}}   
\end{plate*}
The 1.27\,\um\ band has not been adequately examined with modern
laboratory techniques, so we adopt the cross sections for this band
recommended by \cite{SPS98}. 
On the basis of a number of earlier measurements, Solomon et~al.\
constructed the 1.27\,\um\ band by stretching the measured 1.06\,\um\
band to a full width at half maximum (FWHM) of 180\,\xcm\ and scaling
its integrated band intensity by 1.6.
The cross sections of \Ot\ and \Od\ \cite[]{WMO85}, and of \NOd\
\cite[]{DCM88}, are shown for comparison.  
The red curves show that significant \OdOd\ solar absorption occurs in
two spectral regions: $0.3 < \wvl < 0.7$\,\um\ and $1.0 < \wvl <
1.4$\,\um. 
\OdOd\ visible absorption is due to simultaneous collision induced
transitions in both \Od\ molecules, while near-infrared absorption 
involves only one \Od\ transition.

\cite{NeB98} recently measured $\xsxbOdOd$ for $0.455 < \wvl <
0.830$\,\um\ and compared their measurements to those of \cite{GOB90}
(Plate~\ref{plt:abs_xsx}).  
The maximum $\xsxbOdOd$ and the integrated band intensities of
most bands differed by $< 20$\% between the two sets of
measurements. 
Greenblatt et~al.\ found the integrated band intensity of the bands
centered at 360, 380, and 477~nm varied by $< 15$\% from 
$T = 296$\,\K\ to 196\,\K.
Newnham and Ballard found the integrated band intensity of four bands
from 532--630~nm varied by $< 10$\% from $T = 283$\,\K\ to 223\,\K\
but that the intensity of the 477~nm band decreased by 30\% over this 
temperature range. 
Thus the two studies disagree on the temperature dependence of the
the only band examined in both studies, the 477~nm band. 
Why the disagreement at 477~nm exceeds the stated error bars of the
two studies is unexplained. 
Balloon-borne observations of the 447, 532, and 577~nm bands by
\cite{OFH98} recorded a 20\% decrease in $\xsxbOdOd$ as the
temperature increased from $T = 200$\,\K\ to 300\,\K. 
Note that this decrease is in contrast to the increase of $\xsxbOdOd$
with $T$ measured by Newnham and Ballard in the 477~nm band. 
Osterkamp et~al.\ conclude that \OdOd\ may be a metastable complex
with some dimer-like properties.
For simplicity, our study assumes $\xsxbOdX$ is constant with
temperature.
However, we should remain open to the possibility that $\xsxbOdX$
varies significantly with temperature for some bands.

The parameter $\epsNd$ defines the efficiency of \Nd\ relative to
\Od\ as a partner for inducing absorption in the \Od\ $1.27$\,\um\
band. 
The upper and lower blue curves centered at $\wvl = 1.27$\,\um\ in
Plate~\ref{plt:abs_xsx} correspond to $\xsxbOdNd$ assuming $\epsNd =
0.1$ and~0.3, respectively.
All prior studies show $0.1 \le \epsNd \le 0.3$, and the preponderance 
suggest $\epsNd = 0.3$ (see discussion by \cite{SPS98}).  

% \cite{GOB90} found no evidence for temperature dependence in the
% \OdOd\ absorption at temperatures above $-90$~C,
% This shows that \OdOd\ van der Walls complexes are not present at
% temperatures characteristic of Earth's atmosphere and may be ignored 
% in this study.

\cite{SPS98} used differential optical absorption spectroscopy at
large zenith angles to show their high spectral resolution ($\approx
0.6$~nm) in situ measurements agree with the laboratory
cross sections of \cite{GOB90} (Plate~\ref{plt:abs_xsx}) to within
0.3\% for the \OdOd\ band at 0.63\,\um.  
\cite{MCB98} recently inferred $\xsxbOdX$ for $1.0 < \wvl < 1.57$\,\um\ 
from field measurements with a high-resolution ($0.6~$\xcm)
interferometer. 
They inferred comparable peak intensities in the 1.06 and 1.27\,\um\
\OdX\ bands, but narrower FWHMs, than previous investigators
\cite[]{SPS98}.
These narrower FWHMs would reduce absorption in the near-infrared
\OdX\ bands by about 30\% relative to Greenblatt et~al.\ and Solomon
et~al. 
%This discrepancy, and possible explanations, are discussed in both
%\cite{MCB98} and \cite{SPS98}.

The uncertainties in $\xsxbOdX(\wvl)$ contribute roughly half of the 
uncertainty in our estimate of global \OdX\ forcing.
The uncertainty in $\xsxbOdX(\wvl)$ is $< 10\%$ for $0.335 <
\wvl < 1.137$ \um\ \cite[]{GOB90}. 
Uncertainty in $1.27$\,\um\ band absorption is $\sim 30\%$. 
This was estimated by extrapolating to global scale the difference
between single-column simulations employing the continua proposed by
\cite{SPS98} and by \cite{MCB98}.
For consistency, our study adopts the same cross sections as
Solomon et~al., but we note two choices that keep our results for
total \OdX\ absorption closer to those of Mlawer et~al.
First, we use $\epsNd = 0.2$, a somewhat conservative value.
Second, we neglect the \OdOd\ absorption band near 1.58\,\um.
This band absorbs $\sim 9\%$ as much as the 1.27\,\um\ band
\cite[]{MCB98}. 
Appending the 1.58\,\um\ continuum to the Solomon et~al.\
cross sections would increase our computed total \OdX\ solar forcing
by $\sim 4\%$. 
Other absorption features of \OdX\ are negligible in terms of the total
absorbed solar irradiance.
These features, and those of many other well-mixed collision
complexes, are summarized in Table~\ref{tbl:cnc}. 

\section{High Resolution Single Column Simulations}\label{sxn:spc}

%\section{Overlap of \OdX\ with \HdO, \Ot, \Od, \NOd, and
%\COd}\label{sxn:spc}  
\OdX\ complexes are a relatively recent addition to the longstanding
group of solar absorbers considered in large-scale atmospheric models:
\HdO, \Ot, \Od, \NOd, and \COd. 
As such, the spectrally and vertically dependent features of \OdX\
absorption have not yet been thoroughly characterized in the
literature. 
We now analyze these features in high-resolution, single-column
simulations in order to better understand their implications on
global-annual scales.

\subsection{Spectral Characteristics and Overlap}\label{sxn:ovr}
Two previous studies characterized aspects of the spectral overlaps
between \OdX\ and more familiar absorbers. 
\citeauthor{PEP97} [\citeyear{PEP97}, Figure~2] show \OdX\ spectral
features at surface and TOA for $335 < \wvl < 1137$~nm.
\citeauthor{SPS98} [\citeyear{SPS98}, Figure~2] show residual surface
absorption spectra of \OdOd, \Od, and \HdO\ in the region $610 < \wvl
< 680$~nm.    
To complement these studies, our Plate~\ref{plt:odac} shows the
modeled absorption optical depth $\tauabs(\wvl)$ of all significant
gaseous solar absorbers for $200 < \wvl < 1400$~nm.
\begin{plate*}
\begin{center}
\includegraphics[width=\fltwdtdblclm]{/data/zender/fgr/o2x/odxc}\vfill
\end{center}
\caption{
Decomposition of simulated absorption optical depth $\tauabs(\wvl)$ at
the CART site at local solar noontime on October 15, 1995.
Gaseous absorbers included are \OdOd\ (red), \OdNd\ (green), \HdO\
(yellow), \Ot\ (blue), \Od\ (light blue), \NOd\ (light green), and
\COd\ (black).   
The line spectra of \HdO, \COd, and \Od\ are averaged over 10\,\xcm.  
\label{plt:odac}}   
\end{plate*}
The spectra represent local solar noontime (solar zenith angle $\theta
= 45.128\dgr$) for October~15, 1995, at the Clouds and Radiation
Testbed (CART) site in Oklahoma.   
This was a pristine day with minimal aerosol burden ($0.04 <
\tauvisaer < 0.05$, not shown), low precipitable water (11.6\,\kgxmS),
low column \Ot\ (233~DU), and estimated column \NOd\ of $5 \times 
10^{19}$\,\xmS.    
\cite{ZBP97} describe the multistream narrow band (10\,\xcm) Malkmus
model, the experimental uncertainties, and all the input data in more
detail.  

Plate~\ref{plt:odac} isolates the solar absorption continuum 
(using $\xsxbOdX$ from Plate~\ref{plt:abs_xsx}) of each gaseous
absorber considered in the model.
\OdX\ shares with \HdO\ the distinction of overlapping with all the
significant solar absorbers.   
The strongest visible bands of \OdOd\ overlap with the center of the
\Ot\ Chappuis bands near 0.6\,\um.
The \OdOd\ band at $0.630$\,\um\ overlaps the discrete \Od\ $\gamma$
band at 0.629\,\um.
The strong 1.06\,\um\ band partially overlaps a saturated water vapor
band, but its center is the strongest nonoverlapped feature of \OdOd\
in the spectrum.
Absorption at 1.27\,\um\ shows strong overlap among the oxygen infrared 
band, \OdOd, and \OdNd.
Although 1.27\,\um\ is in a water vapor window, both wings of the \OdX\
bands centered there overlap with weak \COd\ bands and with the
wings of \HdO\ bands. 
This overlap with \HdO\ lessens somewhat in dry air masses where
column moisture is $< 5$\,\kgxmS\ (e.g., Antarctica), but \OdX\ 
absorption is only weakly sensitive to variations in \HdO\ (see
section~\ref{sxn:csn}). 
As Plate~\ref{plt:odac} shows, contributions of \OdNd\ and \OdOd\ to
the total 1.27\,\um\ continuum are nearly equal if $\epsNd = 0.2$.
Using $\epsNd = 0.3$ (and weaker $\xsxbOdX(1.27$\,\um)), \cite{MCB98}  
found \OdNd\ absorption may exceed \OdOd\ absorption in this band by
15\%.  

The \OdOd\ band at 0.477\,\um\ occupies a relatively transparent
spectral region where it is often stronger than any other gaseous
absorber, i.e., \Ot\ or \NOd.
This band is discernible in routine optical depth spectra taken at the 
CART site and other locations \cite[]{MBB99}.
Note that neglecting \OdOd\ absorption in its stronger bands
could bias operational retrievals of other absorber amounts (e.g.,
\Ot, \NOd, or aerosol), leading to overestimates of order 0.01 optical 
depths.  
Fortunately, we are aware of no operational retrievals that are 
compromised by neglect of \OdOd\ absorption.
For this particular profile, we estimate \OdX\ causes 2.3\,\wxmS, or
1.4\%, of the total modeled atmospheric solar absorption of 164\,\wxmS.
Accounting for this absorption reduces the discrepancy between
noontime modeled and observed surface insolation (773 and 760\,\wxmS,
respectively) from 13.0 to 11.2\,\wxmS\ \cite[]{ZBP97}.   
% swnb -K -U -p ${DATA}/arese/clm/951015_1200_arese_clm.nc -d ${DATA}/tmp/951015_1200_arese_mdl_clr_aer_O2O2_O2N2.nc &
% swnb -p ${DATA}/arese/clm/951015_1200_arese_clm.nc -d ${DATA}/tmp/951015_1200_arese_mdl_clr_aer.nc &
% ncks -u -H -C -F -d levp,111 -v flx_bb_dwn_sfc,flx_bb_abs_atm,flx_bb_dwn_drc,flx_bb_dwn_dff ${DATA}/tmp/951015_1200_arese_mdl_clr_aer.nc
% ncks -u -H -C -F -d levp,111 -v flx_bb_dwn_sfc,flx_bb_abs_atm,flx_bb_dwn_drc,flx_bb_dwn_dff ${DATA}/tmp/951015_1200_arese_mdl_clr_aer_O2O2_O2N2.nc

% swnb -r 0.2 -A -B -K -U -p ${DATA}/aca/mls_icrccm_92lvl.nc -d ${DATA}/tmp/mls_icrccm_clr_cln_O2O2_O2N2.nc &
% swnb -r 0.2 -A -B -C -K -U -p ${DATA}/aca/mls_icrccm_92lvl.nc -d ${DATA}/tmp/mls_icrccm_clr_cln_O2O2_O2N2_xCO2.nc &
% swnb -r 0.2 -A -B -O -K -U -p ${DATA}/aca/mls_icrccm_92lvl.nc -d ${DATA}/tmp/mls_icrccm_clr_cln_O2O2_O2N2_xO2.nc &
% swnb -r 0.2 -A -B -p ${DATA}/aca/mls_icrccm_92lvl.nc -d ${DATA}/tmp/mls_icrccm_clr_cln.nc &
% swnb -r 0.2 -m 0.1 -A -B -K -U -p ${DATA}/aca/mls_icrccm_92lvl.nc -d ${DATA}/tmp/mls_icrccm_cld_cln_O2O2_O2N2.nc &
% swnb -r 0.2 -m 0.1 -A -B -C -K -U -p ${DATA}/aca/mls_icrccm_92lvl.nc -d ${DATA}/tmp/mls_icrccm_cld_cln_O2O2_O2N2_xCO2.nc &
% swnb -r 0.2 -m 0.1 -A -B -p ${DATA}/aca/mls_icrccm_92lvl.nc -d ${DATA}/tmp/mls_icrccm_cld_cln.nc &
% ncdiff -O -v htg_rate_bb,flx_bb_dwn_sfc,flx_bb_abs_atm,flx_bb_abs_sfc ${DATA}/tmp/mls_icrccm_clr_cln_O2O2_O2N2.nc ${DATA}/tmp/mls_icrccm_clr_cln.nc ${DATA}/tmp/mls_icrccm_clr_cln_O2O2_O2N2_frc.nc
% ncdiff -O -v htg_rate_bb,flx_bb_dwn_sfc,flx_bb_abs_atm,flx_bb_abs_sfc ${DATA}/tmp/mls_icrccm_clr_cln_O2O2_O2N2.nc ${DATA}/tmp/mls_icrccm_clr_cln_O2O2_O2N2_xCO2.nc ${DATA}/tmp/mls_icrccm_clr_cln_CO2_frc.nc
% ncdiff -O -v htg_rate_bb,flx_bb_dwn_sfc,flx_bb_abs_atm,flx_bb_abs_sfc ${DATA}/tmp/mls_icrccm_clr_cln_O2O2_O2N2.nc ${DATA}/tmp/mls_icrccm_clr_cln_O2O2_O2N2_xO2.nc ${DATA}/tmp/mls_icrccm_clr_cln_O2_frc.nc
% ncdiff -O -v htg_rate_bb,flx_bb_dwn_sfc,flx_bb_abs_atm,flx_bb_abs_sfc ${DATA}/tmp/mls_icrccm_cld_cln_O2O2_O2N2.nc ${DATA}/tmp/mls_icrccm_cld_cln.nc ${DATA}/tmp/mls_icrccm_cld_cln_O2O2_O2N2_frc.nc
% ncdiff -O -v htg_rate_bb,flx_bb_dwn_sfc,flx_bb_abs_atm,flx_bb_abs_sfc ${DATA}/tmp/mls_icrccm_cld_cln_O2O2_O2N2.nc ${DATA}/tmp/mls_icrccm_cld_cln_O2O2_O2N2_xCO2.nc ${DATA}/tmp/mls_icrccm_cld_cln_CO2_frc.nc
% ncks -u -H -C -F -v htg_rate_bb,flx_bb_dwn_sfc,flx_bb_abs_atm,flx_bb_abs_sfc ${DATA}/tmp/mls_icrccm_clr_cln_O2O2_O2N2_frc.nc | m

\subsection{Vertical Heating Profile}\label{sxn:htg}
To illustrate the vertical profile of \OdX\ heating, we simulated
``climatological'' clear and cloudy sky, aerosol-free heating profiles
for a 92 level midlatitude summer atmosphere \cite[]{Bri921}.
Figure~\ref{fgr:odx_htg} shows the resulting vertical profile of total 
solar heating for the clear sky profile, the \OdX\ contribution to
this clear sky heating, and the \OdX\ cloudy sky heating.   
\begin{figure}
\begin{center}
% NB: This figure produced by ~/idl/mie.pro:odx_htg_gph()
\includegraphics*[width=\fltwdtsngclm]{/data/zender/fgr/o2x/o2x_htg}\vfill
\end{center}
\caption{
Vertical profile of solar gaseous heating (\kxd) for a standard,
climatological midlatitude summer atmosphere.
Shown are the total solar heating rate in clear sky (solid) and,
exaggerated by a factor of 100, the \OdX\ contributions to the heating
in clear sky (dashed) and in the presence of a 100\,\gxmS, 100\,\mb\ thick
stratus cloud (dotted).
\label{fgr:odx_htg}}   
\end{figure}
The solar zenith angle ($\theta = 60\dgr$), surface albedo ($A
= 0.2$), and cloud properties (100\,\mb thick liquid stratus cloud
centered at 850\,\mb, condensate path 100\,\gxmS) represent
climatological atmospheric properties.  
The \OdX\ heating profiles are multiplied by a factor of 100 for
clarity.  

The clear sky \OdX\ heating increases rapidly with pressure, peaking
at the surface. 
Thus \OdX\ clear sky heating weakly destabilizes the atmospheric
column. 
At moderate zenith angles, \OdX\ absorption is relatively weak (i.e.,
$\odxcOdX < 0.1$) and is not dominated by overlap with other gases
(Plate~\ref{plt:odac}). 
Heating by a well-mixed molecule in this nonoverlapped, optically
thin limit is constant with air density.  
In constrast, it is easy to show that in this limit, heating by a
well-mixed collision complex (or dimer) increases linearly with
air density.  
The high correlations of the \OdX\ heating in Figure~\ref{fgr:odx_htg}
with air density (not shown) confirm that \OdX\ heating obeys the
weakly absorbing linear limit in both clear and cloudy skies.
The absorbed energy actually increases quadratically with dry air
density, i.e., linearly with \OdX\ abundance (\ref{eqn:N}), but one
factor of density (the heat capacity of air in \jxmCK) is removed in
converting from a volumetric heating (\wxmC) to an absolute heating
rate (\kxd).  
We note that \OdX\ is the only significant solar absorber with this 
vertical signature. 
The \OdX\ heating rates in Figure~\ref{fgr:odx_htg} do not appear 
linear because they are expressed as functions of pressure rather
than density. 

Figure~\ref{fgr:odx_htg} shows that clouds significantly enhance
\OdX\ heating above cloud top and reduce heating beneath cloud top
(dotted line).  
Thus clouds shift the \OdX\ heating profile maximum upward to the
level of the cloud top.
This feature appears prominently in climatological averages of \OdX\
heating shown in section~\ref{sxn:vrt}.

In summary, \OdX\ bands are present in much of the visible spectrum 
and in important near-infrared windows.
These bands are well characterized by the linear limit of small
absorber paths. 
The vertical profile of clear sky \OdX\ heating weakly decreases
the atmospheric stability. 

\subsection{Implementation of Absorption in GCM}
To estimate the global abundance and forcing of \OdX, we employed 
the National Center for Atmospheric Research (NCAR) Community Climate
Model, Version~3 (CCM3) general circulation model \cite[]{KHB98}.
The {CCM3} employs an 18 bin $\delta$-Eddington approximation for the 
solar spectral region \cite[]{Bri921}.   
The eight bins for $\lambda < 0.7$\,\um\ represent monochromatic
intervals, but water vapor absorption in the near-infrared is
represented by a seven bin $k$-distribution technique.
We created binary cross sections $\xsxbar$ for each CCM spectral
bin by spectrally averaging $\xsxbOdX(\wvl)$ from the laboratory
resolution ($\sim 0.15$~nm) to the CCM resolution.  
The near-infrared 1.06 and 1.27\,\um\ continuum absorption bands were
mapped into the CCM $k$-distribution bins corresponding to the mean
intensity of the nearest water vapor bands.
This procedure included weighting the high-resolution cross sections
by the incident solar flux at the top of the atmosphere in the more
detailed narrow band multistream (NBM) model \cite[]{ZBP97}.  
In offline single-column comparisons, the CCM3 radiation model
and the NBM model \OdX\ absorption agreed to within 10\% for
% DBG: XXX 
standard atmospheric conditions ranging from Arctic to Tropical, so no
tuning of the $\xsxbar$ was required.  

\section{Global Abundance and Absorption}\label{sxn:res} 

The NCAR CCM3 general circulation model \cite[]{KHB98} has 18 levels
in the vertical and a horizontal resolution of $\sim 2.8\dgr
\times 2.8\dgr$ near the equator.   
The dynamical timestep is 20~min, but radiative fluxes are computed 
hourly. 
Cloud diagnosis is based on convective activity, stability, and
relative and absolute humidity.
Hydrometeor optical properties are partitioned into liquid and ice
components based on temperature and pressure.  
\cite{KHH98} show that compared to observations, the CCM3 produces 
little or no bias in clear sky radiative fluxes and only small biases
in cloudy sky fluxes. 
Thus the CCM3 is suitable for assessing the radiative forcing of other
(noncloud) atmospheric constituents.

We performed three CCM integrations in order to separately estimate
the forcings caused by \OdOd, \OdNd, and $\OdOd + \OdNd$,
respectively. 
Each integration is 1~year long and is initialized from the same
fully spun up data set for January~1. 
The integrations employed climatological sea surface temperatures, and
all forcings are seasonally averaged (i.e., 3~month means) in 
order to minimize the influence of variability on the results.
The solar radiation physics were called twice each solar radiation
time step (60~min): once with and once without \OdX\ heating. 
\OdX\ forcing is the difference between these values.
We archived \OdX\ abundance and forcing every times tep (20~min), but
these forcings were not allowed to affect the predicted climate. 

\subsection{Annual Mean Abundance and Absorption}\label{sxn:ann_avg}
Collision complex abundance is a statistic of the second moment of
dry air density (\ref{eqn:N}).
Plate~\ref{plt:O2O2_ann_avg} shows the annual mean column abundance
of \OdOd, $\npcOdOd$.  
\begin{plate*}
\begin{center}
\includegraphics[width=\fltwdtdblclm]{/data/zender/fgr/o2x/dmr04_8589_NPCO2O2}\vfill
%\includegraphics[width=\fltwdtdblclm]{/data/zender/fgr/o2x/dmr04_8589_NPCO2N2}\vfill
\end{center}
\caption{
Annual mean column abundance $\npcOdOd$ ($\times 10^{42}$ \mlcSxcmF)
of \OdOd.  
\label{plt:O2O2_ann_avg}}   
\end{plate*}
The major meridional gradient is caused by the poleward decrease in
zonal mean temperature.
Cold polar air of a given mass is denser than warm tropical air.
Since $\cncOdOd$ is the square of $\cncOd$ (\ref{eqn:N}), $\npcOdX$ is 
$\sim 20\%$ greater in the Arctic than the tropics at the same sea
level pressure. 
The same would be true of the Antarctic, except the Antarctic plateau
displaces the densest portion of the troposphere, significantly
reducing \OdX\ column abundance.
The Tibetan Plateau, Greenland, East Africa, the Andes, and the
Rockies also show significant orographic reduction in $\npcOdX$.
Note that orographic reduction of $\npcOdX$ is time-invariant, in
contrast to effects of seasonally varying temperature and pressure in
the extratropics, discussed below.  
Whereas the zonal mean column abundance of a well-mixed molecule
(e.g., $\npcOd$) decreases by $30\%$ from sea level in the Arctic to
the Antarctic plateau at 700\,\mb, the corresponding decrease in
$\npcOdX$ exceeds 50\%. 
% ncks -H -u -F -C -v PS ${DATA}/sld012d/sld012d_xavg_8589_01.nc
% ncks -H -u -F -C -v NPCO2O2 ${DATA}/dmr04/dmr04_8589_01_x.nc
In summary, \OdX\ is more sensitive to the thermodynamic and
orographic environment than \Od\ or \Nd\ alone, owing to its quadratic
dependence on these constituents.   

Uncertainties in simulated $\npcOdX$ stem from biases in model
pressure $p$, temperature $T$, and surface elevation.
The maximum seasonal mean CCM temperature bias in the lower
troposphere is $< 6$\,\K\ \cite[]{HKH98}.
The maximum seasonal mean CCM grid point surface pressure bias is 
$< 1\%$ \cite[]{HHB98}. 
Biases in column \Od\ arising from the discretization of surface
elevation do not exceed 1\%.  
Assuming the worst case scenario (errors are additive) yields a
conservative uncertainty in seasonal grid point mean $\npcOd$ and
$\npcNd$ of $\pm 4\%$, implying uncertainty bounds for $\npcOdOd$ and
$\npcOdNd$ of $\pm 8\%$. 
The maximum uncertainty is over the Antarctic shelf and the
Tibetan Plateau, regions where variability and differences with
climatological analyses are relatively large \cite[]{HHB98}.
The simulated annual mean surface pressure is 984.5\,\mb, in excellent
agreement with 984.9\,\mb\ from National Centers for Environmental
Prediction (NCEP) reanalyses.  
% ncks -H ${DATA}/ncep/ncep_8589_xy.nc
This confirms that grid point biases in $\npcOdX$ are generally random,
not systematic, and we place the global mean bias conservatively at
$\pm 2\%$.   
As will be described below, grid point biases in simulated \OdX\
forcing are generally dominated by uncertainties in absorber
cross section and cloud vertical distribution, not \OdX\ abundance.   

The regional abundance of any well-mixed collision complex \XY\ is
determined by scaling Plate~\ref{plt:O2O2_ann_avg} by the product of 
the ratios $\cncX / \cncOd$ and $\cncY / \cncOd$.
Table~\ref{tbl:cnc} summarizes the scale factors (relative to \OdOd)
and global annual mean abundances of many well-mixed collision
complexes. 
\begin{table*}\setlength{\tblwdt}{30.0pc}
\begin{center}
\begin{minipage}{\tblwdt}
\caption{\parbox{\tblwdt}{Global Annual Mean Abundances of Well-Mixed Collision Complexes}}
\vspace{\cpthdrhlnskp}
\begin{tabular*}{\tblwdt}{l 
@{\extracolsep{\fill}} l 
r@{\extracolsep{0.0em}.}l 
@{\extracolsep{\fill}} l}
\tableline \rule{0.0ex}{\hlntblhdrskp}% 
Complex & Scale & \multicolumn{2}{c}{Abundance,} & Collision Induced Absorption Bands \\[0.0ex]
\XY & $\npcXY / \npcOdOd$ & \multicolumn{2}{c}{\mlcSxcmF} & \\[0.5ex]
\tableline \rule{0.0ex}{\hlntblntrskp}%
% ncks -H -C -F -u -v NPCO2O2 ${DATA}/dmr04/dmr04_8589_xy.nc
% NPCO2O2 = 1.23809e+17 MLC2/M5
% Locations of collision induced bands of well-mixed complexes:
% O2-O2: Fundamental (forbidden) vibration band 6.42 um (GoY89 p. 116)
% O2-O2: Collision induced rotation spectrum at 100 cm-1 = 100 um  (GoY89 p. 197)
% O2-O2: 1.58 um (MCB98 p. 3862)
% N2-N2: Fundamental (forbidden) vibration band 4.29 um (GoY89 p. 193)
% N2-N2: Collision induced rotation spectrum at 90 cm-1 = 110 um (GoY89 p. 192)
% O2-N2: N2 is O2 Partner at 1.27 um but not 1.06 um (SPS98 p. 3850)
% O2-N2: N2 is O2 partner in Herzberg (Sha77 p. 525)
% O2-A : A is O2 Partner at 1.27 um (SPS98 p. 3850) and Herzberg (ShP77 p. 433)
% O2-CH4 : CH4 is O2 Partner in Herzberg (OOK95 p. 11830)
% O2-CO2: CO2 is O2 Partner in Herzberg (OOK95 p. 11830)
% CO2-CO2: Rotational band at 200 um (GoY89 p. 206)
\OdOd & $1$ & $1$ & $24 \times 10^{43}$ & HC, 1.58\,\um\, 6.42\,\um\ \\[0.0ex]
\OdNd & $3.728$ & $4$ & $62 \times 10^{43}$ & HC, 4.29\,\um, 6.42\,\um, 100\,\xcm\ \\[0.0ex]
\NdNd & $13.90$ & $1$ & $72 \times 10^{44}$ & 4.29\,\um, 90\,\xcm\ \\[0.0ex]
\OdAr & $0.0446$ & $5$ & $52 \times 10^{41}$ & HC, 1.27\,\um, 6.42\,\um\ \\[0.0ex]
\OdCOd & $1.69 \times 10^{-3}$ & $2$ & $09 \times 10^{40}$ & 1.27\,\um\ \\[0.0ex]
\OdCHq & $8.1 \times 10^{-6}$ & $1$ & $00 \times 10^{38}$ & HC \\[0.0ex]
\COdCOd & $2.87 \times 10^{-6}$ & $3$ & $55 \times 10^{37}$ & 60\,\xcm\ \\[\tblendhlnskp]
\tableline
\end{tabular*}
\vspace*{\hlncptftrskp}\par\parbox{\tblwdt}{\hspace{1em}% NB: Use vspace*
Scale factors relate $\npcXY$ to $\npcOdOd$ (\ref{eqn:N}).
Band locations are approximate centers of absorption bands.
HC denotes Herzberg Continuum ($\sim$~200--250~nm).
\OdOd\ and \OdNd\ bands considered in this study
(Plate~\ref{plt:abs_xsx}) are omitted for brevity.  
Spectral data compiled from
\cite{Sha77}, \cite{ShP77}, \cite{GoY89}, \cite{OTN91}, \cite{OOK95},
\cite{SPS98}, and \cite{MCB98}. 
\label{tbl:cnc}}   
\end{minipage}
\end{center}
\end{table*}
The scale factors were computed using atmospheric abundances of
\citeauthor{GoY89} [\citeyear{GoY89}, p.~9], corrected to present-day
values in the case of \COd\ (355~ppm) and \CHq\ (1.7~ppm). 

The final column of Table~\ref{tbl:cnc} lists the approximate centers 
of the collision induced absorption bands due to each complex, apart
from the \OdX\ bands shown in Plate~\ref{plt:abs_xsx}.
Taken together, such bands appear in the spectrum from the Herzberg
continuum to the far infrared.
In addition to the bands shown in Plate~\ref{plt:abs_xsx}, collisions
involving \OdOd\ and \OdNd\ induce oxygen absorption in the Herzberg
continuum from 195 to 250~nm and in the oxygen (forbidden) 
fundamental vibration band at 6.42\,\um\ \cite[]{ShP77,OTN91}.   
Both \OdNd\ and \NdNd\ induce absorption in the nitrogen (forbidden)
fundamental band at 4.29\,\um. 
Insolation in these spectral regions is too weak to allow significant
solar absorption in absolute terms. 
However, it may be important to account for this absorption when
retrieving atmospheric parameters such as temperature from remotely
sensed spectra, for example, in the \COd\ 4.3\,\um\ band
\cite[][p. 192]{GoY89}.

Plate~\ref{plt:O2O2_O2N2_ann_avg_frc} shows the annual mean increase
in atmospheric absorption due to \OdX.
\begin{plate*}
\begin{center}
\includegraphics[width=\fltwdtdblclm]{/data/zender/fgr/o2x/dmr04_8589_FSATFRC}\vfill
%\includegraphics[width=\fltwdtdblclm]{/data/zender/fgr/o2x/dmr06_8589_FSATFRC}\vfill
%\includegraphics[width=\fltwdtdblclm]{/data/zender/fgr/o2x/dmr05_8589_FSATFRC}\vfill
\end{center}
\caption{
Annual mean increase in column atmospheric absorption (\wxmS) due to
\OdX.
%Note difference in scales.
%White regions exceed maximum scale value by $< 10\%$.
\label{plt:O2O2_O2N2_ann_avg_frc}}   
\end{plate*}
The annual mean forcing approximately resembles the abundance
(Plate~\ref{plt:O2O2_ann_avg}), though some new features are evident.
First note the enhanced atmospheric absorption above bright, low
surfaces (desert, stratus clouds, ice). 
Since the spectral optical depth of most \OdX\ bands is $< 0.05$
(Plate~\ref{plt:odac}), absorption in these bands is in the 
linear limit where a change in absorber path causes a proportionate
change in atmospheric absorption.  
Reflection from low, bright surfaces increases photon path lengths in
the \OdX-rich lower atmosphere.
This increases the mean \OdX\ path and thus the \OdX\ absorption. 
However, reflection from high clouds (e.g., tropical cirrus)
prevents photons from reaching the lower atmosphere and reduces
the mean \OdX\ path traversed by these photons.
This reduces \OdX\ absorption and causes the local minima in \OdX\
forcing in the Intertropical Convergence Zone (ITCZ) region. 
The zonal annual mean atmospheric absorption due to \OdX\ is 
$\sim 0.9$\,\wxmS\ and does not vary strongly with latitude (not
shown). 

It is interesting to note that geographic structure of \OdNd\
absorption closely resembles that of \OdOd\ absorption, so that their
sum, Plate~\ref{plt:O2O2_O2N2_ann_avg_frc}, adequately represents the
structure of each.
Since \OdNd\ absorption occurs entirely in the 1.27\,\um\ band,
it is enhanced relative to \OdOd\ over vegetated surfaces but
diminished over sea-ice and ocean due differences between the
near-infrared and visible albedos of these surfaces.
The magnitude of global mean \OdNd\ absorption is proportional to the 
assumed \Nd\ efficiency. 
This study assumes $\epsNd = 0.2$, which leads to \OdNd\ forcing being
20\% of \OdOd\ absorption.
As discussed in section~\ref{sxn:xsx}, $\epsNd$ may be $0.3$, in which
case \OdNd\ forcing would be 30\% of \OdOd\ forcing.   

It is useful to distinguish atmospheric absorption from other common
metrics of radiative forcing. 
Table~\ref{tbl:frc} summarizes the global mean statistics of \OdX\ 
atmospheric forcing, surface forcing, column radiative forcing, and
net radiative forcing at the tropopause. 
\begin{table}\setlength{\tblwdt}{\fltwdtsngclm}
\caption{Global Annual Mean Forcings of \OdX} 
\vspace{\cpthdrhlnskp}
\begin{tabular*}{\tblwdt}{l 
@{\extracolsep{\fill}} r@{\extracolsep{0.0em}.}l 
@{\extracolsep{\fill}} r@{\extracolsep{0.0em}.}l 
@{\extracolsep{\fill}} r@{\extracolsep{0.0em}.}l}
\tableline \rule{0.0ex}{\hlntblhdrskp}% 
Forcing, \wxmS\ & \multicolumn{2}{c}{\OdOd} & \multicolumn{2}{c}{\OdNd} & \multicolumn{2}{c}{Total} \\[0.5ex]
\tableline \rule{0.0ex}{\hlntblntrskp}%
Atmospheric Absorption & $0$ & $78$ & $0$ & $15$ & $0$ & $93$ \\[0.0ex]
% Atmospheric Absorption & $0.780251$ & $0.152012$ & $0.930421$ \\[0.0ex]
(same, but for clear sky) & ($0$ & $78$) & ($0$ & $17$) & ($0$ & $94$) \\[0.0ex]
% (same, but for clear sky) & ($0.775744$) & ($0.169158$) & ($0.942698$) \\[0.0ex]
Surface Insolation & $-0$ & $50$ & $-0$ & $10$ & $-0$ & $60$ \\[0.0ex]
% Surface Insolation & $-0.496327$ & $-0.100377$ & $-0.595738$ \\[0.0ex]
Surface Absorption & $-0$ & $42$ & $-0$ & $08$ & $-0$ & $51$ \\[0.0ex]
% Surface Absorption & $-0.42418$ & $-0.0849603$ & $-0.508373$ \\[0.0ex]
Sfc.\ $+$ Atm.\ Absorption & $0$ & $36$ & $0$ & $07$ & $0$ & $42$\\[0.0ex]
% Sfc.\ $+$ Atm.\ Absorption & $0.356073$ & $0.0670514$ & $0.422048$\\[0.0ex]
Net Flux at Tropopause & $0$ & $34$ & $0$ & $06$ & $0$ & $40$\\[\tblendhlnskp]
% Net Flux at Tropopause & $0.$ & $$ & $$\\[\tblendhlnskp]
\tableline
\end{tabular*}
\vspace*{\hlncptftrskp}\vfill\parbox{\tblwdt}{\hspace{1em}% NB: Use vspace*
Total forcings may be up to 20\% lower or 30\% higher than indicated.  
See text for uncertainty analysis.
Sfc, surface; Atm, atmosphere.}
\label{tbl:frc}   
\end{table}
The first row, atmospheric forcing, is the mean increase in
atmospheric absorption due to \OdX\
(Plate~\ref{plt:O2O2_O2N2_ann_avg_frc}), 0.93\,\wxmS.  
For comparison, the all sky solar absorption due to \HdO, \Ot,
\Od, and \COd\ is $\sim 43$, 14, 2, and 0.5\,\wxmS, respectively
\cite[]{KiT97} (\NOd\ abundance is highly heterogeneous and a reliable
estimate of its mean absorption has yet to be made).   
The second row shows that on average, the clear sky atmospheric
absorption by \OdX\ slightly exceeds the all sky absorption.

Atmospheric absorption at solar wavelengths reduces surface
insolation (third row of Table~\ref{tbl:frc}) and thus surface
absorption (fourth row). 
The fifth row shows the column radiative forcing, the sum of
atmospheric forcing and surface forcing.
This quantity  (column radiative forcing) equals the change in net
radiative flux at the top of the atmosphere.
The impact of atmospheric constituents on tropospheric climate is, by 
convention, quantified by the change in net radiative flux at the
tropopause, shown in row six.
More than 97\% of \OdX, and thus 97\% of \OdX\ absorption, is in the 
troposphere (see Plate~\ref{plt:O2O2_O2N2_csn_avg_x}).
Therefore the column radiative forcing of \OdX\ nearly equals the net
radiative forcing at the tropopause.

\subsection{Uncertainty in Mean \OdX\ Forcing}\label{sxn:err}  
The largest uncertainties in the estimates in Table~\ref{tbl:frc} stem
from uncertainties in cross section $\xsxbOdOd$, \Nd\ efficiency
$\epsNd$, \OdX\ abundance $\npcOdX$, and cloud vertical distribution.
The uncertainty ranges for these quantities are $\xsxbOdOd(\lambda)$,
$-30$ to $+10\%$, depending on $\lambda$ (see section~\ref{sxn:xsx});
$\epsNd$, $\pm 50\%$; and $\npcOdX$, $\pm 2\%$. 
Also, appending the \OdOd\ 1.58\,\um\ continuum of \cite{MCB98} to the 
\cite{SPS98} cross sections would increase our computed total
\OdX\ solar forcing by $\sim 4\%$.
As discussed in section~2.1, the possible temperature dependence of 
$\xsxbOdX(\lambda)$ is not yet fully understood
\cite[]{GOB90,NeB98,OFH98}.     
Therefore we have not included uncertainty due to possible temperature
dependence of $\xsxbOdX$ in our error analysis.

The simulated mean clear sky and all sky planetary albedos agree with
satellite observations to within 0.005 \cite[]{HKH98}, eliminating the
possibility of gross surface or cloud reflectance biases in estimated
\OdX\ forcing.
Modeled total cloud fraction, 0.588, is bracketed by the Nimbus~7 and 
International Satellite Cloud Climatology Project (ISCCP) estimates of
0.522 and 0.625, respectively \cite[]{HKH98}. 
The vertical distribution of cloudiness generally agrees with
observations for clouds below 400\,\mb\ but disagrees for cloud top
heights above 400\,\mb, a region where clouds efficiently reduce \OdX\
forcing.  
The CCM high cloud fraction, 0.34, exceeds ISCCP analyses by 0.20
\cite[]{HKH98}.  
There is considerable uncertainty in the observations, but if the
ISCCP total and high cloud estimates are correct, then we estimate the
modeled all sky \OdX\ forcing is up to 10\% too low because of
exaggerated amounts of high-level cloud relative to low-level cloud. 

Except for \OdX\ abundance, the aforementioned uncertainties are
systematic, not random. 
Summing the uncertainties for a conservative error estimate yields a
total uncertainty range for the all sky forcings in
Table~\ref{tbl:frc} of $-20$ to $+30\%$.     
Thus the uncertainty range for \OdX\ all sky absorption is
0.75--1.21\,\wxmS. 
The lower bound forcing corresponds to the 1.06 and 1.27\,\um\
$\xsxbOdX$ of \cite{MCB98} and \Nd\ efficiency $\epsNd = 20\%$
(\ref{eqn:tau}).   
The upper bound forcing uses $\xsxbOdX$ from \cite{GOB90} and
\cite{SPS98}, $\epsNd = 30\%$, and assumes that 4\% of \OdX\
absorption occurs at 1.58\,\um\ and that the CCM3 overpredicts high
level clouds relative to low-level clouds.  

The results in Table~\ref{tbl:frc} appear to contrast with two prior
studies. 
\cite{PEP97} estimate \OdOd\ enhances all sky heating relative to
clear sky heating by $\sim 7\%$ (neglecting the 1.27\,\um\ band).
Using identical \OdX\ cross sections to our study, \cite{SPS98}
estimated the mean atmospheric absorption due to \OdX\ is
0.9--1.3\,\wxmS. 
Both these prior studies used the same method to convert single-column
results to global mean results.
The method assumes 50\% of the globe is covered with cloud top heights
at 700\,\mb\ and that these clouds increase the photon path by 30~km
\cite[]{PEP97}.
Thus their conversion method assumes fewer and lower clouds than
predicted by the CCM3.
These assumptions both lead to larger forcings than shown in
Table~\ref{tbl:frc} and, taken together, suffice to explain the
differences between the studies.  
It is unclear whether assumptions regarding surface albedo and
elevation also contribute to these model differences.

% NB: Column 1 (O2O2) is from dmr06
% NB: Column 2 (O2N2) is from dmr05
% NB: Column 3 (O2O2+O2N2) is from dmr04
% ncks -H -u -F -C -v FSATFRC,FSDSFRC,FSNTFRC,FSNSFRC,FSATCFRC,FSNTCFRC,FSNSCFRC ${DATA}/dmr04/dmr04_8589_xy.nc
% Find relation between absorption increase and net radiative forcing
% at TOA and at tropopause for the 951015_1200 ARESE profile:
% ncdiff -O -v flx_bb_net,flx_bb_dwn_sfc,flx_bb_abs_atm,flx_bb_abs_sfc ${DATA}/tmp/951015_1200_arese_mdl_clr_aer_O2O2_O2N2.nc ${DATA}/tmp/951015_1200_arese_mdl_clr_aer.nc ${DATA}/tmp/951015_1200_arese_mdl_clr_aer_O2O2_O2N2_frc.nc
% ncks -u -H -C -F -v flx_bb_net,flx_bb_dwn_sfc,flx_bb_abs_atm,flx_bb_abs_sfc ${DATA}/tmp/951015_1200_arese_mdl_clr_aer_O2O2_O2N2_frc.nc | m
% Tropopause (change in temperature gradient) occurs at about 100 mb 
% Delta flx_bb_abs_atm=2.31 W m-2
% Delta flx_bb_dwn_sfc=-1.86 W m-2
% Delta flx_bb_abs_sfc=-1.51 W m-2
% Delta flx_bb_net(TOA)=0.801 W m-2
% Delta flx_bb_net(100 mb)=0.779 W m-2 
% Delta flx_bb_net(120 mb)=0.764 W m-2 
% Delta flx_bb_net(150 mb)=0.738 W m-2 
% Delta flx_bb_net(200 mb)=0.682 W m-2 
% Fraction of column O2O2 absorption above 100 mb is
% (0.801-0.779)/(2.31) ~= 0.010, or 1%. 
% Relative importance of stratospheric O2O2 increases over high clouds.
% Therefore estimate net flux at tropopause as (98% * FSAT_dlt)-(FSNS_dlt) 
% Show fraction of O2-O2 column above a given level:
% ncks -C -H -v cnc_O2_npl_O2_clm_frc,tpt ${DATA}/aca/trp_afgl_73lvl.nc | m

\subsection{Seasonal Structure of Abundance and Absorption}\label{sxn:csn}  
Owing to the seasonal cycles of insolation and temperature, 
annual mean forcings do not suffice to characterize the
geographic and vertical distribution of \OdX\ forcing as it impacts
the climate system. 
To illustrate this point, Figure~\ref{fgr:NPCO2O2_x} shows the zonal
mean of the column abundance of \OdOd\ for December--February (DJF)
and June--August (JJA). 
\begin{figure}
\begin{center}
\includegraphics[width=\fltwdtsngclm]{/data/zender/fgr/o2x/dmr04_8589_1202_0608_x_NPCO2O2}\vfill
\end{center}
\caption{
Zonal mean column abundance of \OdOd, $\npcOdOd$ ($\times
10^{42}$\,\mlcSxcmF), for DJF (solid line) and JJA (dashed line).   
\label{fgr:NPCO2O2_x}}
\end{figure}
The increases in $\npcOdX$ from summer to winter in the North and
South polar regions are 8--10\% and 1--3\%, respectively. 
% ncks -H -u -F -C -v NPCO2O2 ${DATA}/dmr04/dmr04_8589_01_x.nc
% ncks -H -u -F -C -v NPCO2O2 ${DATA}/dmr04/dmr04_8589_07_x.nc
These increases are entirely due to seasonal changes of pressure and
temperature.  
Although the solar radiative forcing of \OdX\ during polar winter is
zero, this seasonal cycle in $\npcOdX$ does affect \OdX\ absorption
bands in the far infrared (see Table~\ref{tbl:cnc}) throughout the
year. 
The meridional gradient and seasonal changes in $\npcOdX$ depicted in
Figure~\ref{fgr:NPCO2O2_x} apply to all other collision complexes in
Table~\ref{tbl:cnc} as well. 

The strongest modulator of solar radiative forcing is, of course, the
seasonal change of insolation.
Plate~\ref{plt:O2O2_O2N2_csn_avg} shows the geographic distribution
of the seasonal mean atmospheric absorption by \OdX.
\begin{plate*}
\begin{center}
\includegraphics[width=0.5\fltwdtdblclm]{/data/zender/fgr/o2x/dmr04_8589_1202_FSATFRC}%
\includegraphics[width=0.5\fltwdtdblclm]{/data/zender/fgr/o2x/dmr04_8589_0608_FSATFRC}%

\includegraphics[width=0.5\fltwdtdblclm]{/data/zender/fgr/o2x/dmr04_8589_0305_FSATFRC}%
\includegraphics[width=0.5\fltwdtdblclm]{/data/zender/fgr/o2x/dmr04_8589_0911_FSATFRC}%
\end{center}
\caption{
Seasonal mean increase in atmospheric solar absorption (\wxmS) due to \OdX. 
(a) December--February (DJF).
(b) March--May (MAM).
(c) June--August (JJA).
(d) September--November (SON).
White regions exceed maximum scale value by $< 0.7$\,\wxmS.
\label{plt:O2O2_O2N2_csn_avg}}
\end{plate*}
\OdX\ absorption is greatest in polar regions in summer, where peak
insolation coincides with the greatest \OdX\ abundances
(Figure~\ref{fgr:NPCO2O2_x}). 
Plates~\ref{plt:O2O2_O2N2_csn_avg}a and~\ref{plt:O2O2_O2N2_csn_avg}c
show southern and northern summer forcings are nearly symmetric
about the equator.
Arctic forcing is 10--30\% stronger than Antarctic forcing, mainly
the Antarctic plateau reduces $\npcOdX$.
% ncks -H -u -F -C -v FSATFRC ${DATA}/dmr04/dmr04_8589_01_x.nc
Most remarkable are the large polar regions where atmospheric \OdX\
absorption approaches or exceeds 2.0\,\wxmS\ on seasonal timescales.  
Seasonal sea ice boundaries appear in
Plates~\ref{plt:O2O2_O2N2_csn_avg}a--\ref{plt:O2O2_O2N2_csn_avg}d
where they demarcate the open ocean from regions of higher absorption.  
Equinoctial forcings (Plates~\ref{plt:O2O2_O2N2_csn_avg}b
and~\ref{plt:O2O2_O2N2_csn_avg}d) extend farther into the winter
hemisphere than solstitial forcings, and are weaker.  
%Plate~\ref{plt:O2O2_O2N2_csn_avg} can be used to estimate the
%corrections needed to account for \OdX\ in models which do not yet
%incorporate its affects.

Surface albedo, elevation, and clouds all play roles in determining
the seasonal forcing maxima near the poles. 
To help disentangle these roles, we show in Figure~\ref{fgr:FSATFRC_x}
the zonal mean of the winter and summer seasonal forcings
(Plates~\ref{plt:O2O2_O2N2_csn_avg}a
and~\ref{plt:O2O2_O2N2_csn_avg}c) due to \OdX.   
\begin{figure}
\begin{center}
% NB: This figure made with ~/idl/mie.pro:gcm_vy_gph()
\includegraphics[width=\fltwdtsngclm]{/data/zender/fgr/o2x/dmr04_8589_1202_0608_x_FSATFRC}\vfill
\end{center}
\caption{
Zonal seasonal mean increase in atmospheric solar absorption (\wxmS)
due to \OdX: DJF (solid line) and JJA (dashed line).  
\label{fgr:FSATFRC_x}}
\end{figure}
The influence of clouds and surface albedo on \OdX\ forcing is
clarified by comparing the seasonal forcings in
Figure~\ref{fgr:FSATFRC_x} to the seasonal abundances in
Figure~\ref{fgr:NPCO2O2_x}.  
% ncwa -O -a lon,lat -d lat,-90.0,-70.0 -w gw -v FSATFRC,NPCO2O2 ${DATA}/dmr04/dmr04_8589_1202.nc foo.nc
% ncwa -O -a lon,lat -d lat,70.0,90.0 -w gw -v FSATFRC,NPCO2O2 ${DATA}/dmr04/dmr04_8589_0608.nc foo.nc
% ncks -H -u foo.nc
The zonal mean summertime \OdX\ forcing in the Antarctic
(70--90$\dgr$S), 1.69\,\wxmS, is 96\% of the forcing in the Arctic
(70--90$\dgr$N), 1.76\,\wxmS, although the Antarctic has only 74\%
of the \OdX. 
Thus surface and low-level cloud reflectance makes \OdX\ forcing 30\%
more efficient (per unit abundance) in the Antarctic than in the
Arctic.  
Within 5\dgr\ of the poles, the forcing efficiency of \OdX\ is 
75\% greater in the Antarctic than in the Arctic.
The change in forcing efficiency reduces, and nearly eliminates, the
surface elevation-induced disparity in forcing between the poles.

The large difference between the forcing efficiencies of \OdX\ in the 
Arctic and the Antarctic is caused by differences in the surface
albedo, cloud distribution, and water vapor column between the two
polar climate systems. 
First, the simulated visible surface albedo of Arctic sea ice is $\sim 
0.6$, much lower than the simulated visible surface albedo of the
Antarctic continent, $\sim 0.95$ \cite[]{BrB981}.   
Offline sensitivity studies show that in the absence of clouds, the
higher surface reflectance underlying Antarctic \OdX\ increases
forcing (per unit abundance) by $\sim 25\%$ relative to the Arctic.  
Observations suggest the modeled surface albedo and low-level cloud
fraction in the summertime Arctic are $\sim 0.15$ and 0.10 too high,
respectively \cite[]{BrB981}.   
%Corresponding biases in the Antarctic are much smaller, only $\sim
%0.5$ and $\sim 0.10$, respectively. 
This suggests the simulated summertime Arctic forcing is too high,
which would amplify the disparity between Arctic and Antarctic forcing
efficiencies. 

A second, but much weaker, effect is due to the change in simulated
precipitable water content from the summertime Arctic (11.5\,\kgxmS) to 
the Antarctic (2.7\,\kgxmS).
Both are in fair agreement with European Centre for Medium-Range
Weather Forecasts (ECMWF) analyses \cite[]{BrB981}. 
The drier Antarctic column makes the wings of the near-infrared
water vapor bands (compare Plate \ref{plt:odac}) more transmissive
(less saturated).
This increases absorption per unit \OdX\ by $\sim 2\%$. 
% ncwa -O -a lon,lat -d lat,-90.0,-70.0 -w gw -v TMQ ${DATA}/sld012d/sld012d_8589_1202.nc foo.nc
% ncwa -O -a lon,lat -d lat,70.0,90.0 -w gw -v TMQ ${DATA}/sld012d/sld012d_8589_0608.nc foo.nc
% ncwa -O -a lon,lat -d lat,-90.0,-70.0 -w gw -v TMQ ${DATA}/sld012d/sld012d_8589.nc foo.nc
% ncwa -O -a lon,lat -d lat,70.0,90.0 -w gw -v TMQ ${DATA}/sld012d/sld012d_8589.nc foo.nc
% ncks -H -u foo.nc
% clm -n -h 0.2 -i ${DATA}/arese/clm/951015_1200_arese_clm.nc -o ${DATA}/arese/clm/951015_1200_arese_clm_mpcH2O_0.2.nc
% clm -n -h 0.6 -i ${DATA}/arese/clm/951015_1200_arese_clm.nc -o ${DATA}/arese/clm/951015_1200_arese_clm_mpcH2O_0.6.nc
% ncks -u -H -C -v mpc_H2O,cnc_O2_npl_O2_clm ${DATA}/arese/clm/951015_1200_arese_clm_mpcH2O_0.2.nc
% ncks -u -H -C -v mpc_H2O,cnc_O2_npl_O2_clm ${DATA}/arese/clm/951015_1200_arese_clm_mpcH2O_0.6.nc
% swnb -r 0.60 -p ${DATA}/arese/clm/951015_1200_arese_clm_mpcH2O_0.2.nc -d ${DATA}/tmp/951015_1200_arese_mdl_clr_aer_mpcH2O_0.2_alb_0.60.nc &
% swnb -r 0.60 -K -U -p ${DATA}/arese/clm/951015_1200_arese_clm_mpcH2O_0.2.nc -d ${DATA}/tmp/951015_1200_arese_mdl_clr_aer_O2O2_O2N2_mpcH2O_0.2_alb_0.60.nc &
% swnb -r 0.95 -p ${DATA}/arese/clm/951015_1200_arese_clm_mpcH2O_0.2.nc -d ${DATA}/tmp/951015_1200_arese_mdl_clr_aer_mpcH2O_0.2_alb_0.95.nc &
% swnb -r 0.95 -K -U -p ${DATA}/arese/clm/951015_1200_arese_clm_mpcH2O_0.2.nc -d ${DATA}/tmp/951015_1200_arese_mdl_clr_aer_O2O2_O2N2_mpcH2O_0.2_alb_0.95.nc &
% swnb -r 0.95 -p ${DATA}/arese/clm/951015_1200_arese_clm_mpcH2O_0.6.nc -d ${DATA}/tmp/951015_1200_arese_mdl_clr_aer_mpcH2O_0.6_alb_0.95.nc &
% swnb -r 0.95 -K -U -p ${DATA}/arese/clm/951015_1200_arese_clm_mpcH2O_0.6.nc -d ${DATA}/tmp/951015_1200_arese_mdl_clr_aer_O2O2_O2N2_mpcH2O_0.6_alb_0.95.nc &
% swnb -r 0.95 -p ${DATA}/arese/clm/951015_1200_arese_clm.nc -d ${DATA}/tmp/951015_1200_arese_mdl_clr_aer_mpcH2O_1.0_alb_0.95.nc &
% swnb -r 0.95 -K -U -p ${DATA}/arese/clm/951015_1200_arese_clm.nc -d ${DATA}/tmp/951015_1200_arese_mdl_clr_aer_O2O2_O2N2_mpcH2O_1.0_alb_0.95.nc &
% ncks -u -H -C -v flx_bb_abs_atm ${DATA}/tmp/951015_1200_arese_mdl_clr_aer_mpcH2O_0.2_alb_0.60.nc
% ncks -u -H -C -v flx_bb_abs_atm ${DATA}/tmp/951015_1200_arese_mdl_clr_aer_O2O2_O2N2_mpcH2O_0.2_alb_0.60.nc
% ncks -u -H -C -v flx_bb_abs_atm ${DATA}/tmp/951015_1200_arese_mdl_clr_aer_mpcH2O_0.2_alb_0.95.nc
% ncks -u -H -C -v flx_bb_abs_atm ${DATA}/tmp/951015_1200_arese_mdl_clr_aer_O2O2_O2N2_mpcH2O_0.2_alb_0.95.nc
% ncks -u -H -C -v flx_bb_abs_atm ${DATA}/tmp/951015_1200_arese_mdl_clr_aer_mpcH2O_0.6_alb_0.95.nc
% ncks -u -H -C -v flx_bb_abs_atm ${DATA}/tmp/951015_1200_arese_mdl_clr_aer_O2O2_O2N2_mpcH2O_0.6_alb_0.95.nc
% ncks -u -H -C -v flx_bb_abs_atm ${DATA}/tmp/951015_1200_arese_mdl_clr_aer_mpcH2O_1.0_alb_0.95.nc
% ncks -u -H -C -v flx_bb_abs_atm ${DATA}/tmp/951015_1200_arese_mdl_clr_aer_O2O2_O2N2_mpcH2O_1.0_alb_0.95.nc
% h = is H2O scale factor, A is surface albedo
% OdX forcing (h=0.2, A = 0.60) is 3.29 W m-2 (out of 142)
% OdX forcing (h=0.2, A = 0.95) is 4.12 W m-2 (out of 162)
% Thus, Arctic -> Antarctic surface albedo change increases OdX forcing by 25%
% OdX forcing (h=0.2, A=0.95) is 4.12 W m-2 (out of 162)
% OdX forcing (h=0.6, A=0.95) is 4.072 W m-2 (out of 197)
% OdX forcing (h=1.0, A=0.95) is 4.035 W m-2 (out of 215)
% Thus, Arctic -> Antarctic H2O path change increases OdX forcing by 1.2% 
% Hence OdX forcing is not sensitive to H2O path

\subsection{Vertical Structure}\label{sxn:vrt} 
The vertical profile of \OdX\ heating is important in determining
possible dynamical feedbacks as well as the change in net flux at the
tropopause.  
Plate~\ref{plt:O2O2_O2N2_csn_avg_x} shows the vertical profile of the
seasonally and zonally averaged solar heating due to \OdX.  
\begin{plate*}
\begin{center}
\includegraphics[width=0.5\fltwdtdblclm]{/data/zender/fgr/o2x/dmr04_8589_1202_x_QRSFRC}%
\includegraphics[width=0.5\fltwdtdblclm]{/data/zender/fgr/o2x/dmr04_8589_0608_x_QRSFRC}%

\includegraphics[width=0.5\fltwdtdblclm]{/data/zender/fgr/o2x/dmr04_8589_0305_x_QRSFRC}%
\includegraphics[width=0.5\fltwdtdblclm]{/data/zender/fgr/o2x/dmr04_8589_0911_x_QRSFRC}%
\end{center}
\caption{
As in Plate~\ref{plt:O2O2_O2N2_csn_avg}, but for the vertical profile
of the heating ($\times 10^{-2}$\,\kxd) due to \OdX. 
\label{plt:O2O2_O2N2_csn_avg_x}}
\end{plate*}
The vertical axis is the CCM hybrid vertical coordinate times 1000
\cite[]{KHB98}.  
Thus the panels transition from a pure sigma coordinate
representation at the surface to a pure pressure representation near
200\,\mb. 
As mentioned previously, virtually all of the heating is confined 
to the troposphere.
\OdX\ heating decreases toward the winter hemisphere due to decreasing
daylight hours. 
Above 700\,\mb\ (300\,\mb\ above the surface) the heating decreases
with pressure due to decreasing \OdX\ abundance (compare
Figure~\ref{fgr:odx_htg}).  
Seasonal \OdX\ heating in excess of 0.02\,\kxd\ extends throughout
the polar summer troposphere.
This is 2--4\% of local solar heating owing to all other solar
absorbers (compare Figure~\ref{fgr:odx_htg}). 
The heating rate in the Antarctic summer troposphere 
(Plate~\ref{plt:O2O2_O2N2_csn_avg_x}a) is nearly 20\% stronger than
the Arctic summer troposphere (Plate~\ref{plt:O2O2_O2N2_csn_avg_x}c) 
due to its thinner atmosphere.

The heating peaks between 200--300\,\mb\ above the surface in all regions
every season. 
This peak does not coincide with the maximum \OdX\ abundance (which is
always at the surface) even though \OdX\ absorption is in the
optically thin limit (Plate~\ref{plt:odac}). 
As shown in Figure~\ref{fgr:odx_htg}, the location of maximum \OdX\
heating does coincide with maximum \OdX\ abundance except when
efficient scatterers such as clouds screen the \OdX\ in the lower
atmosphere from incoming solar radiation.  
Thus cloud reflection is responsible for shifting the mean vertical
location of maximum \OdX\ absorption, and thus heating, upward from
the surface by 200--300\,\mb.   
Plate~\ref{plt:O2O2_O2N2_csn_avg_x} supports this by showing that the
heating maxima as a function of latitude dips somewhat closer to the
surface in the subtropical regions characterized by large-scale
subsidence and relatively clear skies.

Radiative forcing in other nonoverlapped, optically thin bands of
well-mixed collision complexes (e.g., the \NdNd\ fundamental band at
4.29\,\um, see Table~\ref{tbl:cnc}), will show qualitatively similar
patterns to
Plates~\ref{plt:O2O2_O2N2_ann_avg_frc}--\ref{plt:O2O2_O2N2_csn_avg_x}.   
The magnitude of forcing by other bands is, of course, highly
sensitive to the spectral location of the particular band. 

\section{Discussion and Summary}\label{sxn:dsc} 

We have characterized the spectral, vertical, regional, and seasonal
atmospheric heating by \OdX.
The motivations for modeling the global distribution and forcing of 
\OdX\ were to refine and extend estimates of \OdX\ forcing inferred
from one-dimensional studies to the global climate scale. 
Our global simulations indicate that global annual mean atmospheric
absorption by \OdX\ is 0.75--1.2\,\wxmS\ (Table~\ref{tbl:frc}).  
This falls between the recent estimates of 0.9--1.3\,\wxmS\
\cite[]{SPS98} and 0.84\,\wxmS\ \cite[]{MCB98}. 
This absorption reduces surface insolation by 0.48--0.78\,\wxmS. 

\OdX\ absorption is highly localized in the troposphere and it
increases the net radiative flux at the tropopause by
0.32--0.52\,\wxmS. 
It is useful to compare this absolute radiative forcing by \OdX\
(whose abundance is not changing) to greenhouse gas forcing (due to
changes in the gases' abundance from preindustrial to present-day
levels). 
This comparison enables us to assess the magnitude of the perturbation
to the mean state climate caused by \OdX. 
The 0.32--0.52\,\wxmS\ forcing by \OdX\ is less than the anthropogenic 
forcing of \COd\ (1.75\,\wxmS) but comparable to forcing by \CHq\
(0.44\,\wxmS), stratospheric \HdO\ (0.15\,\wxmS), and \NdO\ (0.11\,\wxmS)
\cite[][p. 57]{SDW90}.   

By extending the simulations to the global scale, we were able to
characterize the spatial and temporal distributions of \OdX\ abundance
and forcing. 
\OdX\ abundance mimics the concentration of \Od\ and \Nd, except its
quadratic dependence on these constituents accentuates both its
spatial and temporal gradients.
This dependence causes a 20\% increase in \OdX\ abundance in the 
Arctic relative to the Tropics for the same sea level pressure.  
\OdX\ abundance depends most on zonal mean temperature, pressure,
and surface elevation.
The variations in zonal mean \OdX\ abundance due to surface elevation,
the annual mean meridional temperature gradient, and to seasonal
temperature variations are 40, 15, and 10\%, respectively.    
These results scale to all collision complexes of well-mixed gases,
for example, \NdNd\ and \OdAr.

Many features of \OdX\ radiative forcing are common to all solar
absorbers, but the relatively sharp gradients in \OdX\ abundance
enhance these features.
Surface and cloud reflectance modulate the solar forcing efficiency of
\OdX\ (forcing per unit abundance) by altering the mean \OdX\ path
traversed by the average photon.
These factors explain why annual mean collision complex forcing peaks
in the subtropics over bright deserts and regions of marine stratus.
The seasonal cycle of insolation causes the seasonal mean solar
forcing to peak in summertime polar regions.  
However, these summertime polar peaks are enhanced by bright surfaces
and extensive low-level clouds, especially in the Arctic.
The forcing efficiency of complexes in the Antarctic is 30--75\%
greater than in the Arctic.

In clear skies the magnitude of \OdX\ heating grows linearly with air
density, peaking at the surface.
This heating profile tends to weakly destabilize the atmospheric
column. 
On seasonal timescales, clouds shift the height of peak \OdX\
heating upward by 200--300\,\mb\ and smooth out the vertical decrease of
heating that would otherwise follow the rapid decrease of \OdX\
abundance with height. 
\OdX\ heating peaks at 0.02--0.03\,\kxd\ in the polar troposphere
in summer.
This is 2--4\% of local heating due to all other solar absorbers.

Many GCMs have strong cold biases at the summertime polar tropopause,
where the CCM3 (not shown) is up to 10--14\,\K\ too cold \cite[]{HKH98}. 
The CCM3 also suffers from a cold bias in the summertime polar
troposphere of 2--6\,\K\ \cite[]{BrB982}.    
Some, but probably not all, of these biases are due to the numerical 
treatment of dynamics \cite[]{WiO98}.
It is likely that allowing \OdX\ heating to affect the thermal
structure of the atmosphere will ameliorate, though not eliminate,
the remainder of these biases.
For example, single-column simulations show the radiative relaxation
timescale at the polar tropopause is $\sim 50$\,days. 
Thus the \OdX\ heating of $\sim 0.01$\,\kxd\
(Plate~\ref{plt:O2O2_O2N2_csn_avg_x}) increases the radiative
equilibrium temperature by $\sim 1$\,\K\ in this region.  

We quantified some fingerprints that distinguish absorption due to
well-mixed collision complexes (and well-mixed dimers) from absorption
due to well-mixed monomers.  
The most significant distinguishing feature is the stronger spatial and
temporal gradients of collision complex abundance relative to monomer
abundance. 
Whereas the zonal mean column abundance of a well-mixed molecule
such as \Od\ decreases by $30\%$ from sea level in the Arctic to the
Antarctic plateau at 700\,\mb, the corresponding decrease in $\npcOdX$ 
exceeds 50\%. 
In the weakly absorbing, nonoverlapped limit, well-mixed molecules
cause an increase in the solar heating rate that is constant with air
density, whereas well-mixed collision complexes such as \OdX\ cause
the solar heating rate to increase linearly with air density.  

\cite{ZBP97} demonstrated that \OdOd\ absorption improved agreement
between modeled and observed solar absorption in both clear and cloudy
skies during the Atmospheric Radiation Measurement (ARM) Enhanced
Shortwave Experiment (ARESE). 
Their total (direct $+$ diffuse) clear sky surface insolation
discrepancies ($\sim 15$\,\wxmS) were no larger than instrumental and
model uncertainties (e.g., aerosol properties). 
Working with an independent set of radiometric observations from
ARESE, \cite{KAC97} found excellent model agreement with direct beam
observations but reported a 30\,\wxmS\ model overestimate of the
instantaneous diffuse surface insolation during clear sky conditions
over multiple days during ARESE. 
They did not account for \OdX\ absorption.
As shown in section~\ref{sxn:ovr}, \OdX\ absorption reduced clear sky
noontime surface insolation by $\sim 2$\,\wxmS\ during ARESE. 
However, nearly $90\%$ of this \OdX\ absorption came from the direct
solar beam, and only $10\%$ occurred in the diffuse field.  
This illustrates the point that unknown or neglected gaseous
absorption processes cannot, by themselves (i.e., without invoking
aerosol processes), remedy a clear sky diffuse radiation bias because
gaseous absorption of the direct beam far exceeds gaseous absorption
of the diffuse field. 

Our global simulations show that in the annual mean, \OdX\ enhances
absorption equally in clear and cloudy skies. 
Thus \OdX\ neither exacerbates nor remediates solar absorption
discrepancies that are associated with clouds
\cite[]{StT902,CZM95,RaV97}.  
The agreement between clear and cloudy sky \OdX\ absorption is due to
the global mean cancellation of photon and absorber path length
changes due to clouds: 
Low-level clouds increase \OdX\ absorption relative to clear skies by
increasing path lengths, while middle- and high-level clouds decrease
\OdX\ absorption by screening photons from the lower atmosphere. 
We hypothesize that this agreement is predominantly due to the global
vertical distributions of cloud reflectivity and well-mixed molecules
and has little to due with spectral features particular to \OdX.   
There are two reasons to expect that this hypothesis is true.
First, \OdX\ absorption occurs at many locations in the visible and
near-infrared and is not correlated with water vapor or condensate
absorption bands.
Second, cloud transmission is nearly spectrally uniform for $0.3 <
\lambda < 1.5$\,\um, so clouds do not preferentially transmit any
energetically significant solar wavelength bands to the lower
atmosphere. 
This hypothesis should be tested because, if it is true, it implies
that globally averaged, well-mixed collision complexes or dimers
do not enhance absorption more in cloudy sky than in clear sky.  
A similar hypothesis may hold for well-mixed molecules in general. 

%Therefore our study suggests that, on a global scale, discrepancies
%between models and observations of cloudy sky absorption cannot be 
%caused by well-mixed collision complexes or dimers. 

%Our global results support an inference which may inform the current
%debate about the existence and causes of enhanced solar absorption in
%cloudy skies \cite[]{StT902,RaV97}.  
%...
%Considering that \OdX\ absorption occurs at many locations in the
%visible and near-infrared, and is not correlated with water vapor
%bands, it is possible that \OdX\ absorption is representative of
%featureles solar continuum absorption by any well-mixed collision
%complex or dimer. 
%Thus it is difficult to imagine that any unknown, well-mixed absorber
%with an unstructured absorption continuum at solar wavelengths can
%cause significantly more absorption in cloudy skies than in clear
%skies.     
%However, differential optical absorption spectroscopy of high
%precision atmospheric spectra has been used to constrain the magnitude
%of unaccounted-for, structured solar absorption.
%This technique has been applied to significant portions of the solar 
%spectrum in both clear sky, 
%$400 < \lambda < 680$~nm and $1.0 < \lambda < 3.3$~nm \cite[][]{PEP97,SPS98,MCB98}, 
%and cloudy sky, $450 < \lambda < 675$~nm \cite[][]{PEP97},
%conditions to show that any significant unaccounted-for absorption in
%these spectral regions must be unstructured.
%Therefore our study suggests that, on a global scale, discrepancies
%between models and observations of cloudy sky absorption cannot be 
%caused by well-mixed collision complexes or dimers. 

The radiative importance of other abundant, but not necessarily
well-mixed, collision complexes and dimers in the atmosphere should be  
explored.    
\cite{SPS98} point out that \Ar\ and \COd\ are also relatively
efficient partners for inducing \Od\ absorption at 1.27\,\um\ but note
that \OdAr\ and \OdCOd\ are too scarce (Table~\ref{tbl:cnc}) to cause
significant absorption.
%For example, the efficiency of \COd\ relative to \Od\ as a collision
%partner for \Od\ absorption at 1.27\,\um\ is $\sim 3.0$.
%Adjusting the \OdNd\ forcing in Table~\ref{tbl:cnc} by the appropriate
%factors yields an estimate for mean \OdCOd\ absorption of 0.001\,\wxmS.
\cite{ChG97} and \cite{SPS98} suggest \OdHdO\ and \HdOHdO\ as
promising candidates to examine in the search for neglected solar
absorbers.   

In summary, the recent discovery that the oxygen collision pairs
\OdOd\ and \OdNd\ absorb a small but significant fraction of the
globally incident solar radiation alters the long-standing view that
\HdO, \Ot, \Od, \COd, and \NOd\ are the only significant gaseous solar
absorbers in Earth's atmosphere. 
%The solar irradiance absorbed by the known remaining photolytic gases 
%(e.g., \NdO, \CHq) was, and still is, deemed negligible relative to
%these five absorbers. 
This study quantified the spectral, regional, vertical, and seasonal
patterns of \OdX\ abundance and radiative forcing.
We now know that globally and annually averaged, \OdX\ absorbs about
1\,\wxmS\ of solar radiation, more than \COd, and that \OdX\ does not
increase cloudy sky solar absorption relative to clear sky absorption.
Improved measurements of \OdX\ absorption cross sections in the near 
infrared will remove about half of the 25\% uncertainty in total
\OdX\ absorption.
Accounting for \OdX\ reduces discrepancies between models and
measurements of solar absorption.
\OdX\ absorption should therefore be included in high-resolution
radiative transfer models, remote-sensing retrieval algorithms, and in
large-scale atmospheric models used to simulate climate and climate
change. 
\OdX\ has the most potential to improve the simulated climate
in summertime polar regions. 
Future versions of the CCM will include this absorption.

% Balance preprint columns
\balance

% Appendices
\appendix

% Acknowledgements
\acknowledgments
S.~Solomon and R.~Portmann provided \OdX\ cross section data and
helpful suggestions at an early stage of this study.   
J.~Burkholder, E.~Mlawer, D.~Newnham, and K.~Pfeilsticker provided
useful comments on \OdOd\ cross section measurements and
uncertainties.  
B.~Briegleb, W.~Collins, and S.~Solomon made numerous suggestions
which improved the quality of the manuscript.
R.~Portmann and one anonymous reviewer provided constructive reviews.  
The NCAR ASP program provided support and encouragement. 
This research was supported in part by DOE ARM Program Grant
DEFG0593ER61376. 

% Bibliography
\bibliographystyle{agu}
\bibliography{bib}

\end{document}

% $Id$

% Purpose: Response to Reviewer A of Zen99 for JGR

% Usage:
% cd ~/crr;make -W o2x_jgr_02.tex o2x_jgr_02.ps;cd -
% cd ~/crr;make -W o2x_jgr_02.tex o2x_jgr_02.ps o2x_jgr_03.ps o2x_jgr_04.ps;cd -
% latex o2x_jgr_02; dvips -o /data/zender/ps/o2x_jgr_02.ps o2x_jgr_02.dvi

\documentclass[12pt,twoside]{article}

% Standard packages
\usepackage{graphicx} % defines \includegraphics*
\usepackage{natbib} % \cite commands from aguplus
\usepackage{ifthen} % Boolean and programming commands

% Personal packages
\usepackage{csz} % Library of personal definitions
\usepackage{dmn} % Dimensional units
\usepackage{chm} % Commands generic to chemistry
\input{jgr_abb} % JGR-sanctioned journal abbreviations

% Commands specific to this file
\usepackage{o2x} % Commands specific to O2*X work

\begin{document}
 
\noindent
Revisions to Manuscript JGRd-1999R036, \\
\begin{center}\normalsize
``Global climatology of abundance and solar absorption of oxygen
collision complexes'' 
\end{center}
\medskip\noindent
by Charlie Zender \\
\medskip\noindent
Date: July 11, 1999

%\section{Response to Reviewer~A}
\medskip\noindent\textbf{Response to Reviewer~A}\medskip

I thank the Reviewer for thoroughly reading the manuscript and
carefully commenting on the study.
This manuscript has benefitted from the Reviewer's constructive
criticism. 

%\subsection{Specific Comments}
\medskip\noindent\textbf{Specific Comments}

\begin{enumerate}
\item Abstract: The abstract reports the estimated radiative forcing
of \OdX\ as ranges whose maxima and minima were determined by the
uncertainty analysis. 
Thus, an important part of the uncertainty analysis is already present
in the abstract, which I have tried to keep concise.
But, since the Reviewer requests that the abstract contain more
information on the uncertainties in our results, I have added to the
abstract the sentence ``These ranges bracket the uncertainties due to
spectral absorption cross sections, \OdNd\ efficiency, \OdX\
abundance, and cloud distribution''.

\item Section 2.2: 
\begin{enumerate}
\item I believe the uncertainty analysis already adequately addresses
the issue raised by the Reviewer.
The Reviewer asks that the manuscript address the sensitivity of the
simulated \OdX\ absorption at 1.27~\um\ to uncertainty in the
partitioning of 1.27~\um\ oxygen absorption into line absorption
(due to magnetic dipole transition) and \OdX\ continuum absorption.  
As stated on p.~18 of the original manuscript, the lower bounds in
\OdX\ absorption that we report correspond to the \OdX\ cross-sections
of \cite{MCB98}, while the upper bounds corresponds to the
cross-sections of \cite{SPS98}.
As \cite{MCB98} note, if the 1.27~\um\ \Od\ line absorption they used
(from HITRAN96) is, in fact 15\% too high, then their 1.27~\um\ \OdX\
continuum absorption is too low and so their estimate of total \OdX\
absorption is a lower bound.
The most recent published measurements of \Od\ line parameters at
1.27~\um\ \cite[]{LSL98} support this interpretation. 

I could not find where \cite{SPS98} stated that updates to HITRAN
``would not impact their results''.
My understanding is that uncertainty in the integrated band strength
of the \Od\ magnetic dipole band may indirectly affect the \OdX\
continuum cross-sections proposed by \cite{SPS98}.
The two may be related because \cite{SPS98} base their estimate of
\OdX\ continuum absorption on previous laboratory measurements which
also had to remove the contribution of \Od\ line absorption at
1.27~\um. 
However, the studies examined by \cite{SPS98} were laboratory
measurements in which it was possible to deduce \OdX\ absorption by
isolating the quadratic (in \Od\ concentration) component of the
absorption \cite[e.g.,][]{GOB90} rather than by inferring \OdX\
absorption as the difference between modeled \Od\ line absorption and
total absorption (i.e., the residual method used by \cite{MCB98}).   
Thus to what extent the 1.27~\um\ \OdOd\ absorption estimated by
\cite{SPS98} may be influenced by a larger \Od\ line absorption band
strength is unclear.
But the above discussion shows why it is reasonable to take the 
cross-sections of \cite{SPS98} (together with $\epsNd = 30\%$)
as an upper bound for \OdX\ absorption.

In summary, the issue identified by the reviewer has, I feel, been
adequately examined by both \citeauthor{SPS98} and
\citeauthor{MCB98}. 
Without anything new to contribute to this issue, the most responsible 
course for the submitted manuscript is to alert the reader to the
existence of discrepancies and to refer them to the original studies. 

\item Side by side comparisons of \cite{MCB98} and \cite{SPS98} in a
line-by-line model are unwarranted because we would learn little that
is not already acknowledged and discussed in these studies.
By construction, the \citeauthor{MCB98} cross-sections will have a
much lower radiance residual when compared to the ASTI radiances
(which, as the reviewer notes, are from a single observation)
than will the \citeauthor{SPS98} cross-sections which are lab
measurements or are inferred from lab measurements taken over a wide 
dynamic range of conditions.
By the same token, the \citeauthor{MCB98} cross-sections are inferior
relative to \citeauthor{SPS98} when compared to the \cite{GOB90}
laboratory measurements of 1.06~\um\ band \OdOd\ absorption. 

In my opinion, Sections~2.2 and~4.2 adequately inform the reader about
the relevant uncertainties and their potential impact on our results.
Page~9 of our original manuscript states that differences in the two
studies' treatments of the near infrared \OdX\ continua at 1.06 and
1.27~\um\ are of order 30\%.  
We mention that \cite{MCB98} and \cite{SPS98} both comment on
this discrepancy.
Further, we conclude (p.~26) that improved laboratory measurements of
\OdX\ cross sections could remove about half of the uncertainty in
total \OdX\ absorption. 
Finally, adding more figures or tables to this study would conflict
with the desire of Reviewer~B to shorten the manuscript.

\item The section on p.~9 has been revised to clarify this point. 
It now reads ``For consistency, our study adopts the same cross-sections as
\cite{SPS98}, but we note two choices that keep our results for total
\OdX\ absorption closer to those of \cite{MCB98}.   
First, we use $\epsNd = 0.2$, a somewhat conservative value.
Second, we neglect the \OdOd\ absorption band near 1.58~\um.
This band absorbs about 9\% as much as the 1.27~\um\ band
\cite[]{MCB98}. 
Had we appended the 1.58~\um\ continuum to the \citeauthor{SPS98}
cross-sections, our computed \OdX\ solar forcing would increase by a
few percent.''
\end{enumerate}
\item Section 3: Agreed. 
The revised manuscript says ``weakly destabilizes''.
\item Section~4.2: 
The error analysis attempts to include all sources of error for which
published estimates exist. 
The uncertainty due to spectral absorption cross sections is described 
in Section~2.2, where the $\xsxbOdX$ data are discussed.
The discussion in Section~4.2 now refers the reader to Section~2.2,
which, on p.~9, has been expanded in response to the Reviewer's
suggestion:   
``The uncertainty in $\xsxbOdX(\wvl)$ is less than 10\% for $0.335 <
\wvl < 1.137$~\um\ \cite[]{GOB90}. 
Uncertainty in $1.27$~\um\ band absorption is $\sim 30\%$. 
This was estimated by extrapolating to global scale the difference
between single column simulations employing the continua proposed by
\citeauthor{SPS98} and by \citeauthor{MCB98}.''

The Reviewer is correct to question the omission of the 1.58~\um\ band
absorption from the error analysis.
The original manuscript did not include 1.58~\um\ band absorption in
the error estimate, because, in my subjective opinion, neglecting it
would offset likely overestimates of 1.27~\um\ band strength.
However, these are logically independent quantities and a conservative
error estimate must account for both to measure the true uncertainty.
Thus, the revised manuscript increases the upper bound error bar on
total \OdX\ absorption from 25\% to 30\% of our best guess to account
for the 1.58~\um\ band. 
The upper ranges of \OdX\ forcing have been been revised upwards
throughout the manuscript.
The ``best guess'' estimates shown in Table~2 are unchanged.
Text on p.~9 and p.~17 now includes the statement that adding the
1.58~\um\ continuum ``would increase our computed total \OdX\ solar
forcing by about~4\%.''

Section~4.2 of the revised manuscript includes the proviso
``As discussed in Section~2.1, the possible temperature dependence of 
$xsxbOdX(\lambda)$ is not yet fully understood
\cite[]{GOB90,NeB98,OFH98}.     
Therefore we have not included uncertainty due to possible temperature
dependence of $xsxbOdX$ in our error analysis.''
The relevant portion of Section~2.1 now includes results from
\cite{OFH98}. 

\item Section~4.3:
The Reviewer's observation that the same $\npcOdX$ produces greater
forcing in the Arctic than in the Southern mid-latitudes during JJA
is correct.
However, Figure~8 clearly shows that the reason for the greater
forcing efficiency in the Arctic is that the length of day in the
Arctic greatly exceeds that in the Southern mid-latitudes during JJA. 
Presumably if the \OdX\ forcing were first normalized by the TOA
insolation, the resulting differences would be due to surface albedo
and clouds.
However, in absolute terms, the seasonal change in insolation causes
a much greater change in forcing efficiency than albedo differences
cause. 

\item Figures: 
\begin{enumerate}
\item Agreed. 
The lines in Figures~1 and~2 are now thicker.
I changed one of the blues to a yellow to make it easier to
differentiate \Od\ from \OdNd. 
\item Agreed.
The publisher will add the `a'--`d' to the panels when they are in the
galley stage, and will do a neater job of it than I have been able to
with my software.    
\item Good eye. 
This gaffe has been fixed.
\end{enumerate}
\end{enumerate}

% Bibliography
\renewcommand\refname{\normalsize References}
\bibliographystyle{agu}
\bibliography{bib}

\end{document}



% $Id$

% Purpose: Response to Reviewer B of Zen99 for JGR

% Usage:
% cd ~/crr;make -W o2x_jgr_03.tex o2x_jgr_03.ps;cd -
% cd ~/crr;make -W o2x_jgr_03.tex o2x_jgr_02.ps o2x_jgr_03.ps o2x_jgr_04.ps;cd -
% latex o2x_jgr_03; dvips -o /data/zender/ps/o2x_jgr_03.ps o2x_jgr_03.dvi

\documentclass[12pt,twoside]{article}

% Standard packages
\usepackage{graphicx} % defines \includegraphics*
\usepackage{natbib} % \cite commands from aguplus
\usepackage{ifthen} % Boolean and programming commands

% Personal packages
\usepackage{csz} % Library of personal definitions
\usepackage{dmn} % Dimensional units
\usepackage{chm} % Commands generic to chemistry
\input{jgr_abb} % JGR-sanctioned journal abbreviations

% Commands specific to this file
\usepackage{o2x} % Commands specific to O2*X work

\begin{document}
 
\noindent
Revisions to Manuscript JGRd-1999R036, \\
\begin{center}\normalsize
``Global climatology of abundance and solar absorption of oxygen
collision complexes'' 
\end{center}
\medskip\noindent
by Charlie Zender \\
\medskip\noindent
Date: July 11, 1999

\medskip\noindent\textbf{Response to Reviewer~B}\medskip

I thank Dr. Portmann for thoroughly reading the manuscript and
carefully commenting on the study.
This manuscript has benefitted from his constructive criticism. 

\medskip\noindent\textbf{Specific Comments}

\begin{enumerate}
% 1
\item Condensing the manuscript: In my opinion, the two tables and
nine figures in the original manuscript were necessary to present a
climatology of some very general properties of atmospheric absorption
which have never been examined before.
In contrast to Reviewer~B, Reviewer~A seems to find all the original
figures useful and even asks for another figure (to compare radiance
residuals). 
In balancing my response to these reviews with my scientific
objectives, I have therefore elected to eliminate Figure~5b but to
keep Figures~6 and~8.
\begin{enumerate}
\item Done.
\item I disagree with Reviewer~B on his points~1b and~1c.
Figures~6 and~8 are essential to adequately explaining a central
result of the study concerning the geographical and seasonal
dependence of the forcing efficiency of \OdX\ (and thus other well
mixed collision complexes).
Figures~6 and~8 show, concisely and unequivocally, that the small
differences between the forcings in the polar summer are the result of
large differences in \OdX\ forcing efficiency between the Northern and
Southern hemispheres. 

Yes, Figure~6 is intended to show that the summertime and wintertime
\OdX\ abundances are very similar.  
But it also shows that the difference between the Northern and
Southern polar regions is nearly a factor of two (which is not
``small''). 
Figure~8 shows the \OdX\ forcing during polar summer is nearly
symmetric about the equator.
Unless Figure~6 is available, most readers will assume that the
JJA forcing is so similar to the DJF forcing in Figure~8 because the
\OdX\ loadings are similar. 
But Figure~6 shows unequivocally that the polar \OdX\ loadings are
significantly different (nearly a factor of two).

This finding is a general property of the atmosphere, illustrated here
with \OdX, but worthy of note for all collision pairs
(and, to a lesser extent, for gaseous solar absorption in general). 
A given mass of collision pairs will soak up 75\% more energy in
summer at the Sorth pole than at the North pole.
The forcing efficiency difference is 30\% for the whole polar region. 
Reviewer~A seemed to find the differences in forcing efficiency
interesting, and I believe many readers will as well.

Figures~4 and~7 alone are insufficient to make this point for two
reasons. 
First, seasonal plots of $\npcOdX$ are needed to show that the DJF
and JJA $\npcOdX$ abundances are similar.
Second, there is too much structure in geographic representation of
the DJF and JJA absorption fields (Figures~7a and~7c) to expect a
reader to compute the zonal averages of each in his head and then 
to mentally intercompare the results.

\end{enumerate}
% 2
\item p.~23: The revised manuscript includes the polar summertime 
tropopause and troposphere temperature biases of the CCM on p.~24.
% 3
\item p.~21: Agreed.
The revised manuscript reads ``Radiative forcing in other
non-overlapped, optically thin bands \ldots''.
% 4
\item p.~24: I agree with the Reviewer's point and that is why the
original statement contained the proviso ``by themselves''---because 
in order for gaseous absorption to explain a clear sky diffuse
radiation bias one needs to invoke the additional argument that the
aerosol optical depth has been systematically mis-measured in a number
of standard aerosol channels. 
Perhaps the original manuscript is too vague, but developing this
point any further would require a lengthy digression into the
literature of aerosol processes, Gas ``X'', and relevant results from
field experiments which are beyond the scope of this study.
As stated, the manuscript simply illustrates the neutral and
non-controversial point that gaseous absorption alone cannot remedy a
clear sky diffuse radiation bias. 
The revised manuscript reads ``\ldots by themselves (i.e., without
invoking aerosol processes) \ldots'', which is more specific but does
not require additional off-topic speculation.
% 5
\item Figure~1: Agreed. 
The lines in Figure~1 are now thicker, and I changed one of the blues
to a yellow to make it easier to differentiate \Od\ from \OdNd.
% 6
\item Figure~2: Agreed.
The caption of Figure~2 now ends with ``The line spectra of \HdO,
\COd, and \Od\ are averaged over 10~\xcm.'' 
The lines in Figures~2 are now thicker---using different line patterns
did not help.
Putting the lines on a log scale de-emphasized the \OdX\ features
too much.
The purpose of the picture is to show the spectral regions which
\OdX\ absorption occurs in spectral windows, and which is overlapped
with other absorption.
The spectral region near 1.27~\um\ is especially hard to read because
there are five gaseous absorption processes to depict.
I have changed the caption to tell the reader that the continua are
all shown in Figure~1, and that line absorption from \HdO, \Od, and
\COd\ has been added.
Figure~1 and the text explain the important features to the
interested reader.
% 7
\item Figure~3: Done.
\end{enumerate}

% Bibliography
%\renewcommand\refname{\normalsize References}
%\bibliographystyle{agu}
%\bibliography{bib}

\end{document}



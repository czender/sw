% $Id$ -*-LaTeX-*-

% Purpose: Theory and observations of particle size distributions (PSDs)

% Copyright (c) 1998--2018, Charles S. Zender
% Permission is granted to copy, distribute and/or modify this document
% under the terms of the GNU Free Documentation License (GFDL), Version 1.3
% or any later version published by the Free Software Foundation;
% with no Invariant Sections, no Front-Cover Texts, and no Back-Cover Texts.
% GFDL: http://www.gnu.org/copyleft/fdl.html

% The original author of this software, Charlie Zender, seeks to improve
% it with your suggestions, contributions, bug-reports, and patches.
% Charlie Zender <zender at uci dot edu>
% Department of Earth System Science
% University of California, Irvine
% Irvine, CA 92697-3100

% URL: http://dust.ess.uci.edu/facts/psd/psd.pdf

% Usage (see also end of file):
% cd ~/sw/crr;make -W psd.tex psd.pdf;cd -
% scp -p ${DATA}/ps/psd.pdf dust.ess.uci.edu:Sites/facts/psd

\documentclass[12pt,twoside]{article}

% Standard packages
\usepackage{ifpdf} % Define \ifpdf
\ifpdf % PDFLaTeX
\usepackage{graphicx} % Defines \includegraphics*
\usepackage{thumbpdf} % Generate thumbnails
\usepackage{epstopdf} % Convert .eps, if found, to .pdf when required
\else % !PDFLaTeX
\usepackage{graphicx} % Defines \includegraphics*
\fi % !PDFLaTeX
\usepackage{amsmath} % \subequations, \eqref, \align
\usepackage{array} % Table and array extensions, e.g., column formatting
\usepackage[dayofweek]{datetime} % \xxivtime, \ordinal
\usepackage{etoolbox} % \newbool, \setbool, \ifxxx
\usepackage{longtable} % Multi-page tables, e.g., acronyms and symbols
\usepackage{lscape} % Landscape environment
\usepackage{makeidx} % Index keyword processor: \printindex and \see
\usepackage{natbib} % \cite commands from aguplus
\usepackage{times} % Postscript Times-Roman font KoD99 p. 375
\usepackage{tocbibind} % Add Bibliography and Index to Table of Contents
\usepackage{url} % Typeset URLs and e-mail addresses
% fxm: 20001028 /usr/share/texmf/tex/latex/base/showidx.sty breaks hyperref
%\usepackage{showidx} % Print index entries as marginal notes

% hyperref is last package since it redefines other packages' commands
% hyperref options, assumed true unless =false is specified:
% backref       List citing sections after bibliography entries
% baseurl       Make all URLs in document relative to this
% bookmarksopen Unknown
% breaklinks    Wrap links onto newlines
% colorlinks    Use colored text for links, not boxes
% hyperindex    Link index to text
% plainpages=false Suppress warnings caused by duplicate page numbers
% pdftex        Conform to pdftex conventions
% Colors used when colorlinks=true:
% linkcolor     Color for normal internal links
% anchorcolor   Color for anchor text
% citecolor     Color for bibliographic citations in text
% filecolor     Color for URLs which open local files
% menucolor     Color for Acrobat menu items
% pagecolor     Color for links to other pages
% urlcolor      Color for linked URLs
\ifpdf % PDFLaTeX
\usepackage[backref,breaklinks,colorlinks,citecolor=blue,linkcolor=blue,urlcolor=blue,hyperindex,plainpages=false]{hyperref} % Hyper-references
%\pdfcompresslevel=9
\else % !PDFLaTeX
\usepackage[backref,breaklinks,colorlinks=false,hyperindex,plainpages=false]{hyperref} % Hyper-references
\fi % !PDFLaTeX

% preview-latex recommends it be last-activated package
\usepackage[showlabels,sections,floats,textmath,displaymath]{preview} % preview-latex equation extraction

\usepackage{csz} % Library of personal definitions
\usepackage{dmn} % Dimensional units
\usepackage{chm} % Commands generic to chemistry
\usepackage{dyn} % Commands generic to fluid dynamics
\usepackage{abc} % Alphabet as three letter macros
\usepackage{aer} % Commands specific to aerosol physics
\usepackage{psd} % Particle size distributions
\usepackage{jrn_agu} % JGR-sanctioned journal abbreviations

% Commands which must be executed in preamble
\makeglossary % Glossary described on KoD95 p. 221
\makeindex % Index described on KoD95 p. 220

% Commands specific to this file

% 1. Primary commands

% 2. Derived commands
\newcommand{\expgsd}{\me^{\gsd}} 
\newcommand{\gggavg}{\overline{\ggg}}
\newcommand{\pdfdmt}{\pdffnc(\dmt)} % Probability density function of diameter
\newcommand{\pdfrds}{\pdffnc(\rds)}
\newcommand{\rdsdst}{\dstfnc(\rds)}
\newcommand{\sdndmt}{\sdn_{\dmt}} % [m] Standard deviation of diameter
\newcommand{\vrndmt}{\sdn_{\dmt}^{2}} % [m2] Variance of diameter
\newcommand{\vrnrds}{\sdn_{\rds}^{2}} % [m2] Variance of radius
\newcommand{\vrnffcxxx}{\sdn_{\xxx,\mathrm{eff}}^{2}}
\newcommand{\vrnxxx}{\sdn_{\xxx}^{2}}
\newcommand{\xxxeff}{\xxx_\mathrm{eff}}

% 3. Doubly-derived commands

\topmargin -24pt \headheight 12pt \headsep 12pt
\textheight 9in \textwidth 6.5in
\oddsidemargin 0in \evensidemargin 0in
%\marginparwidth 0pt \marginparsep 0pt
\setlength{\marginparwidth}{1.5in} % Width of callouts of index terms and page numbers KoD95 p. 220
\setlength{\marginparsep}{12pt} % Add separation for index terms KoD95 p. 220
\footskip 24pt
\footnotesep=0pt

\begin{document} % End preamble

\begin{center}
Online: \url{http://dust.ess.uci.edu/facts} \hfill Built: \shortdate\today, \xxivtime\\
\bigskip
{\Large \textbf{Particle Size Distributions:\\ 
Theory and Application to Aerosols, Clouds, and Soils}}\\
\bigskip
by Charlie Zender\\
University of California, Irvine\\
\end{center}
Department of Earth System Science \hfill zender@uci.edu\\
University of California \hfill Voice: (949)\thinspace 891-2429\\
Irvine, CA~~92697-3100 \hfill Fax: (949)\thinspace 824-3256

% GFDL legalities: http://www.gnu.org/copyleft/fdl.html
\bigskip\noindent
Copyright \copyright\ 1998--2018,  Charles S. Zender\\
Permission is granted to copy, distribute and/or modify this document
under the terms of the GNU Free Documentation License, Version~1.3
or any later version published by the Free Software Foundation;
with no Invariant Sections, no Front-Cover Texts, and no Back-Cover
Texts.
The license is available online at
\url{http://www.gnu.org/copyleft/fdl.html}.

\pagenumbering{roman}
\setcounter{page}{1}
\pagestyle{headings}
\thispagestyle{empty}
%\onecolumn
\tableofcontents
\listoftables
\pagenumbering{arabic}
\setcounter{page}{1}
%\markleft{Size Distribution Monograph}
%\markright{}
\thispagestyle{empty}

\section{Introduction}\label{sxn:ntr}

This document describes mathematical and computational considerations 
pertaining to size distributions.
The application of statistical theory to define meaningful and
measurable parameters for defining generic size distributions is
presented in \S\ref{sxn:stt}.
The remaining sections apply these definitions to the size
distributions most commonly used to describe clouds and aerosol size
distributions in the meteorological literature.
Currently, only the lognormal distribution is presented.

\subsection[Modal vs.\ Sectional Represenatation]{Modal vs.\ Sectional Represenatation}\trmidx{mdlsxn}
\cite{LuB04} designed and optimal non-linear least squares-based
procedure for converting from sectional to modal representations.

\subsection[Nomenclature]{Nomenclature}\trmidx{nomenclature}
There is a bewildering variety of nomenclature associated with size
distributions, probability density functions, and statistics thereof.
The nomenclature in this article generally follows the standard
references, \cite[see, e.g.,][]{HaT74,PaG77,PFT88,FTV89,SeP97}, at
least where those references are in agreement. 
Quantities whose nomenclature is often confusing, unclear, or simply
not standardized are discussed in the text.

\subsection[Distribution Function]{Distribution Function}\index{distribution function}
This section follows the carefully presented discussion of
\cite{FTV89}.
The \trmdfn{size distribution} function $\dstnbrofrds$ is defined such that
$\dstnbrofrds\,\dfr\rds$ is the total concentration (number per unit volume
of air, or \nbrxmC) of particles with sizes in the domain
$[\rds,\rds+\dfr\rds]$. 
The total number concentration of particles $\cncttl$ is obtained by
integrating $\dstnbrofrds$ over all sizes
\begin{equation}
\cncttl = \int_{0}^{\infty} \dstnbrofrds \,\dfr\rds
\label{eqn:cnc_ttl_dfn}
\end{equation}
The size distribution function is also called the \trmdfn{spectral
density function}.
The dimensions of $\dstnbrofrds$ and $\cncttl$ are \nbrxmCm\ and
\nbrxmC, respectively.
Note that $\dstnbrofrds$ is only normalized if $\cncttl = 1.0$ 
(cf.\ Section~\ref{sxn:nrm}). 

Often $\cncttl$ is not an observable quantity.
A variety of functional forms, some of which are overloaded for
clarity, describe the number concentrations actually measured by
instruments. 
Typically an instrument has a lower detection limit $\rdsmin$ and an
upper detection limit $\rdsmax$ of particle sizes which it can
measure.  
\begin{eqnarray}
\cncfnc(\rds < \rdsmax) & = & \int_{0}^{\rdsmax} \dstnbrofrds \,\dfr\rds
\label{eqn:cnc_cml} \\ 
\cncfnc(\rds > \rdsmax) & = & \int_{\rdsmax}^{\infty} \dstnbrofrds \,\dfr\rds
\label{eqn:cnc_gt} \\ 
\cncfnc(\rdsmin, \rdsmax) & = & \cncfnc(\rdsmin < \rds < \rdsmax) = \int_{\rdsmin}^{\rdsmax} \dstnbrofrds \,\dfr\rds
\label{eqn:cnc_ntv}
\end{eqnarray}
Equations (\ref{eqn:cnc_cml})--(\ref{eqn:cnc_ntv}) define the
\trmdfn{cumulative concentration}, \trmdfn{lower bound concentration},
and \trmdfn{truncated concentration}, respectively.
The cumulative concentration is used to define the 
\trmdfn{median radius}~$\rdsnma$.
Half the particles are larger and half smaller than $\rdsnma$
\begin{equation}
\cncfnc(\rds < \rdsnma) = \cncfnc(\rds > \rdsnma) = \frac{\cncttl}{2}
\label{eqn:mdn_dfn}
\end{equation}
These functions are often used to define $\dstnbrofrds$ via 
\begin{equation}
\dstnbrofrds = \frac{\dfr\cnc}{\dfr\rds}
\label{eqn:dst_dfn}
\end{equation}
Note that the concentration nomenclature in (\ref{eqn:dst_dfn}) is
$\cnc$ not $\cncofrds$.
Using $\cncofrds$ would indicate that the concentration has not been
completely integrated over all sizes. 
By definition, the total concentration $\cncttl$ is integrated over
all sizes, as defined by (\ref{eqn:cnc_ttl_dfn}).
A concentration denoted $\cncofrds$ makes no sense without an
associated size bin width $\Delta \rds$, or truncation convention,
as in (\ref{eqn:cnc_cml})--(\ref{eqn:cnc_ntv}).
We try to use $\cnc$ and $\cncttl$ for normalized ($\cnc = 1$) and
non-normalized ($\cncttl \ne 1$, i.e., absolute concentrations).
However this convention is not absolute and (\ref{eqn:cnc_ttl_dfn})
defines both $\cnc$ and~$\cncttl$.

\subsection[Probability Density Function]{Probability Density Function}
Describing size distributions is easier when they are
normalized into \trmdfn{probability density functions}, or \trmidx{PDF}s.
In this context, a PDF is a size distribution function normalized to
unity over the domain of interest, i.e.,  $\pdfrds = \cstnrm \dstnbrofrds$
where the normalization constant $\cstnrm$ is defined such that
\begin{equation}
\int_{0}^{\infty} \pdfrds \,\dfr\rds = 1
\label{eqn:pdf_nrm}
\end{equation}
In the following sections we usually work with PDFs because this
normalization property is very convenient mathematically.
Comparing (\ref{eqn:pdf_nrm}) and (\ref{eqn:cnc_ttl_dfn}), it is clear
that the normalization constant $\cstnrm$ which transforms a size
distribution  function (\ref{eqn:cnc_ttl_dfn}) into a PDF $\pdfrds$ is
$\cncttl^{-1}$  
\begin{equation}
\pdfrds = \frac{1}{\cncttl} \, \dstnbrofrds
\label{eqn:pdf_dfn}
\end{equation}

\subsubsection[Independent Variable]{Choice of Independent Variable}
The merits of using radius $\rds$, diameter $\dmt$, or some other
dimension $\LLL$, as the \trmidx{independent variable} of a size
distribution depend on the application. 
In \trmidx{radiative transfer} applications, $\rds$ prevails in the
literature probably because it is favored in electromagnetic and 
\trmidx{Mie theory}. 
There is, however, a growing recognition of the importance of
aspherical particles in planetary atmospheres. 
Defining an \trmdfn{equivalent radius} or \trmdfn{equivalent diameter}
for these complex shapes is not straighforward (consider, e.g., a
bullet rosette ice crystal).
Important differences exist among the competing definitions, such as
\trmdfn{equivalent area spherical radius}, \trmdfn{equivalent volume
spherical radius}, \cite[e.g.,][]{EbC92,McH972}. 

A direct property of aspherical particles which can often be measured
is its maximum dimension, i.e., the greatest distance between any two
surface points of the particle.
This maximum dimension, usually called $\LLL$, has proven to be
useful for characterizing size distributions of 
\trmidx{aspherical particles}. 
For a sphere, $\LLL$ is also the diameter.
Analyses of \trmidx{mineral dust} sediments in ice core deposits or sediment
traps, for example, are usually presented in terms of $\LLL$.
The surface area and volume of ice crystals have been computed in terms
of power laws of $\LLL$ \cite[e.g.,][]{HeP84,TaL95}.
Since models usually lack information regarding the shape of
particles \cite[early exceptions include][]{Zek941,ChL941}, most modelers
assume \trmidx{spherical particles}, especially for aerosols.
Thus, the advantages of using the diameter $\dmt$ as the independent 
variable in size distribution studies include: $\dmt$ is the
dimension often reported in measurements; $\dmt$ is more analogous
than $\rds$ to $\LLL$.

The remainder of this manuscript assumes spherical particles where
$\rds$ and $\dmt$ are equally useful independent variables.
Unless explicitly noted, our \trmidx{convention} will be to use $\dmt$
as the independent variable.
Thus, it is useful to understand the rules governing conversion of PDFs
from $\dmt$ to $\rds$ and the reverse.

Consider two distinct analytic representations of the same underlying
size distribution.
The first, $\dstnbrdmtofdmt$, expresses the \trmidx{differential number
concentration} per unit diameter.
The second, $\dstnbrrdsofrds$, expresses the differential number
concentration per unit radius.
Both $\dstnbrdmtofdmt$ and $\dstnbrrdsofrds$ share the same dimensions, 
\nbrxmCm. 
\begin{eqnarray}
\dmt & = & 2\rds \\
\dfr\dmt & = & 2 \,\dfr\rds \\
\dstnbrdmtofdmt \,\dfr\dmt & = & \dstnbrrdsofrds \,\dfr\rds \\
\dstnbrdmtofdmt & = & \frac{1}{2} \dstnbrrdsofrds
\label{eqn:dmt_rds}
\end{eqnarray}

\section{Statistics of Size Distributions}\label{sxn:stt}

\subsection[Generic]{Generic}
Consider an arbitrary function $\ggg(\xxx)$ which applies over the
domain of the size distribution $\pdffnc(\xxx)$.
For now the exact definition of $\ggg$ is irrelevant, but imagine that
$\ggg(\xxx)$ describes the variation of some physically meaningful
quantity (e.g., area) with size.
The \trmdfn{mean value} of $\ggg$ is the integral of $\ggg$ over the
domain of the size distribution, weighted at each point by the
concentration of particles
\begin{equation}
\gggavg = \int_{0}^{\infty} \ggg(\xxx) \, \pdffnc(\xxx) \,\dfr\xxx
\label{eqn:stt_gnr}
\end{equation}
Once $\pdffnc(\xxx)$ is known, it is always possible to compute
$\gggavg$ for any desired quantity $\ggg$.
Typical quantities represented by $\ggg(\xxx)$ are size,
$\ggg(\xxx) = \xxx$; area, $\ggg(\xxx) = A(\xxx) \propto \xxx^{2}$; and
volume $\ggg(\xxx) = V(\xxx) \propto \xxx^{3}$.
More complicated statistics represented by $\ggg(\xxx)$ include
variance, $\ggg(\xxx) = (\xxx - \xxxavg)^{2}$. 
The remainder of this section considers some of these examples in more
detail.

\subsection[Mean Size]{Mean Size}
The \trmdfn{number mean size} $\xxxavg$ of a size distribution
$\pdffnc(\xxx)$ is defined as 
\begin{equation}
\xxxavg = \int_{0}^{\infty} \pdffnc(\xxx) \, \xxx \,\dfr\xxx
\label{eqn:szavg_gnr}
\end{equation}
Synonyms for number mean size include \trmdfn{mean size},
\trmdfn{average size}, \trmdfn{arithmetic mean size}, and
\trmdfn{number-weighted mean size} \cite[][]{HaT74}.  
\cite{FTV89} define $\dmtnaa \equiv \dmtavg$, a convention we adopt
in the following.

\subsection[Variance]{Variance}\label{sxn:vrn}
The \trmdfn{variance} $\vrnxxx$ of a size distribution $\pdffnc(\xxx)$
is defined in accord with the statistical variance of a continuous
mathematical distribution.
\begin{equation}
\vrnxxx = 
\int_{0}^{\infty} \pdffnc(\xxx) (\xxx - \xxxavg)^{2} \,\dfr\xxx
\label{eqn:vrn_gnr}
\end{equation}
The variance measures the mean squared-deviation of the distribution
from its mean value.
The units of $\vrnxxx$ are [\mS].
Because $\vrnxxx$ is a complicated function for standard aerosol and
cloud size distributions, many prefer to work with an alternate
definition of variance, called the \trmdfn{effective variance}. 

The \trmdfn{effective variance} $\vrnffcxxx$ of a size distribution 
$\pdffnc(\xxx)$ is the variance about the \trmidx{effective size} of
the distribution, normalized by $\xxxeff$ \cite[e.g.,][]{HaT74}  
\begin{equation}
\vrnffcxxx = \frac{1}{\xxxeff^{2}} 
\int_{0}^{\infty} \pdffnc(\xxx) (\xxx - \xxxeff)^{2} \, \xxx^{2} \,\dfr\xxx
\label{eqn:vrnffc_gnr}
\end{equation}
Because of the $\xxxeff^{-2}$ normalization, $\vrnffcxxx$ is
non-dimensional in contrast to typical variances, e.g.,
(\ref{eqn:vrn_gnr}). 
In the terminology of \cite{HaT74}, $\vrnffcxxx = v$.

\subsection[Standard Deviation]{Standard Deviation}
The \trmdfn{standard deviation} $\sdnxxx$ of a size distribution
$\pdffnc(\xxx)$ is the square root of the variance (\ref{eqn:vrn_gnr}),  
\begin{equation}
\sdnxxx = \sqrt{\vrnxxx}
\label{eqn:sdn_gnr}
\end{equation}
$\sdnxxx$ has units of~[\m].
For standard aerosol and cloud size distributions, $\sdnxxx$ is an
ugly expression.
Therefore many authors prefer to work with alternate definitions of
standard deviation.
Unfortunately, nomenclature for these alternate definitions is not
standardized. 

\section{Cloud and Aerosol Size Distributions}\label{sxn:psd}

\subsection{\normalsize Gamma Distribution}\label{sxn:psd_gmm}
Statistics of the \trmdfn{gamma distribution} are presented in
\url{http://asd-www.larc.nasa.gov/~yhu/paper/thesisall/node8.html}.
Currently, the aerosol property program \cmdidx{mie} implements gamma
distributions in a limited sense.

\subsection{\normalsize Normal Distribution}\label{sxn:psd_nrm}
The \trmdfn{normal distribution} is the most common statistical
distribution. 
The normal distribution $\dst(\xxx)$ is expressed in terms of
its mean $\xxxavg$ (\ref{eqn:szavg_gnr}) and standard deviation
$\sdnxxx$ (\ref{eqn:sdn_gnr}) 
\begin{equation}
\dst(\xxx) = 
\frac{1}{\sqrt{2\mpi} \, \sdnxxx} \exp\! 
\left[-\frac{1}{2}\left( \frac{\xxx - \xxxavg}{\sdnxxx} \right)^{2} \right] 
\label{eqn:dst_nrm}
\end{equation}
With our standard nomenclature for number distribution $\dstnbr$ and
particles diameter~$\dmt$, (\ref{eqn:dst_nrm}) appears as
\begin{equation}
\dstnbr(\dmt) \equiv 
\frac{\dfr\cnc}{\dfr\dmt} = 
\frac{1}{\sqrt{2\mpi} \, \sdndmt} \exp\! 
\left[-\frac{1}{2}\left( \frac{\dmt - \dmtnaa}{\sdndmt} \right)^{2} \right] 
\label{eqn:dst_nrm_dmt}
\end{equation}

The cumulative normal distribution is called the \trmidx{error function} 
and is discussed in Section~(\ref{sxn:gss}). 
Integration of the error function shows that 
68.3\% of the values of (\ref{eqn:dst_nrm_dmt}) are in $\dmtnaa \pm \sdndmt$, 
95.4\% are in $\dmtnaa \pm 2\sdndmt$, and
99.7\% are in $\dmtnaa \pm 3\sdndmt$.

\subsection{\normalsize Lognormal Distribution}\label{sxn:psd_lgn}
The \trmdfn{lognormal distribution} is perhaps the most commonly used
analytic expression in aerosol studies.
\subsubsection[Distribution Function]{Distribution Function}
In a lognormal distribution, the logarithm of abscissa is normally
distributed (Section~\ref{sxn:psd_nrm}).
Substituting $\xxx = \ln\dmt$ into (\ref{eqn:dst_nrm}) yields
\begin{equation}
\dstnbr(\ln\dmt) \equiv 
\frac{\dfr\cnc}{\dfr\ln\dmt} = 
\frac{1}{\sqrt{2\mpi} \, \lngsd} \exp\! 
\left[-\frac{1}{2}\left( \frac{\ln\dmt - \ln\dmtnma}{\lngsd} \right)^{2} \right] 
\label{eqn:dst_lgn_ntm}
\end{equation}
where $\gsd$ and $\dmtnma$ are parameters whose physical
significance is to be defined.
In particular, there is no closed-form algebraic relationship between
$\sdndmt$ (\ref{eqn:dst_nrm_dmt}) and $\gsd$ (\ref{eqn:dst_lgn_ntm}).
The former is a true standard deviation and the properties of the
latter are as yet unknown.

Substituting $\dfr\ln\dmt = \dmt^{-1}\,\dfr\dmt$ in
(\ref{eqn:dst_lgn_ntm}) leads to the most commonly used form the 
\trmdfn{lognormal distribution function} 
\begin{equation}
\dstnbr(\dmt) \equiv 
\frac{\dfr\cnc}{\dfr\dmt} = 
\frac{1}{\sqrt{2\mpi} \, \dmt \lngsd} \exp\! 
\left[-\frac{1}{2}\left( \frac{ \ln(\dmt/\dmtnma)}{\lngsd} \right)^{2} \right] 
\label{eqn:dst_lgn}
\end{equation}

One of the most confusing aspects of size distributions in the
meteorological literature is in the usage of $\gsd$, the 
\trmdfn{geometric standard deviation}.
Some researchers \cite[e.g.,][]{FTV89} prefer a different formulation
(\ref{eqn:dst_lgn}) which is equivalent to 
\begin{equation}
\dstnbrofdmt = \frac{1}{\sqrt{2\mpi} \, \gsdtld \dmt} \exp\! 
\left[-\frac{1}{2}\left( \frac{ \ln(\dmt/\dmtnma)}{\gsdtld} \right)^{2} \right]
\label{eqn:dst_lgn_alt}
\end{equation}
where
\begin{equation}
\gsdtld \equiv \lngsd
\label{eqn:gsd_tld_dfn}
\end{equation}
In practice, (\ref{eqn:dst_lgn}) is used more widely than
(\ref{eqn:dst_lgn_alt}) and we adopt (\ref{eqn:dst_lgn}) in the 
following.

The definition of $\gsdtld$ in (\ref{eqn:dst_lgn_alt}) may be more
satisfactory from a mathematical point of view \cite[]{FTV89}, and it
subsumes an ``$\ln$'' which reduces typing.  
This is seen by transforming $\xxx$ in (\ref{eqn:dst_nrm}) with
\begin{eqnarray}
% FTV89 p. 17
\xxx & = & \frac{1}{\gsdtld} \ln \left( \frac{\dmt}{\dmtnma} \right) \nonumber \\
\dmt & = & \dmtnma \exp (\gsdtld\xxx) \nonumber \\
\dfr\dmt & = & \gsdtld \dmtnma \exp (\gsdtld\xxx) \,\dfr\xxx \nonumber \\
\dfr\xxx & = & (\gsdtld \dmtnma)^{-1} \,\dfr\dmt \nonumber
\label{eqn:cov_xxx}
\end{eqnarray}
This maps $\xxx \in (-\infty, +\infty)$ into $\dmt \in [0,\infty)$.

One is occasionally given a ``standard deviation'' or ``geometric
standard deviation'' parameter without clear specification whether it
represents $\gsd$ (or $\lngsd$, or $\exp \gsd$, or $\sdnxxx$) in
(\ref{eqn:sdn_gnr}), (\ref{eqn:dst_lgn}), or (\ref{eqn:dst_lgn_alt}).
As a true standard deviation, $\sdnxxx$ has dimensions of $\xxx$,
whereas both $\gsd$ and $\gsdtld$ are dimensionless so units cannot 
disambiguate them.
A useful rule of thumb is that $\gsd$ in (\ref{eqn:dst_lgn}) and 
$\me^{\gsdtld}$ in (\ref{eqn:dst_lgn_alt}) are usually between
1.5--2.5 for realistic aerosol populations.   
Since we adopted (\ref{eqn:dst_lgn}), physically realistic values
are $\gsd \in (1.5, 2.5)$.

\cite{SeP97} p.~423 describe the physical meaning of the geometric
standard deviation~$\gsd$.
Define the special particle sizes
\begin{subequations} % Equation~(\ref{eqn:dst_gsd})
\label{eqn:dst_gsd}
\begin{align} 
\dmtgsdpls &\equiv \dmtnma \gsd \\
\dmtgsdmns &\equiv \dmtnma / \gsd
\end{align} 
\end{subequations} % end Equation~(\ref{eqn:dst_gsd})
The \trmidx{cumulative concentration} smaller than $\dmtgsdpls$,
simplifies from (\ref{eqn:lgn_erf}) to 
\begin{eqnarray}
% http://mathworld.wolfram.com/Erf.html
% ncap2 -O -v -s 'gsdnbr=1.0' -s 'cncfncmns=0.5*(1.0+erf(-gsdnbr/sqrt(2.0)))' -s 'cncfncpls=0.5*(1.0+erf(gsdnbr/sqrt(2.0)))' -s 'cncfncspread=cncfncpls-cncfncmns' ~/nco/data/in.nc ~/foo.nc;ncks ~/foo.nc
\cncfnc(\dmt < \dmtgsdpls)
& = & \frac{\cncttl}{2} + 
\frac{\cncttl}{2} \, \erffnc\! \left( \frac{1}{\sqrt{2}} \right)
= 0.841344746069 \cncttl
% fxm: is erf(1/sqrt(2)) analytic?
\label{eqn:cnc_bnd_gsd_pls}
\end{eqnarray}
Numerical integration must be used to obtain the final result, 
$0.841 \cncttl$, as $\erffnc$() has no closed-form solution here.
Using (\ref{eqn:cnc_bnd_gsd_pls}) to invert (\ref{eqn:dst_gsd}), 
we may define $\gsd$ as the ratio of the diameter~$\dmtgsdpls$ (larger
than 84.1\% of all particles) to the median diameter~$\dmtnma$.
Monodisperse populations have $\gsd \equiv 1$.

Similarly the cumulative concentration smaller than $\dmtgsdmns$,
simplifies from (\ref{eqn:lgn_erf}) to 
\begin{eqnarray}
% http://mathworld.wolfram.com/Erf.html
\cncfnc(\dmt < \dmtgsdmns)
& = & \frac{\cncttl}{2} + 
\frac{\cncttl}{2} \, \erffnc\! \left(- \frac{1}{\sqrt{2}} \right)
= 0.158655253931 \cncttl
\label{eqn:cnc_bnd_gsd_mns}
\end{eqnarray}
where we have used the numerical result in (\ref{eqn:cnc_bnd_gsd_mns})
with the error function's anti-symmetric property,
$\erffnc(-\xxx) = -\erffnc(\xxx)$.
Subtracting (\ref{eqn:cnc_bnd_gsd_mns}) from (\ref{eqn:cnc_bnd_gsd_pls})
shows that 68.3\% of all particles in a lognormal distribution lie in 
$\dmt \in [\dmtgsdmns,\dmtgsdpls]$. 

By raising $\gsd$ to any power $\xxx$ in (\ref{eqn:dst_gsd}), it is 
straightforward to verify that the number of particles within
$\dmt \in [\dmtnma\gsd^{-\xxx}, \dmtnma\gsd^{\xxx}]$ is
\begin{eqnarray}
% http://mathworld.wolfram.com/Erf.html
\cncfnc(\dmtnma\gsd^{-\xxx} < \dmt < \dmtnma\gsd^{\xxx}) & = & 
\cncttl \,\erffnc(\xxx/\sqrt{2})
\label{eqn:cnc_bnd_gsd_xxx}
\end{eqnarray}
Application of (\ref{eqn:cnc_bnd_gsd_xxx}) for small integer $\xxx$
shows 
that 68.3\% of all particles lie within 
$\dmtnma/\gsd < \dmt < \dmtnma\gsd$, 
that 95.4\% of all particles\footnote{%
\cite{SeP97} p.~423 has a typo on this point. 
That page erroneously states that the bounds bracketing 95\% of a
lognormal distribution are $\dmtnma/(2\gsd) < \dmt < 2\dmtnma\gsd$.}
lie within $\dmtnma\gsd^{-2} < \dmt < \dmtnma\gsd^{2}$, and 
that 99.7\% of all particles lie within 
$\dmtnma\gsd^{-3} < \dmt < \dmtnma\gsd^{3}$.
These bounds are identical to the fraction of particles enclosed
within one, two, and three standard deviations of the mean of a normal  
distribution (Section~\ref{sxn:psd_nrm}). 

\subsubsection[Lognormal Relations]{Lognormal Relations}
Table~\ref{tbl:lgn} summarizes the standard lognormal distribution
parameters. 
Note that $\gsdtld \equiv \ln \gsd$.
The statistics in Table~\ref{tbl:lgn} are easy to misunderstand
because of the plethora of subtly different definitions.
A common mistake is to assume that patterns which seems to apply to
one distribution, e.g., the number distribution $\dstnbrofdmt$, apply
to distributions of all other moments. 
For example, the number distribution $\dstnbrofdmt$ is the
\textit{only} distribution for which the moment mean size (i.e.,
number mean size~$\dmtnaa$) equals the moment-weighted size (i.e.,
number-weighted size~$\dmtnwa$).  
Also, the number mean size $\dmtnaa$ differs from the number median
size $\dmtnma$ by a factor $\exp(\gsdtldsqr/2)$.
But this factor is not constant and depends on the moment of the
distribution.
For instance, $\dmtsaa$ differs from $\dmtsma$ by $\exp(\gsdtldsqr)$,
while $\dmtsaa$ differs from $\dmtsma$ by $\exp(3\gsdtldsqr/2)$.
Thus converting from mean diameter to median diameter is not the
same for number as for mass distributions.
% Begin tbl:lgn
\begin{landscape}
\begin{longtable}{ >{\raggedright}p{3em}<{} >{$}l<{$} l >{\raggedright}p{14em}<{} >{$\displaystyle}r<{$} }
& & & & \kill % NB: longtable requires caption as table entry
\caption[Lognormal Distribution Relations]{\textbf{Lognormal Distribution
Relations}%
\footnote{\emph{Source:} }%  
% fxm: 20010114 latex2html appears to choke when math mode is used in footnotes
\footnote{For convenience, we have used $\gsdtld \equiv \lngsd$.}%  
\footnote{The total diameter $\dmtttl$ is intended as an example of a
statistic which is first order in $\dmt$.}%
\label{tbl:lgn}} \\
\hline\hline \rule{0.0ex}{\hlntblhdrskp}% 
Symbol & \mbox{Value} & Units & Description & \mbox{Defining Relation} \\[0.0ex]
\hline \rule{0.0ex}{\hlntblntrskp}%
\endfirsthead % Lines between and \endfirsthead appear at top of table
\caption[]{(continued)} \\ % Set label for following pages
Symbol & \mbox{Value} & Units & Description & \mbox{Defining Relation} \\[0.0ex]
\hline \rule{0.0ex}{\hlntblntrskp}%
\endhead % Previous block appears at top of every page
\endlastfoot % Previous block appears at end of table
$\cncttl$ & \cncttl & \nbrxmC & Total number concentration &
\cncttl = \int_{0}^{\infty} \dstnbrofdmt \,\dfr\dmt \\[1.25ex]
$\dmtttl$ &  \cncttl \dmtnma \exp(\gsdtldsqr) & \mxmC & Total diameter &
\dmtttl = \int_{0}^{\infty} \dmt \dstnbrofdmt \,\dfr\dmt \\[1.25ex]
$\xsattl$ & \frac{\mpi}{4} \cncttl \dmtnma^{2} \exp(\gsdtldsqr/2) & \mSxmC & Total cross-sectional area &
\xsattl = \int_{0}^{\infty} \frac{\mpi}{4} \dmt^{2} \dstnbrofdmt \,\dfr\dmt \\[1.25ex]
$\sfcttl$ & \mpi \cncttl \dmtnma^{2} \exp(2\gsdtldsqr) & \mSxmC & Total surface area &
\sfcttl = \int_{0}^{\infty} \mpi \dmt^{2} \dstnbrofdmt \,\dfr\dmt \\[1.25ex]
$\vlmttl$ & \frac{\mpi}{6} \cncttl \dmtnma^{3} \exp(9\gsdtldsqr/2) & \mCxmC & Total volume & 
\vlmttl = \int_{0}^{\infty} \frac{\mpi}{6} \dmt^{3} \dstnbrofdmt \,\dfr\dmt \\[1.25ex]
$\mssttl$ & \frac{\mpi}{6} \cncttl \dns \dmtnma^{3} \exp(9\gsdtldsqr/2) & \kgxmC & Total mass &
\mssttl = \int_{0}^{\infty} \frac{\mpi}{6} \dns \dmt^{3} \dstnbrofdmt \,\dfr\dmt \\[1.25ex]
\hline\hline \rule{0.0ex}{\hlntblhdrskp}% 
$\dmtavg$ & \dmtnma \exp(\gsdtldsqr/2) & \mxnbr & Mean diameter & \cncttl \dmtavg = \cncttl \dmtnaa = \dmtttl \\[0.5ex]
$\xsaavg$ & \frac{\mpi}{4} \dmtnma^{2} \exp(2\gsdtldsqr) & \mSxnbr & Mean
cross-sectional area & \textstyle \cncttl \xsaavg = \cncttl \frac{\mpi}{4} \dmtsaa^{2} = \xsattl \\[0.5ex]
$\sfcavg$ & \mpi \dmtnma^{2} \exp(2\gsdtldsqr) & \mSxnbr & Mean surface area & \cncttl \sfcavg = \cncttl \mpi \dmtsaa^{2} = \sfcttl \\[0.5ex]
$\vlmavg$ & \frac{\mpi}{6} \dmtnma^{3} \exp(9\gsdtldsqr/2) & \mCxnbr & Mean volume & \textstyle \cncttl \vlmavg = \cncttl \frac{\mpi}{6} \dmtvaa^{3} = \vlmttl \\[0.5ex]
$\mssavg$ & \frac{\mpi}{6} \dns \dmtnma^{3} \exp(9\gsdtldsqr/2) & \kgxnbr & Mean mass & \textstyle \cncttl \mssavg = \cncttl \frac{\mpi}{6} \dns \dmtvaa^{3} = \mssttl \\[0.5ex]
\hline\hline \rule{0.0ex}{\hlntblhdrskp}% 
$\cncttl$ & \frac{6}{\mpi \dns} \mssttl \dmtnma^{-3} \exp(-9\gsdtldsqr/2) & \nbrxmC & Number concentration & \cncttl = \int_{0}^{\infty} \dstnbrofdmt \,\dfr\dmt \\[0.5ex]
$\dmtnma$ & \left(\frac{6 \mssttl}{\mpi \cncttl \dns} \right)^{1/3}
\exp(-3\gsdtldsqr/2) & \m & Median diameter & \int_{0}^{\dmtnma} \dstnbrofdmt \,\dfr\dmt = \frac{\cncttl}{2} \\[0.5ex]
$\dmtffc$ & \frac{6\mssttl}{\dns \sfcttl} & \m & Effective diameter & \dmtffc = \frac{1}{\xsattl} \int_{0}^{\infty} \dmt \frac{\mpi}{4} \dmt^{2} \dstnbrofdmt \,\dfr\dmt \\[0.5ex]
$\sfcspc$ & \frac{6}{\dns \dmtffc} & \mSxkg & Specific surface area & \sfcspc = \sfcttl/\mssttl \\[0.5ex]
\hline\hline \rule{0.0ex}{\hlntblhdrskp}% 
$\dmtnma$ & \dmtnaa \exp(-\gsdtldsqr/2) & \m & Median diameter, Scaling diameter, Number median diameter. Half of particles are larger than, and half smaller than, $\dmtnma$ & \int_{0}^{\dmtnma} \dstnbrofdmt \,\dfr\dmt = \frac{\cncttl}{2} \\[0.5ex] 
% SeP97 p. 423 (7.44)
$\dmtnaa$, $\dmtavg$, $\dmtnwa$ & \dmtnma \exp(\gsdtldsqr/2) & \m & Mean diameter, Average diameter, Number-weighted mean diameter & \dmtnaa = \frac{1}{\cncttl} \int_{0}^{\infty} \dmt \dstnbrofdmt \,\dfr\dmt \\[0.5ex]
$\dmtsaa$ & \dmtnma \exp(\gsdtldsqr) & \m & Surface mean diameter & \cncttl \mpi \dmtsaa^{2} = \cncttl \sfcavg = \sfcttl \\[0.5ex]
$\dmtvaa$ & \dmtnma \exp(3\gsdtldsqr/2) & \m & Volume mean diameter, Mass mean diameter & \cncttl \frac{\mpi}{6} \dmtvaa^{3} = \cncttl \vlmavg = \vlmttl \\[0.5ex]
% SeP97 p. 425 (7.50)
$\dmtsma$ & \dmtnma \exp(2\gsdtldsqr) & \m & Surface median diameter & \int_{0}^{\dmtsma} \mpi \dmt^{2} \dstnbrofdmt \,\dfr\dmt = \frac{\sfcttl}{2} \\[0.5ex]
$\dmtswa$, $\dmtffc$ & \dmtnma \exp(5\gsdtldsqr/2) & \m & Area-weighted mean diameter, effective diameter & \dmtswa = \frac{1}{\xsattl} \int_{0}^{\infty} \dmt \frac{\mpi}{4} \dmt^{2} \dstnbrofdmt \,\dfr\dmt \\[0.5ex]
% SeP97 p. 425 (7.52)
$\dmtvma$ & \dmtnma \exp(3\gsdtldsqr) & \m & Volume median diameter \newline Mass median diameter & \int_{0}^{\dmtvma} \frac{\mpi}{6} \dmt^{3} \dstnbrofdmt \,\dfr\dmt = \frac{\vlmttl}{2} \\[0.5ex]
$\dmtvwa$ & \dmtnma \exp(7\gsdtldsqr/2) & \m & Mass-weighted mean diameter, Volume-weighted mean diameter & \dmtvwa = \frac{1}{\vlmttl} \int_{0}^{\infty} \dmt \frac{\mpi}{6} \dmt^{3} \dstnbrofdmt \,\dfr\dmt \\[0.5ex]
\end{longtable}
\end{landscape}
% End tbl:lgn

% fxm: need better segue here
For brevity Table~\ref{tbl:lgn} presents the lognormal relations in
terms of diamter~$\dmt$.
Change the relations to befunctions of radius~$\rds$ is
straightforward. 
For example, direct substitution of $\dmt = 2 \rds$ into
(\ref{eqn:dst_lgn}) yields
\begin{eqnarray}
\dstnbrofdmt & = & \frac{1}{\sqrt{2\mpi} \, 2 \rds \lngsd} \exp\! 
\left[-\frac{1}{2}\left( \frac{ \ln(2 \rds/2 \rdsnma)}{\lngsd}
\right)^{2} \right] \nonumber \\ 
& = & \frac{1}{2} \frac{1}{\sqrt{2\mpi} \, \rds \lngsd} \exp\! 
\left[-\frac{1}{2}\left( \frac{ \ln(\rds/\rdsnma)}{\lngsd}
\right)^{2} \right] \nonumber \\ 
& = & \frac{1}{2} \dstnbrrdsofrds
\label{eqn:dst_lgn_rds}
\end{eqnarray}
in agreement with (\ref{eqn:dmt_rds}).

Table~\ref{tbl:lgn_stt_obs} lists applies the relations in Table~\ref{tbl:lgn}
to specific size distributions typical of tropospheric aerosols.
\begin{longtable}{ >{$}r<{$} >{$}r<{$} >{$}r<{$} >{$}l<{$} r }
& & & & \kill % NB: longtable requires caption as table entry
\caption[Measured Lognormal Dust Size Distributions]{\textbf{Measured Lognormal Dust Size Distributions}% 
\footnote{\emph{References:} 
% fxm: This footnote does not appear in printed matter
\setcounter{enmrfr}{0} % Reset reference counter for this table
\enmrfrstpprn, \citep{PaG77}\label{idx_lgn_stt_obs_PaG77};
\enmrfrstpprn, \citep{She84}\label{idx_lgn_stt_obs_She84};
\enmrfrstpprn, \citep{BSM96}\label{idx_lgn_stt_obs_BSM96};
\enmrfrstpprn, \citep{SBG98}\label{idx_lgn_stt_obs_SBG98};
\enmrfrstpprn, \citep{AGG98}\label{idx_lgn_stt_obs_AGG98};
\enmrfrstpprn, \citep{DHE02}\label{idx_lgn_stt_obs_DHE02};
\enmrfrstpprn, \citep{MSI03}\label{idx_lgn_stt_obs_MSI03};
\enmrfrstpprn, \citep{AKK06}\label{idx_lgn_stt_obs_AKK06};
\enmrfrstpprn, \citep{MGT12}\label{idx_lgn_stt_obs_MGT12};
% fxm uncomment when WGL07 published
%\enmrfrstpprn, \cite{WGL07}\label{idx_lgn_stt_obs_WGL07}.
Values reported in literature were converted to values shown in table
using the analytic expressions summarized in Table~\ref{tbl:lgn}. 
Usually this entailed deriving $\dmtnma$ given $\dmtvma$,
$\rdsvma$, or $\dmtsma$.}% end footnote
\label{tbl:lgn_stt_obs}} \\ % end caption
\hline \rule{0.0ex}{\hlntblhdrskp}% 
\dmtnma & \dmtvma & \gsd & \mssfrc & \setcounter{mltctt}{\value{footnote}} Ref. \\[0.0ex]
\mbox{\um} & \mbox{\um} & & \\[0.0ex]
\hline\hline \rule{0.0ex}{\hlntblntrskp}%
\endfirsthead % Lines between and \endfirsthead appear at top of table
\caption[]{(continued)} \\ % Set label for following pages
\dmtnma & \dmtvma & \gsd & \mssfrc & \setcounter{mltctt}{\value{footnote}} Ref. \\[0.0ex]
\mbox{\um} & \mbox{\um} & & \\[0.0ex]
\hline\hline \rule{0.0ex}{\hlntblntrskp}%
\endhead % Previous block appears at top of every page
\endlastfoot % Previous block appears at end of table
\multicolumn{5}{l}{\citet{PaG77}
\footnote{Detailed fits to dust sampled over Colorado and Texas in \citet{PaG77}, p.~2080 Table~1. 
Original values have been converted from radius to diameter. 
$\mssfrc$ was not given.
\citet{PaG77} showed soil aerosol could be represented with three modes
which they dubbed, in order of increasing size, modes C, A, and~B.
Mode~A is the mineral dust transport mode, seen in source regions and downwind.
Mode~B is seen in the source soil itself, and in the atmosphere during dust events.
Mode~C is seen most everywhere, but does not usually correlate with local dust amount.
Mode~C is usually a global, aged, background, anthropogenic aerosol, typically rich in sulfate and black carbon.
Sometimes, however, Mode~C has a mineral dust component.
Modes~C and~B are averages from \citet{PaG77} Table~1 p.~2080.
Mode~B is based on the summary recommendation that $\rdssma = 1.5$
and $\gsd = 2.2$.}% end footnote
} \\[0.0ex]
0.08169 & 0.27 & 1.88 & & \ref{idx_lgn_stt_obs_PaG77} \\[0.5ex]
0.8674 & 5.6 & 2.2 & & \ref{idx_lgn_stt_obs_PaG77} \\[0.5ex]
28.65 & 57.6 & 1.62 & & \ref{idx_lgn_stt_obs_PaG77} \\[0.5ex]
\multicolumn{5}{l}{\citet{She84}\footnote{Background Desert Model from \citet{She84}, p.~75 Table~1.}} \\[0.0ex]
0.003291 & 0.0111 & 1.89 & 2.6 \times 10^{-4} & \ref{idx_lgn_stt_obs_She84} \\[0.5ex]
0.5972 & 2.524 & 2.0\mbox{\footnote{$\gsd = 2.0$ for transport mode follows \citet{SBG98}, p.~10581, Table~1.}} & 0.781 & \ref{idx_lgn_stt_obs_She84}, \ref{idx_lgn_stt_obs_SBG98} \\[0.5ex]
7.575 & 42.1 & 2.13 & 0.219 & \ref{idx_lgn_stt_obs_She84} \\[0.5ex]
\multicolumn{5}{l}{\citet{BSM96}\footnote{\citet{BSM96}, p.~73 Table~2. 
These are the ``background'' modes of D'Almeida (1987).}} \\[0.0ex]
0.1600 & 0.832 & 2.1 & 0.036 & \ref{idx_lgn_stt_obs_BSM96} \\[0.5ex]
1.401 & 4.82 & 1.90 & 0.957 & \ref{idx_lgn_stt_obs_BSM96} \\[0.5ex]
9.989 & 19.38 & 1.60 & 0.007 & \ref{idx_lgn_stt_obs_BSM96} \\[0.5ex]
\multicolumn{5}{l}{\citet{AGG98}
\footnote{Mass fractions are reported as (\aaa,\bbb) for measurements
 and model, respectively, of Spanish soil sample at 
$\wndfrc = 66$\,\cmxs}% end footnote 
} \\[0.0ex]
0.6445 & 1.5 & 1.7 & (0.22,0.15) & \ref{idx_lgn_stt_obs_AGG98} \\[0.5ex]
3.454 & 6.7 & 1.6 & (0.69,0.76) & \ref{idx_lgn_stt_obs_AGG98} \\[0.5ex]
8.671 & 14.2 & 1.5 & (0.09,0.09) & \ref{idx_lgn_stt_obs_AGG98} \\[0.5ex]
\multicolumn{5}{l}{\citet{DHE02}, Bahrain (1998--2000)
\footnote{All \citet{DHE02} measurements follow certain conventions. 
Standard deviation of measurements follows $\pm$ sign.
Reported $\gsd$ is \textit{not} the geometric standard deviation.
Their $\gsd$ is defined as the standard deviation of the logarithm of
the size distribution  
\cite[][p.~606, Equation~A2]{DHE02}.}% end footnote
\footnote{Bahrain is an island in the Red Sea.}
} \\[0.0ex]
0.1768 & 0.30 \pm 0.08 & 0.42 \pm 0.04 & & \ref{idx_lgn_stt_obs_DHE02} \\[0.5ex]
1.664 & 5.08 \pm 0.08 & 0.61 \pm 0.02 & & \ref{idx_lgn_stt_obs_DHE02} \\[0.5ex]
\multicolumn{5}{l}{\citet{DHE02}, Solar Village Saudi Arabia (1998--2000)
\footnote{The Solar Village AERONET station is located in empty land a few kilometers west of Riyadh's international airport.}
} \\[0.0ex]
0.1485 & 0.24 \pm 0.10 & 0.40 \pm 0.05 & & \ref{idx_lgn_stt_obs_DHE02} \\[0.5ex]
1.576 & 4.64 \pm 0.06 & 0.60 \pm 0.03 & & \ref{idx_lgn_stt_obs_DHE02} \\[0.5ex]
\multicolumn{5}{l}{\citet{DHE02}, Cape Verde (1993--2000)
\footnote{Values of $\dmtnma$ were computed using $\gsd$ based on $\tau = 0.1$.}
} \\[0.0ex]
0.1134 & 0.24 \pm 0.06 & 0.49 + 0.10\tau \pm 0.04 & & \ref{idx_lgn_stt_obs_DHE02} \\[0.5ex]
1.199 & 3.80 \pm 0.06 & 0.63 - 0.10\tau \pm 0.03 & & \ref{idx_lgn_stt_obs_DHE02} \\[0.5ex]
\multicolumn{5}{l}{\citet{MSI03}
\footnote{Measurements during PRIDE, July 2000, from Iza\~{n}a and
  Puerto Rico. 
  Only $\dmtvma$ reported as measurements did not fit lognormal
  distributions.}% end footnote  
} \\[0.0ex]
 & 3.6 \pm 0.3 & \mbox{none} & & \ref{idx_lgn_stt_obs_MSI03} \\[0.5ex]
 & 4.1 & \mbox{none} & & \ref{idx_lgn_stt_obs_MSI03} \\[0.5ex]
\multicolumn{5}{l}{\citet{AKK06}
\footnote{Original manuscript does not contain $\gsd$.}% end footnote 
} \\[0.0ex]
% ncap2 -O -v -s 'dmt_vma_crs=14.2;gsd_anl_crs=1.5;ln_gsd_crs=ln(gsd_anl_crs);dmt_nma_crs=dmt_vma_crs*exp(-3.0*(ln_gsd_crs^2));' ~/nco/data/in.nc ~/foo.nc;ncks -H ~/foo.nc
0.0 & 1.1 & 0.0 & & \ref{idx_lgn_stt_obs_AKK06} \\[0.5ex]
0.0 & 5.5 & 0.0 & & \ref{idx_lgn_stt_obs_AKK06} \\[0.5ex]
0.0 & 14 & 0.0 & & \ref{idx_lgn_stt_obs_AKK06} \\[0.5ex]
\multicolumn{5}{l}{\citet{MGT12}
\footnote{AMMA size distribution used in DEAD coupled with SURFEX.}% end footnote 
} \\[0.0ex]
0.0 &  0.2  & 1.75 & 0.0008 & \ref{idx_lgn_stt_obs_MGT12} \\[0.5ex]
0.0 &  1.67 & 1.76 & 0.0092 & \ref{idx_lgn_stt_obs_MGT12} \\[0.5ex]
0.0 & 11.6  & 1.70 & 0.99 & \ref{idx_lgn_stt_obs_MGT12} \\[0.5ex]
\csznote{ % fxm: Uncomment after WGL07 published
\multicolumn{5}{l}{\citet{WGL07}
\footnote{Measurements reported for 
  SEPS (20041024--20041211, 150--75\,\dgrw, 8--36\,\dgrs) and 
  SOKS (20050110--20050220, 68--78\,\dgre, 49--54\,\dgrs), respectively.}% end footnote 
} \\[0.0ex]
% ncap2 -O -v -s 'dmt_vma_SEPS=2.20;gsd_anl_SEPS=1.36;ln_gsd_SEPS=ln(gsd_anl_SEPS);dmt_nma_SEPS=dmt_vma_SEPS*exp(-3.0*(ln_gsd_SEPS^2));' ~/nco/data/in.nc ~/foo.nc;ncks -H ~/foo.nc
% ncap2 -O -v -s 'dmt_vma_SOKS=2.27;gsd_anl_SOKS=1.54;ln_gsd_SOKS=ln(gsd_anl_SOKS);dmt_nma_SOKS=dmt_vma_SOKS*exp(-3.0*(ln_gsd_SOKS^2));' ~/nco/data/in.nc ~/foo.nc;ncks -H ~/foo.nc
1.66 & 2.20 & 1.36 & & \ref{idx_lgn_stt_obs_WGL07} \\[0.5ex]
1.30 & 2.27 & 1.54 & & \ref{idx_lgn_stt_obs_WGL07} \\[0.5ex]
} % end csznote
\hline
\end{longtable} % end tbl:lgn_stt_obs
\cite{PCE97} and \cite{PeC99} describe measurements and transport of
dust across the Atlantic and Pacific, respectively.
\cite{RJM03} summarize historical measurements of dust size
distributions, and analyze the influence of measurement technique on
the derived size distribution.
They show the derived size distribution is strongly sensitive
to the measurment technique.
During \trmidx{PRIDE}, measured $\dmtvma$ varied from 2.5--9\,\um\ 
depending on the instrument employed.
\cite{MSI03} show that the change in mineral dust size distribution
across the sub-tropical Atlantic is consistent with a slight updraft
of $\sim 0.33$\,\cmxs\ during transport.
\cite{Gin03} and \cite{CTH03} show that the effects of asphericity on
particle settling velocity play an important role in maintaining the
large particle tail of the size distribution during long range
transport. 

Table~\ref{tbl:lgn_stt_anl} applies the relations in Table~\ref{tbl:lgn}
to specific size distributions typical of tropospheric aerosols.
\begin{table}
\begin{minipage}{\hsize} % Minipage necessary for footnotes KoD95 p. 110 (4.10.4)
\renewcommand{\footnoterule}{\rule{\hsize}{0.0cm}\vspace{-0.0cm}} % KoD95 p. 111
\begin{center}
\caption[Analytic Lognormal Statistics]{\textbf{Analytic Lognormal Size Distribution Statistics
\footnote{Shown are statistics for each moment equalling 1\,\um, and
for $\dmtvma = 0.1, 2.5, 3.5, 5.0, 10.0$\,\um.}%
\footnote{
$\dmtnaa$, $\dmtsaa$, and $\dmtvaa$ are number, surface, and
volume-mean diameters, respectively.
$\dmtnma$, $\dmtsma$, and $\dmtvma$ are number, surface, and
volume median diameters, respectively.
$\dmtnwa$, $\dmtswa$, and $\dmtvwa$ are number, surface, and
volume-weighted diameters, respectively.}}% 
\label{tbl:lgn_stt_anl}}   
\vspace{\cpthdrhlnskp}
\begin{tabular}{ *{9}{>{$}r<{$}} } % KoD95 p. 94 describes '*' notation
\hline \rule{0.0ex}{\hlntblhdrskp}% 
\dmtnma & \dmtnaa, \dmtnwa & \dmtsaa & \dmtvaa & \dmtsma & \dmtswa & \dmtvma & \dmtvwa & \gsd \\[0.0ex]
\rdsnma & \rdsnaa, \rdsnwa & \rdssaa & \rdsvaa & \rdssma & \rdsswa & \rdsvma & \rdsvwa & \\[0.0ex]
\mbox{\um} & \mbox{\um} & \mbox{\um} & \mbox{\um} & \mbox{\um} & \mbox{\um} & \mbox{\um} & \mbox{\um} & \\[0.0ex]
\hline \rule{0.0ex}{\hlntblntrskp}%
% NB: Table produced by ~/nco/data/psd.nco
0.1    & 0.1272 & 0.1619 & 0.2056 & 0.2614 & 0.3323 & 0.4227 & 0.5373 & 2.0 \\[0.5ex]
0.1861 & 0.2366 & 0.3009 & 0.3825 & 0.4864 & 0.6185 & 0.7864 & 1.0    & 2.0 \\[0.5ex]
0.2366 & 0.3008 & 0.3825 & 0.4864 & 0.6185 & 0.7864 & 1.0    & 1.272  & 2.0 \\[0.5ex]
0.3009 & 0.3825 & 0.4864 & 0.6185 & 0.7864 & 1.0    & 1.272  & 1.617  & 2.0 \\[0.5ex]
0.3825 & 0.4864 & 0.6185 & 0.7864 & 1.0    & 1.272  & 1.617  & 2.056  & 2.0 \\[0.5ex]
0.4864 & 0.6185 & 0.7864 & 1.0    & 1.272  & 1.617  & 2.056  & 2.614  & 2.0 \\[0.5ex]
0.5915 & 0.7521 & 0.9563 & 1.216  & 1.546  & 1.966  & 2.5    & 3.179  & 2.0 \\[0.5ex]
0.6185 & 0.7864 & 1.0    & 1.272  & 1.617  & 2.056  & 2.614  & 3.324  & 2.0 \\[0.5ex]
0.7864 & 1.0    & 1.272  & 1.617  & 2.056  & 2.614  & 3.324  & 4.225  & 2.0 \\[0.5ex]
0.8281 & 1.053  & 1.339  & 1.702  & 2.165  & 2.753  & 3.5    & 4.450  & 2.0 \\[0.5ex]
1.0    & 1.272  & 1.617  & 2.056  & 2.614  & 3.324  & 4.227  & 5.373  & 2.0 \\[0.5ex]
1.183  & 1.504  & 1.913  & 2.432  & 3.092  & 3.932  & 5.0    & 6.356  & 2.0 \\[0.5ex]
2.366  & 3.008  & 3.825  & 4.864  & 6.184  & 7.864  & 10.0   & 12.72  & 2.0 \\[0.5ex]
\hline
\end{tabular}
\end{center}
\end{minipage}
\end{table} % end tbl:lgn_stt_anl
Values in Table~\ref{tbl:lgn_stt_anl} are valid for radius and
diameter distributions.
Table~\ref{tbl:lgn} shows that all moments of the size distribution
depend linearly on $\dmtnma$ (or $\rdsnma$). 
Therefore all rows in Table~\ref{tbl:lgn_stt_anl} scale linearly
(for a constant geometric standard deviation).
For example, values in the row with $\dmtnma = 1.0$\,\um\ are ten
times the corresponding values for the row $\dmtnma = 0.1$\,\um.
Hence it suffices for Table~\ref{tbl:lgn_stt_anl} to show a decade
range in~$\dmtnma$.

\subsubsection[Related Forms]{Related Forms}
Many important applications make available size distribution
information in a form similar to, but hard to recognize as,
the analytic lognormal PDF (\ref{eqn:dst_lgn}). 
The Aerosol Robotic Network, \trmdfn{AERONET}, for example,  
retrieves size distributions from solar \trmidx{almucantar} 
radiances\footnote{
The almucantar radiances are radiance measurements in a circle of
equal scattering angle centered in a plane about the Sun, i.e.,
radiance measurements at known forward scattering phase function
angles.} \cite[][]{DuK00,DSH00,DHL02}.
AERONET labels the retrieved size distribution
$\dfr\vlmofrds/\dfr\ln\rds$ and reports the values in [\umCxumS]
units. 
The correspondence between the AERONET retrievals and 
$\dfr\cnc/\dfr\ln\rds$ (\ref{eqn:dst_lgn}) in [\nbrxmCm] units
is not exactly clear.
Unfortunately, Table~\ref{tbl:lgn} does not help much here.
Let us now show how to bridge the gap between theory and measurement.

First, total distributions contain $\cncttl$ particles per unit volume
and thus $\cncttl$ applies as a multiplicative factor to
(\ref{eqn:dst_lgn})  
\begin{equation}
\dstnbrofdmt = \frac{\cncttl}{\sqrt{2\mpi} \, \dmt \lngsd} \exp\! 
\left[-\frac{1}{2}\left( \frac{ \ln(\dmt/\dmtnma)}{\lngsd} \right)^{2} \right] 
\label{eqn:dst_lgn_ttl}
\end{equation}
Note that (\ref{eqn:dst_lgn_ttl}) is only normalized if 
$\cncttl = 1.0$ (cf.\ Section~\ref{sxn:nrm}).

Applying (\ref{eqn:dst_dfn}) to (\ref{eqn:dst_lgn_ttl}) yields
% fxm: Add formulae for dV/dlnD, dS/dlnD, dN/dlnD
\begin{equation}
\frac{\dfr\cnc}{\dfr\dmt} = 
\frac{\cncttl}{\sqrt{2\mpi} \, \dmt \lngsd} \exp\! 
\left[-\frac{1}{2}\left( \frac{ \ln(\dmt/\dmtnma)}{\lngsd} \right)^{2} \right] 
\label{eqn:dst_lgn_dNdD}
\end{equation}
Multiplying each side of (\ref{eqn:dst_lgn_dNdD}) by $\dmt$ and
substituting $\dfr\ln\dmt = \dmt^{-1}\,\dfr\dmt$ leads to
\begin{equation}
\frac{\dfr\cnc}{\dfr\ln\dmt} = 
\frac{\cncttl}{\sqrt{2\mpi}\,\lngsd} \exp\! 
\left[-\frac{1}{2}\left( \frac{ \ln(\dmt/\dmtnma)}{\lngsd} \right)^{2} \right] 
\label{eqn:dst_lgn_dNdlnD}
\end{equation}
The derivative in (\ref{eqn:dst_lgn_dNdlnD}) is with respect to the
logarithm of the diameter. 
The change in the independent variable of differentiation defines a
new distribution which could be written $\dstnbr(\ln\dmt)$ to
distinguish it from the normal linear distribution $\dstnbr(\dmt)$
(\ref{eqn:dst_dfn}). 
However, the nomenclature $\dstnbr(\ln\dmt)$ could be misinterpreted. 
We follow \cite{SeP97} and denote logarithmically-defined
distributions with a superscript~$\me$ on the distribution
that re-inforces the use of $\ln\dmt$ as the independent variable
\begin{equation}
\dstlnnbroflndmt \equiv \dstnbr(\ln\dmt) \equiv \frac{\dfr\cnc}{\dfr\ln\dmt}
\label{eqn:dstln_dfn}
\end{equation}
The SI units of $\dstnbrofdmt$ (\ref{eqn:dst_dfn}) and
$\dstlnnbroflndmt$ (\ref{eqn:dstln_dfn}) are [\nbrxmCm] and [\nbrxmC], 
respectively. 

Remote sensing applications often retrieve columnar distributions rather
than volumetric distributions.
The columnar number distribution $\dstnbrpth(\dmt)$, for example, is
simply the vertical integral of the particle number distribution
$\dstnbr(\dmt)$,  
\begin{subequations} 
\label{eqn:dst_pth_all_dfn}
\begin{alignat}{3}
\label{eqn:nbr_pth_dfn}
\dstnbrpth(\dmt) &\equiv \frac{\dfr\cncttlpth}{\dfr\dmt} &&=
\int_{\hgt=0}^{\hgt=\infty} \dstnbr(\dmt,\hgt) \,\dfr\hgt &&=
\mbox{same} \\
\label{eqn:xsa_pth_dfn}
\dstxsapth(\dmt) &\equiv \frac{\dfr\xsattlpth}{\dfr\dmt} &&=
\int_{\hgt=0}^{\hgt=\infty} \dstxsa(\dmt,\hgt) \,\dfr\hgt &&=
\int_{\hgt=0}^{\hgt=\infty} \frac{\mpi}{4} \dmt^{2} \dstnbr(\dmt,\hgt) \,\dfr\hgt \\
\label{eqn:sfc_pth_dfn}
\dstsfcpth(\dmt) &\equiv \frac{\dfr\sfcttlpth}{\dfr\dmt} &&=
\int_{\hgt=0}^{\hgt=\infty} \dstsfc(\dmt,\hgt) \,\dfr\hgt &&=
\int_{\hgt=0}^{\hgt=\infty}       \mpi     \dmt^{2} \dstnbr(\dmt,\hgt) \,\dfr\hgt \\
\label{eqn:vlm_pth_dfn}
\dstvlmpth(\dmt) &\equiv \frac{\dfr\vlmttlpth}{\dfr\dmt} &&=
\int_{\hgt=0}^{\hgt=\infty} \dstvlm(\dmt,\hgt) \,\dfr\hgt &&=
\int_{\hgt=0}^{\hgt=\infty} \frac{\mpi}{6} \dmt^{3} \dstnbr(\dmt,\hgt) \,\dfr\hgt \\
\label{eqn:mss_pth_dfn}
\dstmsspth(\dmt) &\equiv \frac{\dfr\mssttlpth}{\dfr\dmt} &&=
\int_{\hgt=0}^{\hgt=\infty} \dstmss(\dmt,\hgt) \,\dfr\hgt &&=
\int_{\hgt=0}^{\hgt=\infty} \frac{\mpi}{6} \dns \dmt^{3} \dstnbr(\dmt,\hgt) \,\dfr\hgt
\end{alignat}
\end{subequations} 
SI~units of the columnar distributions $\dstxxxpth$ for 
$\xxx = \nbrsbs, \xsasbs, \sfcsbs, \vlmsbs, \msssbs$
(\ref{eqn:dst_pth_all_dfn}) are one less ``per meter'' than the
corresponding volumetric distributions, e.g., $\dstvlm$ and
$\dstvlmpth$ are in [\mCxmCm] and [\mCxmSm], respectively.
This is because of integration over the vertical coordinate.

Combining (\ref{eqn:dst_pth_all_dfn}) with (\ref{eqn:dst_lgn_dNdlnD}) 
leads to 
\begin{subequations} 
% Similar to SeP97 p. 428 7.51
% Variation of DHE02 p. 626 (A3)
\label{eqn:dst_lgn_dNcdlnD}
\begin{alignat}{4}
\label{eqn:dst_nbr_pth_dNcdlnD}
\dstlnnbrpth(\ln\dmt) &\equiv
\frac{\dfr\cncttlpth}{\dfr\ln\dmt} &&=
\frac{\cncttlpth}{\sqrt{2\mpi}\,\lngsd} \exp\! 
\left[-\frac{1}{2}\left( \frac{ \ln(\dmt/\dmtnma)}{\lngsd} \right)^{2} \right] \\
\label{eqn:dst_xsa_pth_dAcdlnD}
\dstlnxsapth(\ln\dmt) &\equiv
\frac{\dfr\xsattlpth}{\dfr\ln\dmt} &&=
\sqrt{\frac{\mpi}{2}} \frac{\cncttlpth\dmt^{2}}{4\lngsd} \exp\! 
\left[-\frac{1}{2}\left( \frac{ \ln(\dmt/\dmtnma)}{\lngsd} \right)^{2} \right] \\
\label{eqn:dst_sfc_pth_dScdlnD}
\dstlnsfcpth(\ln\dmt) &\equiv
\frac{\dfr\sfcttlpth}{\dfr\ln\dmt} &&=
\sqrt{\frac{\mpi}{2}} \frac{\cncttlpth\dmt^{2}}{\lngsd} \exp\! 
\left[-\frac{1}{2}\left( \frac{ \ln(\dmt/\dmtnma)}{\lngsd} \right)^{2} \right] \\
\label{eqn:dst_vlm_pth_dVcdlnD}
\dstlnvlmpth(\ln\dmt) &\equiv
\frac{\dfr\vlmttlpth}{\dfr\ln\dmt} &&=
\sqrt{\frac{\mpi}{2}} \frac{\cncttlpth\dmt^{3}}{6\lngsd} \exp\! 
\left[-\frac{1}{2}\left( \frac{ \ln(\dmt/\dmtnma)}{\lngsd} \right)^{2} \right] \\
\label{eqn:dst_mss_pth_dMcdlnD}
\dstlnmsspth(\ln\dmt) &\equiv
\frac{\dfr\mssttlpth}{\dfr\ln\dmt} &&=
\sqrt{\frac{\mpi}{2}} \frac{\dns\cncttlpth\dmt^{3}}{6\lngsd} \exp\!
\left[-\frac{1}{2}\left( \frac{ \ln(\dmt/\dmtnma)}{\lngsd} \right)^{2} \right]
\end{alignat}
\end{subequations} 
These logarithmic columnar (vertically integrated) distributions 
(\ref{eqn:dst_lgn_dNcdlnD}) are one less ``per meter'' than the
corresponding linear columnar distributions
(\ref{eqn:dst_pth_all_dfn}), e.g., $\dstvlmpth$ and $\dstlnvlmpth$ are in
[\mCxmSm] and [\mCxmS], respectively. 
In order for the area under the curve to be proportional to the
integrated distributions, logarithmic distributions should be plotted
on semi-log axes, e.g., horizontal axis with logarithmic size $\dmt$
and vertical axis with linearly spaced values of $\dstlnvlm(\ln\dmt)$
\cite[][p.~415]{SeP97}.

Measurements (or retrievals such as AERONET) are usually reported in
historical units that can be counted rather than in pure~\trmidx{SI}. 
SI~units for $\dstvlm(\dmt) = \dfr\vlmofdmt/\dfr\dmt$ are
[\mCxmCm], i.e., particle volume per unit air volume per unit particle 
diameter. 
% 20130522: Is the following sentence true? Wouldn't units exponents  
% multiply yielding [m3 m-4]?
These units condense to [\mCxmS], or, multiplying by $10^{6}$,
[\umCxumS].
These condensed units may be confused with particle volume per unit
particle surface area ($\vlmofdmt/\sfcofdmt$), or with columnar
particle volume per unit horizontal surface (e.g., ground or ocean)
area ($\int\vlm(\hgt)\,\dfr\hgt$).
AERONET most definitely does \textit{not} report any of these three 
quantities $\dfr\vlm/\dfr\rds$, $\vlmofdmt/\sfcofdmt$, or 
$\int\vlm(\hgt)\,\dfr\hgt$.
AERONET reports $\dstlnvlmpth(\ln\dmt)$
the vertically integrated logarithmic volume
distribution (\ref{eqn:dst_vlm_pth_dVcdlnD}), the logarithmic
derivative of the \trmidx{columnar volume} $\vlmttlpth$.

\subsubsection[Variance]{Variance}\label{sxn:lgn_vrn}
According to (\ref{eqn:vrn_gnr}), the \trmidx{variance} $\vrndmt$ of the
lognormal distribution (\ref{eqn:dst_lgn}) is
\begin{equation}
\vrndmt = \frac{1}{\sqrt{2 \mpi} \lngsd} 
\int_{0}^{\infty}\frac{1}{\dmt} \exp\!
\left[-\frac{1}{2}\left( \frac{ \ln(\dmt/\dmtnma)}{\lngsd} \right)^{2} \right]
(\dmt - \dmtavg)^{2} \,\dfr\dmt
\label{eqn:vrn_lgn}
\end{equation}

\subsubsection[Non-standard terminology]{Non-standard terminology}\label{sxn:mst}\label{sxn:trm}
% fxm TODO 2884 move to standalone location and submit as comment/note?
Non-standard terminology leads to much confusion in the literature.
For example, \cite{DHE02} provide precise analytic definitions
of their supposedly lognormal size distribution parameters.
However, their terminology is inconsistent with their definitions.
Distributions computed according to their definitions are not
lognormal distributions. 
\cite{DHE02} Equation~A1 (their p.~606) defines the \textit{mean}
logarithmic radius $\rdsvaa$ of the volume distribution which they
confusingly name the volume \textit{median} radius~$\rdsvma$.
\cite{DHE02} Equation~A2 (their p.~606) defines the standard deviation 
of the logarithm of the volume distribution.
This differs from the \trmidx{geometric standard deviation}~$\gsd$
of a lognormal distribution.
The correct parameters of a lognormal distribution (\ref{eqn:dst_lgn})  
are $\rdsnma$ and $\gsd$ (or $\gsdtld \equiv \ln \gsd$)
For a lognormal volume path distribution $\dstlnvlmpth(\ln\dmt)$
(\ref{eqn:dst_vlm_pth_dVcdlnD}) the appropriate parameters are 
$\rdsvma$ and $\gsd$ (or $\gsdtld \equiv \ln \gsd$), not 
$\rdsvaa$ and $\sqrt{\vrnrds}$ (\ref{eqn:vrn_lgn}).
\cite{DHE02} Equation~1 (their p.~593) is the correct form for 
$\dstlnvlmpth(\ln\dmt)$ (\ref{eqn:dst_vlm_pth_dVcdlnD}).

\subsubsection[Bounded Distribution]{Bounded Distribution}\index{bounded distribution}\label{sxn:lgn_bnd}
The statistical properties of a bounded lognormal distribution
are expressed in terms of the \trmidx{error function}
(\S\ref{sxn:erf}).   
The \trmidx{cumulative concentration} bounded by $\dmtmax$ is given by
applying (\ref{eqn:cnc_cml}) to (\ref{eqn:dst_lgn})
\begin{equation}
\cncfnc(\dmt < \dmtmax) = \frac{\cncttl}{\sqrt{2\mpi}\,\lngsd}
\int_{0}^{\dmtmax} \frac{1}{\dmt} \exp\! 
\left[-\frac{1}{2}\left( \frac{ \ln(\dmt/\dmtnma)}{\lngsd} \right)^{2} \right] \,\dfr\dmt
\label{eqn:lgn_bnd}
\end{equation}
We make the change of variable $\zzz = (\ln \dmt - \ln \dmtnma)/\sqrt{2}
\, \lngsd$
\begin{eqnarray}
\zzz & = & (\ln \dmt - \ln \dmtnma)/\sqrt{2}\,\lngsd \nonumber \\ 
\dmt & = & \dmtnma \me^{\sqrt{2} \, \zzz \lngsd} \nonumber \\
& = & \dmtnma \gsd^{\sqrt{2} \, \zzz} \nonumber \\
\dfr\zzz & = & (\sqrt{2} \, \dmt \lngsd)^{-1} \,\dfr\dmt \nonumber \\ 
\dfr\dmt & = & \sqrt{2}\,\lngsd \dmtnma \me^{\sqrt{2} \, \zzz \lngsd} \,
\dfr\zzz \nonumber \\
& = & \sqrt{2}\,\lngsd \dmtnma \gsd^{\sqrt{2} \, \zzz} \,\dfr\zzz
\label{eqn:cov_z}
\end{eqnarray}
which maps $\dmt \in (0, \dmtmax)$ into $\zzz \in (-\infty, \ln \dmtmax -
\ln \dmtnma)/\sqrt{2}\,\lngsd)$. 
In terms of $\zzz$ we obtain
\begin{eqnarray}
\cncfnc(\dmt < \dmtmax) & = & \frac{\cncttl}{\sqrt{2\mpi}\,\lngsd}
\int_{-\infty}^{(\ln \dmtmax - \ln \dmtnma)/\sqrt{2}\,\lngsd} 
\frac{1}{\dmtnma \me^{\sqrt{2} \, \zzz \lngsd} } \,
\me^{-\zzz^{2}} \, 
\sqrt{2}\,\lngsd \dmtnma \me^{\sqrt{2} \, \zzz \lngsd} \,\dfr\zzz \nonumber \\
& = & \frac{\cncttl}{\sqrt{\mpi} }
\int_{-\infty}^{(\ln \dmtmax - \ln \dmtnma)/\sqrt{2}\,\lngsd} 
\me^{-\zzz^{2}} \,\dfr\zzz \nonumber \\
& = & \frac{\cncttl}{\sqrt{\mpi} } \left(
\int_{-\infty}^{0} \me^{-\zzz^{2}} \,\dfr\zzz +
\int_{0}^{(\ln \dmtmax - \ln \dmtnma)/\sqrt{2}\,\lngsd} 
\me^{-\zzz^{2}} \,\dfr\zzz \right) \nonumber \\
& = & \frac{\cncttl}{2} \left(
\frac{2}{\sqrt{\mpi} } \int_{0}^{+\infty} \me^{-\zzz^{2}} \,\dfr\zzz +
\frac{2}{\sqrt{\mpi} } \int_{0}^{(\ln \dmtmax - \ln \dmtnma)/\sqrt{2}\,\lngsd} 
\me^{-\zzz^{2}} \,\dfr\zzz \right) \nonumber \\
& = & \frac{\cncttl}{2} \left[ 
\erffnc(\infty) + 
\erffnc\! \left( \frac{\ln (\dmtmax / \dmtnma)}{\sqrt{2}\,\lngsd} \right)
\right]
\nonumber \\
& = & \frac{\cncttl}{2} + 
\frac{\cncttl}{2} \, \erffnc\! \left( \frac{\ln (\dmtmax / \dmtnma)}{\sqrt{2}\,\lngsd} \right)
\label{eqn:lgn_erf}
\end{eqnarray}
where we have used the properties of the \trmidx{error function}
(\S\ref{sxn:erf}).  
The same procedure can be performed to compute the cumulative
concentration of particles smaller than $\dmtmin$.
When $\cncfnc(\dmt < \dmtmin)$ is subtracted from (\ref{eqn:lgn_erf})
we obtain the truncated concentration (\ref{eqn:cnc_ntv})
\begin{equation}
\cncfnc(\dmtmin, \dmtmax) = \frac{\cncttl}{2}
\left[
\erffnc\! \left( \frac{\ln (\dmtmax / \dmtnma)}{\sqrt{2}\,\lngsd} \right) -
\erffnc\! \left( \frac{\ln (\dmtmin / \dmtnma)}{\sqrt{2}\,\lngsd} \right)
\right]
\label{eqn:lgn_nbr_ntv}
\end{equation}

We are also interested in the bounded distributions of higher moments,
e.g., the mass of particles lying between $\dmtmin$ and $\dmtmax$.
The cross-sectional area, surface area, volume, and mass distributions
of spherical particles are related to their number distribution by 
\begin{subequations} 
\label{eqn:dst_all_dfn}
\begin{align}
\label{eqn:xsa_dfn}
\dstxsa(\dmt) &= \frac{\mpi}{4} \dmt^{2} \dstnbrofdmt \\
\label{eqn:sfc_dfn}
\dstsfc(\dmt) &=       \mpi     \dmt^{2} \dstnbrofdmt \\
\label{eqn:vlm_dfn}
\dstvlm(\dmt) &= \frac{\mpi}{6} \dmt^{3} \dstnbrofdmt \\
\label{eqn:mss_dfn}
\dstmss(\dmt) &= \frac{\mpi}{6} \dns \dmt^{3} \dstnbrofdmt
\end{align}
\end{subequations} 
so that we may simply substitute $\dmtnma = \dmtvma$, for example, in 
(\ref{eqn:lgn_nbr_ntv}) and we obtain
\begin{equation}
\vlmfnc(\dmtmin, \dmtmax) = \frac{\cncttl}{2}
\left[
\erffnc\! \left( \frac{\ln (\dmtmax / \dmtvma)}{\sqrt{2}\,\lngsd} \right) -
\erffnc\! \left( \frac{\ln (\dmtmin / \dmtvma)}{\sqrt{2}\,\lngsd} \right)
\right]
\label{eqn:lgn_mss_ntv}
\end{equation}

\subsubsection[Statistics of Bounded Distributions]{Statistics of Bounded Distributions}\label{sxn:lgn_bnd_stt}
All of the relations given in Table~\ref{tbl:lgn} may be
re-expressed in terms of truncated lognormal distributions, but doing
so is tedious, and requires new terminology.
Instead we derive the expression for a typical size distribution
statistic, and allow the reader to generalize.
We generalize (\ref{eqn:stt_gnr}) to consider
\begin{equation}
\gggavg^* = \int_{\dmtmin}^{\dmtmax}  \dmt \, \pdffnc^*(\dmt) \,
\dfr\dmt
\end{equation}
Note the domain of integration, $\dmt \in (\dmtmin, \dmtmax)$,
reflects the fact that we are considering a bounded distribution.
The superscript $^{*}$ indicates that the average statistic refers
to a truncated distribution and reminds us that $\gggavg^{*} \ne
\gggavg$. 
Defining a closed form expression for $\pdffnc^{*} (\dmt)$ requires some
consideration.  
This truncated distribution has $\cncttlstr$ defined by
(\ref{eqn:lgn_nbr_ntv}), and is completely specified on $\dmt \in
(0,\infty)$ by
\begin{equation}
\pdffnc^{*} (\dmt) = \left\{
\begin{array}
{r@{\quad,\quad}l}
0 & 0 < \dmt < \dmtmin \\
\cncfnc(\dmtmin, \dmtmax) \, \pdfdmt / \cncttl & \dmtmin \le \dmt \le \dmtmax \\
0 & \dmtmax < \dmt < \infty
\end{array}
\right.
\end{equation}
The difficulty is that the three parameters of the lognormal
distribution, $\dmtnma$, $\gsd$, and $\cncttl$ are defined in terms of 
an untruncated distribution.
Using (\ref{eqn:lgn_nbr_ntv}) we can write
\begin{equation}
\pdffnc^{*} (\dmt) = \frac{1}{\cncttlstr} \, \dstnbrofdmt
\cncttlstr = \cncfnc(\dmtmin, \dmtmax)
\end{equation}

If we think of $\pdffnc^{*}$ order to be properly normalized to unity,
note that  (fxm) % fxm
Thus when we speak of truncated distributions it is important to keep
in mind that the parameters $\dmtnma$, $\gsd$, and $\cncttl$ refer to
the untruncated distribution. 

The properties of the truncated distribution will be expressed in
terms of $\dmtnmastr$, $\gsdstr$, and $\cncttlstr$, respectively.

Consider the mean size, $\dmt$.
In terms of (\ref{eqn:stt_gnr}) we have $\ggg(\dmt) = \dmt$ so that
\begin{equation}
\dmtavg = \int_{\dmtmin}^{\dmtmax}  \dmt \, \pdffnc(\dmt) \,\dfr\xxx
\end{equation}

\subsubsection[Overlapping Distributions]{Overlapping Distributions}
Consider the problem of distributing $\srcnbr$ independent and possibly
\trmdfn{overlapping distributions} of particles into $\snknbr$ independent and
possibly overlapping distributions of particles.
To reify the problem we call the $\srcnbr$ bins the \trmidx{source
bins} (these bins represent the parent size distributions in
\trmidx{mineral dust} source areas) and the $\snknbr$ bins as 
\trmidx{sink bins} (which represent sizes transported in the
atmosphere). 
Typically we know the total mass $\mssttl$ or number $\cncttl$ of 
source particles to distribute into the sink bins and we know the
fraction of the total mass to distribute which resides in each source
distribution, $\mssfrcsrc$. 
The problem is to determine matrices of \trmidx{overlap factors}
$\nbrovrsrcsnk$ and $\mssovrsrcsnk$ which determine what number and
mass fraction, respectively, of each source bin $\srcidx$ is blown
into each sink bin $\snkidx$. 

The mass and number fractions contained by the \trmidx{source
distributions} are normalized such that
\begin{equation}
\sum_{\srcidx = 1}^{\srcnbr} \mssfrcsrc = \sum_{\srcidx = 1}^{\srcnbr} \nbrfrcsrc = 1
\label{eqn:frc_nrm}
\end{equation}
In the case of \trmidx{dust emissions}, $\mssfrcsrc$ and $\nbrfrcsrc$
may vary with spatial location.

The overlap factors $\nbrovrsrcsnk$ and $\mssovrsrcsnk$ are defined by the
relations 
\begin{eqnarray}
\label{eqn:ovr_nbr_dfn}
\nbr_\snkidx & = & \sum_{\srcidx = 1}^{\srcnbr} \nbrovrsrcsnk \nbr_\srcidx \nonumber \\
& = & \nbrttl \sum_{\srcidx = 1}^{\srcnbr} \nbrovrsrcsnk \nbrfrcsrc \\
\label{eqn:ovr_mss_dfn}
\mss_\snkidx & = & \sum_{\srcidx = 1}^{\srcnbr} \mssovrsrcsnk \mss_\srcidx \nonumber \\
& = & \mssttl \sum_{\srcidx = 1}^{\srcnbr} \mssovrsrcsnk \mssfrcsrc
\end{eqnarray}

Using (\ref{eqn:lgn_nbr_ntv}) and (\ref{eqn:frc_nrm}) we find
\begin{eqnarray}
\label{eqn:ovr_nbr}
\nbrovrsrcsnk & = & 
\frac{1}{2}
\left[
\erffnc\! \left( \frac{\ln (\dmtmaxj / \dmtnmai)}{\sqrt{2}\,\lngsdi } \right) -
\erffnc\! \left( \frac{\ln (\dmtminj / \dmtnmai)}{\sqrt{2}\,\lngsdi } \right)
\right] \\
\label{eqn:ovr_mss}
\mssovrsrcsnk & = &
\frac{1}{2}
\left[
\erffnc\! \left( \frac{\ln (\dmtmaxj / \dmtvmai)}{\sqrt{2}\,\lngsdi } \right) -
\erffnc\! \left( \frac{\ln (\dmtminj / \dmtvmai)}{\sqrt{2}\,\lngsdi } \right)
\right]
\end{eqnarray}
fxm: The mathematical derivation appears correct but the overlap
factor appears to asymptote to 0.5 rather than to 1.0 for $\dmtmax \gg
\dmtnma \gg \dmtmin$.

A \trmidx{mass distribution} has the same form as a lognormal 
\trmidx{number distribution} but has a different \trmidx{median diameter}. 
Thus the overlap matrix elements apply equally to mass and number
distributions depending on the median diameter used in the following
formulae. 
For the case where both source and sink distributions are complete
lognormal distributions, 
\begin{eqnarray}
\mss(\dmt) & = & \sum_{\srcidx = 1}^{\srcidx=\srcnbr} \mss_\srcidx(\dmt) \nonumber
\label{eqn:mlt_mdl_lgn}
\end{eqnarray}

\subsubsection[Median Diameter]{Median Diameter}\index{median diameter}\label{sxn:mdn}
Substituting $\dmt = \dmtnma$ into (\ref{eqn:lgn_erf}) we obtain
\begin{equation}
\cncfnc(\dmt < \dmtnma) = \frac{\cncttl}{2}
\label{eqn:lgn_mdn}
\end{equation}
This proves that $\dmtnma$ is the median diameter (\ref{eqn:mdn_dfn}). 
The lognormal distribution is the only distribution known (to us)
which is most naturally expressed in terms of its median diameter.

\subsubsection[Mode Diameter]{Mode Diameter}\index{mode diameter}\label{sxn:mod}
The \trmdfn{mode} is the most frequently occuring value of a
distribution. 
The \trmdfn{mode diameter} or \trmdfn{modal diameter} of the 
number distribution $\dstnbrofdmt$ is the diameter $\dmtnda$ that
satisfies 
\begin{equation}
\frac{\dfr\dstnbrofdmt}{\dfr\dmt}\bigg|_{\dmt=\dmtnda} = 0
\label{eqn:lgn_mdl_dfn}
\end{equation}
Applying condition (\ref{eqn:lgn_mdl_dfn}) to (\ref{eqn:dst_lgn})
proves that the median and modal diameters are identical for lognormal 
distributions 
\begin{equation}
\dmtnda = \dmtnma
\label{eqn:lgn_dmt_mdl_mdn}
\end{equation}
The number, surface, volume, and mass distributions are all lognormal 
if any one is.
Therefore (\ref{eqn:lgn_dmt_mdl_mdn}) implies $\dmtsda = \dmtsma$,
and $\dmtvda = \dmtvma$.

\subsubsection[Multimodal Distributions]{Multimodal Distributions}\label{sxn:mlt}
Realistic particle size distributions may be expressed as an
appropriately weighted sum of individual modes. 
\begin{equation}
\dstnbrofdmt = \sum_{\psdidx = 1}^{\psdnbr} \dstnbridxofdmt
\label{eqn:dst_nbr_mlt_dfn}
\end{equation}
where $\dstnbridxofdmt$ is the number distribution of the $\psdidx$th 
component mode\footnote{Throughout this section the $\psdidx$
superscript represents an index of the component mode, not an
exponent.}. 
Such particle size distributions are called \trmdfn{multimodal
istributions} because they contain one maximum for each component
distribution. 
Generalizing (\ref{eqn:cnc_ttl_dfn}), the total number concentration 
becomes
\begin{eqnarray}
\cncttl & = & \sum_{\psdidx = 1}^{\psdnbr} \int_{0}^{\infty} 
\dstnbridxofdmt \,\dfr\dmt \nonumber \\
& = & \sum_{\psdidx = 1}^{\psdnbr} \cncttlidx
\label{eqn:cnc_ttl_dfn_mlt}
\end{eqnarray}
where $\cncttlidx$ is the total number concentration of the
$\psdidx$th component mode. 

The \trmidx{median diameter} of a \trmidx{multimodal distribution} is
obtained by following the logic of
(\ref{eqn:lgn_bnd})--(\ref{eqn:lgn_nbr_ntv}). 
The number of particles smaller than a given size is
\begin{equation}
\cncfnc(\dmt < \dmtmax) = \sum_{\psdidx = 1}^{\psdnbr} 
\frac{\cncttlidx}{2} + 
\frac{\cncttlidx}{2} \, \erffnc\! \left( \frac{\ln (\dmtmax / \dmtnmaidx)}{\sqrt{2}\,\lngsdidx } \right)
\label{eqn:lgn_mdn_mlt}
\end{equation}
For the median particle size, $\dmtmax \equiv \dmtnma$, and we can
move the unknown $\dmtnma$ to the LHS yielding
\begin{eqnarray}
\sum_{\psdidx = 1}^{\psdnbr} \frac{\cncttlidx}{2} + 
\frac{\cncttlidx}{2} \, \erffnc\! \left( \frac{\ln (\dmtnma / \dmtnmaidx)
}{ \sqrt{2}\,\lngsdidx } \right) 
& = & \frac{\cncttl}{2} \nonumber \\ 
\sum_{\psdidx = 1}^{\psdnbr} 
\cncttlidx \, \erffnc\! \left( \frac{\ln (\dmtnma / \dmtnmaidx) }{
\sqrt{2}\,\lngsdidx } \right) & = & 0
\label{eqn:lgn_mdn_mlt2}
\end{eqnarray}
where we have used $\cncttl = \sum_\psdidx^\psdnbr \cncttlidx$.
Obtaining $\dmtnma$ for a multimodal distribution requires 
numerically solving (\ref{eqn:lgn_mdn_mlt2}) given the $\cncttlidx$,
$\dmtnmaidx$, and $\gsdidx$.

% fxm:
%The mean diameter of a multimodal distribution is obtained by
%following the logic of (\ref{eqn:})--(\ref{eqn:}).

\subsection[Higher Moments]{Higher Moments}\label{sxn:mmt}
It is often useful to compute higher moments of the number distribution.  
Each factor of the independent variable weighting the number
distribution function $\dstnbrofdmt$ in the integrand of
(\ref{eqn:szavg_gnr}) counts as a \trmdfn{moment}. 
The $\mmtidx$th moment of $\dstnbrofdmt$ is 
\begin{equation}
\mmtfnc(\mmtidx) = \int_{0}^{\infty} \dstnbrofdmt \dmt^\mmtidx \,\dfr\dmt
\label{eqn:dst_mmt_dfn}
\end{equation}

The statistical properties of higher moments of the lognormal size
distribution may be obtained by direct integration of
(\ref{eqn:dst_mmt_dfn}). 
\begin{eqnarray}
\mmtfnc(\mmtidx) & = & \frac{\cncttl}{\sqrt{2\mpi}\,\lngsd}
\int_{0}^{\infty} \frac{1}{\dmt} \exp\! 
\left[-\frac{1}{2}\left( \frac{ \ln(\dmt/\dmtnma)}{\lngsd} \right)^{2} \right] 
\dmt^{\mmtidx} \,\dfr\dmt \nonumber \\
& = & \frac{\cncttl}{\sqrt{2\mpi}\,\lngsd}
\int_{0}^{\infty} \dmt^{\mmtidx - 1} \exp\! 
\left[-\frac{1}{2}\left( \frac{ \ln(\dmt/\dmtnma)}{\lngsd} \right)^{2} \right] \,\dfr\dmt
\label{eqn:dst_mmt_ntg}
\end{eqnarray}
We make the same change of variable 
$z = (\ln \dmt - \ln \dmtnma)/\sqrt{2}\,\lngsd$ as in
(\ref{eqn:cov_z}).
This maps $\dmt \in (0, +\infty)$ into $\zzz \in (-\infty, +\infty)$.
In terms of $\zzz$ we obtain
\begin{eqnarray}
\mmtfnc(\mmtidx) & = & \frac{\cncttl}{\sqrt{2\mpi}\,\lngsd}
\int_{-\infty}^{+\infty} ( \dmtnma \me^{\sqrt{2} \, \zzz \lngsd} )^{\mmtidx - 1} 
\me^{-\zzz^{2}}
\sqrt{2}\,\lngsd \dmtnma \me^{\sqrt{2} \, \zzz \lngsd} \,\dfr\zzz \nonumber \\
& = & \frac{\cncttl}{\sqrt{\mpi} }
\int_{-\infty}^{+\infty} ( \dmtnma \me^{\sqrt{2} \, \zzz \lngsd} )^{\mmtidx} 
\me^{-\zzz^{2}} \,\dfr\zzz \nonumber \\
& = & \frac{\cncttl \dmtnma^{\mmtidx}}{\sqrt{\mpi} }
\int_{-\infty}^{+\infty} \me^{\sqrt{2} \mmtidx \zzz \lngsd}
\me^{-\zzz^{2}} \,\dfr\zzz \nonumber \\
& = & \frac{\cncttl \dmtnma^{\mmtidx}}{\sqrt{\mpi} }
\int_{-\infty}^{+\infty} \me^{-\zzz^{2} + \sqrt{2} \mmtidx \zzz \lngsd}
\,\dfr\zzz \nonumber \\
& = & \frac{\cncttl \dmtnma^{\mmtidx}}{\sqrt{\mpi} }
\sqrt{ \mpi } \exp\! \left( \frac{ 2 \mmtidx^{2} \lngsdsqr}{4} \right) \nonumber \\
& = & \cncttl \dmtnma^{\mmtidx} 
%\exp\! \left( \frac{ \mmtidx^{2} \lngsdsqr}{2} \right)
\exp ( \mbox{$\frac{1}{2}$} \mmtidx^{2} \lngsdsqr )
\label{eqn:dst_mmt_nsw}
\end{eqnarray}
where we have used (\ref{eqn:nrm_gss_gnr}) with $\alpha = 1$ and
$\beta = \sqrt{2} \mmtidx \lngsd$.

Applying the formula (\ref{eqn:dst_mmt_nsw}) to the first five moments
of the lognormal distribution function we obtain
\begin{equation}
\begin{array}{ r l >{\displaystyle}l<{} r >{\displaystyle}l<{} r
>{\displaystyle}l<{} r >{\displaystyle}l<{} }
\mmtfnc(0) & = & \int_{0}^{\infty} \dstnbrofdmt        \,\dfr\dmt 
& = & \cncttl & = & \cncttl & = & \cncttl \\[1.5ex]
\mmtfnc(1) & = & \int_{0}^{\infty} \dstnbrofdmt \dmt   \,\dfr\dmt
& = & \cncttl \dmtnma     \exp ( \mbox{$\frac{1}{2}$} \lngsdsqr ) 
& = & \dmtttl
& = & \cncttl \dmtnaa \\[1.5ex]
\mmtfnc(2) & = & \int_{0}^{\infty} \dstnbrofdmt \dmt^{2} \,\dfr\dmt
& = & \cncttl \dmtnma^{2} \exp ( 2 \lngsdsqr ) 
& = & \frac{\sfcttl}{\mpi}
& = & \cncttl \dmtsaa^2 \\[1.5ex]
\mmtfnc(3) & = & \int_{0}^{\infty} \dstnbrofdmt \dmt^{3} \,\dfr\dmt
& = & \cncttl \dmtnma^{3} \exp ( \mbox{$\frac{9}{2}$} \lngsdsqr ) 
& = & \frac{6 \vlmttl}{\mpi}
& = & \cncttl \dmtvaa^3 \\[1.5ex]
\mmtfnc(4) & = & \int_{0}^{\infty} \dstnbrofdmt \dmt^{4} \,\dfr\dmt
& = & \cncttl \dmtnma^{4} \exp ( 8 \lngsdsqr )
\end{array}
\label{eqn:dst_mmt_nsw_2}
\end{equation}
Table~\ref{tbl:lgn} includes these relations in slightly different forms.

The first few moments of the number distribution are
related to measurable properties of the size distribution.
In particular, $\mmtfnc(\mmtidx = 0)$ is the 
\trmidx{number concentration}.
Other quantities of merit are ratios of consecutive moments.
For example, the \trmidx{volume-weighted diameter} $\dmtvwa$ is
computed by weighted each diameter by the volume of particles at that  
diameter and then normalizing by the total volume of all particles.
\begin{eqnarray}
\dmtvwa & = &  
\int_{0}^{\infty} \dmt \frac{\mpi}{6} \dmt^{3} \dstnbrofdmt \,\dfr\dmt 
\bigg/ \int_{0}^{\infty} \frac{\mpi}{6} \dmt^{3} \dstnbrofdmt \,\dfr\dmt \nonumber \\
& = & \int_{0}^{\infty} \dmt^{4} \dstnbrofdmt \,\dfr\dmt \bigg/
\int_{0}^{\infty} \dmt^{3} \dstnbrofdmt \,\dfr\dmt \nonumber \\
& = & \mmtfnc(4) / \mmtfnc(3) \nonumber \\
& = & \frac{\cncttl \dmtnma^{4} \exp ( 8 \lngsdsqr )}
{\cncttl \dmtnma^{3} \exp ( \mbox{$\frac{9}{2}$} \lngsdsqr )} \nonumber \\
& = & \dmtnma \exp ( \mbox{$\frac{7}{2}$} \lngsdsqr )
\label{eqn:dmt_vwa_dfn}
\end{eqnarray}

The \trmdfn{surface-weighted diameter} $\dmtswa$ is defined
analogously to $\dmtvwa$.
$\dmtswa$ is more often known by its other name, the \trmidx{effective 
diameter} (twice the \trmidx{effective radius}).
The term ``effective'' refers to the light extinction properties of
the distribution.
Light impinging on a particle distribution is, in the limit of
\trmidx{geometric optics}, extinguished in proportion to the
cross-sectional area of the particles.
Hence the effective diameter (or radius) characterizes the extinction
properties of the distribution. 
Following (\ref{eqn:dmt_vwa_dfn}), the effective diameter of a
lognormal distribution is
\begin{eqnarray}
\dmtswa & = & \mmtfnc(3) / \mmtfnc(2) \nonumber \\
& = & \frac{\cncttl \dmtnma^{3} \exp ( \mbox{$\frac{9}{2}$} \lngsdsqr )} 
{\cncttl \dmtnma^{2} \exp (2 \lngsdsqr) } \nonumber \\
& = & \dmtnma \exp ( \mbox{$\frac{5}{2}$} \lngsdsqr )
\label{eqn:dmt_swa_dfn}
\end{eqnarray}

Moment-weighted diameters, such as the volume-weighted
diameter~$\dmtvwa$ (\ref{eqn:dmt_vwa_dfn}), characterize disperse
distributions.   
A disperse mass distribution $\dstmssofdmt$ behaves most like a
\trmidx{monodisperse distribution} with all mass residing at 
$\dmt = \dmtvwa$.  
Due to approximations, physical operators may be constrained to
act on a single, representative diameter rather than an entire 
distribution. 
The ``least-wrong'' diameter to pick is the moment-weighted diameter
most relevant to the process being modeled.
For example, $\dmtvwa$ best represents the 
\trmidx{gravitational sedimentation} of a distribution of particles.
On the other hand, $\dmtswa$ (\ref{eqn:dmt_swa_dfn}) best represents
the \trmidx{scattering cross-section} of a distribution of particles.

\subsubsection[Aspherical Particles]{Aspherical  Particles}\label{sxn:asp}
The useful relation (\ref{eqn:dst_mmt_nsw}) is a property of the
lognormal distribution itself, rather than the particle shape.
A lognormal distribution of aspherical particles also obeys
(\ref{eqn:dst_mmt_nsw}). 
Important measurable properties of most convex aspherical habits 
may be represented by a constant times the $\mmtidx^{\mathrm{th}}$
moment $\mmtfnc(\mmtidx)$ of the distribution. 
For example, the surface area $\sfchxg$\,[\mS] and volume
$\vlmhxg$\,[\mC] of hexagonal prisms are given by
(\ref{eqn:sfc_hxg_dfn})--(\ref{eqn:vlm_hxg_asp_dfn})\footnote{
My \LaTeX\ skills are too poor to reference equations in other
documents. 
All ``undefined'' analytic formula below are in the aerosol FACT, 
\flprn{aer.pdf}, and that's why their equation numbers appear as
question marks in this FACT. 
20150303: Some equations are now in a new appendix of this 
document.
Look in \flprn{aer.pdf} to see the other equations.}. 
 
To be consistent with the diameter-centric expressions in
Table~\ref{tbl:lgn}, we introduce $\dmt_{\hxgsbs}$, the 
hexagonal prism diameter.
Adopting the convention that $\dmt_{\hxgsbs} \equiv 2\rdshxg$, the
full-width of the basal face, we obtain
\begin{eqnarray}
\label{eqn:sfc_hxg_dmt_dfn}
\sfchxg & = & \left( \frac{3\sqrt{3}}{4} + 3\asphxg \right) \dmt_{\hxgsbs}^{2} \\
\label{eqn:vlm_hxg_dmt_dfn}
\vlmhxg & = & \frac{3\sqrt{3}}{8} \asphxg \dmt_{\hxgsbs}^{3}
\end{eqnarray}

The functional forms for $\sfchxg$ and $\vlmhxg$ consist of constants
multiplying the diameter's second and third moments, respectively. 
The surface area ($\mpi\dmt^{2}$) and volume ($\mpi\dmt^{3}/6$) of
spheres have the same form.  
Therefore the higher moments of aspherical particle distributions
must be the same as spherical particle distributions modulo the
leading constant expressions.
Inserting $\sfchxg$ and $\vlmhxg$ into (\ref{eqn:sfc_hxg_dmt_dfn}),
(\ref{eqn:vlm_hxg_dmt_dfn}), and (\ref{eqn:dst_mmt_nsw}) leads
to analytic expressions for the total surface area
$\sfcttlhxg$\,[\mSxmC] and volume $\vlmttlhxg$\,[\mCxmC] of a 
lognormal distribution of hexagonal prisms: 
\begin{eqnarray}
\label{eqn:sfc_ttl_hxg_asp_dfn}
\sfcttlhxg & = & \left( 3\asphxg+\frac{3\sqrt{3}}{4} \right) \dmtnmahxg^{2} \exp(2\gsdtldsqr) \\
\label{eqn:vlm_ttl_hxg_asp_dfn}
\vlmttlhxg & = & \frac{3\sqrt{3}}{8} \asphxg\dmtnmahxg^{3} \exp(9\gsdtldsqr/2)
\end{eqnarray}

The total concentration $\nbrttlvts$ of equivalent V/S-spheres
corresponding to a known distribution of hexagonal prisms must be
computed numerically unless the size dependence of the aspect ratio
$\asphxg(\dmt)$ takes an analytic form.
In the simplest case, one can imagine or assume distributions of
hexagonal prisms with constant aspect ratio, i.e, 
$\asphxg \ne \asphxg(\dmt)$. 
In this idealized case, the ratio $\cncvts/\cnchxg$
(\ref{eqn:cnc_vts_hxg_dfn}) is constant throughout the distribution.
Then the analytic number concentration of equivalent V/S-spheres
is simply $\cncvts/\cnchxg$ times the analytic number concentration of 
hexagonal prisms which is presumably known directly from the lognormal
size distribution parameters (cf.\ Table~\ref{tbl:lgn}).

\subsubsection[Normalization]{Normalization}\label{sxn:nrm}
We show that (\ref{eqn:dst_lgn}) is normalized by considering
\begin{equation}
\dstnbrofdmt = \frac{\cstnrm}{\dmt} \exp\! 
\left[-\frac{1}{2}\left( \frac{ \ln(\dmt/\dmtnma)}{\lngsd} \right)^{2} \right]
%\label{eqn:dst_lgn_unn}
\end{equation}
where $\cstnrm$ is the \trmidx{normalization constant} determined by 
(\ref{eqn:pdf_nrm}).
First we change variables to $\yyy = \ln(\dmt/\dmtnma)$
\begin{eqnarray}
\yyy & = & \ln \dmt - \ln \dmtnma \nonumber \\ 
\dmt & = & \dmtnma \me^\yyy \nonumber \\
\dfr\yyy & = & \dmt^{-1} \,\dfr\dmt \nonumber \\ 
\dfr\dmt & = & \dmtnma \me^\yyy \,\dfr\yyy
\label{eqn:cov_y}
\end{eqnarray}
This transformation maps $\dmt \in (0, +\infty)$ into $\yyy \in (-\infty,
+\infty)$.
In terms of $\yyy$, the normalization condition (\ref{eqn:pdf_nrm}) 
becomes
\begin{eqnarray*}
\int_{-\infty}^{+\infty}\frac{\cstnrm}{\dmtnma \exp \yyy} \exp\! 
\left[-\frac{1}{2}\left( \frac{\yyy}{\lngsd} \right)^{2} \right]
\dmtnma \exp^\yyy \,\dfr \yyy & = & 1 \\
\int_{-\infty}^{+\infty} \cstnrm \exp\! 
\left[-\frac{1}{2}\left( \frac{\yyy}{\lngsd} \right)^{2} \right]
\,\dfr\yyy & = & 1 \\
\end{eqnarray*}
Next we change variables to $\zzz = \yyy/\lngsd$
\begin{eqnarray}
\zzz & = & \yyy/\lngsd \nonumber \\ 
\yyy & = & \zzz \lngsd \nonumber \\
\dfr\zzz & = & (\lngsd)^{-1} \,\dfr\yyy \nonumber \\ 
\dfr\yyy & = & \lngsd \,\dfr\zzz
\end{eqnarray}
This transformation does not change the limits of integration and we
obtain 
\begin{eqnarray}
\int_{-\infty}^{+\infty} \cstnrm \exp\! \left( \frac{-\zzz^{2}}{2} \right) 
\lngsd \,\dfr\zzz & = & 1 \nonumber \\
\cstnrm \, \sqrt{2 \mpi} \lngsd & = & 1 \nonumber \\
\cstnrm & = & \frac{1}{\sqrt{2 \mpi} \lngsd}
%\label{eqn:}
\end{eqnarray}
In the above we used the well-known normalization property of the
Gaussian distribution function, 
$\int_{-\infty}^{+\infty} \me^{-\xxx^{2}/2} \,\dfr\xxx =
\sqrt{2 \mpi}$ (\ref{eqn:nrm_gss}). 

\section{Implementation in NCAR models}

The discussion thus far has centered on the theoretical considerations
of size distributions.
In practice, these ideas must be implemented in computer codes which
model, e.g., Mie scattering parameters or thermodynamic growth of
aerosol populations.
This section describes how these ideas have been implemented in the 
NCAR-Dust and Mie models.

\subsection[NCAR-Dust Model]{NCAR-Dust Model}
The NCAR-Dust model uses as input a time invariant dataset of surface
soil size distribution.
The two such datasets currently used are from \cite{WRL93} and from
IBIS \cite[]{Fol98}.
The \citeauthor{WRL93} dataset provides global information for three
soil texture types: sand, clay and silt.
At each gridpoint, the mass flux of dust is partitioned into mass
contributions from each of these soil types.
To accomplish this, the partitioning scheme assumes a size
distribution for the source soil of the deflated particles.
\begin{table}
\begin{center}
\begin{tabular}{lllp{10em}}
\hline
Soil Texture & $\dmtnma$ & $\gsd$ & Description \\[0.0ex]
\hline
& & & \\[0.0ex]
Sand & & & Sand \\[0.5ex]
Silt & & & Silt \\[0.5ex]
Clay & & & Clay \\[0.5ex]
\hline
Soil Texture & $\dmtnma$ & $\gsd$ & Description \\[0.0ex]
\hline
Sand & & & Sand \\[0.5ex]
Silt & & & Silt \\[0.5ex]
Clay & & & Clay \\[0.5ex]
\hline
\end{tabular}
\caption[Source Size Distribution]{Source size distribution associated
with surface soil texture data of \cite{WRL93} and of \cite{Fol98}.
\label{tbl:WRL93}}   
\end{center}
\end{table}
Table~\ref{tbl:WRL93} lists the lognormal distribution parameters
associated with the surface soil texture data of \cite{WRL93} and of
\cite{Fol98}. 
The dust model is a size resolving aerosol model. 
Thus, overlap factors are computed to determine the fraction of each
parent size type which is mobilized into each atmospheric dust size
bin during a deflation event.

\clearpage
\subsection[Mie Scattering Model]{Mie Scattering Model}\label{sxn:mie}
This section documents the Mie scattering code \cmdidx{mie}. 
\cmdprn{mie} is box model intended to provide exact simulations of
microphysical processes for the purpose of parameterization into
larger scale models.
\cmdprn{mie} provides instantaneous and equilibrium decriptions of
many processes ranging from surface flux exchange, dust production,
reflection of polarized radiation, and, as its name suggests, the
interaction of particles and radiation.
Thus the inputs to \cmdprn{mie} are the instantaneous state (boundary
and initial conditions) of the environment.
Given these, the program solves for the associated rates of change and
unknown variables.

There is no time-stepping loop primarily because \cmdprn{mie}
generates an extraordinary amount of information about the
instantaneous state.
Time-stepping this environment in a box-model-like format would be
prohibitive if all quantities were allowed to evolve.

\subsubsection[Input switches]{Input switches}\label{sxn:cmd_ln_mie}
The flexibility and power of \cmdprn{mie} can only be exercised by
actively using the hundreds of \trmidx{input switches} which control
its behavior.
This section describes how some of these switches are commonly used
to control fundamental properties of the microphysical environment. 
A complete reference table for these switches, there default values,
and dimensional units, is presented in Appendix~\ref{sxn:apn_cmd_ln_mie}.  

The heart of \cmdprn{mie} is an aerosol size distribution.
Most users will wish to initialize this size distribution to a
particular type of aerosol, and to a particular shape.
This is accomplished with the \cmdidx{cmp\_aer} and \cmdidx{psd\_typ}
keywords. 
The linearity, range, and resolution of the grid on which the analytic
size distribution is discretized are controlled by the
\cmdidx{sz\_grd}, \cmdidx{sz\_mnm}, \cmdidx{sz\_mxm}, \cmdidx{sz\_nbr}
switches, respectively.
Compute size distribution characteristics of a lognormal distribution
\begin{verbatim}
mie -dbg -no_mie --psd_typ=lognormal --sz_grd=log --sz_mnm=0.01 \
--sz_mxm=10.0 --sz_nbr=300 --rds_nma=0.4 --gsd_anl=2.2
mie -dbg -no_mie --psd_typ=lognormal --sz_grd=log --sz_mnm=1.0 \
--sz_mxm=10.0 --sz_nbr=25 --rds_nma=2.0 --gsd_anl=2.2
\end{verbatim}

\subsubsection[Moments of Size Distribution]{Moments of Size Distribution}\label{sxn:mie_mmn}
Determine the analytic (or resolved) moments of an arbitrary size
distribution.
\begin{enumerate}
\item Generate the size distribution. (It may have more than one moment)
\item Select the statistics of interest
\end{enumerate}
\begin{verbatim}
# 1. Lognormal distribution with mass median diameter 3.5 um, GSD = 2.0
mie -no_mie --psd_typ=lognormal --sz_grd=log --sz_nbr=1000 \
--sz_mnm=0.005 --sz_mxm=50.0 --dmt_vma=3.5 --gsd_anl=2.0
# 2. Extract median and weighted analytic moments of diameter
ncks -H -v dmt_vwa,dmt_vma,dmt_swa,dmt_sma,dmt_nwa,dmt_nma ${DATA}/mie/mie.nc
# 3. Extract median and weighted resolved moments of diameter
ncks -H -v dmt_vwr,dmt_vmr,dmt_swr,dmt_smr,dmt_nwr,dmt_nmr ${DATA}/mie/mie.nc
# 4. Extract median and weighted analytic moments of diameter
ncks -H -v rds_vwa,rds_vma,rds_swa,rds_sma,rds_nwa,rds_nma ${DATA}/mie/mie.nc
# 5. Extract median and weighted resolved moments of diameter
ncks -H -v rds_vwr,rds_vmr,rds_swr,rds_smr,rds_nwr,rds_nmr ${DATA}/mie/mie.nc
# 6. Extract number, surface area, and volume distributions at specific sizes
ncks -H -C -F -u -v dst,dst_rds,dst_sfc,dst_vlm -d sz,1.0e-6 ${DATA}/mie/mie.nc
\end{verbatim}

\subsubsection[Generating Properties for Multi-Bin Distributions]{Generating Properties for Multi-Bin}\label{sxn:mie_bin}
On occasion, a seriouly masochistic scientist will decide to create
the ultimate hybrid bin-spectral aerosol method by discretizing the
size distribution into a finite number of bins each with an
independently configurable analytic sub-bin distribution.
Generating properties for all the bins in such a scheme requires
enormous amounts of bookkeeping, or, if a computer is available,
a relatively simple \trmidx{Perl} batch script named \cmdidx{psd.pl}. 

The \cmdprn{psd.pl} batch script calls \cmdprn{mie} repeatedly in a
loop over particle bin.
As input, \cmdprn{psd.pl} accepts concise array representations of
each property of a bin.
For example, \verb'--sz_nbr={200,25,25,25}' specifies that bin~1 is
discretized into 200~sub-bins, and the remaining three bins are each
discretized into only 25~sub-bins.
\begin{verbatim}
~/dst/psd.pl --dbg=1 --CCM_SW --ftn_fxd --psd_nbr=4 --spc_idx_sng={01,02,03,04} \
--sz_mnm={0.05,0.5,1.25,2.5} --sz_mxm={0.5,1.25,2.5,5.0} --sz_nbr={200,25,25,25} \
--dmt_vma_dfl=3.5 > ${DATA}/dst/mie/psd_CCM_SW.txt.v3 2>&1 &
\end{verbatim}

\section{Appendix}\label{sxn:apn}

\subsection[Properties of Gaussians]{Properties of Gaussians}\label{sxn:gss}
The area under a Gaussian distribution may be expressed in closed 
form for infinite domains.
This result can be obtained (IIRC) by transforming to polar
coordinates in the complex plane 
$\xxx = \rds (\cos \theta + \mi \sin \theta)$. 
\begin{equation}
\int_{-\infty}^{+\infty} \me^{-\xxx^{2}/2} \,\dfr\xxx =
\sqrt{2 \mpi}
\label{eqn:nrm_gss}
\end{equation}
This is a special case of a more general result 
\begin{equation}
\int_{-\infty}^{+\infty} \exp (-\alpha \xxx^{2} -\beta \xxx) \,\dfr\xxx = 
\sqrt{\frac{\mpi}{\alpha}} \exp\! \left( \frac{\beta^{2}}{4\alpha} \right) 
\qquad \mathrm{where}\ \alpha > 0
\label{eqn:nrm_gss_gnr}
\end{equation}
To obtain this result, complete the square under the integrand, 
change variables to $\yyy = \xxx + \beta/2\alpha$, 
and then apply (\ref{eqn:nrm_gss}).
Substituting $\alpha = 1/2$ and $\beta = 0$ into
(\ref{eqn:nrm_gss_gnr}) yields (\ref{eqn:nrm_gss}).

\subsection[Error Function]{Error Function}\label{sxn:erf}
The \trmdfn{error function} $\erfxxx$ may be defined as the partial
integral of a Gaussian curve
\begin{equation}
% http://mathworld.wolfram.com/Erf.html
\erfzzz = \frac{2}{\sqrt{\mpi}} \int_{0}^{\zzz} \me^{-\xxx^{2}} \,\dfr\xxx
\label{eqn:erf}
\end{equation}
Using (\ref{eqn:nrm_gss}) and the symmetry of a Gaussian curve, it is
simple to show that the error function is bounded by the limits
$\erffnc(0) = 0$ and $\erffnc(\infty) = 1$.
Thus $\erfzzz$ is the cumulative probability function for a
normally distributed variable $\zzz$ (fxm: True??).
Most compilers implement $\erfxxx$ as an intrinsic function.
Thus $\erfxxx$ is used to compute areas bounded by finite lognormal
distributions (\S\ref{sxn:lgn_bnd}). 

\subsection[Properties of Binomials]{Properties of Binomials}\label{sxn:bnm}
The number $\nbrmlt$ of trials that result in a given outcome 
$\tcmnbr$ times in $\trlnbr$ total trials is
\begin{equation}
% https://en.wikipedia.org/wiki/Bernoulli_process
\nbrmlt(\tcmnbr,\trlnbr) = 
{{\trlnbr}\choose{\tcmnbr}} =
\frac{\trlnbr!}{\tcmnbr! (\trlnbr - \tcmnbr)!}
\label{eqn:bnm_cff_dfn}
\end{equation}
$\nbrmlt(\tcmnbr,\trlnbr)$ is known as the \trmdfn{binomial coefficient}.

If the probability of the outcome of a single independent trial
(e.g., flipping a coin) is~$\prbone$, then the probability $\prbmlt$
of that outcome occurring $\tcmnbr$~times in $\trlnbr$~trials is 
\begin{eqnarray}
% https://en.wikipedia.org/wiki/Bernoulli_process
\prbmlt(\tcmnbr,\trlnbr) & = & 
\nbrmlt(\tcmnbr,\trlnbr) \prbone^{\tcmnbr} (1 - \tcmnbr)^{(\trlnbr-\tcmnbr)} \nonumber \\
& = & 
{{\trlnbr}\choose{\tcmnbr}} \prbone^{\tcmnbr} (1 - \tcmnbr)^{(\trlnbr-\tcmnbr)} \nonumber \\
& = & 
\frac{\trlnbr!}{\tcmnbr! (\trlnbr - \tcmnbr)!} \, \prbone^{\tcmnbr} (1 - \tcmnbr)^{(\trlnbr-\tcmnbr)}
\label{eqn:bnm_dst_dfn}
\end{eqnarray}
$\prbmlt(\tcmnbr,\trlnbr)$ is known as the \trmdfn{binomial distribution}.

The outcome of the binomial distribution is difficult to compute
exactly for large~$\trlnbr$ because evaulating the factorial function 
for large numbers is numerically unwieldy.
Instead we make use of \trmdfn{Stirling's Approximation}:
\begin{equation}
% https://en.wikipedia.org/wiki/Bernoulli_process
\nnn! = \sqrt{2\mpi\nnn} \, \nnn^{\nnn} \me^{-\nnn} 
\left( 1 + \mathcal{O} \left( \frac{1}{\nnn} \right) \right)
\label{eqn:str_apx_dfn}
\end{equation}

Substituting (\ref{eqn:str_apx_dfn}) in (\ref{eqn:bnm_dst_dfn}) we
obtain 
\begin{eqnarray}
% https://en.wikipedia.org/wiki/Bernoulli_process
\prbmlt(\tcmnbr,\trlnbr) & = & 
\frac{ \sqrt{2\mpi\nnn} \, \nnn^{\nnn} \me^{-\nnn} }
     { \sqrt{2\mpi\kkk} \, \kkk^{\kkk} \me^{-\kkk} 
       \sqrt{2\mpi(\nnn-\kkk)} \, (\nnn-\kkk)^{(\nnn-\kkk)} \me^{-(\nnn-\kkk)} }
       \, \prbone^{\tcmnbr} (1 - \tcmnbr)^{(\trlnbr-\tcmnbr)} \nonumber \\
 & = & 
\sqrt{ \frac{\nnn}{2\mpi\kkk(\nnn-\kkk)} } \,
\frac{ \nnn^{\nnn} \me^{-\nnn} }
     { \kkk^{\kkk} \me^{-\kkk} \, (\nnn-\kkk)^{(\nnn-\kkk)} \me^{-(\nnn-\kkk)} }
       \, \prbone^{\tcmnbr} (1 - \tcmnbr)^{(\trlnbr-\tcmnbr)} \\
\label{eqn:bnm_dst_apx}
\end{eqnarray}

\subsection[Formulae from other FACTs]{Formulae from other FACTs}\label{sxn:fct}
% This paragraph stolen from master text in aer.tex:
The surface area $\sfchxg$\,[\mS] and volume $\vlmhxg$\,[\mC] of a 
hexagonal prism are
\begin{eqnarray}
% GrW99 p. 31699 (3)
\sfchxg & = & 6\rdshxg\dmthxg+3\sqrt{3}\rdshxg^{2} = 
              3\rdshxg(2\dmthxg+\sqrt{3}\rdshxg) \\
\label{eqn:sfc_hxg_dfn}
% GrW99 p. 31699 (3) NGW03 p. 3 (4)
\vlmhxg & = & \frac{3\sqrt{3}\rdshxg^{2}\dmthxg}{2}
\label{eqn:vlm_hxg_dfn}
\end{eqnarray}
These definitions may be re-cast in terms of $\rdshxg$ and $\asphxg$
by using $\dmthxg = 2\rdshxg\asphxg$ from (\ref{eqn:asp_hxg_dfn})  
\begin{eqnarray}
\sfchxg & = & 12\rdshxg^{2}\asphxg+3\sqrt{3}\rdshxg^{2} = 
              3\rdshxg^{2}(4\asphxg+\sqrt{3}) \\
\label{eqn:sfc_hxg_asp_dfn}
% NGW03 p. 3 (6c)
\vlmhxg & = & 3\sqrt{3}\rdshxg^{3}\asphxg
\label{eqn:vlm_hxg_asp_dfn}
\end{eqnarray}
It is more convenient to describe hexagons in terms
of ($\rdshxg$,$\asphxg$) than ($\rdshxg$,$\dmthxg$).
This is because natural hexagonal prisms may share similar aspect
ratios over a large range of sizes.

\subsection[Command Line Switches for \texttt{mie} Code]{Command Line Switches for \texttt{mie} Code}\label{sxn:apn_cmd_ln_mie}
Table~\ref{tbl:cmd_ln_mie} summarizes all of the \trmdfn{command line arguments} 
available to control the behavior of the \cmdidx{mie} program. 
This is a summary only---it is impractical to think that written 
documentation could ever convey the exact meaning of all the
switches\footnote{Perhaps the most useful way to begin to contribute
to this \href{http://dust.ess.uci.edu/facts}{FACT} would be to
systematize and extend the documentation of command line switches}.
The most frequently used switches are described in
Section~\ref{sxn:cmd_ln_mie}. 
The only way to learn the full meaning of the more obscure switches is
to read the source code itself.
\begin{landscape} % Begin tbl:cmd_ln_mie
\begin{longtable}{ >{\ttfamily}l<{} >{\raggedright}p{20.0em}<{} r l}
& & & \kill % NB: longtable requires caption as table entry
\caption[Command Line Switches]{\textbf{Command Line Switches for \texttt{mie} code}%
\footnote{\emph{Source:} \cite{ZBP97}}%
\footnote{\cmdprn{cmd\_ln\_dfl} is a standard large value, e.g., 
$1.0 \times 10^{36}$. 
\cmdprn{mie} checks for this value for particular variables
whose treatment depends on whether or not the variable was
user-initialized.}%
\label{tbl:cmd_ln_mie}} \\
\hline\hline \rule{0.0ex}{\hlntblhdrskp}% 
\textrm{Switch} & Purpose & Default & Units \\[0.0ex]
\hline \rule{0.0ex}{\hlntblntrskp}%
\endfirsthead % Lines between and \endfirsthead appear at top of table
\caption[]{(continued)} \\ % Set label for following pages
\textrm{Switch} & Purpose & Default & Units \\[0.0ex]
\hline \rule{0.0ex}{\hlntblntrskp}%
\endhead % Previous block appears at top of every page
\endlastfoot % Previous block appears at end of table
\multicolumn{4}{c}{Boolean flags\rule[-0.5ex]{0ex}{1.5ex}} \\[0.0ex]
--\cmdidx{abc\_flg} & Alphabetize output with \cmdidx{ncks} & true & Flag \\[0.5ex]
--\cmdidx{abs\_ncl\_wk\_mdm\_flg} & Absorbing inclusion in weakly-absorbing sphere & false & Flag \\[0.5ex]
--\cmdidx{bch\_flg} & Batch behavior & false & Flag \\[0.5ex]
--\cmdidx{coat\_flg} & Assume coated spheres & false & Flag \\[0.5ex]
--\cmdidx{drv\_rds\_nma\_flg} & Derive rds\_nma from bin boundaries & false & Flag \\[0.5ex]
--\cmdidx{fdg\_flg} & Tune the extinction of a particular band & false & Flag \\[0.5ex]
--\cmdidx{hxg\_flg} & Aspherical particles are hexagonal prisms & true & Flag \\[0.5ex]
--\cmdidx{vts\_flg} & Apply equal-V/S approximation for aspherical optical properties & false & Flag \\[0.5ex]
--\cmdidx{ftn\_fxd\_flg} & Fortran fixed format & false & Flag \\[0.5ex]
--\cmdidx{hrz\_flg} & Print size-resolved optical properties at debug wavelength & false & Flag \\[0.5ex]
--\cmdidx{mca\_flg} & Multi-component aerosol with effective medium approximation & false & Flag \\[0.5ex]
--\cmdidx{mie\_flg} & Perform Mie scattering calculation & true & Flag \\[0.5ex]
--\cmdidx{no\_abc\_flg} & Set \texttt{abc\_flg} to \texttt{false} & & Flag \\[0.5ex]
--\cmdidx{no\_bch\_flg} & Set \texttt{bch\_flg} to \texttt{false} & & Flag \\[0.5ex]
--\cmdidx{no\_hrz\_flg} & Set \texttt{hrz\_flg} to \texttt{false} & & Flag \\[0.5ex]
--\cmdidx{no\_mie\_flg} & Set \texttt{mie\_flg} to \texttt{false} & & Flag \\[0.5ex]
--\cmdidx{no\_wrn\_ntp\_flg} & Set \texttt{wrn\_ntp\_flg} to \texttt{false} & & Flag \\[0.5ex]
--\cmdidx{wrn\_ntp\_flg} & Print WARNINGs from \texttt{ntp\_vec()} & true & Flag \\[0.5ex]

\multicolumn{4}{c}{Variables\rule[-0.5ex]{0ex}{1.5ex}} \\[0.0ex]
--\cmdidx{RH\_lqd} & Relative humidity w/r/t liquid water & 0.8 & Fraction \\[0.5ex]
--\cmdidx{asp\_rat\_hxg\_dfl} & Hexagonal prism aspect ratio & 1.0 & Fraction \\[0.5ex]
--\cmdidx{asp\_rat\_lps\_dfl} & Ellipsoidal aspect ratio & 1.0 & Fraction \\[0.5ex]
--\cmdidx{bnd\_SW\_LW} & Boundary between SW and LW weighting & $5.0 \times 10^{-6}$ & \m \\[0.5ex]
--\cmdidx{bnd\_nbr} & Number of sub-bands per output band & 1 & Number \\[0.5ex]
--\cmdidx{cmp\_cor} & Composition of core & ``air'' & String \\[0.5ex]
--\cmdidx{cmp\_mdm} & Composition of medium & ``air'' & String \\[0.5ex]
--\cmdidx{cmp\_mnt} & Composition of mantle & ``air'' & String \\[0.5ex]
--\cmdidx{cmp\_mtx} & Composition of matrix & ``air'' & String \\[0.5ex]
--\cmdidx{cmp\_ncl} & Composition of inclusion & ``air'' & String \\[0.5ex]
--\cmdidx{cmp\_prt} & Composition of particle & ``saharan\_dust'' & String \\[0.5ex]
--\cmdidx{cnc\_nbr\_anl\_dfl} & Number concentration analytic, default & 1.0 & \nbrxmC \\[0.5ex]
--\cmdidx{cnc\_nbr\_pcp\_anl} & Number concentration analytic, raindrop & 1.0 & \nbrxmC \\[0.5ex]
--\cmdidx{cpv\_foo} & Intrinsic computational precision temporary variable & 0.0 & Fraction \\[0.5ex]
--\cmdidx{dbg\_lvl} & Debugging level & 0 & Index \\[0.5ex]
--\cmdidx{dmn\_nbr\_max} & Maximum number of dimensions allowed in single variable in output file & 2 & Number \\[0.5ex]
--\cmdidx{dmn\_frc} & Fractal dimensionality of inclusions & 3.0 & Fraction \\[0.5ex]
--\cmdidx{dmn\_rcd} & Record dimension name & ``'' & String \\[0.5ex]
--\cmdidx{dmt\_dtc} & Diameter of detector & 0.001 & \m \\[0.5ex]
--\cmdidx{dmt\_nma\_mcr} & Number median analytic diameter & \texttt{cmd\_ln\_dfl} & \um \\[0.5ex]
--\cmdidx{dmt\_pcp\_nma\_mcr} & Diameter number median analytic, raindrop, microns & 1000.0 & \um \\[0.5ex]
--\cmdidx{dmt\_swa\_mcr} & Surface area weighted mean diameter analytic & \texttt{cmd\_ln\_dfl} & \um \\[0.5ex]
--\cmdidx{dmt\_vma\_mcr} & Volume median diameter analytic & \texttt{cmd\_ln\_dfl} & \um \\[0.5ex]
--\cmdidx{dns\_cor} & Density of core & 0.0 & \kgxmC \\[0.5ex]
--\cmdidx{dns\_mdm} & Density of medium & 0.0 & \kgxmC \\[0.5ex]
--\cmdidx{dns\_mnt} & Density of mantle & 0.0 & \kgxmC \\[0.5ex]
--\cmdidx{dns\_mtx} & Density of matrix & 0.0 & \kgxmC \\[0.5ex]
--\cmdidx{dns\_ncl} & Density of inclusion & 0.0 & \kgxmC \\[0.5ex]
--\cmdidx{dns\_prt} & Density of particle & 0.0 & \kgxmC \\[0.5ex]
--\cmdidx{doy} & Day of year [1.0..367.0) & 135.0 & day \\[0.5ex]
--\cmdidx{drc\_dat} & Data directory & \texttt{/data/zender/aca} & String \\[0.5ex]
--\cmdidx{drc\_in} & Input directory & \texttt{\$\{HOME\}/nco/data} & String \\[0.5ex]
--\cmdidx{drc\_out} & Output directory & \texttt{\$\{HOME\}/c++} & String \\[0.5ex]
--\cmdidx{dsd\_dbg\_mcr} & Debugging size for raindrops & 1000.0 & \um \\[0.5ex]
--\cmdidx{dsd\_mnm\_mcr} & Minimum diameter in raindrop distribution & 999.0 & \um \\[0.5ex]
--\cmdidx{dsd\_mxm\_mcr} & Maximum diameter in raindrop distribution & 1001.0 & \um \\[0.5ex]
--\cmdidx{dsd\_nbr} & Number of raindrop size bins & 1 & Number \\[0.5ex]
--\cmdidx{fdg\_idx} & Band to tune by fdg\_val & 0 & Index \\[0.5ex]
--\cmdidx{fdg\_val} & Tuning factor for all bands & 1.0 & Fraction \\[0.5ex]
--\cmdidx{fl\_err} & File for error messages & ``cerr'' & String \\[0.5ex]
--\cmdidx{fl\_idx\_rfr\_cor} & File or function for refractive indices of core & ``'' & String \\[0.5ex]
--\cmdidx{fl\_idx\_rfr\_mdm} & File or function for refractive indices of medium & ``'' & String \\[0.5ex]
--\cmdidx{fl\_idx\_rfr\_mnt} & File or function for refractive indices of mantle & ``'' & String \\[0.5ex]
--\cmdidx{fl\_idx\_rfr\_mtx} & File or function for refractive indices of matrix & ``'' & String \\[0.5ex]
--\cmdidx{fl\_idx\_rfr\_ncl} & File or function for refractive indices of inclusion & ``'' & String \\[0.5ex]
--\cmdidx{fl\_idx\_rfr\_prt} & File or function for refractive indices of particle & ``'' & String \\[0.5ex]
--\cmdidx{fl\_slr\_spc} & File or function for solar spectrum & ``'' & String \\[0.5ex]
--\cmdidx{flt\_foo} & Intrinsic float temporary variable & 0.0 & Fraction \\[0.5ex]
--\cmdidx{flx\_LW\_dwn\_sfc} & Longwave downwelling flux at surface & 0.0 & \wxmS \\[0.5ex]
--\cmdidx{flx\_SW\_net\_gnd} & Solar flux absorbed by ground & 450.0 & \wxmS \\[0.5ex]
--\cmdidx{flx\_SW\_net\_vgt} & Solar flux absorbed by vegetation & 0.0 & \wxmS \\[0.5ex]
--\cmdidx{flx\_frc\_drc\_sfc\_cmd\_ln} & Surface insolation fraction in direct beam & 0.85 & Fraction \\[0.5ex]
--\cmdidx{flx\_vlm\_pcp\_rsl} & Precipitation volume flux, resolved & $-1.0$ & \mCxmSs \\[0.5ex]
--\cmdidx{gsd\_anl\_dfl} & Geometric standard deviation, default & 2.0 & Fraction \\[0.5ex]
--\cmdidx{gsd\_pcp\_anl} & Geometric standard deviation, raindrop & 1.86 & Fraction \\[0.5ex]
--\cmdidx{hgt\_mdp} & Midlayer height above surface & 95.0 & \m \\[0.5ex]
--\cmdidx{hgt\_rfr} & Reference height (i.e., 10 m) at which surface winds are evaluated for dust mobilization & 10.0 & \m \\[0.5ex]
--\cmdidx{hgt\_zpd\_dps\_cmd\_ln} & Zero plane displacement height & \texttt{cmd\_ln\_dfl} & \m \\[0.5ex]
--\cmdidx{hgt\_zpd\_mbl} & Zero plane displacement height for erodible surfaces & 0.0 & \m \\[0.5ex]
--\cmdidx{idx\_rfr\_cor\_usr} & Refractive index of core & $1.0+0.0\mi$ & Complex \\[0.5ex]
--\cmdidx{idx\_rfr\_mdm\_usr} & Refractive index of medium & $1.0+0.0\mi$ & Complex \\[0.5ex]
--\cmdidx{idx\_rfr\_mnt\_usr} & Refractive index of mantle & $1.33+0.0\mi$ & Complex \\[0.5ex]
--\cmdidx{idx\_rfr\_mtx\_usr} & Refractive index of matrix & $1.0+0.0\mi$ & Complex \\[0.5ex]
--\cmdidx{idx\_rfr\_ncl\_usr} & Refractive index of inclusion & $1.0+0.0\mi$ & Complex \\[0.5ex]
--\cmdidx{idx\_rfr\_prt\_usr} & Refractive index of particle & $1.33+0.0\mi$ & Complex \\[0.5ex]
--\cmdidx{lat\_dgr} & Latitude & 40.0 & \dgr \\[0.5ex]
--\cmdidx{lbl\_sng} & Line-by-line test & ``CO2'' & String \\[0.5ex]
--\cmdidx{lgn\_nbr} & Number of terms in \trmidx{Legendre expansion} of phase function & 8 & Number \\[0.5ex]
--\cmdidx{lnd\_frc\_dry} & Dry land fraction & 1.0 & Fraction \\[0.5ex]
--\cmdidx{mmw\_prt} & Mean molecular weight & 0.0 & \kgxmol \\[0.5ex]
--\cmdidx{mno\_lng\_dps\_cmd\_ln} & Monin-Obukhov length & \texttt{cmd\_ln\_dfl} & \m \\[0.5ex]
--\cmdidx{mss\_frc\_cly} & Mass fraction clay  & 0.19 & Fraction \\[0.5ex]
--\cmdidx{mss\_frc\_snd} & Mass fraction sand & 0.777 & Fraction \\[0.5ex]
--\cmdidx{ngl\_nbr} & Number of angles in Mie computation & 11 & Number \\[0.5ex]
--\cmdidx{oro} & Orography: ocean=0.0, land=1.0, sea ice=2.0 & 1.0 & Fraction \\[0.5ex]
--\cmdidx{pnt\_typ\_idx} & Plant type index & 14 & Index \\[0.5ex]
--\cmdidx{prs\_mdp} & Environmental pressure & 100825.0 & Pa \\[0.5ex]
--\cmdidx{prs\_ntf} & Environmental surface pressure & \texttt{prs\_STP} & Pa \\[0.5ex]
--\cmdidx{psd\_typ} & Particle size distribution type & ``lognormal'' & String \\[0.5ex]
--\cmdidx{q\_H2O\_vpr} & Specific humidity & \texttt{cmd\_ln\_dfl} & \kgxkg \\[0.5ex]
--\cmdidx{rds\_ffc\_gmm\_mcr} & Effective radius of Gamma distribution & 50.0 & \um \\[0.5ex]
--\cmdidx{rds\_nma\_mcr} & Number median analytic radius & 0.2986 & \um \\[0.5ex]
--\cmdidx{rds\_swa\_mcr} & Surface area weighted mean radius analytic & \texttt{cmd\_ln\_dfl} & \um \\[0.5ex]
--\cmdidx{rds\_vma\_mcr} & Volume median radius analytic & \texttt{cmd\_ln\_dfl} & \um \\[0.5ex]
--\cmdidx{rgh\_mmn\_dps\_cmd\_ln} & Roughness length momentum & \texttt{cmd\_ln\_dfl} & \m \\[0.5ex]
--\cmdidx{rgh\_mmn\_ice\_std} & Roughness length over sea ice & 0.0005 & \m \\[0.5ex]
--\cmdidx{rgh\_mmn\_mbl} & Roughness length momentum for erodible surfaces & $100.0 \times 10^{-6}$ & \m \\[0.5ex]
--\cmdidx{rgh\_mmn\_smt} & Smooth roughness length & $10.0 \times 10^{-6}$ & \m \\[0.5ex]
--\cmdidx{rfl\_gnd\_dff} & Diffuse reflectance of ground (beneath snow) & 0.20 & Fraction \\[0.5ex]
--\cmdidx{sfc\_typ} & LSM surface type [0..28] & 2 & Index \\[0.5ex]
--\cmdidx{slf\_tst\_typ} & Self-test type & ``BoH83'' & String \\[0.5ex]
--\cmdidx{slr\_cst} & Solar constant & 1367.0 & \wxmS \\[0.5ex]
--\cmdidx{slr\_spc\_key} & Solar spectrum string & ``LaN68'' & String \\[0.5ex]
--\cmdidx{slr\_zen\_ngl\_cos} & Cosine solar zenith angle & 1.0 & Fraction \\[0.5ex]
--\cmdidx{slv\_sng} & Mie solver to use & ``Wis79'' & String \\[0.5ex]
--\cmdidx{snw\_hgt\_lqd} & Equivalent liquid water snow depth & 0.0 & \m \\[0.5ex]
--\cmdidx{soi\_typ} & LSM soil type [1..5] & 1 & Index \\[0.5ex]
--\cmdidx{spc\_heat\_prt} & Specific heat capacity & 0.0 & \jxkgK \\[0.5ex]
--\cmdidx{spc\_abb\_sng} & Species abbreviation for Fortran data & ``foo'' & String \\[0.5ex]
--\cmdidx{spc\_idx\_sng} & Species index for Fortran data & ``foo'' & String \\[0.5ex]
--\cmdidx{ss\_alb\_cmd\_ln} & Single scattering albedo & 1.0 & Fraction \\[0.5ex]
--\cmdidx{sz\_dbg\_mcr} & Debugging size & 1.0 & \um \\[0.5ex]
--\cmdidx{sz\_grd\_sng} & Type of size grid & ``logarithmic'' & String \\[0.5ex]
--\cmdidx{sz\_mnm\_mcr} & Minimum size in distribution & 0.9 & \um \\[0.5ex]
--\cmdidx{sz\_mxm\_mcr} & Maximum size in distribution & 1.1 & \um \\[0.5ex]
--\cmdidx{sz\_nbr} & Number of particle size bins & 1 & Number \\[0.5ex]
--\cmdidx{sz\_prm\_rsn} & Size parameter resolution & 0.1 & Fraction \\[0.5ex]
--\cmdidx{thr\_nbr} & Thread number & 0 & Number \\[0.5ex]
--\cmdidx{tm\_dlt} & Timestep & 1200.0 & s \\[0.5ex]
--\cmdidx{tpt\_bbd\_wgt} & Blackbody temperature of radiation & 273.15 & K \\[0.5ex]
--\cmdidx{tpt\_gnd} & Ground temperature & 300.0 & K \\[0.5ex]
--\cmdidx{tpt\_ice} & Ice temperature & \texttt{tpt\_frz\_pnt} & K \\[0.5ex]
--\cmdidx{tpt\_mdp} & Environmental temperature & 300.0 & K \\[0.5ex]
--\cmdidx{tpt\_prt} & Particle temperature & 273.15 & K \\[0.5ex]
--\cmdidx{tpt\_soi} & Soil temperature & 297.0 & K \\[0.5ex]
--\cmdidx{tpt\_sst} & Sea surface temperature & 300.0 & K \\[0.5ex]
--\cmdidx{tpt\_vgt} & Vegetation temperature & 300.0 & K \\[0.5ex]
--\cmdidx{tst\_sng} & Name of test to perform (\cmdidx{htg}, \cmdidx{lbl}, \cmdidx{nc}, \cmdidx{nsz}, \cmdidx{psd\_ntg\_dgn}) & ``'' & String \\[0.5ex]
--\cmdidx{var\_ffc\_gmm} & Effective variance of Gamma distribution & 1.0 & Fraction \\[0.5ex]
--\cmdidx{vlm\_frc\_mntl} & Fraction of volume in mantle & 0.5 & Fraction \\[0.5ex]
--\cmdidx{vmr\_CO2} & Volume mixing ratio of \COd & $355.0 \times 10^{-6}$ & \mlcxmlc \\[0.5ex]
--\cmdidx{vmr\_HNO3\_gas} & Volume mixing ratio of gaseous \HNOt & $0.05 \times 10^{-9}$ & \mlcxmlc \\[0.5ex]
--\cmdidx{vwc\_sfc} & Volumetric water content & 0.03 & \mCxmC \\[0.5ex]
--\cmdidx{wbl\_shp} & Weibull distribution shape parameter & 2.4 & Fraction \\[0.5ex]
--\cmdidx{wnd\_frc\_dps\_cmd\_ln} & Friction speed & \texttt{cmd\_ln\_dfl} & \mxs \\[0.5ex]
--\cmdidx{wnd\_mrd\_mdp} & Surface layer meridional wind speed & 0.0 & \mxs \\[0.5ex]
--\cmdidx{wnd\_znl\_mdp} & Surface layer zonal wind speed & 10.0 & \mxs \\[0.5ex]
--\cmdidx{wvl\_dbg\_mcr} & Debugging wavelength & 0.50 & \um \\[0.5ex]
--\cmdidx{wvl\_grd\_sng} & Type of wavelength grid & ``regular'' & String \\[0.5ex]
--\cmdidx{wvl\_dlt\_mcr} & Bandwidth & 0.1 & \um \\[0.5ex]
--\cmdidx{wvl\_mdp\_mcr} & Midpoint wavelength & \texttt{cmd\_ln\_dfl} & \um \\[0.5ex]
--\cmdidx{wvl\_mnm\_mcr} & Minimum wavelength & 0.45 & \um \\[0.5ex]
--\cmdidx{wvl\_mxm\_mcr} & Maximum wavelength & 0.55 & \um \\[0.5ex]
--\cmdidx{wvl\_nbr} & Number of output wavelength bands & 1 & Number \\[0.5ex]
--\cmdidx{wvn\_dlt\_xcm} & Bandwidth & 1.0 & \xcm \\[0.5ex]
--\cmdidx{wvn\_mdp\_xcm} & Midpoint wavenumber & \texttt{cmd\_ln\_dfl} & \xcm \\[0.5ex]
--\cmdidx{wvn\_mnm\_xcm} & Minimum wavenumber & 18182 & \xcm \\[0.5ex]
--\cmdidx{wvn\_mxm\_xcm} & Maximum wavenumber & 22222 & \xcm \\[0.5ex]
--\cmdidx{wvn\_nbr} & Number of output wavenumber bands & 1 & Number \\[0.5ex]
--\cmdidx{xpt\_dsc} & Experiment description & ``'' & String \\[0.5ex]
\hline
\end{longtable}
\end{landscape} % End tbl:cmd_ln_mie

Table~\ref{tbl:fld_nm_swnb} summarizes the \trmdfn{fields} output by SWNB.
\begin{landscape} % Begin tbl:fld_nm_swnb
\begin{longtable}{ >{\ttfamily}l<{} >{\raggedright}p{20.0em}<{} l}
& & \kill % NB: longtable requires caption as table entry
\caption[SWNB Output Fields]{\textbf{SWNB Output Fields}%
\footnote{\emph{Source:} \cite{ZBP97,Zen99}}%
\label{tbl:fld_nm_swnb}} \\
\hline\hline \rule{0.0ex}{\hlntblhdrskp}% 
\textrm{Name(s)} & Purpose & Units \\[0.0ex]
\hline \rule{0.0ex}{\hlntblntrskp}%
\endfirsthead % Lines between and \endfirsthead appear at top of table
\caption[]{(continued)} \\ % Set label for following pages
\textrm{Name(s)} & Purpose & Units \\[0.0ex]
\hline \rule{0.0ex}{\hlntblntrskp}%
\endhead % Previous block appears at top of every page
\endlastfoot % Previous block appears at end of table
\cmdidx{abs\_bb\_SAS} & Broadband absorptance of surface-atmosphere system & fraction \\[0.5ex]
\cmdidx{abs\_bb\_atm} & Broadband absorptance of surface & fraction \\[0.5ex]
\cmdidx{abs\_bb\_sfc} & Broadband absorptance of atmosphere & fraction \\[0.5ex]
\cmdidx{abs\_nst\_SAS} & FSBR absorptance of surface-atmosphere system & fraction \\[0.5ex]
\cmdidx{abs\_nst\_atm} & FSBR absorptance of surface & fraction \\[0.5ex]
\cmdidx{abs\_nst\_sfc} & FSBR absorptance of atmosphere & fraction \\[0.5ex]
\cmdidx{abs\_spc\_SAS} & Spectral absorptance of surface-atmosphere system & fraction \\[0.5ex]
\cmdidx{abs\_spc\_atm} & Spectral absorptance of atmosphere & fraction \\[0.5ex]
\cmdidx{abs\_spc\_sfc} & Spectral absorptance of surface & fraction \\[0.5ex]
\cmdidx{alb\_sfc} & Specified Lambertian surface albedo & fraction \\[0.5ex]
\cmdidx{alt\_cld\_btm} & Highest interface beneath all clouds in column & meter \\[0.5ex]
\cmdidx{alt\_cld\_thick} & Thickness of region containing all clouds & meter \\[0.5ex]
\cmdidx{alt\_ntf} & Interface altitude & meter \\[0.5ex]
\cmdidx{alt} & Altitude & meter \\[0.5ex]
\cmdidx{azi\_dgr} & Azimuthal angle (degrees) & degree \\[0.5ex]
\cmdidx{azi} & Azimuthal angle (radians) & radian \\[0.5ex]
\cmdidx{bnd} & Midpoint wavelength & meter \\[0.5ex]
\cmdidx{flx\_abs\_atm\_rdr} & Flux absorbed in atmosphere at longer wavelengths & \wxmS \\[0.5ex]
\cmdidx{flx\_bb\_abs\_atm} & Broadband flux absorbed by atmospheric column only & \wxmS \\[0.5ex]
\cmdidx{flx\_bb\_abs\_sfc} & Broadband flux absorbed by surface only & \wxmS \\[0.5ex]
\cmdidx{flx\_bb\_abs\_ttl} & Broadband flux absorbed by surface-atmosphere system & \wxmS \\[0.5ex]
\cmdidx{flx\_bb\_abs} & Broadband flux absorbed by layer & \wxmS \\[0.5ex]
\cmdidx{flx\_bb\_dwn\_TOA} & Broadband incoming flux at TOA (total insolation) & \wxmS \\[0.5ex]
\cmdidx{flx\_bb\_dwn\_dff} & Diffuse downwelling broadband flux & \wxmS \\[0.5ex]
\cmdidx{flx\_bb\_dwn\_drc} & Direct downwelling broadband flux & \wxmS \\[0.5ex]
\cmdidx{flx\_bb\_dwn\_sfc} & Broadband downwelling flux at surface & \wxmS \\[0.5ex]
\cmdidx{flx\_bb\_dwn} & Total downwelling broadband flux (direct$+$diffuse) & \wxmS \\[0.5ex]
\cmdidx{flx\_bb\_net} & Net broadband flux (downwelling$-$upwelling) & \wxmS \\[0.5ex]
\cmdidx{flx\_bb\_up} & Upwelling broadband flux & \wxmS \\[0.5ex]
\cmdidx{flx\_nst\_abs\_atm} & FSBR flux absorbed by atmospheric column only & \wxmS \\[0.5ex]
\cmdidx{flx\_nst\_abs\_sfc} & FSBR flux absorbed by surface only & \wxmS \\[0.5ex]
\cmdidx{flx\_nst\_abs\_ttl} & FSBR flux absorbed by surface-atmosphere system & \wxmS \\[0.5ex]
\cmdidx{flx\_nst\_abs} & FSBR flux absorbed by layer & \wxmS \\[0.5ex]
\cmdidx{flx\_nst\_dwn\_TOA} & FSBR incoming flux at TOA (total insolation) & \wxmS \\[0.5ex]
\cmdidx{flx\_nst\_dwn\_sfc} & FSBR downwelling flux at surface & \wxmS \\[0.5ex]
\cmdidx{flx\_nst\_dwn} & Total downwelling FSBR flux (direct$+$diffuse) & \wxmS \\[0.5ex]
\cmdidx{flx\_nst\_net} & Net FSBR flux (downwelling$-$upwelling) & \wxmS \\[0.5ex]
\cmdidx{flx\_nst\_up} & Upwelling FSBR flux & \wxmS \\[0.5ex]
\cmdidx{flx\_slr\_frc} & Fraction of solar flux & fraction \\[0.5ex]
\cmdidx{flx\_spc\_abs\_SAS} & Spectral flux absorbed by surface-atmosphere system & \wxmSm \\[0.5ex]
\cmdidx{flx\_spc\_abs\_atm} & Spectral flux absorbed by atmospheric column only & \wxmSm \\[0.5ex]
\cmdidx{flx\_spc\_abs\_sfc} & Spectral flux absorbed by surface only & \wxmSm \\[0.5ex]
\cmdidx{flx\_spc\_abs} & Spectral flux absorbed by layer & \wxmSm \\[0.5ex]
\cmdidx{flx\_spc\_act\_pht\_TOA} & Spectral actinic photon flux at TOA & \nbrxmSsm \\[0.5ex]
\cmdidx{flx\_spc\_act\_pht\_sfc} & Spectral actinic photon flux at surface & \nbrxmSsm \\[0.5ex]
\cmdidx{flx\_spc\_dwn\_TOA} & Spectral solar insolation at TOA & \wxmSm \\[0.5ex]
\cmdidx{flx\_spc\_dwn\_dff} & Spectral diffuse downwelling flux & \wxmSm \\[0.5ex]
\cmdidx{flx\_spc\_dwn\_drc} & Spectral direct downwelling flux & \wxmSm \\[0.5ex]
\cmdidx{flx\_spc\_dwn\_sfc} & Spectral solar insolation at surface & \wxmSm \\[0.5ex]
\cmdidx{flx\_spc\_dwn} & Spectral downwelling flux & \wxmSm \\[0.5ex]
\cmdidx{flx\_spc\_pht\_dwn\_sfc} & Spectral photon flux downwelling at surface & \nbrxmSsm \\[0.5ex]
\cmdidx{flx\_spc\_up} & Spectral upwelling flux & \wxmSm \\[0.5ex]
\cmdidx{frc\_ice\_ttl} & Fraction of column condensate that is ice & fraction \\[0.5ex]
\cmdidx{htg\_rate\_bb} & Broadband heating rate & \kxs \\[0.5ex]
\cmdidx{j\_NO2} & Photolysis rate for $\ch\NOd + \cstplk\frq \yields \OtP + \NO$ & \xs \\[0.5ex]
\cmdidx{j\_spc\_NO2\_sfc} & Spectral photolysis rate at sfc for $\ch\NOd + \cstplk\frq \yields \OtP + \NO$ & \xsm \\[0.5ex]
\cmdidx{lat\_dgr} & Latitude (degrees) & degree \\[0.5ex]
\cmdidx{lcl\_time\_hr} & Local day hour & hour \\[0.5ex]
\cmdidx{lcl\_yr\_day} & Day of year in local time & day \\[0.5ex]
\cmdidx{levp} & Interface pressure & pascal \\[0.5ex]
\cmdidx{lev} & Layer pressure & pascal \\[0.5ex]
\cmdidx{mpc\_CWP} & Total column Condensed Water Path & \kgxmS \\[0.5ex]
\cmdidx{nrg\_pht} & Energy of photon at band center & joule photon-1 \\[0.5ex]
\cmdidx{ntn\_bb\_aa} & Broadband azimuthally averaged intensity & \wxmSsr \\[0.5ex]
\cmdidx{ntn\_bb\_mean} & Broadband mean intensity & \wxmSsr \\[0.5ex]
\cmdidx{ntn\_spc\_aa\_ndr\_sfc} & Spectral intensity of nadir radiation at surface & \wxmSmsr \\[0.5ex]
\cmdidx{ntn\_spc\_aa\_ndr} & Spectral intensity of nadir radiation & \wxmSmsr \\[0.5ex]
\cmdidx{ntn\_spc\_aa\_sfc} & Spectral intensity of radiation at surface & \wxmSmsr \\[0.5ex]
\cmdidx{ntn\_spc\_aa\_zen\_sfc} & Spectral intensity of zenith radiation at surface & \wxmSmsr \\[0.5ex]
\cmdidx{ntn\_spc\_aa\_zen} & Spectral intensity of zenith radiation & \wxmSmsr \\[0.5ex]
\cmdidx{ntn\_spc\_chn} & Full spectral intensity of particular band & \wxmSmsr \\[0.5ex]
\cmdidx{ntn\_spc\_mean} & Spectral mean intensity & \wxmSmsr \\[0.5ex]
\cmdidx{odac\_spc\_aer} & Aerosol absorption optical depth to surface & fraction \\[0.5ex]
\cmdidx{odac\_spc\_bga} & Background aerosol absorption optical depth to surface & fraction \\[0.5ex]
\cmdidx{odac\_spc\_ice} & Liquid water absorption optical depth to surface & fraction \\[0.5ex]
\cmdidx{odac\_spc\_lqd} & Ice water absorption optical depth to surface & fraction \\[0.5ex]
\cmdidx{odal\_obs\_aer} & Layer aerosol absorption optical depth & fraction \\[0.5ex]
\cmdidx{odal\_obs\_bga} & Layer background aerosol absorption optical depth & fraction \\[0.5ex]
\cmdidx{odsl\_obs\_aer} & Layer aerosol scattering optical depth & fraction \\[0.5ex]
\cmdidx{odsl\_obs\_bga} & Layer background aerosol scattering optical depth & fraction \\[0.5ex]
\cmdidx{odxc\_obs\_aer} & Column aerosol extinction optical depth & fraction \\[0.5ex]
\cmdidx{odxc\_obs\_bga} & Column background aerosol extinction optical depth & fraction \\[0.5ex]
\cmdidx{odxc\_spc\_CO2} & CO2 optical depth to surface & fraction \\[0.5ex]
\cmdidx{odxc\_spc\_H2OH2O} & H2O dimer optical depth to surface & fraction \\[0.5ex]
\cmdidx{odxc\_spc\_H2O} & H2O optical depth to surface & fraction \\[0.5ex]
\cmdidx{odxc\_spc\_NO2} & \NOd optical depth to surface & fraction \\[0.5ex]
\cmdidx{odxc\_spc\_O2N2} & O2N2 optical depth to surface & fraction \\[0.5ex]
\cmdidx{odxc\_spc\_O2O2} & O2O2 optical depth to surface & fraction \\[0.5ex]
\cmdidx{odxc\_spc\_O2} & O2 optical depth to surface & fraction \\[0.5ex]
\cmdidx{odxc\_spc\_O3} & O3 optical depth to surface & fraction \\[0.5ex]
\cmdidx{odxc\_spc\_OH} & OH optical depth to surface & fraction \\[0.5ex]
\cmdidx{odxc\_spc\_Ray} & Rayleigh scattering optical depth to surface & fraction \\[0.5ex]
\cmdidx{odxc\_spc\_aer} & Aerosol extinction optical depth to surface & fraction \\[0.5ex]
\cmdidx{odxc\_spc\_bga} & Background aerosol extinction optical depth to surface & fraction \\[0.5ex]
\cmdidx{odxc\_spc\_ice} & Ice water extinction optical depth to surface & fraction \\[0.5ex]
\cmdidx{odxc\_spc\_lqd} & Liquid water extinction optical depth to surface & fraction \\[0.5ex]
\cmdidx{odxc\_spc\_ttl} & Total extinction optical depth to surface & fraction \\[0.5ex]
\cmdidx{odxl\_obs\_aer} & Layer aerosol extinction optical depth & fraction \\[0.5ex]
\cmdidx{odxl\_obs\_bga} & Layer background aerosol extinction optical depth & fraction \\[0.5ex]
\cmdidx{plr\_cos} & Cosine polar angle (degrees) & fraction \\[0.5ex]
\cmdidx{plr\_dgr} & Polar angle (degrees) & degree \\[0.5ex]
\cmdidx{plr} & Polar angle (radians) & radian \\[0.5ex]
\cmdidx{rfl\_bb\_SAS} & Broadband albedo of entire surface-atmosphere system & fraction \\[0.5ex]
\cmdidx{rfl\_bb\_sfc} & Broadband albedo of surface & fraction \\[0.5ex]
\cmdidx{rfl\_nst\_SAS} & FSBR albedo of entire surface-atmosphere system & fraction \\[0.5ex]
\cmdidx{rfl\_nst\_sfc} & FSBR albedo of surface & fraction \\[0.5ex]
\cmdidx{rfl\_spc\_SAS} & Spectral planetary flux reflectance & fraction \\[0.5ex]
\cmdidx{slr\_zen\_ngl\_cos} & Cosine solar zenith angle & fraction \\[0.5ex]
\cmdidx{tau\_prs} & Optical level (pressure) & pascal \\[0.5ex]
\cmdidx{tau} & Optical level (optical depth) & fraction \\[0.5ex]
\cmdidx{tpt\_ntf} & Interface temperature & kelvin \\[0.5ex]
\cmdidx{tpt} & Layer Temperature & kelvin \\[0.5ex]
\cmdidx{trn\_bb\_atm} & Broadband transmission of atmospheric column & fraction \\[0.5ex]
\cmdidx{trn\_nst\_atm} & FSBR transmission of atmospheric column & fraction \\[0.5ex]
\cmdidx{trn\_spc\_atm\_CO2} & Column transmission due to CO2 absorption & fraction \\[0.5ex]
\cmdidx{trn\_spc\_atm\_H2OH2O} & Column transmission due to H2O dimer absorption & fraction \\[0.5ex]
\cmdidx{trn\_spc\_atm\_H2O} & Column transmission due to H2O absorption & fraction \\[0.5ex]
\cmdidx{trn\_spc\_atm\_NO2} & Column transmission due to \NOd absorption & fraction \\[0.5ex]
\cmdidx{trn\_spc\_atm\_O2N2} & Column transmission due to O2-N2 absorption & fraction \\[0.5ex]
\cmdidx{trn\_spc\_atm\_O2O2} & Column transmission due to O2-O2 absorption & fraction \\[0.5ex]
\cmdidx{trn\_spc\_atm\_O2} & Column transmission due to O2 absorption & fraction \\[0.5ex]
\cmdidx{trn\_spc\_atm\_O3} & Column transmission due to O3 absorption & fraction \\[0.5ex]
\cmdidx{trn\_spc\_atm\_OH} & Column transmission due to OH absorption & fraction \\[0.5ex]
\cmdidx{trn\_spc\_atm\_Ray} & Column transmission due to Rayleigh scattering & fraction \\[0.5ex]
\cmdidx{trn\_spc\_atm\_aer} & Column transmission due to aerosol extinction & fraction \\[0.5ex]
\cmdidx{trn\_spc\_atm\_bga} & Column transmission due to background aerosol extinction & fraction \\[0.5ex]
\cmdidx{trn\_spc\_atm\_ice} & Column transmission due to ice extinction & fraction \\[0.5ex]
\cmdidx{trn\_spc\_atm\_lqd} & Column transmission due to liquid extinction & fraction \\[0.5ex]
\cmdidx{trn\_spc\_atm\_ttl} & Spectral flux transmission of entire column & fraction \\[0.5ex]
\cmdidx{wvl\_ctr} & Midpoint wavelength in band & meter \\[0.5ex]
\cmdidx{wvl\_dlt} & Width of band & meter \\[0.5ex]
\cmdidx{wvl\_grd} & Wavelength grid & meter \\[0.5ex]
\cmdidx{wvl\_max} & Maximum wavelength in band & meter \\[0.5ex]
\cmdidx{wvl\_min} & Minimum wavelength in band & meter \\[0.5ex]
\cmdidx{wvl\_obs\_aer} & Wavelength of aerosol optical depth specification & meter \\[0.5ex]
\cmdidx{wvl\_obs\_bga} & Wavelength of background aerosol optical depth specification & meter \\[0.5ex]
\cmdidx{wvn\_ctr} & Midpoint wavenumber in band & centimeter-1 \\[0.5ex]
\cmdidx{wvn\_dlt} & Bandwidth in wavenumbers & centimeter-1 \\[0.5ex]
\cmdidx{wvn\_max} & Maximum wavenumber in band & centimeter-1 \\[0.5ex]
\cmdidx{wvn\_min} & Minimum wavenumber in band & centimeter-1 \\[0.5ex]
\hline
\end{longtable}
\end{landscape} % End tbl:fld_nm_swnb

Table~\ref{tbl:fld_nm_clm} summarizes the \trmdfn{fields} output by CLM.
\begin{landscape} % Begin tbl:fld_nm_clm
\begin{longtable}{ >{\ttfamily}l<{} >{\raggedright}p{20.0em}<{} l}
& & \kill % NB: longtable requires caption as table entry
\caption[CLM Output Fields]{\textbf{CLM Output Fields}%
\footnote{\emph{Source:} \cite{ZBP97,Zen99}}%
\label{tbl:fld_nm_clm}} \\
\hline\hline \rule{0.0ex}{\hlntblhdrskp}% 
\textrm{Name(s)} & Purpose & Units \\[0.0ex]
\hline \rule{0.0ex}{\hlntblntrskp}%
\endfirsthead % Lines between and \endfirsthead appear at top of table
\caption[]{(continued)} \\ % Set label for following pages
\textrm{Name(s)} & Purpose & Units \\[0.0ex]
\hline \rule{0.0ex}{\hlntblntrskp}%
\endhead % Previous block appears at top of every page
\endlastfoot % Previous block appears at end of table
\cmdidx{CO2\_vmr\_clm} & Carbon Dioxide volume mixing ratio & fraction \\[0.5ex]
\cmdidx{N2O\_vmr\_clm} & Nitrous Oxide volume mixing ratio & fraction \\[0.5ex]
\cmdidx{CH4\_vmr\_clm} & Methane volume mixing ratio & fraction \\[0.5ex]
\cmdidx{CFC11\_vmr\_clm} & CFC11 volume mixing ratio & fraction \\[0.5ex]
\cmdidx{CFC12\_vmr\_clm} & CFC12 volume mixing ratio & fraction \\[0.5ex]
\cmdidx{RH\_ice} & Relative humidity w/r/t ice & fraction \\[0.5ex]
\cmdidx{RH} & Relative humidity & fraction \\[0.5ex]
\cmdidx{RH\_lqd} & Relative humidity w/r/t liquid & fraction \\[0.5ex]
\cmdidx{alb\_sfc\_NIR\_drc} & Albedo for NIR radiation at strong zenith angles & fraction \\[0.5ex]
\cmdidx{alb\_sfc\_NIR\_dff} & Albedo for NIR radiation at weak zenith angles & fraction \\[0.5ex]
\cmdidx{alb\_sfc} & Prescribed surface albedo & fraction \\[0.5ex]
\cmdidx{alb\_sfc\_vsb\_drc} & Albedo for visible radiation at strong zenith angles & fraction \\[0.5ex]
\cmdidx{alb\_sfc\_vsb\_dff} & Albedo for visible radiation at weak zenith angles & fraction \\[0.5ex]
\cmdidx{alt\_cld\_btm} & Highest interface beneath all clouds in column & meter \\[0.5ex]
\cmdidx{alt\_cld\_mid} & Altitude at midpoint of all clouds in column & meter \\[0.5ex]
\cmdidx{alt\_cld\_thick} & Thickness of region containing all clouds & meter \\[0.5ex]
\cmdidx{alt\_cld\_top} & Lowest interface above all clouds in column & meter \\[0.5ex]
\cmdidx{alt\_dlt} & Layer altitude thickness & meter \\[0.5ex]
\cmdidx{alt} & Altitude & meter \\[0.5ex]
\cmdidx{alt\_ntf} & Interface altitude & meter \\[0.5ex]
\cmdidx{cld\_frc} & Cloud fraction & fraction \\[0.5ex]
\cmdidx{cnc\_CO2} & CO2 concentration & \mlcxmC \\[0.5ex]
\cmdidx{cnc\_CH4} & CH4 concentration & \mlcxmC \\[0.5ex]
\cmdidx{cnc\_N2O} & N2O concentration & \mlcxmC \\[0.5ex]
\cmdidx{cnc\_CFC11} & CFC11 concentration & \mlcxmC \\[0.5ex]
\cmdidx{cnc\_CFC12} & CFC12 concentration & \mlcxmC \\[0.5ex]
\cmdidx{cnc\_H2OH2O} & H2O dimer concentration & \mlcxmC \\[0.5ex]
\cmdidx{cnc\_H2O} & H2O concentration & \mlcxmC \\[0.5ex]
\cmdidx{cnc\_N2} & N2 concentration & \mlcxmC \\[0.5ex]
\cmdidx{cnc\_NO2} & \NOd concentration & \mlcxmC \\[0.5ex]
\cmdidx{cnc\_O2O2} & O2O2 concentration & \mlcxmC \\[0.5ex]
\cmdidx{cnc\_O2\_cnc\_N2} & O2 number concentration times N2 number concentration & \mlcSxmSix \\[0.5ex]
\cmdidx{cnc\_O2\_cnc\_O2} & O2 number concentration squared & \mlcSxmSix \\[0.5ex]
\cmdidx{cnc\_O2} & O2 concentration & \mlcxmC \\[0.5ex]
\cmdidx{cnc\_O2\_npl\_N2\_clm} & Column total O2 number concentration times N2 number path & \mlcSxmF \\[0.5ex]
\cmdidx{cnc\_O2\_npl\_N2} & O2 number concentration times N2 number path & \mlcSxmF \\[0.5ex]
\cmdidx{cnc\_O2\_npl\_O2\_clm} & Column total O2 number concentration times O2 number path & \mlcSxmF \\[0.5ex]
\cmdidx{cnc\_O2\_npl\_O2\_clm\_frc} & Fraction of column total O2-O2 at or above each layer & fraction \\[0.5ex]
\cmdidx{cnc\_O2\_npl\_O2} & O2 number concentration times O2 number path & \mlcSxmF \\[0.5ex]
\cmdidx{cnc\_O3} & O3 concentration & \nbrxmC \\[0.5ex]
\cmdidx{cnc\_OH} & OH concentration & \nbrxmC \\[0.5ex]
\cmdidx{cnc\_dry\_air} & Dry concentration & \nbrxmC \\[0.5ex]
\cmdidx{cnc\_mst\_air} & Moist air concentration & \nbrxmC \\[0.5ex]
\cmdidx{dns\_CO2} & Density of CO2 & \kgxmC \\[0.5ex]
\cmdidx{dns\_CH4} & Density of CH4 & \kgxmC \\[0.5ex]
\cmdidx{dns\_N2O} & Density of N2O & \kgxmC \\[0.5ex]
\cmdidx{dns\_CFC11} & Density of CFC11 & \kgxmC \\[0.5ex]
\cmdidx{dns\_CFC12} & Density of CFC12 & \kgxmC \\[0.5ex]
\cmdidx{dns\_H2OH2O} & Density of H20H2O & \kgxmC \\[0.5ex]
\cmdidx{dns\_H2O} & Density of H2O & \kgxmC \\[0.5ex]
\cmdidx{dns\_N2} & Density of N2 & \kgxmC \\[0.5ex]
\cmdidx{dns\_NO2} & Density of \NOd & \kgxmC \\[0.5ex]
\cmdidx{dns\_O2O2} & Density of O2-O2 & \kgxmC \\[0.5ex]
\cmdidx{dns\_O2\_dns\_N2} & O2 mass concentration times N2 mass concentration & \kgSxmSix \\[0.5ex]
\cmdidx{dns\_O2\_dns\_O2} & O2 mass concentration squared & \kgSxmSix \\[0.5ex]
\cmdidx{dns\_O2} & Density of O2 & \kgxmC \\[0.5ex]
\cmdidx{dns\_O2\_mpl\_N2\_clm} & Column total O2 mass concentration times N2 mass path & \kgSxmF \\[0.5ex]
\cmdidx{dns\_O2\_mpl\_N2} & O2 mass concentration times N2 mass path & \kgSxmF \\[0.5ex]
\cmdidx{dns\_O2\_mpl\_O2\_clm} & Column total O2 mass concentration times O2 mass path & \kgSxmF \\[0.5ex]
\cmdidx{dns\_O2\_mpl\_O2} & O2 mass concentration times O2 mass path & \kgSxmF \\[0.5ex]
\cmdidx{dns\_O3} & Density of O3 & \kgxmC \\[0.5ex]
\cmdidx{dns\_OH} & Density of OH & \kgxmC \\[0.5ex]
\cmdidx{dns\_aer} & Aerosol density & \kgxmC \\[0.5ex]
\cmdidx{dns\_bga} & Background aerosol density & \kgxmC \\[0.5ex]
\cmdidx{dns\_dry\_air} & Density of dry air & \kgxmC \\[0.5ex]
\cmdidx{dns\_mst\_air} & Density of moist air & \kgxmC \\[0.5ex]
\cmdidx{eqn\_time\_sec} & foo & second \\[0.5ex]
\cmdidx{ext\_cff\_mss\_aer} & Aerosol mass extinction coefficient & \mSxkg \\[0.5ex]
\cmdidx{ext\_cff\_mss\_bga} & Background aerosol mass extinction coefficient & \mSxkg \\[0.5ex]
\cmdidx{frc\_ice} & Fraction of condensate that is ice & fraction \\[0.5ex]
\cmdidx{frc\_ice\_ttl} & Fraction of column condensate that is ice & fraction \\[0.5ex]
\cmdidx{frc\_str\_zen\_ngl\_sfc} & Surface fraction of strong zenith angle dependence & fraction \\[0.5ex]
\cmdidx{gas\_cst\_mst\_air} & Specific gas constant for moist air & joule kilogram-1 kelvin-1 \\[0.5ex]
\cmdidx{gmt\_day} & foo & day \\[0.5ex]
\cmdidx{gmt\_doy} & foo & day \\[0.5ex]
\cmdidx{gmt\_hr} & foo & hour \\[0.5ex]
\cmdidx{gmt\_mnt} & foo & minute \\[0.5ex]
\cmdidx{gmt\_mth} & foo & month \\[0.5ex]
\cmdidx{gmt\_sec} & foo & second \\[0.5ex]
\cmdidx{gmt\_ydy} & foo & day \\[0.5ex]
\cmdidx{gmt\_yr} & foo & year \\[0.5ex]
\cmdidx{grv} & Gravity & meter second-2 \\[0.5ex]
\cmdidx{oro} & Orography flag & flag \\[0.5ex]
\cmdidx{lat\_cos} & Cosine of latitude & fraction \\[0.5ex]
\cmdidx{lat\_dgr} & Latitude (degrees) & degree \\[0.5ex]
\cmdidx{lat} & Latitude (radians) & radian \\[0.5ex]
\cmdidx{lcl\_time\_hr} & Local day hour & hour \\[0.5ex]
\cmdidx{lcl\_yr\_day} & Day of year in local time & day \\[0.5ex]
\cmdidx{lev} & Layer pressure & pascal \\[0.5ex]
\cmdidx{levp} & Interface pressure & pascal \\[0.5ex]
\cmdidx{lmt\_day} & foo & day \\[0.5ex]
\cmdidx{lmt\_doy} & foo & day \\[0.5ex]
\cmdidx{lmt\_hr} & foo & hour \\[0.5ex]
\cmdidx{lmt\_mnt} & foo & minute \\[0.5ex]
\cmdidx{lmt\_mth} & foo & month \\[0.5ex]
\cmdidx{lmt\_sec} & foo & second \\[0.5ex]
\cmdidx{lmt\_ydy} & foo & day \\[0.5ex]
\cmdidx{lmt\_yr} & foo & year \\[0.5ex]
\cmdidx{lon\_dgr} & foo & degree \\[0.5ex]
\cmdidx{lon} & foo & radian \\[0.5ex]
\cmdidx{lon\_sec} & foo & second \\[0.5ex]
\cmdidx{ltst\_day} & foo & day \\[0.5ex]
\cmdidx{ltst\_doy} & foo & day \\[0.5ex]
\cmdidx{ltst\_hr} & foo & hour \\[0.5ex]
\cmdidx{ltst\_mnt} & foo & minute \\[0.5ex]
\cmdidx{ltst\_mth} & foo & month \\[0.5ex]
\cmdidx{ltst\_sec} & foo & second \\[0.5ex]
\cmdidx{ltst\_ydy} & foo & day \\[0.5ex]
\cmdidx{ltst\_yr} & foo & year \\[0.5ex]
\cmdidx{mmw\_mst\_air} & Mean molecular weight of moist air & kilogram mole-1 \\[0.5ex]
\cmdidx{mpc\_CO2} & Mass path of CO2 in column & \kgxmS \\[0.5ex]
\cmdidx{mpc\_CH4} & Mass path of CH4 in column & \kgxmS \\[0.5ex]
\cmdidx{mpc\_N2O} & Mass path of N2O in column & \kgxmS \\[0.5ex]
\cmdidx{mpc\_CFC11} & Mass path of CFC11 in column & \kgxmS \\[0.5ex]
\cmdidx{mpc\_CFC12} & Mass path of CFC12 in column & \kgxmS \\[0.5ex]
\cmdidx{mpc\_CWP} & Total column Condensed Water Path & \kgxmS \\[0.5ex]
\cmdidx{mpc\_H2OH2O} & Mass path of H2O dimer in column & \kgxmS \\[0.5ex]
\cmdidx{mpc\_H2O} & Mass path of H2O in column & \kgxmS \\[0.5ex]
\cmdidx{mpc\_IWP} & Total column Ice Water Path & \kgxmS \\[0.5ex]
\cmdidx{mpc\_LWP} & Total column Liquid Water Path & \kgxmS \\[0.5ex]
\cmdidx{mpc\_N2} & Mass path of N2 in column & \kgxmS \\[0.5ex]
\cmdidx{mpc\_NO2} & Mass path of \NOd in column & \kgxmS \\[0.5ex]
\cmdidx{mpc\_O2O2} & Mass path of O2-O2 in column & \kgxmS \\[0.5ex]
\cmdidx{mpc\_O2} & Mass path of O2 in column & \kgxmS \\[0.5ex]
\cmdidx{mpc\_O3\_DU} & Mass path of O3 in column & Dobson \\[0.5ex]
\cmdidx{mpc\_O3} & Mass path of O3 in column & \kgxmS \\[0.5ex]
\cmdidx{mpc\_OH} & Mass path of OH in column & \kgxmS \\[0.5ex]
\cmdidx{mpc\_aer} & Total column mass path of aerosol & \kgxmS \\[0.5ex]
\cmdidx{mpc\_bga} & Total column mass path of background aerosol & \kgxmS \\[0.5ex]
\cmdidx{mpc\_dry\_air} & Mass path of dry air in column & \kgxmS \\[0.5ex]
\cmdidx{mpc\_mst\_air} & Mass path of moist air in column & \kgxmS \\[0.5ex]
\cmdidx{mpl\_CO2} & Mass path of CO2 in layer & \kgxmS \\[0.5ex]
\cmdidx{mpl\_CH4} & Mass path of CH4 in layer & \kgxmS \\[0.5ex]
\cmdidx{mpl\_N2O} & Mass path of N2O in layer & \kgxmS \\[0.5ex]
\cmdidx{mpl\_CFC11} & Mass path of CFC11 in layer & \kgxmS \\[0.5ex]
\cmdidx{mpl\_CFC12} & Mass path of CFC12 in layer & \kgxmS \\[0.5ex]
\cmdidx{mpl\_CWP} & Layer Condensed Water Path & \kgxmS \\[0.5ex]
\cmdidx{mpl\_H2OH2O} & Mass path of H2O dimer in layer & \kgxmS \\[0.5ex]
\cmdidx{mpl\_H2O} & Mass path of H2O in layer & \kgxmS \\[0.5ex]
\cmdidx{mpl\_IWP} & Layer Ice Water Path & \kgxmS \\[0.5ex]
\cmdidx{mpl\_LWP} & Layer Liquid Water Path & \kgxmS \\[0.5ex]
\cmdidx{mpl\_N2} & Mass path of N2 in layer & \kgSxmF \\[0.5ex]
\cmdidx{mpl\_NO2} & Mass path of \NOd in layer & \kgxmS \\[0.5ex]
\cmdidx{mpl\_O2O2} & Mass path of O2-O2 in layer & \kgxmS \\[0.5ex]
\cmdidx{mpl\_O2} & Mass path of O2 in layer & \kgSxmF \\[0.5ex]
\cmdidx{mpl\_O3} & Mass path of O3 in layer & \kgxmS \\[0.5ex]
\cmdidx{mpl\_OH} & Mass path of OH in layer & \kgxmS \\[0.5ex]
\cmdidx{mpl\_aer} & Layer mass path of aerosol & \kgxmS \\[0.5ex]
\cmdidx{mpl\_bga} & Layer mass path of aerosol & \kgxmS \\[0.5ex]
\cmdidx{mpl\_dry\_air} & Mass path of dry air in layer & \kgxmS \\[0.5ex]
\cmdidx{mpl\_mst\_air} & Mass path of moist air in layer & \kgxmS \\[0.5ex]
\cmdidx{npc\_CO2} & Column number path of CO2 & \mlcxmS \\[0.5ex]
\cmdidx{npc\_CH4} & Column number path of CH4 & \mlcxmS \\[0.5ex]
\cmdidx{npc\_N2O} & Column number path of N2O & \mlcxmS \\[0.5ex]
\cmdidx{npc\_CFC11} & Column number path of CFC11 & \mlcxmS \\[0.5ex]
\cmdidx{npc\_CFC12} & Column number path of CFC12 & \mlcxmS \\[0.5ex]
\cmdidx{npc\_H2OH2O} & Column number path of H2O dimer & \mlcxmS \\[0.5ex]
\cmdidx{npc\_H2O} & Column number path of H2O & \mlcxmS \\[0.5ex]
\cmdidx{npc\_N2} & Column number path of O2 & \mlcxmS \\[0.5ex]
\cmdidx{npc\_NO2} & Column number path of \NOd & \mlcxmS \\[0.5ex]
\cmdidx{npc\_O2O2} & Column number path of O2O2 & \mlcxmS \\[0.5ex]
\cmdidx{npc\_O2} & Column number path of O2 & \mlcxmS \\[0.5ex]
\cmdidx{npc\_O3} & Column number path of O3 & \mlcxmS \\[0.5ex]
\cmdidx{npc\_OH} & Column number path of OH & \mlcxmS \\[0.5ex]
\cmdidx{npc\_dry\_air} & Column number path of dry air & \mlcxmS \\[0.5ex]
\cmdidx{npc\_mst\_air} & Column number path of moist air & \mlcxmS \\[0.5ex]
\cmdidx{npl\_CO2} & Number path of CO2 in layer & \mlcxmS \\[0.5ex]
\cmdidx{npl\_CH4} & Number path of CH4 in layer & \mlcxmS \\[0.5ex]
\cmdidx{npl\_N2O} & Number path of N2O in layer & \mlcxmS \\[0.5ex]
\cmdidx{npl\_CFC11} & Number path of CFC11 in layer & \mlcxmS \\[0.5ex]
\cmdidx{npl\_CFC12} & Number path of CFC12 in layer & \mlcxmS \\[0.5ex]
\cmdidx{npl\_H2OH2O} & Number path of H2O dimer in layer & \mlcxmS \\[0.5ex]
\cmdidx{npl\_H2O} & Number path of H2O in layer & \mlcxmS \\[0.5ex]
\cmdidx{npl\_N2} & Number path of N2 in layer & \mlcSxmF \\[0.5ex]
\cmdidx{npl\_NO2} & Number path of \NOd in layer & \mlcxmS \\[0.5ex]
\cmdidx{npl\_O2O2} & Number path of O2-O2 in layer & \mlcxmS \\[0.5ex]
\cmdidx{npl\_O2} & Number path of O2 in layer & \mlcSxmF \\[0.5ex]
\cmdidx{npl\_O3} & Number path of O3 in layer & \mlcxmS \\[0.5ex]
\cmdidx{npl\_OH} & Number path of OH in layer & \mlcxmS \\[0.5ex]
\cmdidx{npl\_dry\_air} & Number path of dry air in layer & \mlcxmS \\[0.5ex]
\cmdidx{npl\_mst\_air} & Number path of moist air in layer & \mlcxmS \\[0.5ex]
\cmdidx{odxc\_obs\_aer} & Column aerosol extinction optical depth & fraction \\[0.5ex]
\cmdidx{odxc\_obs\_bga} & Column background aerosol extinction optical depth & fraction \\[0.5ex]
\cmdidx{odxl\_obs\_aer} & Layer aerosol extinction optical depth & fraction \\[0.5ex]
\cmdidx{odxl\_obs\_bga} & Layer background aerosol extinction optical depth & fraction \\[0.5ex]
\cmdidx{oneD\_foo} &  &  \\[0.5ex]
\cmdidx{ppr\_CO2} & Partial pressure of CO2 & pascal \\[0.5ex]
\cmdidx{ppr\_CH4} & Partial pressure of CH4 & pascal \\[0.5ex]
\cmdidx{ppr\_N2O} & Partial pressure of N2O & pascal \\[0.5ex]
\cmdidx{ppr\_CFC11} & Partial pressure of CFC11 & pascal \\[0.5ex]
\cmdidx{ppr\_CFC12} & Partial pressure of CFC12 & pascal \\[0.5ex]
\cmdidx{ppr\_H2OH2O} & Partial pressure of H2O dimer & pascal \\[0.5ex]
\cmdidx{ppr\_H2O} & Partial pressure of H2O & pascal \\[0.5ex]
\cmdidx{ppr\_N2} & Partial pressure of N2 & pascal \\[0.5ex]
\cmdidx{ppr\_NO2} & Partial pressure of \NOd & pascal \\[0.5ex]
\cmdidx{ppr\_O2O2} & Partial pressure of O2O2 & pascal \\[0.5ex]
\cmdidx{ppr\_O2} & Partial pressure of O2 & pascal \\[0.5ex]
\cmdidx{ppr\_O3} & Partial pressure of O3 & pascal \\[0.5ex]
\cmdidx{ppr\_OH} & Partial pressure of OH & pascal \\[0.5ex]
\cmdidx{ppr\_dry\_air} & Partial pressure of dry air & pascal \\[0.5ex]
\cmdidx{prs\_cld\_btm} & Highest interface beneath all clouds in column & pascal \\[0.5ex]
\cmdidx{prs\_cld\_mid} & Pressure at midpoint of all clouds in column & pascal \\[0.5ex]
\cmdidx{prs\_cld\_thick} & Thickness of region containing all clouds & meter \\[0.5ex]
\cmdidx{prs\_cld\_top} & Lowest interface above all clouds in column & pascal \\[0.5ex]
\cmdidx{prs\_dlt} & Layer pressure thickness & pascal \\[0.5ex]
\cmdidx{prs} & Pressure & pascal \\[0.5ex]
\cmdidx{prs\_ntf} & Interface pressure & pascal \\[0.5ex]
\cmdidx{prs\_sfc} & Surface pressure & pascal \\[0.5ex]
\cmdidx{q\_CO2} & Mass mixing ratio of CO2 & \kgxkg \\[0.5ex]
\cmdidx{q\_CH4} & Mass mixing ratio of CH4 & \kgxkg \\[0.5ex]
\cmdidx{q\_N2O} & Mass mixing ratio of N2O & \kgxkg \\[0.5ex]
\cmdidx{q\_CFC11} & Mass mixing ratio of CFC11 & \kgxkg \\[0.5ex]
\cmdidx{q\_CFC12} & Mass mixing ratio of CFC12 & \kgxkg \\[0.5ex]
\cmdidx{q\_H2OH2O} & Water vapor dimer mass mixing ratio & \kgxkg \\[0.5ex]
\cmdidx{q\_H2OH2O\_rcp\_q\_H2O} & Ratio of dimer mmr to monomer mmr & fraction \\[0.5ex]
\cmdidx{q\_H2O} & Water vapor mass mixing ratio & fraction \\[0.5ex]
\cmdidx{q\_N2} & Mass mixing ratio of N2 & \kgxkg \\[0.5ex]
\cmdidx{q\_NO2} & Mass mixing ratio of \NOd & \kgxkg \\[0.5ex]
\cmdidx{q\_O2O2} & Ozone mass mixing ratio & \kgxkg \\[0.5ex]
\cmdidx{q\_O2} & Mass mixing ratio of O2 & \kgxkg \\[0.5ex]
\cmdidx{q\_O3} & Ozone mass mixing ratio & \kgxkg \\[0.5ex]
\cmdidx{q\_OH} & Mass mixing ratio of OH & \kgxkg \\[0.5ex]
\cmdidx{qst\_H2O\_ice} & Saturation mixing ratio w/r/t ice & \kgxkg \\[0.5ex]
\cmdidx{qst\_H2O\_lqd} & Saturation mixing ratio w/r/t liquid & \kgxkg \\[0.5ex]
\cmdidx{r\_CO2} & Dry-mass mixing ratio (r) of CO2 & \kgxkg \\[0.5ex]
\cmdidx{r\_CH4} & Dry-mass mixing ratio (r) of CH4 & \kgxkg \\[0.5ex]
\cmdidx{r\_N2O} & Dry-mass mixing ratio (r) of N2O & \kgxkg \\[0.5ex]
\cmdidx{r\_CFC11} & Dry-mass mixing ratio (r) of CFC11 & \kgxkg \\[0.5ex]
\cmdidx{r\_CFC12} & Dry-mass mixing ratio (r) of CFC12 & \kgxkg \\[0.5ex]
\cmdidx{r\_H2OH2O} & Dry-mass mixing ratio (r) of H2O dimer & \kgxkg \\[0.5ex]
\cmdidx{r\_H2O} & Dry-mass mixing ratio (r) of H2O & \kgxkg \\[0.5ex]
\cmdidx{r\_N2} & Dry-mass mixing ratio (r) of N2 & \kgxkg \\[0.5ex]
\cmdidx{r\_NO2} & Dry-mass mixing ratio (r) of \NOd & \kgxkg \\[0.5ex]
\cmdidx{r\_O2O2} & Dry-mass mixing ratio (r) of O2O2 & \kgxkg \\[0.5ex]
\cmdidx{r\_O2} & Dry-mass mixing ratio (r) of O2 & \kgxkg \\[0.5ex]
\cmdidx{r\_O3} & Dry-mass mixing ratio (r) of O3 & \kgxkg \\[0.5ex]
\cmdidx{r\_OH} & Dry-mass mixing ratio (r) of OH & \kgxkg \\[0.5ex]
\cmdidx{rds\_fct\_ice} & Effective radius of ice crystals & micron \\[0.5ex]
\cmdidx{rds\_fct\_lqd} & Effective radius of liquid droplets & micron \\[0.5ex]
\cmdidx{rgh\_len} & Aerodynamic roughness length & meter \\[0.5ex]
\cmdidx{scl\_hgt} & Local scale height & meter \\[0.5ex]
\cmdidx{sfc\_ems} & Surface emissivity & fraction \\[0.5ex]
\cmdidx{slr\_azi\_dgr} & Solar azimuth angle & degree \\[0.5ex]
\cmdidx{slr\_crd\_gmm\_dgr} & foo & degree \\[0.5ex]
\cmdidx{slr\_cst} & Solar constant & \wxmS \\[0.5ex]
\cmdidx{slr\_dcl\_dgr} & Solar declination & degree \\[0.5ex]
\cmdidx{slr\_dmt\_dgr} & Diameter of solar disc & degree \\[0.5ex]
\cmdidx{slr\_dst\_au} & Earth-sun distance & astronomical units \\[0.5ex]
\cmdidx{slr\_elv\_dgr} & Solar elevation & degree \\[0.5ex]
\cmdidx{slr\_flx\_TOA} & Solar flux at TOA & \wxmS \\[0.5ex]
\cmdidx{slr\_flx\_nrm\_TOA} & Solar constant corrected for orbital
position & \wxmS \\[0.5ex]
\cmdidx{slr\_hr\_ngl\_dgr} & Solar hour angle & degree \\[0.5ex]
\cmdidx{slr\_rfr\_ngl\_dgr} & Solar refraction angle & degree \\[0.5ex]
\cmdidx{slr\_rgt\_asc\_dgr} & Solar right ascension & degree \\[0.5ex]
\cmdidx{slr\_zen\_ngl\_cos} & Cosine solar zenith angle & fraction \\[0.5ex]
\cmdidx{slr\_zen\_ngl\_dgr} & Solar zenith angle in degrees & degree \\[0.5ex]
\cmdidx{slr\_zen\_ngl} & Solar zenith angle & radian \\[0.5ex]
\cmdidx{snow\_depth} & Snow depth & meter \\[0.5ex]
\cmdidx{spc\_heat\_mst\_air} & Specific heat at constant pressure of moist air & joule kilogram-1 kelvin-1 \\[0.5ex]
\cmdidx{time\_lmt} & Seconds between 1969 and LMT of simulation & second \\[0.5ex]
\cmdidx{time\_ltst} & Seconds between 1969 and LTST of simulation & second \\[0.5ex]
\cmdidx{time\_unix} & Seconds between 1969 and GMT of simulation & second \\[0.5ex]
\cmdidx{tpt\_cls} & Layer temperature (Celsius) & celsius \\[0.5ex]
\cmdidx{tpt\_cls\_ntf} & Interface temperature (Celsius) & celsius \\[0.5ex]
\cmdidx{tpt} & Layer Temperature & kelvin \\[0.5ex]
\cmdidx{tpt\_ntf} & Interface temperature & kelvin \\[0.5ex]
\cmdidx{tpt\_sfc} & Temperature of air in contact with surface & kelvin \\[0.5ex]
\cmdidx{tpt\_skn} & Temperature of surface & kelvin \\[0.5ex]
\cmdidx{tpt\_vrt} & Virtual temperature & kelvin \\[0.5ex]
\cmdidx{vmr\_CO2} & Volume mixing ratio of CO2 & number number-1 \\[0.5ex]
\cmdidx{vmr\_CH4} & Volume mixing ratio of CH4 & number number-1 \\[0.5ex]
\cmdidx{vmr\_N2O} & Volume mixing ratio of N2O & number number-1 \\[0.5ex]
\cmdidx{vmr\_CFC11} & Volume mixing ratio of CFC11 & number number-1 \\[0.5ex]
\cmdidx{vmr\_CFC12} & Volume mixing ratio of CFC12 & number number-1 \\[0.5ex]
\cmdidx{vmr\_H2OH2O} & Volume mixing ratio of H2O dimer & number number-1 \\[0.5ex]
\cmdidx{vmr\_H2O} & Volume mixing ratio of H2O & number number-1 \\[0.5ex]
\cmdidx{vmr\_N2} & Volume mixing ratio of N2 & number number-1 \\[0.5ex]
\cmdidx{vmr\_NO2} & Volume mixing ratio of \NOd & number number-1 \\[0.5ex]
\cmdidx{vmr\_O2O2} & Volume mixing ratio of O2O2 & number number-1 \\[0.5ex]
\cmdidx{vmr\_O2} & Volume mixing ratio of O2 & number number-1 \\[0.5ex]
\cmdidx{vmr\_O3} & Volume mixing ratio of O3 & number number-1 \\[0.5ex]
\cmdidx{vmr\_OH} & Volume mixing ratio of OH & number number-1 \\[0.5ex]
\cmdidx{wvl\_obs\_aer} & Wavelength of aerosol optical depth specification & meter \\[0.5ex]
\cmdidx{wvl\_obs\_bga} & Wavelength of background aerosol optical depth specification & meter \\[0.5ex]
\cmdidx{xnt\_fac} & Eccentricity factor & fraction \\[0.5ex]
\hline
\end{longtable}
\end{landscape} % End tbl:fld_nm_clm

% Bibliography
%\renewcommand\refname{\normalsize Publications}
%\nocite{ZeK971}
\bibliographystyle{agu04}
\bibliography{bib}
\printindex % Requires makeidx KoD95 p. 221

\csznote{ 
% Usage: Place usage here at end of file so comment character % not needed
cd ~/sw/crr;make -W psd.tex psd.dvi psd.ps psd.pdf psd.txt;cd -

# dvips -Ppdf -G0 -o ${DATA}/ps/psd.ps ~/sw/crr/psd.dvi;ps2pdf ${DATA}/ps/psd.ps ${DATA}/ps/psd.pdf
# cd ~/sw/crr;texcln psd;make psd.pdf;bibtex psd;makeindex psd;make psd.pdf;bibtex psd;makeindex psd;make psd.pdf;cd -
scp -p ~/sw/crr/psd.tex ~/sw/crr/psd.txt ~/sw/crr/psd.dvi ${DATA}/ps/psd.ps ${DATA}/ps/psd.pdf dust.ess.uci.edu:Sites/facts/psd

# NB: latex2html works well on psd.tex
# cd ~/sw/crr;latex2html -dir Sites/facts/psd psd.tex
# NB: tth works poorly on psd.tex
# cd ~/sw/crr;tth -a -Lpsd -p./:${TEXINPUTS}:${BIBINPUTS} < psd.tex > psd.html
# scp psd.html dust.ess.uci.edu:Sites/facts/psd
# NB: tex4ht works well on psd.tex
# cd ~/sw/crr;htlatex psd.tex # Works OK, index not linked
cd ~/sw/crr;htlatex psd.tex;makeindex -o psd.ind psd.4dx
cd ~/sw/crr;scp psd*.css psd*.html dust.ess.uci.edu:Sites/facts/psd
# NB: tex4moz works poorly on psd.tex
cd ~/sw/crr;/usr/share/tex4ht/mzlatex psd.tex
cd ~/sw/crr;scp psd*.css psd*.html psd*.xml dust.ess.uci.edu:Sites/facts/psd
} % end csznote on usage

\end{document}

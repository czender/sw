% $Id$

% Purpose: Theory and Practice of Radiative Transfer in the Earth System

% Copyright (c) 1998--present, Charles S. Zender
% Permission is granted to copy, distribute and/or modify this document
% under the terms of the GNU Free Documentation License (GFDL), Version 1.3
% or any later version published by the Free Software Foundation;
% with no Invariant Sections, no Front-Cover Texts, and no Back-Cover Texts.
% GFDL: http://www.gnu.org/copyleft/fdl.html

% The original author of this software, Charlie Zender, seeks to improve
% it with your suggestions, contributions, bug-reports, and patches.
% Charlie Zender <zender at uci dot edu>
% Department of Earth System Science
% University of California, Irvine
% Irvine, CA 92697-3100

% URL: http://dust.ess.uci.edu/facts/rt/rt.pdf

% Usage: 
% dvips -Ppdf -G0 -o ${DATA}/ps/rt.ps ~/sw/crr/rt.dvi;ps2pdf ${DATA}/ps/rt.ps ${DATA}/ps/rt.pdf
% cd ~/sw/crr;make rt.pdf;cd -
% cd ~/sw/crr;make rt.dvi rt.pdf rt.ps rt.txt;cd -
% cd ~/sw/crr;texcln rt;make rt.pdf;bibtex rt;makeindex rt;make rt.pdf;bibtex rt;makeindex rt;make rt.pdf;cd -
% scp -p ${HOME}/crr/rt.tex ${HOME}/crr/rt.txt ${HOME}/crr/rt.dvi ${DATA}/ps/rt.ps ${DATA}/ps/rt.pdf dust.ess.uci.edu:Sites/facts/rt

% # NB: latex2html works poorly on rt.tex
% 20010615: latex2html no longer works on rt.tex
% 20030625: latex2html chokes on math in tables in \csznote{}s
% 20050514: latex2html chokes on math in footnotes in tables
% cd ~/sw/crr;latex2html -dir Sites/facts/rt rt.tex
% # NB: tth works well on rt.tex
% cd ${HOME}/crr;tth -a -Lrt -p${TEXINPUTS}:${BIBINPUTS} < rt.tex > rt.html
% scp rt.html dust.ess.uci.edu:Sites/facts/rt
% # NB: tex4ht works well on rt.tex
% cd ${HOME}/crr;htlatex rt.tex
% cd ${HOME}/crr;scp rt*.css rt*.html dust.ess.uci.edu:Sites/facts/rt
% # NB: tex4moz works poorly on rt.tex
% cd ${HOME}/crr;/usr/share/tex4ht/mzlatex rt.tex
% cd ${HOME}/crr;scp rt*.css rt*.html rt*.xml dust.ess.uci.edu:Sites/facts/rt

% Plausible nomenclatural changes: 
% Ste94 discriminates between optical depth and optical thickness
% Optical depth = tau(z) = vertical optical depth
% Optical thickness = tau(s) = slant optical thickness
% Absorption/absorptivity/absorptance/absorbance:
% Absorption is process or absolute quantity, e.g., 100 W m-2
% Absorptance is fractional measure in [0.0,1.0]
% Absorptivity is as defined by Ramanathan
% Absorbance is defined by chemists as log10(I_o/I) (according to SGW)
% Same for reflection, transmission

\documentclass[12pt]{article}

% Standard packages
\usepackage{ifpdf} % Define \ifpdf
\ifpdf % PDFLaTeX
\usepackage{graphicx} % Defines \includegraphics*
\usepackage{thumbpdf} % Generate thumbnails
%\usepackage{epstopdf} % Convert .eps, if found, to .pdf when required
\else % !PDFLaTeX
\usepackage{graphicx} % Defines \includegraphics*
\fi % !PDFLaTeX
\usepackage{amsmath} % \subequations, \eqref, \align
\usepackage{array} % Table and array extensions, e.g., column formatting
\usepackage[dayofweek]{datetime} % \xxivtime, \ordinal
\usepackage[first]{draftcopy} % Blaze ``Draft'' across [none,first,all,etc.] pages
\usepackage{etoolbox} % \newbool, \setbool, \ifxxx
\usepackage{makeidx} % Index keyword processor: \printindex and \see
\usepackage{mdwlist} % Compact list formats \itemize*, \enumerate*
\usepackage{natbib} % \cite commands from aguplus
\usepackage{times} % Postscript Times-Roman font KoD99 p. 375
\usepackage{tocbibind} % Add Bibliography and Index to Table of Contents
\usepackage{url} % Typeset URLs and e-mail addresses

% hyperref is last package since it redefines other packages' commands
% hyperref options, assumed true unless =false is specified:
% backref       List citing sections after bibliography entries
% baseurl       Make all URLs in document relative to this
% bookmarksopen Unknown
% breaklinks    Wrap links onto newlines
% colorlinks    Use colored text for links, not boxes
% hyperindex    Link index to text
% plainpages=false Suppress warnings caused by duplicate page numbers
% pdftex        Conform to pdftex conventions
% Colors used when colorlinks=true:
% linkcolor     Color for normal internal links
% anchorcolor   Color for anchor text
% citecolor     Color for bibliographic citations in text
% filecolor     Color for URLs which open local files
% menucolor     Color for Acrobat menu items
% pagecolor     Color for links to other pages
% urlcolor      Color for linked URLs
\ifpdf % PDFLaTeX
\usepackage[backref,breaklinks,colorlinks,citecolor=blue,linkcolor=blue,urlcolor=blue,hyperindex,plainpages=false]{hyperref} % Hyper-references
\else % !PDFLaTeX
\usepackage[backref=false,breaklinks,colorlinks=false,hyperindex,plainpages=false]{hyperref} % Hyper-references
\fi % !PDFLaTeX

% preview-latex recommends it be last-activated package
\usepackage[showlabels,sections,floats,textmath,displaymath]{preview} % preview-latex equation extraction

% Personal packages
\usepackage{csz} % Library of personal definitions
\usepackage{abc} % Alphabet as three letter macros
\usepackage{dmn} % Dimensional units
% rt.tex does not include aer.sty (fxm: why not?)
\usepackage{aer} % Aerosol physics
\usepackage{chm} % Commands generic to chemistry
\usepackage{dyn} % Commands generic to fluid dynamics
\usepackage{rt} % Commands specific to radiative transfer
\usepackage{psd} % Particle size distributions
\usepackage{hyp} % Hyphenation exception list

% Commands which must be executed in preamble
\allowdisplaybreaks[1] % AMS-LaTeX command to allow multiline math environments to break across pages
%\listfiles % Print required files, versions
\makeglossary % Glossary described on KoD95 p. 221
\makeindex % Index described on KoD95 p. 220
\newcounter{idxpg} % [nbr] Page number of Index

% Commands specific to this file
% 1. Fundamental commands
\newcommand{\Angstrom}{{\AA}ngstr\"{o}m} % [sng] Ångström
\newcommand{\slndnm}{\ensuremath{\mathcal{D}}} % Denominator of two-stream solution
% 2. Derived commands
\newcommand{\slndnmstr}{\ensuremath{\slndnm^{\strsbs}}} % Denominator of two-stream solution
% 3. Doubly-derived commands

% Margins
\topmargin -24pt \headheight 12pt \headsep 12pt
\textheight 9in \textwidth 6.5in
\oddsidemargin 0in \evensidemargin 0in
%\marginparwidth 0pt \marginparsep 0pt
\setlength{\marginparwidth}{1.5in} % Width of callouts of index terms and page numbers KoD95 p. 220
\setlength{\marginparsep}{12pt} % Add separation for index terms KoD95 p. 220
\footskip 24pt
\footnotesep=0pt

\begin{document} % End preamble

\hypersetup{ % A command provided by \hyperref
pdftitle={Radiative Transfer in the Earth System},
pdfsubject={Radiative Transfer in the Earth System},
pdfauthor={Charlie Zender},
pdfkeywords={rt ess236 ess}
} % end \hypersetup

\begin{center}
Online: \url{http://dust.ess.uci.edu/facts} \hfill Built: \shortdate\today, \xxivtime\\
\bigskip
{\Large \textbf{Radiative Transfer in the Earth System}}\\
\bigskip
by Charlie Zender\\
University of California, Irvine\\
\end{center}
Department of Earth System Science \hfill \url{zender@uci.edu}\\
University of California \hfill Voice: (949)\thinspace 891-2429\\
Irvine, CA~~92697-3100 \hfill Fax: (949)\thinspace 824-3256

% GFDL legalities: http://www.gnu.org/copyleft/fdl.html
\bigskip\noindent
Copyright \copyright\ 1998--present, Charles S. Zender\\
Permission is granted to copy, distribute and/or modify this document
under the terms of the GNU Free Documentation License, Version~1.3
or any later version published by the Free Software Foundation;
with no Invariant Sections, no Front-Cover Texts, and no Back-Cover
Texts.
The license is available online at
\url{http://www.gnu.org/copyleft/fdl.html}.
\\
\\
\textbf{Facts about FACTs:}
This document is part of the 
\href{http://dust.ess.uci.edu/facts}{Freely Available Community Text (\acr{fact}) project}.  
\acr{fact}s are created, reviewed, and continuously maintained and updated
by members of the international academic community communicating with
eachother through a well-organized project website. 
\acr{fact}s are intended to standardize and disseminate our fundamental
knowledge of Earth System Sciences in a flexible, adaptive,
distributed framework which can evolve to fit the changing needs and
technology of the geosciences community. 
Currently available \acr{fact}s and their \acr{url}s are listed in
Table~\ref{tbl:fact}. 
\begin{table}[h] % h = ``here'' = position of table
\begin{minipage}{\hsize} % Minipage necessary for footnotes KoD95 p. 110 (4.10.4)
\renewcommand{\footnoterule}{\rule{\hsize}{0.0cm}\vspace{-0.0cm}} % KoD95 p. 111
\begin{center}
\caption[\acr{fact}s]{\textbf{Freely Available Community Texts}%
\label{tbl:fact}}
\vspace{\cpthdrhlnskp}
\begin{tabular}{ r l }
\hline \rule{0.0ex}{\hlntblhdrskp}% 
Format & \acr{URL} Location \\[0.0ex]
\hline \rule{0.0ex}{\hlntblntrskp}%
& \\[-1.0ex]
\multicolumn{2}{c}{\centering{Radiative Transfer in the Earth System}} \\[-0.5ex]
DVI & \url{http://dust.ess.uci.edu/facts/rt/rt.dvi} \\
PDF & \url{http://dust.ess.uci.edu/facts/rt/rt.pdf} \\
Postscript & \url{http://dust.ess.uci.edu/facts/rt/rt.ps} \\[1.0ex]
\multicolumn{2}{c}{Particle Size Distributions: Theory and Application to Aerosols,
Clouds, and Soils \hfill} \\[-0.5ex]
DVI & \url{http://dust.ess.uci.edu/facts/psd/psd.dvi} \\
PDF & \url{http://dust.ess.uci.edu/facts/psd/psd.pdf} \\
Postscript & \url{http://dust.ess.uci.edu/facts/psd/psd.ps} \\
\multicolumn{2}{c}{Natural Aerosols in the Climate System \hfill} \\[-0.5ex]
DVI & \url{http://dust.ess.uci.edu/facts/aer/aer.dvi} \\
PDF & \url{http://dust.ess.uci.edu/facts/aer/aer.pdf} \\
Postscript & \url{http://dust.ess.uci.edu/facts/aer/aer.ps} \\
%& \\[1.0ex]
\hline
\end{tabular}
\end{center}
\end{minipage}
\end{table}
Because of its international scope and availability to students
of all income levels, the \acr{fact} project may impact more students,
and to a greater depth, than imaginable to before the advent of the
Internet. 
If you are interested in learning more about \acr{fact}s and how you
might contribute to or benefit from the project please contact
\url{zender@uci.edu}. 

\pagenumbering{roman}
\setcounter{page}{1}
\pagestyle{headings}
\thispagestyle{empty}
%\onecolumn

\clearpage
\begin{center}
\bigskip
{\Large \textbf{Notes for Students of ESS~223,\\ Earth System Physics:}}\\
\bigskip
\end{center}
This monograph on Radiative Transfer provides some core and some
supplementary reading material for ESS~223.
We will discuss much of the material in the first twenty pages,
and the figures at the end.
\csznote{
% Equivalence to Hou02
Much of the material in \cite{Hou02}, Chapters~2 and~4, is described
at a more advanced level in the rest of this monograph.
\citeauthor{Hou02}~\S\,2.2 is our \S\,\ref{sxn:plk_fnc}, \ref{sxn:ext},
and \ref{sxn:str_atm};
\citeauthor{Hou02}~\S\,2.3 is our \S\,\ref{sxn:rte_plk_sln};
Part of \citeauthor{Hou02}~\S\,4.1 is our \S\,\ref{sxn:sct}; 
Part of \citeauthor{Hou02}~\S\,4.2 is our \S\,\ref{sxn:two_srm}; 
Part of \citeauthor{Hou02}~\S\,4.3 is our \S\,\ref{sxn:eqv_wdt} and \S\,\ref{sxn:lnshp}; 
\citeauthor{Hou02}~\S\,4.4 is our \S\,\ref{sxn:HCG} and \S\,\ref{sxn:lsd}; 
\citeauthor{Hou02}~\S\,4.5 is our \S\,\ref{sxn:rte_sln_frm}--\ref{sxn:rte_plk_sln}; 
This FACT has no equivalent to \citeauthor{Hou02}~\S\,4.6--4.8.
\citeauthor{Hou02}~\S\,4.9 is our \S\,\ref{sxn:nbm}; 
and \citeauthor{Hou02}~\S\,4.10 is our \S\,\ref{sxn:rad_frc}. 
% \stepcounter{idxpg} % [nbr] Page number of Index
} % end csznote
The Index beginning on page~\pageref{sxn:idx} is also helpful.
\begin{center}
\bigskip
{\Large \textbf{Notes for Students of ESS~236,\\ Radiative Transfer and Remote Sensing:}}\\
\bigskip
\end{center}
Yada yada yada.
\clearpage
\tableofcontents
\listoffigures
\listoftables
\pagenumbering{arabic}
\setcounter{page}{1}
%\markleft{Radiative Transfer}
%\markright{}
\thispagestyle{empty}

\section{Introduction}\label{sxn:ntr}

This document describes mathematical and computational considerations 
pertaining to radiative transfer processes and radiative transfer
models of the Earth system. 
Our approach is to present a detailed derivation of the tools of
radiative transfer needed to predict the radiative quantities
(irradiance, mean intensity, and heating rates) which drive climate. 
In so doing we begin with discussion of the intensity field which is
the quantity most often measured by satellite remote sensing
instruments. 
Our approach owes much to \cite{BoH83} (particle scattering), 
\cite{GoY89} (band models), and \cite{ThS99} (nomenclature,
discrete ordinate methods, general approach).
The nomenclature follows these authors where possible.
These sections will evolve and differentiate from their original 
sources as the manuscript takes on the flavor of the researchers who
contribute to it. 

\subsection{Planetary Radiative Equilibrium}\label{sxn:nrg_bln}
The important role that radiation plays in the climate system
is perhaps best illustrated by a simple example showing that
without atmospheric radiative feedbacks (especially, ironically, 
the greenhouse effect), our planet's mean temperature would be
well below the freezing point of water.
Earth is surrounded by the near vacuum of space so the only way to
transport energy to or from the planet is via radiative processes.
If $\nrg$ is the thermal energy of the planet, and $\flxabssw$ and
$\flxolr$ are the absorbed solar radiation and emitted longwave
radiation, respectively, then
\begin{eqnarray}
\frac{\partial \nrg}{\partial \tm} & = & \flxabssw - \flxolr
\label{eqn:nrg_bln_dfn}
\end{eqnarray}
On timescales longer than about a year the Earth as a whole is thought
to be in \trmidx{planetary radiative equilibrium}.
That, is, the global annual mean planetary temperature is nearly
constant because the absorbed solar energy is exactly compensated by
thermal radiation lost to space over the course of a year.
Thus
\begin{eqnarray}
\flxabssw & = & \flxolr
\label{eqn:rdn_eqm_dfn}
\end{eqnarray}
The total amount of solar energy available for the Earth to absorb is
the incoming solar flux (or \trmidx{irradiance}) at the top of Earth's
atmosphere, $\flxslrtoa$ (aka the \trmidx{solar constant}), times the
intercepting area of Earth's disk which is $\mpi \rdsrth^{2}$.
Since Earth rotates, the total mean incident flux $\mpi \rdsrth^{2}
\flxslrtoa$ is actually distributed over the entire surface area of the
Earth. 
The surface area of a sphere is four times its cross-sectional area
so the mean incident flux per unit surface area is $\flxslrtoa / 4$.
The fraction of incident solar flux which is reflected back to space,
and thus unable to heat the planet, is called the
a\trmidx{planetary albedo} or \trmidx{spherical albedo}, $\rfl$.
Satellite observations show that $\rfl \approx 0.3$.
Thus only $(1 - \rfl)$ of the mean incident solar flux contributes
to warming the planet and we have
\begin{eqnarray}
\flxabssw & = & (1 - \rfl) \flxslrtoa / 4
\label{eqn:flx_abs_sw_dfn}
\end{eqnarray}

Earth does not cool to space as a perfect blackbody
(\ref{eqn:plk_frq_dfn}) of a single temperature and emissivity.
Nevertheless the spectrum of thermal radiation $\flxolr$ which escapes
to space and thus cools Earth does resemble blackbody emission with a
characteristic temperature.
The \trmdfn{effective temperature} $\tptffc$ of an object is the
temperature of the blackbody which would produce the same irradiance.
Inverting the \trmidx{Stefan-Boltzmann Law} (\ref{eqn:stf_blt_dfn})
yields
\begin{eqnarray}
\tptffc & \equiv & ( \flxolr / \cststfblt )^{1/4}
\label{eqn:tpt_ffc_dfn}
\end{eqnarray}
For a perfect blackbody, $\tpt = \tptffc$.
For a planet, the difference between $\tptffc$ and the mean surface 
temperature $\tptsfc$ is due to the radiative effects of the overlying
atmosphere.
The insulating behavior of the atmosphere is more commonly known as
the \trmdfn{greenhouse effect}.

Substituting (\ref{eqn:flx_abs_sw_dfn}) and (\ref{eqn:tpt_ffc_dfn})
into (\ref{eqn:rdn_eqm_dfn})
\begin{eqnarray}
(1 - \rfl) \flxslrtoa / 4 & = & \cststfblt \tptffc^{4} \\
\tptffc & = & \left( \frac{(1 - \rfl) \flxslrtoa}
{4 \cststfblt} \right)^{1/4}
\label{eqn:tpt_ffc_rth}
\end{eqnarray}
For Earth, $\rfl \approx 0.3$ and $\flxslrtoa \approx 1367$\,\wxmS.
Using these values in (\ref{eqn:tpt_ffc_rth}) yields $\tptffc =
255$\,K\@.
Observations show the mean surface temperature $\tptsfc = 288$\,K\@.

\subsection[Fundamentals]{Fundamentals}\label{sxn:fnd}
The fundamental quantity describing the electromagnetic spectrum is
\trmdfn{frequency}, $\frq$.
Frequency measures the oscillatory speed of a system, counting the
number of oscillations (waves) per unit time.
Usually $\frq$ is expressed in cycles-per-second, or Hertz.
Units of Hertz may be abbreviated Hz, hz, cps, or, as we prefer, \xs.
Frequency is intrinsic to the oscillator and does not depend on the
medium in which the waves are travelling.
The \trmdfn{energy} carried by a photon is proportional to its
frequency 
\begin{eqnarray}
\nrg & = & \cstplk \frq
\label{eqn:nrg_dfn}
\end{eqnarray}
where $\cstplk$ is \trmdfn{Planck's constant}.
Regrettably, almost no radiative transfer literature expresses
quantities in frequency.

A related quantity, the \trmdfn{angular frequency} $\frqngl$ 
measures the rate of change of wave phase in radians per second.
Wave phase proceeds through $2\mpi$ radians in a complete cycle.
Thus the frequency and angular frequency are simply related
\begin{eqnarray}
\frqngl & = & 2\mpi \frq
\label{eqn:frq_ngl_dfn}
\end{eqnarray}
Since radians are considered dimensionless, the units of $\frqngl$ are
\xs. 
However, angular frequency is also rarely used in radiative transfer. 
Thus some authors use the symbol $\omega$ to denote the element 
of \trmidx{solid angle}, as in $\dfr\omega$.
The reader should be careful not to misconstrue the two meanings.
In this text we use $\omega$ only infrequently.

Radiative transfer literature in the visible, infrared, and microwave
regimes tends to use units of \trmdfn{wavelength},
\trmdfn{wavenumber}, and frequency, respectively.
Wavelength, $\wvl$ (m), measures the distance between two adjacent
peaks or troughs in the wavefield.
The universal relation between wavelength and frequency is
\begin{eqnarray}
\wvl \frq & = & \cstspdlgt
\label{eqn:wvl_dfn}
\end{eqnarray}
where $\cstspdlgt$ is the \trmdfn{speed of light}.
Since $\cstspdlgt$ depends on the medium, $\wvl$ also depends on the
medium. 

The \trmdfn{wavenumber} $\wvn$\,\xm, is exactly the inverse of wavelength 
\begin{eqnarray}
\wvn & \equiv & \frac{1}{\wvl} = \frac{\frq}{\cstspdlgt}
\label{eqn:wvn_dfn}
\end{eqnarray}
Thus $\wvn$ measures the number of oscillations per unit distance,
i.e., the number of wavecrests per meter.
Using (\ref{eqn:wvl_dfn}) in (\ref{eqn:wvn_dfn}) we find 
$\wvn = \frq/\cstspdlgt$ so wavenumber $\wvn$ is indeed proportional
to frequency (and thus to energy). 
Historically spectroscopists have favored $\wvn$ rather than $\wvl$
or $\frq$.
Because of this history, it is more common for the literature to
express $\wvn$ in CGS units of \xcm\ rather than SI units of \xm.
Hereafter, this monograph assumes that $\wvn$ is in \xcm\ unless
specifically stated otherwise.
The CGS wavenumber is used analogously to frequency and to wavelength,
i.e., to identify spectral regions.
The energy of radiative transitions are commonly expressed in CGS
wavenumber units. 
The relations between $\wvn$ expressed in CGS wavenumber units (\xcm)
and wavelength, frequency, and energy in SI units are obtained by using (\ref{eqn:wvn_dfn}) in 
(\ref{eqn:nrg_dfn})
\begin{subequations}
\label{eqn:plk_wien_dfn}
% cf ThS99 p. 94, KiK80 p. 94, GoY89 p. 30 (2.41)
\begin{align}
\label{eqn:nrg_wvn_dfn}
\frq & = \cstspdlgt / \wvl = 100 \cstspdlgt \wvn \\
\wvl & = \cstspdlgt / \frq = 1 / ( 100 \wvn ) \\
\wvn & = \frq / ( 100 \cstspdlgt ) = 1 / ( 100 \wvl ) \\
\nrg & = \cstplk \frq = \cstplk \cstspdlgt / \wvl = 100 \cstplk \cstspdlgt \wvn
\end{align}
\end{subequations}

There is another, distinct quantity also called \trmidx{wavenumber},
though more precisely called \trmidx{angular wavenumber}.
This secondary usage of wavenumber in this text is the traditional
measure of spatial wave propagation and is denoted by $\wvnbr$. 
\begin{eqnarray}
\wvnbr & \equiv & 2\mpi \tilde{\frq}
\label{eqn:wvnbr_dfn}
\end{eqnarray}
The wavenumber $\wvnbr$ is set in Roman typeface as an additional
distinction between it and other symbols
\footnote{The script $\kkk$ is already used for Boltzmann's constant, 
absorption coefficients, and vibrational modes}.

Table~\ref{tbl:wv_cnv} summarizes the relationships between the
fundamental parameters which describe wave-like phenomena.
\begin{table}
\begin{minipage}{\hsize} % Minipage necessary for footnotes KoD95 p. 110 (4.10.4)
\renewcommand{\footnoterule}{\rule{\hsize}{0.0cm}\vspace{-0.0cm}} % KoD95 p. 111
\begin{center}
\caption[Wave Parameter Conversion Table]{\textbf{Wave Parameter Conversion Table}%
% fxm: 20050513 latex2html appears to choke when math mode is used in footnotes
\footnote{The speed of light is $\cstspdlgt = 2.99792458 \times 10^{8}$\,\mxs.}%
\footnote{Table entries express the column in terms of the row.}%
\label{tbl:wv_cnv}}
\vspace{\cpthdrhlnskp}
\begin{tabular}{ >{$\displaystyle}r<{$} *{6}{>{$\displaystyle}c<{$}} } % KoD95 p. 94 describes '*' notation
\hline \rule{0.0ex}{\hlntblhdrskp}% 
\mbox{Variable} & \frq & \wvl & \wvn & \frqngl & \wvnbr & \tau \\[0.0ex]
\mbox{Units} & \mbox{\xs} & \mbox{m} & \mbox{\xcm} & \mbox{\xs} & \mbox{\xm} & \mbox{s} \\[0.0ex]
\hline \rule{0.0ex}{\hlntblntrskp}%
\frq & - & \frac{\cstspdlgt}{\frq} & \frac{\frq}{100\cstspdlgt} & 2\mpi\frq & \frac{2\mpi\frq}{\cstspdlgt} & \frac{1}{\frq} \\[3.0ex]
\wvl & \frac{\cstspdlgt}{\wvl} & - & \frac{1}{100\wvl} & \frac{2\mpi\cstspdlgt}{\wvl} & \frac{2\mpi}{\wvl} & \frac{\wvl}{\cstspdlgt} \\[3.0ex]
\wvn & 100\cstspdlgt\wvn & \frac{1}{100\wvn} & - & \frac{\wvn}{200\mpi\cstspdlgt} & 200\mpi\wvn & \frac{1}{100\cstspdlgt\wvn} \\[3.0ex]
\frqngl & \frac{\frqngl}{2\mpi} & \frac{2\mpi\cstspdlgt}{\frqngl} & \frac{\frqngl}{200\mpi\cstspdlgt} & - & \frac{\frqngl}{\cstspdlgt} & \frac{2\mpi}{\frqngl} \\[3.0ex]
\wvnbr & \frac{\wvnbr\cstspdlgt}{2\mpi} & \frac{2\mpi}{\wvnbr} & \frac{\wvnbr}{200\mpi} & \cstspdlgt\wvnbr & - & \frac{2\mpi}{\cstspdlgt\wvnbr} \\[3.0ex]
\tau & \frac{1}{\tau} & \cstspdlgt\tau & \frac{1}{100\cstspdlgt\tau} & \frac{2\mpi}{\tau} & \frac{2\mpi}{\cstspdlgt\tau} & - \\[3.0ex]
\hline
\end{tabular}
\end{center}
\end{minipage}
\end{table}

\section{Radiative Transfer Equation}\label{sxn:rte}

\subsection{Definitions}

\subsubsection[Intensity]{Intensity}\label{sxn:ntn}
The fundamental quantity defining the radiation field is the
\trmdfn{specific intensity} of radiation.
Specific intensity, also known as \trmdfn{radiance}, measures the flux
of radiant energy transported in a given direction per unit cross 
sectional area orthogonal to the beam per unit time per unit solid
angle per unit frequency (or wavelength, or wavenumber). 
The units of $\ntnwvl$ are
Joule meter$^{-2}$ second$^{-1}$ steradian$^{-1}$ meter$^{-1}$. 
In SI dimensional notation, the units condense to \jxmSssrm.
The SI unit of power (1\,Watt $\equiv$ 1\,Joule~per~second) is
preferred, leading to units of \wxmSsrm.
Often the specific intensity is expressed in terms of spectral
frequency $\ntnfrq$ with units \wxmSsrhz\ or
spectral wavenumber (also $\ntnwvn$) with units \wxmSsrwvn.

Consider light travelling in the direction $\nglhat$ through
the point~$\psnvct$.
Construct an infinitesimal element of surface area $\dfr\sfc$
intersecting $\psnvct$ and orthogonal to~$\nglhat$.
The radiant energy $\dfr\nrg$ crossing $\dfr\sfc$ in time
$\dfr\tm$ in the solid angle $\dfr\ngl$ in the frequency range $[\frq,
\frq+\dfr\frq]$ is related to $\ntnfrq(\psnvct,\nglhat)$ by
\begin{equation}
\dfr\nrg = \ntnfrq(\psnvct,\nglhat,\tm,\frq) \,\dfr\sfc \,\dfr\tm \,\dfr\ngl \,\dfr\frq
\label{eqn:ntn_dfn1}
\end{equation}
It is not convenient to measure the radiant flux across surface
orthogonal to $\nglhat$, as in (\ref{eqn:ntn_dfn1}), when we consider
properties of radiation fields with preferred directions.
If instead, we measure the intensity orthogonal to an arbitrarily
oriented surface element~$\dfr\xsx$ with surface normal~$\nrmhat$, 
then we must alter (\ref{eqn:ntn_dfn1}) to account for projection
of $\dfr\sfc$ onto~$\dfr\xsx$.
If the angle between $\nrmhat$ and $\nglhat$ is $\plr$ then
\begin{equation}
\cos \plr = \nrmhat \cdot \nglhat
\label{eqn:plr_dfn}
\end{equation}
and the projection of $\dfr\sfc$ onto $\dfr\xsx$ yields
\begin{equation}
\dfr\xsx = \cos \plr \,\dfr\sfc
\label{eqn:dxsx_dfn}
\end{equation}
so that
\begin{equation}
\dfr\nrg = \ntnfrq(\psnvct,\nglhat,\tm,\frq) 
\cos \plr \,\dfr\xsx \,\dfr\tm \,\dfr\ngl \,\dfr\frq
\label{eqn:ntn_dfn}
\end{equation}
The conceptual advantage that (\ref{eqn:ntn_dfn}) has over
(\ref{eqn:ntn_dfn1}) is that (\ref{eqn:ntn_dfn}) builds in the 
geometric factor required to convert to any preferred coordinate
system defined by~$\dfr\xsx$ and its normal~$\nrmhat$.
In practice $\dfr\xsx$ is often chosen to be the local horizon.

The radiation field is a seven-dimensional quantity, depending upon
three coordinates in space, one in time, two in angle, and one in
frequency.   
We shall usually indicate the dependence of spectral radiance and 
irradiance on frequency by using $\frq$ as a subscript, as in
$\ntnfrq$, in favor of the more explicit, but lengthier, notation
$\ntn(\frq)$. 
Three of the dimensions are superfluous to climate models and will
be discarded:
The time dependence of $\ntnfrq$ is a function of the atmospheric
state and solar zenith angle and will only be discussed further in
those terms, so we shall drop the explicitly dependence on $\tm$.
We reduce the number of spatial dimensions from three to one by
assuming a \trmdfn{stratified atmosphere} which is horizontally
homogeneous and in which physical quantities may vary only in the
vertical dimension $\zzz$.
Thus we replace $\psnvct$ by $\zzz$.
This approximation is also known as a \trmdfn{plane-parallel}
atmosphere, and comes with at least two important caveats:
The first is the neglect of horizontal photon transport which can be
important in inhomogeneous cloud and surface environments.
The second is the neglect of path length effects at large solar zenith
angles which can dramatically affect the mean intensity of the
radiation field, and thus the atmospheric photochemistry.

With these assumptions, the intensity is a function only of vertical
position and of direction, $\ntnfrq(\zzz,\nglhat)$.
Often the \trmidx{optical depth} $\tau$ (defined below), which
increases monotonically with $\zzz$, is used for the vertical
coordinate instead of $\zzz$. 
The angular direction of the radiation is specified in terms of the
polar angle $\plr$ and the azithumal angle $\azi$.
The polar angle $\plr$ is the angle between $\nglhat$ and the normal  
surface $\nrmhat$ that defines the coordinate system.
The specific intensity of radiation traveling at polar angle
$\plr$ and azimuthal angle $\azi$ at optical depth level $\tau$ in a
plane parallel atmosphere is denoted by $\ntnfrq(\tau,\plr,\azi)$.
Specific intensity is also referred to as intensity.

Further simplification of the intensity field is possible if it
meets certain criteria.
If $\ntnfrq$ is not a function of position ($\tau$), then the field is
\trmdfn{homogeneous}. 
If $\ntnfrq$ is not a function of direction ($\nglhat$), then the
field is \trmdfn{isotropic}. 

\subsubsection[Mean Intensity]{Mean Intensity}\label{sxn:ntn_bar}
The \trmdfn{mean intensity} is an integrated measure of the radiation
field at any point $\psn$.
Mean intensity $\ntnmnfrq$ is defined as the mean value of the
intensity field integrated over all angles.
\begin{eqnarray}
\ntnmnfrq = \frac{\int_{\ngl} \ntnfrq \,\dfr\ngl}{\int_{\ngl} \dfr\ngl}
\label{eqn:ntn_bar_dfn1}
\end{eqnarray}
The solid angle subtended by $\ngl$ is the ratio of the area $\AAA$
enclosed by $\ngl$ on a spherical surface to the square of the radius 
of the sphere. 
Since the area of a sphere is $4\mpi\rds^{2}$, there must be $4\mpi$
steradians in a sphere.
It is straightforward to demonstrate that the differential element of
area in \trmidx{spherical polar coordinates} 
is $\rds^{2} \sin \plr \,\dfr\plr \,\dfr\azi$. 
Thus the element of solid angle is
\begin{eqnarray}
\ngl & = & \AAA / \rds^{2} \nonumber \\
\dfr\ngl & = & \rds^{-2} \,\dfr\AAA \nonumber \\ 
& = & \rds^{-2} \, \rds^{2} \sin \plr \,\dfr\plr \,\dfr\azi \nonumber \\
& = & \sin \plr \,\dfr\plr \,\dfr\azi 
\label{eqn:ngl}
\end{eqnarray}
The \trmdfn{field of view} of an instrument, e.g., a telescope, is
most naturally measured by a solid angle.

The definition of $\ntnmnfrq$ (\ref{eqn:ntn_bar_dfn1}) demands the
radiation field be integrated over all angles, i.e., over all $4\mpi$
steradians. 
Evaluating the denominator demonstrates the properties of angular
integrals.  
The denominator of (\ref{eqn:ntn_bar_dfn1}) is
\begin{eqnarray}
\int_{\ngl} \dfr\ngl & = & \int_{\plr=0}^{\plr=\mpi} \int_{\azi=0}^{\azi=2\mpi} 
\sin \plr \,\dfr\plr \,\dfr\azi \nonumber \\
& = & \left[ \azi \right]_{0}^{2\mpi} \int_{\plr=0}^{\plr=\mpi} \sin
\plr \,\dfr\plr \nonumber \\
& = & 2 \mpi \int_{\plr=0}^{\plr=\mpi} \sin \plr \,\dfr\plr \nonumber \\
& = & 2 \mpi \left[ - \cos \plr \right]_{0}^{\mpi} \nonumber \\
& = & 2 \mpi [-(-1) - (-1)] \nonumber \\
& = & 4 \mpi
\label{eqn:srd_sph_dfn}
\end{eqnarray}
As expected, there are $4\mpi$ steradians in a sphere, and $2\mpi$
steradians in a hemisphere. 

It is convenient to return briefly to the definition of 
\trmdfn{isotropic radiation}.
Isotropic radiation is, by definition, equal intensity in all
directions so that the total emitted radiation is simply $4\mpi$  
times the intensity of emission in any direction.

Applying (\ref{eqn:srd_sph_dfn}) to (\ref{eqn:ntn_bar_dfn1}) yields
\begin{eqnarray}
\ntnmnfrq & = & \frac{1}{4 \mpi} \int_{\ngl} \ntnfrq \,\dfr\ngl
\label{eqn:ntn_bar_dfn}
\end{eqnarray}
$\ntnmnfrq$ has units of radiance, \wxmSsrhz.
If the radiation field is azimuthally independent (i.e., $\ntnfrq$
does not depend on $\azi$), then 
\begin{eqnarray}
\ntnmnfrq & = & \frac{1}{2} \int_{0}^{\mpi} \ntnfrq \sin \plr \,\dfr\plr
\label{eqn:ntn_bar_plr_dfn}
\end{eqnarray}

Let us simplify (\ref{eqn:ntn_bar_plr_dfn}) by introducing the change
of variables 
\begin{eqnarray}
\plru & = & \cos \plr \\
\dfr\plru & = & -\sin \plr \,\dfr\plr
\label{eqn:plru_dfn}
\label{eqn:cov_plru}
\end{eqnarray}
This maps $\plr \in [0, \mpi]$ into $\plru \in [1, -1]$ so that
(\ref{eqn:ntn_bar_plr_dfn}) becomes
\begin{eqnarray}
\ntnmnfrq & = & -\frac{1}{2} \int_{1}^{-1} \ntnfrq \,\dfr\plru \nonumber \\
\ntnmnfrq & = &  \frac{1}{2} \int_{-1}^{1} \ntnfrq \,\dfr\plru
\label{eqn:ntn_bar_plru_dfn}
\end{eqnarray}

The \trmdfn{hemispheric intensities} or \trmdfn{half-range
intensities} are simply the up- and downwelling components 
of which the full intensity is composed
\begin{subequations}
\label{eqn:ntn_hms_dfn}
\begin{align}
\ntnfrq(\tau,\nglhat) = \ntnfrq(\tau,\plr,\azi) & = \left\{ 
\begin{array}{r@{\quad:\quad}ll}
\ntnupwfrq(\tau,\plr,\azi) & 0 < \plr < \mpi/2 \\
\ntndwnfrq(\tau,\plr,\azi) & \mpi/2 < \plr < \mpi
\end{array} \right. \\
\ntnfrq(\tau,\nglhat) = \ntnfrq(\tau,\plru,\azi) & = \left\{ 
\begin{array}{r@{\quad:\quad}l}
\ntnupwfrq(\tau,\plru,\azi) & 0 \le u < 1 \\
\ntndwnfrq(\tau,\plru,\azi) & -1 < u < 0
\end{array} \right.
\end{align}
\end{subequations}

\begin{subequations}
\begin{align}
\label{eqn:ntn_hms_upw}
\ntnupwfrq(\tau,\plrmu,\azi) & = \ntnfrq(\tau,+\plrmu,\azi) 
= \ntnfrq(\tau,0 < \plr < \mpi/2,\azi) = \ntnfrq(\tau,0 < \plru <
1,\azi) \\
\label{eqn:ntn_hms_dwn}
\ntndwnfrq(\tau,\plrmu,\azi) & = \ntnfrq(\tau,+\plrmu,\azi) 
= \ntnfrq(\tau,\mpi/2 < \plr < \mpi,\azi) = \ntnfrq(\tau,-1 < \plru <
0,\azi) 
\end{align}
\end{subequations} 

\subsubsection[Irradiance]{Irradiance}\label{sxn:flx}
The spectral irradiance~$\flxfrq$ measures the radiant energy flux
transported through a given surface per unit area per unit time per
unit wavelength. 
Although it is somewhat ambiguous, ``flux'' is used a synonym for
irradiance, and has become deeply embedded in the literature
\cite[]{Mad87}.  
Consider a surface orthogonal to the $\nglhatprm$ direction.
All radiant energy travelling parallel to $\nglhatprm$ crosses this
surface and thus contributes to the irradiance with 100\% efficiency.
Energy travelling orthogonal to $\nglhatprm$ (and thus parallel to the
surface), however, never crosses the surface and does not contribute
to the irradiance.
In general, the intensity $\ntnfrq(\nglhat)$ projects onto the surface
with an efficiency $\cos \Theta = \nglhat \cdot \nglhatprm$, thus
\begin{eqnarray}
\flxfrq & = & \int_{\ngl} \ntnfrq \cos \plr \,\dfr\ngl \nonumber \\
& = & \int_{\plr=0}^{\plr=\mpi} \int_{\azi=0}^{\azi=2\mpi} 
\ntnfrq \cos \plr \, \sin \plr \,\dfr\plr \,\dfr\azi 
\label{eqn:flx_dfn}
\end{eqnarray}
In a plane-parallel medium, this defines the net specific irradiance
passing through a given vertical level. 
Note the similarity between (\ref{eqn:ntn_bar_dfn}) and
(\ref{eqn:flx_dfn}).  
The former contains the zeroth moment of the intensity with respect to
the cosine of the polar angle, the latter contains the first moment.
Also note that (\ref{eqn:flx_dfn}) integrates the cosine-weighted
radiance over all angles.
If $\ntnfrq$ is isotropic, i.e., $\ntnfrq = \ntnfrqnot$, then $\flxfrq
= 0$ due to the symmetry of $\cos \plr$. 

Let us simplify (\ref{eqn:flx_dfn}) by introducing the change of
variables $u =\cos \plr$, $du = -\sin \plr \,\dfr\plr$.
This maps $\plr \in [0, \mpi]$ into $\plru \in [1, -1]$: 
\begin{eqnarray}
\flxfrq & = & \int_{\plru=1}^{\plru=-1} \int_{\azi=0}^{\azi=2\mpi} 
\ntnfrq \plru \, (- \dfr\plru) \,\dfr\azi \nonumber \\
& = & \int_{\plru=-1}^{\plru=1} \int_{\azi=0}^{\azi=2\mpi} 
\ntnfrq \plru \,\dfr\plru \,\dfr\azi
\label{eqn:flx_udfn}
\end{eqnarray}

The irradiance per unit frequency, $\flxfrq$, is simply related to the
irradiance per unit wavelength, $\flxwvl$.
The total irradiance over any given frequency range, 
$[\frq,\frq+\dfr\frq]$, say, is $\flxfrq \,\dfr\frq$.
The irradiance over the same physical range when expressed in
wavelength, $[\wvl,\wvl-\dfr\wvl]$, say, is $\flxwvl \,\dfr\wvl$.
The negative sign is introduced since $-\dfr\wvl$ increases in the same
direction as $+\dfr\frq$.
Equating the total irradiance over the same region of
frequency/wavelength, we obtain
\begin{eqnarray}
\flxfrq \,\dfr\frq & = & - \flxwvl \,\dfr\wvl \nonumber \\
\flxfrq & = & - \flxwvl \, \frac{\dfr\wvl}{\dfr\frq} \nonumber \\
& = & - \flxwvl \, \frac{d}{\dfr\frq} \left( \frac{\cstspdlgt}{\frq} \right) \nonumber \\
& = & - \flxwvl \, \left( - \frac{\cstspdlgt}{\frq^{2}} \right) \nonumber \\
\flxfrq & = & \frac{\cstspdlgt}{\frq^{2}} \flxwvl =
\frac{\wvl^{2}}{\cstspdlgt} \flxwvl \\
\label{eqn:flx_frq_wvl}
\flxwvl & = & \frac{\cstspdlgt}{\wvl^{2}} \flxfrq =
\frac{\frq^{2}}{\cstspdlgt} \flxfrq 
\label{eqn:flx_wvl_frq}
\end{eqnarray}
Thus $\flxfrq$ and $\flxwvl$ are always of the same sign.

\subsubsection[Actinic Flux]{Actinic Flux}\label{sxn:flx_act}
A quantity of great importance in photochemistry is the total
convergence of radiation at a point.
This quantity, called the \trmdfn{actinic flux}, $\flxact$, determines
the availability of photons for photochemical reactions. 
By definition, the intensity passing through a point~$\pnt$ in the
direction $\nglhat$ within the solid angle $\dfr\ngl$ is 
$\ntnfrq\,\dfr\ngl$. 
We have not multiplied by $\cos \plr$ since we are interested in the
energy passing along $\nglhat$ (i.e., $\plr = 0$). 
The energy from all directions passing through~$\pnt$ is thus
\begin{eqnarray}
\flxactfrq & = & \int_{4\mpi} \ntnfrq \,\dfr\ngl \nonumber \\
& = & 4\mpi \ntnmnfrq
\label{eqn:flx_act_dfn}
\end{eqnarray}
Thus the actinic flux is simply $4\mpi$ times the mean
intensity~$\ntnmnfrq$ (\ref{eqn:ntn_bar_dfn}). 
$\flxactfrq$~has units of \wxmShz\ which are identical to the units of
irradiance~$\flxfrq$ (\ref{eqn:flx_dfn}). 
Although the nomenclature ``actinic flux'' is somewhat appropriate, it
is also somewhat ambiguous.
The ``flux'' measured by~$\flxactfrq$ at a point~$\pnt$ is the energy
convergence (per unit time, frequency, and area) through the surface
of the sphere containing~$\pnt$.  
This differs from the ``flux'' measured by~$\flxfrq$, which is the net
energy transport (per unit time, frequency, and area) through a
defined horizontal surface. 
Thus it is safest to use the terms ``actinic radiation field''
for~$\flxactfrq$ and ``irradiance'' for~$\flxfrq$. 
Unfortunately the literature is permeated with the ambiguous terms
``actinic flux'' and ``flux'', respectively.

The usefulness of actinic flux~$\flxactfrq$ becomes apparent only in
conjunction with additional, species-dependent data describing the
probability of photon absorption, or \trmdfn{photo-absorption}.
Photo-absorption is the process of molecules absorbing photons.
Each absorption removes energy (a photon) from the actinic radiation field.
The amount of photo-absorption per unit volume is proportional to the
number concentration of the absorbing species $\cncA$\,[\xmC], the
actinic radiation field~$\flxactfrq$, and the efficiency with with
each molecule absorbs photons.
This efficiency is called the \trmdfn{absorption cross-section},
\trmdfn{molecular cross section}, or simply \trmdfn{cross-section}.
The absorption cross-section is denoted by $\xsxabs$ and has units of 
[\xmS].
In the literature, however, values of $\xsxabs$ usually appear in CGS
units [\xcmS].
To make explicit the frequency-dependence of $\xsxabs$ we
write~$\xsxabsoffrq$.  
If~$\xsxabs$ depends significantly on temperature, too (as is true for 
ozone), we must consider $\xsxabs(\frq,\tpt)$.

The probability, per unit time, per unit frequency that a single
molecule of species~\A\ will absorb a photon with frequency in
$[\frq,\frq+\dfr\frq]$ is proportional to $\flxactfrq(\frq) \xsxabsoffrq$%
\footnote{The proportionality constant is $(\cstplk\frq)^{-1}$, i.e., 
when the actinic radiation field $\flxactfrq(\frq)$ is converted
from energy (\jxmSshz) to photons (\phtxmSshz) then 
$\flxactfrq(\frq) \xsxabsoffrq$ is the probability of absorption per
second per unit frequency}. 
Thus $\xsxabsoffrq$ is the effective cross-sectional area of a
molecule for absorption.  
The absorption cross-section is the ratio between the number of
photons (or total energy) absorbed by a molecule to the number 
(or total energy) per unit area convergent on the molecule.  
Let $\flxabsfrq$\,[\wxmC] be the energy absorbed per unit time, per
unit frequency, per unit volume of air.
Then
\begin{eqnarray}
\flxabsfrq(\frq) & = & \cncA \flxactfrq(\frq) \xsxabsoffrq
\label{eqn:flx_abs_dfn}
\end{eqnarray}
where $\cncA$\,\xmC\ is the number concentration of~\A.

Photochemists are interested in the probability of absorbed radiation  
severing molecular bonds, and thus decomposing species \AB\ into
constituent species \A\ and \B. 
Notationally this process may be written in any of the equivalent
forms 
\begin{rxnarray}
\AB + \hnu & \yields & \AAA + \BBB \nonumber \\
\AB + \hnu & \yields^{\frq > \frqnot} & \AAA + \BBB \nonumber \\
\AB + \hnu & \yields^{\wvl < \wvlnot} & \AAA + \BBB
\label{rxn:pch_AB}
\end{rxnarray}
Both forms indicate that the efficiency with which reaction
(\ref{rxn:pch_AB}) proceeds is a function of photon energy~$\hnu$.
The second form makes explicit that the photodissociation reaction
does not proceed unless $\frq < \frqnot$, where $\frqnot$ is 
the \trmdfn{photolysis cutoff frequency}.
In any case, photon energy is conventionally written~$\cstplk\frq$,
rather than the less convenient $\cstplk\cstspdlgt/\wvl$.

The probability that a photon absorbed by \AB\ will result in the 
photodissociation of \AB, and the completion of (\ref{rxn:pch_AB}), is 
called the \trmdfn{quantum yield} or \trmdfn{quantum efficiency} and
is represented by~$\qntyld$.
The probability~$\qntyld$ is dimensionless%
\footnote{Azimuthal angle and quantum yield are both
dimensionless quantities denoted by~$\qntyld$.
The meaning of $\qntyld$ should be clear from the context.}. 
In addition to its dependence on $\frq$, $\qntyld$ depends on
temperature~$\tpt$ for some important atmospheric reactions
(such as ozone photolysis).
We explicitly annotate the $\tpt$-dependence of~$\qntyld$ only for
pertinent reactions. 
Measurement of $\qntyldoffrq$ for all conditions and reactions of
atmospheric interest is an ongoing and important laboratory task.

The specific \trmdfn{photolysis rate coefficient} for the
photodissociation of a species \A\ is the number of photodissociations
of~\A\ occuring per unit time, per unit volume of air, per unit
frequency, per molecule of~\A. 
In accord with convention we denote the specific photolysis rate
coefficient by~$\prcfrq$.
The units of~$\prcfrq$ are \xshz.
\begin{eqnarray}
\prcfrq & = & 
\frac{\flxactfrq(\frq) \xsxabs(\frq) \qntyld(\frq)}{\cstplk \frq}
\nonumber \\
& = & 
\frac{4 \mpi \ntnmnfrq(\frq) \xsxabs(\frq) \qntyld(\frq)}{\cstplk \frq}
\label{eqn:prc_frq_dfn}
\end{eqnarray}
The photon energy in the denominator converts the energy per unit area
in~$\flxactfrq$ to units of photons per unit area.
The factor of~$\xsxabs$ turns this photon flux into a photo-absorption rate
per unit area.
The final factor, $\qntyld$, converts the photo-absorption rate
into a photodissociation rate coefficient.
Note that each factor in the numerator of (\ref{eqn:prc_frq_dfn}) 
requires detailed spectral knowledge, either of the radiation field or
of the photochemical behavior of the molecule in question.
This complexity is a hallmark of atmospheric photochemistry.

The total \trmdfn{photolysis rate coefficient} $\prc$ is obtained
by integrating (\ref{eqn:prc_frq_dfn}) over all frequencies that may
contribute to photodissociation 
\begin{eqnarray}
% ThS99 p. 164 (5.80)
\prc & = & \int_{\frq > \frqnot} \prcfrq(\frq) \,\dfr\frq
\nonumber \\
& = & \int_{\frq > \frqnot} 
\frac{\flxactfrq(\frq) \xsxabs(\frq) \qntyld(\frq)}
{\cstplk \frq} \,\dfr\frq \nonumber \\
& = & \frac{4\mpi}{\cstplk} 
\int_{\frq > \frqnot} 
\frac{\ntnmnfrq(\frq) \xsxabs(\frq) \qntyld(\frq)}{\frq} 
\,\dfr\frq
\label{eqn:prc_dfn}
\end{eqnarray}
As mentioned above, evaluation of (\ref{eqn:prc_dfn}) requires
essentially a complete knowledge of the radiative and photochemical
properties of the environment and species of interest.

$\prc$~is notoriously sensitive to uncertainty in the input quantities. 
Integration errors due to the discretization of (\ref{eqn:prc_dfn})
are quite common. 
To compute~$\prc$ with high accuracy, regular grids must have
resolution of $\sim$1\,nm in the ultraviolet, \cite[]{Mad89}.
Much of the difficulty is due to the steep but opposite gradients 
of $\flxactfrq$ and $\qntyld$ that occur in the ultraviolet.
High frequency features in~$\xsxabs$ worsen this problem for some
molecules.

The utility of~$\prc$ has motivated researchers to overcome these
computational difficulties by brute force techniques and by clever
parameterizations and numerical techniques
\cite[][]{CBI87,DaS91,TMA89,StT901,Pet95,LaC981,WZP00}. 
It is common to refer to photolysis rate coeffients as ``J-rates'',
and to affix the name of the molecule to specify which individual 
reaction is pertinent.
Another description for~$\prc$ is the first order rate coefficient 
in photochemical reactions.
For example, $\prcNOd$~is the first order rate coefficient for 
\begin{rxnarray}
\NOd + \hnu & \yields^{\wvl < 420\mathrm{\,nm}} & \NO + \Ou
\label{rxn:NO2+hv_NO+O}
\end{rxnarray}
If [\NOd] denotes the number concentration of \NOd\ 
in a closed system where photolysis is the only sink of \NOd, 
then
\begin{eqnarray}
\frac{\dfr[\NOd]}{\dfr\tm} & = & - \prcNOd [\NOd] + \SSS_{\NOd}
\label{eqn:prc_NO2_dfn}
\end{eqnarray}
where $\SSS_{\NOd}$ represents all sources of \NOd.
The terms in (\ref{eqn:prc_NO2_dfn}) all have dimensions of \xmCs.
The first term on the RHS is the \trmdfn{photolysis rate} of \NOd\
in the system. 

Figure~\ref{fgr:j_NO2_spc} shows the spectral distribution of actinic flux
in a clear mid-latitude summer atmosphere, and the absorption
cross-section and quantum yield of \NOd.
\begin{figure*}
\centering
\includegraphics[width=0.8\hsize]{/Users/zender/data/fgr/rt/j_NO2_act_flx}\vfill
\includegraphics[width=0.8\hsize]{/Users/zender/data/fgr/rt/j_NO2_abs_xsx}\vfill
\includegraphics[width=0.8\hsize]{/Users/zender/data/fgr/rt/j_NO2_qnt_yld}\vfill
\caption[Cross Section and Quantum Yield of Nitrogen Dioxide]{
(a)~Spectral distribution of actinic flux $\flxact$\,[\phtxmSsum] at TOA
and at the surface for a \trmidx{mid-latitude summer} (MLS) atmosphere
with a unit optical depth of dust or sulfate in the lowest kilometer.
(b)~Absorption cross section of \NOd, $\xsxabsNOd$\,[\mSxmlc].
(c)~Quantum yield of \NOd, $\qntyldNOd$ (\ref{rxn:NO2+hv_NO+O}).
\label{fgr:j_NO2_spc}}
\end{figure*}

Figure~\ref{fgr:j_NO2} shows the vertical distribution of
$\prcNOd$\,[\xs] for the conditions shown in
Figure~(\ref{fgr:j_NO2_spc}). 
\begin{figure*}
\centering
\includegraphics[width=0.8\hsize]{/Users/zender/data/fgr/rt/j_NO2_arese_19951011}\vfill
\includegraphics[width=0.8\hsize]{/Users/zender/data/fgr/rt/j_NO2_rlt_arese_19951011}\vfill
\caption[Vertical Distribution of Photodissociation Rates]{
Vertical distribution of $\prcNOd$\,[\xs] (\ref{rxn:NO2+hv_NO+O})
for the conditions shown in Figure~(\ref{fgr:j_NO2_spc}). 
(a)~Absolute rates. 
(b)~Rates normalized by clean sky rates.
\label{fgr:j_NO2}}
\end{figure*}

In this section we have assumed the quantities $\flxactfrq$,
$\xsxabs$, and $\qntyld$ are somehow known and therefore available to 
use to compute $\prc$.
Typically, $\xsxabs$ and $\qntyld$ are considered known quantities
since they usually do not vary with time or space.
Models may store their values in lookup tables or precompute their
contributions to (\ref{eqn:prc_dfn}).
The essence of forward radiative problems is to determine $\ntnwvn$ so  
that quantities such as $\prcfrq$ and $\flxabs$ may be determined. 
In inverse radiative transfer problems, which are encountered in much
of remote sensing, both $\prc$ and the species concentration are
initially unknown and must be determined. 
We shall continue describing the methods of forward radiative transfer
until we have tools at our disposal to solve for $\prcfrq$.
At that point we shall re-visit the inverse problem.

\subsubsection[Actinic Flux Enhancement]{Actinic Flux Enhancement}\label{sxn:flx_act_cld}
The actinic flux~$\flxactfrq$ (\ref{eqn:flx_act_dfn}) is sensitive to
the angular distribution of radiance~$\ntnfrq$.
Nearly all scattering processes diffuse the radiation field, i.e., 
convert collimated photons to more isotropic photons.
Such diffusion causes \trmdfn{actinic flux enhancement}.
It is instructive to examine how the relationship between downwelling
flux and actinic flux changes in the presence of scattering.
Four limiting cases may be identified and are summarized in
Table~\ref{tbl:flxact}. 
\begin{table}
\begin{minipage}{\hsize} % Minipage necessary for footnotes KoD95 p. 110 (4.10.4)
\renewcommand{\footnoterule}{\rule{\hsize}{0.0cm}\vspace{-0.0cm}} % KoD95 p. 111
\begin{center}
\caption[Actinic Flux Enhancement]{\textbf{Actinic Flux Enhancement by
Scattering}%
\footnote{Terminology: $\flxdwnfrq$ is downwelling irradiance of
source, $\rfllmboffrq$ is Lambertian reflectance of surface (or
cloud), $\ntnfrqofngl$ is resulting intensity field, $\flxupwfrq$ is
upwelling irradiance, $\ntnmnfrq$ is mean intensity, $\flxactfrq$ is
actinic flux.}%
\label{tbl:flxact}}
\vspace{\cpthdrhlnskp}
\begin{tabular}{ >{\raggedright}p{8em}<{} *{6}{>{$\displaystyle}c<{$}} } % KoD95 p. 94 describes '*' notation
\hline \rule{0.0ex}{\hlntblhdrskp}% 
Description & \flxdwnfrq & \rfllmboffrq & \ntnfrqofngl & \flxupwfrq & \ntnmnfrq & \flxactfrq \\[0.0ex]
\hline \rule{0.0ex}{\hlntblntrskp}%
Collimated, non-reflecting & \flxslrfrq & 0 & \flxslrfrq\dltfncofnglhatmnglhatnot & 0 & \frac{\flxslrfrq}{4\mpi} & \flxslrfrq \\[1.0ex]
Isotropic, non-reflecting & \flxslrfrq & 0 & 
\begin{array}{rcl}
\ntndwnfrq & = & \flxslrfrq/\mpi \\
\ntnupwfrq & = & 0
\end{array}
& 0 & \frac{\flxslrfrq}{2\mpi} & 2 \flxslrfrq \\[1.0ex]
Collimated, reflecting & \flxslrfrq & 1 & 
\begin{array}{rcl}
\ntndwnfrq & = & \flxslrfrq\dltfncofnglhatmnglhatnot \\
\ntnupwfrq & = & \flxslrfrq/\mpi
\end{array}
& \flxslrfrq & \frac{3\flxslrfrq}{4\mpi} & 3\flxslrfrq \\[1.0ex]
Isotropic, reflecting & \flxslrfrq & 1 & \flxslrfrq/\mpi & \flxslrfrq & \frac{\flxslrfrq}{\mpi} & 4 \flxslrfrq \\[1.0ex]
\hline
\end{tabular}
\end{center}
\end{minipage}
\end{table}
The scenarios differ in the isotropicity of the downwelling radiance
(\trmidx{collimated} or \trmidx{isotropic}) and the
\trmidx{reflectance}~$\rfllmboffrq$ (0 or~1) of the lower boundary,
taken to be a \trmidx{Lambertian surface}.
All scenarios are driven by the same downwelling irradiance, 
taken to be the direct solar beam.
The scenarios are arranged in order of increasing actinic
flux~$\flxactfrq$, shown in the final column.  
$\flxactfrq$~increases with both the number and the
brightness of reflecting surfaces.
The reflectivity of natural surfaces is no more than 90\% nor less than
5\%\footnote{Glaciers are the most reflective surfaces in nature, with
$\rfllmboffrq \lesssim 0.9$. 
Maximum cloud reflectance is $\lesssim 0.7$.
Dark forests and ocean have $\rfllmboffrq \gtrsim 0.05$.
See discussion in \S\ref{sxn:lmb} and Table~\ref{tbl:alb_sfc}.}, so that 
the $\flxactfrq$ in Table~\ref{tbl:flxact} represent bounds on 
realistic systems.

The Collimated-Non-Reflecting scenario assumes all light travels
unidirectionally in a tightly collimated direct beam with irradiance
$\flxfrq = \flxslrfrq$. 
In this case the radiation field is the delta-function in the
direction of the solar beam
$\ntnfrqofngl = \flxslrfrq\dltfncofnglhatmnglhatnot$.
The closest approximation to this scenario in the natural environment
is the pristine atmosphere high above the ocean in daylight.
The actinic flux $\flxactfrq = \flxslrfrq$ follows directly from
(\ref{eqn:flx_act_dfn}).
We define the actinic flux enhancement $\flxactfct$ of a medium as 
the ratio of the actinic flux to the actinic flux of a collimated
beam with the same incident irradiance
\begin{eqnarray}
\flxactfct & \equiv & \flxactfrq/\flxdwnfrq
\label{eqn:flx_act_fct_dfn}
\end{eqnarray}
Table~\ref{tbl:flxact} shows $\flxactfct$ ranges from one for the
direct solar beam to a maximum of four in a completely isotropic
radiation field. 
Thus a collimated beam is the least efficient configuration of radiant 
energy for driving photochemistry.
Scattering processes diffuse the radiation field, and, as a result,
always enhance the photolytic efficiency of a given irradiance.
Absorption in a medium (i.e., in the atmosphere or by the surface)
always reduces $\flxactfrq$ and may lead to $\flxactfct < 1$.

The Isotropic-Non-Reflecting scenario assumes isotropic downwelling 
radiance above a completely black surface.
Under these conditions, the actinic flux is twice the incident
irradiance because the photons are evenly distributed over the
hemisphere rather than collimated. 
Moderately thick clouds ($\tau \gtrsim 3$) over a dark surface
such as the ocean create a radiance field approximately like this.
However, because clouds are efficient at diffusing downwelling
irradiance, they are also efficient reflectors and this significantly
reduces the incident irradiance $\flxdwnfrq$ relative to the total
extraterrestrial irradiance $\flxslrfrq$. 
Whether photochemistry is enhanced or diminished beneath real clouds
depends on whether the actinic flux enhancement factor 
$\flxactfrq = 2$ compensates the reduced sub-cloud insolation due to
photons up-scattering off the cloud and back to space.

The Collimated-Reflecting scenario is a very important limit in nature
because the net effect on photochemistry can approach the theoretical
photochemical enhancement of $\flxactfct = 3$.
In this limit, collimated downwelling radiation and diffuse upwelling
radiation combine to drive photochemistry from both hemispheres.
A clear atmosphere above bright surfaces (clouds, desert, snow)
approaches this limit. 
These conditions describe a large fraction of the atmosphere, which is
50--60\% cloud-covered.  
Moreover, the incident flux is not attenuated by cloud transmission,
so $\flxdwnfrq \approx \flxslrfrq$.
The relatively high frequency of occurance of this scenario, combined
with the large photochemical enhancement of $\flxactfct = 3$, are
unequalled by any other scenario.

We may also define the actinic flux efficiency $\flxactfsh$ of a
medium as the actinic flux relative to the actinic flux of an
isotropic radiation field with the same incident irradiance 
\begin{eqnarray}
\flxactfsh & \equiv & \frac{\flxactfrq}{4 \flxdwnfrq}
\label{eqn:flx_act_fsh_dfn}
\end{eqnarray}
Clearly $0 \le \flxactfsh \le 1$.
Table~\ref{tbl:flxact} shows $\flxactfsh = 0.25$ for a collimated
beam, and $\flxactfsh = 1$ for isotropic radiation.
Just as scattering of the solar beam is required to increase
$\flxactfsh$ above $0.25$, absorption must be present to reduce
$\flxactfsh$ beneath $0.25$.

The Isotropic-Reflecting scenario shows that an isotropic radiation
field is most efficient for driving photochemistry.
The maximum four-fold increase in efficiency relative to the
collimated field arises from the radiation field interacting with the
particle from all directions, rather than from one direction only.
This is geometrically equivalent to multiplying the molecular cross
section by a factor of four, the ratio between the surface and
cross-sectional areas of a sphere.
Note that photochemistry itself is driven by molecular absorption
which reduces $\flxactfrq$ from the values in Table~\ref{tbl:flxact}.  

In summary, we have learned that a reactant molecule with a spherically
symmetric field of influence receives photochemical radiation much
like Earth receives solar irradiance.
In both cases the collimated beam intercepts one fourth of the total
area of matter (molecule or Earth), while an equal flux of diffuse (or
diurnal average) irradiance impinges on four times as much area.
Since, in the geometric limit, absorption probability depends upon
area, not direction, collimated beams have one fourth the
photochemical potential as isotropic radiation.

\subsubsection[Energy Density]{Energy Density}\label{sxn:nrg_dns}
Another quantity of interest is the density of radiant energy per unit
volume of space.
We call this quantity the \trmdfn{energy density} $\nrgdnsfrq$.
The energy density is the number of photons per unit volume in the
frequency range $[\frq,\frq + \dfr\frq]$ times the energy per photon, 
$\cstplk \frq$.
$\nrgdnsfrq$ is simply related to the \trmidx{actinic flux}
$\flxactfrq$ (\ref{eqn:flx_act_dfn}) and thus to the mean intensity
$\ntnmnfrq$.
\begin{eqnarray}
% ThS99 p. 40 (2.9)
\nrgdnsfrq & = & \int_{4\mpi} \,\dfr\nrgdnsfrq \nonumber \\
& = & \frac{4\mpi}{\cstspdlgt} \ntnmnfrq
\label{eqn:nrg_dns_dfn}
\end{eqnarray}
The units of $\nrgdnsfrq$ are \jxmChz.

\subsubsection[Spectral vs.\ Broadband]{Spectral vs.\ Broadband}\label{sxn:spc_bb}
Until now we have considered only spectrally dependent quantities such
as the spectral radiance $\ntnfrq$, spectral irradiance $\flxfrq$,
spectral actinic flux $\flxactfrq$, and spectral energy density
$\nrgdnsfrq$. 
These quantities are called \trmdfn{spectral} and are given a
subscript of $\frq$, $\wvl$, or $\wvn$ because they are expressed per
unit frequency, wavelength, or wavenumber, respectively.  
Each spectral radiant quantity may be integrated over a frequency
range to obtain the corresponding \trmdfn{band-integrated} radiant
quantity.
Band-integrated radiant fields are often called \trmdfn{narrowband} 
or \trmdfn{broadband} Depending on the size of the frequency range, 
\trmdfn{broadband} radiant
are obtained by integrating over all frequencies:
\begin{eqnarray}
\ntn & = & \int_{0}^{\infty} \ntnfrq(\frq) \,\dfr\frq \nonumber \\
\flx & = & \int_{0}^{\infty} \flxfrq(\frq) \,\dfr\frq \nonumber \\
\flxact & = & \int_{0}^{\infty} \flxactfrq(\frq) \,\dfr\frq \nonumber \\
\flxabs & = & \int_{0}^{\infty} \flxabsfrq(\frq) \,\dfr\frq \nonumber \\
\prc & = & \int_{0}^{\infty} \prcfrq(\frq) \,\dfr\frq \nonumber \\
\nrgdns & = & \int_{0}^{\infty} \nrgdnsfrq(\frq) \,\dfr\frq \nonumber \\
\htr & = & \int_{0}^{\infty} \htrfrq(\frq) \,\dfr\frq \nonumber
\label{eqn:bb_dfn}
\end{eqnarray}

\subsubsection[Thermodynamic Equilibria]{Thermodynamic Equilibria}\label{sxn:tdy_eqm}
Temperature plays a fundamental role in radiative transfer because 
$\tpt$ determines the population of excited atomic states, which
in turn determines the potential for \trmdfn{thermal emission}.
Thermal emission occurs as matter at any temperature above absolute
zero undergoes quantum state transitions from higher energy to lower
energy states. 
The difference in energy between the higher and lower level states
is transferred via the electromagnetic field by photons. 
Thus an important problem in radiative transfer is quantifying
the contribution to the radiation field from all emissive matter
in a physical system.
For the atmosphere the system of interest includes, e.g., clouds,
aerosols, and the surface.

To develop this understanding we must discuss various forms of
energetic equilibria in which a physical system may reside.
Earth (and the other terrestrial planets, Mercury, Venus, and Mars)
are said to be in \trmdfn{planetary radiative equilibrium}
because, on an annual timescale the solar energy absorbed by the
Earth system balances the thermal energy emitted to space by Earth. 
Radiation and matter inside a constant temperature enclosure are
said to be in \trmdfn{thermodynamic equilibrium}, or~TE\@.

Radiation in thermodynamic equilibrium with matter plays a fundamental 
role in radiation transfer. 
Such radiation is most commonly known as \trmdfn{blackbody radiation}.  
Kirchoff first deduced the properties of blackbody radiation.

Thermodynamic equilibrium (TE) is an idealized state, but,
fortunately, the properties of radiation in TE can be shown to apply
to a less restrictive equilibrium known as \trmdfn{local thermodynamic  
equilibrium}, or LTE\@.  

\subsubsection[Planck Function]{Planck Function}\label{sxn:plk_fnc}
The Planck function $\plkfrq$ describes the intensity of blackbody
radiation as a function of temperature and wavelength
% 19991011: subequation/align environment is more cramped than equivalent eqnarray
\begin{subequations}
\label{eqn:plk_dfn}
\begin{align}
\label{eqn:plk_frq_dfn}
\plkfrq(\tpt,\frq) & = \frac{2 \cstplk \frq^{3}}
       {\cstspdlgt^{2}( \me^{\cstplk \frq / \cstblt \tpt} - 1)},
       \qquad \mbox{\frq\ in \Hz, \plkfrq\ in \Hz$^{-1}$} \\
\label{eqn:plk_wvl_dfn}
\plkwvl(\tpt,\wvl) & = \frac{2 \cstplk \cstspdlgt^{2}}
       {\wvl^{5} ( \me^{\cstplk \cstspdlgt / \wvl \cstblt \tpt} - 1)},
       \qquad \mbox{\wvl\ in m, \plkwvl\ in m$^{-1}$} \\ 
\label{eqn:plk_wvn_dfn}
\plkwvn(\tpt,\wvn) & = \frac{2 \cstplk \cstspdlgt^{2} \wvn^{3}}
       {\me^{\cstplk \cstspdlgt \wvn / \cstblt \tpt} - 1},
       \qquad \mbox{\wvn\ in \xm, \plkwvn\ in (\xm)$^{-1}$} \\
\label{eqn:plk_wvn_cgs_dfn}
\plkwvn(\tpt,\wvn) & = \frac{2 \times 10^{8} \cstplk \cstspdlgt^{2} \wvn^{3}}
       {\me^{100 \cstplk \cstspdlgt \wvn / \cstblt \tpt} - 1},
       \qquad \mbox{\wvn\ in \xcm, \plkwvn\ in (\xcm)$^{-1}$}
\end{align}
\end{subequations}
Blackbodies emit isotropically.
This considerably simplifies thermal radiative transfer. 
The correct predictions (\ref{eqn:plk_frq_dfn}) resolved one of the
great mysteries in experimental physics in the late 19th century.
In fact, this discovery marked the beginning of the science of quantum 
mechanics. 

The relations (\ref{eqn:plk_frq_dfn})--(\ref{eqn:plk_wvn_cgs_dfn})
predict slightly different quantities:
$\plkfrq$ predicts the blackbody radiance per unit frequency,
$\plkwvl$ predicts the blackbody radiance per unit wavelength,
and the two expressions for $\plkwvn$ predict blackbody radiance per
unit wavenumber in SI units (\xm) and CGS units (\xcm), respectively.
Of course these quantities are related since the blackbody energy
within any given spectral band must be the same regardless of which
formula describes it.
Expressed mathematically, this constraint means
\begin{eqnarray}
\plkfrq \,\dfr\frq & = & - \plkwvl \,\dfr\wvl
\label{eqn:plk_frq_wvl_eqv}
\end{eqnarray}
Once again, the negative sign arises as a result of the opposite
senses of increasing frequency versus increasing wavelength.
We may derive (\ref{eqn:plk_wvl_dfn}) from (\ref{eqn:plk_frq_dfn}) 
by using (\ref{eqn:wvl_dfn}) in (\ref{eqn:plk_frq_wvl_eqv}) 
\begin{eqnarray}
\plkfrq & = & - \plkwvl \, \frac{\dfr\wvl}{\dfr\frq} \nonumber \\
& = & \frac{\plkwvl \cstspdlgt}{\frq^{2}} \nonumber \\
& = & \plkwvl \cstspdlgt \left( \frac{\wvl^{2}}{\cstspdlgt^{2}} \right) \nonumber \\
\plkfrq & = & \frac{\wvl^{2}}{\cstspdlgt} \plkwvl = 
\frac{\cstspdlgt}{\frq^{2}} \plkwvl \\
\label{eqn:plk_frq_wvl_dfn}
\plkwvl & = & \frac{\frq^{2}}{\cstspdlgt} \plkfrq =
\frac{\cstspdlgt}{\wvl^{2}} \plkfrq
\label{eqn:plk_wvl_frq_dfn}
\end{eqnarray}
These relations are analogous to (\ref{eqn:flx_frq_wvl}).

The Planck function (\ref{eqn:plk_dfn}) implicitly defines the
\trmdfn{brightness temperature}.
The brightness temperature $\tptbrt$ associated with an object with
that emits a known radiance $\plkwvl$ or $\plkfrq$ is the temperature
of the blackbody that would produce that radiance.
Hence the brightness temperature is to radiance as the effective
temperature is to flux.
Both are characteristics are measured in terms of temperature, yet
neither is the actual kinetic temperature of the object unless the
object is a blackbody.
These characteristics are most useful when the emissivity of the
object is unknown.
Inverting (\ref{eqn:plk_dfn}) yields the three analytic forms of
$\tptbrt$: 
\begin{subequations}
\label{eqn:tpt_brt_dfn}
\begin{align}
\label{eqn:tpt_brt_frq_dfn}
\plkfrq(\tptbrt,\frq) & = \frac{2 \cstplk \frq^{3}}{\cstspdlgt^{2} 
( \me^{\cstplk \frq / \cstblt \tptbrt} - 1)} \nonumber \\
\me^{\cstplk \frq / \cstblt \tptbrt} - 1 & =
\frac{2 \cstplk \frq^{3}}{\plkfrq \cstspdlgt^{2}} \nonumber \\
\frac{\cstplk \frq}{\cstblt \tptbrt} & =
\ln \left( 1 + \frac{ 2 \cstplk \frq^{3}}{\plkfrq \cstspdlgt^{2}} \right) \nonumber \\
\tptbrt(\plkfrq,\frq) & = \frac{\cstplk \frq}{\cstblt}
\left[ \ln \left( 1 + \frac{ 2 \cstplk \frq^{3}}{\plkfrq \cstspdlgt^{2}} \right) \right]^{-1},
\qquad \mbox{\frq\ in \Hz, \plkfrq\ in \Hz$^{-1}$} \\
\label{eqn:tpt_brt_wvl_dfn}
\plkwvl(\tptbrt,\wvl) & = \frac{2 \cstplk \cstspdlgt^{2}}{\wvl^{5} 
( \me^{\cstplk \cstspdlgt / \wvl \cstblt \tptbrt} - 1)} \nonumber \\
\me^{\cstplk \cstspdlgt / \wvl \cstblt \tptbrt} - 1 & =
\frac{2 \cstplk \cstspdlgt^{2}} {\plkwvl \wvl^{5}} \nonumber \\
\frac{\cstplk \cstspdlgt}{\wvl \cstblt \tptbrt} & =
\ln \left( 1 + \frac{2 \cstplk \cstspdlgt^{2}}{\plkwvl \wvl^{5}} \right) \nonumber \\
\tptbrt(\plkwvl,\wvl) & = \frac{\cstplk \cstspdlgt}{\wvl \cstblt}
\left[ \ln \left( 1 + \frac{2 \cstplk \cstspdlgt^{2}}{\plkwvl \wvl^{5}} \right) \right]^{-1},
\qquad \mbox{\wvl\ in m, \plkwvl\ in m$^{-1}$} \\ 
\label{eqn:tpt_brt_wvn_dfn}
\plkwvn(\tptbrt,\wvn) & = \frac{2 \cstplk \cstspdlgt^{2} \wvn^{3}}
{\me^{\cstplk \cstspdlgt \wvn / \cstblt \tptbrt} - 1} \nonumber \\
\me^{\cstplk \cstspdlgt \wvn / \cstblt \tptbrt} - 1 & =
\frac{2 \cstplk \cstspdlgt^{2} \wvn^{3}}{\plkwvn} \nonumber \\
\frac{\cstplk \cstspdlgt \wvn}{\cstblt \tptbrt} & =
\ln \left( 1 + \frac{2 \cstplk \cstspdlgt^{2} \wvn^{3}}{\plkwvn} \right) \nonumber \\
\tptbrt(\plkwvn,\wvn) & = \frac{\cstplk \cstspdlgt \wvn}{\cstblt}
\left[ \ln \left( 1 + \frac{2 \cstplk \cstspdlgt^{2} \wvn^{3}}{\plkwvn} \right) \right]^{-1},
\qquad \mbox{\wvn\ in \xm, \plkwvn\ in (\xm)$^{-1}$} \\
\label{eqn:tpt_brt_wvn_cgs_dfn}
\tptbrt(\plkwvn,\wvn) & = \frac{100 \cstplk \cstspdlgt \wvn}{\cstblt}
\left[ \ln \left( 1 + \frac{2 \times 10^{8} \cstplk \cstspdlgt^{2} \wvn^{3}}{\plkwvn} \right) \right]^{-1},
\qquad \mbox{\wvn\ in \xcm, \plkwvn\ in (\xcm)$^{-1}$}
\end{align}
\end{subequations}

The Planck function (\ref{eqn:plk_dfn}) has interesting behavior in
both the high and the low energy photon limits. 
In the high energy limit, known as \trmdfn{Wien's limit},
the photon energy greatly exceeds the ambient thermal energy
\begin{eqnarray}
\cstplk \frqmshmax \gg \cstblt \tpt
\label{eqn:wien_dfn}
\end{eqnarray}
In Wien's limit, (\ref{eqn:plk_dfn}) becomes
\begin{subequations}
\label{eqn:plk_wien_dfn}
% cf ThS99 p. 94, KiK80 p. 94, GoY89 p. 30 (2.41)
\begin{align}
\label{eqn:plk_frq_wien_dfn}
\plkfrq(\tpt,\frq) = & \frac{2 \cstplk \frq^{3}}{\cstspdlgt^{2}}
\me^{-\cstplk \frq / \cstblt \tpt} \\
\label{eqn:plk_wvl_wien_dfn}
\plkwvl(\tpt,\wvl) = & \frac{2 \cstplk \cstspdlgt^{2}}{\wvl^{5}} 
\me^{-\cstplk \cstspdlgt / \wvl \cstblt \tpt}
\end{align}
\end{subequations}

In the very low energy limit, known as the \trmdfn{Rayleigh-Jeans
limit}, the photon energy is much less than the ambient thermal energy 
\begin{eqnarray}
\cstplk \frqmshmax & \ll & \cstblt \tpt
\label{eqn:ryl_jean_dfn}
\end{eqnarray}
Thus in the Rayleigh-Jeans limit the arguments to the exponential in
(\ref{eqn:plk_dfn}) are less than~$1$ so the exponentials may be
expanded in Taylor series.
Starting from (\ref{eqn:plk_frq_dfn}) 
\begin{eqnarray}
% cf ThS99 p. 94, KiK80 p. 94
\plkfrq(\tpt,\frq) & \approx & 
\frac{2 \cstplk \frq^{3}}{\cstspdlgt^{2}}
\left(1 + \frac{\cstplk \frq}{\cstblt \tpt} -1 \right)^{-1}
\nonumber \\
& = & 
\frac{2 \cstplk \frq^{3} \cstblt \tpt}{\cstspdlgt^{2} \cstplk \frq}
\nonumber \\
& = & 
\frac{2 \frq^{2} \cstblt \tpt}{\cstspdlgt^{2}}
\end{eqnarray}
Similar manipulation of (\ref{eqn:plk_wvl_dfn}) may be performed 
and we obtain
\begin{subequations}
\label{eqn:plk_ryl_jean_dfn}
\begin{align}
\label{eqn:plk_frq_ryl_jean_dfn}
\plkfrq(\tpt,\frq) & \approx  
\frac{2 \frq^{2} \cstblt \tpt}{\cstspdlgt^{2} } \\
\label{eqn:plk_wvl_ryl_jean_dfn}
\plkwvl(\tpt,\wvl) & \approx  
\frac{2 \cstspdlgt \cstblt \tpt}{\wvl^{4}} \\
\label{eqn:plk_wvn_ryl_jean_dfn}
\plkwvn(\tpt,\wvn) & \approx  
\frac{2 \cstspdlgt \cstblt \tpt}{\wvn^{4}}
\end{align}
\end{subequations}

The frequency of extreme emission is obtained by taking the partial
derivative of (\ref{eqn:plk_frq_dfn}) with respect to frequency with
the temperature held constant
\begin{eqnarray}
\frac{\partial \plkfrq}{\partial \frq} 
\bigg|_{\tpt}
& = &
\frac{2 \cstplk}{\cstspdlgt^{2}} \times
\frac{1}{( \me^{\cstplk \frq / \cstblt \tpt}- 1)^{2}} \times
\left( 3 \frq^{2} ( \me^{\cstplk \frq / \cstblt \tpt}- 1) -
\frq^{3} \frac{\cstplk}{\cstblt \tpt} \me^{\cstplk \frq / \cstblt \tpt} \right)
\nonumber
\end{eqnarray}
To solve for the \trmdfn{frequency of maximum emission}, $\frqmshmax$, 
we set the RHS equal to zero so that one or more of the LHS factors
must equal zero 
\begin{eqnarray}
\frac{2 \cstplk \frqmshmax^{2}
\left[ 3 ( \me^{\cstplk \frqmshmax / \cstblt \tpt}- 1 ) -
\frac{\cstplk \frqmshmax}{\cstblt \tpt} \me^{\cstplk \frqmshmax / \cstblt \tpt} \right] }
{\cstspdlgt^{2} ( \me^{\cstplk \frqmshmax / \cstblt \tpt}- 1)^{2}}
& = & 0
\nonumber \\
3 ( \me^{\cstplk \frqmshmax / \cstblt \tpt}- 1 ) -
\frac{\cstplk \frqmshmax}{\cstblt \tpt} \me^{\cstplk \frqmshmax / \cstblt \tpt}
& = & 0
\nonumber \\
\frac{\cstplk \frqmshmax}{\cstblt \tpt} \me^{\cstplk \frqmshmax / \cstblt \tpt}
& = &
3 ( \me^{\cstplk \frqmshmax / \cstblt \tpt}- 1 )
\nonumber \\
\frac{\cstplk \frqmshmax}{\cstblt \tpt}
& = &
3 ( 1 - \me^{-\cstplk \frqmshmax / \cstblt \tpt})
% \frqmshmax & = &
% \frac{3 \cstblt \tpt ( 1 - \me^{-\cstplk \frqmshmax / \cstblt \tpt})}{\cstplk}
\label{eqn:frq_max_dfn}
\end{eqnarray}
An analytic solution to (\ref{eqn:frq_max_dfn}) is impossible since
$\frqmshmax$ cannot be factored out of this transcendental equation.
If instead we solve $\xxx = 3 ( 1 - \me^{-\xxx})$ numerically we find
that $\xxx \approx 2.8215$ so that 
\begin{eqnarray}
% KiK80 p. 98 (27)
\frac{\cstplk \frqmshmax}{\cstblt \tpt} & \approx & 2.82 \nonumber \\
\frqmshmax & \approx & 2.82 \cstblt \tpt / \cstplk \nonumber \\
& \approx & 5.88 \times 10^{10} \, \tpt \qquad\mbox{\hz}
\label{eqn:frq_max_apx}
\end{eqnarray}
where the units of the numerical factor are \hzxk.
Thus the frequency of peak blackbody emission is directly
proportional to temperature.
This is known as \trmdfn{Wien's Displacement Law}.

A separate relation may be derived for the wavelength of maximum
emission $\wvlmshmax$ by an analogous procedure starting from
(\ref{eqn:plk_wvl_dfn}). 
The result is
\begin{eqnarray}
% Ste94 p. 68 (2.45), ThS99 p. 94 (4.5)
\wvlmshmax & \approx & 2897.8 / \tpt \qquad\mbox{\um}
\label{eqn:wvl_max_apx}
\end{eqnarray}
where the units of the numerical factor are \umk.
Note that (\ref{eqn:frq_max_apx}) and (\ref{eqn:wvl_max_apx})
do not yield the same answer because they measure different 
quantities.
The wavelength of maximum emission per unit wavelength, for example,
is displaced by a factor of approximately 1.76 from the wavelength of
maximum emission per unit frequency.

It is possible to use (\ref{eqn:frq_max_apx}) to estimate the
temperature of remotely sensed surfaces.
For example, a satellite-borne tunable spectral radiometer may measure
the emission of a newly discovered planet at all wavelengths of
interest.  
Assuming the wavelength of peak measured emission is $\wvlmshmax$\,\um.
Then a first approximation is that the planetary temperature is close
to $2897.8 / \wvlmshmax$.

\subsubsection[Hemispheric Quantities]{Hemispheric Quantities}\label{sxn:flx_hms}
In climate studies we are most interested in the irradiance passing
upwards or downwards through horizontal surfaces, e.g., the ground or
certain layers in the atmosphere. 
These \trmdfn{hemispheric} irradiances measure the radiant energy
transport in the vertical direction.
These hemispheric irradiances depend only on the
corresponding \trmdfn{hemispheric intensities}.
Let us assume the intensity field is azimuthally independent, i.e., 
$\ntnfrq = \ntnfrq(\plr)$ only.
Then the azimuthal contribution to (\ref{eqn:flx_udfn}) is $2\mpi$ and
\begin{eqnarray}
\flxfrq & = & 2 \mpi \int_{\plru=-1}^{\plru=1} \ntnfrq \plru \,\dfr\plru \nonumber \\
& = & 2 \mpi \left( 
\int_{\plru=-1}^{\plru=0} \ntnfrq \plru \,\dfr\plru +
\int_{\plru=0}^{\plru=1} \ntnfrq \plru \,\dfr\plru \right)
\label{eqn:hms_u}
\end{eqnarray}
We now introduce the change of variables $\plrmu = |\plru| = |\cos
\plr|$. 
Referring to (\ref{eqn:plru_dfn}) we find
\begin{subequations}
\label{eqn:plrmu_dfn}
\begin{align}
\plrmu =  | \cos \plr | & = \left\{ 
\begin{array}{r@{\quad:\quad}ll}
\cos \plr & 0 < \plr < \mpi/2 \\
-\cos \plr & \mpi/2 < \plr <  \mpi
\end{array} \right. \\
\plrmu =  |u| & = \left\{ 
\begin{array}{r@{\quad:\quad}l}
 u & 0 \le u < 1 \\
-u & -1 < u < 0
\end{array} \right.
\end{align}
\end{subequations}
Most formal work on radiative transfer is written in terms of
$\plrmu$ rather than $\plru$ or $\plr$.
Substituting (\ref{eqn:plrmu_dfn}) into (\ref{eqn:hms_u})
\begin{eqnarray}
\flxfrq & = & 2 \mpi \left( 
\int_{\plrmu=1}^{\plrmu=0} \ntnfrq (-\plrmu) \, (-\dfr\plrmu) +
\int_{\plrmu=0}^{\plrmu=1} \ntnfrq \plrmu \,\dfr\plrmu \right) \nonumber \\
& = & 2 \mpi \left( 
\int_{\plrmu=1}^{\plrmu=0} \ntnfrq \plrmu \,\dfr\plrmu +
\int_{\plrmu=0}^{\plrmu=1} \ntnfrq \plrmu \,\dfr\plrmu \right) \nonumber \\
& = & 2 \mpi \left( 
-\int_{\plrmu=0}^{\plrmu=1} \ntnfrq \plrmu \,\dfr\plrmu +
\int_{\plrmu=0}^{\plrmu=1} \ntnfrq \plrmu \,\dfr\plrmu \right) \nonumber \\
& = & -\flxdwnfrq + \flxupwfrq \nonumber \\
& = & \flxupwfrq - \flxdwnfrq
\end{eqnarray}
where we have defined the \trmdfn{hemispheric fluxes} or
\trmdfn{half-range fluxes}
\begin{subequations}
\label{eqn:flx_hms}
\begin{align}
\label{eqn:flx_upw_frq}
\flxupwfrq & = 2 \mpi \int_{0}^{1} \ntnfrq (+\plrmu) \plrmu \,
\dfr\plrmu \\
\label{eqn:flx_dwn_frq}
\flxdwnfrq & = 2 \mpi \int_{0}^{1} \ntnfrq (-\plrmu) \plrmu \,
\dfr\plrmu
\end{align}
\end{subequations} 
The hemispheric fluxes are positive-definite, and their difference is
the net flux.
\setlength{\fboxsep}{6pt} % KoD95 p. 92
\newline\fbox{\parbox{\hsize}{  % KoD95 p. 138
\begin{equation}
\flxfrq = \flxupwfrq - \flxdwnfrq
\label{eqn:flx_hms_2}
\end{equation} 
}} % end \fbox
\begin{equation} 
\fbox{$ \displaystyle % KoD95 p. 147
\flxfrq = \flxupwfrq - \flxdwnfrq $} % end \fbox
\label{eqn:flx_hms_3}
\end{equation} 
The superscripts $^+$ and $^-$ denote \trmdfn{upwelling}
(towards the upper hemisphere) and \trmdfn{downwelling} (towards the
lower hemisphere) quantities, respectively.
The net flux $\flxfrq$ is the difference between the upwelling and
downwelling hemispheric fluxes, $\flxupwfrq$ and $\flxdwnfrq$, which
are both positive-definite quantities.

The hemispheric flux transport in isotropic radiation fields
is worth examining in detail since this condition is often met
in practice.
It will be seen that isotropy considerably simplifies many of the 
troublesome integrals encountered.
When $\ntnfrq$ has no directional dependence (i.e., it is a constant) 
then $\flxupwfrq = \flxdwnfrq$ (\ref{eqn:flx_hms_3}).
Thus the net radiative energy transport is zero in an isotropic
radiation field, such as a cavity filled with blackbody radiation. 

Let us compute the upward transport of radiation $\plkfrqtpt$
(\ref{eqn:plk_frq_dfn}) emitted by a perfect blackbody such as the
ocean surface.  
Since $\plkfrqtpt$ is isotropic, the intensity may be factored
out of the definition of the upwelling flux (\ref{eqn:flx_upw_frq}) 
and we obtain
\begin{eqnarray} 
\flxplkfrq & = & 2 \mpi \int_{0}^{1} \plkfrq \plrmu \,\dfr\plrmu \nonumber \\
& = & 2 \mpi \plkfrq \int_{0}^{1} \plrmu \,\dfr\plrmu \nonumber \\
& = & 2 \mpi \plkfrq \frac{\plrmu^{2}}{2} \bigg|_{\plrmu = 0}^{\plrmu = 1} \nonumber \\
% fxm: 20040124 tth breaks here on use of \textstyle unless enclosed with braces
& = & 2 \mpi \plkfrq \, ({\textstyle\frac{1}{2}} - 0) \nonumber \\
& = & \mpi \plkfrq
\label{eqn:plk_hms_dfn}
\end{eqnarray} 
Thus the upwelling blackbody irradiance tranports $\mpi$ times the
 constant intensity of the radiation. 
Given that the upper hemisphere contains $2\mpi$ steradians, one might
naively expect the upwelling irradiance to be $2\mpi \plkfrq$.
In fact the divergence of blackbody radiance above the emitting
surface is $2\mpi \plkfrq$.
But the vertical flux of energy is obtained by cosine-weighting the
radiance over the hemisphere and this weight introduces the factor of 
$\frac{1}{2}$ difference between the naive and the correct solutions.

\subsubsection[Stefan-Boltzmann Law]{Stefan-Boltzmann Law}\label{sxn:stf_blt}
The frequency-integrated hemispheric irradiance emanating from a
blackbody of great interest since it describes, e.g., the radiant
power of most surfaces on Earth. 
Although we could integrate the $\plkfrq$ (\ref{eqn:plk_frq_dfn})
directly to obtain the broadband intensity 
$\plkfnc \equiv \int_{0}^{\infty} \plkfrq(\frq) \,\dfr\frq$,
it is traditional to integrate $\plkfrq$ first over the hemisphere 
(\ref{eqn:plk_hms_dfn}). 
By proceeding in this order, we shall obtain the total hemispheric
blackbody irradiance $\flxupwplk$ in terms fundamental physical
constants and the temperature of the body.
\begin{eqnarray}
\flxupwplk & = & \mpi \int_{0}^{\infty} \plkfrq \,\dfr\frq 
\nonumber \\
& = & \mpi \int_{0}^{\infty} 
\frac{2 \cstplk \frq^{3}}{\cstspdlgt^{2} 
( \me^{\cstplk \frq / \cstblt \tpt} - 1)} \,\dfr\frq
\nonumber \\
& = & \frac{2 \mpi \cstplk}{\cstspdlgt^{2}} 
\int_{0}^{\infty} 
\frac{\frq^{3}}{\me^{\cstplk \frq / \cstblt \tpt} - 1} \,\dfr\frq
\label{eqn:stf_blt_1}
\end{eqnarray}
To simplify (\ref{eqn:stf_blt_1}) we make the change of variables 
\begin{eqnarray}
\xxx & = & \frac{\cstplk \frq}{\cstblt \tpt} \nonumber \\
\frq & = & \frac{\cstblt \tpt \xxx}{\cstplk} \nonumber \\
\dfr\frq & = & \frac{\cstblt \tpt}{\cstplk} \,\dfr\xxx \nonumber \\
\dfr\xxx & = & \frac{\cstplk}{\cstblt \tpt} \,\dfr\frq \nonumber
\label{eqn:cov_x}
\end{eqnarray}
This change of variables maps $\frq \in [0,\infty)$ to
$\xxx \in [0,\infty)$.
Substituting this into (\ref{eqn:stf_blt_1}) we obtain
\begin{eqnarray}
\flxupwplk 
& = & \frac{2 \mpi \cstplk}{\cstspdlgt^{2}} 
\int_{0}^{\infty} 
\left( \frac{\cstblt \tpt}{\cstplk} \right)^{3}
\frac{\xxx^{3}}{\me^{\xxx} - 1} \, \frac{\cstblt \tpt}{\cstplk} \,\dfr\xxx
\nonumber \\
& = & \frac{2 \mpi \cstblt^{4} \tpt^{4} }{\cstspdlgt^{2} \cstplk^{3}} 
\int_{0}^{\infty} 
\frac{\xxx^{3}}{\me^{\xxx} - 1} \,\dfr\xxx
\label{eqn:stf_blt_2}
\end{eqnarray}

The definite integral in (\ref{eqn:stf_blt_2}) is $\mpi^{4}/15$. 
Proving this is a classic problem in mathematical physics which
involves the \trmidx{Riemann zeta function} (and thus prime number
theory), the \trmidx{Gamma function}, and contour integration.
The procedure used to obtain this result is interesting so we briefly
summarize it here.
The \trmdfn{Riemann zeta function} $\rmnztafnc(\xxx)$ for real $\xxx > 1$
may be defined as
\begin{eqnarray}
\rmnztafnc(\xxx) & \equiv & \frac{1}{\gmmfnc(\xxx)}
\int_{0}^{\infty} \frac{\uuu^{\xxx-1}}{\me^{\uuu}-1} \,\dfr\uuu
\label{eqn:rmn_zta_dfn}
\end{eqnarray}
Comparing (\ref{eqn:stf_blt_2}) with the Riemann zeta function
definition (\ref{eqn:rmn_zta_dfn}), we see that $\xxx=4$, i.e, 
\begin{eqnarray}
\flxupwplk 
& = & \frac{2 \mpi \cstblt^{4} \tpt^{4} }{\cstspdlgt^{2} \cstplk^{3}} 
\gmmfnc(4)\rmnztafnc(4)
\label{eqn:stf_blt_rmn}
\end{eqnarray}

The integral (\ref{eqn:rmn_zta_dfn}) is analytically solvable for 
the special case of integers $\xxx = \nnn$.
We may transform the integrand from a rational fraction into a
\trmidx{power series} using algebraic manipulation:
\begin{eqnarray}
\frac{\uuu^{\xxx-1}}{\me^{\uuu}-1} 
& = & \frac{\uuu^{\xxx-1}}{\me^{\uuu}-1} \times \frac{\me^{-\uuu}}{\me^{-\uuu}} \nonumber \\
& = & \frac{\uuu^{\xxx-1}\me^{-\uuu}}{1-\me^{-\uuu}} \nonumber \\
& = & \uuu^{\xxx-1}\me^{-\uuu} \times \frac{1}{1-\me^{-\uuu}}
\label{eqn:rmn_ntg}
\end{eqnarray}
The integration limits in (\ref{eqn:rmn_zta_dfn}) are $[0,\infty)$
so it is always true that $\me^{-\uuu} < 1$ in (\ref{eqn:rmn_ntg}),
and thus in the integrand of (\ref{eqn:rmn_zta_dfn}).

Recall that the sum of an infinite \trmdfn{power series} with 
initial term $\aaa_{0}$ and ratio $\rrr$ is
\begin{eqnarray}
\sum_{\kkk=0}^{\infty} \aaa_{0}\rrr^{\kkk} & = & 
\lim_{\kkk \to \infty} \aaa_{0} + \aaa_{0}\rrr + \aaa_{0}\rrr^{2} + \cdots
+ \aaa_{0}\rrr^{\kkk-1} + \aaa_{0}\rrr^{\kkk} \nonumber \\
& = & \aaa_{0}/(1-\rrr) 
\quad \mbox{for\ } |\rrr| < 1
\label{eqn:pwr_srs}
\end{eqnarray}
Hence the last term in (\ref{eqn:rmn_ntg}) is the sum of a power
series (\ref{eqn:pwr_srs}) with initial term $\aaa_{0} = 1$ and 
ratio $\rrr = \me^{-\uuu}$.
\begin{eqnarray}
\frac{1}{1-\me^{-\uuu}}
& = & \sum_{\kkk=0}^{\infty} \me^{-\kkk\uuu}
\label{eqn:xpn_pwr_srs}
\end{eqnarray}
Substituting (\ref{eqn:xpn_pwr_srs}) into (\ref{eqn:rmn_ntg}) 
we obtain
\begin{eqnarray}
% http://mathworld.wolfram.com/RiemannZetaFunction.html
\frac{\uuu^{\xxx-1}}{\me^{\uuu}-1} 
& = & \uuu^{\xxx-1}\me^{-\uuu} \sum_{\kkk=0}^{\infty} \me^{-\kkk\uuu} \nonumber \\
& = & \sum_{\kkk=0}^{\infty} \uuu^{\xxx-1} \me^{-(\kkk+1)\uuu} \nonumber \\
& = & \sum_{\kkk=1}^{\infty} \uuu^{\xxx-1} \me^{-\kkk\uuu}
\label{eqn:rmn_ntg_pwr_srs}
\end{eqnarray}
where the last step shifts the initial index to from zero to one.

Using the inifinite series representation (\ref{eqn:rmn_ntg_pwr_srs}) for
the integrand of (\ref{eqn:rmn_zta_dfn}) yields
\begin{eqnarray}
\rmnztafnc(\xxx) & = & \frac{1}{\gmmfnc(\xxx)}
\int_{0}^{\infty} \left[ \sum_{\kkk=1}^{\infty} \uuu^{\xxx-1}
  \me^{-\kkk\uuu} \right] \,\dfr\uuu \nonumber
\label{eqn:rmn_pwr_srs_dfn_1}
\end{eqnarray}
Integration and addition are commutative operations.
Interchanging their order yields
\begin{eqnarray}
\rmnztafnc(\xxx) & = & \frac{1}{\gmmfnc(\xxx)}
\sum_{\kkk=1}^{\infty} 
\left[ \int_{0}^{\infty} \uuu^{\xxx-1} \me^{-\kkk\uuu}  \,\dfr\uuu \right]
\label{eqn:rmn_pwr_srs_dfn_2}
\end{eqnarray}
We change variables from 
$\uuu \in [0,+\infty]$ to $\yyy \in [0,+\infty]$ with
\begin{eqnarray}
\yyy & = & \kkk\uuu \nonumber \\
\dfr\yyy & = & \kkk \,\dfr\uuu \nonumber \\
\uuu & = & \yyy/\kkk \nonumber \\
\dfr\uuu & = & \kkk^{-1} \,\dfr\yyy
\label{eqn:rmn_cov}
\end{eqnarray}
so that (\ref{eqn:rmn_pwr_srs_dfn_2}) becomes
\begin{eqnarray}
\rmnztafnc(\xxx) & = & \frac{1}{\gmmfnc(\xxx)}
\sum_{\kkk=1}^{\infty} 
\left[ \int_{0}^{\infty} \left( \frac{\yyy}{\kkk} \right)^{\xxx-1} 
\me^{-\yyy} \kkk^{-1} \,\dfr\yyy \right] \nonumber \\
& = & \frac{1}{\gmmfnc(\xxx)}
\sum_{\kkk=1}^{\infty} \kkk^{-(\xxx-1)} \kkk^{-1}
\left[ \int_{0}^{\infty} \yyy^{\xxx-1} \me^{-\yyy} \,\dfr\yyy \right] \nonumber \\
& = & \frac{1}{\gmmfnc(\xxx)}
\sum_{\kkk=1}^{\infty} \kkk^{-\xxx} [ \gmmfnc(\xxx) ] \nonumber \\
& = & \sum_{\kkk=1}^{\infty} \kkk^{-\xxx}
% fxm: Exactly where must we have assumed \xxx is integer \nnn?
\label{eqn:rmn_pwr_srs_dfn_3}
\end{eqnarray}
where we replaced the integral in brackets with the 
\trmidx{Gamma function} it defines.
Hence the Riemann zeta function of a positive integer~$\nnn$ is the
sum of the reciprocals of the postive integers to the power~$\nnn$.

Contour integration in the complex plane gives analytic closed-form
solutions to $\Sigma_{1}^{\infty} \kkk^{-\nnn}$
(\ref{eqn:rmn_pwr_srs_dfn_3}) for positive, even integers~$\nnn$.
Our immediate concern is $\nnn = 4$.
Consider the complex function \cite[][p.~97]{CCP83}
\begin{eqnarray}
% CCP83 p. 97
\fnc(\zzz) & = & \frac{\mpi\cot\mpi\zzz}{\zzz^{4}}
\label{eqn:cnt_ntg_fnc}
\end{eqnarray}
This function 
\begin{enumerate*}
\item Is \trmdfn{analytic} throughout the complex plane
\item Has first order poles at all integer values on the real axis
  (except the origin)
\item Has a fifth order pole at the origin
\item Satisfies $\lim_{|\RRR| \to \infty} \fnc(\zzz) = 0$
  where $\zzz = \RRR\me^{\mi\plr}$
\end{enumerate*}
Therefore (\ref{eqn:cnt_ntg_fnc}) obeys the \trmidx{residue theorem}
for suitably chosen contours.
In other words, fxm.

Having shown 
\begin{eqnarray}
% NB: Contour integration of Riemann zeta function in AM201 notes p. 117
% Full solution is now in end of AM201 notes
\rmnztafnc(4) & = & \mpi^{4}/90
\label{eqn:rmn_zta_four}
\end{eqnarray}

Using $\gmmfnc(4) = 3! = 3 \times 2 \times 1 = 6$
and $\rmnztafnc(4) = \mpi^{4}/90$ from (\ref{eqn:rmn_zta_four}),
Equation~(\ref{eqn:stf_blt_rmn}) becomes
\begin{eqnarray}
\flxupwplk 
& = & \frac{2 \mpi \cstblt^{4}}{\cstspdlgt^{2} \cstplk^{3}} 
\times 6 \times \frac{\mpi^{4}}{90} \times \tpt^{4}
\nonumber \\
& = & \frac{2 \mpi^{5} \cstblt^{4}}{15 \cstspdlgt^{2} \cstplk^{3}} \tpt^{4}
\nonumber
\label{eqn:stf_blt_3}
\end{eqnarray}
This is known as the \trmdfn{Stefan-Boltzmann Law} of radiation,
and is usually written as 
\begin{eqnarray}
\label{eqn:stf_blt_dfn}
\flxupwplk & = & \cststfblt \tpt^{4} \qquad \mbox{where} \\
\label{eqn:cst_stf_blt_dfn}
\cststfblt & \equiv & 
\frac{2 \mpi^{5} \cstblt^{4}}{15 \cstspdlgt^{2} \cstplk^{3}}
\end{eqnarray}
$\cststfblt$ is known as the \trmdfn{Stefan-Boltzmann constant}
and depends only on fundamental physical constants.
The value of $\cststfblt$ is $5.67032 \times 10^{-8}$\,\wxmSkQ.
The thermal emission of matter depends very strongly (quartically) on 
$\tpt$ (\ref{eqn:stf_blt_dfn}). 
This rather surprising result has profound implications for Earth's
climate. 

We derived $\flxupwplk$ (\ref{eqn:stf_blt_dfn}) directly so that
the Stefan-Boltzmann constant would fall naturally from the
derivation. 
For completeness we now present the broadband blackbody intensity
$\plkfnc$
\begin{eqnarray}
\plkfnc & = & \int_{0}^{\infty} \plkfrq(\frq) \,\dfr\frq \nonumber \\
& = & \frac{2 \mpi^{4} \cstblt^{4}}{15 \cstspdlgt^{2} \cstplk^{3}} \tpt^{4} \nonumber \\
& = & \flxupwplk / \mpi = \cststfblt \tpt^{4} / \mpi 
\label{eqn:plk_fnc_dfn}
\end{eqnarray}
The factor of $\mpi$ difference between $\plkfnc$ and $\flxupwplk$ 
is at first confusing.
One must remember that $\plkfnc$ is the broadband
(spectrally-integrated) intensity and that $\flxupwplk$ is the 
broadband hemispheric irradiance
(spectrally-and-angularly-integrated).  

\subsubsection[Luminosity]{Luminosity}\label{sxn:lmn}
The total thermal emission of a body (e.g., star or planet) is called  
its \trmdfn{luminosity}, $\lmn$.
The luminosity is thermal irradiance integrated over the surface-area
of a volume containing the body.
The luminosity of bodies with atmospheres is usually taken to be
the total outgoing thermal emission at the top of the atmosphere.
If the time-mean thermal radiation field of a body is spherically
symmetric, then its luminosity is easily obtained from a time-mean
measurement of the thermal irradiance normal to any unit area of the
surrounding surface.
This technique is used on Earth to determine the intrinsic luminosity
of stars whose distance $\dstslr$ is known (e.g., through parallax)
including our own.
\begin{eqnarray}
\lmnslr & = & 4\mpi\dstslr^{2} \flxslrtoa
\label{eqn:lmn_dfn_1}
\end{eqnarray}
The meaning of the \trmdfn{solar constant} $\flxslrtoa$ is made clear
by (\ref{eqn:lmn_dfn_1}).
$\flxslrtoa$ is the solar irradiance that would be measured normal to
the Earth-Sun axis at the top of Earth's atmosphere.
The mean Earth-Sun distance is $\sim 1.5 \times 10^{11}$\,m\ and 
$\flxslrtoa \approx 1367$\,\wxmS\ so 
$\lmnslr = 3.9 \times 10^{26}$\,W.
Note that $\lmn$ has dimensions of power, i.e., \jxs.

An independent means of estimating the luminosity of a celestial body 
is to integrate the surface thermal irradiance over the surface area
of the body.
For planets without atmospheres, $\lmn$ is simply the integrated
surface emission. 
Assuming the \trmidx{effective temperature} (\ref{eqn:tpt_ffc_dfn}) of
our Sun is $\tptffc$, the radius $\rdsslr$ at which this emission must
originate is defined by combining (\ref{eqn:stf_blt_dfn}) with
(\ref{eqn:lmn_dfn_1})
\begin{eqnarray}
4\mpi\rdsslr^{2} \cststfblt \tptslr^{4} & = & 4\mpi\dstslr^{2} \flxslrtoa 
\nonumber \\
\rdsslr^{2} & = & \frac{\dstslr^{2} \flxslrtoa}{\cststfblt \tptslr^{4}} 
\nonumber \\
\rdsslr & = &
\frac{\dstslr}{\tptslr^{2}}\sqrt{\frac{\flxslrtoa}{\cststfblt}} 
\label{eqn:rds_slr_dfn}
\end{eqnarray}
For the Sun-Earth system, $\tptslr \approx 5800$\,K so
$\rdsslr = 6.9 \times 10^{8}$\,m. 
Solar radiation received by Earth appears to originate from a portion
of the solar atmosphere known as the \trmdfn{photosphere}. 

\subsubsection[Extinction and Emission]{Extinction and Emission}\label{sxn:ext}
Radiation and matter have only two forms of interactions,
\trmdfn{extinction} and \trmdfn{emission}.
Lambert\footnote{Beer is usually credited with first formulating this
law, but actually the similarly-named Bougher deserves the credit.},
first proposed that the extinction (i.e., reduction) of radiation 
traversing an infinitesimal path $\dfr\pth$ is linearly proportional
to the incident radiation and the amount of interacting matter along
the path
\begin{equation}
\frac{\dfr\ntnfrq}{\dfr\pth} = -\extcffoffrq \ntnfrq \qquad\mathrm{Extinction\ only}
\label{eqn:ext_dfn}
\end{equation}
Here $\extcffoffrq$ is the \trmdfn{extinction coefficient}, a
measurable property of the medium, and $\pth$ is the absorber path 
length. 
$\extcffoffrq$~is proportional to the local density of the medium and
is positive-definite.

The term \trmdfn{extinction coefficient} and the exact definition of
$\extcffoffrq$ are somewhat ambiguous until their physical dimensions
are specified.   
Path length, for example, can be measured in terms of 
column mass path $\mpc$\,[\kgxmS],  
number path of molecules $\nbr$\,[\nbrxmS],
and geometric distance $\pth$\,[\m]. 
Each path measure has a commensurate extinction coefficient: 
\trmidx{the mass extinction coefficient} $\extcffmss$\,[\mSxkg] 
(i.e., optical cross-section per unit mass), 
\trmidx{the number extinction coefficient} $\extcffnbr$\,[\mSxmlc] 
(i.e., optical cross-section per molecule), 
\trmidx{the volume extinction coefficient} $\extcffvlm$\,[\mSxmC]$=$[\xm]
(i.e., optical cross-section per unit concentration).
These coefficients are inter-related by 
\begin{eqnarray}
\extcffmss\,\mbox{[\mSxkg]} & = & \frac{\extcffvlm\,\mbox{[\xm]}}{\dns\,\mbox{[\kgxmC]}} \nonumber \\
\extcffnbr\,\mbox{[\mSxmlc]} & = & \frac{\extcffmss\,\mbox{[\mSxkg]} \times \mmw\,\mbox{[\kgxmol]}}{\cstAvagadro\,\mbox{[\mlcxmol]}} \nonumber \\
\extcffvlm\,\mbox{[\xm]} & = & \extcffnbr\,\mbox{[\mSxmlc]} \times \nbrcnc\,\mbox{[\mlcxmC]}
\label{eqn:np}
\end{eqnarray}

We will develop the formalism of radiative transfer in this chapter 
in terms using the geometric path and volume extinction coefficient 
($\pth$, $\extcffvlm$) formalism. 
Our intent is to be concrete, rather than leaving the choice of units
unstated. 
However, there is nothing fundamental about ($\pth$, $\extcffvlm$).
In Section~\ref{sxn:lnshp_fct} we state our preference for working in 
mass-path units ($\mpc$, $\extcffmss$).

Extinction includes all processes which reduce the radiant intensity.
As will be described below, these processes include absorption and
scattering, both of which remove photons from the beam.
Similarly the radiative emission is also proportional to the amount
of matter along the path
\begin{equation}
\frac{\dfr\ntnfrq}{\dfr\pth} = \extcffoffrq \srcfrq \qquad\mathrm{Emission\ only}
\label{eqn:msn_dfn}
\end{equation}
where $\srcfrq$ is known as the \trmdfn{source function}.
The source function plays an important role in radiative transfer
theory.
We show in \S\ref{sxn:rte_sln_frm} that if $\srcfrq$ is known, then
the full radiance field $\ntnfrq$ is determined by an integration of
$\srcfrq$ with the appropriate boundary conditions. 
Emission includes all processes which increase the radiant intensity.
As will be described below, these processes include thermal emission and
scattering which adds photons to the beam.
Determination of $\extcffoffrq$, which contains all the
information about the electromagnetic properties of the media, is the
subject of active theoretical, laboratory and field research. 

Extinction and emission are linear processes, and thus additive. 
Since they are the only two processes which alter the intensity of
radiation,
\begin{eqnarray}
\frac{\dfr\ntnfrq}{\dfr\pth} & = & -\extcffvlm \ntnfrq + \extcffvlm \srcfrq \nonumber \\
\frac{1}{\extcffvlm} \frac{\dfr\ntnfrq}{\dfr\pth} & = & -\ntnfrq + \srcfrq
\label{eqn:rte_dfn_smp}
\end{eqnarray}
Equation~(\ref{eqn:rte_dfn_smp}) is the \trmdfn{equation of radiative
transfer} in its simplest differential form.

\subsubsection[Optical Depth]{Optical Depth}\label{sxn:tau}
We define the \trmdfn{optical path} $\tautld$ between
points $\pnt_{1}$ and $\pnt_{2}$ as
\begin{eqnarray}
\tautld(\pnt_{1},\pnt_{2}) & = & \int_{\pnt_{1}}^{\pnt_{2}} \extcffvlm \,\dfr\pth
\nonumber \\ 
\dfr\tautld & = & \extcffvlm \,\dfr\pth
\label{eqn:tau_tld_dfn}
\end{eqnarray}
The optical path measures the amount of extinction a beam of light
experiences traveling between two points.
When $\tautld > 1$, the path is said to be optically thick.

The most frequently used form of optical path is the \trmidx{optical
depth}. 
The optical depth $\tau$ is the vertical component of the optical path
$\tautld$, i.e., $\tau$ measures extinction between vertical levels.
For historical reasons, the optical depth in planetary atmospheres is
defined $\tau = 0$ at the top of the atmosphere and $\tau = \taustr$
at the surface. 
This convention reflects the astrophysical origin of much of radiative
transfer theory.
Much like pressure, $\tau$ is a positive-definite coordinate which
increases monotonically from zero at the top of the atmosphere to its
surface value. 
Consider the optical depth between two levels $\zzz_{2} > \zzz_{1}$, and
then allow $\zzz_{2} \rightarrow \infty$
\begin{eqnarray}
\tau(\zzz_{1},\zzz_{2}) & = & \int_{\zzzprm=\zzz_{1}}^{\zzzprm=\zzz_{2}}
\extcffvlm \,\dfr\zzzprm \nonumber \\
\tau(\zzz,\infty) & = & 
\int_{\zzzprm=\zzz}^{\zzzprm=\infty} \extcffvlm \,\dfr\zzzprm \\
 & = & 
\int_{\zzzprm=\zzz}^{\zzzprm=\infty} \extcffvlm \,\dfr\zzzprm
\label{eqn:tau_int_dfn}
\end{eqnarray}
Equation~(\ref{eqn:tau_int_dfn}) is the integral definition of optical
depth. 
The differential definition of optical depth is obtained by
differentiating (\ref{eqn:tau_int_dfn}) with respect to the lower
limit of integration and using the fundamental theorem of
differential calculus  
\begin{eqnarray}
\dfr\tau & = & [\extcffvlm(\infty) - \extcffvlm(\zzz)] \,\dfr\zzz \nonumber
\\
& = & -\extcffvlm(\zzz) \,\dfr\zzz
\label{eqn:tau_dfn}
\end{eqnarray}
where the second step uses the convention that $\extcffvlm(\infty) =
0$.  
By convention, $\tau$ is positive-definite, but
(\ref{eqn:tau_dfn}) shows that $\dfr\tau$ may be positive or
negative. 
If this seems confusing, consider the analogy with atmospheric
pressure: pressure increases monotonically from zero at the top of the
atmosphere, and we often express physical concepts such as the
temperature lapse rate in terms of negative pressure gradients.

\subsubsection[Geometric Derivation of Optical Depth]{Geometric Derivation of Optical Depth}\label{sxn:tau_geo}
The optical depth of a column containing spherical particles may be
derived by appealing to intuitive geometric arguments. 
Consider a concentration of $\cnc$\,\xmC\ identical spherical particles 
of radius $\rds$ residing in a rectangular chamber measuring one meter
in the $\xxx$ and $\yyy$ dimensions and of arbitrary height.
If the chamber is uniformly illuminated by a \trmdfn{collimated} beam
of sunlight from one side, how much energy reaches the opposite side?

For this thought experiment, we will neglect the effects of
scattering\footnote{The effects of diffraction will not be explicitly
included either, but are implicit in the assumption of the horizontal
homogeneity of the radiative flux.}.
Moreover, we will assume that the particles are partially opaque so
that the incident radiation which they do not absorb is transmitted
without any change in direction.
Finally, assume the particles are homogeneously distributed in the
horizontal so that the radiative flux $\flx(\zzz)$ is a function only
of height in the chamber.
Let us denote the power per unit area of the incident collimated beam
of sunlight as $\flxslrtoa$.
Our goal is to compute $\flx(\zzz)$ as a function of $\cnc$ and of
$\rds$. 

Since each particle absorbs incident energy in proportion to its
geometric cross-section, the total absorption of radiation per
particle is proportional to $\mpi \rds^{2} \fshext$, where $\fshext = 1$
for perfectly absorbing particles.
If $\fshext < 1$, then each particle removes fewer photons than
suggested by its geometric size\footnote{Although intuition suggests a
spherical particle should not remove more energy from a collimated
light beam than it can geometrically intercept, this is not the case. 
As discussed later, diffraction around the particle is important. 
In fact, for $\rds \gg \wvl$, $\fshext \rightarrow 2$. 
However, most of the extinction due to diffraction occurs as
scattering, not absorption.}. 
Conversely, if $\fshext > 1$ each particle removes more photons than 
suggested by its geometric size.
Each particle encountered removes $\mpi \rds^{2} \fshext
\flx(\hgt)$\,\wxmS\ from the incident beam. 
The maximum flux which can be removed from the beam is, of course, 
$\flxslrtoa$. 

In a section of height $\hgtdlt$, the collimated beam passing
through the chamber will encounter a total of $\cnc \hgtdlt$
particles. 
The number of particles encounted times the flux removed per particle
gives the change in the radiative flux of the beam between the
entrance and the rear wall
\begin{eqnarray}
\flxdlt & = & - \mpi \rds^{2} \fshext \flx \cnc \hgtdlt \nonumber \\
\frac{\flxdlt}{\flx} & = & - \mpi \rds^{2} \fshext \cnc \hgtdlt
\label{eqn:flx_slr_toa_dlt_xmp_1}
\end{eqnarray}
If we take the limit as $\hgtdlt \rightarrow 0$ then
(\ref{eqn:flx_slr_toa_dlt_xmp_1}) becomes 
\begin{eqnarray}
\frac{1}{\flx} \,\dfr\flx & = & -  \mpi \rds^{2} \fshext \cnc \,\dfr\hgt
\nonumber \\
{\dfr(\ln \flx)} & = & - \mpi \rds^{2} \fshext \cnc \,\dfr\hgt
\label{eqn:flx_slr_toa_dlt_xmp_2}
\end{eqnarray}
Let us define the \trmdfn{volume extinction coefficient} $\kkk$ as
\begin{equation}
\kkk \equiv \mpi \rds^{2} \fshext \cnc
\label{eqn:ext_xmp_2}
\end{equation}
Finally we define the optical depth $\tau$ in terms of the extinction
\begin{eqnarray}
\tau & \equiv & \kkk \hgtdlt \\
\tau & = & \mpi \rds^{2} \fshext \cnc \hgtdlt
\label{eqn:tau_xmp_2}
\end{eqnarray}
The dimensions of $\kkk$ are inverse meters [\xm], or, perhaps more
intuitively, square meters of effective surface area per cubic meter
of air [\mSxmC].
Therefore $\tau$ is dimensionless. 
It can be helpful to remember that 
All quantities which compose $\tau$ are positive by convention,
therefore $\tau$ itself is positive-definite.

We are now prepared to solve (\ref{eqn:flx_slr_toa_dlt_xmp_2}) for
$\flx(\tau)$. 
From the theory of first order differential equations, we know that
$\flx$ must be an exponential function whose solution decays from its
initial value with an $e$-folding constant of $\tau$ 
\begin{equation}
\flx(\tau) = \flxslrtoa \me^{-\tau}
\label{eqn:flx_slr_toa_dlt_xmp_4}
\end{equation}
Thus, in the limit of geometrical optics, the optical depth measures
the number of $e$-foldings undergone by the radiative flux of a
collimated beam passing through a given medium.  
This result is one form of the \trmdfn{extinction law}.

The exact value of $\fshext(\rds,\wvl)$ depends on the composition of
the aerosol.
However, there is a limiting value of $\fshext$ as particles become
large compared to the wavelength of light. 
\begin{equation}
\lim_{\rds \gg \lambda} \fshext = 2
\label{eqn:fsh_ext_xmp}
\end{equation}
Thus particles larger than 5--10\,\um\ extinguish twice as much visible
light as their geometric cross-section suggests.

To gain more insight into the usefulness of the optical depth, we
can express $\tau$ (\ref{eqn:tau_xmp_2}) in terms of the aerosol mass
$\mss$, rather than number concentration $\cnc$.
For a monodisperse aerosol of density $\dns$, the mass concentration is
\begin{equation}
\mss = \frac{4}{3} \mpi \rds^{3} \dns \cnc
\label{eqn:mss_xmp}
\end{equation}
If we substitute 
\begin{equation}
\mpi \rds^{2} \cnc = \frac{3 \mss}{4 \rds \dns} \nonumber
\end{equation}
into (\ref{eqn:tau_xmp_2}) we obtain
\begin{equation}
\tau = \frac{3 \mss \fshext \hgtdlt}{4 \rds \dns}
\label{eqn:tau_xmp_3}
\end{equation}
Typical cloud particles have $\rds \sim 10$\,\um\ so that for visible
solar radiation with $\wvl \sim 0.5$\,\um\ we may employ
(\ref{eqn:fsh_ext_xmp}) to obtain 
\begin{equation}
\tau = \frac{3 \mss \hgtdlt}{2 \rds \dns}
\label{eqn:tau_xmp_4}
\end{equation}
Thus $\tau$ increases linearly with $\mss$ for a given $\rds$.
Note however, that a given mass $\mss$ produces an optical depth
that is inversely proportional to the radius of the particles!

\subsubsection[Stratified Atmosphere]{Stratified Atmosphere}\label{sxn:str_atm}
We obtain the radiative transfer equation in terms of optical path
by substituting (\ref{eqn:tau_tld_dfn}) into (\ref{eqn:rte_dfn_smp})
\begin{eqnarray}
% ThS99 p. 151 (5.42)
\frac{\dfr\ntnfrq}{\dfr\tautld} & = & -\ntnfrq + \srcfrq
\label{eqn:rte_tau_tld_dfn}
\end{eqnarray}
A \trmidx{stratified atmosphere} is one in which all atmospheric
properties, e.g., temperature, density, vary only in the vertical. 
As shown in the non-existant Figure, the photon path increment
$\dfr\pth$ at polar angle $\plr$ in a stratified atmosphere is related
to the vertical path increment $\dfr\zzz$ by $\dfr\zzz = \cos \plr \,
\dfr\pth$ or $\dfr\pth = \plru^{-1} \,\dfr\zzz$. 
In other words, the optical path traversed by photons is proportional
to the vertical path divided by the cosine of the trajectory.
\begin{subequations}
% ThS99 p. 154
\label{eqn:dfr_tau_tld_tau_dfn}
\begin{align}
\label{eqn:tau_tld_tau_dfn}
\dfr\tautld & = \plru^{-1} \,\dfr\tau \\
\label{eqn:tau_tau_tld_dfn}
\dfr\tau & = \plru \,\dfr\tautld
\end{align}
\end{subequations}
Substituting 
(\ref{eqn:tau_tld_tau_dfn}) into (\ref{eqn:rte_tau_tld_dfn})  
we obtain
\begin{eqnarray}
% ThS99 p. 159 (5.64)
\plru \frac{\dfr\ntnfrq}{\dfr\tau} & = & -\ntnfrq + \srcfrq
\label{eqn:rte_tau_dfn}
\end{eqnarray}
This is the differential form of the radiative transfer equation in a
plane parallel atmosphere and is valid for all angles.
The solution of (\ref{eqn:rte_tau_dfn}) is made difficult because
$\srcfrq$ \textit{depends} on $\ntnfrq$.
A more tractable set of equations may be obtained by considering
the form of the boundary conditions.
For many (most) problems of atmospheric interest, we know $\ntnfrq$
over an entire hemisphere at each boundary of a ``slab''.
Considering the entire atmosphere as a slab, for example, we know
that, at the top of the atmosphere, sunlight is the only incident
intensity from the hemisphere containing the sun. 
Or, at the surface, we have constraints on the upwelling intensity 
due to thermal emission or the 
surface reflectivity\index{surface reflectance}.
When combined, these two hemispheric boundary conditions span a
complete range of polar angle, and are thus sufficient to solve 
(\ref{eqn:rte_tau_dfn}).
However, in practice it is difficult to apply half a boundary
condition. 
Moreover, we are often interested in knowing the hemispheric flows of 
radiation because many instruments (e.g., pyranometers) are designed
to measure hemispheric irradiance and many models (e.g., climate
models) require hemispheric irradiance to compute surface exchange
properties.
For these reasons we will decouple (\ref{eqn:rte_tau_dfn}) into
its constituent upwelling and downwelling radiation components.

Using the definitions of the \trmidx{half-range intensities}
(\ref{eqn:ntn_hms_dfn}) in (\ref{eqn:rte_tau_dfn}) we obtain 
\begin{subequations}
% ThS99 p. 155 (5.51, 5.52)
\label{eqn:rte_plru_dfn}
\begin{align}
\label{eqn:rte_plru_dwn_dfn}
-\plru \frac{\dfr\ntndwnfrq}{\dfr\tau} & = \ntndwnfrq - \srcdwnfrq 
\qquad 0 < \plr < \mpi/2, \plru = \cos \plr > 0 \\
\label{eqn:rte_plru_upw_dfn}
-\plru \frac{\dfr\ntnupwfrq}{\dfr\tau} & = \ntnupwfrq - \srcupwfrq
\qquad \mpi/2 < \plr < \mpi, \plru = \cos \plr < 0
\end{align}
\end{subequations}
where we have simply multiplied (\ref{eqn:rte_tau_dfn}) by $-1$
in order to place the negative sign on the LHS for reasons that
will be explained shortly.
The definitions of $\srcdwnfrq$ and $\srcupwfrq$ are exactly analogous
to (\ref{eqn:ntn_hms_dfn}).

We now change variables from $\plru$ to $\plrmu$
(\ref{eqn:plrmu_dfn}). 
Replacing $\plru$ by $\plrmu$ in (\ref{eqn:rte_plru_dwn_dfn})
is allowed since $\plrmu = \plru > 0$ in this hemisphere.
In the upwelling hemisphere where $\mpi/2 < \plr < \mpi$, 
$\plru < 0$ so that $\plrmu = -\plru$ (\ref{eqn:plrmu_dfn}).
This negative sign cancels the negative sign on the LHS of 
(\ref{eqn:rte_plru_upw_dfn}), resulting in 
\begin{subequations}
% ThS99 p. 155 (5.51, 5.52)
\label{eqn:rte_hlf_dfn}
\begin{align}
\label{eqn:rte_hlf_dwn_dfn}
-\plrmu \frac{\dfr\ntndwnfrq}{\dfr\tau} & = \ntndwnfrq - \srcdwnfrq \\
\label{eqn:rte_hlf_upw_dfn}
\plrmu \frac{\dfr\ntnupwfrq}{\dfr\tau} & = \ntnupwfrq - \srcupwfrq
\end{align}
\end{subequations}
These are the equations of radiative transfer in slab geometry for  
downwelling ($0 < \plr < \mpi/2$) and upwelling 
($\mpi/2 < \plr < \mpi$) intensities, respectively.
The only mathematical difference between (\ref{eqn:rte_hlf_dwn_dfn})
and (\ref{eqn:rte_hlf_upw_dfn}) is the negative sign.
A helpful mnemonic is that the negative sign on the LHS is associated
with $\ntndwnfrq$ while the implicit unary positive sign is associated
with $\ntnupwfrq$.
Of course the $\ntnfrq$ and $\srcfrq$ terms on the RHS are prefixed
with opposited signs since they represent opposing, but positive
definite, physical processes (absorption and emission).

Equation~(\ref{eqn:rte_plru_dfn}) states that $\ntnupwdwn$ depends
explicitly only on $\ntnupwdwn$ and on $\srcupwdwn$, but has no
\textit{explicit} dependence on $\ntndwnupw$ or on $\srcdwnupw$.
Thus it may appear that $\ntnupw$ and $\ntndwn$ are completely
decoupled from eachother.
However, we shall see that in problems involving scattering, 
$\srcupwdwn$ depends explicitly on $\ntndwnupw$ because scattering
may change the trajectory of photons from upwelling to downwelling and
visa versa.
By coupling $\ntnupw$ to $\ntndwn$, scattering allows the 
entire radiance field to affect the radiance field at every point
and in every direction (modulo the speed of light, of course).
Thus scattering changes the solutions to (\ref{eqn:rte_hlf_dfn}) from 
being locally-dependent to depending on the global radiation field.

As a special case of (\ref{eqn:rte_tau_dfn}), consider a stratified,
non-scattering atmosphere in thermodynamic equilibrium.
Then the source function equals the Planck function $\srcfrq =
\plkfrq = \plkfrqtpt$ and we have
\begin{equation}
% ThS99 p. 151 (5.42)
\plru \frac{\dfr\ntnfrq}{\dfr\tau} = -\ntnfrq + \plkfrq
\label{eqn:rte_plk}
\end{equation}
Equation~(\ref{eqn:rte_plk}) is the basis of radiative transfer in the
thermal infrared, where scattering effects are often negligible.
The solution to (\ref{eqn:rte_plk}) is described in
\S\ref{sxn:rte_plk_sln}. 

\subsection{Integral Equations}\label{sxn:rte_ntg}

\subsubsection[Formal Solutions]{Formal Solutions}\label{sxn:rte_sln_frm}
It is useful to write down the formal solution to
(\ref{eqn:rte_tau_dfn}) before making additional assumptions about
the form of the source function $\srcfrq$.
\begin{eqnarray}
\plru \frac{\dfr\ntnfrq}{\dfr\tau} & = & - \ntnfrq + \srcfrq \nonumber \\
\plru \,\dfr\ntnfrq & = & - \ntnfrq \,\dfr\tau + \srcfrq \,\dfr\tau \nonumber \\
\plru \,\dfr\ntnfrq + \ntnfrq \,\dfr\tau & = & \srcfrq \,\dfr\tau \nonumber \\
\dfr\ntnfrq + \frac{\ntnfrq}{\plru} \,\dfr\tau & = & \plrurcp \srcfrq \,
\dfr\tau
\label{eqn:rte_src_dff}
\end{eqnarray}
Multiplying (\ref{eqn:rte_src_dff}) by the \trmdfn{integrating factor}
$\exptauou$ 
\begin{eqnarray}
\exptauou \,\dfr\ntnfrq + \frac{\exptauou \ntnfrq}{\plru} \,\dfr\tau & =
& \plrurcp \exptauou \srcfrq \,\dfr\tau \nonumber \\
\dfr( \exptauou \, \ntnfrq ) & = & \plrurcp \exptauou \srcfrq
\,\dfr\tau \nonumber \\
\frac{\dfr( \exptauou \, \ntnfrq )}{\dfr\tau} & = & \plrurcp
\exptauou \srcfrq 
\label{eqn:rte_sln_dff}
\end{eqnarray}
The LHS side of (\ref{eqn:rte_sln_dff}) is a complete differential.
The boundary condition which applies to this first degree differential
equation depends on the direction the radiation is traveling. 
Thus we denote the solutions to (\ref{eqn:rte_sln_dff}) as
$\ntnfrqoftaupmu$ and $\ntnfrqoftaummu$ for upwelling and downwelling 
radiances, respectively.
We shall assume that the upwelling intensity at the surface,
$\ntnfrq(\taustr,+\plrmu)$, and the downwelling intensity at the top
of the atmosphere, $\ntnfrq(0,-\plrmu)$, are known quantities. 
Since $\plrmu$ is positive-definite (\ref{eqn:plrmu_dfn}), $+\plrmu$ and 
$-\plrmu$ uniquely specify the angles for which these boundary
conditions apply. 

The solution for upwelling radiance is obtained by replacing $\plru$
in (\ref{eqn:rte_sln_dff}) by $-\plrmu$ because $\plru < 0$ for
upwelling intensities.
\begin{eqnarray}
\frac{\dfr( \expmtauomu \, \ntnupwfrq )}{\dfr\tau} & = & -\plrmurcp
\expmtauomu \srcupwfrq 
\label{eqn:rte_hlf_upw_sln_dff}
\end{eqnarray}
We could have arrived at (\ref{eqn:rte_hlf_upw_sln_dff}) by starting
from (\ref{eqn:rte_hlf_upw_dfn}), and proceeding as above except using 
$\expmtauomu$ as the integrating factor.
We now integrate from the surface to level $\tau$ (i.e., along
a path of decreasing $\tauprm$) and apply the boundary
condition at the surface
\begin{eqnarray}
\left.
\expmtaupomu \ntnfrq(\tauprm,+\plrmu) 
\right|_{\tauprm = \taustr}^{\tauprm=\tau} 
& = & - \plrmurcp \int_{\tauprm = \taustr}^{\tauprm=\tau} 
\expmtaupomu \srcupwfrq \,\dfr\tauprm \nonumber \\
\expmtauomu \ntnfrqoftaupmu -
\expmtausomu \ntnfrq(\taustr,+\plrmu)
& = & \plrmurcp \int_{\tau}^{\taustr}
\expmtaupomu \srcupwfrq \,\dfr\tauprm \nonumber \\
\expmtauomu \ntnfrqoftaupmu  
& = & \expmtausomu \ntnfrq(\taustr,+\plrmu) + 
\plrmurcp \int_{\tau}^{\taustr} 
\expmtaupomu \srcupwfrq \,\dfr\tauprm \nonumber \\
\ntnfrqoftaupmu  
& = & \me^{(\tau - \taustr)/\plrmu} \ntnfrq(\taustr,+\plrmu) + 
\plrmurcp \int_{\tau}^{\taustr} 
\me^{(\tau - \tauprm)/\plrmu} \srcupwfrq \,\dfr\tauprm \nonumber
\end{eqnarray}
Note that $\taustr > \tau$ and $\tauprm > \tau$ so that both of the
transmission factors reduce a beam's intensity between its source (at
$\taustr$ or $\tauprm$) and where it is measured (at $\tau$).
The physical meaning of the transmission factors is more clear if we
write all transmission factors as negative exponentials.
\begin{equation}
% ThS99 p. 157 (5.56)
\ntnfrqoftaupmu  
 = \me^{-(\taustr - \tau)/\plrmu} \ntnfrq(\taustr,+\plrmu) + 
\plrmurcp \int_{\tau}^{\taustr} 
\me^{-(\tauprm - \tau)/\plrmu} \srcfrq(\tauprm,+\plrmu) \,\dfr\tauprm
\label{eqn:rte_sln_upw}
\end{equation}
The first term on the RHS is the contribution of the boundary (e.g.,
Earth's surface) to the upwelling intensity at level $\tau$.
This contribution is attenuated by the optical path of the radiation
between the ground and level $\tau$.
The second term on the RHS is the contribution of the atmosphere to
the upwelling intensity at level $\tau$.
The net upward emission of each parcel of air between the surface and
level $\tau$ is $\srcupwfrq(\tauprm)$, but this internally emitted
radiation is attenuated along the slant path between $\tauprm$ and
$\tau$. 
The $\plrmu^{-1}$ factor in front of the integral accounts for the
slant path of the emitting mass in the atmosphere.

The solution for downwelling radiance is obtained by replacing $\plru$
in (\ref{eqn:rte_sln_dff}) by $\plrmu$ because $\plrmu = \plru$ in the
downwelling hemisphere. 
The resulting expression must be integrated from the upper boundary 
down to level $\tau$, and a boundary condition applied at the top.
\begin{eqnarray}
\frac{\dfr( \exptauomu \, \ntndwnfrq )}{\dfr\tau} & = & \plrmurcp
\exptauomu \srcdwnfrq \nonumber \\
\left. 
\exptaupomu \ntnfrq(\tauprm,-\plrmu) 
\right|_{\tauprm = 0}^{\tauprm=\tau} 
& = & \plrmurcp \int_{\tauprm = 0}^{\tauprm=\tau} 
\exptaupomu \srcdwnfrq \,\dfr\tauprm \nonumber \\
\exptauomu \ntnfrqoftaummu - 
\ntnfrq(0,-\plrmu)
& = & \plrmurcp \int_{0}^{\tau} 
\exptaupomu \srcdwnfrq \,\dfr\tauprm \nonumber \\
\exptauomu \ntnfrqoftaummu  
& = & \ntnfrq(0,-\plrmu) +
\plrmurcp \int_{0}^{\tau} 
\exptaupomu \srcdwnfrq \,\dfr\tauprm \nonumber \\
\ntnfrqoftaummu  
& = & \expmtauomu \ntnfrq(0,-\plrmu) +
\plrmurcp \int_{0}^{\tau} 
\me^{(-\tau + \tauprm)/\plrmu} \srcdwnfrq \,\dfr\tauprm  \nonumber \\
\ntnfrqoftaummu  
& = & \expmtauomu \ntnfrq(0,-\plrmu) +
\plrmurcp \int_{0}^{\tau} 
\me^{-(\tau - \tauprm)/\plrmu} \srcfrq(\tauprm,-\plrmu) \,\dfr\tauprm
\label{eqn:rte_sln_dwn}
\end{eqnarray}

The upwelling and downwelling intensities in a stratified atmosphere
are fully described by (\ref{eqn:rte_sln_upw}) and
(\ref{eqn:rte_sln_dwn}). 
Such formal solutions to the equation of radiative transfer are of
great heuristic value but limited practical use until the source
function is known. 
Note that we have assumed a source function and boundary conditions
which are azimuthally independent, but that the derivation of
(\ref{eqn:rte_sln_upw}) and (\ref{eqn:rte_sln_dwn}) does not 
rely on this assumption. 
It is straightforward to relax this assumption and replace
$\ntnfrqoftaumu$ by $\ntnfrq(\tau,\plrmu,\azi)$ and 
$\srcfrqoftaumu$ by $\srcfrq(\tau,\plrmu,\azi)$ in the above.  

\subsubsection[Thermal Radiation In A Stratified Atmosphere]{Thermal
Radiation In A Stratified Atmosphere}\label{sxn:rte_plk_sln}
Consider a purely absorbing, stratified atmosphere in thermodynamic 
equilibrium.
Then the source function equals the Planck function $\srcfrq =
\plkfrq = \plkfrqtpt$ (\ref{eqn:plk_dfn}) and the radiative transfer
equation is given by (\ref{eqn:rte_plk}). 
It is important to remember that $\plkfrq$ is the \textit{complete}
source function only because we are explicitly neglecting all
scattering processes. 
Thus we need only define the boundary conditions in order to use
(\ref{eqn:rte_sln_upw}) and (\ref{eqn:rte_sln_dwn}) to fully
specify $\ntnfrq$. 
We assume that the surface emits blackbody radiation into the upper
hemisphere 
\begin{equation}
\ntnfrq(\taustr,+\plrmu) = \plkfrq[\tpt(\taustr)]
\label{eqn:plk_bc_btm}
\end{equation}
For brevity we shall define $\plkfrqstr = \plkfrq[\tpt(\taustr)]$.
At the top of the atmosphere, we assume there is no downwelling
thermal radiation.
\begin{equation}
\ntnfrq(0,-\plrmu) = 0
\label{eqn:plk_bc_top}
\end{equation}

The solutions for upwelling and downwelling intensities are then
obtained by using $\srcfrq(\tauprm,\plrmu) = \plkfrq(\tauprm)$ (the
Planck function is isotropic) and substituting (\ref{eqn:plk_bc_btm})
and (\ref{eqn:plk_bc_top}) into (\ref{eqn:rte_sln_upw}) and
(\ref{eqn:rte_sln_dwn}), respectively
\begin{eqnarray}
\label{eqn:plk_sln_upw}
\ntnfrqoftaupmu & = & 
\me^{-(\taustr - \tau)/\plrmu} \plkfrq(\taustr) + 
\plrmurcp \int_{\tau}^{\taustr} 
\me^{-(\tauprm - \tau)/\plrmu} \plkfrq(\tauprm) \,\dfr\tauprm \\
\label{eqn:plk_sln_dwn}
\ntnfrqoftaummu & = & 
\plrmurcp \int_{0}^{\tau} 
\me^{-(\tau - \tauprm)/\plrmu} \plkfrq(\tauprm) \,\dfr\tauprm
\end{eqnarray}
The first term on the RHS of (\ref{eqn:plk_sln_upw}) is the thermal
radiation emitted by the surface, attenuated by absorption in the
atmosphere until it contributes to the upwelling intensity at level
$\tau$.  
The second term on the RHS contains the upwelling intensity arriving
at $\tau$ contributed from the attenuated atmospheric thermal emission
from each parcel between the surface and $\tau$. 
The $\plrmu^{-1}$ factor in front of the integral accounts for the
slant path of the thermally emitting atmosphere.
The RHS of (\ref{eqn:plk_sln_dwn}) is similar but contains no boundary
contribution since the vacuum above the atmosphere is assumed to emit
no thermal radiation.
The upwelling and downwelling intensities in a stratified, thermal
atmosphere are fully described by (\ref{eqn:plk_sln_upw}) and
(\ref{eqn:plk_sln_dwn}).

\subsubsection[Angular Integration]{Angular Integration}\label{sxn:ngl}
Once the solutions for the hemispheric intensities are known, it is
straightforward to obtain the hemispheric fluxes by performing the
angular integrations (\ref{eqn:flx_upw_frq})--(\ref{eqn:flx_dwn_frq}).
\begin{equation}
\flxdwnfrqoftau = 2 \mpi \int_{\plrmu=0}^{\plrmu=1} \ntnfrq
(\tau,-\plrmu) \plrmu \,\dfr\plrmu
\label{eqn:flx_dwn_frq_2}
\end{equation}
Consider first the downwelling flux in a non-scattering, thermal,
isotropic, stratified atmosphere obtained by substituting
(\ref{eqn:plk_sln_dwn}) into (\ref{eqn:flx_dwn_frq_2})
and interchanging the order of integration
\begin{eqnarray}
\flxdwnfrqoftau & = & 2 \mpi \int_{\plrmu=0}^{\plrmu=1}
\left( \plrmurcp \int_{\tauprm = 0}^{\tauprm = \tau} 
\me^{-(\tau - \tauprm)/\plrmu} \plkfrq(\tauprm) \,\dfr\tauprm 
\right) \plrmu \,\dfr\plrmu \nonumber \\
& = & 2 \mpi \int_{\tauprm = 0}^{\tauprm = \tau} 
\int_{\plrmu=0}^{\plrmu=1}
\me^{-(\tau - \tauprm)/\plrmu} \plkfrq(\tauprm) \,\dfr\plrmu \,\dfr\tauprm
\nonumber \\ 
& = & 2 \mpi \int_{\tauprm = 0}^{\tauprm = \tau} 
\plkfrq(\tauprm) \left(
\int_{\plrmu = 0}^{\plrmu = 1}
\me^{-(\tau - \tauprm)/\plrmu} \,\dfr\plrmu \, \right) \,\dfr\tauprm
\label{eqn:flx_dwn_frq_tau}
\end{eqnarray}
Notice that two factors of $\plrmu$ cancelled each other out:
The reduction in irradiance due to non-normal incidence ($\plrmu$)
exactly compensates the increased irradiance due to emission by the
entire slant column which is $\plrmurcp$ times greater than emission
from a vertical column.

In terms of \trmidx{exponential integrals} defined in
Appendix~\ref{sxn:xpn}, the inner integral in parentheses in
(\ref{eqn:flx_dwn_frq_tau}) is $\xpn_{2}(\tau - \tauprm)$
(\ref{eqn:xpn_2_dfn}).  
\begin{equation}
\flxdwnfrqoftau = 2 \mpi \int_{\tauprm = 0}^{\tauprm = \tau} 
\plkfrq(\tauprm) \xpn_{2}(\tau - \tauprm) \,\dfr\tauprm
\label{eqn:flx_dwn_xpn}
\end{equation}
Similar terms arise when we consider the horizontal upwelling flux
obtained by substituting (\ref{eqn:plk_sln_upw}) into
(\ref{eqn:flx_upw_frq}) and we obtain
\begin{eqnarray}
\flxupwfrqoftau & = & 2 \mpi \int_{\plrmu=0}^{\plrmu=1} \ntnfrq
(\tau,+\plrmu) \plrmu \,\dfr\plrmu
\nonumber \\
& = & 2 \mpi \int_{\plrmu=0}^{\plrmu=1} \left( 
\me^{-(\taustr - \tau)/\plrmu} \plkfrq(\taustr) + 
\plrmurcp \int_{\tau}^{\taustr} 
\me^{-(\tauprm - \tau)/\plrmu} \plkfrq(\tauprm) \,\dfr\tauprm
\right) \plrmu \,\dfr\plrmu 
\nonumber \\
& = & 2 \mpi \plkfrq(\taustr) 
\int_{\plrmu=0}^{\plrmu=1} 
\me^{-(\taustr - \tau)/\plrmu} \plrmu \,\dfr\plrmu
+ 2 \mpi \int_{\tau}^{\taustr} 
\plkfrq(\tauprm) \left(
\int_{\plrmu=0}^{\plrmu=1} 
\me^{-(\tauprm - \tau)/\plrmu} 
\,\dfr\plrmu \right) \,\dfr\tauprm
\nonumber \\
& = & 2 \mpi \plkfrq(\taustr) \xpn_{3}(\taustr - \tau) + 
2 \mpi \int_{\tauprm = \taustr}^{\tauprm = \tau}
\plkfrq(\tauprm) \xpn_{2}(\tauprm-\tau) \,\dfr\tauprm
\label{eqn:flx_upw_xpn}
\end{eqnarray}
Subtracting (\ref{eqn:flx_dwn_xpn}) from (\ref{eqn:flx_upw_xpn}) we
obtain the net flux at any layer in a non-scattering, thermal,
stratified atmosphere
\begin{eqnarray}
\flxfrqoftau & = & \flxupwfrqoftau - \flxdwnfrqoftau \nonumber \\
& = & 2 \mpi \left[
\plkfrq(\taustr) \xpn_{3}(\taustr - \tau) + 
\int_{\tau}^{\taustr}
\plkfrq(\tauprm) \xpn_{2}(\tauprm - \tau) \,\dfr\tauprm -
\int_{0}^{\tau} 
\plkfrq(\tauprm) \xpn_{2}(\tau - \tauprm) \,\dfr\tauprm
\right]
\label{eqn:flx_net_xpn}
\end{eqnarray}
Equations~(\ref{eqn:flx_dwn_xpn}) and (\ref{eqn:flx_upw_xpn}) may not
seem useful at this point but their utility becomes apparent in
\S\ref{sxn:flx_trn} where we define the \trmdfn{flux
transmissivity}. 

\subsubsection[Thermal Irradiance]{Thermal Irradiance}\label{sxn:thr}
Assume a non-scattering planetary surface at temperature $\tpt$ emits
blackbody radiation such that $\ntnfrq(\taustr,+\plrmu) =
\plkfrq(\tpt)$ (\ref{eqn:plk_bc_btm}).   
What is the total upwelling thermal irradiance from the surface? 
From (\ref{eqn:flx_upw_frq}) we have
\begin{eqnarray}
\flxupwfrq 
& = & 2 \mpi \int_{0}^{1} \plkfrqtpt \plrmu \, \dfr\plrmu \nonumber \\
& = & 2 \mpi \plkfrqtpt \int_{0}^{1} \plrmu \,\dfr\plrmu \nonumber \\
& = & 2 \mpi \plkfrqtpt \left[ \frac{\plrmu^{2}}{2} \right]_{0}^{1} \nonumber \\
& = & 2 \mpi \plkfrqtpt \left( \frac{1}{2} - 0 \right) \nonumber \\
\flxupwfrq & = & \mpi \plkfrqtpt \nonumber \\
\frac{1}{\mpi} \flxupwfrq & = & \plkfrqtpt 
\label{eqn:flx_frq_sfc_upw}
\end{eqnarray} 
Notice the isotropy of the Planck function allows the factor of 2 from
the azimuthal integration to cancel the mean value of the cosine
weighting function over a hemisphere. 
Moving the remaining factor of $\mpi$ from the azimuthal integration to
the LHS conveniently sets the RHS equal to the Planck function.

We integrate (\ref{eqn:flx_frq_sfc_upw}) over frequency to obtain the
total upwelling thermal irradiance
\begin{eqnarray}
\frac{1}{\mpi} \int_{0}^{\infty} \flxupwfrq  \,\dfr\frq & = &
\int_{0}^{\infty} \plkfrqtpt \,\dfr\frq \nonumber \\ 
\frac{1}{\mpi} \flxupw & = & \int_{0}^{\infty} \plkfrqtpt \,\dfr\frq
\nonumber \\ 
\frac{1}{\mpi} \flxupw & = & \frac{\cststfblt \tpt^{4}}{\mpi} \nonumber \\ 
\flxupw & = & \cststfblt \tpt^{4}
\label{eqn:flx_sfc_upw}
\end{eqnarray} 
Thus the factor of $\mpi$ from the azimuthal integration nicely cancels
the factor of $\mpi$ from the \trmidx{Stefan-Boltzmann Law}.
Equation~(\ref{eqn:flx_sfc_upw}) applies to any surface whose
emissivity is 1. 
Consider, e.g., a thick cloud with cloud base and cloud top
temperatures $\tpt(\zzz_{\btmsbs}) = \tpt_{\btmsbs}$  and
$\tpt(\zzz_{\topsbs}) = \tpt_{\topsbs}$, respectively.  
Then the upwelling thermal flux at cloud top and the downwelling flux
at cloud bottom will be $\flxdwn(\zzz_{\btmsbs}) = \cststfblt \tpt_{\btmsbs}^{4}$
and $\flxupw(\zzz_{\topsbs}) = \cststfblt \tpt_{\topsbs}^{4}$, respectively.

\subsubsection[Grey Atmosphere]{Grey Atmosphere}\label{sxn:gry}
Consider an atmosphere transparent to solar radiation and partially
opaque to thermal radiation governed by (\ref{eqn:rte_plk}).
The exact solution, including angular dependence, is given in
\S\ref{sxn:rte_plk_sln}.
The hemispheric fluxes and net flux may only be obtained exactly by
accounting for the angular dependence as in \S\ref{sxn:ngl}.
We may eliminate the angular dependence of the net flux by making the
simplifying assumption that the hemispheric up and downwelling
irradiances equal a constant times the corresponding intensity. 
A number of methods exist to determine this constant, called the 
\trmidx{diffusivity factor} (e.g., \S\ref{sxn:dff_prx}).
These methods are all related to the \trmidx{two-stream approximation}.

% Method used in Sal96 p. 232
One such method \cite[][p.~232]{Sal96} is to identify an effective
inclination $\plrmubar$ along which all radiation is assumed to
travel.
With this assumption, the contribution to upwelling irradiance from
the lower boundary, the first term on the RHS of \ref{eqn:flx_net_xpn}), is
\begin{eqnarray}
2 \xpn_{3}(\taustr - \tau) & = & 
\exp \left( -\frac{\taustr - \tau}{\plrmubar} \right)
\label{eqn:dfs_fct_apx}
\end{eqnarray}
Inspection (or differentiation) shows that the atmosphere within one
optical depth makes the dominant contribution to~(\ref{eqn:dfs_fct_apx}).
For $\dlt\tau = \taustr - \tau = 1$, the diffusivity factor 
\begin{eqnarray}
\plrmubar^{-1} & \approx & \frac{5}{3}
\label{eqn:plr_mu_bar_dfn}
\end{eqnarray}
The hypothetical collimated beam of radiation is inclined to the zenith
by $\arccos(3/5) \approx 53.13\dgr$.
This is equivalent to a collimated beam of radiation travelling
vertically through an optical depth equal to five thirds the vertical
optical depth traversed by the diffuse radiation. 

Using this assumption (\ref{eqn:dfs_fct_apx}), the upwelling
hemispheric irradiance (\ref{eqn:flx_upw_frq}) for blackbody radiation is
\begin{eqnarray}
\flxupwfrq & = & 
2 \mpi \int_{0}^{1} \ntnfrq (+\plrmu) \plrmu \, \dfr\plrmu \nonumber \\
& \approx & 2 \mpi \int_{0}^{1} \ntnfrq (+\plrmubar) \plrmu \, \dfr\plrmu \nonumber \\
& = & 2 \mpi \plrmubar \ntnfrq(+\plrmubar) \int_{0}^{1} \plrmu \, \dfr\plrmu \nonumber \\
& = & 2 \mpi \ntnfrq (+\plrmubar) \left[ \frac{\plrmu^{2}}{2} \right]_{0}^{1} \nonumber \\
& = & 2 \mpi \ntnfrq (+\plrmubar) \left[ \frac{1}{2} - 0 \right] \nonumber \\
& = & \mpi \ntnfrq (+\plrmubar) \nonumber \\
\label{eqn:flx_ntn_apx}
\end{eqnarray}
% Hou02 p. 12
An analogous relationship holds for the downwelling irradiance.

Based on (\ref{eqn:plr_mu_bar_dfn}) and (\ref{eqn:flx_ntn_apx}),
the irradiance structure of the thermal atmosphere may approximated
by performing a direct angular integration of (\ref{eqn:rte_hlf_dfn}).
With our approximation, radiances $\ntn$ integrate directly to
irradiances $\flx$ modulo the diffusivity factor $\plrmubar$. 
(\ref{eqn:rte_plk})
\begin{subequations}
% Hou02 p. 12 (2.6)
\label{eqn:dff_hlf_dfn}
\begin{align}
\label{eqn:dff_hlf_dwn_dfn}
-\plrmubar \frac{\dfr\flxdwnfrq}{\dfr\tau} & = \flxdwnfrq - \mpi\flxplkfrq \\
\label{eqn:dff_hlf_upw_dfn}
\plrmubar \frac{\dfr\flxupwfrq}{\dfr\tau} & = \flxupwfrq - \mpi\flxplkfrq
\end{align}
\end{subequations}

It is instructive to examine an idealized \trmdfn{grey atmosphere},
where the fluxes of interest have no spectral dependence.
Although this is far from true in Earth's atmosphere, the solution is
straightforward and sheds light on the 
\trmidx{radiative equilibrium temperature profile} and the 
\trmidx{greenhouse effect}. 
We simplify (\ref{eqn:dff_hlf_dfn}) in two ways.
First, we introduce a scaled optical depth
$\dfr\tautld = \plrmubar^{-1}\dfr\tau = \frac{5}{3}\dfr\tau$.
Second, we drop the frequency dependence, which is equivalent to
integrating over a broad range of frequencies.
For heuristic purposes, think of this integration as being over the
relatively narrow $\lambda = 5$--$20$\,\um\ range where most of
Earth's terrestrial radiative energy resides.
\begin{subequations}
% Hou02 p. 12 (2.6)
\label{eqn:gry_hlf_dfn}
\begin{align}
\label{eqn:gry_hlf_dwn_dfn}
-\frac{\dfr\flxdwn}{\dfr\tautld} & = \flxdwn - \mpi\plkfnc \\
\label{eqn:gry_hlf_upw_dfn}
\frac{\dfr\flxupw}{\dfr\tautld} & = \flxupw - \mpi\plkfnc
\end{align}
\end{subequations}

In the absence of dynamical, chemical, and latent heating, the energy
deposition in a parcel of air is entirely radiative.
Under these conditions the idealize grey atmosphere described by
(\ref{eqn:gry_hlf_dfn}) will adjust to a temperature profile
determined by \trmidx{radiative equilibrium}.
Let the time rate of change of temperature $\tpt$ of a parcel be
denoted by $\htr$\,[\kxs], the parcel \trmidx{warming rate}\footnote{% 
Conventionally $\htr$ is called the \trmidx{heating rate}, which is a
misnomer since the dimensions of $\dfr\tpt/\dfr\tm$ are \kxs.
Thus we use the less common but more accurate terminology ``warming rate''.
We reserve ``heating rate'' for a measure power dissipation, i.e.,
energy per unit time, in \jxs, \jxmCs, or \jxkgs.} or \trmidx{cooling rate}. 
The warming rate is the rate of net energy deposition divided by the
\trmidx{specific heat capacity at constant pressure} 
$\heatcpcspcprs$\,[\jxkgK] times the density $\dnsatm$\,[\kgxmC]
\begin{eqnarray}
%\frac{\dfr\flx}{\dfr\hgt} & = & \dnsatm \heatcpcspcprs 
\htr \equiv \frac{\dfr\tpt}{\dfr\tm} & = & 
\frac{1}{\dnsatm \heatcpcspcprs} \frac{\dfr\flx}{\dfr\hgt} \\
\frac{\mbox{K}}{\mbox{s}} & = & 
\left( \frac{\mbox{kg}}{\mbox{\mC}} \times \frac{\mbox{J}}{\mbox{kg\,K}} \right)^{-1} 
\times \frac{\mbox{J}}{\mbox{\mSs}} \times \frac{1}{\mbox{m}} \nonumber
\label{eqn:flx_dvr_hgt_dfn}
\end{eqnarray}
The forcing term on the RHS, $\dfr\flx/\dfr\hgt$ is the 
\trmidx{radiative flux divergence}, the vertical gradient of 
net radiative flux.
Absorption and emission are the only mechanisms which contribute to
the flux divergence.
In terms of hemispheric fluxes,
\begin{eqnarray}
\frac{\dfr\flx}{\dfr\hgt} & = & \frac{\dfr}{\dfr\hgt} (\flxdwn - \flxupw)
\label{eqn:flx_dvr_dfn}
\end{eqnarray}

By definition, the time variation of net radiative heating vanishes 
($\dfr\tpt/\dfr\tm = 0$) at all levels of an atmosphere in radiative
equilibrium. 
By (\ref{eqn:flx_dvr_hgt_dfn}), we see that the vertical gradient in 
net radiative flux must also vanish in radiative equilibrium.
Setting (\ref{eqn:flx_dvr_dfn}) to zero and integrating we obtain
\begin{eqnarray}
\flxdwn - \flxupw & = & \flxnot
\label{eqn:flx_dvr_cst}
\end{eqnarray}
The net radiative flux $\flx(\hgt) = \flxnot$ is constant in 
radiative equilibrium.

Adding and subtracting (\ref{eqn:gry_hlf_dfn}), we obtain
\begin{subequations}
% Hou02 p. 12 (2.6)
\label{eqn:gry_hlf_rdc}
\begin{align}
\label{eqn:gry_hlf_dwn_rdc}
\frac{\dfr}{\dfr\tautld}(\flxupw - \flxdwn) & = \flxupw + \flxdwn - 2\mpi\plkfnc \\
\label{eqn:gry_hlf_upw_rdc}
\frac{\dfr}{\dfr\tautld}(\flxupw + \flxdwn) & = \flxupw - \flxdwn = \flxnot
\end{align}
\end{subequations}
Defining $\psi = \flxupw - \flxdwn$ and $\phi = \flx = \flxupw - \flxdwn$,
\begin{subequations}
% Hou02 p. 12 (2.6)
\label{eqn:gry_hlf_rdc_grk}
\begin{align}
\label{eqn:gry_hlf_dwn_rdc_grk}
\frac{\dfr\phi}{\dfr\tautld} & = \psi - 2\mpi\plkfnc \\
\label{eqn:gry_hlf_upw_rdc_grk}
\frac{\dfr\psi}{\dfr\tautld} & = \phi = \flxnot
\end{align}
\end{subequations}
Since $\phi = \flxupw - \flxdwn$ is constant,
(\ref{eqn:gry_hlf_upw_rdc_grk}) shows that $\dfr\phi/\dfr\tautld = 0$
and thus
\begin{eqnarray}
\psi = 2\mpi\plkfnc
\label{eqn:gry_psi_dfn}
\end{eqnarray}
Substituting this into (\ref{eqn:gry_hlf_upw_rdc_grk}),
\begin{eqnarray}
\frac{\dfr}{\dfr\tautld}(2\mpi\plkfnc) & = & \flxnot \nonumber \\
\frac{\dfr\plkfnc}{\dfr\tautld} & = & \dpysty \frac{\flxnot}{2\mpi} \nonumber \\
\plkfnc(\tautld) & = & \dpysty \frac{\flxnot\tautld}{2\mpi} + \cstone
\label{eqn:flx_dvr_prs_dfn}
\end{eqnarray}
We evaluate the constant of integration by using the boundary
condition at the top of the atmosphere.
By definition $\flxdwn = 0$ and $\tautld = 0$ at \trmidx{TOA}.
This implies that $\phi = \psi(0) = \flxupw(0) = \flxnot$.
We must therefore have $\plkfnc(0) = \phi/2\mpi$ by
(\ref{eqn:gry_psi_dfn}).
Using this result in (\ref{eqn:flx_dvr_prs_dfn}),
\begin{eqnarray}
\plkfnc(0) & = & \dpysty \frac{\phi}{2\mpi} = 
\dpysty \frac{\flxnot(0)}{2\mpi} + \cstone \nonumber \\
\cstone & = & \dpysty \frac{\phi}{2\mpi} = \frac{\flxnot}{2\mpi} \nonumber \\
\label{eqn:cst_one_dfn}
\end{eqnarray}
Substituting (\ref{eqn:cst_one_dfn}) back into
(\ref{eqn:flx_dvr_prs_dfn}) shows  
\begin{eqnarray}
\plkfnc(\tautld) & = & \dpysty \frac{\flxnot\tautld}{2\mpi} + \frac{\flxnot}{2\mpi} \nonumber \\
& = & \dpysty \frac{\flxnot}{2\mpi}(\tautld + 1)
\label{eqn:gry_plk_flx_dfn}
\end{eqnarray}
The thermal absorption and emission in a grey atmosphere increase
linearly with optical depth from TOA to the surface.

The upwelling irradiance at the surface is 
$\flxupw(\tautldstr) = \mpi\plkfnc(\tptsfc)$ where $\tptsfc$\,[\K] is the
surface skin temperature.
The atmospheric temperature just above the surface is given by
(\ref{eqn:gry_plk_flx_dfn}) as
$\plkfnc(\tautldstr) = \frac{\flxnot}{2\mpi}(\tautldstr + 1)$.
Thus there is a temperature discontinuity between the near-surface air
and the ground.

Climate models typically express $\htr$ (\ref{eqn:flx_dvr_hgt_dfn}) in 
terms of the flux gradient with respect to pressure by invoking the
hydrostatic equilibrium condition (\ref{eqn:hyd_eqm}) 
\begin{eqnarray}
\htr \equiv \frac{\dfr\tpt}{\dfr\tm} & = & 
\frac{1}{\dnsatm \heatcpcspcprs} \frac{\dfr\flx}{[-(\dnsatm\grv)^{-1}\dfr\prs]} \nonumber \\
& = & -\frac{\grv}{\heatcpcspcprs} \frac{\dfr\flx}{\dfr\prs} \\
& = & -\frac{\grv}{\heatcpcspcprs} 
\frac{(\flxdwn_{\kkk} - \flxupw_{\kkk})-(\flxdwn_{\kkk+1} - \flxupw_{\kkk+1})}
{\prs_{\kkk}-\prs_{\kkk+1}} \nonumber \\
& = & \frac{\grv}{\heatcpcspcprs} 
\frac{(\flxdwn_{\kkk} - \flxdwn_{\kkk+1}) + (\flxupw_{\kkk+1} - \flxupw_{\kkk})}
{\prs_{\kkk+1}-\prs_{\kkk}}
\label{eqn:flx_dvr_prs_dfn}
\end{eqnarray}
where the subscript denotes the $\kkk$th vertical interface level in
the atmosphere. 

\subsubsection[Scattering]{Scattering}\label{sxn:sct}
Energy interacting with matter undergoes one of two processes,
\trmdfn{scattering} or absorption.
Scattering occurs when a photon reflects off matter without
absorption. 
The direction of the photon after the interaction is usually not the
same as the incoming direction.
The case where the scattered photons are homogeneously distributed
throughout all $4\mpi$ steradians is called \trmdfn{isotropic
scattering}. 
In general, the angular dependence of the scattering is described by
the \trmdfn{phase function} of the interaction. 
The phase function $\phzfnc$ is closely related to the probability
that photons incoming from the direction $\nglhatprm =
(\plrprm,\aziprm)$ will (if scattered) scatter into outgoing
direction $\nglhat = (\plr,\azi)$. 
It is usually assumed that $\phzfnc$ depends only on the
\trmdfn{scattering angle}
$\nglsct$ between incident and emergent directions.
\begin{eqnarray}
\cos \nglsct & = & \nglhatprm \cdot \nglhat
\label{eqn:ngl_sct_dfn}
\end{eqnarray}
The case where incident and emergent directions are equal, i.e.,
$\nglprm = \ngl$ corresponds to $\nglsct = 0$.
When the scattered direction continues moving in the forward
hemisphere (relative to the plane defined by $\nglhatprm$), it is called
\trmdfn{forward scattering}, and corresponds to $\nglsct < \mpi/2$.
When scattered radiation has been reflected back into the
hemisphere from whence it arrived, it is called \trmdfn{back
scattering}, and corresponds to $\nglsct > \mpi/2$. 
The case where incident and emergent directions are opposite, i.e.,
$\nglhatprm = -\nglhat$, corresponds to $\nglsct = \mpi$.

The Cartesian components of $\nglprm$ and $\ngl$ are straigtforward to
obtain in \trmidx{spherical polar coordinates}.
\begin{eqnarray}
\nglhat & = & \sin \plr \cos \azi \, \ihat + \sin \plr \sin \azi \, \jhat +
\cos \plr \, \khat \\
\nglhatprm & = & \sin \plrprm \cos \aziprm \, \ihat + \sin \plrprm \sin
\aziprm \, \jhat + \cos \plrprm \, \khat
\label{eqn:ngl_crt_dfn}
\end{eqnarray}

The scattering angle $\nglsct$ is simply related to the inner product 
of $\nglhatprm$ and $\nglhat$ by the cosine law
\begin{eqnarray}
\cos \nglsct & = & \nglhatprm \cdot \nglhat \nonumber \\
& = & \sin \plrprm \cos \aziprm \sin \plr \cos \azi + \sin \plrprm
\sin \aziprm \sin \plr \sin \azi + \cos \plrprm + \cos \plr \nonumber \\
& = & \sin \plrprm \sin \plr \, ( \cos \aziprm \cos \azi + \sin \aziprm
\sin \azi ) + \cos \plrprm \cos \plr \nonumber \\
& = & \sin \plrprm \sin \plr \cos (\aziprm - \azi) + \cos \plrprm \cos \plr 
\label{eqn:ngl_sct_dfn2}
\end{eqnarray}

\subsubsection[Phase Function]{Phase Function}\label{sxn:phz_fnc}
Accurate treatment of the angular scattering of radiation, i.e., the
phase function, is, perhaps, makes rigorous demands of radiative
transfer applications.
A correspondingly large body of literature is devoted to this topic.
Essential references include \citet{Van57}, \citet{JWW76},
\citet{WiG76}, \citet{Wis771}, \citet{Wis79}, and \citet{Bou98}.

The \trmidx{phase function} $\phzfnc(\cos \nglsct)$ is normalized so
that the total probability of scattering is unity
\begin{subequations}
% ThS99 p. 75 (3.24)
\label{eqn:phz_fnc_nrm}
\begin{align}
\label{eqn:phz_fnc_nrm_sht}
\frac{1}{4\mpi} \int_{\ngl} \phzfnc(\cos \nglsct) \,\dfr\ngl & = 1 \\
\label{eqn:phz_fnc_nrm_lng}
\frac{1}{4\mpi} \int_{\azi = 0}^{\azi = 2\mpi} 
\int_{\plr = 0}^{\plr = \mpi} 
\phzfnc(\plrprm,\aziprm;\plr,\azi) \, \sin \plr \,\dfr\plr
 \,\dfr\azi
& = 1
\end{align}
\end{subequations}
The dimensions of the phase function are somewhat ambiguous.
If the $(4\mpi)^{-1}$ factor in (\ref{eqn:phz_fnc_nrm}) is assumed
to be steradians, then $\phzfnc$ is a true probability and is
dimensionless. 
However, if the $(4\mpi)^{-1}$ factor is considered to be numeric
and dimensionless (i.e., a probability), then $\phzfnc$ has units
of (dimensionless) probability per (dimensional) steradian, \xsr.
The latter convention best expresses the physical meaning of the
phase function and is adopted in this text.
It is therefore important to remember that factors of $(4\mpi)^{-1}$ 
which multiply the scattering integral in the radiative transfer
equation are considered to be dimensionless in the formulations
which follow, e.g., (\ref{eqn:src_sct_frq_dfn}).
Furthermore, the units of $\phzfnc$ are probability per steradian,
\xsr.  

Scattering may depend on the absolute directions $\nglhatprm$ and
$\nglhat$ themselves, rather than just their relative orientations as
measured by the angle $\nglsct$ between them.  
This might be the case, for example, in a broken sea-ice field.
For the time being, however, we shall assume that the phase function
depends only on $\nglsct$.

In atmospheric problems, the phase function may often be independent
of the azimuthal angle $\phi$, and depend only on $\plr$.
In this case the phase function normalization
(\ref{eqn:phz_fnc_nrm_lng}) simplifies to 
\begin{eqnarray}
\label{eqn:phz_fnc_nrm_plr}
\frac{1}{2}
\int_{\plr = 0}^{\plr = \mpi} 
\phzfnc(\plrprm;\plr) \, \sin \plr \,\dfr\plr
& = & 1
\end{eqnarray}
In accord with the discussion above, the factor of $1/2$ is
dimensionless, as is the RHS. 

\subsubsection[Legendre Basis Functions]{Legendre Basis Functions}
% Good discussion of phase functions here:
% http://omlc.ogi.edu/pubs/prahl-pubs/prahl88/node10.html

The phase function (\ref{eqn:phz_fnc_nrm}) specifies the angular
distribution of scattering.
Solution of any radiative transfer equation involving scattering
depends on it.
Numerical approaches aim for a suitable approximation or
discretization which represents $\phzfncofnglsct$ to some desired
level of accuracy. 
The optimal basis functions for representing $\phzfncofnglsct$
are the \trmdfn{Legendre polynomials}.

The Legendre polynomial expansion of the phase function is 
\begin{eqnarray}
\label{eqn:phz_fnc_xpn_lgn}
\phzfnc(\cos\nglsct) & = & \sum_{\plridx=0}^{\plridx=\NNN}
(2\plridx+1) \lgnxpncffplr \lgnplrofcosnglsct 
\end{eqnarray}
An expansion of order~$\NNN$ contains $\NNN+1$ terms. 
The zeroth order polynomial and coefficient are identically~1.

The Legendre polynomials are \trmidx{orthonormal} on the interval
$[-1,1]$.
The factor of $2\plridx+1$ that appears in the numerator of the
Legendre expansion (\ref{eqn:phz_fnc_xpn_lgn}) also appears
in the denominator of the Legendre polynomial orthonormality
property: 
\begin{subequations}
% ThS99 p. 178
\label{eqn:lgn_fnc_nrm_plr}
\begin{align}
\frac{1}{2}
\int_{\plru = -1}^{\plru = 1} 
\lgnplrofplru \lgnfnc_{\kkk}(\plru) \,\dfr\plru & =
\frac{1}{2\plridx + 1} \dltsubtwoarg{\plridx}{\kkk} \\
\frac{1}{2}
\int_{\plru = -1}^{\plru = 1} 
\lgnplrofcosnglsct \lgnfnc_{\kkk}(\cos\nglsct) \,\dfr(\cos\nglsct) & =
\frac{1}{2\plridx + 1} \dltsubtwoarg{\plridx}{\kkk}
\end{align}
\end{subequations}
Some other properties of Legendre polynomials are discussed in
\S\ref{sxn:gss}. 

The \trmidx{expansion coefficients} $\lgnxpncffplr$ are defined
by the projection of the corresponding Legendre polynomial onto
the phase function.
\begin{subequations}
% ThS99 p. 177 (6.29)
% Bou98 p. 128 (2) wrong lower bound of integration (should be -1)
\label{eqn:lgn_xpn_cff}
\begin{align}
\lgnxpncffplr & =
\frac{1}{2}
\int_{\plru = -1}^{\plru = 1} 
\lgnplrofplru \phzfnc(\plru) \,\dfr\plru \\
& =
\frac{1}{2}
\int_{\nglsct = \mpi}^{\nglsct = 0} 
\lgnplrofcosnglsct \phzfnc(\cos\nglsct) \,\dfr(\cos\nglsct) \\
& =
\frac{1}{2}
\int_{\nglsct = 0}^{\nglsct = \mpi} 
\lgnplrofcosnglsct \phzfnc(\cos\nglsct) \sin\nglsct \,\dfr\nglsct
\end{align}
\end{subequations}
map the phase function into a series of Legendre polynomials.
The $\lgnxpncffplr$ are also called the 
\trmidx{phase function moments}.
Radiative transfer programs like \trmidx{DISORT} \citep{STW88}
usually require that directional information be specified as
a Legendre polynomial expansion.

The Legendre expansion for simple, smoothly varying and symmetric
phase functions is highly accurate with just few terms.
For instance, a single moment Legendre expansion exactly describes
Rayleigh scattering.
However, the expansions of asymmetric phase functions converge
much more slowly.
The strongly peaked forward scattering lobes which appear at
large size parameter may require hundreds of moments for accurate
representation.

Computing the $\lgnxpncffplr$ using quadrature methods significantly
increases accuracy and reduces time.
\citet{Wis771} recommends \trmidx{Gauss-Lobatto} quadrature for
highly asymmetric phase functions:
\begin{eqnarray}
% Wis771 p. 1421 (A2)
\label{eqn:lgn_xpn_cff_lbb}
\lgnxpncffplr & = &
\frac{1}{2}
\sum_{\lbbidx = 1}^{\lbbidx = \lbbnbr} 
\wgtlbbidx \lgnplr(\cos\nglsctlbb) \phzfnc(\cos\nglsctlbb) \sin\nglsctlbb
\end{eqnarray}
where $\lbbnbr$ is the number of Gauss-Lobatto abscissae.
Lobatto quadrature is discussed more thorought in \S~\ref{sxn:qdrlbb}.

\subsubsection[Henyey-Greenstein Approximation]{Henyey-Greenstein Approximation}
Computation of the exact phase function of scatterers is laborious but 
desirable when the objective is to predict directional radiances.
However, phase function approximations often suffice when hemispheric
fluxes are the objective.
The \trmdfn{Henyey-Greenstein Phase Function} approximates the full
phase function in terms of its first Legendre moment, i.e., its
\trmidx{asymmetry parameter}~$\asmprm$.
\begin{eqnarray}
% Jac99 p. 286 (10.77)
\label{eqn:phz_fnc_HG}
\phzfnc(\cos\nglsct) 
& = & 
\frac{1-\asmprm^{2}}{(1+\asmprm^{2}-2\asmprm\cos\nglsct)^{3/2}}
\end{eqnarray}
The $\plridx$'th moment in the Legendre expansion of
(\ref{eqn:phz_fnc_HG}) is, conveniently,~$\asmprm^{\plridx}$.
\begin{eqnarray}
\lgnxpncffplr & = & \asmprm^{\plridx}
\label{eqn:phz_fnc_mmn_HG}
\end{eqnarray}
Applying (\ref{eqn:phz_fnc_mmn_HG}) in (\ref{eqn:phz_fnc_xpn_lgn}) gives
\begin{eqnarray}
\label{eqn:phz_fnc_xpn_lgn}
\phzfnc(\cos\nglsct) & = & \sum_{\plridx=0}^{\plridx=\NNN}
(2\plridx+1) \asmprm^{\plridx} \lgnplrofcosnglsct 
\end{eqnarray}
This convenient property (\ref{eqn:phz_fnc_mmn_HG}) makes numerical
quadrature of the Legendre expansion coefficients unnecessary once the
first moment, $\asmprm = \lgnxpncff_{1}$, is known.

\subsubsection[Direct and Diffuse Components]{Direct and Diffuse
Components}\label{sxn:drc_dff}
When working with \trmidx{half-range intensities} it is convenient to
decompose  the downwelling intensity $\ntndwn$ into the sum of a
\trmidx{direct component}, $\ntndwndrcoftaungl$, and a \trmidx{diffuse 
component}, $\ntndwndffoftaungl$ such that 
\begin{eqnarray}
\ntndwnfrqoftaungl & = & \ntndwndrcoftaungl + \ntndwndffoftaungl
\label{eqn:ntn_frq_dwn_drc_dff}
\end{eqnarray}
where we have suppressed the $\nu$ subscript on the RHS to 
simplify notation.
The direct component refers to any photons contributing from a 
collimated source which have not (yet) been scattered.
Typically the collimated source is solar radiation, and so we
subscript the direct component with $\drcsbs$ for ``solar''.
There may be a corresponding direct component of intensity in the
upwelling direction $\ntnupwdrc$ if the reflectance at the lower
surface is \trmidx{specular}, e.g., ocean glint. 
In such situations it is straightforward to define
\begin{eqnarray}
\ntnupwfrqoftaungl & = & \ntnupwdrcoftaungl + \ntnupwdffoftaungl
\label{eqn:ntn_frq_upw_drc_dff}
\end{eqnarray}
Of course there exist planetary atmospheres \textit{somewhere} which
are illuminated by multiple stars.
We shall neglect upwelling solar beams for the time being.
For consistency, though, we shall shall use $\ntnupwdff$ rather than
$\ntnupw$ in equations in which $\ntndwndff$ also appears.

It is plain that $\ntndwndrcoftaungl$ is zero in all directions except
that of the collimated beam.
Moreover, the intensity in the direct beam is, by definition, subject
only to extinction by Bougher's law (i.e., there is no emission).
Thus the direct beam component of the downwelling intensity is the 
irradiance incident at the top of the atmosphere, attenuated by the
extinction law (\ref{eqn:flx_slr_toa_dlt_xmp_4}),  
\begin{eqnarray}
\ntndwndrcoftaungl 
& = & 
\flxslrtoa \me^{-\tau/\plrmunot} \dltfncofnglhatmnglhatnot \\
& = & 
\flxslrtoa \me^{-\tau/\plrmunot} \dltfnc(\plrmu - \plrmunot) 
\dltfnc(\azi - \azinot) \nonumber
\label{eqn:ntn_dwn_drc_dfn}
\end{eqnarray}

\subsubsection[Source Function]{Source Function}\label{sxn:src_slb}
The source function for thermal emission $\srcemsfrq$ is
\begin{eqnarray}
\srcemsfrq & = & \frac{\vlmemscff(\frq)}{\xsxext(\frq)} \nonumber \\
& = & \frac{\xsxabs(\frq)}{\xsxext(\frq)}\plkfrq(\tpt) \nonumber \\
& = & \frac{\xsxext(\frq)-\xsxabs(\frq)}{\xsxext(\frq)}\plkfrq(\tpt) \nonumber \\
& = & (1-\ssa)\plkfrq(\tpt)
\label{eqn:src_ems_frq_dfn}
\end{eqnarray}
With knowledge of the phase function $\phzfncofnglprmngl$ as well as
the scattering and absorption coefficients of the particular
extinction process, we can determine the contribution of scattering
to the total source function $\srcfrq$.
Consider the change in radiance $\dfr\ntnfrqofngl$ occuring over a
small change in path $\dfr\tautld$ due to a scattering process with
phase function $\phzfncofnglprmngl$.
The scattering contribution to $\dfr\ntnfrqofngl$ from every incident 
direction $\nglhatprm$ is proportional the radiance of the incident
beam, $\ntnfrqofnglprm$, the probability that extinction of
$\ntnfrqofnglprm$ at location $\tautld$ is due to scattering (not to
absorption), and to the normalized phase function
$\phzfncofnglprmngl/4\mpi$ (\ref{eqn:phz_fnc_nrm}). 
\begin{eqnarray}
\frac{\dfr \ntnfrqofngl}{\dfr\tautld} & = & 
\ntnfrqofnglprm \frac{\xsxsctoffrq}{\xsxsctoffrq + \xsxabsoffrq} 
\frac{\phzfncofnglprmngl}{4\mpi} \nonumber \\
& = & 
\frac{\ssa}{4\mpi} \ntnfrqofnglprm \phzfncofnglprmngl
\label{eqn:src_sct_dfr_dfn}
\end{eqnarray}
The source function for scattering $\srcsctfrq$ is obtained by
integrating (\ref{eqn:src_sct_dfr_dfn}) over all possible incident 
directions $\nglhatprm$ that contribute the exiting radiance in 
direction $\nglhat$
\begin{eqnarray}
\srcsctfrq & = & \frac{\ssa}{4\mpi} 
\int_{\nglprm} \ntnfrq(\tautld,\nglhatprm)
 \phzfnc(\tautld,\nglhatprm,\nglhat) \,\dfr\nglprm
\label{eqn:src_sct_frq_dfn}
\end{eqnarray}
The LHS and RHS of (\ref{eqn:src_sct_frq_dfn}) must have equal
dimensions. 
This is easily verified: $\ssa$ is dimensionless, $(4\mpi)^{-1}$ is
dimensionless (cf.~\S\ref{sxn:phz_fnc}), $\ntnfrq$ and $\srcsctfrq$
both have dimensions of intensity, the dimensions of $\phzfnc$ are
\xsr\ and the cancel the dimensions of $\dfr\nglprm$ which are \sr. 

All terms in (\ref{eqn:src_sct_frq_dfn}) are functions of position and
scattering process.
If there are $\nnn$ distinct scattering processes (e.g., Rayleigh
scattering, aerosol scattering, cloud scattering) then we must know 
the properties of each individual process (e.g., $\xsxsct_{\iii}$,
$\phzfnc_{\iii}$) and sum their contributions to obtain to total
scattering source function $\srcsctfrq = \sum_{\iii=1}^{\iii=\nnn}
\srcfnc_{\frq,\iii}^{\sctsbs}$.
We neglect such details for now and merely remind the reader that
applications need to consider multiple extinction processes
at the same time.

The total source function is obtained by adding the source functions
for thermal emission (\ref{eqn:src_ems_frq_dfn}) and for scattering
(\ref{eqn:src_sct_frq_dfn}) 
\begin{eqnarray}
\srcfrq & = & \srcemsfrq + \srcsctfrq \nonumber \\
\srcfrq & = & (1-\ssa)\plkfrq(\tpt) +
\frac{\ssa}{4\mpi} 
\int_{\nglprm} \ntnfrq(\tautld,\nglhatprm)
\phzfnc(\tautld,\nglhatprm,\nglhat) \,\dfr\nglprm
\label{eqn:src_frq_dfn}
\end{eqnarray}
Although (\ref{eqn:src_frq_dfn}) is complete for sources within a 
medium, additional terms may need to be added to account for
boundary sources, such as 
surface reflection\index{surface reflectance}.
Boundary sources are discussed in \S\ref{sxn:rta}.

\subsubsection[Radiative Transfer Equation in Slab Geometry]{Radiative
Transfer Equation in Slab Geometry}\label{sxn:rte_slb}
Inserting (\ref{eqn:src_frq_dfn}) into (\ref{eqn:rte_tau_tld_dfn}) we
obtain the radiative transfer equation including absorption, thermal
emission and scattering
\begin{eqnarray}
\frac{\dfr\ntnfrq}{\dfr\tautld} & = & -\ntnfrq + \srcfrq \nonumber \\
& = & -\ntnfrq + (1-\ssa)\plkfrq(\tpt) +
\frac{\ssa}{4\mpi} \int_{\nglprm} \ntnfrq(\tautld,\nglhatprm) 
\phzfnc(\tautld,\nglhatprm,\nglhat) \,\dfr\nglprm
\label{eqn:rte_dfn_src}
\end{eqnarray}
This is a general form for the equation of radiative transfer in
one dimension, and the jumping off point for our discussion of 
various solution techniques.
The physics of thermal emission, absorption, and scattering are all
embodied in (\ref{eqn:rte_dfn_src}).
The most appropriate solution technique for (\ref{eqn:rte_dfn_src})
depends on the boundary conditions of the particular problem, the
fields required in the solution (e.g., if irradiance is required but
radiance is not), and the required accuracy of the solution.

We are most interested in solving the radiative transfer equations in 
a slab geometry.
To do this we transform from the path optical depth coordinate
$\tautld$ to the vertical optical depth coordinate $\tau$.
Applying the procedure described in
(\ref{eqn:rte_tau_tld_dfn})--(\ref{eqn:rte_hlf_dfn}) to
(\ref{eqn:rte_dfn_src}) we obtain the radiative transfer equation for
the half range intensities in a slab geometry
\begin{subequations}
% ThS99 p. 171 (6.3)--(6.4)
\label{eqn:rte_dfn_fll_1}
\begin{align}
\label{eqn:rte_dfn_fll_1_dwn}
- \plrmu \frac{\dfr\ntndwn(\tau,\nglhat)}{\dfr\tau}
& = 
\ntndwn(\tau,\nglhat) - (1-\ssa) \plkfnc 
- \frac{\ssa}{4\mpi} \int_{\upwsbs} \ntnupw(\tau,\nglhatprm) 
\phzfnc(+\nglhatprm,-\nglhat) \,\dfr\nglprm
\nonumber \\ & {} % KoD99 p. 138 for spacing info
- \frac{\ssa}{4\mpi} \int_{\dwnsbs} \ntndwn(\tau,\nglhatprm) 
\phzfnc(-\nglhatprm,-\nglhat) \,\dfr\nglprm \\
\label{eqn:rte_dfn_fll_1_upw}
\plrmu \frac{\dfr\ntnupw(\tau,\nglhat)}{\dfr\tau}
& = 
\ntnupw(\tau,\nglhat) - (1-\ssa) \plkfnc 
- \frac{\ssa}{4\mpi} \int_{\upwsbs} \ntnupw(\tau,\nglhatprm) 
\phzfnc(+\nglhatprm,+\nglhat) \,\dfr\nglprm
\nonumber \\ & {} % KoD99 p. 138 for spacing info
- \frac{\ssa}{4\mpi} \int_{\dwnsbs} \ntndwn(\tau,\nglhatprm) 
\phzfnc(-\nglhatprm,+\nglhat) \,\dfr\nglprm
\end{align}
\end{subequations}
If we substitute (\ref{eqn:ntn_frq_dwn_drc_dff}) into
(\ref{eqn:rte_dfn_fll_1_dwn}) we obtain
\begin{eqnarray}
% ThS99 p. 171 (6.5)
\lefteqn{- \plrmu \frac{\dfr\ntndwndff(\tau,\nglhat)}{\dfr\tau}
- \plrmu \frac{\dfr\ntndwndrc(\tau,\nglhat)}{\dfr\tau} = } \nonumber \\
& & \ntndwndff(\tau,\nglhat) + \ntndwndrc(\tau,\nglhat) - (1-\ssa) \plkfnc 
- \frac{\ssa}{4\mpi} \int_{\dwnsbs} \ntndwndrc(\tau,\nglhatprm) 
\phzfnc(-\nglhatprm,-\nglhat) \,\dfr\nglprm \nonumber \\
\nonumber \\ & & {} % KoD99 p. 138 for spacing info
- \frac{\ssa}{4\mpi} \int_{\upwsbs} \ntnupwdff(\tau,\nglhatprm) 
\phzfnc(+\nglhatprm,-\nglhat) \,\dfr\nglprm
- \frac{\ssa}{4\mpi} \int_{\dwnsbs} \ntndwndff(\tau,\nglhatprm) 
\phzfnc(-\nglhatprm,-\nglhat) \,\dfr\nglprm \nonumber
\label{eqn:rte_dfn_fll_2_dwn}
\end{eqnarray}
The direct beam (\ref{eqn:ntn_dwn_drc_dfn}) satisfies the extinction
law (\ref{eqn:flx_slr_toa_dlt_xmp_4}) so that 
$-\plrmu \,\dfr\ntndwndrc/\dfr\tau = \ntndwndrc$.
Thus the second terms on the LHS and the RHS of
(\ref{eqn:rte_dfn_fll_2_dwn}) cancel eachother and we are left with  
\begin{eqnarray}
% ThS99 p. 171 (6.6)
\lefteqn{ - \plrmu \frac{\dfr\ntndwndff(\tau,\nglhat)}{\dfr\tau} = 
\ntndwndff(\tau,\nglhat) - (1-\ssa) \plkfnc - \srcstr(\tau,-\nglhat) }
\nonumber \\ & & {} % KoD99 p. 138 for spacing info
- \frac{\ssa}{4\mpi} \int_{\upwsbs} \ntnupwdff(\tau,\nglhatprm) 
\phzfnc(+\nglhatprm,-\nglhat) \,\dfr\nglprm
- \frac{\ssa}{4\mpi} \int_{\dwnsbs} \ntndwndff(\tau,\nglhatprm) 
\phzfnc(-\nglhatprm,-\nglhat) \,\dfr\nglprm 
\label{eqn:rte_dfn_fll_3_dwn}
\end{eqnarray}
where $\srcstr$ is called the \trmdfn{single-scattering source
function}. 
$\srcstr$ is defined by the integral containing the direct beam on the
RHS of (\ref{eqn:rte_dfn_fll_3_dwn})
\begin{eqnarray}
% ThS99 p. 171 (6.6)
\srcstr(\tau,-\nglhat) 
& = &
\frac{\ssa}{4\mpi} \int_{\dwnsbs} \ntndwndrc(\tau,\nglhatprm) 
\phzfnc(-\nglhatprm,-\nglhat) \,\dfr\nglprm \nonumber \\
& = &
\frac{\ssa}{4\mpi} \int_{\dwnsbs} 
\flxslrtoa \me^{-\tau/\plrmunot} \dltfnc(\nglhatprm,\nglhatnot)
\phzfnc(-\nglhatprm,-\nglhat) \,\dfr\nglprm \nonumber \\
& = &
\frac{\ssa}{4\mpi} 
\flxslrtoa \me^{-\tau/\plrmunot}
\phzfnc(-\nglhatnot,-\nglhat)
\label{eqn:src_str_dwn_dfn}
\end{eqnarray}
Note that (\ref{eqn:rte_dfn_fll_3_dwn}) is an equation for the diffuse 
downwelling radiance, not for the total downwelling radiance.
The relation between the diffuse and total radiance is given by
(\ref{eqn:ntn_frq_dwn_drc_dff}).
The direct component is always known (\ref{eqn:ntn_dwn_drc_dfn}) once
the optical depth has been determined. 

Likewise, inserting (\ref{eqn:ntn_frq_dwn_drc_dff}) and 
(\ref{eqn:ntn_dwn_drc_dfn}) into (\ref{eqn:rte_dfn_fll_1_upw}), we
obtain the equation for the diffuse upwelling intensity
\begin{eqnarray}
\lefteqn{ \plrmu \frac{\dfr\ntnupwdff(\tau,\nglhat)}{\dfr\tau} =  
\ntnupwdff(\tau,\nglhat) - (1-\ssa) \plkfnc - \srcstr(\tau,+\nglhat) }
\nonumber \\ & & {} % KoD99 p. 138 for spacing info
- \frac{\ssa}{4\mpi} \int_{\upwsbs} \ntnupwdff(\tau,\nglhatprm) 
\phzfnc(+\nglhatprm,+\nglhat) \,\dfr\nglprm
- \frac{\ssa}{4\mpi} \int_{\dwnsbs} \ntndwndff(\tau,\nglhatprm) 
\phzfnc(-\nglhatprm,+\nglhat) \,\dfr\nglprm
\label{eqn:rte_dfn_fll_3_upw}
\end{eqnarray}
where
\begin{eqnarray}
% ThS99 p. 171 (6.6)
\srcstr(\tau,+\nglhat) 
& = &
\frac{\ssa}{4\mpi} \int_{\dwnsbs} \ntndwndrc(\tau,\nglhatprm) 
\phzfnc(-\nglhatprm,+\nglhat) \,\dfr\nglprm \nonumber \\
& = &
\frac{\ssa}{4\mpi} \int_{\dwnsbs} 
\flxslrtoa \me^{-\tau/\plrmunot} \dltfnc(\nglhatprm,\nglhatnot)
\phzfnc(-\nglhatprm,+\nglhat) \,\dfr\nglprm \nonumber \\
& = &
\frac{\ssa}{4\mpi} 
\flxslrtoa \me^{-\tau/\plrmunot}
\phzfnc(-\nglhatnot,+\nglhat)
\label{eqn:src_str_upw_dfn}
\end{eqnarray}

\subsubsection[Azimuthal Mean Radiation Field]{Azimuthal Mean Radiation Field}\label{sxn:rte_slb_azi_avg} 
Often we are not concerned with the azimuthal dependence of the
radiation field. 
In some cases this is because the azimuthal dependence is very weak.
For example, heavily overcast skies, or diffuse reflectance from a
uniform surface.
In other cases the azimuthal dependence may not be weak, but there
insufficient information to fully determine the solutions.
For example, the full surface \trmidx{BRDF} or the shapes of clouds
are not available.
The azimuthal dependence of a radiative quantity $\XXX(\azi) $ is
removed by applying the \trmdfn{azimuthal mean operator} 
\begin{eqnarray}
% ThS99 p. 176 (6.22)
\bar{\XXX}_{\azi} & = & 
\frac{1}{2\mpi} \int_{0}^{2\mpi} \XXX(\azi) \,\dfr\azi
\label{eqn:azi_avg_opr}
\end{eqnarray}
For future reference we present also the polar angle integration
operator which will be used to formulate the \trmidx{two stream
equations}. 
\begin{eqnarray}
% ThS99 p. 176 (6.22)
\bar{\XXX}_{\plr} & = & \int_{0}^{\mpi} \XXX(\plr) 
\sin \plr \,\dfr\plr \nonumber \\
\bar{\XXX}_{\plrmu} & = & \int_{0}^{1} \XXX(\plrmu) \,\dfr\plrmu
\label{eqn:plr_avg_opr}
\end{eqnarray}

Applying (\ref{eqn:azi_avg_opr}) to (\ref{eqn:ntn_hms_dfn}) and
(\ref{eqn:phz_fnc_nrm}), we obtain the azimuthal mean
\trmidx{hemispheric intensity}, \trmidx{phase function}, and
\trmidx{single-scattering source function} respectively, 
\begin{eqnarray}
% ThS99 p. 176 (6.22)
\label{eqn:ntn_azi_avg}
\ntnupwdwn(\tau,\plrmu) & = & \frac{1}{2\mpi} \int_{0}^{2\mpi} 
\ntnupwdwn(\tau,\plrmu,\azi) \,\dfr\azi \\
\label{eqn:phz_fnc_azi_avg}
\phzfnc(\pm\plrmuprm;\pm\plrmu) & = & \frac{1}{2\mpi} \int_{0}^{2\mpi}
\phzfnc(\pm\plrmuprm,\aziprm;\pm\plrmu,\azi) \,\dfr\azi \\
\label{eqn:src_str_azi_avg}
\srcstr(\tau,\pm\plrmu) & = &
\frac{\ssa}{4\mpi} \flxslrtoa \me^{-\tau/\plrmunot}
\phzfnc(-\plrmunot,\pm\plrmu)
\end{eqnarray}
No additional symbols are used to indicate azimuthal mean quantities.
The presence of only the polar angle ($\plr$ or $\plrmu$) on the LHS 
of (\ref{eqn:ntn_azi_avg}) indicates that azimuthal mean quantities
are involved. 

Applying (\ref{eqn:azi_avg_opr}) to the full radiative transfer
equations for slab geometry, 
(\ref{eqn:rte_dfn_fll_3_dwn}) and (\ref{eqn:rte_dfn_fll_3_upw}), 
we obtain the radiative transfer equations for the azimuthal mean
intensity in slab geometry
\begin{subequations}
% ThS99 p. 171 (6.6), p. 176 (6.20)
\label{eqn:rte_azi_avg}
\begin{align}
\label{eqn:rte_azi_avg_dwn}
\lefteqn{ - \plrmu \frac{\dfr\ntndwndff(\tau,\plrmu)}{\dfr\tau} = 
\ntndwndff(\tau,\plrmu) - (1-\ssa) \plkfnc - 
\frac{\ssa}{4\mpi} \flxslrtoa \me^{-\tau/\plrmunot}
\phzfnc(-\plrmunot,-\plrmu) }
\nonumber \\ & {} % KoD99 p. 138 for spacing info
- \frac{\ssa}{2} \int_{\upwsbs} \ntnupwdff(\tau,\plrmuprm) 
\phzfnc(+\plrmuprm,-\plrmu) \,\dfr\plrmuprm
- \frac{\ssa}{2} \int_{\dwnsbs} \ntndwndff(\tau,\plrmuprm) 
\phzfnc(-\plrmuprm,-\plrmu) \,\dfr\plrmuprm \\
\label{eqn:rte_azi_avg_upw}
\lefteqn{ \plrmu \frac{\dfr\ntnupwdff(\tau,\plrmu)}{\dfr\tau} =  
\ntnupwdff(\tau,\plrmu) - (1-\ssa) \plkfnc -
\frac{\ssa}{4\mpi} \flxslrtoa \me^{-\tau/\plrmunot}
\phzfnc(-\plrmunot,+\plrmu) }
\nonumber \\ & {} % KoD99 p. 138 for spacing info
- \frac{\ssa}{2} \int_{\upwsbs} \ntnupwdff(\tau,\plrmuprm) 
\phzfnc(+\plrmuprm,+\plrmu) \,\dfr\plrmuprm
- \frac{\ssa}{2} \int_{\dwnsbs} \ntndwndff(\tau,\plrmuprm) 
\phzfnc(-\plrmuprm,+\plrmu) \,\dfr\plrmuprm
\end{align}
\end{subequations} 
The azimuthal mean equations (\ref{eqn:rte_azi_avg}) are very similar
to the full equations (\ref{eqn:rte_dfn_fll_3_dwn}) and
(\ref{eqn:rte_dfn_fll_3_upw}). 
While the remainder of the terms in (\ref{eqn:rte_azi_avg}) contain
only one azimuthally dependent intensity, the scattering integrals
contain two, $\phzfnc$ and $\ntnupwdwn$.
This causes the factor of $\ssa/4\mpi$ in front of the scattering
integrals has to become a factor of $\ssa/2$. 

We apply the two stream formalism introduced in \S\ref{sxn:two_srm} 
to the radiative transfer equation for the azimuthal mean radiation 
field (\ref{eqn:rte_azi_avg}).
Recall that the two stream formalism is obtained by applying three
successive approximations.
The approximation is achieved in three stages.
First, we operate on both sides of (\ref{eqn:rte_azi_avg_dwn}) with
the hemispheric averaging operator (\ref{eqn:plr_avg_opr}).
The procedure for (\ref{eqn:rte_azi_avg_dwn}) is identical.
\begin{eqnarray}
% ThS99 p. 171 (6.6), p. 176 (6.20)
\label{eqn:rte_plr_avg_1}
\lefteqn{ - \int_{0}^{1} \plrmu
\frac{\dfr\ntndwndff(\tau,\plrmu)}{\dfr\tau} \,\dfr\plrmu = 
\int_{0}^{1} \ntndwndff(\tau,\plrmu) \,\dfr\plrmu - 
(1-\ssa) \int_{0}^{1} \plkfnc  \,\dfr\plrmu - 
\frac{\ssa}{4\mpi} \flxslrtoa \me^{-\tau/\plrmunot}
\int_{0}^{1} \phzfnc(-\plrmunot,-\plrmu) \,\dfr\plrmu }
\nonumber \\ & & {} % KoD99 p. 138 for spacing info
- \frac{\ssa}{2} \int_{0}^{1} \int_{\upwsbs} \ntnupwdff(\tau,\plrmuprm) 
\phzfnc(+\plrmuprm,-\plrmu) \,\dfr\plrmuprm \,\dfr\plrmu
- \frac{\ssa}{2} \int_{0}^{1} \int_{\dwnsbs} \ntndwndff(\tau,\plrmuprm) 
\phzfnc(-\plrmuprm,-\plrmu) \,\dfr\plrmuprm \,\dfr\plrmu
\end{eqnarray}

Further approximations are required to simplify equations
(\ref{eqn:rte_plr_avg_2}). 
These approximations allow us to decouple products of functions 
with $\plrmu$ dependence.
The first approximation, (\ref{eqn:two_srm_plr_mu_apx}),
replaces the continuous value $\plrmu$ on the LHS of
(\ref{eqn:rte_azi_avg}) by a suitable hemispheric mean value
$\plrmubar$.
% fxm: Write a sxn:plr_mu_avg which discusses choices of plrmubar
\begin{eqnarray}
\int_{0}^{1} \plrmu \frac{\dfr\ntndwndff(\tau,\plrmu)}{\dfr\tau} 
\,\dfr\plrmu & \approx &
\plrmubar \frac{\dfr}{\dfr\tau} 
\int_{0}^{1} \ntndwndff(\tau,\plrmu) \,\dfr\plrmu
= \plrmubar \frac{\dfr \ntndwndff(\tau)}{\dfr\tau}
\label{eqn:two_srm_plr_mu_apx}
\end{eqnarray}
Extracting $\plrmubar$ from the integral on the LHS of 
(\ref{eqn:two_srm_plr_mu_apx}) is an approximation whose validity
worsens as the correlation between $\plrmu$ and
$\ntnupwdwndff(\tau,\plrmu)$ increases.

The first two terms on the RHS of (\ref{eqn:rte_plr_avg_1})
are simple hemispheric integrals of hemispherically-varying
intensities.
We will replace $\ntnupwdwn(\tau,\plrmu)$ by the hemispheric mean 
intensity $\ntnupwdwn(\tau)$ (\ref{eqn:two_str_ntp_1}).
The Planck function is isotropic and so may be pulled outside the
hemispheric integral which promptly vanishes since 
$\int_{0}^{1} \,\dfr\plrmu = 1$.
Thus the hemispherically integrated intensities which result
explicitly are replaced by the hemispheric mean intensity
(\ref{eqn:two_str_ntp_1}) associated with $\plrmubar$.

The final three terms on the RHS of (\ref{eqn:rte_plr_avg_1}) involve
products of the phase function and the intensity.
We approximate the scattering integrals by applying the hemispheric
integral over $\plrmu$ to the intensities, and then extracting the
intensities from original integrals over $\plrmuprm$.
\begin{eqnarray}
\int_{0}^{1} \int_{\upwsbs}
\phzfnc(+\plrmuprm,-\plrmu) \ntnupwdff(\tau,\plrmuprm)
\,\dfr\plrmuprm
& \approx &
\int_{0}^{1} \ntnupwdff(\tau,\plrmuprm) \,\dfr\plrmu
\int_{\upwsbs} \phzfnc(+\plrmuprm,-\plrmu) \,\dfr\plrmuprm
\nonumber \\
& = & \ntnupwdff(\tau) \int_{\upwsbs}
\phzfnc(+\plrmuprm,-\plrmu) \,\dfr\plrmuprm
\label{eqn:two_srm_phz_fnc_apx}
\end{eqnarray}
These terms define the backscattered fraction of the radiation field.  
Equation~(\ref{eqn:two_srm_phz_fnc_apx}), for example, represents the
fraction of upwelling energy backscattered into the downwelling
direction. 
This is one of four similar terms that appear in the coupled equations 
(\ref{eqn:rte_plr_avg_1}).
We define the azimuthal mean \trmdfn{backscattering function}
$\bckcffofplrmu$ as
\begin{eqnarray}
% ThS99 p. 251 (7.116)
\bckcffofplrmu & \equiv &
\frac{1}{2} \int_{0}^{1} \phzfnc( -\plrmuprm, \plrmu ) \,\dfr\plrmuprm =
\frac{1}{2} \int_{0}^{1} \phzfnc( \plrmuprm, -\plrmu ) \,\dfr\plrmuprm
\label{eqn:bck_fnc_dfn}
\end{eqnarray}
The complement of the backscattering function is the \trmdfn{forward
scattering function} $\fsfcff$.
\begin{eqnarray}
% ThS99 p. 251 (7.117)
\fsfcffofplrmu = 1 - \bckcffofplrmu & \equiv &
\frac{1}{2} \int_{0}^{1} \phzfnc( -\plrmuprm, -\plrmu ) \,\dfr\plrmuprm = 
\frac{1}{2} \int_{0}^{1} \phzfnc( \plrmuprm, \plrmu ) \,\dfr\plrmuprm
\label{eqn:fsf_fnc_dfn}
\end{eqnarray}
Pairs of the terms in (\ref{eqn:bck_fnc_dfn})--(\ref{eqn:fsf_fnc_dfn})
are identical due to \trmdfn{reciprocity relations}.
Reciprocity relations state that photon paths are reversible. 

The hemispheric mean \trmdfn{backscattering coefficient} $\bckcff$ is 
the hemispheric integral of (\ref{eqn:bck_fnc_dfn})
\begin{eqnarray}
% ThS99 p. 251 (7.116)
\bckcff & \equiv &
\int_{0}^{1} \bckcffofplrmu \,\dfr\plrmu \\
& = & \frac{1}{2} \int_{0}^{1} 
\int_{0}^{1} \phzfnc( -\plrmuprm, \plrmu ) \,\dfr\plrmuprm 
 \,\dfr\plrmu
\label{eqn:bck_cff_dfn}
\end{eqnarray}
The hemispheric mean forward scattering coefficient is defined
analogously. 

The result of these two steps is
\begin{subequations}
% ThS99 p. 171 (6.6), p. 176 (6.20)
\label{eqn:rte_plr_avg_2}
\begin{align}
\label{eqn:rte_plr_avg_2_dwn}
\lefteqn{ - \plrmubar \frac{\dfr\ntndwndff(\tau)}{\dfr\tau} = 
\ntndwndff(\tau) - (1-\ssa) \plkfnc - 
\frac{\ssa}{4\mpi} \flxslrtoa \me^{-\tau/\plrmunot}
\phzfnc(-\plrmunot,-\plrmu) }
\nonumber \\ & {} % KoD99 p. 138 for spacing info
- \frac{\ssa}{2} \int_{\upwsbs} \ntnupwdff(\tau) 
\phzfnc(+\plrmuprm,-\plrmu) \,\dfr\plrmuprm
- \frac{\ssa}{2} \int_{\dwnsbs} \ntndwndff(\tau) 
\phzfnc(-\plrmuprm,-\plrmu) \,\dfr\plrmuprm \\
\label{eqn:rte_plr_avg_2_upw}
\lefteqn{ \plrmubar \frac{\dfr\ntnupwdff(\tau)}{\dfr\tau} =  
\ntnupwdff(\tau) - (1-\ssa) \plkfnc -
\frac{\ssa}{4\mpi} \flxslrtoa \me^{-\tau/\plrmunot}
\phzfnc(-\plrmunot,+\plrmu) }
\nonumber \\ & {} % KoD99 p. 138 for spacing info
- \frac{\ssa}{2} \int_{\upwsbs} \ntnupwdff(\tau) 
\phzfnc(+\plrmuprm,+\plrmu) \,\dfr\plrmuprm
- \frac{\ssa}{2} \int_{\dwnsbs} \ntndwndff(\tau) 
\phzfnc(-\plrmuprm,+\plrmu) \,\dfr\plrmuprm
\end{align}
\end{subequations} 
These are the hemispheric mean, azimuthal mean, two stream equations
for the radiation field in slab geometry. 
It should be clear that these equations (\ref{eqn:rte_plr_avg_2}) 
are not derived simply by performing a hemispheric integral
(\ref{eqn:plr_avg_opr}) on (\ref{eqn:rte_azi_avg}). 
A strict hemispheric averaging operation would applies to the product
of the phase function and and the intensity, not to each separately,
so that extracting $\ntnupwdwndff$ from the scattering integrals is an
approximation, as is the definition of $\plrmubar$.

\subsubsection[Anisotropic Scattering]{Anisotropic Scattering}\label{sxn:two_srm_asm_sct} 
Armed with techniques introduced in solving the two stream equations 
for an isotropically scattering medium \S\ref{sxn:two_srm_eqn},  
we now solve analytically the coupled two stream equations for
an anisotropic medium.
First we recast (\ref{eqn:rte_plr_avg_3}) into a simpler notation 
\begin{subequations}
\label{eqn:two_str_asm_apx}
\begin{align}
\label{eqn:two_str_asm_apx_upw}
\plrmubar \frac{\dfr\ntnupwfrqoftau}{\dfr\tau} & =  
\ntnupwfrqoftau - \ssa (1 - \bckcff) \ntnupwfrqoftau 
- \ssa \bckcff \ntndwnfrqoftau  - (1-\ssa) \plkfrq - \srcstrupwfrq \\
\label{eqn:two_str_asm_apx_dwn}
- \plrmubar \frac{\dfr\ntndwnfrqoftau}{\dfr\tau} & =  
\ntndwnfrqoftau - \ssa (1 - \bckcff) \ntndwnfrqoftau 
- \ssa \bckcff \ntnupwfrqoftau  - (1-\ssa) \plkfrq - \srcstrdwnfrq
\end{align}
\end{subequations} 
For the remainder of the derivation we drop the $\frq$
subscript and the explicit dependence of $\ntn^{\pm}$ on $\tau$.
Adding and subtracting we obtain
\begin{subequations}
\label{eqn:two_str_asm_abb}
\begin{align}
\label{eqn:two_str_asm_pls}
\frac{\dfr(\ntnupw + \ntndwn)}{\dfr\tau} & =  
-(\alpha - \beta) (\ntnupw - \ntndwn) \\
\label{eqn:two_str_asm_mns}
\frac{\dfr(\ntnupw - \ntndwn)}{\dfr\tau} & =  
-(\alpha + \beta) (\ntnupw + \ntndwn)
\end{align}
\end{subequations} 
where we have defined 
\begin{subequations}
\label{eqn:two_str_asm_abb}
\begin{align}
\label{eqn:two_str_asm_alpha}
\alpha & \equiv -[ 1 - \ssa ( 1 - \bckcff ) ] / \plrmubar \\
\label{eqn:two_str_asm_beta}
\beta & \equiv \ssa \bckcff / \plrmubar
\end{align}
\end{subequations} 

\subsubsection[Diffusivity Approximation]{Diffusivity Approximation}\label{sxn:dff_prx} 
The angular integration required to convert the intensity field into a
hemispheric irradiance is a time consuming aspect of numerical models  
and should be avoided where possible.
According to (\ref{eqn:flx_upw_frq})--(\ref{eqn:flx_dwn_frq})
\begin{subequations}
\label{eqn:flx_hms_4}
\begin{align}
\label{eqn:flx_hms_4_a}
\frac{1}{\mpi} \flxupwfrq & = 2 \int_{0}^{1} \ntnfrq (+\plrmu) \plrmu \,
\dfr\plrmu \\
\label{eqn:flx_hms_4_b}
\frac{1}{\mpi} \flxdwnfrq & = 2 \int_{0}^{1} \ntnfrq (-\plrmu) \plrmu \,
\dfr\plrmu 
\end{align}
\end{subequations} 
The factor of $\mpi$ has been moved to the LHS so that, when the source
function is thermal, the RHS is the integral Planck function
(\ref{eqn:flx_frq_sfc_upw}). 
These angular integrals may be reduced to a calculation of the
intensity at certain quadrature points (zenith angles) in each
hemisphere with surprising accuracy.
The location of the optimal quadrature points may be arrived at
through both theoretical and empirical methods.

Two point \trmdfn{full Gaussian quadrature}
(\S\ref{sxn:gss}) tells us the optimal angles to evaluate 
(\ref{eqn:flx_hms_4}) at are $\plru = \pm 3^{-1/2}$, $\plr = \pm
54.7356\dgr$. 
\begin{subequations}
\label{eqn:flx_hms_gss}
\begin{align}
\label{eqn:flx_hms_gss_a}
\frac{1}{\mpi} \flxupwfrq & \simeq \frac{2}{\sqrt{3}} \ntnfrq (+3^{-1/2}) \\
\label{eqn:flx_hms_gss_b}
\frac{1}{\mpi} \flxdwnfrq & \simeq \frac{2}{\sqrt{3}} \ntnfrq (-3^{-1/2})
\end{align}
\end{subequations} 
The relation in Equation~(\ref{eqn:flx_hms_gss}) is \textit{exact}
when $\ntnfrqofplrmu$ is a polynomial of degree $< 2$.
For two point Gaussian quadrature the quadrature weights happen to
be unity, but not so for three point (or higher order) Gaussian
quadrature.
Here is the four point quadrature version of (\ref{eqn:flx_hms_gss})
\newline\parbox{6in}{ % KoD95 p. 138
\begin{eqnarray}
\plru_{1} = -\plru_{4} & = & 0.339981 \nonumber \\
\plru_{2} = -\plru_{3} & = & 0.861136 \nonumber \\
\AAA_{1} = \AAA_{4} & = & 0.652145 \nonumber \\
\AAA_{2} = \AAA_{3} & = & 0.347855 \nonumber \\
\frac{1}{2 \mpi} \flxupwfrq & \simeq & 
\AAA_{1} \plru_{1} \ntnfrq(\plru_{1}) + \AAA_{2} \plru_{2} \ntnfrq(\plru_{2})
\nonumber \\
\frac{1}{2 \mpi} \flxdwnfrq & \simeq & 
\AAA_{3} \plru_{3} \ntnfrq(\plru_{3}) + \AAA_{4} \plru_{4} \ntnfrq(\plru_{4})
\nonumber
\end{eqnarray} 
}\hfill % end parbox KoD95 p. 138
\parbox{1cm}{\begin{eqnarray}\label{eqn:flx_hms_gss_4}\end{eqnarray}}\newline

What is often used to evaluate (\ref{eqn:flx_hms_4}) instead of
Gaussian quadrature is a \trmdfn{diffusivity approximation}.  
Instead of choosing optimal quadrature angles $\plrmu_{\kkk}$, the
diffusivity approximation relies on choosing the optimal effective
absorber path.
Thus the \trmdfn{diffusivity factor}, $\dff$, is defined by
\begin{subequations}
\label{eqn:dff_dfn}
\begin{align}
\label{eqn:dff_dfn_a}
\frac{1}{\mpi} \flxupwfrq & \simeq \ntnfrq (\plru = +\dff^{-1}) \\
\label{eqn:dff_dfn_b}
\frac{1}{\mpi} \flxdwnfrq & \simeq \ntnfrq (\plru = -\dff^{-1})
\end{align}
\end{subequations} 
Comparing (\ref{eqn:dff_dfn}) to (\ref{eqn:flx_hms_gss}) and
(\ref{eqn:flx_hms_4}) shows that $\dff$ replaces, simultaneously, the
quadrature angle, its weight, the factor of $\plrmu$ in
(\ref{eqn:flx_hms_4}), and a factor of 2 from the azimuthal
integration. 
The angle $\plrdff = \arccos \dff^{-1}$ may be interpreted as the
mean slant path of radiation in an isotropic, non-scattering
atmosphere. 

The reasons for subsuming so many factors into $\dff$ are historical.  
Nevertheless, (\ref{eqn:dff_dfn}) is much more common than
(\ref{eqn:flx_hms_gss}) in the literature.
Heating rate calculations in isotropic, non-scattering atmospheres
show that using $\dff = 1.66$ results in errors $< 2\%$
\cite[][p. 221]{GoY89}.  

In applying (\ref{eqn:dff_dfn}), $\dff$ only affects the optical depth
factor of the intensity.
% fxm: I'm not sure I understand this: In computing intensity,
% doesn't D appear both in the optical depth factor AND the mass path
% factor? 
Thus computing the intensity in (\ref{eqn:dff_dfn}) at the angle
$\plr = \plrdff$ is formally equivalent to computing the intensity in
the vertical direction $\plr = 0$ but through an atmosphere in which
the absorber densities have been increased by a factor of $\dff$.

\subsubsection[Transmittance]{Transmittance}\label{sxn:trn}
The \trmdfn{transmittance} between two points measures the transparency
of the atmosphere between those points.
The transmittance $\trntld$ from point $\pnt_{1}$ to  
point $\pnt_{2}$ is the likelihood a photon traveling in direction
$\nglhat$ at $\pnt_{1}$ will arrive at $\pnt_{2}$ without having
interacted with the matter in between. 
\begin{equation}
\trntld(\pnt_{1},\pnt_{2}) = \exp[-\tautld(\pnt_{1},\pnt_{2})]
\label{eqn:trn_dfn}
\end{equation}
We have left implicit the dependence on frequency.
In a stratified atmosphere, we are interested in the transmission
of light traveling from layer $\zzzprm$ to layer $\zzz$ at angle
$\plr$ from the vertical, thus
\begin{equation}
\trnbm(\zzzprm,\zzz;\plrmu) = \exp[-\tau(\zzzprm,\zzz)/\plrmu]
\label{eqn:trnbm_ppa}
\end{equation}
Thus, in common with absorptance $\absbm$ and reflectance $\rflbm$,
the transmittance $\trnbm \in [0,1]$\footnote{This text distinguishes
between \trmidx{transmittance} and \trmidx{transmission}. 
The former measures the potential for the latter to occur.
Thus a transmittance of unity means the transmission of energy is very
high.
A similar rule holds for \trmidx{reflectance} and \trmidx{reflection}
and for \trmidx{absorptance} and \trmidx{absorption}.}

The vertical gradient of $\trnbm$ is frequently used to formulate
solutions of the radiative transfer equation in non-scattering,
thermal atmospheres.
\begin{eqnarray}
\frac{\partial}{\partial \zzzprm} \trnbm(\zzzprm,\zzz;\plrmu) & = &
\frac{\partial}{\partial \zzzprm} \exp[-\tau(\zzzprm,\zzz;\plrmu)/\plrmu]
\nonumber \\
\frac{\partial \trnbm(\zzzprm,\zzz;\plrmu)}{\partial \zzzprm} & = &
-\frac{1}{\plrmu} \exp[-\tau(\zzzprm,\zzz;\plrmu)/\plrmu] 
\frac{\partial \tau(\zzzprm,\zzz;\plrmu)}{\partial \zzzprm}
\nonumber \\
& = & -\frac{1}{\plrmu} \trnbm(\zzzprm,\zzz;\plrmu) 
\frac{\partial \tau}{\partial \zzzprm}
\label{eqn:trn_grd}
\end{eqnarray}
% Equation~{\ref{eqn:trn_grd}} says that 
Inverting the above, we obtain
\begin{equation}
\plrmurcp \trnbm(\zzzprm,\zzz;\plrmu) =  
- \frac{\partial \trnbm(\zzzprm,\zzz;\plrmu)}{\partial \zzzprm}
\frac{\partial \zzzprm}{\partial \tau}
\label{eqn:trn_grd_2}
\end{equation}

The integral solutions for the upwelling and downwelling intensities
in a non-scattering, thermal, stratified atmosphere
(\ref{eqn:plk_sln_upw})--(\ref{eqn:plk_sln_dwn}) may be rewritten in
terms of $\trnbm$. 
(\ref{eqn:plk_sln_dwn}) from $\tau$ to $\zzz$.
To do this we use
\begin{eqnarray}
\zzz(\tau = \taustr) & = & 0 \nonumber \\
\zzz(\tau = 0) & = & \infty \nonumber \\
\trnbm(0,\zzz;\plrmu) & = & \me^{-(\taustr - \tau)/\plrmu} \nonumber \\
\trnbm(\zzzprm,\zzz;\plrmu) & = & \me^{-(\tauprm - \tau)/\plrmu} \nonumber \\
\trnbm(\zzz,\zzzprm;\plrmu) & = & \me^{-(\tau - \tauprm)/\plrmu} \nonumber \\
\frac{\partial \trnbm(\zzzprm,\zzz;\plrmu)}{\partial \zzzprm} & = &
-\frac{\partial \trnbm(\zzz,\zzzprm;\plrmu)}{\partial \zzzprm} 
\end{eqnarray}
The last relation simply states that the transmission decreases as the
distance between $\zzzprm,\zzz$ increases and visa versa.
Substituting the above into (\ref{eqn:plk_sln_upw}) and
(\ref{eqn:plk_sln_dwn}) yields
\begin{eqnarray}
\ntnfrqofzzzpmu & = & 
\trnbm(0,\zzz;\plrmu) \plkfrq(0) + 
\plrmurcp \int_{\zzz}^{0} 
\trnbm(\zzzprm,\zzz;\plrmu) \plkfrq(\zzzprm) \, (-\dfr\zzzprm) 
\nonumber \\
\ntnfrqofzzzmmu & = & 
\plrmurcp \int_{\infty}^{\zzz} 
\trnbm(\zzz,\zzzprm;\plrmu) \plkfrq(\zzzprm) \, (-\dfr\zzzprm) \nonumber  
\end{eqnarray}
Rearranging terms we obtain
\begin{subequations}
\label{eqn:zzz_sln}
\begin{align}
\label{eqn:zzz_sln_upw}
\ntnfrqofzzzpmu & = 
\trnbm(0,\zzz;\plrmu) \plkfrq(0) + 
\plrmurcp \int_{0}^{\zzz} 
\trnbm(\zzzprm,\zzz;\plrmu) \plkfrq(\zzzprm) \,\dfr\zzzprm \\
\label{eqn:zzz_sln_dwn}
\ntnfrqofzzzmmu & = 
\plrmurcp \int_{\zzz}^{\infty} 
\trnbm(\zzz,\zzzprm;\plrmu) \plkfrq(\zzzprm) \,\dfr\zzzprm
\end{align}
\end{subequations} 
Substituting (\ref{eqn:trn_grd_2}) in the above leads to
\begin{subequations}
\label{eqn:trn_sln}
\begin{align}
\label{eqn:trn_sln_upw}
\ntnfrqofzzzpmu & =  
\trnbm(0,\zzz;\plrmu) \plkfrq(0) + 
\int_{0}^{\zzz} 
\frac{\partial \trnbm(\zzzprm,\zzz;\plrmu)}{\partial \zzzprm}
\plkfrq(\zzzprm) \,\dfr\zzzprm \\
\label{eqn:trn_sln_dwn}
\ntnfrqofzzzmmu & = 
- \int_{\zzz}^{\infty} 
\frac{\partial \trnbm(\zzz,\zzzprm;\plrmu)}{\partial \zzzprm}
\plkfrq(\zzzprm) \,\dfr\zzzprm
\end{align}
\end{subequations} 
If $\trnbm$ is known, then Equations
(\ref{eqn:trn_sln_upw})--(\ref{eqn:trn_sln_dwn}) are suitable for use
in quadrature formulae to obtain irradiances.

\subsection[Reflection, Transmission, Absorption]{Reflection, Transmission, Absorption}\label{sxn:rta}
Perhaps the most useful metrics of the radiative properties 
of entire systems are the quantities reflection, transmission, and
absorption. 
These metrics describe normalized properties of a system and thus
are somewhat more fundamental than absolute quantities (like
transmitted irradiance), which may, for example, change depending on
time of day. 
We shall define the reflection, transmission, and absorption first
of the radiance field, and then integrate these definitions with
the appropriate weighting to obtain the $\rfl$, $\trn$, and $\abs$
pertinent to fluxes.
We shall introduce a painful but necessary menagerie of terminology
to describe the various species of $\rfl$, $\trn$, and $\abs$.

\cite{MPB04} show how remotely sensed \trmidx{surface reflectance}
differs significantly from that derived from snow-age models.
Using the more realistic albedos improved models of snowmelt by 
providing more accurate net surface shortwave radiation estimates. 
This term, $\flxnetswsfc$, dominates the snow-melt energy budget.

\cite{WBZ05} derived a simple functional fit to the desert surface
albedo observed by \trmidx{MODIS}.

\subsubsection[BRDF]{BRDF}\label{sxn:BRDF}
The \trmdfn{bidirectional reflectance distribution function} (BRDF)
$\brdf$ is the ratio of the reflected intensity to the energy in the
incident beam. 
As such, $\brdf$ is a function of frequency, incident angle, and
scattered angle
\begin{eqnarray}
\brdfoffrqmnglhatprmnglhat & \equiv & 
\frac{\dfr\ntnupwfrqrfl(\nglhat)}{\ntndwnfrq(\nglhatprm) \cos \plrprm \,\dfr\nglprm}
\label{eqn:brdf_dfn}
\end{eqnarray}

The dimensions of $\brdf$ are \xsr\ and they convert irradiance to
intensity. 
The reflected intensity in any particular direction $\nglhat$ is the
sum of contributions from all incident directions $\nglhatprm$ that
have a finite probability of reflecting into $\nglhat$.
\begin{eqnarray}
\ntnupwfrqrfl(\nglhat) 
& = & 
\int_{-\nglhatprm} \,\dfr\ntnupwfrqrfl(\nglhat)
\nonumber \\
& = & 
\int_{-\nglhatprm} 
\ntndwnfrq(\nglhatprm) \brdfoffrqmnglhatprmnglhat \cos \plrprm
\,\dfr\nglprm
\label{eqn:ntn_upw_frq_rfl_dfn}
\end{eqnarray}
The dependence of $\brdfoffrqmnglhatprmnglhat$ on two directions and
on frequency makes it a difficult property to measure.

\subsubsection[Lambertian Surfaces]{Lambertian Surfaces}\label{sxn:lmb}
Fortunately many surfaces found in nature obey simpler reflectance
properties first characterized by Lambert.
% fxm: Lambert or Lambertian? ThS99 use Lambertian
A \trmdfn{Lambertian surface} is one whose reflectance is independent
of both incident and reflected directions.
The reflectance of a Lambertian surface depends only on frequency
\begin{eqnarray}
\brdfoffrqmnglhatprmnglhat & = & \rfllmboffrq
\label{eqn:rfl_lmb_dfn}
\end{eqnarray}
The intensity reflected from a Lambertian surface is given by 
inserting (\ref{eqn:rfl_lmb_dfn}) into (\ref{eqn:brdf_dfn}) and then
using (\ref{eqn:flx_dwn_frq})
\begin{eqnarray}
\ntnupwfrqrfl(\nglhat) 
& = & 
\int_{-\nglhatprm} 
\ntndwnfrq(\nglhatprm) \rfllmboffrq \cos \plrprm
\,\dfr\nglprm \nonumber \\
& = & 
\rfllmboffrq 
\int_{-\nglhatprm} \ntndwnfrq(\nglhatprm) \cos \plrprm
\,\dfr\nglprm \nonumber \\
& = & 
\rfllmboffrq \flxdwnfrq
\label{eqn:ntn_rfl_lmb_dfn}
\end{eqnarray}
As expected, the intensity reflected from a Lambertian surface depends 
only on the incident irradiance, and not at all on the details of the 
angular distribution of the incident intensity field.

The irradiance reflected from a Lambertian surface $\flxupwfrqrfl$ is
the cosine-weighted integral of the reflected intensity
(\ref{eqn:ntn_rfl_lmb_dfn}) over all reflected angles 
\begin{eqnarray}
\flxupwfrqrfl
& = & 
\int_{\nglhat}
\ntnupwfrq(\nglhat) \cos \plr \,\dfr\ngl \nonumber \\
& = & 
\int_{\nglhat}
\rfllmboffrq \flxdwnfrq \cos \plr \,\dfr\ngl \nonumber
\end{eqnarray}
Neither $\flxdwnfrq$ and $\rfllmboffrq$ depend on the emergent angle
$\nglhat$ so the integral reduces to the familiar integral of 
$\cos \plr$ over the hemisphere (\ref{eqn:plk_hms_dfn}) which is $\mpi$.
\begin{eqnarray}
\flxupwfrqrfl
& = & 
\mpi \rfllmboffrq \flxdwnfrq 
\label{eqn:flx_rfl_lmb_dfn}
\end{eqnarray}

The reflected irradiance may not exceed the incident irradiance
or the requirement of energy conservation will be violated.
Therefore (\ref{eqn:flx_rfl_lmb_dfn}) shows that 
\begin{eqnarray}
\rfllmboffrq & \le &\mpi^{-1}
\label{eqn:rfl_lmb_lmt}
\end{eqnarray}
with equality holding only for a perfectly reflective (non-absorbing)
Lambertian surface. 

Many researchers prefer the Lambertian BRDF to have an upper limit of
1, not $\mpi^{-1}$ (\ref{eqn:rfl_lmb_lmt}).
Thus it is common to encounter in the literature BRDF defined as
$\rrr = \mpi \brdf$.

\subsubsection[Albedo]{Albedo}\label{sxn:alb}
The reflectance of a surface illuminated by a collimated source
such as the Sun or a laser is of great interest in planetary studies
and in remote sensing.
For an incident collimated beam of intensity $\flxslrtoa$, the
diffusely reflected intensity is 
\begin{eqnarray}
\ntnupwfrqrfl(\nglhat) 
& = & 
\int_{-\nglhatprm} 
\flxslrtoa \dltfncofnglhatprmmnglhatnot 
\brdfoffrqmnglhatprmnglhat \cos \plrprm
\,\dfr\nglprm \nonumber \\
& = & 
\flxslrtoa \brdfoffrqmnglhatnotnglhat \cos \plrnot
\label{eqn:ntn_rfl_cll_dfn}
\end{eqnarray}
Integrating (\ref{eqn:ntn_rfl_cll_dfn}) over all reflected angles we 
obtain the diffusely reflected hemispheric irradiance
\begin{eqnarray}
\flxupwfrqrfl
& = & 
\int_{+\nglhat} 
\flxslrtoa \brdfoffrqmnglhatnotnglhat \cos \plrnot \cos \plr
\,\dfr\ngl \nonumber \\
& = & 
\flxslrtoa \cos \plrnot
\int_{+\nglhat} 
\brdfoffrqmnglhatnotnglhat \cos \plr \,\dfr\ngl
\label{eqn:flx_rfl_cll_dfn}
\end{eqnarray}
The \trmdfn{flux reflectance} or \trmdfn{plane albedo}
is the ratio of the reflected (diffuse) irradiance
(\ref{eqn:flx_rfl_cll_dfn}) to the incident (collimated) irradiance  
(\ref{eqn:ntn_dwn_drc_dfn}) 
\begin{eqnarray}
\brdfoffrqmnglhatnottwopi
& = & \frac{\flxupwfrqrfl}{\flxdwnfrq} \nonumber \\
& = & 
\int_{+\nglhat} \brdfoffrqmnglhatnotnglhat \cos \plr \,\dfr\ngl
\label{eqn:flx_rfl_dfn}
\end{eqnarray}
The $2\mpi$ indicates that the plane albedo pertains to the entire
reflected hemisphere.

When the reflecting body is of finite size then the corresponding
ratio of reflected to incident fluxes is more complicated because it
contains ``edge effects'', e.g., diminishing contributions from
planetary limbs.
First we consider the reflection, called the \trmdfn{spherical
albedo}, \trmdfn{planetary albedo} or \trmdfn{Bond albedo}, of an
entire planetary disk illuminated by collimated sunlight.

The \trmdfn{directional reflectance} $\brdfoffrqmtwopinglhat$ is 
the 
\begin{eqnarray}
\flxupwfrqrfl(\nglhat)
& = & 
\int_{-\nglhatprm} 
\flxdwnfrq(-\nglhatprm) \brdfoffrqmtwopinglhat \cos \plrprm \,
\dfr\nglprm \nonumber \\ 
\label{eqn:flx_rfl_drc_dfn}
\end{eqnarray}

The flux reflectance of a \trmidx{Lambertian surface} to collimated
light is found by subsituting (\ref{eqn:rfl_lmb_dfn}) in
(\ref{eqn:flx_rfl_dfn}) 
\begin{eqnarray}
\brdfoffrqmnglhatnottwopi
& = & 
\int_{+\nglhat} \rfllmboffrq \cos \plr \,\dfr\ngl \nonumber \\
& = & 
\mpi \rfllmboffrq 
\label{eqn:flx_rfl_lmb_cll_dfn}
\end{eqnarray}
which agrees with (\ref{eqn:flx_rfl_lmb_dfn}).

The BRDF must be integrated over all possible angles of reflectance
in order to obtain the \trmdfn{flux reflectance},
$\brdfoffrqmnglhatprmtwopi$. 
Representative flux reflectances of various surfaces in the Earth
system are presented in Table~\ref{tbl:alb_sfc}.
\begin{table}
\begin{minipage}{\hsize} % Minipage necessary for footnotes KoD95 p. 110 (4.10.4)
\renewcommand{\footnoterule}{\rule{\hsize}{0.0cm}\vspace{-0.0cm}} % KoD95 p. 111
\begin{center}
\caption[Surface Albedo]{\textbf{Surface Albedo}%
\footnote{\emph{Sources: Bird Entrails (2000)}}%
\label{tbl:alb_sfc}}
\vspace{\cpthdrhlnskp}
\begin{tabular}[c]{ r l l }
\hline \rule{0.0ex}{\hlntblhdrskp}% 
Surface Type & Reflectance \\[0.0ex]
\hline \rule{0.0ex}{\hlntblntrskp}%
Glacier & 0.9 \\[1.0ex]
Snow & 0.8 \\[1.0ex]
Sea-ice & 0.6 \\[1.0ex]
Clouds & 0.2--0.7 \\[1.0ex]
Desert & 0.2--0.3 \\[1.0ex]
Savannah & 0.2--0.25 \\[1.0ex]
Forest & 0.05--0.1 \\[1.0ex]
Ocean & 0.05--0.1 \\[1.0ex]
\hline\hline
Planetary & 0.3 \\[1.0ex]
\end{tabular}
\end{center}
\end{minipage}
\end{table}

\subsubsection[Flux Transmission]{Flux Transmission}\label{sxn:flx_trn}
The \trmdfn{flux transmission} $\trnflxfrq$ between two layers
is (twice) the cosine-weighted integral of the spectral transmission  
\begin{eqnarray}
\trnflxfrq(\zzzprm,\zzz) & = & 2 \int_{0}^{1}
\trnbm(\zzzprm,\zzz;\plrmu) \plrmu \,\dfr\plrmu \nonumber \\ 
\trnflxfrq(\zzzprm,\zzz) & = & 2 \int_{0}^{1}
\exp[-\tau(\zzzprm,\zzz)/\plrmu] \plrmu \,\dfr\plrmu
\label{eqn:trn_flx_dfn}
\end{eqnarray}
We can rewrite (\ref{eqn:trn_flx_dfn}) in terms of exponential
integrals (\ref{eqn:xpn_3_dfn})
\begin{equation}
\trnflxfrq(\zzzprm,\zzz) = 2 \xpn_{3}[\tau(\zzzprm,\zzz)]
\label{eqn:trn_flx_xpn}
\end{equation}
The factor of 2 ensures that expressions for the hemispheric fluxes
in non-scattering, thermal atmospheres closely resemble the equations
for intensities, as will be shown next.

The vertical gradient of the flux transmission is frequently used to
formulate solutions of the radiative transfer equation in
non-scattering, thermal atmospheres.
It is obtained by differentiating (\ref{eqn:trn_flx_xpn}) then
applying (\ref{eqn:xpn_rcr_1}) 
\begin{eqnarray}
\frac{\partial}{\partial \zzzprm} \trnflxfrq(\zzzprm,\zzz) & = & 
2 \frac{\partial}{\partial \zzzprm} \xpn_{3}[\tau(\zzzprm,\zzz)] 
\frac{\partial \trnflxfrq(\zzzprm,\zzz)}{\partial \zzzprm} 
\nonumber \\
\frac{\partial \trnflxfrq(\zzzprm,\zzz)}{\partial \zzzprm}
& = & 2 [-\xpn_{2}(\zzzprm,\zzz)] 
\frac{\partial \tau(\zzzprm,\zzz;\plrmu)}{\partial \zzzprm}
\nonumber \\
& = & - 2 \xpn_{2}(\zzzprm,\zzz) 
\frac{\partial \tau}{\partial \zzzprm}
\label{eqn:trn_flx_drv}
\end{eqnarray}

The integral solutions for the upwelling and downwelling fluxes
in a non-scattering, thermal, stratified atmosphere
(\ref{eqn:flx_dwn_xpn}) and (\ref{eqn:flx_upw_xpn}) may be rewritten in
terms of $\trnflxfrq$.  
The procedure for doing so is analogous to the procedure applied to
intensities in \S\ref{sxn:trn}. 
First, we change the independent variable in (\ref{eqn:flx_dwn_xpn})
and (\ref{eqn:flx_upw_xpn}) from $\tau$ to $\zzz$.
\begin{eqnarray}
\label{eqn:flx_upw_xpn_zzz}
\flxupwfrqofzzz & = & 2 \mpi \plkfrq(0) \xpn_{3}(\zzz) +  
2 \mpi \int_{\zzzprm = 0}^{\zzzprm = \zzz}
\plkfrq(\zzzprm) \xpn_{2}(\zzz-\zzzprm) \,\dfr\zzzprm \nonumber \\
\label{eqn:flx_dwn_xpn_zzz}
\flxdwnfrqofzzz & = & 2 \mpi \int_{\zzzprm = \zzz}^{\zzzprm = \infty} 
\plkfrq(\zzzprm) \xpn_{2}(\zzzprm - \zzz) \,\dfr\zzzprm \nonumber
\end{eqnarray}
Substituting (\ref{eqn:trn_flx_drv}) into the above yields
\begin{subequations}
\label{eqn:flx_trn_zzz}
\begin{align}
\label{eqn:flx_upw_trn_zzz}
\frac{1}{\mpi} \flxupwfrqofzzz & = 
\plkfrq(0) \trnflxfrq(0,\zzz) +  
\int_{\zzzprm = 0}^{\zzzprm = \zzz}
\plkfrq(\zzzprm) 
\frac{\partial \trnflxfrq(\zzzprm,\zzz)}{\partial \zzzprm} 
\,\dfr\zzzprm \\
\label{eqn:flx_dwn_trn_zzz}
\frac{1}{\mpi} \flxdwnfrqofzzz & = 
- \int_{\zzzprm = \zzz}^{\zzzprm = \infty} 
\plkfrq(\zzzprm) 
\frac{\partial \trnflxfrq(\zzzprm,\zzz)}{\partial \zzzprm} 
\,\dfr\zzzprm
\end{align}
\end{subequations} 
The analogy between
(\ref{eqn:flx_upw_trn_zzz})--(\ref{eqn:flx_dwn_trn_zzz}) and 
(\ref{eqn:trn_sln_upw})--(\ref{eqn:trn_sln_dwn}) is complete. 
$\trnflxfrq$ plays the same role in former that $\trnbm$ plays in the
latter. 

Equations (\ref{eqn:flx_upw_trn_zzz})--(\ref{eqn:flx_dwn_trn_zzz})
are particularly useful because they contain no explicit reference to
the angular integration.
The polar integration, however, is still implicit in the definition of 
$\trnflxfrq$ (\ref{eqn:trn_flx_dfn}).
The hemispherical irradiances may be determined without any angular
integration if the diffusivity approximation is applied to 
(\ref{eqn:flx_upw_trn_zzz})--(\ref{eqn:flx_dwn_trn_zzz}).
This is accomplished by replacing $\trnflxfrq$ by the vertical
spectral transmission $\trnbm$ through a factor $\dff$
(\ref{eqn:dff_dfn}) times as much mass  
\begin{equation}
\trnflxfrq(\zzzprm,\zzz) \simeq \trnbm(\zzzprm,\zzz;\arccos \dff^{-1}) 
\label{eqn:dff_prx}
\end{equation}
Determination of $\trnbm$ requires the use of a spectral gaseous
extinction database in conjunction with either a \trmidx{line-by-line}
model, a narrow band model, or a broadband emissivity approach.

\subsection[Two-Stream Approximation]{Two-Stream Approximation}\label{sxn:two_srm}
The \trmdfn{two-stream approximation} to the radiative transfer
equation assumes that up- and down-welling radiances travel at mean
inclinations $\plrmubarupw$ and $\plrmubardwn$, respectively.
With this assumption, the hemispheric intensities $\ntnfrq^{\pm}$ in
the azimuthally averaged radiative transfer equation
(\ref{eqn:rte_azi_avg}) lose their explicit dependence on $\plrmu$
and depend solely on optical depth.
For an isotropically scattering atmosphere in a slab geometry,
the two-stream approximation may be written 
\begin{subequations}
\label{eqn:two_str}
\begin{align}
\label{eqn:two_str_upw}
% ThS99 p. 228 (7.23--7.24)
\plrmubarupw \frac{\dfr\ntnupwfrqoftau}{\dfr\tau} & =  
\ntnupwfrqoftau - \frac{\ssa}{2} \ntnupwfrqoftau - \frac{\ssa}{2}
\ntndwnfrqoftau - (1 - \ssa) \plkfrq[\tpt(\tau)] \\
\label{eqn:two_str_dwn}
-\plrmubardwn \frac{\dfr\ntndwnfrqoftau}{\dfr\tau} & =  
\ntndwnfrqoftau - \frac{\ssa}{2} \ntnupwfrqoftau - \frac{\ssa}{2}
\ntndwnfrqoftau - (1 - \ssa) \plkfrq[\tpt(\tau)]
\end{align}
\end{subequations} 
These equations are no longer exact, but only approximations.
Nevertheless, and as we shall show, the two-stream approximation
yields remarkably accurate solutions for a wide variety of scenarios
including the most important ones in planetary atmospheres.
We emphasize that the solutions to the two-stream equation,
$\ntnupwdwnfrq$, are hemispheric mean intensities each associated with
a hemispheric mean direction $\plrmubarupwdwn$.
In this interpretation of $\ntnupwdwnfrqoftau$,
\begin{eqnarray}
\ntnupwdwnfrqoftau = 
\frac{\int_{0}^{1} \ntnupwdwnfrq(\tau,\plrmu)\,\dfr\plrmu}
{\int_{0}^{1}  \,\dfr\plrmu}
= \int_{0}^{1} \ntnupwdwnfrq(\tau,\plrmu)\,\dfr\plrmu
\label{eqn:two_str_ntp_1}
\end{eqnarray}
As emphasized by (\ref{eqn:two_str_ntp_1}), $\ntnupwdwnfrq$ are
\textit{not} isotropic intensities in each hemisphere. 

We now re-define the other important properties of the radiation field
to be consistent with the two-stream approximation.
Perhaps most important is the hemispheric irradiance.
Starting from (\ref{eqn:flx_hms}) we obtain
\begin{eqnarray}
% ThS99 p. 226 (7.26)
\flxupwdwnfrq(\tau) & = & 
2 \mpi \int_{0}^{1} \ntnupwdwnfrq (\tau,\plrmu) \plrmu \,\dfr\plrmu 
\nonumber \\
& \approx & 2 \mpi \plrmubar \ntnupwdwnfrq (\tau)
\label{eqn:flx_hms_two_str}
\end{eqnarray}
where $\approx$ indicates that (\ref{eqn:flx_hms_two_str}) is an
approximation. 
Equation~(\ref{eqn:flx_hms_two_str}) reveals one interesting
physical interpretation of the two-stream approximation:
the radiance field comprises only two discrete streams
travelling at angles $\plrmubarupwdwn$. 
Mathematically, this radiance field could be represented by the
union of the two delta-functions representing collimated beams
travelling in the $\plrmubarupwdwn$ directions.
This interpretation (presumably) gives the ``two stream'' method its
name. 
Physically, it is more appropriate to interpret the two-stream
radiance field as being continuously distributed in each hemisphere
(\ref{eqn:two_str_ntp_1}), such that the angular moments (e.g., in the
definition of irradiance) have the properties in
(\ref{eqn:flx_hms_two_str}).

A common mis-understanding of the two-stream approximation is that
the radiance field is isotropic in each hemisphere.
If two-stream radiance were isotropic in each hemisphere then the
appropriate measure of hemispheric irradiance would be
$\flxupwdwnfrq = \mpi \ntnupwdwnfrq$ (\ref{eqn:plk_hms_dfn}) as opposed to 
$\flxupwdwnfrq = 2 \mpi \plrmubarupwdwn \ntnupwdwnfrq$
(\ref{eqn:flx_hms_two_str}). 
In fact, assuming the intensity field varies linearly with $\plrmu$ 
(and is thus anisotropic) is a reasonable interpretation of the
two-stream approximation.

The mean intensity $\ntnmnfrq$ (\ref{eqn:ntn_bar_dfn}) for an
azimuthally independent, isotropically scattering radiation field is 
\begin{eqnarray}
% ThS99 p. 226 (7.25)
\ntnmnfrqoftau
& = & \frac{1}{2} \int_{-1}^{1} \ntnfrq(\tau,\plru) \,\dfr\plru
\nonumber \\
& = & \frac{1}{2} \int_{0}^{1} \ntnupwfrq(\tau,\plrmu) +
\ntndwnfrq(\tau,\plrmu) \,\dfr\plrmu \nonumber \\
& \approx & \frac{1}{2} [\ntnupwfrqoftau + \ntndwnfrqoftau] 
\label{eqn:ntnmn_two_str}
\end{eqnarray}
Although $\ntnmnfrqoftau$ (\ref{eqn:ntnmn_two_str}) appears
independent of $\plrmubar$, the two-stream intensities in
(\ref{eqn:ntnmn_two_str}) do depend on $\plrmubar$.

Continuing to write the two-stream approximations for radiative
quantities of interest, we turn now to the source function.
The source function $\srcfrq$ (\ref{eqn:src_frq_dfn}) for an azimuthally
independent, isotropically scattering ($\phzfnc = 1$) radiation field is  
\begin{eqnarray}
% ThS99 p. 226 (7.25)
\srcfrqoftau
& = & 
(1-\ssa)\plkfrq[\tpt(\tau)] + \frac{\ssa}{2} 
\int_{-1}^{1} \ntnfrq(\tau,\plru) \,\dfr\plru \nonumber \\
& \approx & 
(1-\ssa)\plkfrq[\tpt(\tau)] + \frac{\ssa}{2}
[\ntnupwfrqoftau + \ntndwnfrqoftau]
\label{eqn:src_two_str}
\end{eqnarray}
The final term on the RHS is the two-stream mean intensity
$\ntnmnfrq$~(\ref{eqn:ntnmn_two_str}). 
$\ntnmnfrq$~appears in $\srcfrq$~(\ref{eqn:src_two_str}) due to the 
assumption of isotropic scattering. 
For more general phase functions, $\srcfrq$~will not
contain~$\ntnmnfrq$.  

A comment on vertical homogeneity is appropriate here.
The Planck function in (\ref{eqn:src_two_str}) varies
continuously with $\tau$, whereas $\ssa$ is assumed to be constant in
a given layer over which (\ref{eqn:src_two_str}) is discretized.
Thus $\tau$-dependence appears explicitly for $\plkfrq$, and is absent
for $\ssa$.
More generally, the radiation field (intensity, irradiance, source
function, etc.) varies continuously with $\tau$.
The discretization of $\tau$ (or, in general, all spatial coordinates) 
is required in order to solve for the continuously varying radiation
field. 
The discretization determines the scale over which the properties of
the matter in the medium may change.

The radiative heating rate $\htrfrq$ in the two-stream approximation
may be obtained in at least two distinct forms.
% fxm: Define heating rate earlier on
\begin{eqnarray}
% ThS99 p. 226 (7.27)
\htrfrqoftau
& = & 
-\frac{\partial \flxfrq}{\partial \zzz} \nonumber \\
& = & 
4\mpi \ntnmnfrqoftau \xsxabs - 4\mpi \xsxabs \plkfrq[\tpt(\tau)] \nonumber \\
& \approx & 
2 \mpi \xsxabs [ \ntnupwfrqoftau + \ntndwnfrqoftau ] 
- 4\mpi \xsxabs \plkfrq[\tpt(\tau)]
\label{eqn:htr_two_str}
\end{eqnarray}
The first term on the RHS represents the rate of absorption of
radiative energy, while the second term is the rate of radiative
emission. 
Note that (\ref{eqn:htr_two_str}) depends on the absorption cross
section $\alpha$ rather that the single scattering albedo $\ssa$.
This emphasizes that scattering has no direct effect on heating.

Together the relations for $\flxfrq$, $\ntnmnfrq$, $\srcfrq$, and
$\htrfrq$ in (\ref{eqn:flx_hms_two_str})--(\ref{eqn:htr_two_str})
define a complete and self-consistent set of radiative properties 
under the two-stream approximation.
We turn now to obtaining the two-stream intensities $\ntnupwdwnfrq(\tau)$
from which these quantities may be computed.

\subsubsection[Two-Stream Equations]{Two-Stream Equations}\label{sxn:two_srm_eqn}
We shall simplify both the physics and nomenclature of
(\ref{eqn:two_str}) before attempting an analytical solution.
First, for algebraic simplicity we set
$\plrmubarupw = \plrmubardwn = \plrmubar$. 
Also, we neglect the thermal source term $\plkfrq[\tpt(\tau)]$.
With these assumptions, (\ref{eqn:two_str}) becomes
\begin{subequations}
\label{eqn:two_str_apx}
\begin{align}
\label{eqn:two_str_apx_upw}
% ThS99 p. 228 (7.23--7.24)
\plrmubar \frac{\dfr\ntnupwfrqoftau}{\dfr\tau} & =  
\ntnupwfrqoftau - \frac{\ssa}{2} \ntnupwfrqoftau 
- \frac{\ssa}{2} \ntndwnfrqoftau \\
\label{eqn:two_str_apx_dwn}
-\plrmubar \frac{\dfr\ntndwnfrqoftau}{\dfr\tau} & =  
\ntndwnfrqoftau - \frac{\ssa}{2} \ntnupwfrqoftau 
- \frac{\ssa}{2} \ntndwnfrqoftau
\end{align}
\end{subequations} 

For the remainder of the derivation we drop the $\frq$
subscript and the explicit dependence of $\ntn^{\pm}$ on $\tau$.
Adding and subtracting the equations in (\ref{eqn:two_str_apx}) we
obtain
\begin{subequations}
\label{eqn:two_str_apx2}
\begin{align}
% ThS99 p. 228 (7.28--7.29)
\plrmubar \frac{\dfr ( \ntnupw - \ntndwn ) }{\dfr\tau} & =  
\ntnupw + \ntndwn - \ssa ( \ntnupw + \ntndwn )
= (1 - \ssa) ( \ntnupw + \ntndwn ) \\
\plrmubar \frac{\dfr ( \ntnupw + \ntndwn ) }{\dfr\tau} & =  
\ntnupw - \ntndwn
\end{align}
\end{subequations} 
These may be rewritten as first order, coupled equations in $\YYYpm$
where 
\begin{subequations}
\begin{align}
\label{eqn:YYY_dfn}
\YYYpm & \equiv \ntnupw \pm \ntndwn \\
\label{eqn:ntn_upwdwn_dfn}
\ntnupwdwn & = {\textstyle\frac{1}{2}} ( \YYYpls \pm \YYYmns )
\end{align}
\end{subequations} 
Using (\ref{eqn:YYY_dfn}) in (\ref{eqn:two_str_apx2})
\begin{subequations}
\label{eqn:two_str_apx3}
\begin{align}
% ThS99 p. 228 (7.28--7.29)
\frac{\dfr\YYYmns}{\dfr\tau} & =  
\frac{(1 - \ssa)}{\plrmubar} \YYYpls \\
\frac{\dfr\YYYpls}{\dfr\tau} & =  
\frac{1}{\plrmubar} \YYYmns
\end{align}
\end{subequations} 
We differentiate (\ref{eqn:two_str_apx3}) with respect to $\tau$
to uncouple the equations, 
\begin{alignat*}{3} % KoD99 p. 424 describes alignment and &&
% ThS99 p. 228 (7.30)
\frac{\dfrsqr\YYYmns}{\dfr\tausqr} &=  
\frac{(1 - \ssa)}{\plrmubar} \frac{\dfr\YYYpls}{\dfr\tau} 
&&= \frac{(1 - \ssa)}{\plrmubar} \frac{1}{\plrmubar} \YYYmns
&&= \frac{1 - \ssa}{\plrmubar^{2}} \YYYmns \\
\frac{\dfrsqr\YYYpls}{\dfr\tausqr} &= 
\frac{1}{\plrmubar} \frac{\dfr\YYYmns}{\dfr\tau}
&&= \frac{1}{\plrmubar} \frac{(1 - \ssa)}{\plrmubar} \YYYpls
&&= \frac{1 - \ssa}{\plrmubar^{2}} \YYYpls
\end{alignat*}
The uncoupled quantities $\YYYpm$ satisfy the same second
order ordinary differential equation
\begin{eqnarray}
% ThS99 p. 228 (7.30)
\frac{\dfrsqr\YYYpm}{\dfr\tausqr} & = & \Gamma^{2} \YYYpm
\label{eqn:two_str_apx5}
\end{eqnarray}
where
\begin{eqnarray}
\Gamma^{2} & \equiv & \frac{(1 - \ssa)}{\plrmubar^{2}}
\label{eqn:gmm_dfn}
\end{eqnarray}
Elementary differential equation theory teaches us that the solutions
to (\ref{eqn:two_str_apx5}) are 
\begin{eqnarray}
% ThS99 p. 228 (7.30)
\YYYpm(\tau) & = & \cst^{\pm}_{1} \me^{\Gamma\tau} + \cst^{\pm}_{2} \me^{-\Gamma\tau}
\label{eqn:two_str_apx6}
\end{eqnarray}
The general forms of $\ntnupwdwn$ are obtained by inserting
(\ref{eqn:two_str_apx6}) into (\ref{eqn:ntn_upwdwn_dfn})
\begin{subequations}
\label{eqn:ntn_sln_gnr}
\begin{align}
% ThS99 p. 229 (7.34--7.35)
\ntnupwoftau & = \cstone\me^{\Gamma\tau} + \csttwo\me^{-\Gamma\tau} \\
\ntndwnoftau & = \cstthree\me^{\Gamma\tau} + \cstfour\me^{-\Gamma\tau} 
\end{align}
\end{subequations} 

Equation~(\ref{eqn:ntn_sln_gnr}) suggests that there are four
unknown constants of integration $\cstone$--$\cstfour$.
This is a consequence of solving for the hemispheric intensities
(\ref{eqn:ntn_hms_dfn}) separately in two equations
(\ref{eqn:two_str_apx}) rather solving the full radiative transfer
equation (\ref{eqn:rte_dfn_src}) for the full domain intensity $\ntn$.
As discussed in \S\ref{sxn:str_atm}, the motivation for solving the
half-range equations is that the physical boundary condition 
(at least in planetary science applications) is usually known for half
the angular domain at both the top and the bottom of the atmosphere.
These two independent boundary conditions for the first order
equations for the hemispheric intensities (\ref{eqn:two_str_apx})
mean there must be two additional relationships among the unknowns
$\cstone$--$\cstfour$.
Note that the boundary conditions themselves are still unspecified.

To obtain the relationships among $\cstone$--$\cstfour$ we substitute
the general solution forms (\ref{eqn:ntn_sln_gnr}) into the governing
equations (\ref{eqn:two_str_apx}). 
It can be shown (after much tedious algebra) that
\begin{eqnarray*}
% ThS99 p. 229 (7.33)
\csttwo & = & \rflinf \cstfour \\
\cstthree & = & \rflinf \cstone
\label{eqn:cst_rlt}
\end{eqnarray*}
where 
\begin{eqnarray}
% ThS99 p. 229 (7.33)
\rflinf & \equiv & \frac{1 - \sqrt{1 - \ssa}}{1 + \sqrt{1 - \ssa}}
\label{eqn:rfl_inf_dfn}
\end{eqnarray}
With these additional constraints, (\ref{eqn:ntn_sln_gnr}) becomes
\begin{subequations}
\label{eqn:ntn_sln_cns}
\begin{align}
% ThS99 p. 229 (7.34--7.35)
\ntnupwoftau & = \cstone\me^{\Gamma\tau} + \rflinf\cstfour\me^{-\Gamma\tau} \\
\ntndwnoftau & = \rflinf\cstone\me^{\Gamma\tau} + \cstfour\me^{-\Gamma\tau} 
\end{align}
\end{subequations} 
where now only the constants $\cstone$ and $\cstfour$ remain to be
determined by the particular boundary conditions of the problem.

The most relevant two-stream problem that is easily tractable is that of an 
isotropic, downwelling hemispheric radiation field of intensity
$\ntncal$ illuminating a homogeneous slab which overlies a black
surface (so that upwelling intensity at the lower boundary is zero).
\begin{subequations}
\label{eqn:two_str_bc}
\begin{align}
% ThS99 p. 228 (7.32)
\ntnupw(\tau = \taustr) & = 0 \\
\ntndwn(\tau = 0) & = \ntncal 
\end{align}
\end{subequations} 

Substituting (\ref{eqn:two_str_bc}) into the solutions for the case of
an isotropically scattering medium (\ref{eqn:ntn_sln_cns}) we obtain
\begin{alignat*}{3} % KoD99 p. 424 describes alignment and &&
% ThS99 p. 229 (7.34--7.35)
\ntnupw(\tau = \taustr) &= 0 
&&= \cstone\me^{\Gamma\taustr} + \rflinf\cstfour\me^{-\Gamma\taustr} \\
\ntndwn(\tau = 0) &= \ntncal 
&&= \rflinf\cstone + \cstfour
\end{alignat*}
which lead to
\begin{subequations}
\label{eqn:cst_ntg_dfn}
\begin{alignat}{2} % KoD99 p. 424 describes alignment and &&
% ThS99 p. 229
\label{eqn:cst_ntg_dfna}
\cstone &=
\frac{\rflinf \ntncal \me^{-\Gamma \taustr}}
{\rflinfsqr \me^{-\Gamma \taustr} - \me^{\Gamma \taustr}}
&&= \frac{\rflinf \ntncal}{\rflinfsqr - \me^{2 \Gamma \taustr}} \\
\label{eqn:cst_ntg_dfnb}
\cstfour &=
\frac{\ntncal \me^{\Gamma \taustr}}
{\me^{\Gamma \taustr} - \rflinfsqr \me^{-\Gamma \taustr}}
&&= \frac{\ntncal}{1 - \rflinfsqr \me^{-2\Gamma \taustr}}
\end{alignat}
\end{subequations} 
It will prove convenient to define the denominators of
(\ref{eqn:cst_ntg_dfnb}) as
\begin{subequations}
\label{eqn:sln_dnm}
\begin{align}
% ThS99 p. 229 (7.38)
\label{eqn:sln_dnm_dfn}
\slndnm &= \me^{\Gamma \taustr} - \rflinfsqr \me^{-\Gamma \taustr} \\
\label{eqn:sln_dnm_str}
\slndnmstr &= 1 - \rflinfsqr \me^{-2 \Gamma \taustr}
\end{align}
\end{subequations} 
where $\slndnmstr = \me^{-\Gamma \taustr} \slndnm$.
The initial forms of $\cstone$ and $\cstfour$ contain both positive
and negative exponentials and are pleasingly symmetrical.
These forms have been divided top and bottom by 
$\me^{-\Gamma \taustr}$ to obtain the final forms on the RHS
for reasons that will be discussed shortly.

Substituting (\ref{eqn:cst_ntg_dfn}) and (\ref{eqn:sln_dnm}) into
(\ref{eqn:ntn_sln_cns}) we can finally write the intensities at
any optical depth $\tau$ in terms of the known quantities $\taustr$
and $\ssa$ 
\begin{subequations}
\label{eqn:ntn_sln}
\begin{alignat}{2} % KoD99 p. 424 describes alignment and &&
% ThS99 p. 229 (7.36--7.37)
\ntnupwoftau &= 
\frac{\rflinf \ntncal}{\slndnm} 
[ \me^{\Gamma(\taustr - \tau)} - \me^{-\Gamma (\taustr - \tau) } ]
&&= \frac{\rflinf \ntncal}{\slndnmstr} 
[ \me^{-\Gamma \tau} - \me^{-2\Gamma (\taustr - \tau/2) } ] \\
\ntndwnoftau &= 
\frac{\ntncal}{\slndnm} 
[ \me^{\Gamma(\taustr - \tau)} - \rflinfsqr \me^{-\Gamma (\taustr - \tau) } ]
&&= \frac{\ntncal}{\slndnmstr} 
[ \me^{-\Gamma \tau} - \rflinfsqr \me^{-2\Gamma (\taustr - \tau/2) } ]
\end{alignat}
\end{subequations} 
Two forms of the solution are presented.
The initial forms on the RHS are differences between positive and
negative exponential terms.
In this form, the solutions may appear to depend only on the
distance from the lower boundary ($\taustr - \tau$) but notice that
the $\slndnm$ term (\ref{eqn:sln_dnm_dfn}) depends on the absolute
layer thickness ($\taustr$) as well.
This solution form is not recommended for computational implementation
because the positive exponentials are difficult to handle for large
optical depths. 
Moreover, the difference between the exponential terms quickly
leads to a loss of numerical precision as $\tau \rightarrow \infty$.

The final forms on the RHS of (\ref{eqn:ntn_sln}) contain only
negative exponentials and are numerically well-behaved as 
$\tau \rightarrow \infty$.
Since $0 < \tau < \taustr$, it is clear that all exponentials in
(\ref{eqn:ntn_sln}) and (\ref{eqn:sln_dnm_dfn}) are negative
exponentials and thus well-conditioned for computational
applications. 
There are no physical processes which would lead to positive
exponential terms in the solutions, so it is always worthwhile 
doublechecking for accuracy or stability any formulae which contain
positive exponentials.

As expected, the solutions (\ref{eqn:ntn_sln}) are proportional to the
incident intensity $\ntncal$, which is the only source of energy in
the problem since we neglected the thermal source term while
constructing the governing equations (\ref{eqn:two_str_apx}). 
The upwelling radiance is proportional to the parameter $\rflinf$. 
The physical meaning of $\rflinf$ is elucidated by noting that
$\ntnupw \rightarrow \rflinf \ntncal$ as $\taustr \rightarrow \infty$.
Thus $\rflinf$ is the maximum reflectance of semi-infinite layer of
matter with the same optical properties as the layer in question.

\subsubsection[Layer Optical Properties]{Layer Optical Properties}\label{sxn:two_srm_lop}
The solutions for the intensities (\ref{eqn:ntn_sln}) allow us to
derive the other optical properties of the layer using
(\ref{eqn:flx_hms_two_str})--(\ref{eqn:htr_two_str}).
The \trmdfn[flux reflectance, total]{total flux reflectance}
$\brdfoffrqmtwopitwopi$ is the ratio of reflected irradiance to all
incident irradiance.  
\begin{eqnarray}
% ThS99 p. 230
\rfloffrq & \equiv & \brdfoffrqmtwopitwopi \equiv 
\frac{\flxupwfrq(\tau = 0)}{\flxdwnfrq(\tau = 0)} \approx
\frac{2 \mpi \plrmubar \ntnupwfrq(\tau = 0)}
{2 \mpi \plrmubar \ntndwnfrq(\tau = 0)}
=
\frac{\ntnupwfrq(\tau = 0)}{\ntndwnfrq(\tau = 0)} \nonumber \\
& = & 
\frac{\rflinf \ntncal}{\slndnmstr} ( 1 - \me^{-2\Gamma \taustr} )
\times
\left[ \frac{\ntncal}{\slndnmstr}
(1 - \rflinfsqr \me^{-2\Gamma \taustr}) \right]^{-1} \nonumber \\
& = & 
\frac{\rflinf}{\slndnmstr}(1 - \me^{-2\Gamma \taustr})
= \frac{\rflinf}{\slndnm}(\me^{\Gamma \taustr} - \me^{-\Gamma \taustr})
\nonumber \\
& = & 
\frac{\rflinf(1 - \me^{-2\Gamma \taustr})}{1 - \rflinfsqr \me^{-2\Gamma \taustr}}
= \frac{\rflinf(\me^{\Gamma \taustr} - \me^{-\Gamma \taustr})}
{\me^{\Gamma \taustr} - \rflinfsqr \me^{-\Gamma \taustr}}
\label{eqn:flx_rfl_two_srm}
\end{eqnarray}
where we have used the definition of $\slndnmstr$
(\ref{eqn:sln_dnm_str}) in the last step.

The \trmdfn[flux transmittance, total]{total flux transmittance}
$\trnoffrq$ is the ratio of transmitted irradiance to incident
irradiance.  
\footnote{A non-reflecting lower boundary ensures there is no 
ambiguity to the meaning of transmittance.  
A reflecting lower boundary artificially enhances the apparent
transmittance by allowing surface-reflected radiation to scatter off
the bottom of the layer and thus be counted as transmitted radiation.}
\begin{eqnarray}
% ThS99 p. 230
\trnoffrq & \equiv & \trn(\frq,-2\mpi,-2\mpi) \equiv
\frac{\flxdwnfrq(\tau=\taustr)}{\flxdwnfrq(\tau=0)} \approx
\frac{2 \mpi \plrmubar \ntndwnfrq(\tau = \taustr)}
{2 \mpi \plrmubar \ntndwnfrq(\tau = 0)} =
\frac{\ntndwnfrq(\tau = \taustr)}{\ntndwnfrq(\tau = 0)} \nonumber \\
& = &
\frac{\ntncal}{\slndnmstr} 
( \me^{-\Gamma \taustr} - \rflinfsqr \me^{-\Gamma \taustr} )
\times
\left[ \frac{\ntncal}{\slndnmstr} 
( 1 - \rflinfsqr \me^{-2 \Gamma \taustr} )
\right]^{-1}
 \nonumber \\ 
& = &
\frac{( 1 - \rflinfsqr ) \me^{-\Gamma \taustr}}{\slndnmstr}
= \frac{1 - \rflinfsqr}{\slndnm}
 \nonumber \\ 
& = &
\frac{( 1 - \rflinfsqr ) \me^{-\Gamma \taustr}}
{1 - \rflinfsqr \me^{-2 \Gamma \taustr}}
= \frac{1 - \rflinfsqr}
{\me^{\Gamma \taustr} - \rflinfsqr \me^{- \Gamma \taustr}}
\label{eqn:flx_trn_two_srm}
\end{eqnarray}
The first expressions in (\ref{eqn:flx_rfl_two_srm}) and 
(\ref{eqn:flx_trn_two_srm}) were derived using the $\slndnmstr$ forms 
of the solutions (\ref{eqn:ntn_sln}) and contain no positive
exponential terms which blow up as $\taustr \rightarrow \infty$.
The second expressions, on the other, hand, are numerically
ill-conditioned as $\taustr \rightarrow \infty$.

Conservation of energy requires that the 
\trmdfn[flux absorptance, total]{total flux absorptance}
$\absoffrq$ is one minus the sum of the transmittance and the
reflectance 
\begin{eqnarray}
% ThS99 p. 230
\absoffrq & \equiv & \abs(\frq,-2\mpi) \equiv
\frac{\flxdwnfrq(0)-\flxdwnfrq(\taustr)+\flxupwfrq(\taustr)-\flxupwfrq(0)}{\flxdwnfrq(0)} \approx
\frac{\ntndwnfrq(0)-\ntndwnfrq(\taustr)+\ntnupwfrq(\taustr)-\ntnupwfrq(0)}{\ntndwnfrq(0)} \nonumber \\
& = &
1 - \trnoffrq - \rfloffrq
\label{eqn:flx_abs_two_srm}
\end{eqnarray}

\subsubsection[Conservative Scattering Limit]{Conservative Scattering Limit}\label{sxn:two_srm_csl}

The preceding section evaluated $\rfloffrq$ and $\trnoffrq$ for the
entire layer for a non-conservative scattering ($\ssa \ne 1$)
atmosphere. 
When $\ssa = 1$ the two-stream equations take a simpler form with the
result:
\begin{eqnarray}
% ThS99 p. 233 (7.53)
\rfloffrq & \equiv & \brdfoffrqmtwopitwopi \equiv 
\frac{\ntnupwfrq(\tau = 0)}{\ntndwnfrq(\tau = 0)} \nonumber \\
& = &
\frac{\ntncal(\taustr - \tau)}{2\plrmubar + \taustr}
\times
\left[ \frac{\ntncal(2\plrmubar + \taustr - \tau)}{2\plrmubar + \taustr}
\right]^{-1} \nonumber \\
& = &
\frac{\taustr - \tau}{2\plrmubar + \taustr - \tau} \\
& = &
\frac{\taustr}{2\plrmubar + \taustr} \nonumber
\label{eqn:flx_rfl_two_srm_csl}
\end{eqnarray}

\begin{eqnarray}
% ThS99 p. 233 (7.53)
\trnoffrq & \equiv & \trn(\frq,-2\mpi,-2\mpi) \equiv
\frac{\ntndwnfrq(\tau = \taustr)}{\ntndwnfrq(\tau = 0)} \nonumber \\
& = &
\frac{\ntncal (2\plrmubar + \taustr - \tau)}{2\plrmubar + \taustr}
\times \frac{1}{\ntncal} \nonumber \\
& = &
\frac{2\plrmubar + \taustr - \tau}{2\plrmubar + \taustr} \\
& = &
\frac{2\plrmubar}{2\plrmubar + \taustr} \nonumber
\label{eqn:flx_trn_two_srm_csl}
\end{eqnarray}

\clearpage
% PDF does not allow nested hyperlinks in TOC
\subsection[Solar Heating]{Example: Solar Heating by Uniformly Mixed Gases}\label{sxn:slr_htg}
In hydrostatic balance, the density of a uniformly mixed species
$\gasidx$ is
\begin{eqnarray}
% Har94 p. 54 (3.18)
\dnsidx(\hgt) & = & \dnsidx(\hgt = 0) \me^{-\hgt/\hgtscl} \\
\label{eqn:dns_hgt}
\end{eqnarray}
The optical depth of a purely absorbing species with volume
absorption coefficient $\abscffvlm$ is (\ref{eqn:tau_int_dfn})  
\begin{eqnarray}
% Har94 p. 54 (3.18)
fxm
\label{eqn:dns_hgt}
\end{eqnarray}


\subsection[Chapter Exercises]{Exercises for Chapter~\ref{sxn:rte}}\label{sxn:xrsz_rte}

\begin{enumerate*}
\item Consider a vertically homogeneous stratiform liquid water cloud
extending from $\zzz = 1$--$2$\,km above the surface, i.e., the cloud
is 1\,km thick.
Assume the sun is directly overhead and let $\flx(\zzz)$ denote the
downwelling flux (in \wxmS) in the direct solar beam.
At the top of the cloud $\flx(\zzz = 2) = \flxslrtoa$.
In the middle of the cloud it is found that $\flx(\zzz = 1.5) =
\flxslrtoa$/2. 
\begin{enumerate}
\item What causes $\flx(\zzz)$ to decrease from the top to the middle
of the cloud?\\
\vspace{\hmwskplng}
Answer: Absorption and scattering by cloud droplets.
\vspace{\hmwskpsht}
\item What is $\flx(\zzz = 1)$, i.e., the downwelling flux exiting the 
bottom of the cloud?
Answer: $\flxslrtoa/4$
\item What is the extinction optical depth of the cloud $\tauext$?
Answer: $\ln 2$
\item Assuming the cloud droplets are uniform in size, what additional
information about the cloud is needed to estimate the droplet radius? 
Answer: The mass or number concentration of cloud droplets and their
density (i.e., the density of water, 1\,\gxcmC).
Also, the extinction efficiency $\fshext$ must be known, but it can be
assumed to be 2.
% Assume the cloud droplets are uniform in size and cause an extinction of
%$\kkk = 0.0001$\,\xm.
%Measurements from a Microwave Radiometer (MWR) reveal the liquid water
%path (LWP) in the cloud is about 100\,\gxmS.
\end{enumerate}
\end{enumerate*}
\clearpage

\section[Remote Sensing]{Remote Sensing}\label{sxn:rmt_sns}

\csznote{
% Sor01 Sorensen review of soot fractal aggregates
s_2 is an average number of monomers per aggregrate 
s_i is M_i/M_{i-1}
s_2 is M_2/M_1 is a mean size
n_2 is fxm
R_v is fxm
n(N) is number concentration of soot aggregates with N monomers
N is number of monomers per aggregate
M_i is i'th moment of size distribution n(N)
} % end csznote
\subsection[Rayleigh Limit]{Rayleigh Limit}\label{sxn:ryl_lmt}
% Ste94 p. 216
The \trmidx{size parameter} $\szprm$ is the ratio of particle
circumference to wavelength, times the real part of the refractive
index of the surrounding medium $\idxrfrmdmrl$
\begin{eqnarray}
\szprm & = & \wvnbr \rds \idxrfrmdmrl \\
& = & 2 \mpi \rds \idxrfrmdmrl / \wvl
\label{eqn:sz_prm_dfn}
\end{eqnarray}
For particles in air, $\idxrfrmdmrl \approx 1$ so 
$\szprm \approx 2\mpi\rds/\wvl$.
The \trmdfn{Rayleigh limit} is the regime where the wavelength is
large compared to the particle size, $\wvl \gg \rds$, or,
equivalently, where $\szprm \rightarrow 0$.
Note that this regime applies equally to visible light interacting
with nanoparticles, and to microwave radiation interacting with cloud
droplets. 
The absorption and scattering efficiencies reduce to simple, closed
form expressions in the Rayleigh limit
\begin{subequations}
\label{eqn:ryl_lmt}
\begin{align}
\label{eqn:ryl_lmt_abs}
% Van57 p. 432 Ste94 p. 216 (5.24)
\lim_{\szprm\to 0} \fshabs & = -4\szprm \, \im \! \left( 
\frac{\idxrfr^{2}-1}{\idxrfr^{2}+2} \right) \\
\label{eqn:ryl_lmt_sct}
\lim_{\szprm\to 0} \fshsct & = \frac{8\szprm^{4}}{3}
\left| \frac{\idxrfr^{2}-1}{\idxrfr^{2}+2} \right|^{2}
\end{align}
\end{subequations} 
where $\im(\zzz)$ is the imaginary part of $\zzz$.
We see that $\fshabs > \fshsct$ as $\szprm\to 0$. 
Thus we may rewrite (\ref{eqn:xsx_ffc}) as
\begin{eqnarray}
% Ste94 p. 216 (5.25)
\xsxextffc(\hgt) & = & \int_{0}^{\infty}
-4 \left( \frac{2\mpi \rds}{\wvl} \right) \, \im \! \left( 
\frac{\idxrfr^{2}-1}{\idxrfr^{2}+2} \right) 
\mpi \rds^{2} \dstnbr(\rds,\hgt) \,\dfr\rds \nonumber \\
& = &
-\frac{8\mpi^{2}}{\wvl} \, \im \! \left( 
\frac{\idxrfr^{2}-1}{\idxrfr^{2}+2} \right) 
 \int_{0}^{\infty} \rds^{3} \dstnbr(\rds,\hgt) \,\dfr\rds \nonumber \\
& = &
-\frac{8\mpi^{2}}{\wvl} \, \im \! \left( 
\frac{\idxrfr^{2}-1}{\idxrfr^{2}+2} \right) 
\frac{3}{4 \mpi \dns} \int_{0}^{\infty} 
\frac{4 \mpi \dns}{3} \rds^{3} \dstnbr(\rds,\hgt) \,\dfr\rds \nonumber \\
& = &
-\frac{6\mpi}{\dns \wvl} \, \im \! \left( 
\frac{\idxrfr^{2}-1}{\idxrfr^{2}+2} \right) 
\mssttl(\hgt)
\end{eqnarray} 
where in the last step we have replaced the integrand by the total
mass concentration, $\mssttl$\,\kgxmC.  
If the cloud particles are liquid droplets then 
the density $\dns$ and index of refraction $\idxrfr$ are well known
and $\mssttl$ is referred to as the \trmidx{liquid water content}.
Assuming the extinction by cloud droplets dominates the volume
extinction coefficient then we may integrate over the thickness of the
cloud to determine the total optical depth.
Referring to (\ref{eqn:tau_ext_ffc}) we have
\begin{eqnarray} 
% Ste94 p. 216 (5.26)
\tauextffc
& = & 
-\frac{6\mpi}{\dns \wvl} \, \im \! \left( 
\frac{\idxrfr^{2}-1}{\idxrfr^{2}+2} \right) 
\int_{\hgtdlt} \mssttl(\hgt) \,\dfr\hgt \nonumber \\
& = & 
-\frac{6\mpi}{\dns \wvl} \, \im \! \left( 
\frac{\idxrfr^{2}-1}{\idxrfr^{2}+2} \right) 
\msspth
\label{eqn:tau_cld_ryl_lmt}
\end{eqnarray} 
where $\msspth$ is the path-integrated water content in the cloud,
called the \trmdfn{liquid water path}.
It is notable that $\tauextffc$ explicitly depends on the 
total condensed water mass, but not on the cloud droplet size.
In contrast, (\ref{eqn:tau_xmp_4}) shows that $\tauextffc$ in visible 
wavelengths depends on both $\msspth$ and $\rds$.
If the microwave optical depth of the cloud is known, through any
independent measurement, then we may use it to obtain $\msspth$ by
inverting (\ref{eqn:tau_cld_ryl_lmt}).
Fortunately $\tauextffc(\wvl)$ may be independently estimated 
using the microwave \trmidx{brightness temperature} of clouds.

\subsection[Anomalous Diffraction Theory]{Anomalous Diffraction Theory}\label{sxn:adt}
The full \trmidx{Mie theory} solutions are amenable to approximations
in certain limits, the Rayleigh approximation (\S\ref{sxn:ryl_lmt}) 
for example.
Scattering and absorption by particles with indices of refraction near
unity ($\idxrfr \approx 1$) may be treated with \trmdfn{anomalous
diffraction theory} (\trmidx{ADT}) in the regime of large \trmidx{size 
parameters} $\szprm \gg 1$. 
\trmprn{ADT}, described in \cite{Van57}, estimates particle-radiation
interaction in this regime by considering the effect refraction 
has on the phase delay between scattered and diffracted waves.
Consider incident plane waves of the form
\begin{eqnarray}
\lctvctcpx(\drcvct,\tm) 
& = & 
\lctvctnot \exp[\mi (\wvnbrvct \cdot \drcvct - \frqngl \tm ) ] \nonumber \\
& = & 
\lctvctnot \exp[\mi (\wvnbrvct \cdot \drcvct - \mi \frqngl \tm +
\nglphz) ] \nonumber \\
\lctvct(\drcvct,\tm) & = & \re(\lctvctcpx) \nonumber \\
& = & 
\lctfldnot \cos(\wvnbrvct \cdot \drcvct - \frqngl \tm + \nglphz)
\nonumber \\
& = & 
\lctfld_{0,\xxx} \cos(\wvnbrvct \cdot \drcvct - \frqngl \tm + \nglphz_{\xxx}) \ihat +
\lctfld_{0,\yyy} \cos(\wvnbrvct \cdot \drcvct - \frqngl \tm + \nglphz_{\yyy}) \jhat +
\lctfld_{0,\zzz} \cos(\wvnbrvct \cdot \drcvct - \frqngl \tm + \nglphz_{\zzz}) \khat
\label{eqn:pln_wv_dfn}
\end{eqnarray}
where the phase angle $\nglphz$ is an arbitrary angular offset.
Of course plane waves which explicitly include $\nglphz$ satisfy
\trmidx{Maxwell's equations} (\ref{eqn:mxw_eqn_SI}).
$\nglphz$ is often omitted from plane wave solutions since,
being a constant offset, it has no effect on the physical properties
of the solution.
We include $\nglphz$ in (\ref{eqn:pln_wv_dfn}) because it will prove
useful to examine the phase delay which accrues in plane waves due to
particle scattering. 

After particle scattering, the electric field is comprised of
two waves: the incident plane wave $\lctvctnot$ which travels
unrefracted through the particle (because $\idxrfr \approx 1$),
and the scattered wave, $\lctvctsct$ which has been diffracted by the 
edge of the particle. 
\begin{eqnarray}
\lctfld & = & \lctfldnot + \lctfldsct
\end{eqnarray}
We assume that rays suffer no reflection or deviations at the particle
boundaries since $\idxrfr \approx 1$. 
Thus the particle changes $\lctvctsct$ only in phase, not in amplitude.
The change in phase may be determined by considering the geometric
path length through the particle.
Collimated rays passing through the particle of radius $\rds$ a
distance $\mpcprm$ (the \trmidx{impaction parameter}) from the center
define chords of length $2\mpcprm \sin \nglntr$ from the entry point 
to the exit point. 
The angle $\nglntr$ lays between the radius to the point where the light
ray enters the sphere, and the line segment (of length $\mpcprm$)
joining the center of the sphere to the chord the ray travels through
the sphere.
\begin{eqnarray}
\nglntr & = & \cos^{-1} \mpcprm / \rds
\label{eqn:ngl_ntr_dfn}
\end{eqnarray}
Thus the angle between the entry and exit points and the center
of the particle is $2\nglntr$.
The phase lag $\phzdlt$ is determined by the phase of the electric
field measured on a plane behind the particle but normal to the
propagation of the incident wave.
The electric field measured within the geometric shadow of the
particle has suffered a \trmdfn{phase lag} relative to the field
outside the geometric shadow, which has not interacted with the
particle.   
The path length traveled within the particle is $\raycrd$
\begin{eqnarray}
\raycrd & = & 2 \rds \sin \nglntr
\label{eqn:ray_crd_dfn}
\end{eqnarray}
The phase lag $\phzdlt$ suffered during particle transit is $\raycrd$ 
times the difference in propagation speeds times the spatial
wavenumber $\wvnbr = 2\mpi / \wvl$ (\ref{eqn:wvnbr_dfn}).
We assume the particle is suspended in air so the propagation
speeds differ by a factor of $\idxrfr - 1$.
\begin{eqnarray}
\phzdlt & = & 
\raycrd \times (\idxrfr - 1 ) \times \wvnbr \nonumber \\
& = &  2 \rds \sin \nglntr \times (\idxrfr - 1 ) \times \wvnbr \nonumber \\
& = & \phzdlyctr \sin \nglntr
\label{eqn:phz_dlt_dfn}
\end{eqnarray}
where $\phzdlyctr$ is the phase delay suffered by a ray crossing the
full diameter of the particle
\begin{eqnarray}
\phzdlyctr & = & 2 \rds (\idxrfr - 1 ) (2 \mpi / \wvl) \nonumber \\
& = & 2 \szprm (\idxrfr - 1 )
\label{eqn:phz_dly_ctr}
\end{eqnarray}
The phase delay (\ref{eqn:phz_dlt_dfn}) causes a radially varying
reduction the electric field amplitude behind the particle, so that
\begin{eqnarray}
\lctfldsct & = & \lctfldnot \me^{-\mi \phzdlt} - 1
\label{eqn:lct_fld_sct_adt_dfn}
\end{eqnarray}

We shall determine the extinction efficiency due to the particle
by examining the scattering \trmdfn{amplitude function} 
$\sctfnc(\nglsct)$ at the \trmidx{scattering angle} $\nglsct = 0$.
The amplitude function $\sctfnc(0)$ is the integral over the geometric
shadow region of the ratio of the scattered wave to the incident wave.
\begin{eqnarray}
% Ste94 p. 233 (5.53), Van57 p. 175
\sctfnc(0) & = & \frac{\wvnbr^{2}}{2\mpi} 
\int_{\xsa} \frac{\lctfldsct}{\lctfldnot} \,\dfr \xsa \nonumber
\label{eqn:sct_fnc_dfn}
\end{eqnarray}
Following \cite{Van57}, we employ polar coordinates within the
geometric shadow region and begin by integrating over the impact
parameter (\ref{eqn:ngl_ntr_dfn})
\begin{eqnarray}
\mpcprm & = & \rds \cos \nglntr \nonumber
\label{eqn:mpc_prm_dfn}
\end{eqnarray}
from the center of the particle to the edge.
The element of area is a ring of radius $\mpcprm$ about the center of
the particle.
The elemental area is $2\mpi\mpcprm\,\dfr\mpcprm$ and area 
\begin{eqnarray}
% Van57 p. 175
\sctfnc(0)
& = & \frac{\wvnbr^{2}}{2\mpi} 
\int_{0}^{\rds} (1 - \me^{-\mi \phzdlyctr \sin \nglntr})
(2 \mpi \mpcprm) \,\dfr \mpcprm \nonumber \\
& = & \wvnbr^{2}
\int_{0}^{\rds} (1 - \me^{-\mi \phzdlyctr \sin \nglntr})
\,\mpcprm \,\dfr \mpcprm
\label{eqn:adt_dfn}
\end{eqnarray}
The first term in the integrand represents the incident wave.
The exponential term which represents the scattered wave is more
difficult to integrate. 
We now change integration variables from impact parameter to entry
angle using (\ref{eqn:mpc_prm_dfn}),
\begin{eqnarray}
\mpcprm & = & \rds \cos \nglntr \nonumber \\
\nglntr & = & \cos^{-1} (\mpcprm / \rds) \nonumber \\
\dfr \mpcprm & = & - \rds \sin \nglntr \,\dfr \nglntr \nonumber
\label{eqn:cov_mpc_prm_dfn}
\end{eqnarray}
The change of variables maps $\mpcprm \in [0,\rds]$ to
$\nglntr \in [0,\mpi/2]$.
Thus
\begin{eqnarray}
\sctfnc(0)
& = & \wvnbr^{2} \int_{0}^{\mpi/2}
(1 - \me^{-\mi \phzdlyctr \sin \nglntr})
(\rds \cos \nglntr)
(\rds \sin \nglntr) \,\dfr \nglntr \nonumber \\
& = & \wvnbr^{2} \rds^{2} \int_{0}^{\mpi/2}
(1 - \me^{-\mi \phzdlyctr \sin \nglntr})
\cos \nglntr \sin \nglntr \,\dfr \nglntr \nonumber \\
& \equiv & \wvnbr^{2} \rds^{2} \KKK(\www) 
\label{eqn:adt_ntg}
\end{eqnarray}
where $\www \equiv \mi \phzdlyctr$.
The first term in the integrand in $\KKK(\www)$ (\ref{eqn:adt_ntg}) is
a simple trigonometric function. 
The second term may be \trmidx[integration by parts]{integrated by
parts} with the formula 
$\int \uuu \,\dfr \vvv = \uuu \vvv - \int \vvv \,\dfr \uuu$
where
\begin{eqnarray}
\uuu & = & \sin \nglntr \nonumber \\
\dfr \vvv & = & \me^{-\www \sin \nglntr} \cos \nglntr \,\dfr \nglntr \nonumber \\
\vvv & = & -\frac{\me^{-\www \sin \nglntr}}{\www} \nonumber \\
\dfr \uuu & = & \cos \nglntr \,\dfr\nglntr
\label{eqn:cov_prt_ntg_dfn}
\end{eqnarray}
Using these standard techniques we obtain
\begin{eqnarray}
\KKK(\www) 
& = & 
\int_{0}^{\mpi/2} \sin \nglntr \cos \nglntr \,\dfr\nglntr
- (\uuu \vvv - \int \vvv \,\dfr \uuu) \nonumber \\
& = & 
\int_{0}^{\mpi/2} \frac{\sin 2 \nglntr}{2} \,\dfr\nglntr
- \left[ - \sin \nglntr 
\frac{\me^{-\www \sin \nglntr}}{\www} \right]_{0}^{\mpi/2}
+ \int_{0}^{\mpi/2} \frac{\me^{-\www \sin \nglntr}}{\www} 
\cos \nglntr \,\dfr\nglntr \nonumber \\
& = & 
\left[ - \frac{\cos 2 \nglntr}{4} \right]_{0}^{\mpi/2} -
\left( - \frac{\me^{-\www}}{\www} - 0 \right) +
\frac{1}{\www} \left[
- \frac{\me^{-\www \sin \nglntr}}{\www} \right]_{0}^{\mpi/2} \nonumber \\
& = & 
\frac{-(-1)-[-(1)]}{4} +
\frac{\me^{-\www}}{\www} +
% fxm: there is a sign error here
\frac{\me^{-\www} - 1}{\www^{2}} \nonumber \\
& = & 
\frac{1}{2} +
\frac{\me^{-\www}}{\www} +
\frac{\me^{-\www} - 1}{\www^{2}} 
\label{eqn:adt_kkk_sln}
\end{eqnarray}
With $\KKK(\www)$ known we can finally determine the scattering
function and the extinction efficiency in the anomalous diffraction
limit 
\begin{eqnarray}
% Lio02 p. 101 (3.3.31)
\sctfnc(0) & = & \szprm^{2} \KKK(\www) \nonumber \\
\fshext & = & \frac{4}{\szprm^{2}} \, \re \, [\sctfnc(0)] \nonumber \\
& = & 4 \, \re \, [\KKK(\www)]
\label{eqn:adt_sln}
\end{eqnarray}
For non-absorbing spheres, $\idxrfr$ is real so (\ref{eqn:adt_sln})
becomes
\begin{eqnarray}
% Lio02 p. 101 (3.3.31)
\fshext & = & 2 - \frac{4}{\phzdlyctr} \sin \phzdlyctr +
\frac{4}{\phzdlyctr^{2}} ( 1 - \cos \phzdlyctr )
\label{eqn:adt_sln_ext}
\end{eqnarray}
Following a similar procedure \cite[]{Lio02} leads to the ADT
approximation for absorption efficiency
\begin{eqnarray}
% Lio02 p. 101 (3.3.33)
\fshabs & = & 1 + \frac{2}{\bbb} \me^{-\bbb} + \frac{2}{\bbb^{2}} (\me^{-\bbb}-1)
\label{eqn:adt_sln_abs}
\end{eqnarray}
where $\bbb = 4\szprm\idxrfrimg$.

\subsection[Geometric Optics Approximation]{Geometric Optics Approximation}\label{sxn:goa}
\cite{Nus03} estimate the fraction of absorption due to Mie resonance
effects. 
First, they summarize previous studies \cite[][]{Nus92,Van57} which
show Mie scattering in transparent spheres results in quasi-periodic
resonances with approximate period $\prdszprm$. 
For water-based liquid aerosols typical of the atmosphere, the period
is of order unity in size parameter space.
A more precise estimate is
\begin{eqnarray}
% Nus03 p. 1588
\sqrtidxrfrsqrmnsone & \equiv & \sqrt{\idxrfrrl^{2}-1} \\ 
\prdszprm & \approx & \frac{\arctan \sqrtidxrfrsqrmnsone}{\sqrtidxrfrsqrmnsone}
\label{eqn:prd_sz_prm_dfn}
\end{eqnarray}

\cite{Nus03} estimate the absorption due to resonance effects based  
on \trmidx{complex angular momentum} (\trmidx{CAM}) theory.
Let $\sttdnsrsnfrc$ be the fractional contribution of resonances to the
total mean density of states.
In the asymptotic limit as $\szprm \gg 1$,
\begin{eqnarray}
% Nus03 p. 1589 (1)
\sttdnsrsnfrc & = & (\sqrtidxrfrsqrmnsone/\idxrfrrl)^{3}, \quad \mbox{for~}\szprm \ll 1
\label{eqn:stt_dns_rsn_frc}
\end{eqnarray}
For \HdO, $\idxrfrrl = 1.33$ so $\sttdnsrsnfrc \sim 29\%$.
Let $\fshabsrsnavg$ be the mean absorption efficiency due to
resonances across the approximate resonance period $\prdszprm$.
\citeauthor{Nus03} argues that, since each resonance adds to the
absorption efficiency, $\sttdnsrsnfrc$ is a good estimate of 
$\fshabsrsnavg/\fshabs$, the fractional contribution of resonances
to the overall absorption efficiency.

A second estimate of $\fshabsrsnavg/\fshabsrsn$ can be obtained from
the \trmidx{geometric optics approximation} (\trmidx{GOA}).
A fundamental result of GOA \cite[]{BoH83} is that
\begin{eqnarray}
% BoH83 p. 168 (7.2), Nus03 p. 1589 (1)
\fshabsgoa & = & \frac{8}{3} \idxrfrrl^{2} (1-\sttdnsrsnfrc)
\idxrfrimg \szprm, \quad \mbox{for~}\idxrfrimg\szprm \ll 1
\label{eqn:goa_sln_abs}
\end{eqnarray}
\cite{Nus03} combines (\ref{eqn:goa_sln_abs}) with
\trmidx{Wentzel-Kramers-Brillouin} (\trmidx{WKB}) theory to show
\begin{eqnarray}
% Nus03 p. 1589 (2) 
\frac{\fshabsrsnavg}{\fshabsgoa} & = & \frac{3}{4} 
\frac{\arctan\sqrtidxrfrsqrmnsone}{\idxrfrrl^{3}-\sqrtidxrfrsqrmnsone^{3}}
\left[ \left( \frac{\sqrtidxrfrsqrmnsone}{\arctan \sqrtidxrfrsqrmnsone} \right)^{2}-1 \right]
\quad \mbox{for $\szprm \gg 1$, $\idxrfrimg \ll 1$}  
\label{eqn:abs_rsn_frc}
\end{eqnarray}
For $\idxrfrrl = 1.33$, (\ref{eqn:abs_rsn_frc}) yields
$\fshabsrsnavg/\fshabsgoa \sim 16\%$, similar to the estimate based on
$\sttdnsrsnfrc$ (\ref{eqn:stt_dns_rsn_frc}).

Dave recommended a size parameter resolution for Mie integrations
of $\Delta \szprm \approx 0.1$.
\cite{Mar02} shows this leads to an underestimate of the absorptance
of soot in water spheres by about a factor of two.

\subsection[Single Scattered Intensity]{Single Scattered Intensity}\label{sxn:ssi}
In clear sky conditions it is reasonable to assume that most of the
majority photons measured by 
Downward looking satellite instruments measure upwelling radiation
reflected by the surface plus any photons scattered by the atmosphere
into the satellite viewing geometry. 
In clear sky conditions it is reasonable to approximate the solar
radiance measured by a satellite as consisting entirely of
singly-scattered photons. 

The upwelling intensity measured by a downward-looking instrument
mounted, e.g., on an aircraft or satellite, is formally described 
by the exact solution for upwelling radiance presented in
(\ref{eqn:rte_sln_upw}).
The full, multiple scattering source function with thermal emission
$\srcupwfrq$ (\ref{eqn:src_frq_dfn}) is difficult to obtain and we
shall instead make the \trmdfn{single-scattering approximation}. 
In the single-scattering approximation, we replace
$\srcupwfrq(\tau,\nglhat)$ by the 
\trmidx{single-scattering source function} 
$\srcstr(\tau,\nglhat)$ (\ref{eqn:src_str_upw_dfn}).

In the single-scattering approximation, the upwelling intensity
leaving the surface consists solely of reflected photons
(\ref{eqn:ntn_rfl_cll_dfn}) from the direct downwelling beam
(\ref{eqn:ntn_dwn_drc_dfn}) 
\begin{eqnarray}
\ntnupwfrq(\taustr,\nglhat) 
& = & \int_{-\nglhatprm} 
\ntndwndrc(\taustr,\nglhatprm) \brdfoffrqmnglhatprmnglhat 
\cos \plrprm \,\dfr\nglprm \nonumber \\
& = & \int_{-\nglhatprm} 
\flxslrtoa \me^{-\taustr/\plrmunot} \dltfncofnglhatprmmnglhatnot
\brdfoffrqmnglhatprmnglhat \plrmuprm \,\dfr\nglprm \nonumber \\
& = & 
\plrmunot \flxslrtoa \me^{-\taustr/\plrmunot} 
\brdfoffrqmnglhatnotnglhat
\label{eqn:ntn_rfl_sfc_dfn}
\end{eqnarray}
% fxm: Must finish this section by finishing definition of ssi

\subsection[Satellite Orbits]{Satellite Orbits}\label{sxn:stl_orb}
Assume a satellite is in a circular orbit about Earth
so that its Cartesian position $\psnbld$ at radial distance $\rdl$ as
a function of time $\tm$ is 
\begin{eqnarray}
\psnbld & = & \rdl (\sin \vlcngl \tm \ihat + \cos \vlcngl \tm \jhat)
\label{eqn:orb_crc_psn_dfn}
\end{eqnarray}
where $\vlcngl$ is the \trmdfn{angular velocity}.
As its name implies, $\vlcngl$ measures the rapidity with which
the angular coordinate changes (in radians per second, denoted \xs).
Let $\tauorb$ denote the \trmdfn{orbital period} of the satellite.
We assume the orbital speed $\vlc$ of a satellite in a circular orbit
is a constant.
Circular geometry dictates that
\begin{eqnarray}
\tauorb & = & 2 \mpi \rds / \vlc
\label{eqn:tau_orb_dfn}
\end{eqnarray}
By definition
\begin{eqnarray}
\vlcngl & = & 2\mpi \tauorb^{-1} \nonumber \\
& = & 2\mpi \left( \frac{2\mpi \rdl}{\vlc} \right)^{-1} \nonumber \\
& = & \vlc / \rdl
\label{eqn:vlc_ngl_dfn}
\end{eqnarray}
The acceleration $\xclbld$ of the satellite is the second derivative
of the position.
\begin{eqnarray}
\label{eqn:orb_crc_vlc_dfn}
\vlcbld & = & \rdl \vlcngl (\cos \vlcngl \tm \ihat - \sin \vlcngl \tm \jhat) \\
\label{eqn:orb_crc_xcl_dfn}
\xclbld & = & - \rdl \vlcngl^{2} (\sin \vlcngl \tm \ihat + \cos \vlcngl
\tm \jhat) \\
& = & - \vlcngl^{2} \psnbld
\end{eqnarray}
It is easy to show that $\vlcbld \cdot \psnbld = 0$ because they are
orthogonal. 
Thus the velocity of an object in a circular trajectory is tangential
to the radial vector.
In plane polar coordinates, $\psnbld = \rdl \rdlhat$ so
\begin{eqnarray}
\xclbld & = & - \rdl \vlcngl^{2} \rdlhat
\label{eqn:orb_crc_xcl_dfn2}
\end{eqnarray}
Moreover, (\ref{eqn:orb_crc_xcl_dfn2}) shows that the acceleration of
an object in a circular trajectory is opposite in direction to the
radial vector. 

Since the orbit is circular, the magnitudes of $\psnbld$, $\vlcbld$,
and $\xclbld$ are all constant with time and only their direction
changes. 
This can be verified by computing the vector magnitudes, e.g., 
\begin{eqnarray}
\xcl \equiv |\xclbld| & = & \sqrt{\xclbld \cdot \xclbld} \nonumber \\
& = & \rdl \vlcngl^{2} \sqrt{\sin^{2} \vlcngl \tm + \cos^{2} \vlcngl \tm} \nonumber \\
& = & \rdl \vlcngl^{2}
\label{eqn:xcl_mgn_dfn}
\end{eqnarray}
$\xcl$ is known as the \trmdfn{centripetal acceleration}.
Substituting (\ref{eqn:vlc_ngl_dfn}) into (\ref{eqn:xcl_mgn_dfn})
we obtain 
\begin{eqnarray}
\xcl & = & \rdl (\vlc / \rdl)^{2} \nonumber \\
& = & \vlc^{2} / \rdl \nonumber \\
\xclbld & = & - \frac{\vlc^{2}}{\rdl} \rdlhat
\label{eqn:orb_crc_xcl_dfn3}
\end{eqnarray}

According to \trmidx{Newton's Second Law} (\ref{eqn:2nd_law_2}), the
gravitational force $\frcbld$ (\ref{eqn:law_grv}) holding the
satellite in orbit must exactly balance (\ref{eqn:orb_crc_xcl_dfn3}). 
Let $\mssstl$ and $\mssrth$ denote the mass of the satellite and of
Earth, respectively.
Then
\begin{eqnarray}
\frcbld & = & \mssstl \xclbld  \nonumber \\
- \frac{\cstgrv \mssstl \mssrth}{\rdl^{2}} \rdlhat & = & - \mssstl \frac{\vlc^{2}}{\rdl} \rdlhat \nonumber \\
\frac{\cstgrv \mssstl \mssrth}{\rdl^{2}} & = & \mssstl \frac{\vlc^{2}}{\rdl} \nonumber \\
\frac{\cstgrv \mssrth}{\rdl^{2}} & = & \frac{\vlc^{2}}{\rdl} \nonumber \\
\vlc^{2} & = & \frac{\cstgrv \mssrth}{\rdl} \nonumber \\
\vlc & = & \sqrt{ \frac{\cstgrv \mssrth}{\rdl} }
\label{eqn:orb_stl_vlc_dfn}
\end{eqnarray}
We note that $\rdl$ is the distance from Earth's center of mass,
not its surface, to the satellite. 
Hence  satellite's speed increases slowly as its distance from
Earth decreases.
The orbital period $\tauorb$ is determined by substituting
(\ref{eqn:orb_stl_vlc_dfn}) into (\ref{eqn:tau_orb_dfn}) 
\begin{eqnarray}
\tauorb & = & 2 \mpi \rds \sqrt \frac{\rdl}{\cstgrv \mssrth} \nonumber \\
\tauorb^{2} & = & \frac{4 \mpi^{2} \rds^{3}}{\cstgrv \mssrth}
\label{eqn:orb_stl_vlc_dfn2}
\end{eqnarray}
Hence the satellite period squared is proportional to the cube of the
orbital size.
This relationship is known as \trmdfn{Kepler's Law} in honor of
its discoverer, the astronomer Johannes Kepler.

\subsection[Aerosol Characterization]{Aerosol Characterization}\label{sxn:aer_chr}
The wavelength and size dependence of aerosol optical properties 
are generally well-predicted if the aerosol composition and shape are
known. 
However, in situ aerosol composition and size are often difficult or 
impossible to obtain in the field.
Thus indirect estimates of particle size and composition are often
derived from the measurements that are available.
Many field sites and experiments are equipped to measure 
the spectrally resolved, column-integrated, aerosol extinction optical 
depth $\tauaerext(\wvliii)$ where $\iii$ denotes the spectral channel.
The following sections discuss techniques for inferring aerosol
optical depth from surface measurements, and the subsequent uses of
$\tauaerext(\wvliii)$ to calibrate radiometry, and to infer
information about aerosol size distribution and composition.

\subsubsection{Measuring Aerosol Optical Depth}\label{sxn:aod_msr}
Surface radiometry is incapable of directly measuring 
\trmdfn{aerosol optical depth} (AOD)\footnote{AOD implicitly refers to 
aerosol extinction optical depth $\tauaerext$, the sum of aerosol
absorption optical depth $\tauaerabs$ and aerosol scattering optical
depth $\tauaersct$.} because aerosol is vertically distributed in the
atmosphere. 
We now describe how AOD is inferred from surface measurements.
This will enable us to qualify what is usually meant by the
phrase ``measured optical depth'' as it appears in the literature.
Upward pointing radiometers may measure one or more parts of the
radiance field.
Downwelling total solar irradiance $\flxdwn$ is traditionally measured by
\trmdfn{global solar pyranometers}. 
The downwelling direct beam irradiance $\flxdwndrc$ is most accurately
measured by a \trmdfn{normal incidence pyrheliometer} (NIP).
Downwelling diffuse irradiance $\flxdwndff$ is measured by
\trmdfn{shaded pyranometers}. 
Instruments that measure all three irradiances also exist.
A class of instruments called \trmdfn{shadowband radiometers} do this
by periodically blocking the detector from the direct solar beam
using an occluding device called a shadowband.

\subsubsection{Aerosol Indirect Effects on Climate}\label{sxn:aer_ndr}
Greater than the direct radiative effect of aerosols on climate is the
uncertainty are the effects on climate of changes in cloud due to
aerosols.
These changes are collectively known as \trmdfn{Aerosol Indirect
  Effects} (AIE). 
The first AIE is the \trmdfn{droplet size effect} or \trmdfn{albedo
  effect} described by \cite{Two77}.
% fxm
The second AIE is the so-called \trmdfn{cloud lifetime effect} or 
\trmdfn{LWC effect}.
Other AIEs include the \trmdfn{cloud glaciation effect}.

Dust may suppress precipitation \cite[]{RRL01,RKR02} including
tropical storms and hurricanes \cite[]{} or enhance precipitation 
\cite[]{LeG96,WRL00}. % fxm add reference from Feingold 2002 AGU

\subsubsection{Aerosol Effects on Snow and Ice Albedo}\label{sxn:aer_alb}
Snow and ice always contain a number of particulate and bubble
inclusions.
The effect on reflectance is non-negligible when the particles are
substantially darker than the snow or ice.
\cite{WaW80} and \cite{WiW80} show that Mie theory is an appropriate,
but not exact, tool with which to model these effects.
% fxm get, read, and summarize paper
\cite{LEM98} present updated data on these effects. 

\subsubsection{\Angstrom\ Exponent}\label{sxn:ang_xpn}
The sensitivity of the particle extinction efficiency to wavelength
i.e., $|\partial \fshext(\rds,\wvl)/\partial \wvl|$, generally
increases with decreasing particle size.
This sensitivity holds true any paritcle composition, and may be
easily demonstrated using Mie codes.
An empirical measure of this sensitivity is obtained by defining
the \trmdfn{\Angstrom\ exponent} or
\trmdfn{\Angstrom\ parameter} $\angxpn$.
\begin{eqnarray}
\angxpn & = & \frac{\ln [ \tauext(\wvlone) / \tauext(\wvltwo) ] }
{\ln ( \wvlone / \wvltwo ) }
\label{eqn:ang_xpn_dfn}
\end{eqnarray}
Typical values of $\angxpn$ are $\angxpn > 2$ for small carbonaceous
aerosols (smoke) and urban pollution (sulfate) to $\angxpn \approx 0$
for large dust particles.

\section[Gaseous Absorption]{Gaseous Absorption}\label{sxn:abs}

\subsection{Line Shape}\label{sxn:lnshp}
Three fundamental processes interact to determine the observed shape
of spectral lines.
These are the natural line width, pressure broadening, and Doppler
broadening. 
\trmdfn{Natural line width}, which arises from the Heisenberg
uncertainty principle, is unimportant in planetary atmospheres.
Pressure- and Doppler-broadening effects dominate line shapes here.
However, the processes determining natural line widths and
\trmdfn{pressure broadening} of lines are statistically identical, so 
these two processes have the same analytical form.
In planetary atmospheres, \trmdfn{Doppler broadening} is intermediate
in importance between natural line shape and \trmdfn{pressure
broadening} or \trmdfn{collision broadening}.
At pressures weaker than about 1\,\mb\ collisions become infrequent
enough that the line shapes transition from Lorentzian to Doppler. 
In this transitional regime the line shape is a convolution of
Lorentzian and Doppler shapes.
This convolution is known as the \trmdfn{Voigt line shape}.

\subsubsection[Line Shape Factor]{Line Shape Factor}\label{sxn:lnshp_fct}
The fundamental property of a discrete line transition is the
\trmdfn{spectral absorption cross section}, $\xsxabsoffrq$, which
measures the strength of the transition as a function of frequency (or 
wavelength, etc.).
As discussed in Section~\ref{sxn:ext}, the units of $\xsxabsoffrq$
depend on the application, and they must be consistent with the units
of the absorber path, $\abspth$.  
The absorber mass path is integral of the absorber amount
$\abscnc$ along the path. 
\begin{equation}
\abspth(\pnt_{1},\pnt_{2}) = \int_{\pnt_{1}}^{\pnt_{2}} \abscnc(\pth) \,\dfr\pth
\label{eqn:abs_pth_dfn}
\end{equation}
Thus, if $\xsxabsoffrq$ is expressed in units of \mSxkg, then
$\abspth$ and $\dnsabs$ should be expressed in units of \kgxmS\ and
\kgxmC, respectively.
We adopt the convention that, when referrring to gaseous absorption,
$\xsxabsoffrq$ is the \trmidx{number absorption coefficient} and is
expressed in \mSxmlc. 
Closely related is the \trmidx{volume absorption coefficient}
$\abscffvlm$\,[\xm]
\begin{eqnarray}
\abscffvlmoffrq & = & \xsxabsoffrq \cncttl
\label{eqn:abs_cff_nbr_dfn}
\end{eqnarray}
where $\cncttl$ is the number concentration [\mlcxmC] of the species.

The value of $\xsxabsoffrq$ depends on the absorber type.
For a single spectral line, $\xsxabsoffrq$ is obtained by converting
the parameters in the \trmidx{\acr{HITRAN}} compilation \cite[]{RRG98} to
the local temperature and pressure. 
For a continuum absorption process, e.g., \Ot\ Chappuis bands,
$\xsxabsoffrq$ is usually obtained from a standard table (which may
include a parameterized temperature dependence).
The JPL compilation \cite[]{JPL94} is the standard reference for
absorption cross sections of photochemical species.

Given these definitions, the optical depth due to absorption
between points $\pnt_{1}$ and $\pnt_{2}$ is 
\begin{eqnarray}
\tauabs(\pnt_{1},\pnt_{2}) & = & \int_{\pnt_{1}}^{\pnt_{2}} \xsxabsoffrq
\abscnc(\pth) \,\dfr\pth 
\nonumber \\
& = & \xsxabsoffrq \int_{\pnt_{1}}^{\pnt_{2}} \abscnc(\pth) \,\dfr\pth \nonumber \\
& = & \xsxabsoffrq \abspth(\pnt_{1},\pnt_{2})
\label{eqn:tau_abs_dfn}
\end{eqnarray}
Comparing this to (\ref{eqn:tau_tld_dfn}) we see that $\xsxabsoffrq$ is the
extinction due to absorption.

The integrated strength of a transition is called the \trmdfn{line
strength} or \trmdfn{line intensity} and is denoted by $\lnstr$ 
\begin{equation}
\lnstr = \int_{0}^{\infty} \xsxabsoffrq \,\dfr\frq
\label{eqn:lnstr_dfn}
\end{equation}
The dimensions of $\lnstr$ are thus the dimensions of the absorption
coefficient times the dimensions of frequency (or wavelength), e.g.,
\mSxkgxhz\ or \mSxkgxm.
The relation between $\xsxabsoffrq$ and the line strength is called the
\trmdfn{line shape factor} $\lnshpofdltfrq$
\begin{subequations}
\label{eqn:lnshp_dfn}
\begin{align}
\label{eqn:xsx_lnstr_lnshp}
\xsxabsoffrq & = \lnstr \lnshpofdltfrq \\
\label{eqn:lnstr_xsx_lnshp}
\lnstr & = \xsxabsoffrq / \lnshpofdltfrq \\
\label{eqn:lnshp_xsx_lnstr}
\lnshpofdltfrq & = \xsxabsoffrq / \lnstr
\end{align}
\end{subequations} 
where $\frqnot$ is the central frequency of the unperturbed
transition. 
Thus $\lnshpofdltfrq$ relates the frequency-integrated absorption amount
(i.e., line strength) to the specific absorption a distance $\frq -
\frqnot$ from the line center.
The shape factor for each transition (i.e., line) is normalized to
unity
\begin{eqnarray}
\int_{0}^{\infty} \lnshpofdltfrq \,\dfr\frq & = & 1
\label{eqn:lnshp_nrm}
\end{eqnarray}
The dimensions of $\lnshp$ are inverse frequency or wavelength, e.g.,
\xfrq, \xhz, or \xm.

In this section we do not need to specify the which set of
self-consistent dimensions we choose for $\abscnc$, $\abspth$, and 
$\xsxabsoffrq$. 
Nevertheless, for concreteness we list the three most common choices.
First, we may employ dimensions relative to absorber number
concentration 
\newline\parbox{6in}{\begin{eqnarray} % KoD95 p. 138
\abscnc & = & \nbrcnc \qquad\mbox{\mlcxmC} \nonumber \\
\abspth & = & \nbrpth \qquad\mbox{\mlcxmS} \nonumber \\
\xsxabsoffrq & = & \abscffvlmoffrq \qquad\mbox{\mSxmlc}\nonumber 
\end{eqnarray}}\hfill % end parbox KoD95 p. 138
\parbox{1cm}{\begin{eqnarray}\label{eqn:abs_nbr_dfn}\end{eqnarray}}\newline
where $\abscffvlmoffrq$ is a \trmidx{number absorption coefficient}.
Alternatively, we may employ dimensions relative to absorber mass
concentration
\newline\parbox{6in}{\begin{eqnarray} % KoD95 p. 138
\abscnc & = & \msscnc \qquad\mbox{\kgxmC} \nonumber \\
\abspth & = & \msspth \qquad\mbox{\kgxmS} \nonumber \\
\xsxabsoffrq & = & \abscffmssoffrq \qquad\mbox{\mSxkg} \nonumber 
\end{eqnarray}}\hfill % end parbox KoD95 p. 138
where $\abscffmssoffrq$ is a \trmidx{mass absorption coefficient}.
\parbox{1cm}{\begin{eqnarray}\label{eqn:abs_mss_dfn}\end{eqnarray}}\newline
Finally, we may employ dimensions relative to absorber number
\newline\parbox{6in}{\begin{eqnarray} % KoD95 p. 138
\abscnc & = & \msscnc \qquad\mbox{\kgxmC} \nonumber \\
\abspth & = & \msspth \qquad\mbox{\kgxmS} \nonumber \\
\xsxabsoffrq & = & \abscffmssoffrq \qquad\mbox{\mSxkg} \nonumber 
\end{eqnarray}}\hfill % end parbox KoD95 p. 138
where $\abscffmssoffrq$ is a \trmidx{mass absorption coefficient}.
\parbox{1cm}{\begin{eqnarray}\label{eqn:abs_mss_dfn}\end{eqnarray}}\newline
We prefer to work in units absorption per unit mass
(\ref{eqn:abs_mss_dfn}) because, at least in models, the mass of
absorbers is more readily available than the number of absorbers.

\subsubsection[Natural Line Shape]{Natural Line Shape}\label{sxn:lnshp_ntr}
Radiative transitions from an upper to a lower state may occur 
spontaneously, in a process known as \trmdfn{spontaneous emission} or
\trmdfn{spontaneous decay}.
The time intervals $\tm$ between spontaneous emission are described by
a Poisson distribution 
\begin{eqnarray}
\pdfpss(\tm) & = & \frac{1}{\tauxct} \me^{-\tm / \tauxct}
\label{eqn:pss_dfn}
\end{eqnarray}
where $\tauxct$ is the mean lifetime of the excited state.

The time-dependent Schroedinger equation predicts that the relation
between the time intervals between such decays and the energy of the
decays gives rise to a continuous, rather than discrete, profile of 
absorption.
The resulting profile is called the  \trmidx{natural line shape} 
and is described by  
\begin{eqnarray}
\lnshpntr(\frq-\frqnot) & = & \frac{\hwhmntr}{\mpi [ (\frq - \frqnot)^{2} + \hwhmntr^{2} ] }
\label{eqn:lnshp_ntr_dfn}
\end{eqnarray}
where $\hwhmntr$ is the natural line shape half width at half-maximum
(HWHM). 
This is seen by noting that $\lnshpntr(\frq = \frqnot) =
(\mpi\hwhmntr)^{-1}$ is the full maximum value of $\lnshpntr(\frq)$, while
$\lnshpntr(\frq = \frqnot + \hwhmntr) = (2\mpi\hwhmntr)^{-1}$.

Natural broadening is one of two important broadening processes that
are described by the \trmdfn{Lorentz line shape}, $\lnshplrnofdltfrq$.  
The Lorentz line shape is defined identically to
(\ref{eqn:lnshp_ntr_dfn}) as 
\begin{eqnarray}
% RyL79 p. 99 (3.56), p. 289, ThS p. 66 (3.6), Shu91 p. 101 (10.6), p. 148 (14.29)
\lnshplrnofdltfrq & = & \frac{\hwhmlrn}{\mpi [ (\frq - \frqnot)^{2} + \hwhmlrn^{2} ] }
\label{eqn:lnshp_lrn_dfn}
\end{eqnarray}
The Lorentz profile (\ref{eqn:lnshp_lrn_dfn}) may also be written in
terms of the full width at half-maximum $\fwhmlrn = 2\hwhmntr$ as
\begin{eqnarray}
\lnshplrnofdltfrq & = & \frac{\fwhmlrn/2}{\mpi [ (\frq - \frqnot)^{2} +
(\fwhmlrn/2)^{2} ] }
\label{eqn:lnshp_lrn_fwhm_dfn}
\end{eqnarray}
The HWHM of a Lorentz profile is often written in terms of a parameter 
$\Hwhmlrn$ such that 
\begin{eqnarray}
\lnshplrnofdltfrq & = & \frac{\Hwhmlrn/4\mpi}{\mpi [ (\frq - \frqnot)^{2} +
(\Hwhmlrn/4\mpi)^{2} ] }
\label{eqn:lnshp_lrn_Hwhm_dfn}
\end{eqnarray}
The full meaning of $\Hwhmlrn$ is described below in
\S\ref{sxn:lnshp_prs}.  

The Lorentz line shape describes the relative absorption as a function 
of distance from line center.
A~separate parameter, the line strength, $\lnstr$
(\ref{eqn:lnstr_dfn}), defines the abolute absorption. 
Thus $\lnshplrnofdltfrq$ must be normalized such that its integral over
all frequencies is unity.
\begin{eqnarray}
\int_{0}^{\infty} \lnshplrnofdltfrq \,\dfr\frq
& = & \int_{0}^{\infty} 
\frac{\hwhmntr}{\mpi [ (\frq - \frqnot)^{2} + \hwhmntr^{2} ] } \,\dfr\frq
\nonumber \\
& = & \frac{\hwhmntr}{\mpi} \int_{0}^{\infty} 
\frac{1}{(\frq - \frqnot)^{2} + \hwhmntr^{2}} \,\dfr\frq
\nonumber \\
& = & \frac{\hwhmntr}{\mpi} \frac{1}{1/\hwhmntr^{2}} \frac{1/\hwhmntr^{2}}{1} 
\int_{0}^{\infty} 
\frac{1}{(\frq - \frqnot)^{2} + \hwhmntr^{2}} \,\dfr\frq
\nonumber \\
& = & \frac{1}{\mpi \hwhmntr} \int_{0}^{\infty}
\frac{1}{\frac{1}{\hwhmntr^{2}}(\frq - \frqnot)^{2} + 1} \,\dfr\frq
\nonumber 
\end{eqnarray}
We now make the change of variable 
\begin{eqnarray}
\yyy & = & (\frq - \frqnot)/\hwhmntr \nonumber \\
\frq & = & \frqnot + \hwhmntr \yyy \nonumber \\
\dfr\yyy & = &  \dfr\frq / \hwhmntr \nonumber \\
\dfr\frq & = & \hwhmntr \,\dfr\yyy \nonumber
\label{eqn:cov_y}
\end{eqnarray}
This change of variables maps $\frq \in [0,\infty)$ to 
$\yyy \in [-\frqnot/\hwhmntr,\infty)$.
Since $\frqnot \gg 1$, and the integrand is peaked near the origin,
we will replace the lower limit of integration by $-\infty$.
\begin{eqnarray}
\int_{0}^{\infty} \lnshplrnofdltfrq \,\dfr\frq
& = & 
\frac{1}{\mpi \hwhmntr} \int_{-\infty}^{\infty} 
\frac{1}{\yyy^{2} + 1} \hwhmntr \,\dfr\yyy \nonumber \\
& = & 
\frac{1}{\mpi} \int_{-\infty}^{\infty} 
\frac{1}{\yyy^{2} + 1} \,\dfr\yyy \nonumber \\
& = & 
\frac{1}{\mpi} \bigg[ \tan^{-1} \yyy \bigg]_{-\infty}^{\infty} \nonumber \\
& = & 
\frac{1}{\mpi} \left[ \frac{\mpi}{2} - \left( - \frac{\mpi}{2} \right) \right] \nonumber \\
& = & 1
\label{eqn:lnshplrn_ntr_nrm}
\end{eqnarray}
Thus the Lorentz line shape (\ref{eqn:lnshp_lrn_dfn}) is
correctly normalized. 

Some insight into the origin of natural broadening may be obtained
from simple examination of the Heisenberg uncertainty principle
which may be expressed as the fundamental uncertainty relating two
conjugate coordinates such as position and momentum, or, in our case,
time and energy:
\begin{eqnarray}
\dlttm \dltnrg  & \sim & \hbar 
\label{eqn:hsn_unc_prn}
\end{eqnarray}
When a molecule is in an excited state it has a probability of
spontaneously decaying to the lower state. 
Its actual state is uncertain on the timescale $\dlttm$, during which 
it may be in a superposition of allowed states.
Corresponding to this uncertainty, then, is an uncertainty in energy
$\dltnrg = \cstplk \dltfrq$ so that
\begin{eqnarray}
\dlttm \, \cstplk \dltfrq  & \sim & \cstplk/2\mpi \nonumber \\
\dltfrq  & \sim & \frac{1}{2\mpi \dlttm}
\label{eqn:hsn_unc_prn_2}
\end{eqnarray}
If we identify $\dlttm$ with $\tauxct$, the mean lifetime of the
excited state, and $\dltfrq$ with $\hwhmntr$, the natural half width,
then 
\begin{eqnarray}
\hwhmntr & = & \frac{1}{2 \mpi \tauxct}
\label{eqn:hwhm_ntr_dfn}
\end{eqnarray}
For spontaneous emission, $\tauxct$ is the reciprocal of the Einstein
$\AAA$ coefficient \cite[]{Shu91}.
A typical value of $\tauxct$ is 10$^{-8}$\,s.
This is an extremely long time relative to the typical mean time
between (optical) collisions in a planetary atmosphere, order
10$^{-10}$\,s.  
In practice, therefore, the line shapes we are concerned with are
dominated by collisional effects.

\subsubsection[Pressure Broadening]{Pressure Broadening}\label{sxn:lnshp_prs}
Collisions between molecules are the most important cause of line
broadening in the lower atmosphere.
Statistically, collisions in a gas of molecules of uniform speeds take 
place at random time intervals about a mean value.
As with spontaneous decay, the probability distribution of time
intervals between collisions is therefore a \trmidx{Poisson
distribution} (\ref{eqn:pss_dfn}).
In reality, molecular velocities are not uniform but obey the
\trmidx{Maxwell distribution} 
\begin{eqnarray}
\pdfmxw(\vlc) \,\dfr\vlc & = & 
\left( \frac{\mss}{2\mpi \cstblt \tpt} \right)^{1/2} 
\exp \left( -\frac{\mss \vlc^{2}}{2 \cstblt \tpt} \right) \,\dfr\vlc
\label{eqn:mxw_dst_dfn}
\end{eqnarray}
$\pdfmxw(\vlc)$ is the probability that a molecule of mass $\mss$ 
at temperature $\tpt$ has a velocity in $[\vlc,\vlc+\dfr\vlc]$.
In all other respects, collision-induced line broadening, or
\trmidx{pressure broadening} of lines is equivalent to
(\ref{eqn:lnshp_ntr_dfn}) and is thus governed by
a Lorentz line shape (\ref{eqn:lnshp_lrn_dfn}).
\begin{eqnarray}
\lnshpprs(\frq-\frqnot) & = & \frac{\hwhmprs}{\mpi [ (\frq - \frqnot)^{2} + \hwhmprs^{2} ] } 
\label{eqn:lnshp_prs_dfn}
\end{eqnarray}
where the half width at half-maximum intensity due to pressure
(collision) broadening, $\hwhmprs$ is called the
\trmdfn{pressure-broadened line width}. 
Because $\hwhmprs$ charaterizes collisional interactions, its value
depends on the concentrations and masses of all the molecular species
which are present.
Thus its determination is somewhat involved.

Molecular collisions occur in two senses, the kinetic and the
optical. 
A collision might change the kinetic energy of one or both of the
molecules, but this is not neccessary. 
Collisions that do alter the kinetic energy of one or both of the 
particpants are \trmdfn{kinetic collisions}. 
Collisions may also cause a radiative excitation or de-excitation
of the radiative energy levels of either or both of the molecules
involved. 
Collisional de-excitation, for example, might release one vibrational
quanta of energy without changing the kinetic energy of either of the
molecules.  
Collisions that alter the rotational or vibrational quanta of one or
both of the particpants are called \trmdfn{optical collisions}.
Rotational energy levels are, in general, much smaller than
translational energy levels and thus optical collisions may occur more
frequently than kinetic collisions.

Many methods of determining the pressure-broadened line shape have
been proposed.
A particularly satisfying model is that of molecules radiating with
constant intensity between elastic collisions that change the phase 
of the emitted wavetrain randomly.
The radiation is a delta-function in frequency with harmonic time
dependence
\begin{eqnarray}
% ThS99 p. 81 (3.6) 
\fnc(\tm) & = & \me^{\mi \frqnot \tm}
\label{eqn:rdn_pre_cll}
\end{eqnarray}
$\frqnot$ is the center of the unbroadened line.
We may restate (\ref{eqn:rdn_pre_cll}) in frequency space by taking
its Fourier transform
\begin{eqnarray}
% ThS99 p. 81 (3.6) 
\Fnc(\frq) & \equiv & \int_{-\infty}^{\infty} 
\fnc(\tm) \me^{-\mi \frq \tm} \,\dfr\tm \\
& = & \int_{-\infty}^{\infty} 
\me^{\mi \frqnot \tm} \me^{-\mi \frq \tm} \,\dfr\tm \nonumber \\
& = & \left[
\frac{\me^{\mi (\frqnot - \frq) \tm}}{\mi (\frqnot - \frq)} 
\right]_{}^{} \nonumber \\
& = & 
\frac{\me^{\mi (\frqnot - \frq) \taucll}}{\mi (\frqnot - \frq)} 
\nonumber \\
\label{eqn:rdn_pre_cll_frr}
\end{eqnarray}
The phase of the oscillation, and thus of $\fnc(\tm)$ is reset
randomly by each collision.
The waiting period between collisions determines the length of each
wavetrain.
The probability of a collision occuring between time 
$\tm$ and $\tm + \dfr\tm$ follows a \trmidx{Poisson distribution}
\begin{eqnarray}
% ThS99 p. 81 (3.6) 
\pdfpss(\tm) \,\dfr\tm & = & \me^{\tm/\taucll} \,\dfr\tm
\label{eqn:pdf_pss_dfn}
\end{eqnarray}

Let $\taucll$ and $\tauopt$ denote the mean times between kinetic
collisions and optical collisions, respectively.
By definition the mean collision frequency $\frqcll$ is the inverse
of $\taucll$.
The relationship between distance, rate, and time may be formulated to 
obtain $\taucll$ 
\begin{eqnarray}
% Bri86 p. 17
\taucll & = & \frqcll^{-1} = \frac{\mfp}{\vlcmlcavg}
\label{eqn:tau_cll_dfn}
\end{eqnarray}
where $\vlcmlcavg$ is the \trmidx{thermal speed} 
and $\mfp$ is the \trmidx{mean free path}, both of which are
known properties of the thermodynamic state of the atmosphere. 
First, let us recall that \trmidx{Maxwell-Boltzmann statistics} 
prescribe a square-root dependence of $\vlcmlcavg$ on $\tpt$ 
\begin{eqnarray}
% [m s-1] SeP97 p. 453 (8.2)
\vlcmlcavgA & = & \sqrt{\frac{8 \gascst \tpt}{\mpi}} = \sqrt{\frac{8 \gascstunv \tpt}{\mpi \mmw}} 
\label{eqn:vlc_mlc_avg_dfn}
\end{eqnarray}
where $\mmw$ is the mean molecular weight of the gas,
$\gascstunv = \mmw \gascst$ is the universal gas constant,
$\gascst$ is the specific gas constant, and $\tpt$ is the ambient
temperature.  
Kinetic theory tells us \citep[][p.~457]{Sep97} that the mean free
path $\mfpAA$ of molecular species $\AAA$ in a gas of molecular
species $\AAA$ is inversely related to the total concentration of
molecules and to their cross-sectional area for collisions,
$\mpi\dmtcllA^{2}/4$ 
\begin{eqnarray}
% [m] SeP97 p. 455 (8.5)
\label{eqn:mfp_BB_dfn}
% NB: Missing factor of cstAvagadro reported by Rod Brick and fixed 20100802
\mfpAA & = & \frac{1}{\mpi \sqrt{2} \cncA \dmtcllA^{2}} \\
& = & \frac{\mmw}{\mpi \sqrt{2} \dns \cstAvagadro \dmtcllA^{2}} \nonumber \\
& = & \frac{\mmw \gascst \tpt}{\mpi \sqrt{2} \prs \cstAvagadro \dmtcllA^{2}} \nonumber \\
& = & \frac{\gascstunv \tpt}{\mpi \sqrt{2} \prs \cstAvagadro \dmtcllA^{2}}
\label{eqn:mfp_dfn}
\end{eqnarray}
where%
\footnote{Unfortunately, terminology has placed in close proximity two 
  symbols that are easy to confuse: $\cncA$\,[\nbrxmC] is the number
  concentration of molecular species~\A, while $\cstAvagadro$\,[\nbrxmol] is
  Avagadro's constant.} 
we have re-expressed $\mfpAA$ in terms of $\prs$ and $\tpt$
by using $\cnc = \dns \cstAvagadro / \mmw$ and then
applying the ideal gas law $\dns = \prs / (\gascst \tpt)$
(\ref{eqn:igl}) to (\ref{eqn:mfp_BB_dfn}). 
There are many subtle assumptions embedded in (\ref{eqn:mfp_BB_dfn})
that should be clarified before proceeding:
First is that $\mfpAA$ is the mean free path of molecular species
$\A$ in a gas of molecular species $\A$, i.e., of a homogeneous
gas of~$\A$.
Often~$\A$ is a trace species (e.g., \COd) in air, represented by
species $\B$. 
The mean free path of $\A$ in $\B$ incorporates properties of both
$\A$ and $\B$ \citep[][p.~457]{Sep97}:

The pressure of an inhomogeneous gaseous mixture of $\A$ and $\B$ is
due, mainly, to the presence of the bulk medium which, for our
purposes is air represented by species~$\B$. 
Although the temperatures $\tpt$ of both gases in a mixture are equal  
in thermodynamic equilibrium, their densities and partial pressures
are unequal.
Hence the expressions for pressure-broadened line shapes of~$\A$ will
contain molecular properties of~$\A$ as well as bulk thermodynamics 
state (temperature, pressure) due mainly to~$\B$. 

For a homogeneous gas of~$\A$, substituting
(\ref{eqn:vlc_mlc_avg_dfn}) and (\ref{eqn:mfp_dfn}) into
(\ref{eqn:tau_cll_dfn}) yields
\begin{eqnarray}
\hwhmprs & = & \frqcll = \frac{\vlcmlcavg}{\mfp} \nonumber \\
& = & \sqrt{\frac{8 \gascstunv \tpt}{\mpi \mmwA}} \times
\frac{\mpi \sqrt{2} \prs \cstAvagadro \dmtcllA^{2}}{\gascstunv \tpt} \nonumber \\
& = & \sqrt{\frac{8 \gascstunv \tpt}{\mpi \mmwA}} \times
\frac{\mpi \sqrt{2} \prs \cstAvagadro \dmtcllA^{2}}{\gascstunv \tpt} \nonumber \\
& = & 4 \cstAvagadro \dmtcllA^{2} \sqrt{\frac{\mpi}{\gascstunv \mmwA}}
\frac{\prs}{\sqrt{\tpt}} \nonumber \\
& = & \frac{4 \cstAvagadro \dmtcllA^{2}}{\mmwA}
\sqrt{\frac{\mpi}{\gascstA}} \frac{\prs}{\sqrt{\tpt}}
\label{eqn:hwhm_prs_pre}
\end{eqnarray}
We see the interesting result that $\hwhmprs$ decreases as $\tpt$ increases.
This is because although thermal speed increases as $\sqrt{\tpt}$
(\ref{eqn:vlc_mlc_avg_dfn}), the mean free path $\mfp$ between
collisions increases linearly as $\tpt$ (\ref{eqn:mfp_dfn}).
The net result of increasing $\tpt$ is therefore to decrease collision
frequency $\frqcll$, and $\hwhmprs$ (\ref{eqn:hwhm_prs_pre}).

If we define
\begin{eqnarray}
\hwhmprsnot & \equiv & 4 \cstAvagadro \dmtcllA^{2} 
\sqrt{\frac{\mpi}{\gascstunv \mmwA}}
\frac{\prsnot}{\sqrt{\tptnot}} \nonumber \\
& \equiv & \frac{4 \cstAvagadro \dmtcllA^{2}}{\mmwA}
\sqrt{\frac{\mpi}{\gascstA}}
\frac{\prsnot}{\sqrt{\tptnot}}
\label{eqn:hwhm_prs_dfn}
\end{eqnarray}
then
\begin{eqnarray}
\hwhmprs(\prs,\tpt) & = & 
\hwhmprsnot \frac{\prs}{\prsnot} \left(\frac{\tptnot}{\tpt}\right)^{\hwhmxpn}
\label{eqn:hwhm_prs_scl_dfn}
\end{eqnarray}
where $\hwhmxpn = \frac{1}{2}$.
The exponent $\hwhmxpn$ determining the temperature dependence in
(\ref{eqn:hwhm_prs_scl_dfn}) is called the \trmdfn{line width exponent}. 
Two of the key line parameters measured in laboratory experiments 
are $\hwhmprsnot$ and $\hwhmxpn$.
Both are tabulated in databases such as \acr{HITRAN}
(\S\ref{sxn:htrn}).
Table~\ref{tbl:hwhmxpn} shows the mean, median, and range of the
temperature-dependent exponent $\hwhmxpn$ for the pressure-broadened
half width $\hwhmprs$ of various optically active gases.
\begin{table}
\begin{minipage}{\hsize} % Minipage necessary for footnotes KoD95 p. 110 (4.10.4)
\renewcommand{\footnoterule}{\rule{\hsize}{0.0cm}\vspace{-0.0cm}} % KoD95 p. 111
\begin{center}
\caption[Temperature Dependence of $\hwhmprs$]{\textbf{Temperature Dependence of $\hwhmprs$}%
\footnote{\emph{Sources:} \cite{RRG98}, \cite{Lio92}}%
\label{tbl:hwhmxpn}}
\vspace{\cpthdrhlnskp}
\begin{tabular}{ l >{$}c<{$} >{$}c<{$} >{$}c<{$} >{$}c<{$} }
\hline \rule{0.0ex}{\hlntblhdrskp}% 
Molecule & \mbox{Min} & \mbox{Mean} & \mbox{Median} & \mbox{Max} \\[0.0ex]
\hline \rule{0.0ex}{\hlntblntrskp}%
% ncwa -O -a ln_ctr -y min -v HWHM_tpt_dpn_xpn ${DATA}/hitran/H2O.nc ${DATA}/hitran/foo.nc;ncks -C -H -v HWHM_tpt_dpn_xpn ${DATA}/hitran/foo.nc
% fxm: the line strength-weighted temperature exponent would be more meaningful
\HdO & 0.28 & 0.66 & & 0.97 \\[1.0ex]
\COd & 0.49 & 0.75 & & 0.78 \\[1.0ex]
\Ot  & 0.76 & 0.76 & 0.76 & 0.76 \\[1.0ex]
\NdO & 0.64 & 0.78 & & 0.82 \\[1.0ex]
\CHq & 0.75 & 0.75 & & 0.75 \\[1.0ex]
\Od  & 0.63 & 0.71 & & 0.74 \\[1.0ex]
\OH  & 0.50 & 0.66 & & 0.66 \\[1.0ex]
\SOd & 0.50 & 0.60 & & 0.75 \\[1.0ex]
%& & & & \\[1.0ex]
\hline
\end{tabular}
\end{center}
\end{minipage}
\end{table}
For most optically active gases, $0.6 < \hwhmxpnavg < 0.8$.
This range differs considerably from $\hwhmxpn = \frac{1}{2}$
derived in (\ref{eqn:hwhm_prs_scl_dfn}), which is known as the
\trmdfn{classical value} of $\hwhmxpn$.
This discrepancy arises becauses the optical cross section
$\xsxopt$ of the molecules have a temperature dependence.

Table~\ref{tbl:hwhm} shows the mean, median, and range of the
pressure-broadened half width $\hwhmprs$ of various optically active
gases. 
\begin{table}
\begin{minipage}{\hsize} % Minipage necessary for footnotes KoD95 p. 110 (4.10.4)
\renewcommand{\footnoterule}{\rule{\hsize}{0.0cm}\vspace{-0.0cm}} % KoD95 p. 111
\begin{center}
\caption[Pressure-Broadened Half Widths]{\textbf{Pressure-Broadened Half Widths}%
\footnote{Units are \wvnxatm\ at 296\,\dgrk. \emph{Sources:} \cite{RRG98}}%
\label{tbl:hwhm}}
\vspace{\cpthdrhlnskp}
\begin{tabular}{ l >{$}c<{$} >{$}c<{$} >{$}c<{$} >{$}c<{$} }
\hline \rule{0.0ex}{\hlntblhdrskp}% 
Molecule & \mbox{Min} & \mbox{Mean} & \mbox{Median} & \mbox{Max} \\[0.0ex]
\hline \rule{0.0ex}{\hlntblntrskp}%
% ncwa -O -a ln_ctr -y min -v HWHM_tpt_dpn_xpn,HWHM_air ${DATA}/hitran/H2O.nc ${DATA}/hitran/foo.nc;ncks -C -H -v HWHM_tpt_dpn_xpn,HWHM_air ${DATA}/hitran/foo.nc
% fxm: the line strength-weighted HWHM would be more meaningful
\HdO & 0.0077 & 0.071 & & 0.11  \\[1.0ex]
\COd & 0.055  & 0.071 & & 0.095 \\[1.0ex]
\Ot  & 0.049  & 0.069 & & 0.084 \\[1.0ex]
\NdO & 0.069  & 0.075 & & 0.097 \\[1.0ex]
\CHq & 0.018  & 0.054 & & 0.16  \\[1.0ex]
\Od  & 0.028  & 0.043 & & 0.060 \\[1.0ex]
\OH  & 0.040  & 0.044 & & 0.095 \\[1.0ex]
\SOd & 0.10   & 0.11  & & 0.15  \\[1.0ex]
%& & & & \\[1.0ex]
\hline
\end{tabular}
\end{center}
\end{minipage}
\end{table}

Pure kinetic theory must be combined with the \trmdfn{optical cross
section} $\xsxopt$ to determine $\hwhmprs$%
\footnote{(\ref{eqn:hwhm_prs_dfn_GoY89}) is taken from \cite{GoY89}, p.~99.
The equation appears to contain typos and should be viewed
qualitatively until it is checked against a trustworthy reference on
kinetic theory. 
For example, the quadratic dependence on the concentration of
the perturber appears to be typo.
Also the optical collision diameter $\xsxopt$ is missing.}
\begin{eqnarray}
% GoY89 p. 99 (3.51)
% fxm: quadratic dependence on concentration of perturber must be a typo
% fxm: how exactly does optical collision diameter fit in here?
\hwhmprs & = & \sum_{\iii} \cnc_{\iii}^{2} 
\left[ \left( \frac{2 \cstblt \tpt}{\mpi} \right)
\left( \frac{1}{\mss} + \frac{1}{\mss_{\iii}}\right) \right]^{1/2}
\label{eqn:hwhm_prs_dfn_GoY89}
\end{eqnarray}
where the summation is over all perturber species $\iii$ (i.e., \Od,
\Nd, \ldots) and $\mss$ is the mass of the absorber.
Thus $\hwhmprs$ depends quadratically (fxm: typo, must be linear) upon
the number-density of collision partners and linearly upon the
velocity of the molecules (note the factor of $\sqrt{\cstblt \tpt}$). 
A scaling approximation is therefore frequently used to extrapolate
$\hwhmprs(\tpt)$ based on some measured or tabulated value
$\hwhmprs(\tptnot)$ 
\begin{eqnarray}
% ThS99 p. 69 (3.11), Ste94 p. 101 (3.20) Lio92 p. 28 (2.2.6)
% fxm: GoY89 use T/T_{0}, GoY89 p. 106 (3.71), GoY89 p. 99 (3.51)
% HITRAN, ThS99, Lio92, Bri86 use T_{0}/T, Why?
\hwhmprs(\tpt,\prs) & \approx & \hwhmprs(\tptnot,\prsnot) 
\frac{\prs}{\prsnot} \left( \frac{\tpt}{\tptnot} \right)^{1/2}
\label{eqn:hwhm_lrn_scl_dfn}
\end{eqnarray}
% See discussion in Ste94 p. 101

Atmospheric pressure changes by three orders of magnitude, from
1--1000\,mb, from 50\,km to the surface.
Temperature changes by only a factor of 2 over the same altitude
range. 
Since line widths (\ref{eqn:hwhm_lrn_scl_dfn}) depend linearly on
$\prs$, but only on the square root of $\tpt$, pressure variation
dominates vertical changes in $\hwhmprs$.

The mean time between optical collisions $\tauopt$ is then related to  
$\hwhmprs$ by
\begin{eqnarray}
\tauopt & = & \frac{1}{2 \mpi \hwhmprs} \nonumber \\
\hwhmprs & = & \frac{1}{2 \mpi \tauopt} \nonumber \\
\hwhmprs & = & \frac{\frqcll}{2 \mpi}
\label{eqn:tau_opt_dfn}
\end{eqnarray}
This is exactly analogous to the relation between $\tauxct$ and
$\hwhmntr$ (\ref{eqn:hwhm_ntr_dfn}) because both processes are
described by the Lorentz line shape.

When natural- and collision-broadening are considered simultaneously,
it can be shown that the combined line shape is also a Lorentzian with
HWHM $\Hwhmlrn/4\mpi$ (\ref{eqn:lnshp_lrn_Hwhm_dfn}).
The parameter $\Hwhmlrn$ is related to both the natural line width and
the mean frequency of collisions $\frqcll = \taucll^{-1}$
(\ref{eqn:tau_opt_dfn}) by
\begin{eqnarray}
% RyL79  p. 290 (10.75b)
\Hwhmlrn & = & 4\mpi\hwhmntr + 2 \nu
\label{eqn:Hwhm_lrn_dfn}
\end{eqnarray}
These relations may be proved by examining the power spectrum of a
sinusoidal electric field which is randomly interrupted $\frqcll$ times
per second. 
In practice in the lower atmosphere $\hwhmprs \gg \hwhmntr$ so the
Lorentz line width is nearly equivalent to the pressure-broadened
line width.
Therefore, \trmidx{Lorentz broadening} and the \trmidx{Lorentz line  
width} will hereafter refer to the Lorentzian line shape
(\ref{eqn:lnshp_lrn_dfn}) due to the convolution of natural- and
collision-broadening.  
With reference to (\ref{eqn:Hwhm_lrn_dfn}) and
(\ref{eqn:tau_opt_dfn}) 
\begin{eqnarray}
% RyL79  p. 290 (10.75b)
\hwhmlrn & = & \Hwhmlrn/4\mpi \nonumber \nonumber \\
\hwhmlrn & = & \hwhmntr + \frqcll / 2\mpi \nonumber \\
\hwhmlrn & = & \hwhmntr + \hwhmprs
\label{eqn:hwhm_lrn_dfn}
\end{eqnarray}

When discussing a given transition, it is convenient to translate the
origin of the frequency axis to the line center.
Rather than defining a new variable to do this, we continue to use
$\frq$.
Thus we replace $\frq - \frqnot$ by $\frq$ alone. 
The intended meaning of $\frq$ should be clear from the context.
Using this notation (\ref{eqn:lnshp_lrn_dfn}) becomes
\begin{eqnarray}
\lnshplrnoffrq & = & \frac{\hwhmlrn}{\mpi(\frq^{2} + \hwhmlrn^{2})}
\label{eqn:lnshp_lrn_dfn2}
\end{eqnarray}
The absorption cross-section (\ref{eqn:lnshp_dfn}) for
Lorentzian lines may now be written
\begin{eqnarray}
\xsxabsoffrq & = & \frac{\lnstr \hwhmlrn}{\mpi(\frq^{2} + \hwhmlrn^{2})}
\label{eqn:abs_lrn_dfn}
\end{eqnarray}

\subsubsection[Doppler Broadening]{Doppler Broadening}\label{sxn:lnshp_dpp}
Brownian motion causes molecules to constantly change their velocity
relative to the photons which may interact with them.
While the frequencies of resonant absorption are constant in the frame
of motion of the molecule, this random motion broadens the range of
resonant frequencies in the frame of a stationary observer.
This form of line broadening is called \trmdfn{Doppler broadening}. 
The probability that the molecule and a stationary reference frame
(an ``observer'') have relative velocity in $[\vlc,\vlc+\dfr\vlc]$ is
given by the Maxwell distribution (\ref{eqn:mxw_dst_dfn})
\begin{eqnarray}
\pdfmxw(\vlc) \,\dfr\vlc & = & 
\left( \frac{\mss}{2\mpi \cstblt \tpt} \right)^{1/2} 
\exp \left( -\frac{\mss \vlc^{2}}{2 \cstblt \tpt} \right) \,\dfr\vlc
\label{eqn:mxw_vlc_dfn}
\end{eqnarray}
We convert this Maxwellian PDF directly to the \trmdfn{Doppler line
shape} PDF $\lnshpdpp(\frq - \frqnot)$ by noting that each value of
$\vlc$ describes a shift in line center $\dltfrq \equiv \frq - \frqnot$.
For $\vlc/\cstspdlgt \ll 1$, the relation between $\vlc$ and the 
Doppler shift $\dltfrq$ is
\begin{eqnarray}
\dltfrq & = & \frqnot \vlc / \cstspdlgt \nonumber \\
\vlc & = & \cstspdlgt \dltfrq / \frqnot \nonumber \\
\dfr\frq & = & \dfr\dltfrq = \frac{\frqnot}{\cstspdlgt} \,\dfr\vlc \nonumber \\
\dfr\vlc & = & \frac{\cstspdlgt}{\frqnot} \,\dfr\frq \nonumber
\label{eqn:vlc_dpp_dfn}
\end{eqnarray}
When we re-express $\pdfmxw(\vlc)$ (\ref{eqn:mxw_vlc_dfn}) in terms of
$\dltfrq$ by imposing the condition
\begin{eqnarray}
\pdfmxw(\vlc) \,\dfr\vlc & = & \pdfmxw(\dltfrq) \,\dfr\dltfrq 
\label{eqn:dpp_dfr_dfn}
\end{eqnarray}
This relation between $\pdfmxw(\vlc)$ and $\pdfmxw(\dltfrq)$ 
is like that between $\plkwvl(\wvl)$ and $\plkfrq(\frq)$
(\ref{eqn:plk_frq_wvl_eqv}). 
Inserting (\ref{eqn:vlc_dpp_dfn}) into (\ref{eqn:mxw_vlc_dfn}) leads 
to 
\begin{eqnarray}
\pdfmxw[\vlc(\frq)] \,\dfr\vlc & = & 
\left( \frac{\mss}{2\mpi \cstblt \tpt} \right)^{1/2} 
\exp \left( -\frac{\mss \cstspdlgt^{2} \dltfrq^{2}}{2 \frqnot^{2} \cstblt
\tpt} \right) \, \frac{\dfr\vlc}{\dfr\frq} \,\dfr\frq \nonumber \\ 
& = & 
\frac{\cstspdlgt}{\frqnot} \left( \frac{\mss}{2\mpi \cstblt \tpt} \right)^{1/2} 
\exp \left[ - \dltfrq^{2} \left( 
\frac{2 \frqnot^{2} \cstblt \tpt}{\mss \cstspdlgt^{2}} \right)^{-1} \right] \,\dfr\frq
\label{eqn:dpp_vlc_pre_dfn}
\end{eqnarray}
The RHS of (\ref{eqn:dpp_vlc_pre_dfn}) is now strictly a function
of $\frq$.
We name this function the \trmidx{Doppler line shape},
$\lnshpdpp(\frq)$ and define the \trmidx{Doppler width} $\hwemdpp$
as the square root of the portion of the exponential in parentheses 
\begin{eqnarray}
% Lio92 p. 30 (2.2.9) ThS99 p. 71 (3.15) GoY89 p. 111 (3.82)
\hwemdpp
& = & 
\frac{\frqnot}{\cstspdlgt} 
\sqrt{ \frac{2 \cstblt \tpt}{\mss} }
\label{eqn:hwem_dpp_dfn}
\end{eqnarray}
so that (\ref{eqn:dpp_vlc_pre_dfn}) becomes
\begin{eqnarray}
\lnshpdpp(\frq) \,\dfr\frq & = & 
\frac{\cstspdlgt}{\frqnot} \left( \frac{\mss}{2\mpi \cstblt \tpt} \right)^{1/2} 
\exp \left( -\frac{\dltfrq^{2}}{\hwemdpp^{2}} \right) \,\dfr\frq \nonumber \\ 
& = & 
\frac{1}{\sqrt{\mpi}} \frac{\cstspdlgt}{\frqnot}
\left( \frac{\mss}{2 \cstblt \tpt} \right)^{1/2} 
\exp \left( -\frac{\dltfrq^{2}}{\hwemdpp^{2}} \right) \,\dfr\frq \nonumber \\ 
& = & 
\frac{1}{\hwemdpp \sqrt{\mpi}}
\exp \left[ - \left( \frac{\frq-\frqnot}{\hwemdpp} \right)^{2} \right] \,\dfr\frq
\label{eqn:lnshp_dpp_dfn}
\end{eqnarray}
The normalization of (\ref{eqn:lnshp_dpp_dfn}) follows from the fact
that it is simply a re-expression of the Maxwell distribution
(\ref{eqn:mxw_vlc_dfn}), which is already known to be normalized.
The full maximum value of $\lnshpdpp$ is 
$\lnshpdpp(\frq = \frqnot) = (\hwemdpp \sqrt{\mpi})^{-1}$, and 
$\lnshpdpp(\frq = \frqnot + \hwemdpp) = (\me \hwemdpp
\sqrt{\mpi})^{-1}$, so that $\hwemdpp$ is the half width at $\me^{-1}$
of the maximum of $\lnshpdpp$.
Care should be taken not to confuse $\hwemdpp$ with a half-width at 
half-maximum.
The half width at half-maximum of $\lnshpdpp(\frq)$ is 
\begin{eqnarray}
\hwhmdpp & = & \hwemdpp \sqrt{\ln 2}
\label{eqn:hwhm_dpp_dfn}
\end{eqnarray}
It is more appropriate to compare $\hwhmlrn$ (\ref{eqn:hwhm_prs_dfn})
to $\hwhmdpp$ (\ref{eqn:hwhm_dpp_dfn}), since they are identical
measures of line width, than to $\hwemdpp$ (\ref{eqn:hwem_dpp_dfn}).

We recognize that (\ref{eqn:lnshp_dpp_dfn}) is a form of
\trmdfn{Gaussian distribution}, although it is not expressed in the
canonical form of a Gaussian
\begin{eqnarray}
\pdfgss(\frq) \,\dfr\frq & = & 
\frac{1}{\stddvn \sqrt{2\mpi}}
\exp \left[ - \left( \frac{\frq-\frqnot}{\stddvn \sqrt{2}} \right)^{2} \right] \,\dfr\frq
\label{eqn:pdf_gss_dfn}
\end{eqnarray}
in which $\stddvn$ represents the standard deviation and $\frqnot$ the
mean value.
Comparing (\ref{eqn:pdf_gss_dfn}) to (\ref{eqn:lnshp_dpp_dfn}) we see
that the standard deviation of the Doppler line shape $\stddvndpp =
\hwemdpp/\sqrt{2}$.
Thus Doppler broadening alone allows absorptions to occur within
$\hwemdpp/\sqrt{2}$ and $\sqrt{2}\hwemdpp$ of line center with
efficiencies (relative to line center) of $1-\me^{-1} = 0.683$, 
and $1-\me^{-2} = 0.954$, respectively.

\subsubsection[Voigt Line Shape]{Voigt Line Shape}\label{sxn:lnshp_vgt}
Line broadening processes occur simultaneously in nature. 
In particular, \trmidx{Collision-broadening} occurs in tandem with  
\trmidx{Doppler broadening}.
If we assume these two processes are independent but occur
simulateously, then the net line shape of the total process will be
the collision-broadened line shape, shifted by the Doppler shift
(\ref{eqn:vlc_dpp_dfn}),  and averaged over the Maxwell distribution
(\ref{eqn:mxw_vlc_dfn}).
The resulting line shape is call the \trmdfn{Voigt profile},
$\lnshpvgt(\frq)$. 
\begin{eqnarray}
\lnshpvgt(\frq-\frqnot) & = & 
\int_{-\infty}^{\infty} 
\left( \frac{\mss}{2\mpi \cstblt \tpt} \right)^{1/2} 
\exp \left( - \frac{\mss \vlc^{2}}{2 \cstblt \tpt} \right) 
\frac{\hwhmlrn}{\mpi [ (\frq - \frqnot + \frqnot \vlc / \cstspdlgt )^{2} + \hwhmlrn^{2} ] }
\,\dfr\vlc \\
& = & 
\sqrt{ \frac{\mss}{2\mpi \cstblt \tpt} }
\int_{-\infty}^{\infty} 
\exp \left( - \frac{\mss \vlc^{2}}{2 \cstblt \tpt} \right) 
\frac{\hwhmlrn}{\mpi [ (\frq - \frqnot + \frqnot \vlc / \cstspdlgt )^{2} + \hwhmlrn^{2} ] }
\,\dfr\vlc
\label{eqn:vgt_pre_dfn}
\end{eqnarray}
Analytic approximations to and asymptotic behavior of
$\lnshpvgt(\frq-\frqnot)$ are discussed in \cite{GoY89,Lio92}.

With appropriate definitions \cite[e.g.,][p.30]{Lio92}
\begin{eqnarray}
\ttt & = & \sqrt{\ln 2} (\frq - \frqprm)/\hwemdpp \nonumber \\
\xxx & = & \sqrt{\ln 2} (\frq - \frqnot)/\hwemdpp \nonumber \\
\hwhmdpp & = & \sqrt{\ln 2} \hwemdpp \nonumber \\
\yyy & = & \sqrt{\ln 2} \hwhmprs / \hwhmdpp
\label{eqn:vgt_cov}
\end{eqnarray}
we may re-express (\ref{eqn:vgt_pre_dfn}) as
\begin{eqnarray}
\lnshpvgt(\frq-\frqnot) & = & 
\frac{1}{\hwhmdpp} \sqrt{\frac{\ln 2}{\mpi}} \KKK(\xxx,\yyy)
\label{eqn:vgt_err_fnc_cpx}
\end{eqnarray}
where the \trmdfn{complex error function} $\KKK(\xxx,\yyy)$ is defined
\begin{eqnarray}
\KKK(\xxx,\yyy) & = &
\frac{\yyy}{\mpi} \int_{-\infty}^{+\infty} 
\frac{\me^{-\ttt^{2}}}{\yyy^{2}+(\xxx-\ttt)^{2}} \,\dfr\ttt
\label{eqn:err_fnc_cpx_dfn}
\end{eqnarray}
Efficient algorithms for evaluating $\KKK(\xxx,\yyy)$ have been
developed \cite[]{HAW78,Hum82,Kun97} to reduce computational expense
in detailed line-by-line calculations.

\subsubsection[Collision-Induced Absorption (CIA)]{Collision-Induced Absorption (CIA)}\label{sxn:cia}
Symmetric molecules have no permanent electric dipole, yet can develop
a transient electric dipole when perturbed such as during collisions
\citep{Fro06,RGR12}.
In the absence of such perturbations, electric and magnetic quadrupole
moments cause the ``forbidden'' absorption of such molecules,
including \Hd, \Od, \Nd, and \CHq.
The additional absorption that occurs during collisional interactions
is called Collision-Induced Absorption (CIA).

CIA has unusual physical units.
Traditional absorption is expressed as cross-sectional area per
molecule (\mSxmlc).
When multiplied by the number concentration of the molecule (\mlcxmC)
the result is the absorption optical depth per unit pathlength (\xm).
The intensity of CIA varies with the product of the concentrations of
the partner molecules.
By analogy, the cross-sectional area per molecular pair is the
cross-sectional area per molecule per unit concentration of perturber
(\mFxmlcS). 
The CIA absorption optical depth per unit pathlength (\xm) is the
nominal cross-section per pair multiplied by the product of the 
concentrations of the partner molecules (\mlcSxmSix).

\section[Molecular Absorption]{Molecular Absorption}\label{sxn:mlc_abs}

We now attempt to develop an introductory understanding of
the location of absorption lines in gases.
The science of atomic and molecular \trmdfn{spectroscopy} is extremely
detailed and requires a quantum mechanical treatment for a
satisfactory explanation of all phenomenon.
Nevertheless, important insights as to the spectral location, 
temperature dependence, and relative population of radiative energy
levels may be gained by resorting to a semi-classical treatment of 
the radiation field.
In essence, our first task is to characterize the distribution of 
observed molecular transitions in the atmosphere, i.e., to
characterize $\dltnrg$ from Planck's relation
\begin{eqnarray}
\dltnrg & = & \cstplk \frq
\label{eqn:mlc_abs_dfn}
\end{eqnarray}
To begin, we enumerate the contributions to a molecule's total energy
$\nrgttl$.
The total molecular energy comprises the \trmdfn{tranlational kinetic
energy} $ngrtrn$, the \trmdfn{electronic energy} $\nrglct$, 
the \trmdfn{vibrational energy} $\nrgvbr$, 
the \trmdfn{rotational energy} $\nrgrtt$, and the \trmdfn{nuclear
energy} $\nrgncl$.
\begin{eqnarray}
\label{eqn:nrg_ttl_dfn}
\nrgttl & = & \nrgtrn + \nrglct + \nrgvbr + \nrgrtt + \nrgncl
\end{eqnarray}
These energy components have been listed in order of decreasing
magnitude. 
We are mainly interested in describing vibrational and rotational
energy transitions in the next sections.

\subsection[Mechanical Analogues]{Mechanical Analogues}\label{sxn:mch_anl}
In the quantum mechanical view, radiative absorption occurs when an
incident photon of energy level $\cstplk \frq$ encounters a molecule
with energy $\nrg_{1}$. 
If $\nrg_{1} + \cstplk \frq$ is ``close enough'' to an available energy
state $\nrg_{2}$ of the molecule, then absorption may occur.
If absorption does occur, the energy of the photon is transferred to
to the corresponding energetic mode or modes of the molecule.
These molecular modes, e.g., spin, vibration, rotation, take their
names from mechanical systems which form their classical analogues.
To understand the distribution of energy levels in these modes
it is thus useful to describe classical mechanical analogues 
which are known to behave like molecular systems.

Molecules may be thought of as rotating structures.
When the molecular structure is fixed, i.e., the separation between
the atoms does not change, we call the molecule obeys the
\trmdfn{rigid rotator} model.
Diatomic molecules, for example, may be visualized as barbell-shaped,
with an atom at each end, separated by a massless but rigid rod.
The separation of the atoms means their charges are separated.
If the center of mass between the atoms does not coincide with the 
center of charge, then the molecule has a permanent \trmidx{dipole
moment}. 
Molecules comprised of a single element, e.g., \Nd\ or \Od, are called
\trmdfn{homonuclear}.
Homonuclear molecules are perfectly symmetric rigid rotators, and thus
have no permanent dipole moment.

To continue with the quantum mechanical behavior of the rigid rotator
molecule, we must determine the available energy states of the system.
Let the atoms of masses $\mss_{1}$ and $\mss_{2}$ be separated by a
distance $\rds$.
Then the distances of each from the center of mass of the system are
\begin{eqnarray}
\rds_{1} & = & \frac{\mss_{1} \rds}{\mss_{1} + \mss_{2}} =
\frac{\mssrdc}{\mss_{2}} \rds \\
\rds_{2} & = & \frac{\mss_{2} \rds}{\mss_{1} + \mss_{2}} = 
\frac{\mssrdc}{\mss_{1}} \rds
\label{eqn:rgd_rtt_dst_dfn}
\end{eqnarray}
where the \trmdfn{reduced mass} $\mssrdc$ of the system is 
\begin{eqnarray}
\mssrdc & \equiv & \frac{\mss_{1} \mss_{2}}{\mss_{1} + \mss_{2}}
\label{eqn:mss_rdc_dfn}
\end{eqnarray}
The \trmdfn{moment of inertia} $\mmnrsh$ of the system about the
center of mass is the total mass-weighted mean square distance from
the axis of rotation
\begin{eqnarray}
\mmnrsh & = & \mss_{1} \rds_{1}^{2} + \mss_{2} \rds_{2}^{2}
\label{eqn:mmn_rsh_dfn}
\end{eqnarray}
The \trmdfn{angular momentum} $\mmnngl$ of the system is the
total mass-weighted product of the distance from the axis of rotation
times the velocity.
Denoting the angular frequency of the rotation by $\frqngl$, 
the linear velocity of the atoms are $\frqngl \rds_{1}$ and
$\frqngl \rds_{2}$.
\begin{eqnarray}
\mmnngl & = & \mss_{1} \rds_{1} \vlc_{1} + \mss_{2} \rds_{2} \vlc_{2} \nonumber \\
& = & \mss_{1} \frqngl \rds_{1}^{2} + \mss_{2} \frqngl \rds_{2}^{2} \nonumber \\
& = & \frqngl (\mss_{1} \rds_{1}^{2} + \mss_{2} \rds_{2}^{2}) \nonumber
\label{eqn:mmn_ngl_dfn}
\end{eqnarray}
The classical energy of a rigidly rotating dumbbell may be expressed
in terms of $\mmnrsh$ and $\mmnngl$
\begin{eqnarray}
\nrgrtt & = & {\textstyle\frac{1}{2}} \mmnrsh \frqngl^{2} \nonumber \\
\label{eqn:rgd_rtt_nrg}
\nrgrtt & = & \frac{\mmnngl^{2}}{2 \mmnrsh}
% fxm: are the following two classical analogues instructive or not?
% & = & \frqngl^{2} (\mss_{1} \rds_{1}^{2} + \mss_{2} \rds_{2}^{2}) \nonumber \\
% & = & \mss_{1} \frqngl \rds_{1}^{2} + \mss_{2} \frqngl \rds_{2}^{2} \nonumber
\end{eqnarray}

The quantum mechanical angular momentum operator, $\mmnnglopr$, 
quantizes the angular momentum component of the energy Hamiltonian.
The eigenvalues of the square of the angular momentum operator, 
$\mmnnglopr^{2}$, are 
\begin{eqnarray}
\mmnnglopr^{2} \wvfnc & = & \hbar^{2} \qntrtt (\qntrtt + 1) \wvfnc
\label{eqn:mmn_ngl_opr_dfn}
\end{eqnarray}
where $\wvfnc$ is the wavefunction and $\qntrtt$, an integer, is the 
\trmdfn{rotational quantum number} of the system. 
Substituting (\ref{eqn:mmn_ngl_opr_dfn}) into (\ref{eqn:rgd_rtt_nrg})
we obtain the quantum mechanical energy of a rigid rotator
\begin{eqnarray}
% ThS99 p. 114 (4.48) Ste94 p. 92 (3.9), GoY89 p. 81 (3.19)
\nrgrtt & = & \frac{\hbar^{2}}{2\mmnrsh} \qntrtt (\qntrtt + 1) 
\nonumber \\
& = & \frac{\cstplk^{2}}{4\mpi^{2}} \frac{1}{2\mmnrsh}
\frac{\cstspdlgt}{\cstspdlgt} \qntrtt (\qntrtt + 1) \nonumber \\
& = & \frac{\cstplk}{8\mpi^{2} \cstspdlgt \mmnrsh} 
\cstplk \cstspdlgt \qntrtt (\qntrtt + 1) \nonumber \\
& = & \cstplk \cstspdlgt \cstrtt \qntrtt (\qntrtt + 1)
\label{eqn:nrg_rtt_dfn}
\end{eqnarray}
where
\begin{eqnarray}
\cstrtt & = & \frac{\cstplk}{8\mpi^{2} \cstspdlgt \mmnrsh}
\label{eqn:cst_rtt_dfn}
\end{eqnarray}
$\cstrtt$ is the \trmdfn{rotational constant} of the species.
$\cstrtt$ is purely a function of the atomic masses and geometry of
the species through $\mmnrsh$ (\ref{eqn:mmn_rsh_dfn}).
The $\vbrsbs$ subscript\footnote{$\cstrtt$ is not to be confused with
the Planck function, $\plkfrq$.} indicates that $\cstrtt$ is a
function of vibrational state.

Spectroscopists have adopted a variety of equivalent but usually
unintuitive notations to describe the lower (lower energy) and upper
(higher energy) states of transitions. 
The most common convention is that lower and upper state quantum
numbers are superscripted by a double prime ``$^{\prime\prime}$'' and  
a single prime ``$^{\prime}$'', respectively.
Thus lower and upper vibrational and rotational quantum states are
denoted by $(\qntvbrlwr,\qntrttlwr)$ and $(\qntvbrupr,\qntrttupr)$,
respectively.
Examples of equivalent representations of emission include
\begin{eqnarray}
\nu^{\prime} & \rightarrow & \nu \nonumber \\
\nu & \leftarrow & \nu^{\prime} \nonumber \\
(\qntvbrupr,\qntrttupr) & \rightarrow & (\qntvbrlwr,\qntrttlwr) +
\cstplk \frq \nonumber \\
(\qntvbrlwr,\qntrttlwr) + \cstplk \frq & \leftarrow &
(\qntvbrupr,\qntrttupr) 
\label{eqn:qnt_trn_dfn}
\end{eqnarray}

Due to destructive interference patterns between quantum mechanical
wavefunctions, not all conceivable vibrotational transitions are 
\trmdfn{allowed transitions}.
Rotational transitions are subject to the \trmidx{selection rule}
that 
\begin{eqnarray}
\Delta \qntrtt & = & \qntrttupr \pm \qntrttlwr = \pm 1
\label{eqn:slc_rtt_dfn}
\end{eqnarray}
Transitions which break selection rules like (\ref{eqn:slc_rtt_dfn})
are called \trmdfn{forbidden transitions}.

Applying this rule to (\ref{eqn:nrg_rtt_dfn}) we obtain the energy of
photons released in purely rotational emission
\begin{eqnarray}
\Delta \nrgrtt 
& = & 
\cstplk \cstspdlgt \cstrtt (\qntrtt + 1) (\qntrtt + 2)
- \cstplk \cstspdlgt \cstrtt \qntrtt (\qntrtt + 1) \nonumber \\
\cstplk \frq
& = & \cstplk \cstspdlgt \cstrtt [ \qntrtt^{2} + 3\qntrtt + 2
- ( \qntrtt^{2} + \qntrtt ) ] \nonumber \\
\frq
& = & \cstspdlgt \cstrtt ( 2 \qntrtt + 2 ) \nonumber \\
& = & 2 \cstspdlgt \cstrtt ( \qntrtt + 1 )
\label{eqn:frq_rtt_trn_dfn}
\end{eqnarray}
Thus lines in pure rotational bands are spaced linearly in frequency. 
Linear, symmetric molecules such as \Nd, \Od, and \COd\ have no
permanent dipole moments in their ground vibrational states because
the center of mass coincides with the center of charge.
These molecules therefore have no rotational bands in their ground
vibrational state.

The factor of $\cstspdlgt$ on the RHS of (\ref{eqn:frq_rtt_trn_dfn}) 
disappears if we work in SI wavenumber rather than frequency units.
Using $\wvn = \frq / \cstspdlgt$ results in 
\begin{eqnarray}
\wvn & = & \Delta \nrgrtt = 2 \cstrtt ( \qntrtt + 1 )
\label{eqn:wvn_rtt_trn_dfn}
\end{eqnarray}
This direct relation between wavenumber and quantum number is one
reason spectroscopists prefer to work in wavenumber space.

\subsubsection[Vibrational Transitions]{Vibrational Transitions}\label{sxn:vbr}
A molecule composed of $\nclnbr$ atoms has $\modvbrnbr$ vibrational
modes where
\begin{eqnarray}
\modvbrnbr & = & \left\{
\begin{array}{r@{\quad:\quad}ll}
3\nclnbr -3 & \mbox{Non-linear molecules} \\
3\nclnbr -2 & \mbox{Linear molecules}
\end{array} \right.
\label{eqn:mod_vbr_nbr_dfn}
\end{eqnarray}
Often modes are degenerate, and so have fewer distinct vibrational
quantum numbers than indicated by (\ref{eqn:mod_vbr_nbr_dfn}).
As shown in Table~\ref{tbl:lnr}, linear molecules include all
\trmidx{diatomic molecules} (\CO, \NO, \OH), as well as \COd\ and
\NdO. 
Non-linear molecules include \HdO, \Ot, \CHq, and \CFClt.
\begin{table}
\begin{minipage}{\hsize} % Minipage necessary for footnotes KoD95 p. 110 (4.10.4)
\renewcommand{\footnoterule}{\rule{\hsize}{0.0cm}\vspace{-0.0cm}} % KoD95 p. 111
\begin{center}
\caption[Mechanical Analogues of Important Gases]{\textbf{Mechanical
Analogues for Radiatively Important Atmospheric Gases}%
\footnote{Based on \cite{GoY89}, p.~81, \cite{RRG98}}%
\label{tbl:lnr}}
\vspace{\cpthdrhlnskp}
\begin{tabular}{ >{$}r<{$} l >{$\ch}l<{$} }
\hline \rule{0.0ex}{\hlntblhdrskp}% 
\mbox{Moments of Inertia} & Class & Members \\[0.0ex]
\hline \rule{0.0ex}{\hlntblntrskp}%
\mmnrsh_{\AAA} = 0, \mmnrsh_{\BBB} = \mmnrsh_{\CCC} \ne 0 &
\footnote{Diatomic molecules and linear polyatomic molecules with
integer \qntrtt}Linear & \COd,
\NdO, \CO, \NO, \HF, \HCl, \HBr, \HI, \OCS \\[1.0ex]
\mmnrsh_{\AAA} \ne 0, \mmnrsh_{\BBB} = \mmnrsh_{\CCC} \ne 0 & Symmetric top
& \CFClt, \NHt, \CHtCl, \CdHs, \SFs \\[1.0ex]
\mmnrsh_{\AAA} = \mmnrsh_{\BBB} = \mmnrsh_{\CCC} & Spherical top & \CHq \\[1.0ex]
\mmnrsh_{\AAA} \ne \mmnrsh_{\BBB} \ne \mmnrsh_{\CCC} & Asymmetric top &
\HdO, \Ot, \SOd, \NOd, \HNOt \\[1.0ex]
\hline
\end{tabular}
\end{center}
\end{minipage}
\end{table}

The equation of motion for the position $\psnvct$ of a classical
oscillator of mass $\mss$ oscillating with restoring force $\cstrst$
is 
\begin{eqnarray}
\mss \frac{\dfr^{2} \psnvct}{\dfr \tm^{2}} & = & -\cstrst \psnvct \nonumber \\
\psnvctddot + \frac{\cstrst}{\mss} & = & 0 \nonumber \\
\psnvct & = & \AAA \sin \frqngl \tm + \BBB \cos \frqngl \tm \nonumber \\
\frqngl & = & 2\mpi \frq = \sqrt{\frac{\cstrst}{\mss}} \nonumber \\
\frq & = & \frac{1}{2\mpi} \sqrt{\frac{\cstrst}{\mss}}
\label{eqn:frq_cls_dfn}
\end{eqnarray}
where $\cstrst$ is the constant of the linear restoring force,
known as the \trmdfn{spring constant} or the \trmdfn{bond strength}.
Since $\cstrst$ is the restoring force divided by the displacement
distance, its SI units are \nxm\ (Newtons per meter) and its CGS
units are \dxcm\ (dynes per centimeter).
The spring constant $\cstrst$ depends only on the total particle mass, 
not on the mass distribution within the particle.
The solution to a two particle sysytem where particle equilibrium
separation is $\psnvct$ is identical to (\ref{eqn:frq_cls_dfn}) with
the mass $\mss$ replaced by the reduced mass $\mssrdc$
(\ref{eqn:mss_rdc_dfn})  
\begin{eqnarray}
\frq & = & \frac{1}{2\mpi} 
\sqrt{\frac{\cstrst(\mss_{1} + \mss_{2})}{\mss_{1} \mss_{2}}}
\label{eqn:frq_cls_dfn2}
\end{eqnarray}
The bond strength $\cstrst$ depends on the exact arrangement of 
nuclei within a molecule.
For diatomic molecules $\cstrst$ approximately fits the following
progression: 
Single bond, double bond, and triple bond diatomic molecules have
$\cstrst$ of  approximately $500$, $1000$, and $1500$\,\nxm
($5 \times 10^{5}$, $10 \times 10^{5}$, and $15 \times 10^{5}$\,\dxcm)
respectively.

\csznote{
% Where did this CO example come from?
% fxm: received from Lararos Oreopoulos 20060918
I am writing to you about a specific example in your book for which I am a
little bit puzzled regarding the numbers at which you arrive. Specifically
(in the version I have at least), in page 84 you calculate the fundamental
frequency of CO. You arrive at an answer 1120 cm-1 which contradicts the
literature value of ~2150 cm-1. I think the problem lies with the bond
strength of 500 Nm-1 you are using. I have found a value of ~1900 Nm-1 for
the triple (not single) bond of CO. This value will give you approximately
the published fundamental frequency. Please let me know if I have missed
something. } % end csznote
Let us consider the aborption spectrum due to stretching of
the \trmidx{carbon monoxide} molecule.
\trmidx{\CO} is approximately a triply bonded diatomic molecule with 
a bond strength of about 1900\,\nxm.
We will use (\ref{eqn:frq_cls_dfn2}) to predict the fundamental
frequency $\frq_{\CO}$ of the absorption spectrum.
We shall include the details of the calculation to demonstrate
the equivalence of the results obtained using SI and CGS units.
\begin{eqnarray}
\frq_{\CO} & \approx & 
\frac{1}{2\mpi} 
\sqrt{ \frac{ 1900\,\mbox{\nxm} \times (12.0 \times 10^{-3} + 16.0
\times 10^{-3})\,\mbox{\kgxmol}}
{12.0 \times 10^{-3}\,\mbox{\kgxmol} \times 16.0 \times
10^{-3}\,\mbox{\kgxmol} } } \nonumber \\ 
& \approx & 
\frac{1}{6.28}
\sqrt{ \frac{ 53.2\,\mbox{\nxm} }
{192 \times 10^{-6}\,\mbox{\kgxmol} } } \nonumber 
\approx
0.160
\sqrt{ 2.77 \times 10^{5} \frac{\mbox{kg m}}{\mbox{s$^{2}$}} \times 
\frac{\mbox{1}}{\mbox{m}} \times \frac{\cstAvagadro}{\mbox{kg}} }
\nonumber \\
\frq_{\CO} & \approx &
84.2 \sqrt{ \frac{1}{\mbox{s$^{2}$}}
\times \frac{6.02 \times 10^{23}}{1} } \approx
65.3 \times 10^{12}\,\mbox{\xs} = 65.3\,\mbox{\Thz} \nonumber \\
\wvl_{\CO} & = & \frac{\cstspdlgt}{\frq_{\CO}} \approx 
\frac{3.00 \times 10^{8}\,\mbox{\mxs}}{65.3 \times 10^{12}\,\mbox{\xs}}
= 4.59 \times 10^{-6}\,\mbox{m} = 4.59\,\mbox{\um} \nonumber \\
\wvn_{\CO} & = & \frac{\frq_{\CO}}{100\cstspdlgt}
\approx \frac{65.3 \times 10^{12}\,\mbox{\xs}}{100\,\mbox{\cmxm} 
\times 3.00 \times 10^{8}\,\mbox{\mxs}}
\approx 2182\,\xcm
\label{eqn:wvn_CO_SI}
\end{eqnarray}
The conversion of $\frq_{\CO}$ to $\wvn_{\CO}$ and $\wvl_{\CO}$
is straightforward, and shows that the stretching mode of the \CO\
molecule produces an absorption spectrum in the infrared. 
For pedagogical reasons, let us repeat the derivation of $\frq_{\CO}$
using CGS units to demonstrate that neither system is clearly superior 
in terms of its simplicity or ease of manipulation.
\begin{eqnarray}
\frq_{\CO} & \approx & 
\frac{1}{2\mpi} 
\sqrt{ \frac{ 1.9 \times 10^{6}\,\mbox{\dxcm} \times (12.0 + 16.0)\,\mbox{\gxmol}}
{12.0\,\mbox{\gxmol} \times 16.0\,\mbox{\gxmol} } } \nonumber \\
& \approx & 
\frac{1}{6.28}
\sqrt{ \frac{ 53.2 \times 10^{6}\,\mbox{\dxcm} }
{192\,\mbox{\gxmol} } } \nonumber 
\approx
0.160
\sqrt{ 27.7 \times 10^{4} \frac{\mbox{g\,cm}}{\mbox{s$^{2}$}} \times 
\frac{\mbox{1}}{\mbox{cm}} \times \frac{\cstAvagadro}{\mbox{g}} }
\nonumber
\label{eqn:wvn_CO_CGS}
\end{eqnarray}
and the rest of the computation is identical to (\ref{eqn:wvn_CO_SI}).
It appears that CGS units, historically the system of choice for
radiative transfer, are slightly more simple to work with than SI
units because of the CGS-centric definition of atomic weights.
However, in this era of automated computations, the slight advantage
of CGS does not appear to outweigh the usefulness of consistently
using SI units (\ref{eqn:wvn_CO_SI}).
All other things being equal, we recommend using SI units wherever
practical in radiative transfer, a convention we follow in this text.

The vibrational energy of a quantum harmonic oscillator is
\begin{eqnarray}
\nrgvbr & = & \sum_{\modvbridx=1}^{\modvbrnbr} 
\cstplk \frqmod
(\qntvbrmod + {\textstyle\frac{1}{2}})
\label{eqn:nrg_vbr_dfn}
\end{eqnarray}
where 
$\frqmod$ is the frequency of the $\modvbridx$th mode, 
$\cstplk \frqmod$ is the \trmdfn{vibrational constant} of the mode,  
and $\qntvbrmod$ is the \trmdfn{vibrational quantum number} of the
mode. 
Like other quantum numbers, $\qntvbr$ is an integer 
$\qntvbr \in [0, 1, 2, \ldots]$.
The term of $\frac{1}{2}$ in (\ref{eqn:nrg_vbr_dfn}) is called the
\trmdfn{zero-point energy}.
The \trmdfn{vibrational ground state} of a molecule is reached when 
$\qntvbrmod = 0$ for all $\modvbrnbr$ vibrational modes.
The linear relation between $\qntvbrmod$ and $\nrgvbr$
(\ref{eqn:nrg_vbr_dfn}) is an approximation.
In reality \trmdfn{anharmonicities} cause oscillations to deviate from
linear behavior.
This causes vibrational energy levels to become more closely spaced   
as $\qntvbrmod$ increases.

Allowed transitions must match applicable \trmdfn{selection rules}.
The selection rule for vibrational transitions is
\begin{eqnarray}
\Delta \qntvbr & = & \qntvbrupr \pm \qntvbrlwr = \pm 1
\label{eqn:slc_vbr_dfn}
\end{eqnarray}
The \trmdfn{fundamental transition} refers to transitions between the
first excited state ($\qntvbr = 1$) and the vibrational ground state
($\qntvbr = 0$).
Using $\Delta \qntvbr = \pm 1$ in (\ref{eqn:nrg_vbr_dfn}) leads to
\begin{eqnarray}
\Delta \nrgvbr & = & 
\cstplk \frqmod (\qntvbr + 1 + {\textstyle\frac{1}{2}})
- \cstplk \frqmod (\qntvbr + {\textstyle\frac{1}{2}}) \nonumber \\
& = & 
\cstplk [ \frqmod ( \qntvbr + {\textstyle\frac{3}{2}} ) 
- \frqmod (\qntvbr + {\textstyle\frac{1}{2}}) ] \nonumber \\
& = & \cstplk \frqmod
\label{eqn:nrg_vbr_trn_dfn}
\end{eqnarray}
For simpler molecules such as diatomic molecules, the relation between
the type of vibrational mode $\modvbridx$ (e.g., stretching, bending)
and the fundamental energy of the mode (\ref{eqn:nrg_vbr_trn_dfn}) may
depend on only a few parameters.

\subsubsection[Isotopic Lines]{Isotopic Lines}\label{sxn:iso_ln}

The quantum mechanical analogue of (\ref{eqn:frq_cls_dfn}) is
\begin{eqnarray}
\frqmod & = & \frac{1}{2\mpi}\sqrt{\frac{\cstrstmod}{\mssrdc}} 
\label{eqn:frq_qnt_dfn}
\end{eqnarray}
where now $\cstrstmod$ is the force constant of vibrational mode
$\modvbridx$ which depends on the distribution of charge in the molecule but
is completely independent of the mass distribution, and thus
independent of isotope. 
Two distinct isotopes $\iii$ and $\jjj$ of the same species
will thus have differing equilibrium frequencies $\frq_{\iii}$
and $\frq_{\jjj}$ for the same vibrational mode:
\begin{eqnarray}
% GoY89 p. 80 (3.18)
\frac{\frq_{\iii}}{\frq_{\jjj}} & = & \sqrt{\frac{\mssrdc_{\jjj}}{\mssrdc_{\iii}}}
\label{eqn:frq_iso_dfn}
\end{eqnarray}
For instance, the frequency shift for $\chm{{}^{13}C{}^{16}O}$
relative to the 2140\,\xcm\ band of $\chm{{}^{12}C{}^{16}O}$ is
47.7\,\xcm. % GoY89 p. 80 

\subsubsection[Combination Bands]{Combination Bands}\label{sxn:cmb}
Transition selection rules for many important molecules (including all
diatomic molecules), require that $\Delta \qntvbr \ne 0$ and $\Delta
\qntrtt \ne 0$, i.e., vibrational transitions must occur
simultaneously with rotational transitions.
Thus simultaneous transitions of both vibrational and rotational
states are the norm, not the exception, in the atmosphere.
These simultaneous transitions form what are called
\trmdfn{combination bands} or \trmdfn{vibration-rotation bands}.
The energy of such transitions is obtained by subtracting the 
lower energy state from the upper energy state
\begin{eqnarray}
\wvn_{\RRR} & = & \wvn_{\modvbridx} + 
\cstrttupr \qntrttupr (\qntrttupr + 1) -
\cstrttlwr \qntrttlwr (\qntrttlwr + 1) 
\qntrtt^{2}
\label{eqn:cmb_dfn}
\end{eqnarray}
where $\wvn_{\modvbridx}$ is the energy of the pure vibrational
transition (\ref{eqn:nrg_vbr_dfn}).
We have not cancelled like terms, as in the idealized case of
(\ref{eqn:frq_rtt_trn_dfn}), since $\cstrttupr \ne \cstrttlwr$ in
combination bands. 
Substituting $\Delta \qntrtt = \pm 1$ into (\ref{eqn:cmb_dfn}) leads
to the energy spacings between the \trmdfn{\RRR-branch} transitions 
and the \trmdfn{\PPP-branch} transitions, respectively.
\begin{eqnarray}
\wvn_{\RRR} & = & \wvn_{\modvbridx} + 2 \cstrttupr + 
(3\cstrttupr - \cstrttlwr) \qntrtt + (\cstrttupr - \cstrttlwr)
\qntrtt^{2} \\
\wvn_{\PPP} & = & \wvn_{\modvbridx} - (\cstrttupr + \cstrttlwr) \qntrtt 
+ (\cstrttupr - \cstrttlwr) \qntrtt^{2} 
\label{eqn:PR_dfn}
\end{eqnarray}

There may be a number of absorption bands in addition to the
fundamental bands of a molecule.
These include \trmdfn{overtone bands}, \trmdfn{combination bands},
\trmdfn{vibration-rotation bands}, and 
\trmdfn{harmonic coupling bands}. 
Overtone bands occur at frequencies which are multiples of the
fundamental frequencies. 
Combination bands are due to the interaction of two fundamental
vibration bands.
The combined frequency \ldots
Vibration-rotation bands have already been discussed.
Harmonic coupling bands occur when interactions among closely spaced
oscillation frequencies produces distinct, unexpected bands.
This is relatively uncommon.
% fxm: Develop this section until we can parse at least
% quantum configurations found in O2X work, like 
% the 1.27-\um, {$a ^{1}\Delta_g-X ^{3}\Sigma_g^-, v = 0-0$} band of {\Od}

\subsection[Partition Functions]{Partition Functions}\label{sxn:prt_fnc}
The \trmdfn{statistical weight} $\sttwgt$ measures the number of
available states in a given quantum configuration.
Quantum theory teaches us that rotational energy levels are 
$2\qntrtt + 1$ degenerate owing to $2\qntrtt + 1$ indistinguishable
orientations of the component of angular momentum in a fixed direction
in space.
The statistical weight for rotational levels is therefore
\begin{eqnarray}
\sttwgtrtt & = & 2\qntrtt + 1
\label{eqn:stt_wgt_rtt_dfn}
\end{eqnarray}

In LTE, rotational states are populated according to \trmdfn{Boltzmann 
statistics}, so that the probability of occupancy is proportional
to $\me^{-\nrgrtt/\cstblt \tpt}$.
At temperature $\tpt$, the ratio of molecules in state $\qntrttupr$ to
those in state $\qntrttlwr$ is therefore
\begin{eqnarray}
\frac{\cnc(\qntrttupr)}{\cnc(\qntrttlwr)} & \equiv & 
\frac{\sttwgt_{\qntrttupr}}{\sttwgt_{\qntrttlwr}}
\frac{\me^{-\nrgrtt(\qntrttupr)/\cstblt\tpt}}{\me^{-\nrgrtt(\qntrttlwr)/\cstblt\tpt}}
\nonumber \\ 
& = & 
\frac{2\qntrttupr + 1}{2\qntrttlwr + 1}
\exp\! \left\{ 
-\frac{\cstplk \cstspdlgt \cstrtt}{\cstblt \tpt}
[ \qntrttupr ( \qntrttupr + 1 ) - \qntrttlwr ( \qntrttlwr + 1 )]
\right\}
\label{eqn:rlt_ppl_rtt_dfn}
\end{eqnarray}
where now $\qntrttupr$ and $\qntrttlwr$ refer to any values of
$\qntrtt$, not just those satisfying selection rules.
To examine the fractional abundance of molecules in state $\qntrtt$ 
relative to those in all other rotational states, we must sum 
(\ref{eqn:rlt_ppl_rtt_dfn}) over all $\qntrttlwr$ while holding
$\qntrttupr = \qntrtt$ fixed.
This leads to 
\begin{eqnarray}
\frac{\cnc(\qntrtt)}{\cnc} & = & 
\frac{2\qntrtt + 1}{\prtrtt}
\exp\! \left[
-\frac{\cstplk \cstspdlgt \cstrtt}{\cstblt \tpt}
\qntrtt ( \qntrtt + 1 ) \right]
\label{eqn:ppl_frc_rtt_dfn}
\end{eqnarray}
where we have defined the \trmdfn{rotational partition function}
$\prtrtt$ as the total, probability-weighted number of available
rotation states in the system
\begin{eqnarray}
\prtrtt & = & 
\sum_{\qntrtt = 1}^{\infty} (2\qntrtt + 1)
\exp\! \left[
-\frac{\cstplk \cstspdlgt \cstrtt}{\cstblt \tpt}
\qntrtt ( \qntrtt + 1 ) \right]
\label{eqn:prt_rtt_dfn}
\end{eqnarray}
% Shu91 p. 357
For atmospheric cases of interest, 
$\cstblt \tpt \gg \cstplk \cstspdlgt \cstrtt$
so the exponential term becomes vanishingly small.
To express and evaluate (\ref{eqn:prt_rtt_dfn}) as an integral,
we make a change of variables 
\begin{eqnarray}
\xxx & = & \frac{\cstplk \cstspdlgt \cstrtt}{\cstblt \tpt} \qntrtt (\qntrtt + 1) \nonumber \\
\dfr\xxx & = & \frac{\cstplk \cstspdlgt \cstrtt}{\cstblt \tpt} (2\qntrtt + 1) \dfr\qntrtt \nonumber \\
(2\qntrtt + 1) \dfr\qntrtt & = & \frac{\cstblt \tpt}{\cstplk \cstspdlgt \cstrtt} \dfr\xxx
\label{eqn:cov_J}
\end{eqnarray}
This maps $\qntrtt \in [0,\infty)$ to $\xxx \in [0,\infty)$.
Thus $\xxx$ is considered a continuous function of $\qntrtt$.
% fxm: this is hand-wavy, what really allows conversion of sum to integral?
Substituting (\ref{eqn:cov_J}) into (\ref{eqn:prt_rtt_dfn}) results in 
\begin{eqnarray}
\prtrtt & \approx & 
\int_{0}^{\infty} (2\qntrtt + 1)
\exp\! \left[
-\frac{\cstplk \cstspdlgt \cstrtt}{\cstblt \tpt}
\qntrtt ( \qntrtt + 1 ) \right] \,\dfr\qntrtt \nonumber \\
& = & 
\int_{0}^{\infty} 
\frac{\cstblt \tpt}{\cstplk \cstspdlgt \cstrtt}
\me^{-\xxx} \,\dfr\xxx \nonumber \\
& = & 
\frac{\cstblt \tpt}{\cstplk \cstspdlgt \cstrtt} \left[ -\me^{-\xxx} \right]_{0}^{\infty} \nonumber \\
& = & 
\frac{\cstblt \tpt}{\cstplk \cstspdlgt \cstrtt} [ - 0 - ( - 1 ) ] \nonumber \\
& = & 
\frac{\cstblt \tpt}{\cstplk \cstspdlgt \cstrtt}
\label{eqn:prt_rtt_val}
\end{eqnarray}
Substituting (\ref{eqn:prt_rtt_val}) into (\ref{eqn:ppl_frc_rtt_dfn})
we obtain
\begin{eqnarray}
\frac{\cnc(\qntrtt)}{\cnc} & = & 
\frac{\cstplk \cstspdlgt \cstrtt(2\qntrtt + 1)}{\cstblt \tpt}
\exp\! \left[
-\frac{\cstplk \cstspdlgt \cstrtt}{\cstblt \tpt}
\qntrtt ( \qntrtt + 1 ) \right]
\label{eqn:ppl_frc_rtt_dfn2}
\end{eqnarray}
Examination of (\ref{eqn:ppl_frc_rtt_dfn2}) shows that
$\cnc(\qntrtt)/\cnc$ is small for $\qntrtt \rightarrow 0$ and 
for $\qntrtt \rightarrow \infty$ and maximal in between.
The derivation of (\ref{eqn:ppl_frc_rtt_dfn2}) neglected any
vibrational-dependence of $\cstrtt$.

The \trmdfn{vibrational partition function} $\prtvbr$ is defined by a
procedure analogous to
(\ref{eqn:rlt_ppl_rtt_dfn})--(\ref{eqn:prt_rtt_val}).
% GoY89 p. 85
Contrary to rotational states (\ref{eqn:stt_wgt_rtt_dfn}), vibrational
states all have equal statistical weights
\begin{eqnarray}
\sttwgtvbr & = & \sttwgt_{0}
\label{eqn:stt_wgt_vbr_dfn}
\end{eqnarray}

In LTE, \trmdfn{Boltzmann's Law}, (\ref{eqn:stt_wgt_vbr_dfn}) and
(\ref{eqn:nrg_vbr_dfn}) lead to
\begin{eqnarray}
\frac{\cnc(\qntvbrupr)}{\cnc(\qntvbrlwr)} & \equiv & 
\frac{\sttwgt_{\qntvbrupr}}{\sttwgt_{\qntvbrlwr}}
\frac{\me^{-\nrgvbr(\qntvbrupr)/\cstblt\tpt}}{\me^{-\nrgvbr(\qntvbrlwr)/\cstblt\tpt}}
\nonumber \\ 
& = & 
\frac{\sttwgt_{0}}{\sttwgt_{0}}
\exp\! \left\{
- \left[ \frac{\cstplk \frqnot (\qntvbrupr + \frac{1}{2})}{\cstblt \tpt}
-\frac{\cstplk \frqnot (\qntvbrlwr + \frac{1}{2})}{\cstblt \tpt}
\right] \right\} \nonumber \\
& = & 
\exp\! \left[
-\frac{\cstplk \frqnot}{\cstblt \tpt}(\qntvbrupr - \qntvbrlwr)
\right]
\label{eqn:rlt_ppl_vbr_dfn}
\end{eqnarray}
Notice that the statistical weight and the \trmidx{zero-point energy} 
contribution of $\cstplk \frqnot /2$ factor out of the relative
abundance of molecules in a given vibrational level
(\ref{eqn:rlt_ppl_vbr_dfn}).
Nevertheless, the zero-point energy does contribute to the total
internal energy of the system and so is properly included in the 
definition of the vibrational partition function $\prtvbr$
\begin{eqnarray}
\prtvbr & = & 
\sum_{\qntvbr = 0}^{\infty} 
\exp\! \left[
-\frac{\cstplk \frqnot}{\cstblt \tpt} ( \qntvbr + {\textstyle\frac{1}{2}} )
\right] \nonumber \\
& = & 
\me^{-\cstplk \frqnot/2\cstblt \tpt}
\sum_{\qntvbr = 0}^{\infty} 
\left( \me^{-\cstplk \frqnot / \cstblt \tpt} \right)^{\qntvbr}
\nonumber \\
& = & 
\frac{\me^{-\cstplk \frqnot/2\cstblt \tpt}}
{1-\me^{-\cstplk \frqnot / \cstblt \tpt}}
\label{eqn:prt_vbr_val}
\end{eqnarray}
where in the last step we used the solution for an infinite
geometric series whose ratio between terms $\rrr < 1$, 
\begin{eqnarray}
\sum_{\nnn=0}^{\infty} \aaa_{0} \rrr^{\nnn} & = &
\frac{\aaa_{0}}{1 - \rrr}
\label{eqn:gmt_srs_dfn}
\end{eqnarray}

The fractional abundance of molecules in vibrational state
$\qntvbr$ is thus
\begin{eqnarray}
\frac{\cnc(\qntvbr)}{\cnc} 
& = &
\frac{1}{\prtvbr}
\exp\! \left[
-\frac{\cstplk \frqnot}{\cstblt \tpt} (\qntvbr + {\textstyle\frac{1}{2}}) \right]
\nonumber \\
& = &
\frac{1-\me^{-\cstplk \frqnot / \cstblt \tpt}}
{\me^{-\cstplk \frqnot/2\cstblt \tpt}}
\me^{-\cstplk \frqnot/2\cstblt \tpt}
\me^{-\qntvbr \cstplk \frqnot /\cstblt \tpt} \nonumber \\
& = &
(1-\me^{-\cstplk \frqnot / \cstblt \tpt})
\me^{-\qntvbr \cstplk \frqnot /\cstblt \tpt}
\label{eqn:ppl_frc_vbr_dfn}
\end{eqnarray}
Vibrational levels are so widely spaced that most molecules in the
lower atmosphere are in $\qntvbr = 0$ or $\qntvbr = 1$ states.

It is common to approximate the total internal partition function of 
atmospheric transitions $\prtfnc$ as the product of the vibrational and 
the rotational partition functions.
Using (\ref{eqn:prt_vbr_val}) and (\ref{eqn:prt_rtt_val}) we
obtain 
\begin{eqnarray}
\prtfnc & \approx & 
\prtvbr \prtrtt \nonumber \\
& \approx & 
\frac{\me^{-\cstplk \frqnot/2\cstblt \tpt}}
{1-\me^{-\cstplk \frqnot / \cstblt \tpt}}
\frac{\cstblt \tpt}{\cstplk \cstspdlgt \cstrtt} \nonumber
\end{eqnarray}
In the high temperature limit where $\cstblt \tpt \gg \cstplk \frqnot$
then we may use the behavior $\me^{-\xxx} \approx 1 - \xxx$ so this
becomes 
\begin{eqnarray}
\prtfnc & \approx & 
\frac{1 - \frac{\cstplk \frqnot}{2\cstblt \tpt}}
{1 - 1 + \frac{\cstplk \frqnot}{\cstblt \tpt} }
\frac{\cstblt \tpt}{\cstplk \cstspdlgt \cstrtt} \nonumber \\
& \approx & 
\frac{\frac{2\cstblt \tpt - \cstplk \frqnot}{2\cstblt \tpt}}
{\frac{\cstplk \frqnot}{\cstblt \tpt} } 
\frac{\cstblt \tpt}{\cstplk \cstspdlgt \cstrtt} \nonumber \\
& \approx & 
\frac{2\cstblt \tpt - \cstplk \frqnot}{2 \cstplk \frqnot}
\frac{\cstblt \tpt}{\cstplk \cstspdlgt \cstrtt} \nonumber
\label{eqn:prt_fnc_dfn}
\end{eqnarray}

\subsection[Dipole Radiation]{Dipole Radiation}\label{sxn:dpl}
The probability $\prbif$ of absorption of a photon leading to a
change in molecular state from state $\iii$ to state $\fff$ is
\begin{eqnarray}
% Shu91 p. 230 (22.33)
\prbif & = & \tm 
\left( \frac{\chglct^{2}}{\cstplk \cstspdlgt^{3} \msslct^{2}} \right)
\sum_{\alpha = 1}^{2} \oint
\left[ \frqngl \pplphtplr(\wvnbrvct) 
|\langle \wvfncf | \me^{\mi \wvnbrvct \cdot \psn} \eeeplr(\wvnbrvcthat)
\cdot \mmnvct | \wvfnci \rangle |^{2} \right]_{fi} \,\dfr\ngl
\label{eqn:abs_prb}
\end{eqnarray}
The \trmdfn{matrix element} 
$\langle \wvfncf | \me^{\mi \wvnbrvct \cdot \psn}
\eeeplr(\wvnbrvcthat) \cdot \mmnvct | \wvfnci \rangle$
determines the absorption probability.
The \trmdfn{dipole approximation} retains only the first term in the
expansion
\begin{eqnarray}
% Shu91 p. 233 (23.2)
\me^{\mi \wvnbrvct \cdot \psn} & = & 1 + \mi \wvnbrvct \cdot \psn +
\cdots \nonumber
\label{eqn:dpl_apx}
\end{eqnarray}
so that the matrix element simplifies to
\begin{eqnarray}
% Shu91 p. 233 (23.3)
\langle \wvfncf | \me^{\mi \wvnbrvct \cdot \psn}
\eeeplr(\wvnbrvcthat) \cdot \mmnvct | \wvfnci \rangle
& = & \eeeplr \cdot \langle \wvfncf | \mmnvct | \wvfnci \rangle
\label{eqn:mtx_lmn}
\end{eqnarray}
This matrix element is related to the \trmidx{line strength} $\lnstr$
as follows \cite[][p.~709]{RRG98}.... % fxm finish this link...

\subsection[Two Level Atom]{Two Level Atom}\label{sxn:atm}
% RyL79 p. 27
Einstein created an elegant paradigm for analyzing the
interaction of matter and radiation.
His tool is called the \trmdfn{two level atom}.
It consists of an ensemble of molecules with two discrete energy
levels, $\nrgone$ and $\nrgtwo > \nrgone$.
The energy levels have statistical weights $\sttwgtone$ and
$\sttwgttwo$, respectively.
There are $\pplone$ molecules per unit volume in the state with
$\nrgone$, and $\ppltwo$ in the state with $\nrgtwo$.
The frequency of transition between the two levels is $\hnunot =
\nrgtwo - \nrgone$.

The five processes by which an idealized two-level molecule may change
state are
\begin{rxnarray}
% ThS99 p. (103)
\ppltwo & \yields^{\atwoone} & \pplone + \hnunot \\
\pplone + \hnunot & \yields^{\bonetwo} & \ppltwo \\
\ppltwo + \hnunot & \yields^{\btwoone} & \pplone + 2\hnunot \\
% ThS99 p. 98, p. 107 (4.37)
\ppltwo + \M + \KE & \yields^{\ctwoone} & \pplone + \M + \KEprm + \hnunot \\
\pplone + \M + \KE & \yields^{\conetwo} & \ppltwo + \M + \KEprm
\label{rxn:two_lvl_dfn}
\end{rxnarray}
where \M\ is a collision partner and \KE\ and \KEprm\ represent the
kinetic energy of the two molecule system before and after the
collision, respectively. 
The most frequent collision partners are, naturally, nitrogen and
oxygen molecules.

The Einstein coefficient $\atwoone$\,\xs\ is the transition
probability per unit time for \trmidx{spontaneous emission}.
Spontaneous emission, as its name implies, requires no external
stimulus.
Thus decay of excited molecules occurs even in the absence of a
radiation field.
Each emission reduces the population of excited molecules and
increases the population of ground state molecules by one.
Hence
\begin{eqnarray}
% ThS99 p. (), Shu91 p. ()
\frac{\dfr \pplone}{\dfr\tm} 
& = & \atwoone \ppltwo
\label{eqn:dfeq_a21}
\end{eqnarray}

The Einstein coefficient $\bonetwo$ determines the rate of radiative
absorption. 
$\bonetwo$ is also called the Einstein coefficient for
\trmdfn{stimulated absorption}.
More specifically, $\bonetwo$ is the proportionality constant
between the mean intensity of the radiation field $\ntnmnfrq$ and the
probability per unit time of an absorption occuring.
The transition probability per unit time for radiative absorption
is $\bonetwo \ntnmnfrq$.
$\bonetwo$ has units of [\xs(\wxmSsrhz)$^{-1}$].
The rate of radiative absorptions per unit time per unit volume is 
\begin{eqnarray}
% ThS99 p. (104), Shu91 p. ()
\frac{\dfr \ppltwo}{\dfr\tm} 
& = & \pplone 
\int_{0}^{\infty} \int_{\ngl} 
\xsxabsoffrq \frac{\ntnfrq}{\hnu}
\,\dfr\ngl \,\dfr\frq
\nonumber \\
& = & 4\mpi \pplone 
\int_{0}^{\infty} \xsxabsoffrq \frac{\ntnmnfrq}{\hnu} \,\dfr\frq
\label{eqn:dfeq_b12}
\end{eqnarray}
where $\xsxabsoffrq$ refers to absorption cross-section of the
single transition available in the two-level atom.
Comparing the final two expressions we see that 
\begin{eqnarray}
\bonetwo & = & \frac{4\mpi \xsxabsoffrq}{\hnu \lnshpoffrq}
\label{eqn:b12_mcr_dfn}
\end{eqnarray}
$\bonetwo$ is intimately related to the \trmidx{line strength} $\lnstr$
of the molecular transition \cite[][p.~709]{RRG98}. 
In particular, $\bonetwo$ may be expressed in terms of the weighted
transition-moment squared.

The Einstein coefficient $\btwoone$ determines the rate of 
\trmidx{stimulated emission} or \trmidx{induced emission}.
$\btwoone \ntnmnfrq$ is transition probability per
unit time for stimulated emission.
$\btwoone$ has the same dimensions as $\bonetwo$.
\begin{eqnarray}
% ThS99 p. (), Shu91 p. ()
\frac{\dfr \pplone}{\dfr\tm} 
& = & \ppltwo
\int_{0}^{\infty} \int_{\ngl} 
\xsxabsstmoffrq \frac{\ntnfrq}{\hnu}
\,\dfr\ngl \,\dfr\frq
\nonumber \\
& = & 4\mpi \ppltwo
\int_{0}^{\infty} \xsxabsstmoffrq \frac{\ntnmnfrq}{\hnu} \,\dfr\frq
\label{eqn:dfeq_b21}
\end{eqnarray}
where $\xsxabsstmoffrq$ is the absorption cross-section for
stimulated emission.

Gaseous absorption cross-sections $\xsxabsoffrq$ are extremely narrow
and so the integrals over frequency domain in (\ref{eqn:dfeq_b12}) and
(\ref{eqn:dfeq_b21}) are determined by a small range of frequencies
about $\frqnot$, the line center frequency.
Exact descriptions of line broadening about $\frqnot$ are discussed in
\S\ref{sxn:lnshp_ntr}--\S\ref{sxn:lnshp_vgt}.
It is instructive to ignore the details of the finite shape of line
transitions for now and to assume the line absorption cross
section behaves as a delta-function with integrated cross section
$\aaaonetwo$ 
\begin{eqnarray}
% Shu91 p. 73 (8.23)
\xsxabsoffrq & \approx & \aaaonetwo \dltfncoffrqfrqnot \\
\aaaonetwo & = & \frac{\mpi \chglct^{2}}{\msslct \cstspdlgt}
\label{eqn:xsx_abs_dlt_dfn}
\end{eqnarray}
where $\msslct$ is the \trmdfn{electron rest mass},
$\chglct$ is the electron charge, and
$\oscstr$ is the \trmdfn{oscillator strength} of the transition.
With these definitions.

In thermodynamic equilibrium 
the number of transitions from state~1 to state~2 per unit time per unit volume 
equals 
the number of transitions from state~2 to state~1 per unit time per unit volume.
Combining (\ref{eqn:dfeq_a21}), (\ref{eqn:dfeq_b12}), and
(\ref{eqn:dfeq_b21}) we obtain 
\begin{eqnarray}
% RyL79 p. 29 (1.69)
\frac{\dfr \pplone}{\dfr\tm} & = & \frac{\dfr \ppltwo}{\dfr\tm}
\nonumber \\
\pplone \bonetwo \ntnmnfrq & = & \ppltwo \atwoone + \ppltwo \btwoone \ntnmnfrq
\label{eqn:_dfn}
\end{eqnarray}

\subsection[Line Strengths]{Line Strengths}\label{sxn:lnstr}
We repeat the definition of \trmidx{line strength}
(\ref{eqn:lnstr_dfn}) for convenience
\begin{equation}
\lnstr = \int_{0}^{\infty} \xsxabsoffrq \,\dfr\frq
\label{eqn:lnstr_dfn2}
\end{equation}
In practice, (\ref{eqn:lnstr_dfn2}) is either measured in the
laboratory or predicted from \textit{ab initio} methods.

\subsubsection[\acr{HITRAN}]{\acr{HITRAN}}\label{sxn:htrn}
The \acr{HITRAN} (``high-resolution transmission'') database provides the
parameters required to computed absorption coefficients for all
atmospheric transitions of interest. 
The \acr{HITRAN} database is defined by \cite{RRG98}, and its usage is
defined in their Appendix~A.

The idiosyncratic units of line parameters can be confusing.
Most \acr{HITRAN} data are provided in CGS units. 
For reference, Table~\ref{tbl:htrn} presents the equivalence between
the symbols used in \cite{RRG98}, their Appendix~A, and our symbols.
\begin{table}
\begin{minipage}{\hsize} % Minipage necessary for footnotes KoD95 p. 110 (4.10.4)
\renewcommand{\footnoterule}{\rule{\hsize}{0.0cm}\vspace{-0.0cm}} % KoD95 p. 111
\begin{center}
\caption[\acr{HITRAN} database]{\textbf{\acr{HITRAN} database}%
\footnote{Equivalance between symbols employed in this work and those
used by \cite{RRG98}, Appendix~A.}%
\footnote{\acr{HITRAN} data are tabulated at a temperature $\tptnot$
and pressure $\prsnot$ of $296$\,K and $101325.0$\,Pa, respectively.}% 
\label{tbl:htrn}}
\vspace{\cpthdrhlnskp}
\begin{tabular}{ >{$}c<{$} >{$}c<{$} l }
\hline \rule{0.0ex}{\hlntblhdrskp}% 
\mbox{This Text} & \mbox{\acr{HITRAN}} & Description \\[0.0ex]
\hline \rule{0.0ex}{\hlntblntrskp}%
\lnstrlnnot & S_{\eta\eta^{\prime}} & Line strength \\[1.0ex]
\hwhmlnnot & \gamma_{\mbox{\scriptsize air}} & Pressure-broadened HWHM \\[1.0ex]
& \gamma_{\mbox{\scriptsize self}} & Self-broadened HWHM \\[1.0ex]
\wvnnot & \nu_{\eta\eta^{\prime}} & Transition frequency \\[1.0ex]
\wvnlwrln & E_{\nu} & Lower state energy \\[1.0ex]
\prsbrdtptdpnxpn & n & Exponent in temperature-dependence of pressure-broadened halfwidth \\[1.0ex] 
\prsshf & \delta & Air-broadened pressure shift of transition frequency \wvnnot \\[1.0ex]
\wvn & \nu & Frequency in wavenumbers \\[1.0ex]
%& & \\[1.0ex]
\hline
\end{tabular}
\end{center}
\end{minipage}
\end{table}

The \trmidx{line strength} $\lnstrlnnot$ (\ref{eqn:lnstr_dfn}) is
tabulated in \acr{HITRAN} in CGS units of wavenumbers times centimeter
squared per molecule, \wvncmSxmlc, often written \cmxatm. 
Line strengths for weak lines can be less than
$10^{-36}$\,\wvncmSxmlc, i.e., unrepresentable as IEEE single
precision (4~byte) floating point numbers. 
Thus it is more robust to work with $\lnstrlnnot$ units of per mole
(rather than per molecule) in applications where using IEEE double
precision (8~byte) floating point numbers is not an option either due
to memory limitations or computational overhead. 
Following are the conversions from the tabulated values of
$\lnstrlnnot$ in \wvncmSxmlc\ to more useful units, including 
\wvnmSxmlc, \wvnmSxmol, and to fully SI units of \frqmSxmol\ and 
\wvlmSxmol.
\begin{subequations}
\label{eqn:lnstr_units}
\begin{align}
\label{eqn:lnstr_wvn_mlc}
\lnstrlnnot [\mbox{\wvnmSxmlc}] 
& = \lnstrlnnot [\mbox{\wvncmSxmlc}] \times 10^{-4} \\
\label{eqn:lnstr_wvn}
\lnstrlnnot [\mbox{\wvnmSxmol}] 
& = \lnstrlnnot [\mbox{\wvncmSxmlc}] \times 10^{-4} \times \cstAvagadro \\
\label{eqn:lnstr_frq}
\lnstrlnnot [\mbox{\frqmSxmol}] 
& = \lnstrlnnot [\mbox{\wvncmSxmlc}] \times 10^{-4} \times 
100 \cstspdlgt \times \cstAvagadro \\
\label{eqn:lnstr_wvl}
\lnstrlnnot [\mbox{\wvlmSxmol}] 
& = \lnstrlnnot [\mbox{\wvncmSxmlc}] \times 10^{-4} \times
100 \wvl^{2} \times \cstAvagadro
\end{align}
\end{subequations} 
where $\cstAvagadro$ is \trmdfn{Avagadro's number}.
Equation~(\ref{eqn:lnstr_wvl}) is uncertain.

The energy of the lower state of each transition, $\nrglwrln$, is
required to determine the relative population of that state available
for transitions. 
\acr{HITRAN} supplies this energy in wavenumber units in the tabulated
parameter $\wvnlwrln$ which is simply related to $\nrglwrln$ by
\begin{eqnarray}
\nrglwrln & = & \cstplk \cstspdlgt \wvnlwrln
\label{eqn:nrg_lwr_ln_dfn}
\end{eqnarray}
Of course, consistent with \acr{HITRAN} and spectroscopic conventions,
$\wvnlwrln$ is archived in CGS, \xcm.

All \acr{HITRAN} data have been scaled from the conditions of the experiment
to a temperature and pressure of $296$\,K and $1013.25$\,mb, respectively.
Thus line strengths must typically be scaled from \acr{HITRAN}-standard
conditions, $\lnstrlnnot(\tptnot)$, to the atmospheric conditions of
interest, $\lnstrln(\tpt)$.
\begin{eqnarray}
% ThS99 p. 121 (4.63), RRG98 p. 710 (A11)
\lnstrln(\tpt) & = & \lnstrlnnot(\tptnot)
\frac{\prtfnc(\tptnot)}{\prtfnc(\tpt)}
\frac{\me^{-\cstplk \cstspdlgt \wvn_{\iii} / \cstblt \tpt }}
{\me^{-\cstplk \cstspdlgt \wvn_{\iii} / \cstblt \tptnot}}
\frac{1 - \me^{-\cstplk \cstspdlgt \wvn_{\iii \jjj} / \cstblt \tpt }}
{1 - \me^{-\cstplk \cstspdlgt \wvn_{\iii \jjj} / \cstblt \tptnot }}
\label{eqn:lnstr_scl}
\end{eqnarray}
where $\prtfnc$ is the total internal partition function.
The second, third and fourth factors on the RHS of
(\ref{eqn:lnstr_scl}) are the temperature-dependent scalings of,
respectively, the total partition function, the Boltzmann factor for
the lower state, and the stimulated emission factor.
The scaling presented in (\ref{eqn:lnstr_scl}) is general and contains 
no approximation. 

The classical approximation \cite[]{RoY89,ThS99} assumes that
$\prtfnc$ is the product of the rotational and the vibrational
partition functions 
\begin{eqnarray}
\prtfnc(\tpt) & \approx & \prtvbr(\tpt) \prtrtt(\tpt)
\label{eqn:prt_fnc_sov_dfn}
\end{eqnarray}
In this approximation, $\prtvbr$ (\ref{eqn:prt_vbr_val}) and $\prtrtt$ 
(\ref{eqn:prt_rtt_val}) are treated as independent components of
$\prtfnc$.
This allows $\prtfnc$ to be computed as discussed in
\S\ref{sxn:prt_fnc}. 
Most high precision calculation improve this approach by using more
exact methods described in \cite{GHR90} and \cite{GKH00}.

For many applications, a simplified version of (\ref{eqn:lnstr_scl})
is employed
\begin{eqnarray}
% ThS99 p. 121 (4.64)
\lnstrln(\tpt) & = & \lnstrlnnot(\tptnot)
\left( \frac{\tptnot}{\tpt} \right)^{\mmm}
\exp\left[ -\frac{\nrglwrln}{\cstblt} 
\left( \frac{1}{\tpt} - \frac{1}{\tptnot} \right) \right] \nonumber \\
& = & \lnstrlnnot(\tptnot)
\left( \frac{\tptnot}{\tpt} \right)^{\mmm}
\exp\left[ \frac{\nrglwrln}{\cstblt} 
\left( \frac{1}{\tptnot} - \frac{1}{\tpt} \right) \right]
\label{eqn:lnstr_apx}
\end{eqnarray}
where $\mmm$ is an empirical parameter of order unity and 
The parameterization (\ref{eqn:lnstr_apx}) incorporates the
temperature dependence of both partition functions and the stimulated 
emission factor into the single factor $(\tptnot/\tpt)^{\mmm}$.
The exact temperature dependence of Boltzmann factors is retained.
If the parameter $\mmm$ is available, use of (\ref{eqn:lnstr_apx}) 
offers decreased computational overhead relative to
(\ref{eqn:lnstr_scl}). 

Pressure-broadened halfwidths $\hwhmprsnot$ are also provided at the 
reference temperature and pressure, 
$\hwhmprsnot \equiv \hwhmprs(\prsnot,\tptnot)$.
\acr{HITRAN} literature refers to broadening features are as
\trmdfn{air-broadened} (rather than pressure-broadened) because all
parameters in the \acr{HITRAN} database are normalized to Earth's natural
isotopic mixing ratios. 
Thus while the broadening process is collision-broadening, the
tabulated line width refers to collision-broadening by air of Earth's  
isotopic composition, hence the label. 
\acr{HITRAN} actually provides two halfwidths, the self-broadened halfwidth
$\hwhmslfnot$ and the air-broadened or \trmdfn{foreign-broadened}
halfwidth $\hwhmfrnnot$.
Air-broadening accounts for collision-broadening by foreign molecules,
i.e., all molecules except the species undergoing radiative
transition. 
Thus air-broadening is, to a good approximation, due nitrogen and
oxygen molecules.
The other form of broadening, \trmdfn{self-broadening} refers to
collision-broadening by molecules of the species undergoing the
radiative transition. 
Because reasonances may develop between members of identical species,
in general $\hwhmslfnot \ne \hwhmfrnnot$.
The pressure-broadened halfwidth of transitions of radiatively active
species \AAA\ at arbitrary pressure and temperature scales as 
\begin{eqnarray}
% RRg98 p. 710 (A12)
\hwhmprs(\prs,\tpt) & = & 
\left( \frac{\tptnot}{\tpt} \right)^{\prsbrdtptdpnxpn} 
\left( \frac{\prs - \prsprtA}{\prsnot} \hwhmfrnnot +
\frac{\prsprtA}{\prsnot} \hwhmslfnot \right)
\label{eqn:hwhm_scl}
\end{eqnarray}
where $\prsbrdtptdpnxpn$ is an empirical fitting parameter supplied by
\acr{HITRAN} and $\prsprtA$ is the partial pressure of species \AAA.
Clearly self-broadening effects are appreciable only for species with 
signifcant partial pressures, i.e., \Od\ and \Nd. 
For minor trace gases, $\prsprtA \ll \prs$ and $\prsprtA \ll \prsnot$
so $\prs - \prsprtA \approx \prs$ and $\prsprtA/\prsnot \approx 0$.
For such gases, (\ref{eqn:hwhm_scl}) simplifies to
\begin{eqnarray}
\hwhm(\prs,\tpt) & \approx & 
\hwhmnot \frac{\prs}{\prsnot}
\left( \frac{\tptnot}{\tpt} \right)^{\prsbrdtptdpnxpn} 
\label{eqn:hwhm_apx_scl}
\end{eqnarray}
which has no dependence on self-broadening.

Collision-broadening may also cause a \trmidx{pressure-shift} of the
line transition frequency away from the tabulated line center
frequency $\wvnnot = \wvn(\prsnot)$ to a shifted frequency 
$\wvnshf = \wvn(\prs)$.
\acr{HITRAN} supplies a parameter $\prsshf(\prsnot)$ in CGS
wavenumbers per atmosphere (\wvnxatm) with which to calculate the line
center frequency change due to pressure-shifting.
\begin{eqnarray}
\wvnshf(\prs) & = & \wvnnot + \prsshf(\prsnot) \frac{\prs}{\prsnot}
\label{eqn:prs_shf_dfn}
\end{eqnarray}

The prescriptions for adjusting line centers (\ref{eqn:prs_shf_dfn})
and half-widths (\ref{eqn:hwhm_scl}) from $(\prsnot,\tptnot)$ to
to arbitrary $\prs$ and $\tpt$ should be used to determine the 
corrected line shape profile of each transition considered.
In the lower atmosphere, application of (\ref{eqn:prs_shf_dfn})
and (\ref{eqn:hwhm_scl}) leads to a corrected Lorentzian profile 
(\ref{eqn:lnshp_lrn_dfn}) 
\begin{eqnarray}
\lnshplrn(\wvn,\wvnnot,\prs,\tpt) & = & 
\frac{1}{\mpi} \,
\frac{\hwhmlrn(\prs,\tpt)}{\
\{ \wvn - [\wvnnot + \prs \prsshf(\prsnot)/\prsnot] \}^{2} + 
\hwhmlrn(\prs,\tpt)^{2}} \nonumber \\
\lnshplrn(\frq,\frqnot,\prs,\tpt) & = & 
\frac{1}{\mpi} \,
\frac{\hwhmlrn(\prs,\tpt)}{\
(\frq - \frqshf)^{2} + \hwhmlrn(\prs,\tpt)^{2}}
\label{eqn:lnshp_lrn_crc_dfn}
\end{eqnarray}
Of course this section has only discussed application of \acr{HITRAN}
database parameters to atmospheres in \trmidx{LTE} conditions.
Application of \acr{HITRAN} to \trmdfn{nonlocal thermodynamic equilibrium}
conditions is discussed in \cite{GaR92}.

It is necessary to compute statistics of $\lnstrln$ for use in
\trmidx{narrow band models} (\S\ref{sxn:nbm}).
There is no reason not to apply the full approach of
(\ref{eqn:lnstr_scl}) when computing these statistics.
Certain well-documented narrow band models use more approximate forms
appropriate for specific applications.
\cite{Bri922} uses
\begin{eqnarray}
\lnstrln(\tpt) & = & \lnstrlnnot(\tptnot)
\left( \frac{\tptnot}{\tpt} \right)^{3/2}
\frac{\me^{-\cstplk \cstspdlgt \wvn_{\iii} / \cstblt \tpt }}
{\me^{-\cstplk \cstspdlgt \wvn_{\iii} / \cstblt \tptnot}}
\frac{1 -  \me^{-\cstplk \cstspdlgt \wvn_{\iii \jjj} / \cstblt \tpt }}
{1 -  \me^{-\cstplk \cstspdlgt \wvn_{\iii \jjj} / \cstblt \tptnot }}
\label{eqn:lnstr_mlk}
\end{eqnarray}
which is a hybrid of (\ref{eqn:lnstr_scl}) and (\ref{eqn:lnstr_apx})
with $\mmm = 1.5$.

\subsection[Line-By-Line Models]{Line-By-Line Models}\label{sxn:lbl}

\subsubsection[Literature]{Literature}\label{sxn:lbl_ltr}
The classic paper on the water vapor continuum is \cite{CKD89}.
\cite{FuL92} compare a correlated-$\kkk$ method to line-by-line
model results.
\cite{ElW96} describe the a field experiment to directly compared
measured and modeled spectral radiances to both band and line-by-line
models. 
\cite{Cri972} uses a line-by-line model to determine atmospheric solar
absorption. 
\cite{Spa97} presents an economical algorithm for selecting variable
wavelength grid resolution such that absorption coefficients may be
computed to a given level of accuracy.
\cite{MTB97} compare a correlated-$\kkk$ model (\acr{rrtm}) to 
the well-known line-by-line model \acr{lblrtm}.
\cite{MoC97} describe a fully scattering line-by-line model.
\cite{WWM97} presents test-case spectra for evaluating line-by-line
radiative transfer models in cold and dry atmospheres.
\cite{VRC981} use observations to constrain magnitude of any
non-Lorentzian continuum in the near-infrared.
\cite{PHS00} discuss the \trmidx{equivalance theorem}.
\cite{QuD01} present an algorithm that computes absorption
coefficients to a specified error tolerance by using a pre-computed
lookup table of where interpolation is appropriate.

\section[Band Models]{Band Models}\label{sxn:nbm}

Band models, also called narrow band models, discretize the radiative
transfer equation  intervals for which the line statistics and the
Planck function are relatively constant. 
The literature describing these models extends back to Goody (1957).  
Band models have traditionally been applied to thermal source
functions (\ref{eqn:trn_sln_upw})--(\ref{eqn:trn_sln_dwn}), but they
may also be applied to the solar spectral region \cite[]{ZBP97}
(MODTRAN3) if additional assumptions are made. 

\subsection{Generic}\label{sxn:gnr}
The following presentation assumes that the absorption path is
homogeneous, i.e., at constant temperature and pressure. 
Important corrections to these assumptions are necessary in
inhomogeneous atmospheres.
These corrections are discussed in \S\ref{sxn:HCG}.

This discussion makes use of an arbitrary frequency 
interval $\dltfrq$ which represents the discretization interval of  
narrow band approximation.
It is important to remember that $\dltfrq$ is best determined
empirically by comparison of the narrow band approximation to
\trmidx{line-by-line} approximations.
Typically, $\dltfrq$ is 5--10\,\xcm.
\cite{KiR83} showed $\dltfrq = 5\,\xcm$ bands are optimal for \COd. 
\cite{Bri922} uses $\dltfrq = 5\,\xcm$ for \COd, \Ot, \CHq, \NdO, but
$\dltfrq = 10\,\xcm$ for \HdO. 
\cite{Kie97} recommends $\dltfrq = 5\,\xcm$ for \COd, 10\,\xcm\ for
\HdO, 5--10\,\xcm\ for \Ot, and 5\,\xcm\ for all other trace gases. 

\subsubsection[Beam Transmittance]{Beam Transmittance}\label{sxn:trnbm}
The gaseous absorption optical depth $\tauabsoffrq$ is the product of
the spectrally resolved molecular cross-section $\xsxabsoffrq$
(\ref{eqn:xsx_abs_dfn}) and the absorber path $\abspth$
(\ref{eqn:abs_pth_dfn}) 
\begin{eqnarray}
\tauabsoffrq & = & \xsxabsoffrq \abspth
\label{eqn:tau_abs_mlk_dfn}
\end{eqnarray}
The transmittance $\trnbmoffrq$ between two points in a homogeneous
atmosphere is the negative exponential of $\tauabsoffrq$
(\ref{eqn:trnbm_ppa}),  
\begin{eqnarray}
\trnbmoffrq & = & \me^{-\tauabsoffrq} \nonumber \\
& = & \me^{-\xsxabsoffrq \abspth}
\label{eqn:trnbm_ppa2}
\end{eqnarray}
Note that $\trnbmoffrq$ is the spectrally and directionally resolved 
transmittance because (a) it pertains to a monochromatic frequency
interval and (b) it applies to radiances, not to irradiances 
(which require an additional angular integration).
Narrow band models are based on the \trmdfn{mean transmittance}
$\trndltfrq$ of a narrow but finite frequency interval $\dltfrq$
between $\frqnot-\dltfrq/2$ and $\frqnot+\dltfrq/2$.
\begin{eqnarray}
\trndltfrq & = & \frac{1}{\dltfrq} \int_{\dltfrq} \trnbm \,\dfr\frq \nonumber \\  
& = & \frac{1}{\dltfrq} \int_{\dltfrq} \me^{-\tauabs} \,\dfr\frq
\label{eqn:trn_dlt_frq_dfn}
\end{eqnarray}
where $\frqnot$ is the central frequency of the interval in question

Using (\ref{eqn:tau_abs_dfn}), we rewrite (\ref{eqn:trn_dlt_frq_dfn})
as 
\begin{equation}
\trndltfrq = \frac{1}{\dltfrq} 
\int_{\dltfrq} \exp[-\xsxabsoffrq \abspth] \,\dfr\frq \nonumber
\end{equation}
If the lines are Lorentzian (i.e., pressure-broadened)
then the transmittance may be written in terms of the line strength and
absorber amount using (\ref{eqn:xsx_lnstr_lnshp}) and
(\ref{eqn:lnshp_lrn_dfn2}) 
\begin{equation}
\xsxabsoffrq \abspth = \frac{\lnstr \hwhmlrn \abspth}{\mpi(\frq^{2} + \hwhmlrn^{2})} 
\label{eqn:lrn_pth_dfn}
\end{equation}
Thus (\ref{eqn:trn_dlt_frq_dfn}) becomes
\begin{equation}
\trndltfrq = \frac{1}{\dltfrq} 
\int_{\dltfrq} \exp \left[ 
-\frac{\lnstr \hwhmlrn \abspth}{\mpi(\frq^{2} + \hwhmlrn^{2})} 
\right] \,\dfr\frq
\label{eqn:lrn_pth_dfn2}
\end{equation}

\subsubsection[Beam Absorptance]{Beam Absorptance}\label{sxn:absbm}
In analogy to $\trnbm$ (\ref{eqn:trn_dfn}) we define the
\trmdfn{monochromatic beam absorptance} $\absbm$
\begin{eqnarray}
\absbm & = & 1 - \trnbm \nonumber \\
& = & 1 - \exp[-\xsxabsoffrq \abspth]
\label{eqn:abs_dfn}
\end{eqnarray}
where $\xsxabsoffrq$ is the absorption cross-section and
$\abspth$ is the absorber path.
As described in \S\ref{sxn:trnbm}, we refer to $\absbm$ as the beam
absorptance becuase it pertains to a monochromatic frequency, and
refers to absorption of radiance, not irradiance.

The \trmdfn{band absorptance} $\absdltfrq$ is the complement of the
band transmittance
\begin{eqnarray}
\absdltfrq & = & 1 - \trndltfrq \nonumber \\
& = & \frac{1}{\dltfrq} \int_{\dltfrq} \absbm \,\dfr\frq \nonumber \\
& = & \frac{1}{\dltfrq} \int_{\dltfrq} 1 - \me^{-\tauabs} \,\dfr\frq \nonumber \\
& = & \frac{1}{\dltfrq} \int_{\dltfrq} 1 - \exp[-\xsxabsoffrq \abspth]
\,\dfr\frq
\label{eqn:abs_dlt_frq_dfn}
\end{eqnarray}
The relation between the band absorptance and band transmittance
is the same as the relation between the monochromatic beam absorptance
and the monochromatic beam transmittance (\ref{eqn:abs_dfn}).
$\absdltfrq$ measures the mean absorptance within a finite frequency
(or wavelength) range comprising many lines.
$\absdltfrq$ does not include any contribution from outside the
$\dltfrq$ range.
Thus $\absdltfrq$ may neglect contribution from the far wings of 
some lines in the band.
Moreover, it is important to remember that energy absorption follows 
the band absorptance only if the energy distribution within the band is
close to linear.

If the line shape is Lorentzian (\ref{eqn:lnshp_lrn_dfn}), then
(\ref{eqn:lrn_pth_dfn2}) applies so that
\begin{equation}
\absdltfrq = \frac{1}{\dltfrq} \int_{\dltfrq} 1 - 
\exp \left[ 
-\frac{\lnstr \hwhmlrn \abspth}{\mpi(\frq^{2} + \hwhmlrn^{2})} 
\right]
\,\dfr\frq
\label{eqn:abs_dlt_frq_dfn2}
\end{equation}

\subsubsection[Equivalent Width]{Equivalent Width}\label{sxn:eqv_wdt}
The spectrally integrated monochromatic beam absorptance of a line is
called its \trmdfn{equivalent width}
\begin{eqnarray}
\eqvwth & = & \int_{-\infty}^{+\infty} \absbm \,\dfr\frq \nonumber \\
\eqvwthabspth & = & \int_{-\infty}^{+\infty} 1 - \exp[-\xsxabsoffrq \abspth] \,\dfr\frq
\label{eqn:eqv_wth_dfn}
\end{eqnarray}
Thus the absorber amount must be specified in order to determine
the equivalent width.
The lower limit of integration, $-\infty$, is actually an
approximation which is mathematically convenient to retain
so that (\ref{eqn:eqv_wth_dfn}) is analytically integrable for the
important line shape functions.
Recall that we are working with a frequency coordinate $\frq$ which is
defined to be $\frq = 0$ at line center $\frqnot$ (\ref{eqn:abs_lrn_dfn}).
In practice the error caused by using the analytic expressions
resulting from $\int_{-\infty}^{+\infty}$ rather than those resulting
from $\int_{-\frqnot}^{+\infty}$ is negligible except for lines 
in the microwave. 

The name ``equivalent width'' reminds us that $\eqvwth$ is the width 
of a completely saturated rectangular line profile that has the same
total absorptance.
The relation between $\abspth$ and $\eqvwthabspth$ is called the
\trmdfn{curve of growth}.
The curve of growth of a gas can be measured in the laboratory.
The interpretation of the curve of growth played a very important role
in advancing our understanding and representation of gaseous
absorption. 

If the line shape is Lorentzian (\ref{eqn:lnshp_lrn_dfn}), then
(\ref{eqn:abs_dlt_frq_dfn2}) applies
\begin{equation}
% GoY89 p. 130 (4.7) ThS99 p. 387 (10.6)
\eqvwth = \int_{-\infty}^{+\infty} 
1 - \exp \left[ 
-\frac{\lnstr \hwhmlrn \abspth}{\mpi(\frq^{2} + \hwhmlrn^{2})} 
\right]
\,\dfr\frq
\label{eqn:eqv_wth_dfn2}
\end{equation}

\subsubsection[Mean Absorptance]{Mean Absorptance}\label{sxn:abs_avg}
Although the spectrally resolved absorptance (\ref{eqn:abs_dfn}) and
the band absorptance (\ref{eqn:abs_dlt_frq_dfn}) are dimensionless, the
equivalent width (\ref{eqn:eqv_wth_dfn}) has units of frequency (or
wavelength).   
It is convenient to define a dimensionless mean absorptance for a
transition line $\absavg$
\begin{eqnarray}
% ThS99 p. 387 (10.4) GoY89 p. 129 (4.6)
\absavg & = & \frac{1}{\mls} \int_{-\infty}^{+\infty} \absbm \,\dfr\frq
\nonumber \\
& = & \frac{1}{\mls} \int_{-\infty}^{+\infty}
1 - \exp[-\xsxabsoffrq \abspth] \,\dfr\frq
\nonumber \\
& = & \frac{\eqvwth}{\mls}
\label{eqn:abs_avg_dfn}
\end{eqnarray}
where $\mls$ is the mean line spacing between adjacent lines in a
given band.
Note that $\absdltfrq$ (\ref{eqn:abs_dlt_frq_dfn}) differs from
$\absavg$ (\ref{eqn:abs_avg_dfn}). 
$\absavg$ accounts for the absorptance due to a single line over the
entire spectrum and renormalizes that to a mean absorptance within
a specified frequency (or wavelength) range ($\mls$).
In contrast to $\absdltfrq$, $\absavg$ does not neglect any
absorption in the far wings of a line.
Therefore the sum of the average absorptances of all the lines
centered within $\dltfrq$ may exceed, by a small amount, the band 
absorptance computed from (\ref{eqn:abs_dlt_frq_dfn})
\begin{equation}
\sum_{\iii = 1}^{\iii = \NNN} \absavg_i \ge \absdltfrq
\label{eqn:absavg_absdlt}
\end{equation}
This difference should make it clear that $\absavg$ is much more
closely related to $\eqvwth$ than to $\absdltfrq$.

For lines with Lorentzian profiles (\ref{eqn:lnshp_lrn_dfn}),
$\eqvwthavg$ and $\absavg$ are
\begin{eqnarray}
% ThS99 p. 387 (10.6) GoY89 p. 130 (4.9)
\eqvwth = \absavg \mls & = & \int_{-\infty}^{+\infty} 
1 - \exp \left[ 
-\frac{\lnstr \hwhmlrn \abspth}{\mpi(\frq^{2} + \hwhmlrn^{2})} 
\right]
\,\dfr\frq
\label{eqn:absavg_dfn2}
\end{eqnarray}
We shall rewrite (\ref{eqn:absavg_dfn2}) in terms of three
dimensionless variables, $\xxx$, $\yyy$, and $\uuu$ 
\begin{subequations}
\label{eqn:absavg_cov}
\begin{align}
\label{eqn:absavg_cov_xxx}
\xxx & = \frq / \mls \\
\frq & = \mls \xxx \nonumber \\
\dfr\xxx & = \mls^{-1} \,\dfr\frq \nonumber \\
\dfr\frq & = \mls \,\dfr\xxx \nonumber \\
\label{eqn:absavg_cov_yyy}
\yyy & = \hwhmlrn / \mls \\
\hwhmlrn & = \mls \yyy \nonumber \\
\label{eqn:absavg_cov_uuu}
\uuu & = \frac{\lnstr \abspth}{2 \mpi \hwhmlrn} \\
\frac{\lnstr \abspth}{\mpi} & = 2 \hwhmlrn \uuu \nonumber
\end{align}
\end{subequations} 
This change of variables maps $\frq \in [-\infty,+\infty]$ to 
$\xxx \in [-\infty,+\infty]$.
We obtain
\begin{eqnarray}
% ThS99 p. 387 (10.6) GoY89 p. 130 (4.9)
\absavg & = & \frac{1}{\mls} \int_{-\infty}^{+\infty} 
1 - \exp \left[ 
-(2 \hwhmlrn \uuu)
\left(\frac{\hwhmlrn}{\mls^{2} \xxx^{2} + \mls^{2} \yyy^{2}}\right)
\right]
\, (\mls \,\dfr\xxx)
\nonumber \\
& = & \int_{-\infty}^{+\infty} 
1 - \exp \left[ 
-\frac{2 \hwhmlrn^{2} \uuu}{\mls^{2} (\xxx^{2} + \yyy^{2})}
\right]
\,\dfr\xxx
\nonumber \\
& = & \int_{-\infty}^{+\infty} 
1 - \exp \left[ 
-\frac{2 \mls^{2} \yyy^{2} \uuu}{\mls^{2} (\xxx^{2} + \yyy^{2})}
\right]
\,\dfr\xxx
\nonumber \\
& = & \int_{-\infty}^{+\infty} 
1 - \exp \left[ 
-\frac{2 \uuu \yyy^{2}}{\xxx^{2} + \yyy^{2}}
\right]
\,\dfr\xxx
\label{eqn:abs_avg_xy}
\end{eqnarray}
The solution to this definite integral may be written in terms of
modified \trmidx{Bessel functions} of the first kind $\bslIcpx(\uuu)$
(see \S\ref{sxn:bsl}) 
\begin{eqnarray}
% ThS99 p. 387 (10.7) GoY89 p. 130 (4.10)
\absavg & = & 2 \mpi \yyy \uuu \me^{-\uuu} [ \bslIfnc_{0}(\uuu) + \bslIfnc_{1}(\uuu) ]
\nonumber \\
& \equiv & 2 \mpi \yyy \ldnfnc(\uuu) \nonumber \\
& = & 2 \mpi \hwhmlrn \mls^{-1} \ldnfnc(\uuu) \nonumber \\
\eqvwth & = & 2 \mpi \hwhmlrn \ldnfnc(\uuu)
\label{eqn:lr_dfn}
\end{eqnarray}
where $\ldnfnc(\uuu)$ is the \trmdfn{Ladenburg and Reiche function}
\cite[]{LaR13}.
The dimensionless optical path $\uuu$ (\ref{eqn:absavg_cov}) is
therefore a key parameter in determining the gaseous absorptance.
Two important limiting cases of (\ref{eqn:lr_dfn}) are $\uuu \ll 1$
and $\uuu \gg 1$.

For small optical paths, $\lnstr \abspth \ll 1$
(\ref{eqn:abs_pth_dfn}), or, equivalently, $\uuu \ll 1$
(\ref{eqn:absavg_cov_uuu}).  
This is the \trmdfn{weak-line limit}.
In this limit the exponential in (\ref{eqn:absavg_dfn2}) or
(\ref{eqn:abs_avg_xy}) may be replaced by the first two terms in its
Taylor series expansion.   
Thus all line shapes have the same weak-line limit
\begin{eqnarray}
% GoY89 p. 131 (4.14) ThS99 p. 387 (10.8)
\eqvwth = \absavg \mls & \approx & \int_{-\infty}^{+\infty}
\xsxabsoffrq \abspth \,\dfr\frq
\nonumber \\
& \approx & \lnstr \abspth
\int_{-\infty}^{+\infty} \lnshplrnoffrq \,\dfr\frq
\nonumber \\
& \approx & \lnstr \abspth
\label{eqn:abs_avg_wll}
\end{eqnarray}
where we have use the generic normalization property of the line shape
profile (\ref{eqn:lnshplrn_ntr_nrm}) in the final step.

For large optical paths, $\lnstr \abspth \gg 1$
(\ref{eqn:abs_pth_dfn}), or, equivalently, $\uuu \gg 1$
(\ref{eqn:absavg_cov_uuu}).  
This is the \trmdfn{strong line limit}.
To examine this limit we first note that the line HWHM is much smaller
than the frequencies where line absorptance is strong, i.e., 
$\hwhmlrn \ll \frq$ in (\ref{eqn:absavg_dfn2}). 
Equivalently, $\yyy \ll \xxx$ 
(\ref{eqn:absavg_cov_xxx})--(\ref{eqn:absavg_cov_yyy}) 
so that $\yyy^{2}$ may be neglected relative to $\xxx^{2}$ in the
denominator of (\ref{eqn:abs_avg_xy})
\begin{eqnarray}
% GoY89 p. 132 (4.15), ThS99 p. 388 (10.9)
\absavg & \approx & \int_{-\infty}^{+\infty} 
1 - \exp ( -2 \uuu \yyy^{2}/\xxx^{2} ) \nonumber
\,\dfr\xxx \nonumber
\end{eqnarray}
One further simplification is possible.
Both terms in the integrand are symmetric about the origin so we may
consider only positive $\xxx$ if we double the value of the integral. 
\begin{eqnarray}
\absavg & = & 2 \int_{0}^{+\infty} 
1 - \exp ( -2 \uuu \yyy^{2}/\xxx^{2} )
\,\dfr\xxx
\label{eqn:abs_avg_xy_sll}
\end{eqnarray}
The change of variables $\zzz = 2 \uuu \yyy^{2} \xxx^{-2}$
maps $\xxx \in (0,+\infty)$ to $\zzz \in (+\infty,0)$
\begin{subequations}
\label{eqn:sll_cov}
\begin{align}
\label{eqn:sll_cov_zzz}
% GoY89 p. 132 (4.15), ThS99 p. 388 (10.10)
\zzz & = 2 \uuu \yyy^{2} \xxx^{-2} \\
\label{eqn:sll_cov_xxx}
\xxx & = (2 \uuu)^{1/2} \yyy \zzz^{1/2} \\
\dfr\xxx & = (2 \uuu)^{1/2} \yyy 
(-{\textstyle{\frac{1}{2}}})\zzz^{-3/2} \,\dfr\zzz \nonumber \\
\label{eqn:sll_cov_dx}
& = - 2^{-1/2} \uuu^{1/2} \yyy \zzz^{-3/2} \,\dfr\zzz
\end{align}
\end{subequations} 
Substituting this into (\ref{eqn:abs_avg_xy_sll}) leads to
\begin{eqnarray}
% GoY89 p. 132 (4.15), ThS99 p. 388 (10.10)
\absavg & \approx & 
2 \int_{+\infty}^{0} 
(1 - \me^{-\zzz}) ( - 2^{-1/2} \uuu^{1/2} \yyy ) \zzz^{-3/2} 
\,\dfr\zzz \nonumber \\
& \approx & 
2^{1/2} \uuu^{1/2} \yyy \int_{0}^{+\infty}
(1 - \me^{-\zzz}) \zzz^{-3/2} 
\,\dfr\zzz \nonumber \\
& \approx & 
\yyy \sqrt{2 \uuu} \int_{0}^{+\infty}
\zzz^{-3/2} - \zzz^{-3/2} \me^{-\zzz} 
\,\dfr\zzz \nonumber
\label{eqn:sll_pre_dfn}
\end{eqnarray}
The first term in the integrand of (\ref{eqn:sll_pre_dfn}) is directly
integrable $\int \zzz^{-3/2} \,\dfr\zzz = -2 \zzz^{-1/2}$.
The second term in the integrand of (\ref{eqn:sll_pre_dfn})
is the complete \trmdfn{gamma function} $\gmmfnc(-1/2) = $.
Section~\ref{sxn:gmm} describes the properties of gamma
functions\footnote{This section is currently in 
\url{http://dust.ess.uci.edu/facts/aer/aer.pdf}.}.

Combining these results we find that the mean absorptance of an
isolated Lorentz line in the \trmidx{strong-line limit}
(\ref{eqn:sll_pre_dfn}) reduces to 
\begin{eqnarray}
% GoY89 p. 132 (4.15), ThS99 p. 388 (10.10)
\absavg & \approx & 
\yyy \sqrt{2 \uuu} \left(
-2 \zzz^{-1/2} \bigg|_{0}^{+\infty} 
- (- 2 \sqrt{\mpi}) \right)
\nonumber \\
& = & 
\yyy \sqrt{2 \uuu} ( \infty + 2 \sqrt{\mpi} ) \nonumber \\
& = & 
2 \yyy \sqrt{2 \mpi \uuu}
\label{eqn:sll_dfn}
\end{eqnarray}
fxm: The first term must vanish, but how?
Substituting (\ref{eqn:absavg_cov_yyy})--(\ref{eqn:absavg_cov_uuu}) 
into (\ref{eqn:sll_dfn})
\begin{eqnarray}
% GoY89 p. 132 (4.16), ThS99 p. 388 (10.10)
\absavg 
& \approx & 
2 \left( \frac{\hwhmlrn}{\mls} \right) \sqrt{2 \mpi} 
\sqrt{ \frac{\lnstr \abspth}{2 \mpi \hwhmlrn} }
\nonumber \\
\eqvwth = \absavg \mls
& = & 
2 \sqrt{ \lnstr \abspth \hwhmlrn }
\label{eqn:abs_avg_sll}
\end{eqnarray}
In the \trmidx{strong line limit} we see that the mean absorptance 
$\absavg$ of an isolated line increases only as the square-root of the 
mass path.
Physically, the strong line limit is approached as the line core
becomes saturated and any additional absorption must occur in the line
wings. 
Put another way, transition lines do not obey the exponential
\trmidx{extinction law} on which our solutions to the radiative
transfer equation are based.
One important consequence of this result is that complicated gaseous  
spectra must either be decomposed into a multitude of monochromatic
intervals, each narrow enough to resolve a small portion of a
transition line, or some new statistical means must be developed which 
correctly represents line absorption in both the weak line and strong
line limits.

In summary, the equivalent width $\eqvwth$ (\ref{eqn:eqv_wth_dfn})  of
an isolated spectral line behaves distinctly differently in the two
limits (\ref{eqn:abs_avg_wll}) and (\ref{eqn:abs_avg_sll})
\begin{eqnarray}
\eqvwth = \absavg \mls & = \left\{
\begin{array}{r@{\quad:\quad}ll}
% GoY89 p. 132 (4.16), ThS99 p. 388 (10.10), Lio92 p. 52 (2.4.6,2.4.7)
%\label{eqn:eqv_wth_wll}
\lnstr \abspth & \mbox{Weak-line limit} \\
%\label{eqn:eqv_wth_sll}
2 \sqrt{ \lnstr \abspth \hwhmlrn } & \mbox{Strong-line limit}
\end{array} \right.
\label{eqn:eqv_wth_lmt}
\end{eqnarray}

\subsection[Line Distributions]{Line Distributions}\label{sxn:ld}
Inspection of realistic gaseous absorption spectra reveals that line
strengths in complex bands occupy a seemingly continuous distribution
space, with line strengths $\lnstr$ varying over many orders of
magnitude in a single band.
A key point discussed further in \cite{GoY89} is that the variabilities
in line spacing and in line widths within a band are negligible
(and usually order of magnitude smaller) in comparison to the dynamic
range of $\lnstr$.
A band containing a suitably large number of lines, therefore, 
may be amenable to the approximation that the line strength
distribution may be represented by a continuous function of $\lnstr$. 
The \trmdfn{line strength distribution function} $\pdfoflnstr$ is the
probability that a line in a given spectral region will have a line
strength between $\lnstr$ and $\lnstr + \dfr\lnstr$.
The function $\pdfoflnstr$ must be correctly normalized so that
probability of a line having a finite, positive strength is unity
\begin{eqnarray}
\int_{0}^{\infty} \pdfoflnstr \,\dfr\lnstr & = & 1
\label{eqn:pdf_lnstr_nrm}
\end{eqnarray}
A number of functional forms for $\pdfoflnstr$ have been proposed.

\subsubsection[Line Strength Distributions]{Line Strength Distributions}\label{sxn:lsd}
\cite{Goo52} proposed the \trmdfn{exponential line strength
distribution}, now also known as the \trmdfn{Goody distribution} 
\begin{eqnarray}
% Goy89 p. 138 (4.25)
\pdfoflnstr & = & \lnstravg^{-1} \, \me^{-\lnstr/\lnstravg}
\label{eqn:dst_xpn}
\end{eqnarray}
where $\lnstravg$ is a constant which, in the next section,
we show to be equal to the mean line intensity.
A prime advantage of (\ref{eqn:dst_xpn}) is its arithmetic
tractability. 
The zeroth and first moments of (\ref{eqn:dst_xpn}) are both
solvable analytically. 
The following sections use this property to demonstrate the
absorptive characteristics of line distributions.
However, more complex distributions can improve upon the exponential
distribution (\ref{eqn:dst_xpn}) by better representing the observed 
line shape distribution of many important species in Earth's
atmosphere, such as \HdO.

\cite{Mal67} proposed an improved, albeit more complex,
analytic form for the line strength distribution function.
First we present a simplified, approximate version of this PDF
which illustrates the essential differences between the Malkmus
distribution and simpler line strength distributions.
\begin{eqnarray}
% Kie97 p. 112 (38) Mal67 p. 324 (6)
\pdfoflnstr & = & \lnstr^{-1} \, \me^{-\lnstr/\lnstravg}
\label{eqn:dst_mlk_apx}
\end{eqnarray}
The difference with (\ref{eqn:dst_xpn}) is the prefactor has changed
from $\lnstravg^{-1}$ to $\lnstr^{-1}$.
The $\lnstr^{-1}$ dependence increases the number of weaker lines
relative to stronger lines, which is an improvement over
(\ref{eqn:dst_xpn}) and simpler distributions (such as the Elsasser or
Godson distributions). 
The Malkmus distribution (\ref{eqn:dst_mlk_apx}) is now perhaps the most
commonly used line distribution function. 

The astute reader will note that (\ref{eqn:dst_mlk_apx}) is not
normalizable on the interval $[0,+\infty)$
(cf. \S\ref{sxn:lnstr_dst_nrm}). 
A key point of the Malkmus distribution, therefore, is the
truncation or tapering of $\pdfoflnstr$ so that statistics of the
resulting $\pdfoflnstr$ are well-behaved.
A straightforward line strength distribution function with most
of the desired properties is the ``truncated'' distribution which is
non-zero only between a fixed maximum and minimum line strength, 
$\lnstrmax$ and $\lnstrmin$, respectively.
By convention, $\lnstrmin$ is deprecated in favor of the ratio
$\lnstrrat$ between the maximum and minimum line strengths considered
in the band  
\begin{eqnarray}
\lnstrmin & = & \lnstrmax / \lnstrrat \nonumber \\
\lnstrrat & \equiv & \lnstrmax / \lnstrmin
\label{eqn:lnstr_rat_dfn}
\end{eqnarray}
The line strength ratio $\lnstrrat$ measures the dynamic range of line
strengths considered in a band, which can be quite large, e.g.,
greater than $10^{6}$ for \HdO\ bands. 
With this convention, 
\begin{eqnarray}
% Kie97 p. 112 (38) Mal67 p. 324 (6)
\pdfoflnstr & = & \left\{
\begin{array}{r@{\quad:\quad}ll}
0 & \lnstr < \lnstrmax/\lnstrrat \\
(\lnstr \ln \lnstrrat)^{-1}
& \lnstrmax/\lnstrrat < \lnstr < \lnstrmax \\
0 & \lnstr > \lnstrmax \\
\end{array} \right.
\label{eqn:dst_mlk_trnc}
\end{eqnarray}
\cite{God53} appears to have been the first to examine
(\ref{eqn:dst_mlk_trnc}), although it is closely related to the full
Malkmus distribution. 
The normalization of (\ref{eqn:dst_mlk_trnc}) is demonstrated
in \S\ref{sxn:lnstr_dst_nrm}.
Clearly (\ref{eqn:dst_mlk_trnc}) contains no contributions from lines
weaker than $\lnstrmax/\lnstrrat$ or stronger than $\lnstrmax$.
The main disadvantage to (\ref{eqn:dst_mlk_trnc}) is that it is
discontinuous, and thus more difficult to treat computationally.

\cite{Mal67} had the insight to identify the following continuous
function
\begin{eqnarray}
% Kie97 p. 112 (38) Mal67 p. 324 (6)
\pdfoflnstr & = & 
\frac{\me^{-\lnstr/\lnstrmax} - \me^{-\lnstrrat \lnstr / \lnstrmax }}
{\lnstr \ln \lnstrrat} 
\label{eqn:dst_mlk}
\end{eqnarray}
We shall call (\ref{eqn:dst_mlk}) the full Malkmus line strength
distribution. 
In practice, (\ref{eqn:dst_mlk}) is only applied in the limit
$\lnstrrat \rightarrow \infty$.
Rather than simply truncating (\ref{eqn:dst_mlk_apx}) so that
$\pdfoflnstr$ is non-zero only between some lower and upper bound
(\ref{eqn:dst_mlk_trnc}), \cite{Mal67} developed (\ref{eqn:dst_mlk})
because it has many useful properties:
% 20040125: fxm tth chokes here if use enumerate* not enumerate
\begin{enumerate*}
\item $\pdfoflnstr$ is continuous
\item $\pdfoflnstr$ has $\lnstr^{-1}$ dependence over the bandwidth of
interest
\item $\pdfoflnstr$ approaches (\ref{eqn:dst_mlk_trnc})
\item $\pdfoflnstr$ is analytically integrable
\end{enumerate*}

Following \cite{Mal67}, let us ellucidate the behavior of
(\ref{eqn:dst_mlk}) in important limits. 
For large $\lnstrrat$ and 
$\lnstrmax/\lnstrrat \ll \lnstr \ll \lnstrmax$ the 
bracketed term in (\ref{eqn:dst_mlk}) approaches unity so that
$\pdfoflnstr \approx (\lnstr \ln \lnstrrat)^{-1}$, i.e.,
the behavior is identical to (\ref{eqn:dst_mlk_trnc}).
For $\lnstr \gg \lnstrmax$, 
$\pdfoflnstr \approx (\lnstr \ln \lnstrrat)^{-1}
\me^{-\lnstr/\lnstrmax}$. 
Thus for $\lnstr \gg \lnstrmax$, $\pdfoflnstr$ is non-zero but much
less than what extrapolating $(\lnstr \ln \lnstrrat)^{-1}$ into this 
region would yield.
Finally, for large $\lnstrmax$ and $\lnstr \ll \lnstrmax/\lnstrrat$,  
$\pdfoflnstr \approx (\lnstrrat)^{-1}(\lnstrrat -1)/\lnstrmax$.
To summarize, the limits of (\ref{eqn:dst_mlk}) are as follows
\begin{eqnarray}
% Mal67 p. 324 (7)
\pdfoflnstr & = & \left\{
\begin{array}{r@{\quad:\quad}ll}
(\ln \lnstrrat)^{-1}(\lnstrrat -1)/\lnstrmax
& \lnstr \ll \lnstrmax/\lnstrrat \\
(\lnstr \ln \lnstrrat)^{-1} 
& \lnstrmax/\lnstrrat \ll \lnstr \ll \lnstrmax \\
(\lnstr \ln \lnstrrat)^{-1} \me^{-\lnstr/\lnstrmax} 
& \lnstr \gg \lnstrmax \\
\end{array} \right.
\label{eqn:dst_mlk_lmt}
\end{eqnarray}
Thus (\ref{eqn:dst_mlk}) behaves like (\ref{eqn:dst_mlk_trnc})
everywhere except near the truncation points $\lnstrmax/\lnstrrat$ 
and $\lnstrmax$.
Beyond these points, (\ref{eqn:dst_mlk}) is sharply tapered so that
lines in these regions do not influence the statistics of the line
strength distribution very much.

\subsubsection[Normalization]{Normalization}\label{sxn:lnstr_dst_nrm}
The normalization properties of the line strength distribution
functions will now be shown.
The exponential distribution (\ref{eqn:dst_xpn}) is easily shown to be  
normalized 
\begin{eqnarray}
\int_{0}^{\infty} \lnstravg^{-1} \, \me^{-\lnstr/\lnstravg} \,
\dfr\lnstr 
& = &
\lnstravg^{-1} \left[ -\lnstravg \me^{-\lnstr/\lnstravg} \right]_{0}^{\infty} \nonumber \\
& = &
- 0 - (-1) = 1
\label{eqn:xpn_nrm}
\end{eqnarray}

For heuristic purposes we first demonstrate that the approximate
Malkmus distribution (\ref{eqn:dst_mlk_apx}) is not normalizeable
despite its apparent simplicity.
With the substitution $\xxx = \lnstr/\lnstravg$ we have
\begin{eqnarray}
\int_{0}^{\infty} \lnstr^{-1} \, \me^{-\lnstr/\lnstravg} \,
\dfr\lnstr 
& = &
\int_{0}^{\infty} \xxx^{-1} \, \me^{-\xxx} \,\dfr\xxx 
\label{eqn:mlk_apx_nrm}
\end{eqnarray}
The RHS resembles the complete \trmidx{gamma function} of zero,
$\gmmfnc(0)$ (\ref{eqn:wbl_gmm_dfn}).  
However, $\gmmfnc(0)$ is indeterminate, a singularity between negative
and positive infinities much like $\tan \frac{\mpi}{2}$.
Thus (\ref{eqn:dst_mlk_apx}) is not normalizeable.

Next we consider the truncated Malkmus distribution
(\ref{eqn:dst_mlk_lmt}).
We integrate over the non-zero region, $\lnstrmax/\lnstrrat$ to  
$\lnstrmax$, to demonstrate its normalization properties
\begin{eqnarray}
\int_{\lnstrmax/\lnstrrat}^{\lnstrmax} 
(\lnstr \ln \lnstrrat)^{-1} \,\dfr\lnstr
& = &
(\ln \lnstrrat)^{-1} 
[ \ln \lnstr]_{\lnstrmax/\lnstrrat}^{\lnstrmax} \nonumber \\
& = &
(\ln \lnstrrat)^{-1} 
[ \ln \lnstrmax - \ln (\lnstrmax/\lnstrrat) ] \nonumber \\
& = &
(\ln \lnstrrat)^{-1} 
[ \ln \lnstrmax - \ln \lnstrmax + \ln \lnstrrat ] \nonumber \\
& = & 1
\label{eqn:mlk_lmt_nrm}
\end{eqnarray}

It is more challenging to prove that the full Malkmus distribution
(\ref{eqn:dst_mlk}) is normalized. 
To begin, we simplify $\pdfoflnstr$ (\ref{eqn:dst_mlk}) with the
substituions $\aaa = \lnstrmax^{-1}$, $\bbb = \lnstrrat
\lnstrmax^{-1}$, and thus $\lnstrrat = \bbb/\aaa$.
\begin{eqnarray}
% Mal67 p. 324 (6) Lio92 p. 56 (2.4.27)
\int_{0}^{\infty} \pdfoflnstr \,\dfr\lnstr 
& = &
(\ln \lnstrrat)^{-1} \int_{0}^{\infty} 
\lnstr^{-1} ( \me^{-\aaa \lnstr} - \me^{-\bbb \lnstr} ) 
\,\dfr\lnstr 
\label{eqn:mlk_nrm_1}
\end{eqnarray}
We must show, therefore, that this rather complex definite integral
always equals $\ln (\bbb/\aaa) = \ln \lnstrrat$. 
Since both terms in the integrand are of the same form,
we shall explicitly evaluate $\lnstr^{-1} \me^{-\aaa \lnstr}$
before subtracting the analogous term involving $\bbb$.
These terms are amenable to integration by parts with the change of
variables 
\begin{eqnarray}
\uuu & = & \me^{-\aaa\lnstr} \nonumber \\
\dfr\uuu & = & -\aaa \me^{\aaa\lnstr} \,\dfr\lnstr \nonumber \\
\dfr\vvv & = & \lnstr^{-1} \,\dfr\lnstr \nonumber \\
\vvv & = & \ln \lnstr
\label{eqn:cov_mlk}
\end{eqnarray}
The result is
\begin{eqnarray}
% Mal67 p. 324 (6) Lio92 p. 56 (2.4.27)
\ln \lnstrrat \int_{0}^{\infty} \lnstr^{-1} \me^{-\aaa \lnstr} \,\dfr\lnstr 
& = &
\left[ \me^{-\aaa\lnstr} \ln \lnstr \right]_{0}^{\infty}
- \int_{0}^{\infty} \ln \lnstr (-\aaa \me^{-\aaa\lnstr}) \,\dfr\lnstr
\nonumber \\
& = &
\left[ \me^{-\aaa\lnstr} \ln \lnstr \right]_{0}^{\infty}
+ \aaa \int_{0}^{\infty} \me^{-\aaa\lnstr} \ln \lnstr \,\dfr\lnstr
\nonumber
\label{eqn:mlk_nrm_2}
\end{eqnarray}
The first term on the RHS contains a singularity since $\ln (0)$ is
undefined. 
This term however, will disappear once the similar term arising from
the integration of $\int \lnstr^{-1} \me^{-\bbb \lnstr} \,\dfr\lnstr$
is subtracted from (\ref{eqn:mlk_nrm_2}).
Now we change variables to isolate the parameter $\aaa$ from the dummy
variable of integration. 
Letting $\xxx = \aaa\lnstr$, $\lnstr = \xxx/\aaa$, $\dfr\xxx =
\aaa\,\dfr\lnstr$, and $\dfr\lnstr = \aaa^{-1}\,\dfr\xxx$ leads to
\begin{eqnarray}
\ln \lnstrrat \int_{0}^{\infty} \lnstr^{-1} \me^{-\aaa \lnstr} \,\dfr\lnstr 
& = &
\left[ \me^{-\aaa\lnstr} \ln \lnstr \right]_{0}^{\infty}
+ \aaa \int_{0}^{\infty} \me^{-\xxx} \ln (\xxx/\aaa) \aaa^{-1} \,\dfr\xxx
\nonumber \\
& = &
\left[ \me^{-\aaa\lnstr} \ln \lnstr \right]_{0}^{\infty}
+ \int_{0}^{\infty} \me^{-\xxx} \ln \xxx - \me^{-\xxx} \ln \aaa \,\dfr\xxx
\nonumber \\
& = &
\left[ \me^{-\aaa\lnstr} \ln \lnstr \right]_{0}^{\infty}
+ \int_{0}^{\infty} \me^{-\xxx} \ln \xxx \,\dfr\xxx
- \ln \aaa \left[ - \me^{-\xxx} \right]_{0}^{\infty}
\nonumber \\
& = &
\left[ \me^{-\aaa\lnstr} \ln \lnstr \right]_{0}^{\infty}
+ \int_{0}^{\infty} \me^{-\xxx} \ln \xxx \,\dfr\xxx
- \ln \aaa [ - 0 - (- 1 ) ] 
\nonumber \\
& = &
\left[ \me^{-\aaa\lnstr} \ln \lnstr \right]_{0}^{\infty}
+ \int_{0}^{\infty} \me^{-\xxx} \ln \xxx \,\dfr\xxx
- \ln \aaa
\label{eqn:mlk_nrm_3}
\end{eqnarray}
There is no closed form solution to the second term on the RHS, 
$\int \me^{-\xxx} \ln \xxx \,\dfr\xxx$, which is very close in
appearance to the original integral (\ref{eqn:mlk_nrm_1}).
However, this term will also disappear once the integration involving
$\bbb$ is subtracted from (\ref{eqn:mlk_nrm_3}).
Multiplying both sides by $(\ln \lnstrrat)^{-1}$, and subtracting
the integration involving $\bbb$ leads to
\begin{eqnarray}
\int_{0}^{\infty} \pdfoflnstr \,\dfr\lnstr 
& = &
\frac{1}{\ln\lnstrrat} \bigg\{
\left[ \me^{-\aaa\lnstr} \ln \lnstr \right]_{0}^{\infty}
-\left[ \me^{-\bbb\lnstr} \ln \lnstr \right]_{0}^{\infty}
\nonumber \\ & & {} % KoD99 p. 138 describes long equations and avoiding unary + spacing
+ \int_{0}^{\infty} \me^{-\xxx} \ln \xxx \,\dfr\xxx
- \int_{0}^{\infty} \me^{-\xxx} \ln \xxx \,\dfr\xxx
- \ln \aaa
+ \ln \bbb \bigg\} \nonumber \\
& = &
\frac{\ln \bbb - \ln \aaa}{\ln \lnstrrat} =
\frac{\ln (\bbb/\aaa)}{\ln \lnstrrat} =
\frac{\ln \lnstrrat}{\ln \lnstrrat} \nonumber \\
& = & 1 
\label{eqn:mlk_nrm_4}
\end{eqnarray}
All the like terms in the numerator cancelled in the first step with
the exception of $\ln \lnstrrat$.

\subsubsection[Mean Line Intensity]{Mean Line Intensity}\label{sxn:lnstr_dst_avg}
The \trmdfn{mean line intensity} $\lnstravg$ of a line strength
distribution $\pdfoflnstr$ is defined as
\begin{eqnarray}
% GoY89 p. 138 (4.29)
\lnstravg & = & \int_{0}^{\infty} \lnstr \pdfoflnstr \,\dfr\lnstr
\label{eqn:lnstr_avg_dfn}
\end{eqnarray}
$\lnstravg$ is an important statistic of a line strength distribution 
and any realistic line strength distribution should adequately predict
$\lnstravg$ when compared to observed line strengths.
Thus many line strength distributions, e.g., the exponential
distribution (\ref{eqn:dst_xpn}) contain $\lnstravg$ in their
definition. 
The claim that $\lnstravg$ is, in fact, the mean strength of these
distributions will now be proved.

The \trmidx{mean line intensity} of the exponential distribution
(\ref{eqn:dst_xpn}) is 
\begin{eqnarray}
% GoY89 p. 138 (4.29)
\int_{0}^{\infty} \lnstr \pdfoflnstr \,\dfr\lnstr & = & 
\int_{0}^{\infty} \frac{\lnstr}{\lnstravg} \, \me^{-\lnstr/\lnstravg} \,\dfr\lnstr
\nonumber \\
& = & \frac{1}{\lnstravg} \int_{0}^{\infty} \lnstr \me^{-\lnstr/\lnstravg} \,\dfr\lnstr \nonumber \\
& = & \frac{1}{\lnstravg} \int_{0}^{\infty} \lnstravg \xxx \me^{-\xxx}
\lnstravg \,\dfr\xxx \nonumber \\
& = & \lnstravg \int_{0}^{\infty} \xxx \me^{-\xxx} \,\dfr\xxx \nonumber \\
& = & \lnstravg \left[ -\xxx \me^{-\xxx} - \me^{-\xxx} \right]_{0}^{\infty} \nonumber \\
& = & \lnstravg \left[ -0 - 0 - (0 - 1) \right] \nonumber \\
& = & \lnstravg
\label{eqn:lnstr_avg_xpn}
\end{eqnarray}
This demonstrates $\lnstravg$ is the mean line strength of the
exponential distribution.

The mean line intensity of the approximate Malkmus distribution,
(\ref{eqn:dst_mlk_apx}) is 
\begin{eqnarray}
% GoY89 p. 138 (4.29)
\int_{0}^{\infty} \lnstr \pdfoflnstr \,\dfr\lnstr & = & 
\int_{0}^{\infty} \frac{\lnstr}{\lnstr} \, \me^{-\lnstr/\lnstravg} \,\dfr\lnstr
\nonumber \\
& = & \int_{0}^{\infty} \me^{-\lnstr/\lnstravg} \,\dfr\lnstr \nonumber \\
& = & \left. -\lnstravg \me^{-\lnstr/\lnstravg} \right|_{0}^{\infty} \nonumber \\
& = & -\lnstravg ( \me^{-\infty/\lnstravg} - \me^{-0/\lnstravg} ) \nonumber \\
& = & \lnstravg
\label{eqn:lnstr_avg_mlk_apx}
\end{eqnarray}
This proves that $\lnstravg$ is also the mean line strength in the 
approximate Malkmus line distribution function
(\ref{eqn:dst_mlk_apx}).  

The \trmidx{mean line intensity} of the full Malkmus distribution
is derived directly from (\ref{eqn:dst_mlk})
\begin{eqnarray}
% GoY89 p. 138 (4.29) Lio92 p. 56 (2.4.28)
\int_{0}^{\infty} \lnstr \pdfoflnstr \,\dfr\lnstr 
& = & 
\int_{0}^{\infty} \lnstr \times
\frac{\me^{-\lnstr/\lnstrmax} - \me^{-\lnstrrat \lnstr / \lnstrmax }}
{\lnstr \ln \lnstrrat}
\,\dfr\lnstr \nonumber \\
& = & 
(\ln \lnstrrat)^{-1} 
\int_{0}^{\infty} 
\me^{-\lnstr/\lnstrmax} - \me^{-\lnstrrat \lnstr / \lnstrmax }
\,\dfr\lnstr \nonumber \\
& = & 
\frac{1}{\ln \lnstrrat}
\left[ 
-\lnstrmax \me^{-\lnstr/\lnstrmax} +
\frac{\lnstrmax}{\lnstrrat} \me^{-\lnstrrat \lnstr / \lnstrmax }
\right]_{0}^{\infty} 
\nonumber \\
& = & 
\frac{\lnstrmax}{\ln \lnstrrat}
\left[ -0 + 0 - (-1) - (\lnstrrat^{-1}) \right]
\nonumber \\
& = & 
\frac{\lnstrmax}{\ln \lnstrrat}
( 1 - \lnstrrat^{-1} )
=
\frac{\lnstrmax}{\ln \lnstrrat}
\frac{\lnstrrat - 1}{\lnstrrat}
\nonumber \\
& = & 
\frac{\lnstrrat - 1}{\lnstrrat \ln \lnstrrat} \lnstrmax
\label{eqn:lnstr_avg_mlk}
\end{eqnarray}
This expression for $\lnstravg$ does not converge in the limit as  
$\lnstrrat \rightarrow \infty$. 
Thus the mean line strength of the Malkmus distribution
(\ref{eqn:dst_mlk}) depends on the parameter $\lnstrrat$
(\ref{eqn:lnstr_rat}), the ratio between the strongest and weakest
lines in a given band.
It is clear from (\ref{eqn:lnstr_avg_mlk}) that $\lnstravg$
decreases as $\lnstrrat$ increases.
Adequately representing the importance of weak lines is one 
of the chief advantages of the Malkmus distribution relative to other 
distributions. 

The maximum line strength intensity $\lnstrmax$ may be expressed
in terms of the mean line strength $\lnstr$ by inverting
(\ref{eqn:lnstr_avg_mlk}) 
\begin{eqnarray}
% GoY89 p. 138 (4.29) Lio92 p. 56 (2.4.28)
\lnstrmax
& = & 
\frac{\lnstrrat \ln \lnstrrat}{\lnstrrat - 1}\lnstravg
\label{eqn:lnstr_max_mlk}
\end{eqnarray}

\subsubsection[Mean Absorptance of Line Distribution]{Mean Absorptance of Line Distribution}\label{sxn:mlk_abs}
This section develops the concept of mean absorptance of a band of
non-overlapping lines, whereas \S\ref{sxn:abs_avg} derived the mean
absorptance $\absavg$ of an isolated Lorentz line. 
Since the lines are assumed to be non-overlapping, we may weight
the contribution of each line to the mean absorptance
(\ref{eqn:abs_avg_dfn}) by the normalized line strength probability 
(\ref{eqn:pdf_lnstr_nrm}) 
\begin{eqnarray}
% GoY89 p. 138 (4.24) ThS99 p. 390 (10.14)
\eqvwthavg = \absavg \mls & = & 
\int_{0}^{\infty} \pdfoflnstr \left(
\int_{-\infty}^{+\infty} \absbm \,\dfr\frq 
\right) \,\dfr\lnstr \nonumber \\
& = & 
\int_{0}^{\infty} \pdfoflnstr
\int_{-\infty}^{+\infty} 
1 - \exp[ - \lnstr \lnshp \abspth ]
\,\dfr\frq 
\,\dfr\lnstr
\label{eqn:abs_lndst_dfn}
\end{eqnarray}

Inserting the exponential line strength distribution function
(\ref{eqn:dst_xpn}) for  $\pdfoflnstr$ in (\ref{eqn:abs_lndst_dfn}) 
and interchanging orders of integration yields
\begin{eqnarray}
% GoY89 p. 140 (4.32) ThS99 p. 391 (10.19)
\eqvwthavg = \absavg \mls 
& = & 
\int_{0}^{\infty} \lnstravg^{-1} \, \me^{-\lnstr/\lnstravg}
\int_{-\infty}^{+\infty} 
1 - \exp[ - \lnstr \lnshp \abspth ]
\,\dfr\frq 
\,\dfr\lnstr 
\nonumber \\
& = & 
\lnstravg^{-1} 
\int_{-\infty}^{+\infty} 
\int_{0}^{\infty} \me^{-\lnstr/\lnstravg}
 - \exp \left( -\frac{\lnstr}{\lnstravg} - \lnstr \lnshp \abspth
\right) 
\,\dfr\lnstr 
\,\dfr\frq 
\nonumber \\
& = & 
\lnstravg^{-1} 
\int_{-\infty}^{+\infty} 
\int_{0}^{\infty} \me^{-\lnstr/\lnstravg}
 - \exp \left( -\lnstr ( \lnstravg^{-1} + \abspth \lnshp ) \right) 
\,\dfr\lnstr 
\,\dfr\frq 
\nonumber \\
& = & 
\lnstravg^{-1} 
\int_{-\infty}^{+\infty} 
\left[ -\lnstravg \me^{-\lnstr/\lnstravg}
+ ( \lnstravg^{-1} + \abspth \lnshp )^{-1}
\exp \left( -\lnstr ( \lnstravg^{-1} + \abspth \lnshp ) \right) 
\right]_{0}^{\infty}
\,\dfr\frq 
\nonumber \\
& = & 
\lnstravg^{-1} 
\int_{-\infty}^{+\infty} 
\left[ 0 + 0 - \left(-\lnstravg \times 1 + 
( \lnstravg^{-1} + \abspth \lnshp )^{-1} \times 1 \right) \right]
\,\dfr\frq 
\nonumber \\
& = & 
\lnstravg^{-1} 
\int_{-\infty}^{+\infty} 
\lnstravg - ( \lnstravg^{-1} + \abspth \lnshp )^{-1}
\,\dfr\frq 
\nonumber \\
& = & 
\lnstravg^{-1} 
\int_{-\infty}^{+\infty} 
\lnstravg - \frac{\lnstravg}{1 + \lnstravg \abspth \lnshp}
\,\dfr\frq 
\nonumber \\
& = & 
\int_{-\infty}^{+\infty} 
1 - \frac{1}{1 + \lnstravg \abspth \lnshp}
\,\dfr\frq 
\nonumber \\
& = & 
\int_{-\infty}^{+\infty} 
\frac{1 + \lnstravg \abspth \lnshp - 1}{1 + \lnstravg \abspth \lnshp}
\,\dfr\frq 
\nonumber \\
& = & 
\int_{-\infty}^{+\infty} 
\frac{\lnstravg \abspth \lnshp}{1 + \lnstravg \abspth \lnshp}
\,\dfr\frq 
\label{eqn:abs_xpn_ntg}
\end{eqnarray}
Thus the mean absorptance of the exponential line distribution 
depends straightforwardly on the line shape function.
Substituting the Lorentz line shape 
$\lnshplrnoffrq = \hwhmlrn/[\mpi ( \frq^{2} + \hwhmlrn^{2} ) ]$
(\ref{eqn:lnshp_lrn_dfn}) into (\ref{eqn:abs_xpn_ntg}) we obtain
\begin{eqnarray}
% GoY89 p. 141 (4.36) ThS99 p. 391 (10.20) Mal67 p. 325 (22)
\eqvwthavg = \absavg \mls 
& = & 
\int_{-\infty}^{+\infty} 
\frac{\hwhmlrn}{\mpi (\frq^{2} + \hwhmlrn^{2})} \times
\frac{\lnstravg \abspth}{1 + \lnstravg \abspth 
\frac{\hwhmlrn}{\mpi ( \frq^{2} + \hwhmlrn^{2} )}}
\,\dfr\frq \nonumber \\
& = & 
\frac{\lnstravg \abspth \hwhmlrn}{\mpi}
\int_{-\infty}^{+\infty} 
\frac{1}{\frq^{2} + \hwhmlrn^{2}} \times
\frac{1}{\frac{\mpi ( \frq^{2} + \hwhmlrn^{2} ) + 
\lnstravg \abspth \hwhmlrn}{\mpi ( \frq^{2} + \hwhmlrn^{2} )}}
\,\dfr\frq \nonumber \\
& = & 
\frac{\lnstravg \abspth \hwhmlrn}{\mpi}
\int_{-\infty}^{+\infty} 
\frac{1}{\frq^{2} + \hwhmlrn^{2}} \times
\frac{\mpi ( \frq^{2} + \hwhmlrn^{2} )}{\mpi ( \frq^{2} + \hwhmlrn^{2} ) + 
\lnstravg \abspth \hwhmlrn}
\,\dfr\frq \nonumber \\
& = & 
\lnstravg \abspth \hwhmlrn
\int_{-\infty}^{+\infty} 
\frac{1}{\mpi ( \frq^{2} + \hwhmlrn^{2} ) + 
\lnstravg \abspth \hwhmlrn}
\,\dfr\frq \nonumber \\
& = & 
\frac{\lnstravg \abspth \hwhmlrn}{\mpi}
\int_{-\infty}^{+\infty} 
\frac{1}{\frq^{2} + \hwhmlrn^{2} + \mpi^{-1} \lnstravg \abspth \hwhmlrn}
\,\dfr\frq \nonumber \\
\label{eqn:abs_xpn_lrn1}
\end{eqnarray}
Letting 
$\aaa = \sqrt{\hwhmlrn^{2} + \mpi^{-1} \lnstravg \abspth \hwhmlrn}
= \hwhmlrn \sqrt{1 + \lnstravg \abspth /(\mpi \hwhmlrn)}$ 
and noting that 
$\int (\xxx^{2} + \aaa^{2})^{-1} \,\dfr\xxx = \aaa^{-1} \tan^{-1} (\xxx/\aaa)$,
\begin{eqnarray}
\eqvwthavg = \absavg \mls 
& = & 
\frac{\lnstravg \abspth \hwhmlrn}{\mpi}
\int_{-\infty}^{+\infty} \frac{1}{\frq^{2} + \aaa^{2}}
\,\dfr\frq \nonumber \\
& = & 
\frac{\lnstravg \abspth \hwhmlrn}{\mpi}
\frac{1}{\aaa} \left[ \tan^{-1} \frac{\xxx}{\aaa}
\right]_{-\infty}^{+\infty} \nonumber \\
& = & 
\frac{\lnstravg \abspth \hwhmlrn}{\mpi}
\left[ \hwhmlrn \left(1 + \frac{\lnstravg \abspth}{\mpi \hwhmlrn} \right)^{1/2} \right]^{-1} 
\left[ \frac{\mpi}{2} - \left(-\frac{\mpi}{2} \right) \right] \nonumber \\
& = & 
\lnstravg \abspth
\left(1 + \frac{\lnstravg \abspth}{\mpi \hwhmlrn} \right)^{-1/2}
\nonumber \\
\frac{\eqvwthavg}{\mls} = \absavg 
& = & 
% Mal67 p. 325 (22) GoY89 p. 141 (4.36) ThS99 p. 391 (10.20) 
\frac{\lnstravg \abspth}{\mls}
\left(1 + \frac{\lnstravg \abspth}{\mpi \hwhmlrn} \right)^{-1/2}
\label{eqn:abs_xpn_lrn}
\end{eqnarray}
In addition to $\abspth$, the mean band absorptance $\absavg$ 
depends on only two statistics of the band, namely $\lnstravg/\mls$
and $\lnstravg/(\mpi \hwhmlrn)$.

We recall that (\ref{eqn:abs_xpn_lrn}) is based on the idealized,
continuous line strength distribution (\ref{eqn:dst_xpn}) rather than
on real (observed) line strengths.
Observed line strengths and positions are tabulated in the
\trmidx{\acr{HITRAN}} database (\S\ref{sxn:htrn}).
Therefore we may compute any desired statistics of the actual line
strength distribution.
Of particular interest is the observed mean absorptance of a band
in both the weak and strong line limits.
Summing (\ref{eqn:abs_avg_wll}) and (\ref{eqn:abs_avg_sll}) over
the $\lnnbr$ lines in the given band
\begin{eqnarray}
% GoY89 p. 142 (4.41) (4.42)
\eqvwth = \absavg \mls & = & \left\{
\begin{array}{r@{\quad:\quad}ll}
\lnnbr^{-1} \sum_{\lnidx = 1}^{\lnnbr} \lnstrln \abspth
& \lnstr \abspth \ll 1 \\
\lnnbr^{-1} \sum_{\lnidx = 1}^{\lnnbr} 2 \sqrt{ \lnstrln \abspth \hwhmlrnln } 
& \lnstr \abspth \gg 1
\end{array} \right.
\label{eqn:abs_avg_lmt_htrn}
\end{eqnarray}
The weak and strong line limits of the mean absorptance of the 
exponential line strength distribution function
(\ref{eqn:abs_xpn_lrn}) are
\begin{subequations}
% Mal67 p. 325 (22) GoY89 p. 141 (4.44) ThS99 p. 391 (10.20) 
\label{eqn:abs_xpn_lrn_lmt}
\begin{align}
\label{eqn:abs_xpn_lrn_wll}
\lim_{\lnstravg \abspth \ll 1} \eqvwthavg = \absavg \mls & \approx 
\lnstravg \abspth \\
\label{eqn:abs_xpn_lrn_sll}
\lim_{\lnstravg \abspth \gg 1} \eqvwthavg = \absavg \mls & \approx 
\lnstravg \abspth
\left(\frac{\lnstravg \abspth}{\mpi \hwhmlrn} \right)^{-1/2}
= \sqrt{ \mpi \lnstravg \abspth \hwhmlrn }
\end{align}
\end{subequations}
We now require (\ref{eqn:abs_xpn_lrn_lmt}) to equal
(\ref{eqn:abs_avg_lmt_htrn}) in the appropriate limits.
In the weak line limit
\begin{eqnarray}
% GoY89 p. 141 (4.44) (4.45)
\lnstravg \abspth & = & 
\lnnbr^{-1} \sum_{\lnidx = 1}^{\lnnbr} \lnstrln \abspth \nonumber \\
\frac{\lnstravg}{\mls}
& = & 
(\lnnbr\mls)^{-1} \sum_{\lnidx = 1}^{\lnnbr} \lnstrln
\end{eqnarray}
In the strong line limit
\begin{eqnarray}
% GoY89 p. 141 (4.44) (4.45)
\sqrt{ \mpi \lnstravg \abspth \hwhmlrn } 
& = & 
\lnnbr^{-1} \sum_{\lnidx = 1}^{\lnnbr} 2 \sqrt{ \lnstrln \abspth \hwhmlrnln } 
\nonumber \\
\mpi \lnstravg \abspth \hwhmlrn
& = & 
\frac{4 \abspth}{\lnnbr^{2}} \left( \sum_{\lnidx = 1}^{\lnnbr} \sqrt{ \lnstrln \hwhmlrnln } \right)^{2}
\nonumber \\
\frac{1}{\mpi \lnstravg \hwhmlrn }
& = & 
\frac{\lnnbr^{2}}{4} \left( \sum_{\lnidx = 1}^{\lnnbr} \sqrt{ \lnstrln \hwhmlrnln } \right)^{-2}
\nonumber \\
\frac{1}{\mpi \lnstravg \hwhmlrn } \times \lnstravg^{2}
& = & 
\lnstravg^{2} \times \frac{\lnnbr^{2}}{4} \left( \sum_{\lnidx = 1}^{\lnnbr} \sqrt{ \lnstrln \hwhmlrnln } \right)^{-2}
\nonumber \\
\frac{\lnstravg}{\mpi \hwhmlrn }
& = & 
\frac{1}{\lnnbr^{2}} \left( \sum_{\lnidx = 1}^{\lnnbr} \lnstrln \right)^{2}
\times \frac{\lnnbr^{2}}{4} \left( \sum_{\lnidx = 1}^{\lnnbr} \sqrt{ \lnstrln \hwhmlrnln } \right)^{-2}
\nonumber \\
\frac{\lnstravg}{\mpi \hwhmlrn}
& = & 
\frac{1}{4} \left( \sum_{\lnidx = 1}^{\lnnbr} \lnstrln \bigg/
\sum_{\lnidx = 1}^{\lnnbr} \sqrt{ \lnstrln \hwhmlrnln } \right)^{2}
\end{eqnarray}

The tractable analytic form of (\ref{eqn:abs_xpn_lrn}) led to its
adoption as the dominant formulation for band models for many years.
The mean band absorptances $\absavg$ of other line strength
distributions have forms similar to (\ref{eqn:abs_xpn_lrn}).

Inserting the Malkmus line strength distribution function
(\ref{eqn:dst_mlk}) for  $\pdfoflnstr$ in (\ref{eqn:abs_lndst_dfn})  
and interchanging orders of integration yields
\begin{eqnarray}
% GoY89 p. 140 (4.34) ThS99 p. 391 (10.21) Lio92 p. 56 (2.4.29)
\eqvwthavg = \absavg \mls 
& = & 
\int_{0}^{\infty} 
\frac{\me^{-\lnstr/\lnstrmax} - \me^{-\lnstrrat \lnstr / \lnstrmax }}
{\lnstr \ln \lnstrrat} 
\int_{-\infty}^{+\infty} 
1 - \exp (- \lnstr \lnshp \abspth)
\,\dfr\frq 
\,\dfr\lnstr 
\nonumber \\
& = & 
\frac{1}{\ln \lnstrrat}
\int_{-\infty}^{+\infty} 
\int_{0}^{\infty} 
\frac{1}{\lnstr} ( \me^{-\lnstr/\lnstrmax} 
 - \me^{-\lnstr \lnstrrat / \lnstrmax} )
( 1 - \me^{- \lnstr \lnshp \abspth}) 
\,\dfr\lnstr 
\,\dfr\frq 
\nonumber \\
& = & 
\frac{1}{\ln \lnstrrat}
\int_{-\infty}^{+\infty} 
\int_{0}^{\infty} 
\frac{\me^{-\lnstr/\lnstrmax}}{\lnstr}
-\frac{\me^{-\lnstr \lnstrrat / \lnstrmax}}{\lnstr}
+\frac{\me^{-\lnstr(\lnshp \abspth + \lnstrrat / \lnstrmax)}}{\lnstr}
-\frac{\me^{-\lnstr(\lnshp \abspth + 1/\lnstrmax)}}{\lnstr}
\,\dfr\lnstr 
\,\dfr\frq 
\nonumber
\label{eqn:abs_mlk_ntg_1}
\end{eqnarray}
The first two terms and the last two terms in the integrand are both
of the form $\lnstr^{-1}(\me^{-\aaa\lnstr}-\me^{-\bbb\lnstr})$.
In proving the Malkmus distribution is correctly normalized 
(\ref{eqn:mlk_nrm_1})--(\ref{eqn:mlk_nrm_4}) we showed the value of
this type of definite integral is $\ln (\bbb/\aaa)$.
For the first two terms, 
$\aaa = \lnstrmax^{-1}$ and 
$\bbb = \lnstrrat/\lnstrmax$.
For the final two terms, 
$\aaa = \lnshp \abspth + \lnstrrat / \lnstrmax$ and 
$\bbb = \lnshp \abspth + 1/\lnstrmax$.
Therefore the integration over $\lnstr$ is complete and results in
\begin{eqnarray}
% GoY89 p. 140 (4.34) ThS99 p. 391 (10.21) Lio92 p. 56 (2.4.29)
\eqvwthavg = \absavg \mls 
& = & 
\frac{1}{\ln \lnstrrat}
\int_{-\infty}^{+\infty} 
\ln \lnstrrat + \ln \left( 
\frac{\lnshp \abspth + 1 / \lnstrmax}
{\lnshp \abspth + \lnstrrat / \lnstrmax}
\right) \,\dfr\frq 
\nonumber \\
& = & 
\frac{1}{\ln \lnstrrat}
\int_{-\infty}^{+\infty} 
\ln \lnstrrat + \ln \left( 
\frac{1+ \lnstrmax \lnshp \abspth}
{\lnstrrat + \lnstrmax \lnshp \abspth}
\right) \,\dfr\frq 
\nonumber \\
& = & 
\frac{1}{\ln \lnstrrat}
\int_{-\infty}^{+\infty} 
\ln \lnstrrat + \ln \left[ 
\left( \frac{1}{\lnstrrat} \right)
\frac{1 + \lnstrmax \lnshp \abspth}
{1 + \lnstrmax \lnshp \abspth / \lnstrrat}
\right] \,\dfr\frq 
\nonumber \\
& = & 
\frac{1}{\ln \lnstrrat}
\int_{-\infty}^{+\infty} 
\ln \lnstrrat + \ln \left( 
\frac{1 + \lnstrmax \lnshp \abspth}
{1 + \lnstrmax \lnshp \abspth / \lnstrrat}
\right) - \ln \lnstrrat \,\dfr\frq 
\nonumber \\
& = & 
\frac{1}{\ln \lnstrrat}
\int_{-\infty}^{+\infty} 
\ln \left( 
\frac{1 + \lnstrmax \lnshp \abspth}
{1 + \lnstrmax \lnshp \abspth / \lnstrrat}
\right) \,\dfr\frq 
\label{eqn:abs_mlk_ntg}
\end{eqnarray}
The mean absorptance of the Malkmus distribution also depends
simply on the line shape $\lnshpoffrq$. 
The mean absorptance of the exponential line distribution
(\ref{eqn:abs_xpn_ntg}) is analogous to the Malkmus result
(\ref{eqn:abs_mlk_ntg}).
Note that the former is a rational function of the mean line intensity 
$\lnstravg$, while the latter depends logarithmically on $\lnstravg$.
Moreover, the mean absorptance of the Malkmus distribution depends on 
the additional parameter $\lnstrrat$, the ratio between the maximum
and minimum line strengths with the band.
The dependence is problematic in that (\ref{eqn:abs_mlk_ntg}) is
not self-evidently convergent as $\lnstrrat \rightarrow \infty$.
We now show that for the most important line shapes, 
\absavg (\ref{eqn:abs_mlk_ntg}) does converge in this limit.

Substituting the Lorentz line shape 
$\lnshplrnoffrq = \hwhmlrn/[\mpi ( \frq^{2} + \hwhmlrn^{2} ) ]$
(\ref{eqn:lnshp_lrn_dfn}) into (\ref{eqn:abs_xpn_ntg}) we obtain
\begin{eqnarray}
% GoY89 p. 140 (4.34) ThS99 p. 391 (10.21) Lio92 p. 56 (2.4.30)
\eqvwthavg = \absavg \mls 
& = & 
\frac{1}{\ln \lnstrrat}
\int_{-\infty}^{+\infty} 
\ln \left( 
\frac{1 + \frac{\lnstrmax \abspth \hwhmlrn}{\mpi (\frq^{2} + \hwhmlrn^{2})}}
{1 + \frac{\lnstrmax \abspth \hwhmlrn}{\mpi (\frq^{2} + \hwhmlrn^{2})\lnstrrat}}
\right) \,\dfr\frq \nonumber \\
& = & 
\frac{1}{\ln \lnstrrat}
\int_{-\infty}^{+\infty} 
\ln \left( 
\frac{\lnstrmax \abspth \hwhmlrn + \mpi (\frq^{2} + \hwhmlrn^{2})}
{\lnstrmax \abspth \hwhmlrn \lnstrrat^{-1} + \mpi (\frq^{2} + \hwhmlrn^{2})}
\right) \,\dfr\frq \nonumber \\
& = & 
\frac{1}{\ln \lnstrrat}
\int_{-\infty}^{+\infty} 
\ln \left( 
\frac{\frq^{2} + \hwhmlrn^{2} + \lnstrmax \abspth \hwhmlrn / \mpi}
{\frq^{2} + \hwhmlrn^{2} + \lnstrmax \abspth \hwhmlrn / (\mpi \lnstrrat)}
\right) \,\dfr\frq \nonumber \\
& = & 
\frac{1}{\ln \lnstrrat}
\int_{-\infty}^{+\infty} 
\ln \left( \frq^{2} + \hwhmlrn^{2} + \frac{\lnstrmax \abspth \hwhmlrn}{\mpi} \right)
- \ln \left( \frq^{2} + \hwhmlrn^{2} + \frac{\lnstrmax \abspth
\hwhmlrn}{\mpi \lnstrrat} \right)
\,\dfr\frq
\label{eqn:abs_mlk_lrn1}
\end{eqnarray}
We simplify (\ref{eqn:abs_mlk_lrn1}) by making the substitutions
$\ccc^{2} = \hwhmlrn^{2} + \lnstrmax \abspth \hwhmlrn / \mpi$ and
$\ddd^{2} = \hwhmlrn^{2} + \lnstrmax \abspth \hwhmlrn / ( \mpi
\lnstrrat)$, or, equivalenty
\begin{subequations}
\label{eqn:mlk_lrn_ccc_ddd}
\begin{align}
\label{eqn:mlk_lrn_ccc}
\ccc & = \hwhmlrn \sqrt{1 + \frac{\lnstrmax \abspth}{\mpi \hwhmlrn}} \\ 
\label{eqn:mlk_lrn_ddd}
\ddd & = \hwhmlrn \sqrt{1 + \frac{\lnstrmax \abspth \hwhmlrn}{\mpi \hwhmlrn \lnstrrat}}
\end{align}
\end{subequations}
With these definitions, (\ref{eqn:abs_mlk_lrn1}) becomes
\begin{eqnarray}
% GoY89 p. 140 (4.34) ThS99 p. 391 (10.21) Lio92 p. 56 (2.4.30)
\eqvwthavg = \absavg \mls 
& = & 
\frac{1}{\ln \lnstrrat}
\int_{-\infty}^{+\infty} 
\ln ( \frq^{2} + \ccc^{2} ) - \ln ( \frq^{2} + \ddd^{2} )
\,\dfr\frq 
\label{eqn:abs_mlk_lrn2}
\end{eqnarray}
This is the difference of two identical, standard integrals
\cite[e.g.,][p.~205]{GrR65} and reduces to elementary functions  
\begin{eqnarray}
% GrR65 p. 205 (2.773) Lio92 p. 57 (2.4.30)
\int \ln (\frq^{2} + \aaa^{2}) \,\dfr\frq & = & 
\frq \ln ( \frq^{2} + \aaa^{2} ) - 2 \frq + 2 \frq \tan^{-1} \frac{\frq}{\aaa}
\label{eqn:GrR65_2.773}
\end{eqnarray}
It is straightforward to differentiate (\ref{eqn:GrR65_2.773}) and
verify the result.
Using (\ref{eqn:GrR65_2.773}) to complete the frequency integration in
(\ref{eqn:abs_mlk_lrn2}) results in
\begin{eqnarray}
% GoY89 p. 141 (4.38) ThS99 p. 391 (10.21) Lio92 p. 56 (2.4.31)
\eqvwthavg = \absavg \mls 
& = & 
(\ln \lnstrrat)^{-1} \left[ 
 \frq \ln ( \frq^{2} + \ccc^{2} ) - 2 \frq + 2 \ccc \tan^{-1} \frac{\frq}{\ccc}
-\frq \ln ( \frq^{2} + \ddd^{2} ) + 2 \frq - 2 \ddd \tan^{-1} \frac{\frq}{\ddd}
\right]_{-\infty}^{+\infty} \nonumber \\
& = & 
(\ln \lnstrrat)^{-1} \left[ 
 2 \ccc \tan^{-1} \frac{\frq}{\ccc} - 2 \ddd \tan^{-1} \frac{\frq}{\ddd}
\right]_{-\infty}^{+\infty} \nonumber
\label{eqn:abs_mlk_lrn3}
\end{eqnarray}
The mutual cancellation of the $2\frq$ terms is obvious, but the
equivalence of the logarithmic terms may not be self-evident.
Consider, however, that both $\ccc$ and $\ddd$ (\ref{eqn:mlk_lrn_ccc_ddd})
are finite-valued as $\frq \rightarrow \pm \infty$, and it is clear
that these terms must cancel. 
Evaluating the remaining terms leads to 
\begin{eqnarray}
% GoY89 p. 141 (4.38) ThS99 p. 391 (10.21) Lio92 p. 56 (2.4.31)
\eqvwthavg = \absavg \mls 
& = & 
(\ln \lnstrrat)^{-1} \left\{ 
 2 \ccc \frac{\mpi}{2} - 2 \ddd \frac{\mpi}{2} - \left[
 2 \ccc \left( - \frac{\mpi}{2} \right) - 
 2 \ddd \left( - \frac{\mpi}{2} \right) \right] 
\right\} \nonumber \\
& = & 
(\ln \lnstrrat)^{-1} 
( \mpi \ccc - \mpi \ddd + \mpi \ccc - \mpi \ddd ) \nonumber \\
& = & 
(\ln \lnstrrat)^{-1} 2 \mpi ( \ccc - \ddd ) \nonumber \\
& = & 
\frac{2 \mpi \hwhmlrn}{\ln \lnstrrat} 
\left( 
\sqrt{1 + \frac{\lnstrmax \abspth}{\mpi \hwhmlrn}} -
\sqrt{1 + \frac{\lnstrmax \abspth}{\mpi \hwhmlrn \lnstrrat}}
\right)
\label{eqn:abs_mlk_lrn4}
\end{eqnarray}
Although (\ref{eqn:abs_mlk_lrn4}) does convey the behavior of
the mean absorptance of the Malkmus band model with Lorentzian lines, 
the presence of the $\lnstrmax$ term and the $\lnstrrat$ term is
undesirable because their values are interrelated and depend 
on the specific gaseous absorption band being considered. 
It would be more convenient to express the mean absorptance
in terms of universal statistics applicable to all band models,
so that $\absavg$ and $\eqvwthavg$ may be easily intercompared among
band models.
The mean line strength $\lnstravg$ (\ref{eqn:lnstr_avg_dfn}) appears
in both the exponential (\ref{eqn:dst_xpn}) and Godson line strength
distributions.
We use (\ref{eqn:lnstr_max_mlk}) to replace $\lnstrmax$ by $\lnstravg$
in (\ref{eqn:abs_mlk_lrn4})  
\begin{eqnarray}
% GoY89 p. 141 (4.38) ThS99 p. 391 (10.21) Lio92 p. 56 (2.4.31)
\eqvwthavg = \absavg \mls 
& = & 
\frac{2 \mpi \hwhmlrn}{\ln \lnstrrat} 
\left( 
\sqrt{1 + \frac{\lnstravg \abspth}{\mpi \hwhmlrn} 
\frac{\lnstrrat \ln \lnstrrat}{\lnstrrat - 1}} -
\sqrt{1 + \frac{\lnstravg \abspth}{\mpi \hwhmlrn \lnstrrat}
\frac{\lnstrrat \ln \lnstrrat}{\lnstrrat - 1}}
\right) \nonumber \\
& = & 
\frac{2 \mpi \hwhmlrn}{\ln \lnstrrat} 
\left( 
\sqrt{1 + \frac{\lnstravg \abspth}{\mpi \hwhmlrn} 
\frac{\lnstrrat \ln \lnstrrat}{\lnstrrat - 1}} -
\sqrt{1 + \frac{\lnstravg \abspth}{\mpi \hwhmlrn}
\frac{\ln \lnstrrat}{\lnstrrat - 1}}
\right)
\label{eqn:abs_mlk_lrn}
\end{eqnarray}

The weak line limit of the mean absorptance of the Malkmus line
strength distribution function (\ref{eqn:abs_mlk_lrn}) is obtained
by expanding the radical into a Taylor series and retaining only the
first two terms
$\lim_{\xxx \ll 1} \sqrt{1 + \xxx} \approx 1 + \xxx/2$
\begin{eqnarray}
% Mal67 p. 325 (22) GoY89 p. 141 (4.38) ThS99 p. 391 (10.20) Lio92 p. 57 (2.4.32)
\lim_{\lnstravg \abspth \ll 1} \eqvwthavg = \absavg \mls & \approx &
\frac{2 \mpi \hwhmlrn}{\ln \lnstrrat} 
\left( 
1 + \frac{\lnstravg \abspth}{2 \mpi \hwhmlrn} 
\frac{\lnstrrat \ln \lnstrrat}{\lnstrrat - 1} -
1 - \frac{\lnstravg \abspth}{2 \mpi \hwhmlrn}
\frac{\ln \lnstrrat}{\lnstrrat - 1}
\right) \nonumber \\
& = &
\frac{2 \mpi \hwhmlrn}{\ln \lnstrrat} 
\times
\frac{\lnstravg \abspth \ln \lnstrrat (\lnstrrat - 1)}
{2 \mpi \hwhmlrn (\lnstrrat - 1)} 
\nonumber \\
& = &
\lnstravg \abspth
\label{eqn:abs_mlk_lrn_wll}
\end{eqnarray}
which is identical to (\ref{eqn:abs_xpn_lrn_wll}).
The strong line limit of (\ref{eqn:abs_mlk_lrn}) is
\begin{eqnarray}
% Mal67 p. 325 (22) GoY89 p. 141 (4.38) ThS99 p. 391 (10.20) Lio92 p. 57 (2.4.32)
\lim_{\lnstravg \abspth \gg 1} \eqvwthavg = \absavg \mls & \approx &
\frac{2 \mpi \hwhmlrn}{\ln \lnstrrat} 
\left( 
\sqrt{\frac{\lnstravg \abspth}{\mpi \hwhmlrn} 
\frac{\lnstrrat \ln \lnstrrat}{\lnstrrat - 1}} -
\sqrt{\frac{\lnstravg \abspth}{\mpi \hwhmlrn}
\frac{\ln \lnstrrat}{\lnstrrat - 1}}
\right) \nonumber \\
& = & 
\frac{2 \mpi \hwhmlrn}{\ln \lnstrrat} 
\sqrt{\frac{\lnstravg \abspth \ln \lnstrrat}{\mpi \hwhmlrn}}
\left(
\sqrt{\frac{\lnstrrat}{\lnstrrat - 1}} -
\sqrt{\frac{1}{\lnstrrat - 1}} \right)
\nonumber \\
& = & 
2 \left(\frac{\mpi \hwhmlrn \lnstravg \abspth}{\ln \lnstrrat}\right)^{1/2}
\left(
\frac{\sqrt{\lnstrrat}}{\sqrt{\lnstrrat - 1}} -
\frac{1}{\sqrt{\lnstrrat - 1}} \right)
\nonumber \\
& = & 
2\sqrt{\mpi \hwhmlrn \lnstravg \abspth} \times
\frac{\sqrt{\lnstrrat} - 1}{\sqrt{(\lnstrrat - 1) \ln \lnstrrat}}
\label{eqn:abs_mlk_lrn_sll1}
\end{eqnarray}
Let us encapsulate the $\lnstrrat$-dependence into the function
$\sllfctofrat$ and and also present $\sllfct^{2}(\lnstrrat)$ for
future use.
\begin{subequations}
\label{eqn:sll_fct}
\begin{align}
\label{eqn:sll_fct_dfn}
\sllfctofrat & =  
\frac{\sqrt{\lnstrrat} - 1}{\sqrt{(\lnstrrat - 1) \ln \lnstrrat}} \\
[\sllfctofrat]^{2} 
& =  
\frac{(\sqrt{\lnstrrat} - 1)(\sqrt{\lnstrrat} - 1)}
{(\lnstrrat - 1) \ln \lnstrrat} \nonumber \\
& =  
\frac{(\sqrt{\lnstrrat} - 1)(\sqrt{\lnstrrat} - 1)}
{(\sqrt{\lnstrrat} - 1)(\sqrt{\lnstrrat} + 1) \ln \lnstrrat} \nonumber \\
\label{eqn:sll_fct_sqr_dfn}
\sllfct^{2}(\lnstrrat) 
& =  
\frac{\sqrt{\lnstrrat} - 1}{(\sqrt{\lnstrrat} + 1) \ln \lnstrrat}
\end{align}
\end{subequations}
where we have made use of the factoring 
$\lnstrrat - 1 = (\sqrt{\lnstrrat} - 1)(\sqrt{\lnstrrat} + 1)$.
Rewriting (\ref{eqn:abs_mlk_lrn_sll1}) with $\sllfctofrat$
(\ref{eqn:sll_fct_dfn})
\begin{eqnarray}
\lim_{\lnstravg \abspth \gg 1} \eqvwthavg = \absavg \mls & \approx &
2\sqrt{\mpi \hwhmlrn \lnstravg \abspth} \sllfctofrat
\label{eqn:abs_mlk_lrn_sll2}
\end{eqnarray}
The strong line limit of the Malkmus distribution of Lorentzian lines  
(\ref{eqn:abs_mlk_lrn_sll2}), and the strong line limit of an
individual Lorentzian line (\ref{eqn:eqv_wth_lmt}), differ by the
factor $\sqrt{\mpi} \sllfctofrat$.

The advantage of encapsulating the $\lnstrrat$-dependence in the
$\sllfctofrat$ function is that the weak and strong line limits of the
Malkmus distribution may now be cast in exactly the same notation as
the exponential distribution (\ref{eqn:abs_xpn_lrn_lmt}). 
First introduce a transformed mean line intensity $\lnstravgmlk$ and
mean line spacing $\mlsmlk$ which are defined in terms of the line
strength ratio of the Malkmus distribution
\begin{subequations}
\label{eqn:mlk_lmt_cov}
\begin{align}
\label{eqn:lnstr_avg_mlk_dfn}
\lnstravgmlk & = \frac{\lnstravg}{\mpi \sllfct^{2}} \leftrightarrow 
\lnstravg = \mpi \lnstravgmlk \sllfct^{2} \\
\label{eqn:mls_mlk_dfn}
\mlsmlk & = \frac{\mls}{\mpi \sllfct^{2}} \leftrightarrow 
\mls = \mpi \mlsmlk \sllfct^{2}
\end{align}
\end{subequations}
The definitions (\ref{eqn:mlk_lmt_cov}) ensure that 
$\lnstravg/\mls = \lnstravgmlk/\mlsmlk$.
In terms of the original mean line strength and spacing, and the
transformed line strength and line spacing, the weak and strong line
limits of the mean absorptance $\absavg$ of the Malkmus line strength
distribution applied to Lorentzian lines (\ref{eqn:abs_mlk_lrn}) is
\begin{subequations}
\label{eqn:abs_mlk_lrn_lmt3}
\begin{align}
\label{eqn:abs_mlk_lrn_wll3}
\lim_{\lnstravg \abspth \ll 1} \eqvwthavg = \absavg \mls & \approx
\frac{\lnstravg \abspth}{\mls}
= \frac{\lnstravgmlk \abspth}{\mlsmlk} \\
\label{eqn:abs_mlk_lrn_sll3}
\lim_{\lnstravg \abspth \gg 1} \eqvwthavg = \absavg \mls & \approx
2\sqrt{\mpi \hwhmlrn \lnstravg \abspth} \sllfctofrat
= 2\sqrt{\hwhmlrn \lnstravgmlk \abspth}
\end{align}
\end{subequations}

We now rewrite the mean absorptance (\ref{eqn:abs_mlk_lrn}) in terms 
of the transformed parameters $\lnstravgmlk$ and $\mlsmlk$
(\ref{eqn:mlk_lmt_cov}), making use of (\ref{eqn:sll_fct_sqr_dfn})
\begin{eqnarray}
% GoY89 p. 141 (4.38) ThS99 p. 391 (10.21) Lio92 p. 56 (2.4.37)
\frac{\eqvwthavg}{\mls} = \absavg 
& = & 
\frac{2 \mpi \hwhmlrn}{\mls \ln \lnstrrat}
\left( 
\sqrt{1 + \frac{\lnstravg \abspth}{\mpi \hwhmlrn} 
\frac{\lnstrrat \ln \lnstrrat}{\lnstrrat - 1}} -
\sqrt{1 + \frac{\lnstravg \abspth}{\mpi \hwhmlrn}
\frac{\ln \lnstrrat}{\lnstrrat - 1}}
\right) \nonumber \\
& = & 
\frac{1}{\mpi \mlsmlk \sllfct^{2}}
\frac{2 \mpi \hwhmlrn}{\ln \lnstrrat}
\left( 
\sqrt{1 + \frac{\mpi \sllfct^{2} \lnstravgmlk \abspth}{\mpi \hwhmlrn} 
\frac{\lnstrrat \ln \lnstrrat}{\lnstrrat - 1}} -
\sqrt{1 + \frac{\mpi \sllfct^{2} \lnstravgmlk \abspth}{\mpi \hwhmlrn}
\frac{\ln \lnstrrat}{\lnstrrat - 1}}
\right) \nonumber \\
& = & 
\frac{(\sqrt{\lnstrrat} + 1) \ln \lnstrrat}{\sqrt{\lnstrrat} - 1}
\frac{2 \hwhmlrn}{\mlsmlk \ln \lnstrrat}
\left( 
\sqrt{1 + \frac{\lnstravgmlk \abspth}{\hwhmlrn} \times
\frac{\sqrt{\lnstrrat} - 1}{(\sqrt{\lnstrrat} + 1) \ln \lnstrrat} \times
\frac{\lnstrrat \ln \lnstrrat}{(\sqrt{\lnstrrat} - 1)(\sqrt{\lnstrrat} + 1)}} 
\right. \nonumber \\ & & \left. {} - % Spoof new equation
\sqrt{1 + \frac{\lnstravgmlk \abspth}{\hwhmlrn} \times
\frac{\sqrt{\lnstrrat} - 1}{(\sqrt{\lnstrrat} + 1) \ln \lnstrrat} \times
\frac{\ln \lnstrrat}{(\sqrt{\lnstrrat} - 1)(\sqrt{\lnstrrat} + 1)}}
\right) \nonumber \\
& = & 
\frac{2 \hwhmlrn}{\mlsmlk} \left(
\frac{\sqrt{\lnstrrat} + 1}{\sqrt{\lnstrrat} - 1} \right)
\left( 
\sqrt{1 + \frac{\lnstravgmlk \abspth}{\hwhmlrn} \times
\frac{\lnstrrat}{(\sqrt{\lnstrrat} + 1)^{2}}} -
\sqrt{1 + \frac{\lnstravgmlk \abspth}{\hwhmlrn} \times
\frac{1}{(\sqrt{\lnstrrat} + 1)^{2}}}
\right) \nonumber \\
& = & 
\frac{2 \hwhmlrn}{\mlsmlk} \left(
\frac{\sqrt{\lnstrrat} + 1}{\sqrt{\lnstrrat} - 1} \right)
\left( 
\sqrt{1 + \frac{\lnstravgmlk \abspth}
{\hwhmlrn (1 + 1/\sqrt{\lnstrrat})^{2}}} -
\sqrt{1 + \frac{\lnstravgmlk \abspth}
{\hwhmlrn (1 + \sqrt{\lnstrrat})^{2}}}
\right) \nonumber
\label{eqn:abs_mlk_lrn_trn}
\end{eqnarray}
The asymmetric nature of the optical path terms under the radicals in
(\ref{eqn:abs_mlk_lrn_trn}) is clear.
For large $\lnstrrat$, the first absorptance term approaches a
finite-valued function of $\lnstravgmlk$, $\abspth$, and $\hwhmlrn$. 
The second absorptance term, on the other hand, approaches unity.
Thus the mean absorptance $\absavg$ asymptotes to a finite value as
$\lnstrrat \rightarrow \infty$.
Formally, this amounts to taking the limit as $\sqrt{\lnstrrat} \gg 1$
\begin{eqnarray}
% GoY89 p. 141 (4.38) ThS99 p. 391 (10.21) Lio92 p. 56 (2.4.38)
\frac{\eqvwthavg}{\mls} = \absavg 
& \approx & 
\frac{2 \hwhmlrn}{\mlsmlk} \left(
\frac{\sqrt{\lnstrrat}}{\sqrt{\lnstrrat}} \right)
\left( 
\sqrt{1 + \frac{\lnstravgmlk \abspth}
{\hwhmlrn (1 + 0)}} -
\sqrt{1 + 0}
\right) \nonumber \\
& = & 
\frac{2 \hwhmlrn}{\mlsmlk}
\left( \sqrt{1 + \frac{\lnstravgmlk \abspth}{\hwhmlrn}} - 1 \right) 
\label{eqn:abs_mlk_lrn_trn_lmt}
\end{eqnarray}
To obtain $\eqvwthavg$ and $\absavg$ in terms of $\lnstravg$ and
$\mls$ we use (\ref{eqn:sll_fct_dfn}) in
(\ref{eqn:abs_mlk_lrn_trn_lmt}) 
\begin{eqnarray}
% GoY89 p. 141 (4.38) ThS99 p. 391 (10.21) Lio92 p. 58 (2.4.42)
\frac{\eqvwthavg}{\mls} = \absavg 
& = & 
2 \hwhmlrn \times \frac{\mpi \sllfct^{2}}{\mls}
\left( \sqrt{1 + \frac{\lnstravg}{\mpi \sllfct^{2}}
\times \frac{\abspth}{\hwhmlrn}} - 1 \right) \nonumber \\
& = & 
\frac{2 \mpi \hwhmlrn \sllfct^{2}}{\mls}
\left( \sqrt{1 + \frac{\lnstravg \abspth}{\mpi \hwhmlrn \sllfct^{2}}}
- 1 \right)
\label{eqn:abs_mlk_lrn_asm}
\end{eqnarray}
The mean absorptance of the Malkmus-Lorentz distribution $\absavg$
(\ref{eqn:abs_mlk_lrn_asm}) is thus a straightforward function of
$\lnstravg$, $\abspth$, $\hwhmlrn$, and the parameter $\sllfct$
(\ref{eqn:sll_fct_dfn}). 
Equation~(\ref{eqn:abs_mlk_lrn_asm}) is valid only in the limit
$\sqrt{\lnstrrat} \gg 1$.

It is more computationally economical to choose a single value of
$\sllfct$ than to determine a new value for each band considered.
The traditional approach to this has been to choose $\sllfct$ such
that the mean absorptance of the Malkmus distribution
(\ref{eqn:abs_mlk_lrn_asm}) equals the mean absorptance of the
exponential distribution (\ref{eqn:abs_xpn_lrn}) in the weak and
strong line limits. 
The weak and strong line limits of (\ref{eqn:abs_mlk_lrn_asm}) are
\begin{subequations}
\label{eqn:abs_mlk_lrn_lmt4}
\begin{align}
\label{eqn:abs_mlk_lrn_wll4}
\lim_{\lnstravg \abspth \ll 1} \eqvwthavg = \absavg \mls & \approx 
2 \mpi \hwhmlrn \sllfct^{2}
\left( 1 + \frac{\lnstravg \abspth}{2 \mpi \hwhmlrn \sllfct^{2}}
- 1 \right) \nonumber \\
& =
\lnstravg \abspth \\
\label{eqn:abs_mlk_lrn_sll4}
\lim_{\lnstravg \abspth \gg 1} \eqvwthavg = \absavg \mls & \approx 
2 \mpi \hwhmlrn \sllfct^{2}
\left( \sqrt{\frac{\lnstravg \abspth}{\mpi \hwhmlrn \sllfct^{2}}}
- 1 \right) \nonumber \\
& \approx
2 \mpi \hwhmlrn \sllfct^{2}
\sqrt{\frac{\lnstravg \abspth}{\mpi \hwhmlrn \sllfct^{2}}}
\nonumber \\
& =
2 \sllfct \sqrt{\mpi \lnstravg \abspth \hwhmlrn}
\end{align}
\end{subequations}
Equations (\ref{eqn:abs_mlk_lrn_wll4}) and (\ref{eqn:abs_xpn_lrn_wll})
are an identity and yield no information on the value of
$\sllfctofrat$.  
Equating (\ref{eqn:abs_mlk_lrn_sll4}) to (\ref{eqn:abs_xpn_lrn_sll})
yields
\begin{eqnarray}
2 \sllfct \sqrt{\mpi \lnstravg \abspth \hwhmlrn}
& = &
\sqrt{\mpi \lnstravg \abspth \hwhmlrn} \nonumber \\
2 \sllfct & = & 1 \nonumber \\
\sllfct & = & 1/2
\label{eqn:sll_fct_lmt}
\end{eqnarray}
Thus the value $\sllfct = \frac{1}{2}$ causes 
(\ref{eqn:abs_mlk_lrn_lmt4}) to agree with (\ref{eqn:abs_xpn_lrn_lmt}).
Substituting $\sllfct^{2} = 1/4$ into (\ref{eqn:abs_mlk_lrn_asm})
yields the traditional form of the mean absorptance of the Malkmus
line strength distribution for Lorentzian lines
\begin{eqnarray}
% GoY89 p. 141 (4.38) ThS99 p. 391 (10.21) Lio92 p. 58 (2.4.42)
\frac{\eqvwthavg}{\mls} = \absavg 
& = & 
\frac{\mpi \hwhmlrn}{2 \mls}
\left( \sqrt{1 + \frac{4 \lnstravg \abspth}{\mpi \hwhmlrn}}
- 1 \right)
\label{eqn:abs_mlk_lrn_apx}
\end{eqnarray}

As mentioned above, the choice $\sllfct^{2} = \frac{1}{4}$
(\ref{eqn:sll_fct_lmt}) implies a specific choice of $\lnstrrat$.
Starting from (\ref{eqn:sll_fct_sqr_dfn})
\begin{eqnarray}
\frac{\sqrt{\lnstrrat} - 1}{(\sqrt{\lnstrrat} + 1) \ln \lnstrrat}
& = & \frac{1}{4} \nonumber \\
\ln \lnstrrat & = & 4 \left( \frac{\sqrt{\lnstrrat} - 1}
{\sqrt{\lnstrrat} + 1} \right )
\nonumber \\
\lnstrrat & = &
\exp \left[ 4 \left( \frac{\sqrt{\lnstrrat}-1}{\sqrt{\lnstrrat}+1}
\right) \right]
\label{eqn:lnstr_rat_val}
\end{eqnarray}
which is a transcendental equation in $\lnstrrat$.
The solution of (\ref{eqn:lnstr_rat_val}) appears to be imaginary,
which is inconsistent with the assumption that 
$\sqrt{\lnstrrat} \gg 1$ (\ref{eqn:abs_mlk_lrn_trn_lmt}).

\subsubsection[Transmittance]{Transmittance}\label{sxn:lnstr_dst_trn}
% fxm: need to write trn_gnr, According to (\ref{eqn:trn_gnr}), 
The band transmittance $\trndltfrq$ of the exponential distribution
(\ref{eqn:dst_xpn}) is the negative exponential of
(\ref{eqn:abs_xpn_lrn}) 
% fxm: Develop this more
\begin{eqnarray}
% GoY89 p. , ThS99 p. 
\trndltfrq & = & \exp \left[ - 
\frac{\lnstravg \abspth}{\mls}
\left(1 + \frac{\lnstravg \abspth}{\mpi \hwhmlrn} \right)^{-1/2}
\right]
\label{eqn:trn_xpn_lrn}
\end{eqnarray}

The band transmittance $\trndltfrq$ of the Malkmus distribution
(\ref{eqn:dst_mlk}) is 
\begin{eqnarray}
% KiR83 p. 5192 (4) Kie97 p. 112 (41)
\trndltfrq & = & \exp \left\{ - \frac{ ( \lnstr / \mls )_{\dltfrq}}{2 ( \lnstr
/ \mpi \hwhmlrn )_{\dltfrq} } \left[ \left( 1 + 4 \mpl \left( \frac{\lnstr
}{\mpi \hwhmlrn} \right)_{\dltfrq} \right)^{1/2} - 1 \right] \right\} 
\label{eqn:trn_mlk_lrn}
\end{eqnarray}

% fxm: Need to write (\ref{eqn:trn_flx_dlt_dfn}) 
The band flux transmittance $\trnflxdltfrq$ 
is simplified by replacing the angular integration with a
\trmidx{diffusivity factor}, $\dff$.  
\begin{eqnarray}
% Kie97 p. 112 (41)
\trnflxdltfrq & = & \exp \left\{ - \frac{ ( \lnstr / \mls )_{\dltfrq}}{2 ( \lnstr
/ \mpi \hwhmlrn )_{\dltfrq} } \left[ \left( 1 + 4 \dff \mpl \left( \frac{\lnstr
}{\mpi \hwhmlrn} \right)_{\dltfrq} \right)^{1/2} - 1 \right] \right\} 
\label{eqn:trn_flx_mlk_LW}
\end{eqnarray}
As discussed in \S\ref{sxn:dff_prx}, $\dff = 1.66$ gives the
best agreement between (\ref{eqn:trn_flx_mlk_LW}) and models which
integrate the RT equation directly % fxm: (\ref{eqn:trn_dlt_dfn}) directly.

Applying the Malkmus model to shortwave spectral regions is also
possible if we redefine the diffusivity factor. 
For solar radiative transfer in clear skies most of the downwelling
radiation follows the path of the direct solar beam.
Thus the downwelling radiation traverses a mass path which increases
with the cosine of the solar zenith angle $\theta$. 
Defining $\mu = \cos \theta$ we have
\begin{eqnarray}
% Kie97 p. 112 (41)
\trnflxdltfrq & = & \exp \left\{ - \frac{ ( \lnstr / \mls )_{\dltfrq}}{2 ( \lnstr
/ \mpi \hwhmlrn )_{\dltfrq} } \left[ \left( 1 + \frac{4 \mpl}{\mu} \left( \frac{\lnstr
}{\mpi \hwhmlrn} \right)_{\dltfrq} \right)^{1/2} - 1 \right] \right\} 
\label{eqn:trn_flx_mlk_SW}
\end{eqnarray}
The role of $\mu$ in (\ref{eqn:trn_flx_mlk_SW}) is analogous to the
role of $\dff$ in (\ref{eqn:trn_flx_mlk_LW}).
Note that both $\mu$ and $\dff$ are approximations to the angular
integral. 
Diffuse radiation, i.e., thermal emission, upwelling (scattered) solar
radiation and downwelling solar radiation beneath optically thick
scatterers, is better characterized by $\dff = 1.66$ than by $\mu$.
Direct downwelling solar radiation, on the other hand, is better
characterized by $\mu$ than by $\dff = 1.66$.

\subsubsection[Multiplication Property]{Multiplication Property}\label{sxn:mlt}
It is observed that the spectrally resolved transmittance of a mixture
of $\NNN$ gases obeys the \trmdfn{multiplication property}.
Let $\trnbm(1, 2, \ldots \NNN)$ be the spectral transmittance of a
mixture of $\NNN$ gases, and $\trnbm(\iii)$ be the transmittance of
the $\iii$th gas alone. 
\begin{equation}
\trnbm(1, 2, \ldots \NNN) = \prod_{\iii = 1}^{\NNN} \trnbm(\iii)
\label{eqn:mlt_dfn}
\end{equation}
The usefulness of the multiplication property of transmittance cannot 
be overstated.
It is basic to the construction of efficient wide band (emissivity)
models. 

Equation~(\ref{eqn:mlt_dfn}) may be integrated over a finite frequency
interval.
\begin{eqnarray}
\int_{\dltfrq} \trnbm(1, 2, \ldots \NNN) \,\dfr\frq & = & 
\int_{\dltfrq} \prod_{\iii = 1}^{\NNN} \trnbm(\iii) \,\dfr\frq
\nonumber \\
\trndltfrq(\NNN) & = & 
\int_{\dltfrq} \prod_{\iii = 1}^{\NNN} \trnbm(\iii) \,\dfr\frq
\nonumber \\
& = & \prod_{\iii = 1}^{\NNN} \trndltfrq(\iii) 
\label{eqn:mlt_bnd_dfn}
\end{eqnarray}
The final step is strictly valid only if the spectral transmittances of
the $\NNN$ gases are uncorrelated over the range of $\dltfrq$.
This will always be the case if the lines in these gases are randomly
distributed in the interval.
Fortunately, this \trmdfn{random band} assumption is met by all
important atmospheric bands.
Henceforth we shall consider (\ref{eqn:mlt_bnd_dfn}) valid unless
otherwise specified.

\subsection[Transmission in Inhomogeneous Atmospheres]{Transmission in Inhomogeneous Atmospheres}\label{sxn:tia}
Until now we have focused on methods designed to obtain accurate 
spectral mean absorptance and transmittance of one or many lines
over homogeneous atmospheric paths, i.e., paths along which
temperature, pressure, and gas mixing ratio do not vary.
This is generally a reasonable assumpution only along horizontal
atmospheric paths, i.e., at constant altitude.
In realistic atmospheres, at least pressure and temperature
will vary vertically, and cause corresponding changes in line strength
and line width (\ref{eqn:lnstr_scl}). 

\subsubsection[Constant mixing ratio]{Constant mixing ratio}\label{sxn:cmr}
Exact, analytic solutions for mean transmission through inhomogeneous
atmospheres exist for Lorentzian lines in the idealized case where the
gas mixing ratio is uniform, the atmosphere is isothermal at
temperature $\tpt$, and where, therefore, the half-width depends only 
on pressure $\prs$.  
Let $\mmr$ be the gas mass mixing ratio, and $\dnsatm$ be the
atmospheric mass density so that the total gas mass concentration is
$\mmr\dnsatm$.
The Lorentzian line shape has a linear dependence on pressure
(\ref{eqn:hwhm_prs_dfn}).
We shall assume \trmidx{hydrostatic equilibrium} applies,
\begin{eqnarray}
% GoY89 p. 223 (6.17)
\dfr\prs & = & -\dnsatm \grv \,\dfr\hgt \nonumber \\
\dfr\hgt & = & -(\dnsatm \grv)^{-1} \,\dfr\prs
\label{eqn:hyd_eqm}
\end{eqnarray}
and use (\ref{eqn:hyd_eqm}) to change variables to pressure
coordinates. 
Referring to (\ref{eqn:tau_abs_dfn}) and (\ref{eqn:xsx_lnstr_lnshp}), 
we may write the absorption optical depth through a uniformly mixed
layer in hydrostatic equilibrium between heights $\hgt$ and $\hgtprm$
as 
\begin{eqnarray}
% GoY89 p. 223 (6.18) ThS99 p. 400 (10.49) Lio92 p. 60
\tauabs(\hgt,\hgtprm)
& = & \int_{0}^{\abspth} \lnstr \lnshplrn(\hgtprmprm) \,\dfr\abspth
\nonumber \\
& = & \int_{\hgt}^{\hgtprm} \lnstr \lnshplrn(\hgtprmprm) \mmr \dnsatm(\hgtprmprm)
\,\dfr\hgtprmprm \nonumber \\
& = & \int_{\hgt}^{\hgtprm} 
\frac{\lnstr \hwhmlrn(\hgtprmprm) \mmr \dnsatm(\hgtprmprm) }
{\mpi [ \frq^{2} + \hwhmlrn(\hgtprmprm)^{2} ] }
\,\dfr\hgtprmprm \nonumber \\
& = & \int_{\prs}^{\prsprm} 
\frac{\lnstr \hwhmlrn(\prsprmprm) \mmr \dnsatm(\prsprmprm) }
{\mpi [ \frq^{2} + \hwhmlrn(\prsprmprm)^{2} ] }
\left( -\frac{\dfr\prsprmprm}{\dnsatm \grv} \right) \nonumber \\
& = & \frac{\lnstr \mmr}{\mpi \grv} \int_{\prsprm}^{\prs} 
\frac{\hwhmlrn(\prsprmprm)}
{ \frq^{2} + \hwhmlrn(\prsprmprm)^{2} }
\,\dfr\prsprmprm \nonumber \\
& = & \frac{\lnstr \mmr}{\mpi \grv} \int_{\hwhmlrn(\prsprm)}^{\hwhmlrn(\prs)} 
\frac{\hwhmlrn}{ \frq^{2} + \hwhmlrn^{2} }
\,\dfr\hwhmlrn \nonumber \\
& = & \frac{\lnstr \mmr}{\mpi \grv} 
\left[
\frac{1}{2} \ln \left( \frq^{2} + \hwhmlrn^{2} \right) 
\right]_{\hwhmlrn(\prsprm)}^{\hwhmlrn(\prs)} \nonumber \\
& = & \frac{\lnstr \mmr}{2 \mpi \grv} 
\ln \frac{ \frq^{2} + \hwhmlrn(\zzz)^{2} }
{ \frq^{2} + \hwhmlrn(\zzzprm)^{2} } \nonumber \\
& & \ldots \mbox{fxm: find error in above!} \nonumber \\
& = & \frac{\lnstr \abspth}{2 \mpi \hwhmlrn(\zzz)} 
\ln \frac{ \frq^{2} + \hwhmlrn(\zzz)^{2} }
{ \frq^{2} + \hwhmlrn(\zzzprm)^{2} }
\label{eqn:tau_ihg_lyr}
\end{eqnarray}
% fxm: According to GoY89 and ThS99, answer has factor of pressure
where we have used the simple relationship 
$\abspth = \mmr \prs / \grv$ to determine the mass path of a gas with
constant mixing ratio $\mmr.$ 

Near the top of the atmosphere, the line shape approaches the Doppler
shape $\lnshpdpp$ (\ref{eqn:lnshp_dpp_dfn}).
In an isothermal atmosphere $\hwemdpp$ does not depend on height.
The optical depth from level $\zzz$ to space is 
\begin{eqnarray}
% GoY89 p. 223 (6.18) ThS99 p. 400 (10.50) Lio92 p. 60
\tauabs(\hgt,\hgtprm)
& = & \frac{\lnstr \mmr}{2 \mpi \grv} 
\ln \frac{ \frq^{2} + \hwhmlrn(\zzz)^{2} }
{ \frq^{2} + \hwemdpp^{2} }
\label{eqn:tau_ihg_toa}
\end{eqnarray}

\subsubsection[H-C-G Approximation]{van~de~Hulst-Curtis-Godson Approximation}\label{sxn:HCG}
The general problem is to account for the effects of inhomogeneous
mass, temperature, and pressure paths on transmission.
We shall attempt to determine a \trmdfn{scaled absorber path}
$\abspthscl$, \trmdfn{scaled temperature} $\tptpthscl$, and
\trmdfn{scaled pressure} $\prspthscl$ such that $\abspthscl$, 
$\tptpthscl$, and $\prspthscl$ yield optimal results when employed
in our machinery for homogeneous paths, e.g.,
(\ref{eqn:trn_xpn_lrn}), (\ref{eqn:trn_mlk_lrn}). 
We shall show that the scaled parameters can be chosen 
to yield correct band transmittances in certain limits.

The extra degree of freedom permitted by each parameter means that 
$\abspthscl$, $\tptpthscl$, and $\prspthscl$ form a three-parameter 
\trmdfn{scaling approximation}.
The researchers van~de~Hulst, Curtis, and Godson independently
discovered the following scaling approximation, now known as the 
\trmdfn{H-C-G approximation}. 
First we note that absorption along a path in Earth's atmosphere
is generally more affected by changes in $\prs$ than by changes in
$\tpt$. 
This stands to reason since $\prs$ changes by many orders of magnitude
from the surface to space, while $\tpt$ generally changes by less than
30\%.
Thus the usual practice, often adeequate, is to set
$\tptpthscl$ equal to the mean temperature along a path, 
\begin{eqnarray}
% GoY89 p. 224 (6.22) Lio92 p. 60
\tptpthscl & = & \tptavg
\label{eqn:tpt_pth_scl_dfn}
\end{eqnarray}
This leaves two parameters, $\abspthscl$ and $\prspthscl$, to account for
the effects of inhomogeneity on path absorption.
We first derive a single parameter scaling approximation based solely
on $\abspthscl$. 
This will prove useful afterwards in developing the two parameter
H-C-G approximation.

Consider the absorption optical depth through an arbitrary
inhomogeneous atmosphere whose mass absorption coefficient
$\abscffmss(\prs,\tpt)$ (\ref{eqn:abs_mss_dfn}) depends only on $\tpt$ 
and $\prs$. 
The optical path through this medium is (\ref{eqn:tau_abs_dfn})
\begin{eqnarray}
% GoY89 p. 224 (6.22) Lio92 p. 60 (2.5.1)
\tauabs & = & \int \abscffmss(\prs,\tpt) \,\dfr\abspth
\label{eqn:tau_abs_dfn2}
\end{eqnarray}
A one-parameter scaling approximation is exact if
$\abscffmss(\prs,\tpt)$ can be separated into a frequency-dependent
function $\etascloffrq$ times a $\prs$-$\tpt$ dependent function
$\phiscl(\prs,\tpt)$ \cite[][p.~224]{GoY89}
\begin{eqnarray}
% GoY89 p. 224 (6.22)
\abscffmss(\prs,\tpt) & = & \phiscl(\prs,\tpt) \, \etascloffrq \nonumber \\
\phiscl(\prs,\tpt) & = & \abscffmss(\prs,\tpt) / \etascloffrq
\label{eqn:phi_scl_dfn}
\end{eqnarray}
Applying this separation of variables (\ref{eqn:phi_scl_dfn}) to
(\ref{eqn:tau_abs_dfn2}) yields
\begin{eqnarray}
% GoY89 p. 224 (6.23)
\tauabs & = & \int \phiscl(\prs,\tpt) \, \etascloffrq \,\dfr\abspth \nonumber \\
& = & \int \phiscl(\prs,\tpt) 
\frac{\phiscl(\prspthscl,\tptpthscl)}{\phiscl(\prspthscl,\tptpthscl)}
\, \etascloffrq \,\dfr\abspth
\nonumber \\
& = & \int 
\frac{\phiscl(\prs,\tpt)}{\phiscl(\prspthscl,\tptpthscl)}
\frac{\abscffmss(\prspthscl,\tptpthscl)}{\etascloffrq}
\, \etascloffrq \,\dfr\abspth
\nonumber \\
& = & \int 
\abscffmss(\prspthscl,\tptpthscl)
\frac{\phiscl(\prs,\tpt)}{\phiscl(\prspthscl,\tptpthscl)}
\,\dfr\abspth \nonumber
\end{eqnarray}
We see that \abscffmss(\prspthscl,\tptpthscl) is independent of the 
variable of integration, $\abspth$, since both $\prspthscl$ and
$\tptpthscl$ are path-mean quantities.
Therefore $\abscffmss(\prspthscl,\tptpthscl)$ may be placed outside
the integrand 
\begin{eqnarray}
% GoY89 p. 224 (6.23) Lio92 p. 60 (2.5.8)
\tauabs
& = & 
\abscffmss(\prspthscl,\tptpthscl) 
\int \frac{\phiscl(\prs,\tpt)}{\phiscl(\prspthscl,\tptpthscl)} \,\dfr\abspth
\nonumber \\
& = & 
\abscffmss(\prspthscl,\tptpthscl) \abspthscl \qquad \mbox{where} \nonumber \\
\abspthscl & \equiv & 
\int \frac{\phiscl(\prs,\tpt)}{\phiscl(\prspthscl,\tptpthscl)} \,\dfr\abspth
\label{eqn:abs_pth_scl_dfn}
\end{eqnarray}
The scaled absorber amount, $\abspthscl$, together with the absorption
coefficient evaluated for the scaled temperature and pressure, exactly
predicts the absorption optical path (\ref{eqn:tau_abs_dfn2}).

The exact form of the scaling function $\phiscl(\prs,\tpt)$ is
chosen to optimize the accuracy of (\ref{eqn:abs_pth_scl_dfn}).
The definition (\ref{eqn:phi_scl_dfn}) makes
clear that $\phiscl(\prs,\tpt)$ is a weighting function that does
not explicitly depend on frequency, and yet must somehow account for
the temperature dependence of all the lines with the band.
Line strengths are highly temperature dependent (\ref{eqn:lnstr_scl}).
Considering first Doppler lines, we note that the Doppler line width
$\hwemdpp$ (\ref{eqn:hwem_dpp_dfn}) has no pressure 
dependence, but does depend on $\sqrt{\tpt}$.
Both strong and weak Doppler lines within a given band will therefore 
be well-represented by identifying the scaling function
$\phiscl(\prs,\tpt)$ with the line strength $\lnstr$ 
\begin{eqnarray}
% GoY89 p. 225 (6.25) Lio92 p. 60 (2.5.8)
\phiscl(\prs,\tpt) & = & \sum_{\lnsbs}^{\lnnbr} \lnstrlnoftpt
\label{eqn:phi_scl_dpp}
\end{eqnarray}
Thus $\phiscl(\prs,\tpt)$ has no explicit pressure dependence in this
case. 

\cite{ChA80} present a more formalized one-parameter scaling based on
the presumption (appropriate for water vapor) that errors due to
inhomogeneity are likely to bias line wing absorption more than line
center absorption.
\cite{Lio92}, p.  presents a detailed account of this scaling.

The \trmidx{H-C-G approximation}, sometimes called the \trmdfn{C-G
approximation}, \ldots

\subsection[Temperature Dependence]{Temperature Dependence}\label{sxn:tpt}
SWNB evaluates the temperature dependence of the random band
parameters. 
Let $\tptnot$ be the temperature at which the line parameters
(strengths, half-widths) are known.
Typically $\tptnot = 296\,$K, the reference temperature of the
\acr{HITRAN} database discussed in \S\ref{sxn:htrn}.
The H-C-G approximation accounts for the influence of temperature and
pressure inhomogeneity along the path on the absorption
(\ref{eqn:tau_abs_dfn2}). 
The scaling functions $\phiscloftpt$ and $\psiscloftpt$ at an
arbitrary temperature $\tpt$ are
\begin{eqnarray}
% BPB94 p. 1, GoY89 p. 225 (6.25), Lio92 p. 60 (2.5.8)
\phiscloftpt & \equiv & 
\sum_{\lnsbs}^{\lnnbr} \lnstrlnoftpt \bigg/ 
\sum_{\lnsbs}^{\lnnbr} \lnstrln(\tptnot) \\
\psiscloftpt & \equiv & 
\left( \sum_{\lnsbs}^{\lnnbr} \sqrt{\lnstrlnoftpt
\hwhmlrn(\prsnot,\tpt) } \right)^{2} \bigg/ 
\left( \sum_{\lnsbs}^{\lnnbr} \sqrt{\lnstrln(\tptnot)
\hwhmlrn(\prsnot,\tptnot) } \right)^{2}
\label{eqn:u_bar_dfn}
\end{eqnarray}
where $\lnstrln(\tpt)$ is scaled from $\lnstrln(\tptnot)$ using
(\ref{eqn:lnstr_scl}) 
and $\hwhmlrn(\prsnot,\tpt)$ is scaled from
$\hwhmlrn(\prsnot,\tptnot)$ using (\ref{eqn:hwhm_scl}). 

The scaled absorber path $\abspthscl$ is
\begin{eqnarray}
\abspthscl & = & \int q \phiscloftpt \, \frac{\dfr\prs}{\grv}
\label{eqn:abs_pth_scl_dfn2}
\end{eqnarray}
The scaled pressure $\prspthscl$ is
\begin{eqnarray}
\prspthscl & = & \frac{1}{\abspthscl}
\int q \psiscloftpt \frac{\prs}{\prsnot} \, \frac{\dfr\prs}{\grv}
\label{eqn:prs_pth_scl_dfn}
\end{eqnarray}

Following \cite{RoW66}, we fit the logarithms of $\phiscloftpt$
and $\psiscloftpt$ to quadratic polynomials in temperature
\begin{subequations}
\label{eqn:phi_psi_scl_pzn}
\begin{align}
\label{eqn:phi_scl_pzn}
\ln \phiscloftpt & =  
\phiscla (\tpt - \tptnot) +  \phisclb (\tpt - \tptnot)^{2} \\
\label{eqn:psi_scl_pzn}
\ln \psiscloftpt & =  
\psiscla (\tpt - \tptnot) +  \psisclb (\tpt - \tptnot)^{2}
\end{align}
\end{subequations}
The linear term dominates (\ref{eqn:phi_scl_pzn}) since $\phiscloftpt$ 
(\ref{eqn:phi_scl_dpp}) depends mostly exponentially on 
$\tpt$ (\ref{eqn:lnstr_scl}), i.e., $\phiscla \gg \phisclb$.
On the other hand the quadratic term is important in
(\ref{eqn:psi_scl_pzn}) since $\psiscloftpt$ (\ref{eqn:fxm}) 
contains significant nonlinear behavior.
The coefficients $\phiscla$, $\phisclb$, $\psiscla$, and $\psisclb$ 
are determined by least-squares fitting $\phiscl$ and $\psiscl$ at
discrete temperatures $\tpttpt$ over the temperature range of
interest: 
\begin{subequations}
\label{eqn:phi_psi_scl_cff}
\begin{align}
\label{eqn:phi_psi_scl_cff_a}
\phiscla & = 
\frac{
\sum_{\tptidx} (\tpttpt-\tptnot) \ln \phiscl(\tpttpt)
\sum_{\tptidx} (\tpttpt-\tptnot)^{4} - 
\sum_{\tptidx} (\tpttpt-\tptnot)^{2} \ln \phiscl(\tpttpt)
\sum_{\tptidx} (\tpttpt-\tptnot)^{3}
}{
\sum_{\tptidx} (\tpttpt-\tptnot)^{2}
\sum_{\tptidx} (\tpttpt-\tptnot)^{4} - 
\sum_{\tptidx} (\tpttpt-\tptnot)^{3}
\sum_{\tptidx} (\tpttpt-\tptnot)^{3}
} \\
\label{eqn:phi_psi_scl_cff_b}
\phisclb & = 
\frac{
\sum_{\tptidx} (\tpttpt-\tptnot)^{2} \ln \phiscl(\tpttpt)
\sum_{\tptidx} (\tpttpt-\tptnot)^{2} - 
\sum_{\tptidx} (\tpttpt-\tptnot) \ln \phiscl(\tpttpt)
\sum_{\tptidx} (\tpttpt-\tptnot)^{3}
}{
\sum_{\tptidx} (\tpttpt-\tptnot)^{2}
\sum_{\tptidx} (\tpttpt-\tptnot)^{4} - 
\sum_{\tptidx} (\tpttpt-\tptnot)^{3}
\sum_{\tptidx} (\tpttpt-\tptnot)^{3}
}
\end{align}
\end{subequations}
and $\psiscla$ and $\psisclb$ are defined analogously.

In practice SWNB adjusts all band parameters from the
\acr{HITRAN}-standard temperature (296\,K) to 250\,K, roughly
the mass-weighted mean temperature of Earth's atmosphere.
This reduces any biases caused by non-quadratic temperature-dependent
behavior of the logarithms of $\phiscloftpt$ and $\psiscloftpt$
(\ref{eqn:phi_psi_scl_cff}).
$\phiscloftpt$ and $\psiscloftpt$ are evaluated exactly in the 
temperature range $180 < \tpttpt < 320$\,K with a 1\,K resolution
and the results are fit using (\ref{eqn:phi_psi_scl_cff}).
The expressions (\ref{eqn:phi_psi_scl_cff}) are generally accurate to 
within 1\% over this 140\,K range.

\subsection[Transmission in Spherical Atmospheres]{Transmission in Spherical Atmospheres}\label{sxn:tsa}
Until this point we have assumed plane-parellel boundary conditions.
That is we have assumed the increase in path length due to
off-vertical transmission is given by the secant of the solar zenith
angle times the vertical path. 
Planetary atmospheres are, of course, spherical and the plane parallel
approximation must be corrected for spherical effects when the zenith
angle is large.
Horizontal irradiances must be corrected if the zenith angle is
greater than about $80\dgr$.

\subsubsection[Chapman Function]{Chapman Function}\label{sxn:chp_fnc}
The earliest quantitative computations of the absorber path to space
at arbitrary angles through a spherical atmosphere were made by Sidney
Chapman. 
Let $\hgtscl$ be the scale height of the species, so that
\begin{eqnarray}
\nbrcnc(\hgt) & = & \nbrcnc(\hgtnot) \me^{-(\hgt-\hgtnot)/\hgtscl}
\label{eqn:hgt_scl_dfn}
\end{eqnarray}
The scale height well-mixed gases in Earth's atmosphere is about
7~km. 
Let $\rdsrth$ be the Earth's radius, and $\hgtnot$ be the altitude at
which the absorber path is to begin.
With the assumption that the species is distributed exponentially
with altitude (\ref{eqn:hgt_scl_dfn}), the number path $\nbrpth$ from
height $\hgtnot$ to space at a polar angle $\plrnot$ reduces to 
\begin{eqnarray}
\nbrpth(\hgtnot,\plrnot) & = & \int_{\hgt=\hgtnot}^{\hgt=\infty} 
\nbrcnc(\hgt) \left[ 
1 - \frac{(\rdsrth+\hgtnot) \sin(\plrnot)}{(\rdsrth+\hgt)^{2}}
\right]^{-1/2} \,\dfr\hgt \nonumber
\label{eqn:nbr_pth_chp_dfn}
\end{eqnarray}
It can be shown that the number path $\nbrpth$
(\ref{eqn:nbr_pth_chp_dfn}) is a function of a dimensionless 
measure called the \trmdfn{radius of curvature} of the absorber  
distribution $\rdscrv$ 
\begin{eqnarray}
\rdscrv & = & (\rdsrth+\hgtnot)/\hgtscl 
\label{eqn:rds_crv_dfn}
\end{eqnarray}
According to Huestis, $\rdscrv$ varies from about 300 to 1300 on
Earth. 
The \trmdfn{Chapman function} $\chpfnc$ \cite[e.g.,][]{Ree89} 
is defined implicitly by the RHS of (\ref{eqn:nbr_pth_chp_dfn}).
It directly relates $\nbrpth(\hgtnot,\plrnot)$ to the scale height
$\hgtscl$ and local number concentration $\nbrcnc(\hgt)$ of a
species via
\begin{eqnarray}
\nbrpth(\hgtnot,\plrnot) & = & \hgtscl \nbrcnc(\hgtnot)
\chpfnc(\rdscrv,\plrnot)
\label{eqn:chp_fnc_dfn}
\end{eqnarray}
Note that $\chpfnc$ (\ref{eqn:chp_fnc_dfn}) depends only on $\rdscrv$,  
not on the individual values of $\rdsrth$, $\hgtnot$, and $\hgtscl$.

\section[Radiative Effects of Aerosols and Clouds]{Radiative Effects of Aerosols and Clouds}\label{sxn:aer}

Interaction of radiation and matter occurs on three scales.
On the molecular scale, the scattering and absorption of radiation is
described by the ambient concentration, scattering cross sections,
reasonant frequencies, and quantum states of the molecule.
Interaction of radiation with bulk phase matter (e.g., the sea
surface) is described by the change in the index of refraction of the
media (e.g., Snell's law) and macroscopic thermodynamic properties
such as $\flxupw = \cststfblt \tpt^{4}$ (\ref{eqn:flx_sfc_upw}).
In between these two scales is the interaction of radiation with
atmospheric particulates, i.e., aerosols and hydrometeors.
Particulates are, by definition, isolated bodies of condensed matter
surrounded by a continuous media.
We now turn our attention to describing the effects of particulates on
the radiation field.

\subsection{Single Scattering Properties}\label{sxn:ssp}
Three properties are required to exactly specify the radiative effects 
of particles.
The properties define the total extinction (scattering plus
absorption) due to the particle, the probability an interaction
results in absorption rather than scattering, and, finally, the
angular distribution of scattered photons as a function of the
incident angle. 
By convention, the properties defining the above attributes are
usually specified as the \trmdfn{extinction optical depth} $\tauext$,
the \trmdfn{single scattering albedo} $\ssa$, and the
\trmdfn{asymmetry parameter} $\asmprm$. 
These three parameters, $\tauext$, $\ssa$, and $\asmprm$, are known
collectively as the \trmdfn{single scattering properties} of the
particles. 
These properties depend in turn on the mass, size, and composition of
the particle species.

A particle's chemical composition determines its \trmdfn{index of
refraction}\footnote{Care must be taken not to confuse the complex
index of refraction $\idxrfr$\,(dimensionless) with the size
distribution $\dstnbr$\,[\nbrxmCm]. 
The former is a function of wavelength for a given chemical
composition, while the latter is a function of size for a given
particle population. 
The meaning should be clear from the context.}
$\idxrfr$.
The index of refraction is a dimensionless, complex number 
\begin{eqnarray}
\idxrfr & = & \idxrfrrl + \mi \idxrfrimg
\label{eqn:idx_rfr_dfn}
\end{eqnarray}
All the physics describing the fundamental electromagnetic properties
of the material are consolidated into~$\idxrfr$.
$\idxrfrrl$ describes the scattering properties of the medium while 
$\idxrfrimg$ describes the absorption properties of the medium.
The literature disagrees about the sign convention for~$\idxrfrimg$.
We choose to represent $\idxrfrimg$ as positive-definite, i.e.,
$\idxrfrimg > 0$. 
This convention is employed by \cite{BoH83}, among others.

$\idxrfrrl$ and $\idxrfrimg$ are fundamental properties of matter and
must be determined from laboratory studies
\cite[e.g.,][]{HKS98,PGS77,Pat81,PeG91,Tan97,Vol73,BoH83},
or from models constrained by field measurements 
\cite[e.g.,][]{CTT02,STD03}.
The \acr{HITRAN} database \cite[]{RRG98} contains a compilation of
$\idxrfr$ for many aerosols of atmospheric interest.

\subsubsection[Maxwell Equations]{Maxwell Equations}\label{sxn:mxw_eqn}
The macroscopic form of \trmidx{Maxwell's equations} in Rationalized
SI (MKSA) units is \cite[][p.~12]{BoH83}
\begin{subequations}
% BoH83 p. 58 (3.1), Jac75 p. 818
\label{eqn:mxw_eqn_SI}
\begin{align}
\label{eqn:mxw_eqn_SI_clb}
\nabla \cdot \lctdspvct & = \chgdnsfre \qquad\mbox{Coulomb's law} \\
\label{eqn:mxw_eqn_SI_mgn_ndc}
\nabla \cdot \mgnndcvct & = 0 \\
\label{eqn:mxw_eqn_SI_frd}
\nabla \times \lctvct + \frac{\partial \mgnndcvct}{\partial \tm} & = 0 \qquad\mbox{Faradays's law} \\
\label{eqn:mxw_eqn_SI_amp}
\nabla \times \mgnvct & = \crrdnsfrevct + \frac{\partial \lctdspvct}{\partial \tm} \qquad\mbox{Maxwell's generalization of Ampere's law}
\end{align}
\end{subequations} 
% fxm this is verbatim from BoH83 p. 13
where $\lctvct$ is the \trmdfn{electric field} measured in Volts per
meter [\vxm], $\mgnndcvct$ is the \trmdfn{magnetic induction}
measured in \trmdfn{Tesla}~[T], and $\crrdnsfrevct$ is the
\trmdfn{free current density} measured in Amperes per square meter
[\axmS].
In this context, \trmidx{macroscopic} refers to spatio-temporal
averaging of the Maxwell equations in a vacuum. 
Clearly (\ref{eqn:mxw_eqn_SI}) applies to microscopic spaces, since 
it is expressed in terms of differentials, though it does not apply
at the smallest molecular scales.

The \trmdfn{electric displacement} $\lctdspvct$ and \trmdfn{magnetic
field} $\mgnvct$ are defined by
\begin{subequations}
% BoH83 p. 12 (2.5)
\label{eqn:dsp_mgn_eqn}
\begin{align}
\label{eqn:lct_dsp_dfn}
\lctdspvct & = \lctprmtvcm \lctvct + \lctplrvct \\
\label{eqn:mgn_dfn}
\mgnvct & = \frac{\mgnndcvct}{\lctmuvcm} - \mgntznvct
\end{align}
\end{subequations} 
where $\lctplrvct$ is the \trmdfn{electric polarization}
(mean electric dipole moment per unit volume), $\mgntznvct$ is the
\trmdfn{magnetization} (mean magnetic dipole moment per unit volume).
$\lctplrvct$ is measured in [\cxmS], and $\mgntznvct$ is measured in
[\axm]. 
The scalar symbols are 
the \trmdfn{free charge density} $\chgdnsfre$, measured in [\cxmC], 
the \trmdfn{permittivity of free space} $\lctprmtvcm$, measured in [\fxm], 
the \trmdfn{permeability of free space} $\lctmuvcm$, measured in [\hxm]. 

The \trmidx{constitutive equations} in SI units are
\cite[][p.~13]{BoH83} 
\begin{subequations}
% BoH83 p. 13 (2.7)
\label{eqn:cns_eqn_SI}
\begin{align}
\label{eqn:cns_eqn_SI_crr_dns}
\crrdnsfrevct & = \lctcnd \lctvct \\
\label{eqn:cns_eqn_SI_mgn_ndc}
\mgnndcvct & = \lctmu \mgnvct \\
\label{eqn:cns_eqn_SI_plr}
\lctplrvct & = \lctprmtvcm \lctscp \lctvct
\end{align}
\end{subequations} 
Here $\lctcnd$ is the \trmdfn{conductivity} which is 
measured in Siemens~[S], Amperes per volt [\axv], or \trmidx{mhos} per
meter [\mhoxm].  
$\lctmu$ is the \trmdfn{permeability} measured in [\hxm]. 
$\lctscp$ is the \trmdfn{electric susceptibility} measured in Farads
per meter [\fxm] or [\cxvm].
Together, $\lctcnd$, $\lctmu$, and $\lctscp$ are known as the 
\trmdfn{phenomenological coefficients}---they define the medium under 
consideration. 
In a \trmdfn{linear} medium, these coefficients do not depend upon
the electromagnetic field.
In a \trmdfn{homogeneous} medium, these coefficients are independent
of position.
In an \trmdfn{isotropic} medium, these coefficients are independent
of direction.
The media we consider is assumed to be linear, homogeneous, and
isotropic. 

We shall now simplify the Maxwell equations (\ref{eqn:mxw_eqn_SI}) to
the problem of scattering of radiation by matter described by the 
the \trmidx{constitutive equations} (\ref{eqn:cns_eqn_SI}).
Subsituting (\ref{eqn:cns_eqn_SI_plr}) into (\ref{eqn:lct_dsp_dfn})
and (\ref{eqn:lct_dsp_dfn}) into (\ref{eqn:mxw_eqn_SI_clb}), we
obtain  
\begin{eqnarray}
\nabla \cdot \lctdspvct & = & 
\nabla \cdot ( \lctprmtvcm \lctvct + \lctprmtvcm \lctscp \lctvct)
= \lctprmtvcm ( 1 + \lctscp ) \nabla \cdot \lctvct = \chgdnsfre
\label{eqn:mxw_eqn_gss_cns}
\end{eqnarray} 
Subsituting (\ref{eqn:cns_eqn_SI_mgn_ndc}) into
(\ref{eqn:mxw_eqn_SI_mgn_ndc}) leads to 
\begin{eqnarray}
\nabla \cdot \mgnndcvct & = & \nabla \cdot ( \lctmu \mgnvct ) = 
\lctmu \nabla \cdot \mgnvct = \nabla \cdot \mgnvct = 0
\label{eqn:mxw_eqn_mgn_ndc_cns}
\end{eqnarray} 
Subsituting (\ref{eqn:cns_eqn_SI_mgn_ndc}) into
(\ref{eqn:mxw_eqn_SI_frd}) leads to 
\begin{eqnarray}
\nabla \times \lctvct + \frac{\partial \lctmu \mgnvct}{\partial \tm} 
& = & 0
\label{eqn:mxw_eqn_frd_cns}
\end{eqnarray} 
Subsituting (\ref{eqn:cns_eqn_SI_plr}) into
(\ref{eqn:lct_dsp_dfn}), and then sustituting 
(\ref{eqn:cns_eqn_SI_crr_dns}) and (\ref{eqn:lct_dsp_dfn}) into
(\ref{eqn:mxw_eqn_SI_amp}), we obtain  
\begin{eqnarray}
\nabla \times \mgnvct & = &
\lctcnd \lctvct + 
\frac{\partial(\lctprmtvcm \lctvct + \lctprmtvcm \lctscp
\lctvct)}{\partial \tm} =
\lctcnd \lctvct + \lctprmtvcm ( 1 + \lctscp) 
\frac{\partial \lctvct}{\partial \tm}
\label{eqn:mxw_eqn_amp_cns}
\end{eqnarray} 
where $\lctprmt$, the \trmdfn{complex permittivity} is defined by
\begin{eqnarray}
\lctprmt & = & \lctprmtvcm ( 1 + \lctscp ) + \mi \frac{\lctcnd}{\frqngl}
\label{eqn:lct_prmt_dfn}
\end{eqnarray}

We assume the time-variation of the electro-magnetic field is
harmonic, i.e., of the form 
$\lctvct \propto \exp(-\mi \frqngl \tm)$ 
where $\frqngl$ is the \trmidx{angular frequency}.
The particular problem we wish to solve is the scattering of incident
plane waves by a sphere.
The harmonic plane wave definition (\ref{eqn:pln_wv_dfn}) is
\begin{eqnarray}
\lctvctcpx(\drcvct,\tm) 
& = & 
\lctvctnot \exp[\mi (\wvnbrvct \cdot \drcvct - \frqngl \tm ) ] \nonumber \\
\label{eqn:pln_wv_gnr_dfn}
\end{eqnarray}
where the subscript $\cpxsbs$ denotes \trmdfn{complex plane wave}.
With this assumption, the Maxwell equations for the medium 
(\ref{eqn:mxw_eqn_gss_cns})--(\ref{eqn:mxw_eqn_amp_cns}) impose
a dispersive relationship upon wave characteristics of
(\ref{eqn:pln_wv_dfn}). 
\begin{subequations}
% BoH83 p. 15 (2.12)
% Maxwell equations dispersive relationships
% Note that these relationships assume only a time-harmonic field
% No assumption is made about plane waves
\label{eqn:mxw_dsp}
\begin{align}
\label{eqn:mxw_dsp_gss}
\nabla \cdot \lctvctcpx & = 0 \\
\label{eqn:mxw_dsp_mgn}
\nabla \cdot \mgnvctcpx & = 0 \\
\label{eqn:mxw_dsp_frd}
\nabla \times \lctvctcpx & = \mi \frqngl \lctmu \mgnvctcpx \\
\label{eqn:mxw_dsp_amp}
\nabla \times \mgnvctcpx & = - \mi \frqngl \lctprmt \lctvctcpx
\end{align}
\end{subequations} 
Electrodynamics texts are often written in CGS (Gaussian) units. 
The \trmidx{Maxwell equations} in all common units appear in
\cite{Jac75}~p.~818.

The coupled vector equations (\ref{eqn:mxw_dsp}) will prove more
tractable once they are reduced to coupled scalar equations.
This will be accomplished by exercising (\ref{eqn:mxw_dsp}) with 
some vector mathematics (\ref{eqn:vct_idn}). 
First we take the curl of equations (\ref{eqn:mxw_dsp_frd}) and
(\ref{eqn:mxw_dsp_amp}):
\begin{subequations}
% BoH83 p. 58 (3.4)
\label{eqn:mxw_dsp_crl}
\begin{align}
\label{eqn:mxw_dsp_frd_crl}
\nabla \cross \nabla \cross \lctvctcpx 
& = \nabla \cross (\mi \frqngl \lctmu \mgnvctcpx) \nonumber \\
& = \mi \frqngl \lctmu \nabla \cross \mgnvctcpx \nonumber \\
& = \mi \frqngl \lctmu (- \mi \frqngl \lctprmt \lctvctcpx) \nonumber \\
& = \frqngl^{2} \lctmu \lctprmt \lctvctcpx \\
\label{eqn:mxw_dsp_amp_crl}
\nabla \cross \nabla \cross \mgnvctcpx 
& = \nabla \cross (- \mi \frqngl \lctprmt \lctvctcpx) \nonumber \\
& = - \mi \frqngl \lctprmt \nabla \cross \lctvctcpx \nonumber \\
& = - \mi \frqngl \lctprmt (\mi \frqngl \lctmu \mgnvctcpx) \nonumber \\
& = \frqngl^{2} \lctmu \lctprmt \mgnvctcpx
\end{align}
\end{subequations} 
We introduce the \trmidx{wavenumber} $\wvnbr$ as
\begin{eqnarray}
% fxm: Why is this k, the magnitude of the plane-wave vector?
% I think this is answered by BoH83 p. 26 (2.45)
\wvnbr^{2} & = & \frqngl^{2} \lctmu \lctprmt
\label{eqn:wvnbr_dsp_dfn}
\end{eqnarray}

Applying (\ref{eqn:vct_idn_crl_crl}) to (\ref{eqn:mxw_dsp_crl})
results in 
\begin{subequations}
% BoH83 p. 58 (3.5)
\label{eqn:lct_mgn_wv_eqn}
\begin{align}
\label{eqn:lct_wv_eqn}
\nabla \cross \nabla \cross \lctvctcpx 
= \nabla (\nabla \cdot \lctvctcpx) - \nabla \cdot (\nabla
\lctvctcpx) & = \frqngl^{2} \lctmu \lctprmt \lctvctcpx \nonumber \\
\nabla (0) - \nabla \cdot (\nabla \lctvctcpx) +  
\frqngl^{2} \lctmu \lctprmt \lctvctcpx & = 0 \nonumber \\
\nabla^{2} \lctvctcpx + \wvnbr^{2} \lctvctcpx & = 0 \\
\label{eqn:mgn_wv_eqn}
\nabla^{2} \mgnvctcpx + \wvnbr^{2} \mgnvctcpx & = 0
\end{align}
\end{subequations} 
where the derivation of (\ref{eqn:mgn_wv_eqn}) follows
(\ref{eqn:lct_wv_eqn}). 
Thus both the electric and magnetic fields satisfy the \trmdfn{vector 
wave equation} (\ref{eqn:lct_wv_eqn})--(\ref{eqn:mgn_wv_eqn}), as well
as the Maxwell equations for time-harmonic fields (\ref{eqn:mxw_dsp}).
In fact, any two vector fields $\vshm$ and $\vshn$ which satisfy 
(\ref{eqn:lct_mgn_wv_eqn}) and (\ref{eqn:mxw_dsp}) are valid electric
and magnetic fields.
In keeping with convention \cite[][]{BoH83}, we let $\vshm$ and
$\vshn$ denote a candidate pair of solutions to the electro-magnetic
field equations.

We shall apply a \trmdfn{gauge transformation} to simplify
the Maxwell equations for a plane wave scattering from a sphere.
This transformation identifies a scalar \trmdfn{generating function}
which satisfies the scalar wave equation.
This generating function then determines $\vshm$ and $\vshn$.
The problem thus reduces to identifying the correct scalar field. 
Denote the scalar field by $\sclfld$ and introduce a constant vector
function $\cccbld$ such that
\begin{eqnarray}
\label{eqn:vshm_dfn}
\vshm & = & \nabla \cross (\sclfld \cccbld) \\
\label{eqn:vshm_dfn_2}
& = & \nabla \sclfld \cross \cccbld + \sclfld \nabla \cross \cccbld
\end{eqnarray}
where (\ref{eqn:vshm_dfn_2}) follows from the vector identity for the   
curl of a product (\ref{eqn:vct_idn_crl_prd}).
The divergence of $\vshm$ (\ref{eqn:vshm_dfn}) is zero since $\vshm$ 
is the curl of a vector field (\ref{eqn:vct_idn_dvr_crl})
\begin{eqnarray}
\nabla \cdot \vshm & = & 0
\label{eqn:vshm_dvr_dfn}
\end{eqnarray}
The curl of $\vshm$ (\ref{eqn:vshm_dfn}) is
\begin{eqnarray}
\nabla \cross \vshm 
& = & \nabla \cross [ \nabla \cross (\sclfld \cccbld) ] \nonumber \\
\label{eqn:vshm_wv_eqn_drv_1}
& = & \nabla [ \nabla \cdot (\sclfld \cccbld) ]
- \nabla^{2} ( \sclfld \cccbld ) \\
\label{eqn:vshm_wv_eqn_drv_2}
& = & \nabla \cross ( 
\nabla \sclfld \cross \cccbld + \sclfld \nabla \cross \cccbld 
)
\end{eqnarray}
where (\ref{eqn:vshm_wv_eqn_drv_1}) comes from the vector identity for
the curl of a curl (\ref{eqn:vct_idn_crl_crl}) and
(\ref{eqn:vshm_wv_eqn_drv_2}) arises from direct subsitution of
(\ref{eqn:vshm_dfn_2}). 
Neither of these expressions is particularly appealing for direct
evaluation. 
We proceed from (\ref{eqn:vshm_wv_eqn_drv_1}) armed with the vector 
identities for the gradient of a divergence
(\ref{eqn:vct_idn_grd_dot}) and the \trmidx{Laplacian} of a vector  
(\ref{eqn:vct_idn_lpl_vct})
\begin{eqnarray}
& = & \nabla \cross ( 
\nabla \sclfld \cross \cccbld + \sclfld \nabla \cross \cccbld 
) \nonumber \\
& = & \nabla [ \cccbld \cdot \nabla \sclfld + \sclfld \nabla \cdot \cccbld ] 
- \nabla [\nabla \cdot (\sclfld\cccbld)] - 
\nabla \cross [\nabla \cross (\sclfld\cccbld)] \nonumber
\label{eqn:vshm_wv_eqn_drv_1}
\end{eqnarray}
Applying (\ref{eqn:vct_idn_grd_dvr}) to (\ref{eqn:vshm_dvr_dfn})
\begin{eqnarray}
fxm
\label{eqn:vshm_wv_eqn_drv_3}
\end{eqnarray}
\ldots leads to the desired result
\begin{eqnarray}
\nabla^{2} \vshm + \wvnbr^{2} \vshm & = & 
\nabla \cross [ \cccbld ( \nabla^{2} \sclfld + \wvnbr^{2} \sclfld ) ]
\label{eqn:vshm_wv_eqn_inh}
\end{eqnarray}
If $\sclfld$ satisfies the scalar wave equation then the RHS of
(\ref{eqn:vshm_wv_eqn}) vanishes and $\vshm$ must satisfy the
\trmdfn{vector wave equation}.
\begin{eqnarray}
\nabla^{2} \vshm + \wvnbr^{2} \vshm & = & 0
\label{eqn:vshm_wv_eqn}
\end{eqnarray}
As $\vshm$ was constructed from the curl of $\cccbld$
(\ref{eqn:vshm_dfn}) , so may we construct a new vector function
$\vshn$ from the curl of $\vshm$ 
\begin{eqnarray}
% BoH83 p. 84
\vshn & = & \frac{\nabla \cross \vshm}{\wvnbr}
\label{eqn:vshn_dfn}
\label{eqn:vshm_crl}
\end{eqnarray}
The divergence of $\vshn$ vanishes since $\vshn$ is the curl of a
vector field (\ref{eqn:vct_idn_dvr_crl}).
Hence $\vshn$ also satisfies the vector wave equation
\begin{eqnarray}
\nabla^{2} \vshn + \wvnbr^{2} \vshn & = & 0
\label{eqn:vshn_wv_eqn}
\end{eqnarray}
This implies that
\begin{eqnarray}
\nabla \cross \vshn & = & \wvnbr \vshm
\label{eqn:vshn_crl}
\end{eqnarray}

We constructed $\vshm$ and $\vshn$ to behave like electric and
magnetic fields, respectively.  
First, they are construced from the curls of vector fields and
so their divergence vanishes (\ref{eqn:vct_idn_dvr_crl}), as 
(\ref{eqn:mxw_dsp_gss})--(\ref{eqn:mxw_dsp_mgn}).
Second, relations (\ref{eqn:vshm_crl}) and (\ref{eqn:vshn_crl}) show
that the curls of $\vshm$ and $\vshn$ are mutually proportional, as
are (\ref{eqn:mxw_dsp_frd})--(\ref{eqn:mxw_dsp_amp}).
Third, relations (\ref{eqn:vshm_wv_eqn}) and (\ref{eqn:vshn_wv_eqn})
show that $\vshm$ and $\vshn$ satisfy the vector wave equations.

\subsection[Separation of Variables]{Separation of Variables}\label{sxn:SOV}
We now seek a function $\sclfld$ which satisfies the scalar wave
equation (the \trmidx{Helmholtz equation}) in spherical coordinates 
\begin{eqnarray}
% BoH83 p. 84 (4.2)
% Arf85 p. 115 (2.83) 
\nabla^{2} \sclfld + \wvnbr^{2} \sclfld & = & 0 \\
\frac{1}{\rdl^{2}} \frac{\partial}{\partial \rdl} 
\left( \rdl^{2} \frac{\partial \sclfld}{\partial \rdl} \right) +
\frac{1}{\rdl^{2} \sin \plr} \frac{\partial}{\partial \plr}
\left( \sin \plr \frac{\partial \sclfld}{\partial \plr} \right) + 
\frac{1}{\rdl^{2} \sin^{2} \plr} \frac{\partial^{2} \sclfld}{\partial \azi^{2}}
+ \wvnbr^{2} \sclfld & = & 0
\label{eqn:wv_eqn_scl_sph}
\end{eqnarray}
The \trmdfn{separation of variables} procedure attempts to find
$\sclfld$ as the product of three independent functions---one for each
radial and angular coordinate: 
\begin{eqnarray}
\sclfld( \rds, \plr, \azi) & = & \rdlfnc (\rds) \plrfnc (\plr) \azifnc (\azi)
\label{eqn:scl_sov_eqn}
\end{eqnarray}
where $\rdlfnc(\rds)$, $\plrfnc(\plr)$, and $\azifnc(\azi)$ contain
the radial, polar, and azimuthally dependent components of $\sclfld$. 
Substituting (\ref{eqn:scl_sov_eqn}) into (\ref{eqn:wv_eqn_scl_sph})
and dividing by $\rdlfnc\plrfnc\azifnc$ we obtain
\cite[][p.~115]{Arf85} 
\begin{eqnarray}
% SOV on scalar wave equation shown on Arf85 p. 115 (2.83) 
\frac{1}{\rdlfnc\rdssqr} \frac{\dfr}{\dfr\rds} 
\left( \rdssqr \frac{\dfr\rdlfnc}{\dfr\rds} \right) +
\frac{1}{\plrfnc\rdssqr\sin\plr} \frac{\dfr}{\dfr\plr}
\left( \sin\plr \frac{\dfr\plrfnc}{\dfr\plr} \right) +
\frac{1}{\azifnc\rdssqr\sin^{2}\plr} \frac{\dfr^{2}\azifnc}{\dfr\azi^{2}}
& = & -\wvnbr^{2}
\label{eqn:wvn_eqn_sph_sov}
\end{eqnarray}
The differentials are full, not partial.

\subsubsection[Azimuthal Solutions]{Azimuthal Solutions}\label{sxn:mie_azi}
We isolate the second order derivative of $\azifnc$ by multiplying
(\ref{eqn:wvn_eqn_sph_sov}) by $\rdssqr\sin^{2}\plr$---
this is ill-defined at the poles, a detail we neglect:
\begin{eqnarray}
% SOV on scalar wave equation shown on Arf85 p. 115 (2.83) 
\frac{1}{\azifnc} \frac{\dfr^{2}\azifnc}{\dfr\azi^{2}}
& = & \rdssqr\sin^{2}\plr \left[ -\wvnbr^{2} -
\frac{1}{\rdlfnc\rdssqr} \frac{\dfr}{\dfr\rds} 
\left( \rdssqr \frac{\dfr\rdlfnc}{\dfr\rds} \right) -
\frac{1}{\plrfnc\rdssqr\sin\plr} \frac{\dfr}{\dfr\plr}
\left( \sin\plr \frac{\dfr\plrfnc}{\dfr\plr} \right) 
\right]
\label{eqn:wvn_eqn_sph_sov_azi_spr}
\end{eqnarray}
Equation~(\ref{eqn:wvn_eqn_sph_sov_azi_spr}) has the interesting
property that its LHS and RHS are equal yet depend on completely
independent variables ($\azi$ and $\rds,\plr$, respectively).
This can be true for all values of $\rds$, $\plr$, and~$\azi$ if and
only if both sides are constant---the so-called 
\trmidx{separation constant}.
Since the LHS is an azimuthal angle, we expect $\azifnc(\azi)$ to be
periodic, i.e., to repeat every $2\mpi$ radians.
Recognizing the LHS allows simple trigonometric solutions, we denote
the separation constant by~$-\aziidx^{2}$.
\begin{eqnarray}
\frac{1}{\azifnc} \frac{\dfr^{2}\azifnc}{\dfr\azi^{2}} & = &
-\aziidx^{2} \nonumber \\
\frac{\dfr^{2}\azifnc}{\dfr\azi^{2}} & = & -\aziidx^{2}\azifnc \nonumber \\
\azifncidx(\azi) & = & \cst_{1,\aziidx}\me^{\mi\aziidx\azi} + \cst_{2,\aziidx}\me^{-\mi\aziidx\azi} \\ 
\azifncidx(\azi) & = & \cst_{1,\aziidx}\cos\aziidx\azi + \cst_{2,\aziidx}\sin\aziidx\azi
\label{eqn:wvn_eqn_sph_azi_sln}
\end{eqnarray}
where $\cst_{1,\aziidx}$ and $\cst_{2,\aziidx}$ are constants of
integration determined by the boundary conditions. 
A requirement of our physical problem, Mie scattering, is that
the resulting electromagnetic fields be single-valued.
In other words, a given azimuthal angle $\azi$ must be associated
with only one functional value so that
\begin{eqnarray}
\azifnc(\azi + 2\mpi) & = & \azifnc(\azi)
\label{eqn:azi_spr_cst}
\end{eqnarray}
Requirement (\ref{eqn:azi_spr_cst}) can be met only if $\aziidx$
is an integer. 
Consider, for example $\aziidx = \tfrac{1}{2}$.
In this hypothetical case, $\azifnc(\azi)$ is double-valued since
\begin{equation}
\cos\aziidx\azi = \cos\tfrac{\azi}{2} \neq \cos\tfrac{\azi+2\mpi}{2}
= \cos(\tfrac{\azi}{2}+\mpi) = -\cos\tfrac{\azi}{2}
\label{eqn:azi_spr_sng_val}
\end{equation}
This does not satisfy requirement (\ref{eqn:azi_spr_cst}).
Hence the physical requirements of the Mie problem require that
$\aziidx$ is integral.

The symmetries involved in Mie scattering make it useful to
distinguish the even and odd components of~$\azifnc$. 
We note that $\cos\aziidx\azi$ and $\sin\aziidx\azi$ are symmetric and 
anti-symmetric, respectively, so that
\begin{subequations}
\begin{align}
\cst_{1,\aziidx}\cos\aziidx\azi & = \cst_{1,\aziidx}\cos[(-\aziidx)\azi] \\
\cst_{2,\aziidx}\sin\aziidx\azi & = -\cst_{2,\aziidx}\sin[(-\aziidx)\azi]
\label{eqn:evnodd}
\end{align}
\end{subequations}
Following \cite{BoH83}, p.~85, we subscript even functions
with~$\evnsbs$ and odd function with~$\oddsbs$. 
\begin{subequations}
\begin{align}
\azievnidx & = \cst_{1,\aziidx}\cos\aziidx\azi \nonumber \\
\azioddidx & = \cst_{2,\aziidx}\sin\aziidx\azi \nonumber \\
\azifncidx(\azi) & = \azievnidx + \azioddidx
\label{eqn:wvn_eqn_sph_azi_sln_evn_odd}
\end{align}
\end{subequations}
The negative sign arising from anti-symmetric functions 
(\ref{eqn:evnodd}) may be subsumed into $\cst_{2,\aziidx}$ so that all
linear all linear combinations of $\azievnidx$ and $\azioddidx$ are
obtainable with positive integers~$\aziidx$.
Hence we may assume that $\aziidx$ takes only positive (or zero)
integer values.

Since $\cos\aziidx\azi$ and $\sin\aziidx\azi$ are
\trmidx{orthonormal}, any azimuthal function can be decomposed into 
a series of these \trmidx{basis functions}
\begin{eqnarray}
\azifnc(\azi) & = & \sum_{\aziidx=0}^{\infty} 
\cst_{1,\aziidx}\cos\aziidx\azi + \cst_{2,\aziidx}\sin\aziidx\azi
\label{eqn:wvn_eqn_sph_azi_sln_cst}
\end{eqnarray}

\subsubsection[Polar Solutions]{Polar Solutions}\label{sxn:mie_plr}
The polar and radial components of (\ref{eqn:wvn_eqn_sph_sov_azi_spr}) 
now must satisfy
\begin{eqnarray}
\frac{1}{\rdlfnc\rdssqr} \frac{\dfr}{\dfr\rds} 
\left( \rdssqr \frac{\dfr\rdlfnc}{\dfr\rds} \right) +
\frac{1}{\plrfnc\rdssqr\sin\plr} \frac{\dfr}{\dfr\plr}
\left( \sin\plr \frac{\dfr\plrfnc}{\dfr\plr} \right) -
\frac{\aziidx^{2}}{\rdssqr\sin^{2}\plr}
& = & -\wvnbr^{2}
\label{eqn:wvn_eqn_sph_rdl_plr}
\end{eqnarray}
To separate the radial and polar terms, multiply
(\ref{eqn:wvn_eqn_sph_rdl_plr}) by $\rdssqr$ and re-group
\begin{eqnarray}
\frac{1}{\rdlfnc} \frac{\dfr}{\dfr\rds} 
\left( \rdssqr \frac{\dfr\rdlfnc}{\dfr\rds} \right) +
\frac{1}{\plrfnc\sin\plr} \frac{\dfr}{\dfr\plr}
\left( \sin\plr \frac{\dfr\plrfnc}{\dfr\plr} \right) -
\frac{\aziidx^{2}}{\sin^{2}\plr}
& = & -\wvnbr^{2}\rdssqr \nonumber \\
\frac{1}{\rdlfnc} \frac{\dfr}{\dfr\rds} 
\left( \rdssqr \frac{\dfr\rdlfnc}{\dfr\rds} \right) +
\wvnbr^{2}\rdssqr
& = & 
- \frac{1}{\plrfnc\sin\plr} \frac{\dfr}{\dfr\plr}
\left( \sin\plr \frac{\dfr\plrfnc}{\dfr\plr} \right) 
+ \frac{\aziidx^{2}}{\sin^{2}\plr}
\label{eqn:wvn_eqn_sph_rdl_plr_sov}
\end{eqnarray}
The LHS and RHS of Equation~(\ref{eqn:wvn_eqn_sph_rdl_plr_sov}) must
each equal a (new) separation constant for the same reasons given for 
the separation constant in
Equation~(\ref{eqn:wvn_eqn_sph_sov_azi_spr}).  
We label this separation constant~$\QQQ$.
\begin{eqnarray}
- \frac{1}{\plrfnc\sin\plr} \frac{\dfr}{\dfr\plr}
\left( \sin\plr \frac{\dfr\plrfnc}{\dfr\plr} \right) 
+ \frac{\aziidx^{2}}{\sin^{2}\plr}
& = & \QQQ \nonumber \\
\frac{1}{\sin\plr} \frac{\dfr}{\dfr\plr}
\left( \sin\plr \frac{\dfr\plrfnc}{\dfr\plr} \right) 
+ \left[ \QQQ - \frac{\aziidx^{2}}{\sin^{2}\plr} \right] \plrfnc
& = & 0
\label{eqn:wvn_eqn_sph_plr_sov}
\end{eqnarray}
The radial component is
\begin{eqnarray}
\frac{1}{\rdlfnc} \frac{\dfr}{\dfr\rds} 
\left( \rdssqr \frac{\dfr\rdlfnc}{\dfr\rds} \right) -
\wvnbr^{2}\rdssqr
& = & \QQQ \nonumber \\
\frac{\dfr}{\dfr\rds} 
\left( \rdssqr \frac{\dfr\rdlfnc}{\dfr\rds} \right) +
[ \wvnbr^{2}\rdssqr - \QQQ ] \rdlfnc
& = & 0
\label{eqn:wvn_eqn_sph_rdl_sov}
\end{eqnarray}

It is a well-known result that the separation constant $\QQQ$ is
quantized and takes positive integral values defined by 
$\QQQ = \plridx(\plridx+1)$.
The reasoning for this is fxm.
Substituting this into (\ref{eqn:wvn_eqn_sph_plr_sov})
and~(\ref{eqn:wvn_eqn_sph_rdl_sov}) and we obtain
\begin{eqnarray}
\label{eqn:wvn_eqn_sph_plr_sln}
\frac{1}{\sin\plr} \frac{\dfr}{\dfr\plr}
\left( \sin\plr \frac{\dfr\plrfnc}{\dfr\plr} \right) 
+ \left[ \plridx(\plridx+1) - \frac{\aziidx^{2}}{\sin^{2}\plr} \right] \plrfnc
& = & 0 \\
\label{eqn:wvn_eqn_sph_rdl_sln}
\frac{\dfr}{\dfr\rds} 
\left( \rdssqr \frac{\dfr\rdlfnc}{\dfr\rds} \right) +
[ \wvnbr^{2}\rdssqr - \plridx(\plridx+1) ] \rdlfnc
& = & 0
\end{eqnarray}
Equation~(\ref{eqn:wvn_eqn_sph_plr_sln}) is the 
\trmidx{associated Legendre equation} and its solutions are the 
associated Legendre polynomials.
The solutions are \trmidx{Legendre polynomials} of degree~$\plridx$
and order~$\aziidx$, denoted by $\lgnassplrazi(\cos\plr)$.
Appendix~\ref{sxn:lgn} describes many properties of Legendre
polynomials including series expansions and recurrence formulae.

Legendre polynomials are usually expressed in terms of 
$\plrmu = \cos\plr$.
Note that the rest of this monograph uses 
$\plrmu = |\cos\plr|$ to refer to the zenith angle with respect to
an entire atmosphere.
The present section is concerned with the polar angle of a single
particle, and for this use only, and to maintain consistency with
other Mie theory presentations \cite[e.g.,][]{BoH83,Lio02} we define 
$\plrmu = \cos\plr$.
\begin{eqnarray}
\frac{\dfr}{\dfr\plrmu}
\left[ (1 - \plrmu^{2}) \frac{\dfr\plrfnc}{\dfr\plrmu} \right] +
\left[ \plridx(\plridx+1) - \frac{\aziidx^{2}}{1-\plrmu^{2}} \right] \plrfnc
& = & 0
\label{eqn:wvn_eqn_sph_plrmu_sln}
\end{eqnarray}
The associated Legendre polynomials of degree~$\plridx$ and
order~$\aziidx$ which satisfy (\ref{eqn:wvn_eqn_sph_plrmu_sln}) are
denoted by $\lgnassplrazi(\plrmu)$. 

\subsubsection[Radial Solutions]{Radial Solutions}\label{sxn:mie_rdl}
We turn now to the solution of the radial
equation~(\ref{eqn:wvn_eqn_sph_rdl_sln}), which we re-write as  
\begin{eqnarray}
\rdssqr \frac{\dfr^{2}\rdlfnc}{\dfr\rds^{2}} +
2\rds \frac{\dfr\rdlfnc}{\dfr\rds} +
[ \wvnbr^{2}\rdssqr - \plridx(\plridx+1) ] \rdlfnc
& = & 0
\label{eqn:wvn_eqn_sph_rdl_xpn}
\end{eqnarray}
The algebraic form of (\ref{eqn:wvn_eqn_sph_rdl_xpn}) simplifies if we
change independent variables from~$\rds$ to the dimensionless~$\wvnrds$
\begin{subequations}
\begin{align}
\wvnrds & = \wvnbr\rds \\
\rds & = \wvnrds/\wvnbr \\
\dfr\rds & = \wvnbr^{-1} \,\dfr\wvnrds \\
\dfr\wvnrds & = \wvnbr \,\dfr\rds
\label{eqn:bsl_cov}
\end{align}
\end{subequations}
to yield
\begin{eqnarray}
\left( \frac{\wvnrds}{\wvnbr} \right)^{2}
\left( \frac{1}{\wvnbr^{-1}} \right)^{2}
\frac{\dfr^{2}\rdlfnc}{\dfr\wvnrds^{2}} +
2 \left( \frac{\wvnrds}{\wvnbr} \right)
\left( \frac{1}{\wvnbr^{-1}} \right)
\frac{\dfr\rdlfnc}{\dfr\wvnrds} +
\left[ \wvnbr^{2} 
\left( \frac{\wvnrds}{\wvnbr} \right)^{2}
- \plridx(\plridx+1) \right] \rdlfnc 
& = & 0 \nonumber \\
\wvnrds^{2} \frac{\dfr^{2}\rdlfnc}{\dfr\wvnrds^{2}} +
2 \wvnrds \frac{\dfr\rdlfnc}{\dfr\wvnrds} +
[ \wvnrds^{2} - \plridx(\plridx+1) ] \rdlfnc 
& = & 0
\label{eqn:bsl_wvnrds}
\end{eqnarray}
Equation~(\ref{eqn:bsl_wvnrds}) is not tractable in its present form.
However, a re-defined radial function will transform
(\ref{eqn:bsl_wvnrds}) into \trmidx{Bessel's equation}.
Define $\bslZfnc$ by
\begin{subequations}
\begin{align}
\rdlfnc & = \wvnrds^{-1/2} \bslZfnc \\
\bslZfnc & = \sqrt{\wvnrds} \rdlfnc 
\label{eqn:bsl_cov}
\end{align}
\end{subequations}
and substitute it into (\ref{eqn:bsl_wvnrds}) to obtain
\begin{eqnarray}
\wvnrds^{2} \frac{\dfr^{2}(\wvnrds^{-1/2}\bslZfnc)}{\dfr\wvnrds^{2}} +
2 \wvnrds \frac{\dfr(\wvnrds^{-1/2}\bslZfnc)}{\dfr\wvnrds} +
[ \wvnrds^{2} - \plridx(\plridx+1) ] (\wvnrds^{-1/2}\bslZfnc)
& = & 0 \nonumber \\
\wvnrds^{2} 
\frac{\dfr}{\dfr\wvnrds} 
\left( -\frac{\wvnrds^{-3/2}}{2} \bslZfnc + \wvnrds^{-1/2} 
\frac{\dfr\bslZfnc}{\dfr\wvnrds} \right) + \cdots
& & \nonumber \\
{}\cdots + 2 \wvnrds 
\left( -\frac{\wvnrds^{-3/2}}{2} \bslZfnc + \wvnrds^{-1/2} 
\frac{\dfr\bslZfnc}{\dfr\wvnrds} \right) +
[ \wvnrds^{2} - \plridx(\plridx+1) ] \wvnrds^{-1/2} \bslZfnc
& = & 0 \nonumber \\
\wvnrds^{2} 
\left(
\frac{3\wvnrds^{-5/2}}{4} \bslZfnc -
\frac{\wvnrds^{-3/2}}{2} \frac{\dfr\bslZfnc}{\dfr\wvnrds} -
\frac{\wvnrds^{-3/2}}{2} \frac{\dfr\bslZfnc}{\dfr\wvnrds} +
\wvnrds^{-1/2} \frac{\dfr^{2}\bslZfnc}{\dfr\wvnrds^{2}} 
\right) - \cdots
& & \nonumber \\
{}\cdots -
\wvnrds^{-1/2} \bslZfnc +
2 \wvnrds^{1/2} \frac{\dfr\bslZfnc}{\dfr\wvnrds} +
[ \wvnrds^{2} - \plridx(\plridx+1) ] \wvnrds^{-1/2} \bslZfnc
& = & 0 \nonumber \\
\frac{3\wvnrds^{-1/2}}{4} \bslZfnc -
\wvnrds^{1/2} \frac{\dfr\bslZfnc}{\dfr\wvnrds} +
\wvnrds^{3/2} \frac{\dfr^{2}\bslZfnc}{\dfr\wvnrds^{2}} -
\wvnrds^{-1/2} \bslZfnc +
2 \wvnrds^{1/2} \frac{\dfr\bslZfnc}{\dfr\wvnrds} +
[ \wvnrds^{2} - \plridx(\plridx+1) ] \wvnrds^{-1/2} \bslZfnc
& = & 0 \nonumber \\
\frac{3}{4} \bslZfnc -
\wvnrds \frac{\dfr\bslZfnc}{\dfr\wvnrds} +
\wvnrds^{2} \frac{\dfr^{2}\bslZfnc}{\dfr\wvnrds^{2}} -
\bslZfnc +
2 \wvnrds \frac{\dfr\bslZfnc}{\dfr\wvnrds} +
[ \wvnrds^{2} - \plridx(\plridx+1) ] \bslZfnc
& = & 0 \nonumber \\
\wvnrds^{2} \frac{\dfr^{2}\bslZfnc}{\dfr\wvnrds^{2}} +
\wvnrds \frac{\dfr\bslZfnc}{\dfr\wvnrds} -
\frac{\bslZfnc}{4} +
[ \wvnrds^{2} - \plridx(\plridx+1) ] \bslZfnc
& = & 0 \nonumber \\
\wvnrds^{2} \frac{\dfr^{2}\bslZfnc}{\dfr\wvnrds^{2}} +
\wvnrds \frac{\dfr\bslZfnc}{\dfr\wvnrds} +
[ \wvnrds^{2} - (\plridx^{2} + \plridx + \tfrac{1}{4}) ] \bslZfnc
& = & 0 \nonumber \\
\wvnrds^{2} \frac{\dfr^{2}\bslZfnc}{\dfr\wvnrds^{2}} +
\wvnrds \frac{\dfr\bslZfnc}{\dfr\wvnrds} +
[ \wvnrds^{2} - (\plridx + \tfrac{1}{2})^{2} ] \bslZfnc
& = & 0
\label{eqn:bsl_wvnrds_sph}
\end{eqnarray}
Equation~(\ref{eqn:bsl_wvnrds_sph}) is \trmdfn{Bessel's equation}
\cite[][p.~19]{Wat58} for half-integral values.
The solutions are \trmidx{Bessel functions} of half-integral order,
$\bslZfnc = \bslZfnc_{\plridx+\sfrac{1}{2}}(\wvnrds)$. 
Appendix~\ref{sxn:bsl} describes many properties of Bessel functions, 
including 
\trmidx[Bessel functions, modified]{modified Bessel functions},
asymptotic limits, and recurrence formulae.

In terms of the radial coordinate~$\rds$ and wavenumber~$\wvnbr$, the
solutions to (\ref{eqn:wvn_eqn_sph_rdl_xpn}) are 
\begin{eqnarray}
\rdlfnc(\plridx;\rds,\wvnbr) & = & \frac{1}{\sqrt{\wvnbr\rds}}
\bslZfnc_{\plridx+\sfrac{1}{2}}(\wvnbr\rds)
\label{eqn:wvn_eqn_sph_rdl_sln}
\end{eqnarray}
The general solution to our Helmholtz equation in spherical
coordinates (\ref{eqn:wv_eqn_scl_sph}), is the product
(\ref{eqn:scl_sov_eqn}) of the radial, polar, and azimuthal solutions
from 
(\ref{eqn:wvn_eqn_sph_rdl_sln}), (\ref{eqn:wvn_eqn_sph_plr_sln}), and
(\ref{eqn:wvn_eqn_sph_azi_sln}), respectively
\begin{eqnarray}
\sclfld(\plridx,\aziidx;\rds,\plr,\azi) & = & 
\frac{1}{\sqrt{\wvnbr\rds}} \bslZfnc_{\plridx+\sfrac{1}{2}}(\wvnbr\rds)
\lgnassplrazi(\cos\plr)
(\cst_{1,\aziidx}\cos\aziidx\azi + \cst_{2,\aziidx}\sin\aziidx\azi)
\label{eqn:wv_eqn_sph_sln}
\end{eqnarray}

The physical environment of Mie scattering allows us to restrict the
radial solutions $\bslZfnc_{\plridx+\sfrac{1}{2}}(\wvnbr\rds)$ to 
certain subsets of Bessel functions.
It is advantageous to explain the rationale for doing so now before
the nomenclature becomes even more cumbersome as it will once we
expand the incident wave into spherical harmonics
Section~\ref{sxn:xpn} and apply the boundary conditions
Section~\ref{sxn:bc}.
As described in Appendix~\ref{sxn:bsl}, 
the spherical Bessel functions $\bslzntg(\wvnbr\rds)$
(\ref{eqn:bsl_sph_dfn}) are suitably normalized re-definitions of 
$\bslZfnc_{\plridx+\sfrac{1}{2}}(\wvnbr\rds)$. 
The two linearly independent sets of basis function which
$\bslzntg(\wvnbr\rds)$ represents are 
(1)~$\bsljntg(\wvnbr\rds)$ and $\bslyntg(\wvnbr\rds)$,  
and (2)~$\bslhonentg(\wvnbr\rds)$ and $\bslhtwontg(\wvnbr\rds)$. 

The first physical restriction is that the scattered electromagnetic 
field must be finite at the origin.
$\bslyntg(\wvnbr\rds)$ is unbounded at the origin so
$\bsljntg(\wvnbr\rds)$ must suffice for Mie solutions interior to   
the sphere. 
The second physical restriction is that the scattered electromagnetic 
field must be finite at large distances from the sphere, as 
$\rds \to \infty$.
$\bsljntg(\wvnbr\rds)$ is unbounded as $\rds \to \infty$ whereas
the Hankel functions are well-behaved.
Hence $\bslhonentg(\wvnbr\rds)$ and $\bslhtwontg(\wvnbr\rds)$ will
be used for Mie solutions exterior to the sphere.

Note that (\ref{eqn:wv_eqn_sph_sln}) depends on $\wvnbr\rds$ rather 
than $\rds$. 
The radius-to-wavelength ratio is physically meaningful in the
Helmholtz equation, rather than the absolute radius or wavelength.
Of course (\ref{eqn:wv_eqn_sph_sln}) only defines the generating
function that accounts for the geometry of the scattering problem.  
In practice, the dielectric constant (i.e., index of refraction) does
depend on the absolute wavelenth, and this information propogates
into the solutions for the electric and magnetic fields which are 
coupled through the Maxwell equations.
Some explications of Mie theory \cite[e.g.,][p.~180]{Lio02} include
refractive index dependence in the solution of the Helmholtz equation
rather than waiting until applying the boundary conditions, as we
shall do.

\subsubsection[Plane Wave Expansion]{Expansion of Plane Wave into Spherical Harmonics}\label{sxn:xpn}
Consider the scattering of a linearly polarized plane wave
(\ref{eqn:pln_wv_gnr_dfn}) by a sphere of radius~$\rdsmie$ centered at
the origin of the Cartesian coordinate system.  
In this coordinate system, an \trmdfn{incident plane wave}
may be written as 
\begin{eqnarray}
\lctvctncd(\drcvct,\tm) & = & 
\lctvctnot \exp[\mi (\wvnbrvctncd \cdot \drcvct - \frqngl \tm ) ]
\label{eqn:pln_wv_ncd_gnr}
\end{eqnarray}
where the subscript $\ncdsbs$ denotes ``incident''.
We may simplify the representation of $\lctvctncd(\drcvct,\tm)$
without loss of generality by picking our coordinate system to
align with the incident plane wave.
By convention we choose (1)~the incident wave to propogate toward the
$\zzz$-direction, and (2)~the direction of polarization of the
incident electric vector is parallel to the $\xxx$-axis.
Under our assumptions, the incident plane waves have no components
in the $\jhat$ or $\khat$ directions because we assumed linear
polarization in the $\ihat$~direction. 
Hence, $\lctvctncd$ may be represented by the simple
scalar~$\lctfldnot$.  
Furthermore, the waves are traveling in the positive $\zzz$-direction,
so the wavenumber vector~$\wvnbrvctncd$ in (\ref{eqn:pln_wv_ncd_gnr})
collapses to the scalar~$\wvnbr$, and its dot product with position
simplifies $\wvnbrvctncd \cdot \drcvct = \wvnbr\zzz$.
Finally, we assume the incident field is constant in time so that
$\lctvctncd(\drcvct,\tm) = \lctvctncd(\drcvct)$.
Accordingly, (\ref{eqn:pln_wv_ncd_gnr}) simplifies to
\begin{eqnarray}
\lctvctncd(\drcvct) & = & \lctfldnot \me^{\mi\wvnbr\zzz} \ihat
\label{eqn:wv_eqn_ncd}
\end{eqnarray}

fxm
% Applying \label{eqn:crt_sph_unit_vct_trn} to

\subsubsection[Boundary Conditions]{Boundary Conditions}\label{sxn:bc}

\subsubsection[Mie Theory]{Mie Theory}\label{sxn:mie}
The solution of Maxwell's equations (\ref{eqn:mxw_eqn_SI}) for the
geometry of the aerosol (usually considered to be spherical) in the
medium of interest (e.g., air or ocean water) yields the connection
between the index of refraction~$\idxrfr$ and the single scattering
properties $\tauext$, $\ssa$, and~$\asmprm$.
The complete solution to this important problem was derived
independently by physicists Ludwig Lorenz in 1890 and Gustav Mie in
1908.
The subject is usually called \trmdfn{Mie theory} in honor of the
latter, but \trmdfn{Lorenz-Mie} theory would be more appropriate.
In-depth discussions of Mie theory are presented in \cite{Van57},
\cite{HaT74}, and \cite{BoH83}.

Consider the scattering of a linearly polarized plane wave
(\ref{eqn:pln_wv_dfn}) by a sphere of radius $\rds$ centered at the
origin of the Cartesian coordinate system.  
The plane wave propogates toward the $\zzz$-direction. 
The direction of polarization of the incident electric vector is along 
the $\xxx$-axis.
The subscripts $\ncdsbs$, $\prtsbs$, and $\sctsbs$ refer to the 
\trmidx{incident wave}, \trmidx{particle wave} (i.e., the wave
interior to the particle), and \trmidx{scattered wave}, respectively.
\begin{subequations}
\label{eqn:lct_mgn_fld_dfn}
\begin{align}
\label{eqn:lct_fld_ncd_dfn}
\lctvctncd(\drcvct,\tm) & = \sum_{\srsidx=1}^{\infty} 
\lctfldtrm [ \vshmoonen^{\sbfksbs} - \mi \vshneonen^{\sbfksbs} ] \\
\label{eqn:mgn_fld_ncd_dfn}
\mgnvctncd(\drcvct,\tm) & = -\frac{\wvnbrmdm}{\frqngl \lctmumdm}
\sum_{\srsidx=1}^{\infty} \lctfldtrm [ \vshmeonen^{\sbfksbs} + \mi
\vshnoonen^{\sbfksbs} ] \\
\label{eqn:lcd_fld_prt_dfn}
\lctvctprt(\drcvct,\tm) & = \sum_{\srsidx=1}^{\infty} 
\lctfldtrm [ \mieccctrm \vshmoonen^{\sbfksbs} - 
\mi \miedddtrm \vshneonen^{\sbfksbs} ] \\
\label{eqn:mgn_fld_prt_dfn}
\mgnvctprt(\drcvct,\tm) & = -\frac{\wvnbrprt}{\frqngl \lctmuprt}
\sum_{\srsidx=1}^{\infty} \lctfldtrm [ \miedddtrm \vshmoonen^{\sbfksbs} 
+ \mi \mieccctrm \vshnoonen^{\sbfksbs} ] \\
\label{eqn:lct_fld_sct_dfn}
\lctvctsct(\drcvct,\tm) & = \sum_{\srsidx=1}^{\infty} 
\lctfldtrm [ \mi \mieaaatrm \vshneonen^{\sbfhsbs} - 
\miebbbtrm \vshmoonen^{\sbfhsbs} ) \\
\label{eqn:mgn_fld_sct_dfn}
\mgnvctsct(\drcvct,\tm) & = -\frac{\wvnbrmdm}{\frqngl \lctmumdm}
\sum_{\srsidx=1}^{\infty} \lctfldtrm [ \mi \miebbbtrm \vshnoonen^{\sbfhsbs} 
+ \mieaaatrm \vshmeonen^{\sbfhsbs} ]
\end{align}
\end{subequations} 
where 
\begin{eqnarray}
\lctfldtrm & = & \frac{\mi^{\srsidx}(2\srsidx + 1)}
{\srsidx(\srsidx + 1)} \lctfldnot
\label{eqn:lct_fld_trm_dfn}
\end{eqnarray}
and $\lctfldnot$, the amplitude of the incident 
\trmidx{electric field}, is defined in (\ref{eqn:pln_wv_dfn}). 
The superscripts $\sbfksbs$ and $\sbfhsbs$ in
(\ref{eqn:lct_fld_trm_dfn}) indicate the type of \trmidx{spherical
Bessel function} into which the electric and magnetic fields are
decomposed.
The complex wavenumbers within the particle, $\wvnbrprt$, and the
medium, $\wvnbrmdm$, are defined by, respectively
\begin{subequations}
\label{eqn:wvnbrmie_dfn}
\begin{align}
\label{eqn:wvnbrprt_dfn}
\wvnbrprt & = 2 \mpi \idxrfrprt / \wvlvcm \\
\label{eqn:wvnbrmdm_dfn}
\wvnbrmdm & = 2 \mpi \idxrfrmdm / \wvlvcm
\end{align}
\end{subequations} 

The \trmdfn{vector spherical harmonic} expansions of $\vshm$ and
$\vshn$ are
\begin{subequations}
\label{eqn:vsh_dfn}
\begin{align}
\label{eqn:vshmoonen_dfn}
\vshmoonen & = 
\cos \azi \miepitrm (\cos \plr) \bslzntg (\rdsrho) \plrhat -
\sin \azi \mietautrm (\cos \plr) \bslzntg (\rdsrho) \azihat \\
\label{eqn:vshmeonen_dfn}
\vshmeonen & = \\
\label{eqn:vshnoonen_dfn}
\vshnoonen & = \\
\label{eqn:vshneonen_dfn}
\vshneonen & =  
\end{align}
\end{subequations} 
where the definition of the coordinate $\rdsrho$ at which the radial
Bessel functions are evaluated depends on the field location.
Within the particle, the radial coordinate is the radial distance
times the internal wavenumber, while the radial coordinate used to
evaluate the incident and scattered fields (which are outside the
particle) 
\begin{eqnarray}
\rdsrho & = & \left\{
\begin{array}{l@{\quad:\quad}ll}
\rds \wvnbrprt & \mbox{Internal field} \\
\rds \wvnbrmdm & \mbox{Incident and Scattered fields}
\end{array} \right.
\label{eqn:rds_rho_dfn}
\end{eqnarray}

\subsubsection[Resonances]{Resonances}\label{sxn:rsn}
Recent studies suggest that \trmidx{resonant absorption} of sunlight
by cloud droplets may constitute a significant and unaccounted-for
solar energy sink in the atmosphere \cite[e.g,][]{Nus03}.
Many studies refer to the impact of \trmidx{resonances} on absorptance 
\cite[e.g.,][]{CKK781,CKK782,BeR78,BoH83,GuN94,MaS99,Mit00,Mar02,Nus03,ZeT06}. 
Resolving all sharp resonances requires a resolution in size parameter 
$\szprm = 2\mpi\rds/\wvl$ ($\rds$---droplet radius, $\wvl$---incident
wavelength) of about~$10^{-7}$ \cite[][]{CKK781}.

This section describes application of resonance enhancement 
by absorbing particles such as soot in otherwise weakly absorbing
spheres such as liquid cloud droplets.
\cite{MaS99} present an exact theory for this enhancement.
\cite{Mar02} shows how resonances amplify this enhancement.

\cite{MaS99} present a general theory of energy absorption by
absorbing spheres in weakly-absorbing media. 
The theory has three main assumptions:
\begin{enumerate*}
\item fxm
\end{enumerate*}
The absorption enhancement factor $\abshnsfct$ is the ratio of
absorption $\xsxabs$ by an inclusion inside a relatively weakly
absorbing medium to the absorption $\xsxabsxtr$ of the same inclusion
in a vacuum. 
For concreteness, we may henceforth refer to the absorbing inclusion
as the soot cluster and the relatively weakly absorbing spherical
medium as the cloud droplet.
\begin{equation}
\abshnsfct(\rdscld,\wvl) =
\frac{\xsxabs(\rdscld,\wvl)}{\xsxabsxtr(\rdsncl,\wvl)} 
\label{eqn:abs_hns_fct_dfn}
\end{equation}
where subscripts $\cldsbs$ and $\nclsbs$ refer to the cloud droplet
and the soot inclusion, respectively.
Hence $\rdscld$ is the cloud droplet radius, $\rdsncl$ is the soot 
inclusion radius, and the cloud droplet size parameter
$\szprmcld = 2\mpi\rdscld/\wvl$.

With the above assumptions, the enhancement factor $\abshnsfct$
(\ref{eqn:abs_hns_fct_dfn}) depends only on the wavelength and the
size of the cloud droplet, not on the size of the inclusion.

\csznote{
Hello Vadim,

I am working with Jorge Talamantes to apply your theory of 
absorption by strongly absorbing inclusions (soot) in weakly 
absorbing spheres (water) to the case of marine stratocumulus clouds. 
I don't believe we have directly communicated in the past, but I have
been following Jorge's exchanges with you closely and I appreciate
your helpful responses to our past questions.
I have a few more questions designed to ensure we apply your theory
correctly. I apologize in advance for the length of this message.
I wanted to get the terminology down on paper.
It may be easier to speak by phone to actually answer these questions.
If so, call me at the number below or let me know and I'll call you.

We are confident that we are predicting the same enhancement factor G
you presented in your 1999 and 2002 JQSRT papers because you gave 
Jorge the code (thanks) and our results with it are quite similar.
My questions have to do with the correct application of G to obtain
the final microphysical optical properties (absorption, extinction,
and phase function) for use as inputs to our atmospheric radiative
transfer model. 

As I mentioned we are working on sooty liquid water clouds.
We use your algorithm to compute G=G(rds_cld,wvl), a function of cloud
droplet radius (rds_cld) and wavelength (wvl) (which together determine
the size parameter) and the real refractive index of liquid water.
This is relatively straightforward so far.
My questions have to do with how to apply G to determine the
droplet absorption cross sections.

Your paper instructs us to apply G as a multiplicative factor to the
absorption cross section of the externally mixed soot aerosol.
We assume the soot to be spherical (for simplicity), and use Mie
theory to compute the externally mixed absorption (abs) efficiency
q_abs of each size of soot particle. 
Then the absorption cross-section of each soot particle, as a function
of wavelength (wvl) and soot aerosol particle radius (rds_aer) is
sigma_abs(wvl,rds_aer) = q_abs(wvl,rds_aer)*pi*rds_aer^2.  
sigma_abs has units of m^2 per soot particle.

We use an assumed lognormal distribution of soot aerosol size to
integrate the soot absorption cross-sections weighted by their number
concentration to obtain the total (ttl) effective absorption cross
section of the entire soot particle distribution sigma_abs_ttl(wvl).
The units of sigma_abs_ttl are also m^2 per soot particle.
Multiplying sigma_abs_ttl by the total number (nbr) of soot aerosol
(aer) particles per square meter in a given column, nbr_aer, would
yield the dimensionless aerosol absorption optical depth (tau)
tau_abs_aer, i.e., tau_abs_aer=sigma_abs_ttl*nbr_aer. 

Since your G-factor is independent of the size of the soot inclusions
(I find this property of G rather amazing) we multiply sigma_abs_ttl
by G to obtain the corrected absorption cross-section of the soot
particles embedded in cloud (cld) droplets. 
Hence sigma_abs_ttl_G(rds_cld,wvl)=G(rds_cld,wvl)*sigma_abs_ttl(wvl).
Here is where I get confused:

1. Are the units of sigma_abs_ttl_G (a) m^2 per soot particle or
   (b) m^2 per cloud particle?

Only (b) makes sense to me, i.e., I think G converts absorption
cross sections from units of per soot particle to per cloud droplet.
If so, then sigma_abs_ttl_G(rds_cld,wvl) plays the same role for
cloud droplet absorption (including internally mixed soot effects)
as sigma_abs(wvl,rds_aer) plays for externally mixed soot absorption.  

2. Is this correct?

If so, then we can integrate sigma_abs_ttl_G(rds_cld,wvl) over
the assumed lognormal cloud droplet number concentration to obtain
sigma_abs_ttl_G_cld(wvl).
Then the dimensionless cloud droplet absorption optical depth
tau_abs_cld through nbr_cld cloud droplets per square meter is
tau_abs_cld(wvl)=sigma_abs_ttl_G_cld(wvl)*nbr_cld.

3. Is this correct?

Finally, there is the issue of double-counting absorption.
fxm
For vanishinginly small concentrations of soot, 

Thanks,
Charlie
} % end csznote

\subsubsection[Optical Efficiencies]{Optical Efficiencies}\label{sxn:fsh_opt}
Mie theory predicts the \trmdfn{optical efficiencies} $\fshxxx$ of
particles of a given size at a given wavelength. 
Here $\xxx$ stands for $\abssbs$, $\sctsbs$, or $\extsbs$ which
represent the processes of absorption, scattering, and extinction,
respectively. 
The optical efficiency for each of these processes is the ratio
between a particle's effective cross-sectional area for the specified
interaction (absorption, scattering, or both) with light and its
geometric cross sectional area. 
\begin{subequations}
\label{eqn:fsh_opt}
\begin{align}
\label{eqn:fsh_abs_dfn}
\fshabs(\rds,\wvl) & = \frac{\xsxabs(\rds,\wvl)}{\mpi \rds^{2}} \\
\label{eqn:fsh_sct_dfn}
\fshsct(\rds,\wvl) & = \frac{\xsxsct(\rds,\wvl)}{\mpi \rds^{2}} \\
\label{eqn:fsh_ext_dfn}
\fshext(\rds,\wvl) & = \frac{\xsxext(\rds,\wvl)}{\mpi \rds^{2}}
\end{align}
\end{subequations} 
Thus the optical efficiencies are dimensionless.
In \cmdidx{mie}, the band-mean versions of $\fshabs$, $\fshsct$, and 
$\fshext$ are named \cmdidx{abs\_fsh}, \cmdidx{sca\_fsh}, and
\cmdidx{ext\_fsh}, respectively. 

These optical efficiencies are not independent of one another.
Two of the efficiency factors, usually the \trmdfn{extinction
efficiency} $\fshext$ and the \trmdfn{scattering efficiency}
$\fshsct$, are predicted directly by Mie theory.
The third efficiency factor, the \trmdfn{absorption efficiency}
$\fshabs$, is the residual that satisfies energy conservation
\begin{eqnarray}
\fshabs(\rds,\wvl) & = & \fshext(\rds,\wvl) - \fshsct(\rds,\wvl)
\label{eqn:fsh_abs_rsd}
\end{eqnarray}
This relation states that extinction is the sum of absorption and
scattering.

As mentioned previously (\ref{eqn:fsh_ext_xmp}), Mie theory shows that  
\begin{eqnarray}
\lim_{\szprm\to\infty} \fshext(\rds,\wvl) & = & 2
\label{eqn:fsh_ext_lmt}
\end{eqnarray}

\subsubsection[Optical Cross Sections]{Optical Cross Sections}\label{sxn:xsx_opt}
As mentioned above, the optical efficiencies $\fshxxx$ are the ratios
between a particle's effective cross-sectional area for interacting
with light (i.e., absorbing or scattering photons) and its geometric
cross-sectional area. 
The interaction cross-sections per particle are 
\begin{subequations}
\label{eqn:xsx_opt}
\begin{align}
\label{eqn:xsx_abs_dfn}
% mie: abs_xsx
\xsxabs(\rds,\wvl) & = \mpi \rds^{2} \fshabs(\rds,\wvl) \\
\label{eqn:xsx_sct_dfn}
% mie: sct_xsx
\xsxsct(\rds,\wvl) & = \mpi \rds^{2} \fshsct(\rds,\wvl) \\
\label{eqn:xsx_ext_dfn}
% mie: ext_xsx
\xsxext(\rds,\wvl) & = \mpi \rds^{2} \fshext(\rds,\wvl)
\end{align}
\end{subequations} 
The relationship between the interaction cross-sections is exactly
analogous to the relationship between the optical efficiencies
(\ref{eqn:fsh_opt}), so that 
\begin{equation}
\xsxext(\rds,\wvl) = \xsxabs(\rds,\wvl) + \xsxsct(\rds,\wvl)
\label{eqn:tau_ext_dfn}
\end{equation}
The optical cross sections have dimensions of area per particle.

\subsubsection[Optical Depths]{Optical Depths}\label{sxn:tau_opt}
Using (\ref{eqn:tau_dfn}) we see that a column of depth
$\hgtdlt$\,[\m] with a homogeneous particle concentration of
$\cncfnc(\rds)$\,[\xmC] produces optical depths of
\begin{subequations}
\label{eqn:tau_opt}
\begin{align}
\label{eqn:tau_opt_abs}
% mie: tau_abs
\tauabs(\rds,\wvl) & = \mpi \rds^{2} \fshabs(\rds,\wvl)
\cncfnc(\rds) \hgtdlt \\
\label{eqn:tau_opt_sct}
% mie: tau_sct
\tausct(\rds,\wvl) & = \mpi \rds^{2} \fshsct(\rds,\wvl)
\cncfnc(\rds) \hgtdlt \\
\label{eqn:tau_opt_ext}
% mie: tau_ext
\tauext(\rds,\wvl) & = \mpi \rds^{2} \fshext(\rds,\wvl)
\cncfnc(\rds) \hgtdlt
\end{align}
\end{subequations} 

\subsubsection[Single Scattering Albedo]{Single Scattering Albedo}\label{sxn:ssa_opt}
The single scattering albedo is simply the probability that, given an
interaction between the photon and particle, the particle will be
scattered rather than absorbed.
For a single particle size, this probability may easily be expressed
in terms of the optical efficiencies (\ref{eqn:fsh_opt}), 
optical cross-sections (\ref{eqn:xsx_opt}), or
optical depths (\ref{eqn:tau_opt})
\begin{eqnarray}
% mie: ss_alb
\ssa(\rds,\wvl) & = & \fshsct(\rds,\wvl) / \fshext(\rds,\wvl)
\nonumber \\
& = & \xsxsct(\rds,\wvl) / \xsxext(\rds,\wvl) \nonumber \\
& = & \tausct(\rds,\wvl) / \tauext(\rds,\wvl) \nonumber \\
& = & \frac{ 1 - \tauabs(\rds,\wvl)}{\tauext(\rds,\wvl) }
\label{eqn:ssa_opt}
\end{eqnarray}
In \cmdidx{mie}, the band-mean version of $\ssa$ is named
\cmdidx{ss\_alb\_fsh}.  

\subsubsection[Asymmetry Parameter]{Asymmetry Parameter}\label{sxn:asm_prm_opt}
In \cmdidx{mie}, the band-mean version of $\asmprm$ is named
\cmdidx{asm\_prm\_fsh}.  

\subsubsection[Mass Absorption Coefficient]{Mass Absorption Coefficient}\label{sxn:mac_opt}
It is often a reasonable approximation to neglect the effects of
particulate scattering of radiation for wavelengths longer than about
5\,\um, i.e., in the longwave spectral region. 
This approximation to longwave radiative transfer means that many
longwave band models require only one parameter to account for the
absorption (and emission) of radiation by particles, the \trmdfn{mass 
absorption coefficient} $\abscffmss$.

\subsection{Effective Single Scattering Properties}\label{sxn:ffc}
In nature, particles are not \trmdfn{monodisperse} but rather appear
continuous size distributions.
Therefore the single scattering properties of each particle size must
be appropriately weighted and combined into the net or
\trmdfn{effective single scattering properties} of the entire size
distribution.
These effective properties are the optical properties of an infinitely
narrow spectral region.   

The size distribution, $\dstnbr(\rds,\hgt)$\,[\nbrxmCm], describes
the rate of change of particle concentration with particle size,
and is a function of position, which we denote by $\hgt$ for height.
\begin{equation}
\dstnbrofrds = \frac{\dfr\cncofrds}{\dfr\rds}
\label{eqn:dst_dfn}
\end{equation}
The total particle concentration is the integral of the number size 
distribution 
\begin{equation}
\cncttl = \int_{0}^{\infty} \dstnbrofrds \,\dfr\rds
\label{eqn:cnc_ttl_dfn}
\end{equation}
$\cncttl$ has dimensions of [\nbrxmC], particle number per unit air volume.

The total cross-sectional area $\xsattl$ is the integral of the
cross-sectional area weighted by the size distribution 
\begin{equation}
\xsattl = \int_{0}^{\infty} \mpi \rds^{2} \dstnbrofrds \,\dfr\rds
\label{eqn:xsa_ttl_dfn}
\end{equation}
$\xsattl$ has dimensions of [\mSxmC], particle area per unit air
volume.  
Since Maxwell's equations (\ref{eqn:mxw_eqn_SI}) are linear, the
solutions to the equations, i.e., the optical efficiencies, are linear
and additive.  
The appropriate weight for each property is the particle number
distribution and a factor which depends on the particular property. 

For the rest of this section we assume that the single scattering
properties of the aerosol are known or can be obtained.
Section~\ref{sxn:mdl} describes a computer program which computes 
these properties for arbitrary size distributions of spherical
aerosols. 

% fxm: There is basically an arbitrary factor of N_0 floating around these definitions
% Pick a definition and stick to it!
\subsubsection[Effective Efficiencies]{Effective Efficiencies}\label{sxn:fsh_ffc}
\begin{subequations}
\label{eqn:fsh_ffc}
\begin{align}
\label{eqn:fsh_abs_ffc}
% mie: abs_fsh_ffc
\fshabsffc(\hgt,\wvl) & = \frac{1}{\xsattl} \int_{0}^{\infty}
\mpi \rds^{2} \fshabs(\rds,\wvl) \dstnbr(\rds,\hgt) \,\dfr\rds \\
\label{eqn:fsh_sct_ffc}
% mie: sca_fsh_ffc
\fshsctffc(\hgt,\wvl) & = \frac{1}{\xsattl} \int_{0}^{\infty}
\mpi \rds^{2} \fshsct(\rds,\wvl) \dstnbr(\rds,\hgt) \,\dfr\rds \\
\label{eqn:fsh_ext_ffc}
% mie: ext_fsh_ffc
\fshextffc(\hgt,\wvl) & = \frac{1}{\xsattl} \int_{0}^{\infty}
\mpi \rds^{2} \fshext(\rds,\wvl) \dstnbr(\rds,\hgt) \,\dfr\rds
\end{align}
\end{subequations}
The effective efficiencies, like the fundamental optical efficiencies
(\ref{eqn:fsh_ffc}), are dimensionless.

\subsubsection[Effective Cross Sections]{Effective Cross Sections}\label{sxn:xsx_ffc}
The effective cross sections are the the fundamental optical cross 
sections (\ref{eqn:xsx_opt}) integrated over the size distribution.
\begin{subequations}
\label{eqn:xsx_ffc}
\begin{align}
\label{eqn:xsx_abs_ffc}
\xsxabsffc(\hgt,\wvl) & = \xsattl \fshabsffc =
\int_{0}^{\infty}
\mpi \rds^{2} \fshabs(\rds,\wvl) \dstnbr(\rds,\hgt) \,\dfr\rds \\
\label{eqn:xsx_sct_ffc}
\xsxsctffc(\hgt,\wvl) & = \xsattl \fshsctffc =
\int_{0}^{\infty}
\mpi \rds^{2} \fshsct(\rds,\wvl) \dstnbr(\rds,\hgt) \,\dfr\rds \\
\label{eqn:xsx_ext_ffc}
\xsxextffc(\hgt,\wvl) & = \xsattl \fshextffc =
\int_{0}^{\infty}
\mpi \rds^{2} \fshext(\rds,\wvl) \dstnbr(\rds,\hgt) \,\dfr\rds
\end{align}
\end{subequations}
The dimensions of the effective cross-sections are [\mSxmC], particle
area per unit air volume. 
Hence, they are also known as the volume absorption, scattering, and 
extinction coefficients.
These units are usually expressed as~[\xm]---we feel that [\mSxmC] is
much clearer.

\subsubsection[Effective Specific Extinction Coefficients]{Effective Specific Extinction Coefficients}\label{sxn:mac_ffc}
As mentioned earlier, \trmidx{specific extinction} is the extinction
per unit mass of particle.
Thus the specific extinction coefficients are the effective specific
cross-sections (Section~\ref{sxn:xsx_ffc}) divided by the total mass
of particles $\mssttl$.
For completeness, we list the explicit definition here:
\begin{subequations}
\label{eqn:spc_ffc}
\begin{align}
\label{eqn:abs_spc_ffc}
\absspcffc(\hgt,\wvl) & = \frac{\xsxabsffc}{\mssttl} = 
\frac{\xsattl \fshabsffc}{\mssttl} =
\frac{\int_{0}^{\infty}
\mpi \rds^{2} \fshabs(\rds,\wvl) \dstnbr(\rds,\hgt) \,\dfr\rds}
{\int_{0}^{\infty}
\frac{4\mpi}{3} \dns \rds^{3} \dstnbr(\rds,\hgt) \,\dfr\rds} \\
\label{eqn:sct_spc_ffc}
\sctspcffc(\hgt,\wvl) & = \frac{\xsxsctffc}{\mssttl} = 
\frac{\xsattl \fshsctffc}{\mssttl} =
\frac{\int_{0}^{\infty}
\mpi \rds^{2} \fshsct(\rds,\wvl) \dstnbr(\rds,\hgt) \,\dfr\rds}
{\int_{0}^{\infty}
\frac{4\mpi}{3} \dns \rds^{3} \dstnbr(\rds,\hgt) \,\dfr\rds} \\
\label{eqn:ext_spc_ffc}
\extspcffc(\hgt,\wvl) & = \frac{\xsxextffc}{\mssttl} = 
\frac{\xsattl \fshextffc}{\mssttl} =
\frac{\int_{0}^{\infty}
\mpi \rds^{2} \fshext(\rds,\wvl) \dstnbr(\rds,\hgt) \,\dfr\rds}
{\int_{0}^{\infty}
\frac{4\mpi}{3} \dns \rds^{3} \dstnbr(\rds,\hgt) \,\dfr\rds}
\end{align}
\end{subequations}

\subsubsection[Effective Optical Depths]{Effective Optical Depths}\label{sxn:tau_ffc}
\begin{subequations}
\label{eqn:tau_ffc}
\begin{align}
\label{eqn:tau_abs_ffc}
\tauabsffc(\hgt) & = \xsxabsffc \hgtdlt = \absspcffc \mpl =
\hgtdlt \int_{0}^{\infty}
\mpi \rds^{2} \fshabs(\rds,\wvl) \dstnbr(\rds,\hgt) \,\dfr\rds \\
\label{eqn:tau_sct_ffc}
\tausctffc(\hgt) & = \xsxsctffc \hgtdlt = \sctspcffc \mpl =
\hgtdlt \int_{0}^{\infty}
\mpi \rds^{2} \fshsct(\rds,\wvl) \dstnbr(\rds,\hgt) \,\dfr\rds \\
\label{eqn:tau_ext_ffc}
\tauextffc(\hgt) & = \xsxextffc \hgtdlt = \extspcffc \mpl =
\hgtdlt \int_{0}^{\infty}
\mpi \rds^{2} \fshext(\rds,\wvl) \dstnbr(\rds,\hgt) \,\dfr\rds
\end{align}
\end{subequations}

\subsubsection[Effective Single Scattering Albedo]{Effective Single Scattering Albedo}\label{sxn:ssa_ffc}
\begin{equation}
\ssaffc(\hgt,\wvl) = \xsxsctffc(\hgt,\wvl)/\xsxextffc(\hgt,\wvl) =
\left( \int_{0}^{\infty}
\mpi \rds^{2} \fshsct(\rds,\wvl) \dstnbr(\rds,\hgt) \,\dfr\rds \right)
\bigg/ \xsxextffc(\hgt,\wvl)
\label{eqn:ssa_ffc}
\end{equation}

\subsubsection[Effective Asymmetry Parameter]{Effective Asymmetry Parameter}\label{sxn:asm_prm_ffc}
The \trmdfn{effective asymmetry parameter} $\asmprmffc$ 
\begin{equation}
\asmprmffc(\hgt,\wvl) = \frac{\int_{0}^{\infty}
\mpi \rds^{2} \asmprm(\rds,\wvl) \fshsct(\rds,\wvl) \dstnbr(\rds,\hgt) \,
\dfr\rds }{
\int_{0}^{\infty} \mpi \rds^{2} \fshsct(\rds,\wvl) \dstnbr(\rds,\hgt) \,
\dfr\rds } 
\label{eqn:asm_prm_ffc}
\end{equation}

\subsection{Mean Effective Single Scattering Properties}\label{sxn:ffc_avg}
Atmospheric radiative transfer models (with the possible exception of
\trmidx{line-by-line models}) work by discretizing the spectral region
of interest into a reasonable number of finite width spectral bands.
These bands can be, and usually are, much wider that the spectral
structure of the absorption and scattering features they contain.
For example, many \trmidx{GCMs} divide the solar spectrum into about
twenty bands, and the infrared spectrum into about
ten\footnote{\trmidx{CAM} uses nineteen solar bands and six-to-eight
  infrared bands.
The Malkmus narrow band model \trmidx{SWNB2} covers the solar spectrum
with 1690 bands.}. 
The appropriate optical properties for such finite bands are called the
\trmidx{mean effective single scattering properties}.

Naturally, we construct the mean effective properties as a weighted
mean of the \trmidx{effective single scattering properties}
(Section~\ref{sxn:ffc}).   
For the single scattering properities of atmospheric size
distributions, the appropriate spectral weight is the specific flux or
radiance. 
For solar or infrared radiative transfer, weight by the fractional
solar or infrared flux, respectively.
These procedures are sometimes called \trmidx{Rayleigh weighting} or
\trmidx{Planck weighting}, respectively
If possible, use the ambient temperature for determining infrared
weights. 

Other potential spectral weighting factors depend on the application.
For example, the single scatter properties of the surface particle
size distribution determine the \trmidx{surface reflectance}.
The spectral surface reflectance is a more optimal weight than the
incident specific flux for determining the band average reflectance of
snow and ice surfaces.

We denote the spectral weighting function by $\flxwgtfrcofwvl$.
Spectral flux weighting in Earth's atmosphere may be done by setting
\begin{eqnarray}
\flxwgtfrcofwvl & = & \left\{
\begin{array}{r@{\quad:\quad}ll}
\flxslrwvlofwvl & \wvl \lesssim 5\,\um \\
\plkwvlofwvltpt & \wvl \gtrsim 5\,\um
\end{array} \right.
\label{eqn:flx_wgt_frc_dfn}
\end{eqnarray}
where $\flxslrwvlofwvl$ is the solar spectral irradiance and
$\plkwvlofwvltpt$ is the specific blackbody radiance
(\ref{eqn:plk_wvl_dfn}). 
For brevity, we omit the explicit dependence of $\flxwgtfrcofwvl$ on
any factor (e.g., altitude, atmospheric composition) except wavelength.  
The mean effective single scattering properties must be normalized
by the integral of the spectral weighting function over the region of
interest, $\flxwgtfrcnot$. 
\begin{equation}
\flxwgtfrcnot \equiv \int_{\wvlmin}^{\wvlmax} \flxwgtfrcofwvl \,\dfr\wvl
\label{eqn:flx_wgt_frc_ttl_dfn}
\end{equation}
Unless the spectral weighting function is normalized, i.e., integrates
to unity over the region of interest, the computation of
(\ref{eqn:flx_wgt_frc_ttl_dfn}) should be done once, outside the
spectral integration loop.

\subsubsection[Mean Effective Efficiencies]{Mean Effective Efficiencies}\label{sxn:fsh_ffc_avg}
The \trmidx{mean effective efficiencies} are the flux-weighted
opacities of a size distribution of particles.
\begin{subequations}
\label{eqn:fsh_ffc_avg}
\begin{align}
\label{eqn:fsh_abs_ffc_avg}
\fshabsffcavg(\hgt) & = \frac{1}{\xsattl \flxwgtfrcnot} 
\int_{\wvlmin}^{\wvlmax} \int_{0}^{\infty}
\mpi \rds^{2} \fshabs(\rds,\wvl) \dstnbr(\rds,\hgt) \flxwgtfrcofwvl
\,\dfr\rds \,\dfr\wvl \\
\label{eqn:fsh_sct_ffc_avg}
\fshsctffcavg(\hgt) & = \frac{1}{\xsattl \flxwgtfrcnot} 
\int_{\wvlmin}^{\wvlmax} \int_{0}^{\infty}
\mpi \rds^{2} \fshsct(\rds,\wvl) \dstnbr(\rds,\hgt) \flxwgtfrcofwvl
\,\dfr\rds \,\dfr\wvl \\
\label{eqn:fsh_ext_ffc_avg}
\fshextffcavg(\hgt) & = \frac{1}{\xsattl \flxwgtfrcnot} 
\int_{\wvlmin}^{\wvlmax} \int_{0}^{\infty}
\mpi \rds^{2} \fshext(\rds,\wvl) \dstnbr(\rds,\hgt) \flxwgtfrcofwvl
\,\dfr\rds \,\dfr\wvl
\end{align}
\end{subequations}
The mean effective efficiencies, like the effective efficiencies
(\ref{eqn:fsh_ffc}) and the fundamental optical efficiencies
(\ref{eqn:fsh_opt}), are dimensionless. 
In \cmdidx{mie}, $\fshabsffcavg$, $\fshsctffcavg$, and
$\fshextffcavg$, are named \cmdidx{abs\_fsh\_ffc},
\cmdidx{sct\_fsh\_ffc}, and \cmdidx{ext\_fsh\_ffc}, respectively. 

\subsubsection[Mean Effective Cross Sections]{Mean Effective Cross Sections}\label{sxn:xsx_ffc_avg}
\begin{subequations}
\label{eqn:xsx_ffc_avg}
\begin{align}
\label{eqn:xsx_abs_ffc_avg}
\xsxabsffcavg(\hgt) & = \xsattl \fshabsffcavg =
\frac{1}{\flxwgtfrcnot} \int_{\wvlmin}^{\wvlmax} \int_{0}^{\infty}
\mpi \rds^{2} \fshabs(\rds,\wvl) \dstnbr(\rds,\hgt) \flxwgtfrcofwvl
\,\dfr\rds \,\dfr\wvl \\
\label{eqn:xsx_sct_ffc_avg}
\xsxsctffcavg(\hgt) & = \xsattl \fshsctffcavg =
\frac{1}{\flxwgtfrcnot} \int_{\wvlmin}^{\wvlmax} \int_{0}^{\infty}
\mpi \rds^{2} \fshsct(\rds,\wvl) \dstnbr(\rds,\hgt) \flxwgtfrcofwvl
\,\dfr\rds \,\dfr\wvl \\
\label{eqn:xsx_ext_ffc_avg}
\xsxextffcavg(\hgt) & = \xsattl \fshextffcavg =
\frac{1}{\flxwgtfrcnot} \int_{\wvlmin}^{\wvlmax} \int_{0}^{\infty}
\mpi \rds^{2} \fshext(\rds,\wvl) \dstnbr(\rds,\hgt) \flxwgtfrcofwvl
\,\dfr\rds \,\dfr\wvl
\end{align}
\end{subequations}
The mean effective cross sections, like the effective cross section
(\ref{eqn:xsx_ffc}), and the fundamental optical cross sections
(\ref{eqn:xsx_opt}), have dimensions of area.
In \cmdidx{mie}, $\xsxabsffcavg$, $\xsxsctffcavg$, and
$\xsxextffcavg$, are named \cmdidx{abs\_cff\_vlm},
\cmdidx{sca\_cff\_vlm}, and \cmdidx{ext\_cff\_vlm}, respectively. 

\subsubsection[Mean Effective Specific Extinction Coefficients]{Mean Effective Specific Extinction Coefficients}\label{sxn:mac_ffc_avg}
\begin{subequations}
\label{eqn:spc_ffc_avg}
\begin{align}
\label{eqn:abs_spc_ffc_avg}
\absspcffcavg(\hgt) & = \frac{\xsxabsffcavg}{\mssttl} = 
\frac{\xsattl \fshabsffcavg}{\mssttl} =
\frac{\int_{\wvlmin}^{\wvlmax} \int_{0}^{\infty}
\mpi \rds^{2} \fshabs(\rds,\wvl) \dstnbr(\rds,\hgt) \flxwgtfrcofwvl \,\dfr\rds \,\dfr\wvl}
{\int_{\wvlmin}^{\wvlmax} \flxwgtfrcofwvl \,\dfr\wvl \int_{0}^{\infty}
\frac{4\mpi}{3} \dns \rds^{3} \dstnbr(\rds,\hgt) \,\dfr\rds} \\
\label{eqn:sct_spc_ffc_avg}
\sctspcffcavg(\hgt) & = \frac{\xsxsctffcavg}{\mssttl} = 
\frac{\xsattl \fshsctffcavg}{\mssttl} =
\frac{\int_{\wvlmin}^{\wvlmax} \int_{0}^{\infty}
\mpi \rds^{2} \fshsct(\rds,\wvl) \dstnbr(\rds,\hgt) \flxwgtfrcofwvl \,\dfr\rds \,\dfr\wvl}
{\int_{\wvlmin}^{\wvlmax} \flxwgtfrcofwvl \,\dfr\wvl \int_{0}^{\infty}
\frac{4\mpi}{3} \dns \rds^{3} \dstnbr(\rds,\hgt) \,\dfr\rds} \\
\label{eqn:ext_spc_ffc_avg}
\extspcffcavg(\hgt) & = \frac{\xsxextffcavg}{\mssttl} = 
\frac{\xsattl \fshextffcavg}{\mssttl} =
\frac{\int_{\wvlmin}^{\wvlmax} \int_{0}^{\infty}
\mpi \rds^{2} \fshext(\rds,\wvl) \dstnbr(\rds,\hgt) \flxwgtfrcofwvl \,\dfr\rds \,\dfr\wvl}
{\int_{\wvlmin}^{\wvlmax} \flxwgtfrcofwvl \,\dfr\wvl \int_{0}^{\infty}
\frac{4\mpi}{3} \dns \rds^{3} \dstnbr(\rds,\hgt) \,\dfr\rds}
\end{align}
\end{subequations}
The mean effective specific extinction coefficients, like the
effective specific extinction coefficients (\ref{eqn:spc_ffc}) have
dimensions of area per unit mass.
In \cmdidx{mie}, $\absspcffcavg$, $\sctspcffcavg$, and
$\extspcffcavg$ are named \cmdidx{abs\_cff\_mss},
\cmdidx{sca\_cff\_mss}, and \cmdidx{ext\_cff\_mss}, respectively. 

\subsubsection[Mean Effective Optical Depths]{Mean Effective Optical Depths}\label{sxn:tau_ffc_avg}
\begin{subequations}
\label{eqn:tau_ffc_avg}
\begin{align}
\label{eqn:tau_abs_ffc_avg}
\tauabsffcavg & = \xsxabsffcavg \hgtdlt = \absspcffcavg \mpl =
\frac{\hgtdlt}{\flxwgtfrcnot} \int_{\wvlmin}^{\wvlmax} \int_{0}^{\infty}
\mpi \rds^{2} \fshabs(\rds,\wvl) \dstnbr(\rds,\hgt) \flxwgtfrcofwvl \,\dfr\rds \,\dfr\wvl \\
\label{eqn:tau_sct_ffc_avg}
\tausctffcavg & = \xsxsctffcavg \hgtdlt = \sctspcffcavg \mpl =
\frac{\hgtdlt}{\flxwgtfrcnot} \int_{\wvlmin}^{\wvlmax} \int_{0}^{\infty}
\mpi \rds^{2} \fshsct(\rds,\wvl) \dstnbr(\rds,\hgt) \flxwgtfrcofwvl \,\dfr\rds \,\dfr\wvl \\
\label{eqn:tau_ext_ffc_avg}
\tauextffcavg & = \xsxextffcavg \hgtdlt = \extspcffcavg \mpl =
\frac{\hgtdlt}{\flxwgtfrcnot} \int_{\wvlmin}^{\wvlmax} \int_{0}^{\infty}
\mpi \rds^{2} \fshext(\rds,\wvl) \dstnbr(\rds,\hgt) \flxwgtfrcofwvl \,\dfr\rds \,\dfr\wvl
\end{align}
\end{subequations}

\subsubsection[Mean Effective Single Scattering Albedo]{Mean Effective Single Scattering Albedo}\label{sxn:ssa_ffc_avg}
\begin{equation}
\ssaffcavg(\hgt) = \xsxsctffcavg(\hgt)/\xsxextffcavg(\hgt) =
\frac{1}{\flxwgtfrcnot \xsxextffcavg(\hgt)} \int_{\wvlmin}^{\wvlmax} \int_{0}^{\infty}
\mpi \rds^{2} \fshsct(\rds,\wvl) \dstnbr(\rds,\hgt) \flxwgtfrcofwvl \,\dfr\rds \,\dfr\wvl
\label{eqn:ssa_ffc_avg}
\end{equation}
In \cmdidx{mie}, $\ssaffcavg$ is named \cmdidx{ss\_alb}.

\subsubsection[Mean Effective Asymmetry Parameter]{Mean Effective Asymmetry Parameter}\label{sxn:asm_prm_ffc_avg}
The \trmdfn{mean effective asymmetry parameter} $\asmprmffcavg$ 
\begin{equation}
\asmprmffcavg(\hgt) = \frac{
\int_{\wvlmin}^{\wvlmax} \int_{0}^{\infty}
\mpi \rds^{2} \asmprm(\rds,\wvl) \fshsct(\rds,\wvl) \dstnbr(\rds,\hgt) \flxwgtfrcofwvl
\,\dfr\rds \,\dfr\wvl}
{
\int_{\wvlmin}^{\wvlmax} \int_{0}^{\infty}
\mpi \rds^{2} \fshsct(\rds,\wvl) \dstnbr(\rds,\hgt) \flxwgtfrcofwvl
\,\dfr\rds \,\dfr\wvl}
\label{eqn:asm_prm_ffc}
\end{equation}
In \cmdidx{mie}, $\asmprmffcavg$ is named \cmdidx{asm\_prm}.

\subsection{Bulk Layer Single Scattering Properties}\label{sxn:blk}
The real atmosphere is characterized by multiple species coexisting
and interacting with the radiation field. 
The radiation field is determined by the combined optical properties
of all the radiatively active constituents.
Knowing only the individual radiative properties of the constituents
is not helpful as the radiation field is not in any sense linear,
i.e., the additive result of the radiation fields produced by each
constituent individually.
Instead the individual radiative properties of all constituents,
gases, particles, and boundary surfaces must be combined into what we
shall call the \trmdfn{bulk single scattering properties} or
\trmdfn{layer single scattering properties}.
Combination of optical properties proceeds as in the previous section.

\subsubsection{Addition of Optical Properties}\label{sxn:opt_add}
Let $\spcnbr$ denote the number of radiatively active species in a
volume.
If we assume that these species do not interact with eachother then
the scattering and absorbing properties of the medium are additive.
This may be called the \trmidx{independent scatterers} assumption.
The combination rules that result from this assumption were described
by \cite{Ces85}.

The \trmidx{independent scatterers} assumption has many critics.
\cite{Mel08} believes that enough interactions occur on the scale of
interstitial aerosol and gases in clouds, that the independent
scatterers assumption significantly underestimates the effective
optical depth (and thus absorption) of these media.
\cite{KMW02} may also question the assumption.

\subsubsection[Bulk Optical Depths]{Bulk Optical Depths}\label{sxn:tau_blk}
Optical depths add linearly.
With reference to (\ref{eqn:tau_ffc}), 
\begin{subequations}
\label{eqn:tau_blk}
\begin{align}
\label{eqn:tau_abs_blk}
\tauabs & = \sum_{\spcidx = 1}^{\spcnbr} \tauabsffcspc \\
\label{eqn:tau_sct_blk}
\tausct & = \sum_{\spcidx = 1}^{\spcnbr} \tausctffcspc \\
\label{eqn:tau_ext_blk}
\tauext & = \sum_{\spcidx = 1}^{\spcnbr} \tauextffcspc
\end{align}
\end{subequations}

\subsubsection[Bulk Single Scattering Albedo]{Bulk Single Scattering Albedo}\label{sxn:ssa_blk}
With reference to (\ref{eqn:ssa_ffc}), (\ref{eqn:tau_ffc}) and
(\ref{eqn:tau_ext_blk}), 
\begin{equation}
\ssa = \frac{1}{\tauext} \sum_{\spcidx = 1}^{\spcnbr} \ssaffcspc \tauextffcspc
\label{eqn:ssa_blk}
\end{equation}

\subsubsection[Bulk Asymmetry Parameter]{Bulk Asymmetry Parameter}\label{sxn:asm_prm_blk}
With reference to (\ref{eqn:asm_prm_ffc}), (\ref{eqn:ssa_ffc}),
(\ref{eqn:tau_ffc}), (\ref{eqn:tau_ext_blk}), and (\ref{eqn:ssa_blk}),
\begin{equation}
\asmprm = \frac{1}{\ssa \tauext} \sum_{\spcidx = 1}^{\spcnbr} \asmprmffcspc \ssaffcspc \tauextffcspc
\label{eqn:asm_prm_blk}
\end{equation}

\subsubsection[Diagnostics]{Diagnostics}\label{sxn:dgn_blk}
When the radiation field is known, it is possible to diagnose the
contribution of individual elements to bulk properties such as
heating.  
For instance, the contribution of given-size particles to the net
radiative heating of a layer is the difference between the absorption
and the emission of the particles.
\begin{equation}
q_{\rm R}(\rds,\hgt) = 4\mpi \int_{0}^{\infty} \mpi \rds^{2} 
\fshabs(\rds,\wvl)
[\ntnmnfrq(\hgt,\wvl) - \plkfrq(\tpt_{\rm a},\wvl)] \dfr\wvl
\label{eqn:qrad_dfn}
\end{equation}

\csznote{
\begin{equation}
q_{\rm R,SW}(\rds,\hgt) = 
\sum_{\wvlidx = 1}^{\wvlidx = 18} \left(
\frac{\partial F_{\rm SW}}{\partial \hgt } \right)_{\wvlidx,k} \mpi \rds^{2} 
\fshabs(\rds,\wvl_{\wvlidx}) \bigg/ 
\left[ 
\int_{0}^{\infty} \mpi \rds^{2} 
\fshabs(\rds,\wvl_{\wvlidx}) \dstnbr(\rds,\hgt) \,\dfr\rds
\right] 
\label{eqn:qrad_dfn2}
\end{equation}
} % end csznote

\section[Global Radiative Forcing]{Global Radiative Forcing}\label{sxn:rad_frc}
Figure~\ref{fgr:FSNT} shows the balance between incident, reflected,
and absorbed solar radiation observed by the ERBE satellite system
from 1985--1989.  
\begin{figure*}
\centering
\includegraphics[height=0.3\vsize]{/Users/zender/data/fgr/rt/erbe_b_8589_SOLIN}\vfill
\includegraphics[height=0.3\vsize]{/Users/zender/data/fgr/rt/erbe_b_8589_FSUT}\vfill
\includegraphics[height=0.3\vsize]{/Users/zender/data/fgr/rt/erbe_b_8589_FSNT}\vfill
\caption[Climatological Mean Absorbed Solar Radiation]{
Geographic distribution of 1985--1989 climatological mean (a)
insolation $\FSDT$, (b) reflected shortwave irradiance $\FSUT$, and
(c) absorbed shortwave radiation $\FSNT$~[\wxmS] from ERBE observations.
\label{fgr:FSNT}}
\end{figure*}

Figure~\ref{fgr:FLNT} shows the balance between terrestrial radiation
emitted by the surface, trapped by the atmosphere, and escaping to
space as observed by the ERBE satellite system from 1985--1989.  
\begin{figure*}
\centering
\includegraphics[height=0.3\vsize]{/Users/zender/data/fgr/rt/erbe_b_8589_FLUS}\vfill
\includegraphics[height=0.3\vsize]{/Users/zender/data/fgr/rt/erbe_b_8589_GCLD}\vfill
\includegraphics[height=0.3\vsize]{/Users/zender/data/fgr/rt/erbe_b_8589_FLNT}\vfill
\caption[Climatological Mean Emitted Longwave Radiation]{
Geographic distribution of 1985--1989 climatological mean terrestrial
radiation (a) emitted by the surface $\FLUS$, (b) trapped by the
atmosphere $\GCLD$, and (c) escaping to space $\FLNT$ in
[\wxmS] from ERBE observations. 
\label{fgr:FLNT}}
\end{figure*}

Figure~\ref{fgr:ENSO_tpt} shows the temperature and OLR response to ENSO
in terms of Hovm\"{o}ller diagrams.
\begin{figure*}
\centering
\includegraphics[width=0.5\hsize,height=0.9\vsize]{/Users/zender/data/fgr/rt/sld012d_yavg_10S10N_8589_0160_TS1}%
\includegraphics[width=0.5\hsize,height=0.9\vsize]{/Users/zender/data/fgr/rt/erbe_b_yavg_10S10N_8589_0160_FLNT}
\caption[ENSO Temperature and OLR]{
Hovm\"{o}ller diagrams of (a) sea surface temperature [\K] and (b)
outgoing longwave radiation [\wxmS] over the
Equatorial Pacific (averaged 10\,\dgrs--10\,\dgrn).
Month~1 is January 1985.  
Contour intervals are 0.5\,K and 10\,\wxmS, respectively.
\label{fgr:ENSO_tpt}}
\end{figure*}

Figure~\ref{fgr:SWCF} shows the shortwave cloud forcing ($\SWCF$)
observed by the ERBE satellite system from 1985--1989.
\begin{figure*}
\centering
\includegraphics[width=0.8\hsize]{/Users/zender/data/fgr/rt/erbe_b_8589_01_SWCF}\vfill
\includegraphics[width=0.8\hsize]{/Users/zender/data/fgr/rt/erbe_b_8589_07_SWCF}%
\caption[Seasonal Shortwave Cloud Forcing]{
Geographic distribution of shortwave cloud forcing $\SWCF$ [\wxmS] 
for 1985--1989 from ERBE observations for (a) January and (b) July.
\label{fgr:SWCF}}
\end{figure*}

Figure~\ref{fgr:SWCF_x} shows the zonal mean shortwave cloud forcing
($\SWCF$) observed by the ERBE satellite system from 1985--1989.
\begin{figure*}
\centering
\includegraphics*[width=0.8\hsize]{/Users/zender/data/fgr/rt/erbe_b_sld012d_8589_01_x_SWCF}\vfill
\includegraphics*[width=0.8\hsize]{/Users/zender/data/fgr/rt/erbe_b_sld012d_8589_07_x_SWCF}%
\caption[Zonal Mean Shortwave Cloud Forcing]{
Zonal mean shortwave cloud forcing $\SWCF$ [\wxmS] 
for 1985--1989 from ERBE observations and from CCM simulations for (a)
January and (b) July. 
\label{fgr:SWCF_x}}
\end{figure*}

Figure~\ref{fgr:LWCF} shows the longwave cloud forcing ($\LWCF$)
observed by the ERBE satellite system from 1985--1989.
\begin{figure*}
\centering
\includegraphics[width=0.8\hsize]{/Users/zender/data/fgr/rt/erbe_b_8589_01_LWCF}\vfill
\includegraphics[width=0.8\hsize]{/Users/zender/data/fgr/rt/erbe_b_8589_07_LWCF}%
\caption[Seasonal Longwave Cloud Forcing]{
Geographic distribution of shortwave cloud forcing $\LWCF$ [\wxmS] 
for 1985--1989 from ERBE observations for (a) January and (b) July.
\label{fgr:LWCF}}
\end{figure*}

Figure~\ref{fgr:LWCF_x} shows the zonal mean longwave cloud forcing
($\LWCF$) observed by the ERBE satellite system from 1985--1989.
\begin{figure*}
\centering
\includegraphics*[width=0.8\hsize]{/Users/zender/data/fgr/rt/erbe_b_sld012d_8589_01_x_LWCF}\vfill
\includegraphics*[width=0.8\hsize]{/Users/zender/data/fgr/rt/erbe_b_sld012d_8589_07_x_LWCF}%
\caption[Zonal Mean Longwave Cloud Forcing]{
Zonal mean shortwave cloud forcing $\LWCF$ [\wxmS] 
for 1985--1989 from ERBE observations and from CCM simulations for (a)
January and (b) July. 
\label{fgr:LWCF_x}}
\end{figure*}

Figure~\ref{fgr:NCF} shows the balance between solar and terrestrial
cloud forcing as observed by the ERBE satellite system from
1985--1989.   
\begin{figure*}
\centering
\includegraphics[height=0.3\vsize]{/Users/zender/data/fgr/rt/erbe_b_8589_SWCF}\vfill
\includegraphics[height=0.3\vsize]{/Users/zender/data/fgr/rt/erbe_b_8589_LWCF}\vfill
\includegraphics[height=0.3\vsize]{/Users/zender/data/fgr/rt/erbe_b_8589_NCF}\vfill
\caption[Climatological Mean Net Cloud Forcing]{
Geographic distribution of 1985--1989 climatological mean 
(a) shortwave cloud forcing $\SWCF$, (b) longwave cloud forcing
$\LWCF$, and (c) net cloud forcing $\NCF$~[\wxmS] from ERBE.
\label{fgr:NCF}}
\end{figure*}

Figure~\ref{fgr:ENSO_crf} shows the interannual variability of the cloud
forcing over the Pacific in terms of Hovm\"{o}ller diagrams of cloud
forcing. 
\begin{figure*}
\centering
\includegraphics[width=0.5\hsize,height=0.9\vsize]{/Users/zender/data/fgr/rt/erbe_b_yavg_10S10N_8589_0160_LWCF}%
\includegraphics[width=0.5\hsize,height=0.9\vsize]{/Users/zender/data/fgr/rt/erbe_b_yavg_10S10N_8589_0160_SWCF}
\caption[ENSO Cloud Forcing]{
Hovm\"{o}ller diagrams of cloud forcing [\wxmS] in the 
Equatorial Pacific (averaged 10\,\dgrs--10\,\dgrn).
ERBE observations of (a) shortwave cloud forcing $\SWCF$, and 
(b) longwave cloud forcing.
Month~1 is January 1985.  
Contour interval is 10\,\wxmS. 
\label{fgr:ENSO_crf}}
\end{figure*}

\section[Implementation in NCAR models]{Implementation in NCAR models}\label{sxn:mdl}

The discussion thus far has centered on the theoretical considerations
of radiative transfer.
In practice, these ideas must be implemented in computer codes which
model, e.g., shortwave or longwave atmospheric fluxes and heating
rates.
This section describes how these ideas have been implemented in the 
NCAR narrow band models.

\clearpage
\section{Appendix}

\subsection[Vector Identities]{Vector Identities}\label{sxn:vct_idn}
A number of \trmdfn{vector identities} are useful in the derivation 
of Mie theory.
For arbitrary vectors $\aaabld$, $\bbbbld$, and~$\cccbld$; 
arbitrary scalars $\alpha$ and~$\beta$; 
and scalar fields $\sclfld = \sclfld(\xxx,\yyy,\zzz)$ 
and $\vvv = \vvv(\xxx,\yyy,\zzz)$, we have:
\begin{subequations}
% Back inside cover of Jac75
\label{eqn:vct_idn}
\begin{align}
\label{eqn:vct_idn_grd_lnr}\mbox{Grad-linearity:}\qquad
\nabla (\alpha \sclfld + \beta \vvv) & = \alpha \nabla \sclfld + \beta \nabla \vvv \\
\label{eqn:vct_idn_dvr_lnr}\mbox{Div-linearity:}\qquad
\nabla \cdot (\alpha \aaabld + \beta \bbbbld) & = \alpha \nabla \cdot \aaabld + \beta \nabla \cdot \bbbbld \\
\label{eqn:vct_idn_crl_lnr}\mbox{Curl-linearity:}\qquad
\nabla \cross (\alpha \aaabld + \beta \bbbbld) & = \alpha \nabla \cross \aaabld + \beta \nabla \cross \bbbbld \\
\label{eqn:vct_idn_dot_cmm}\mbox{Dot-commutativity:}\qquad
\aaabld \cdot \bbbbld & = \bbbbld \cdot \aaabld \\
\label{eqn:vct_idn_crs_cmm}\mbox{Cross-(non)-commutativity:}\qquad
\aaabld \cross \bbbbld & = - \bbbbld \cross \aaabld \\
\label{eqn:vct_idn_trp_prd}\mbox{Scalar Triple Product:}\qquad
\aaabld \cdot \bbbbld \cross \cccbld & = 
\cccbld \cdot \aaabld \cross \bbbbld = 
\bbbbld \cdot \cccbld \cross \aaabld \\
\label{eqn:vct_idn_crs_crs}\mbox{Cross-cross:}\qquad
\aaabld \cross (\bbbbld \cross \cccbld) & = 
\bbbbld (\aaabld \cdot \cccbld) - 
\cccbld (\aaabld \cdot \bbbbld) \\
(\aaabld \cross \bbbbld) \cross \cccbld & = 
\bbbbld (\aaabld \cdot \cccbld) - 
\aaabld (\bbbbld \cdot \cccbld) \\
\label{eqn:vct_idn_crl_crs}\mbox{Curl-cross:}\qquad
\nabla \cross (\aaabld \cross \bbbbld) & = 
\aaabld (\nabla \cdot \bbbbld) - \bbbbld (\nabla \cdot \aaabld) + 
(\bbbbld \cdot \nabla) \aaabld - (\aaabld \cdot \nabla) \bbbbld \\
\label{eqn:vct_idn_crl_crl}\mbox{Curl-curl:}\qquad 
\nabla \cross (\nabla \cross \aaabld) 
& = \nabla (\nabla \cdot \aaabld) - \nabla \cdot (\nabla \aaabld) \nonumber \\
& = \nabla (\nabla \cdot \aaabld) - \nabla^{2} \aaabld \\
\label{eqn:vct_idn_lpl_vct}\mbox{Vector Laplacian:}\qquad 
\nabla^{2} \aaabld & \equiv (\nabla \cdot \nabla ) \aaabld \equiv \nabla \cdot (\nabla \aaabld) \\
& = \nabla (\nabla \cdot \aaabld) - \nabla \cross (\nabla \cross \aaabld) \\
\label{eqn:vct_idn_grd_dvr}\mbox{Grad-div:}\qquad
\nabla ( \nabla \cdot \aaabld ) & = \nabla^{2} \aaabld + \nabla \cross (\nabla \cross \aaabld) \\
& \ne \mbox{(not equal)} \nabla^{2} \aaabld \nonumber \\
\label{eqn:vct_idn_grd_dot}\mbox{Grad-dot:}\qquad
\nabla (\aaabld \cdot \bbbbld) & = % BoH83 p. 83
\aaabld \cross (\nabla \cross \bbbbld) + \bbbbld \cross (\nabla \cross \aaabld) + 
(\bbbbld \cdot \nabla) \aaabld + (\aaabld \cdot \nabla) \bbbbld \nonumber \\
& =  % Inside back cover of Jac75
( \aaabld \cdot \nabla ) \bbbbld + ( \bbbbld \cdot \nabla ) \aaabld +
\aaabld \cross ( \nabla \cross \bbbbld ) + \bbbbld \cross ( \nabla \cross \aaabld ) \\
\label{eqn:vct_idn_del_vct}\mbox{Del-vector:}\qquad
\nabla \aaabld & \equiv \nabla_{\!\iii\iii} \aaa_{\iii} \xhatidx = \mbox{Dyadic} \\
\label{eqn:vct_idn_dvr_grd}\mbox{Div-grad:}\qquad
\nabla \cdot ( \nabla \sclfld ) & \equiv \nabla^{2} \sclfld \\
\label{eqn:vct_idn_dvr_del}\mbox{Div-del:}\qquad
(\nabla \cdot \nabla ) \aaabld & \equiv \nabla^{2} \aaabld \\
\label{eqn:vct_idn_dot_crs}\mbox{Dot-cross:}\qquad
\aaabld \cdot (\bbbbld \cross \cccbld) & = 
\bbbbld \cdot (\cccbld \cross \aaabld) = 
\cccbld \cdot (\aaabld \cross \bbbbld) \\
\label{eqn:vct_idn_dvr_crl}\mbox{Div-curl:}\qquad
\nabla \cdot \nabla \cross \aaabld & = 0 \\
\label{eqn:vct_idn_crl_grd}\mbox{Curl-grad:}\qquad
\nabla \cross \nabla \sclfld & = 0 \\
\label{eqn:vct_idn_dvr_prd}\mbox{Div-product:}\qquad
\nabla \cdot (\sclfld \aaabld) & = 
\sclfld \nabla \cdot \aaabld + \aaabld \cdot \nabla \sclfld \\
\label{eqn:vct_idn_crl_prd}\mbox{Curl-product:}\qquad
\nabla \cross (\sclfld \aaabld) & = 
\nabla \sclfld \cross \aaabld + \sclfld \nabla \cross \aaabld \\
\label{eqn:vct_idn_dot_del}\mbox{Dot-del:}\qquad
(\aaabld \cdot \nabla) \aaabld & = 
\frac{1}{2} \nabla |\aaabld|^{2} + (\nabla \cross \aaabld) \cross \aaabld
\end{align}
\end{subequations} 
The divergence of the curl vanishes (\ref{eqn:vct_idn_dvr_crl}). 
The curl of the gradient vanishes (\ref{eqn:vct_idn_crl_grd}). 

The Laplacian $\nabla^{2}$ is shorthand for $(\nabla \cdot \nabla)$, 
and is usually read as ``del-squared'', though it is not the
``square'' of the del operator.
The scalar Laplacian is the divergence of the gradient
(\ref{eqn:vct_idn_dvr_grd}).
However, the vector Laplacian is \textit{not} the gradient of the
divergence (\ref{eqn:vct_idn_grd_dvr}). 
The vector Laplacian $\nabla^{2}$ is defined by (\ref{eqn:vct_idn_lpl_crl}) 
which is obtained by simply re-arranging (\ref{eqn:vct_idn_crl_crl}). 
The vector Laplacian (\ref{eqn:vct_idn_lpl_crl}) is the gradient of
the divergence (\ref{eqn:vct_idn_grd_dvr}) minus the curl of the curl 
(\ref{eqn:vct_idn_crl_crl}).
\clearpage

\subsection{Legendre Polynomials}\label{sxn:lgn}
The \trmidx{associated Legendre equation} 
of degree~$\plridx$ and order~$\aziidx$ is
\begin{eqnarray}
\frac{\dfr}{\dfr\plrmu}
\left[ (1 - \plrmu^{2}) \frac{\dfr\plrfnc}{\dfr\plrmu} \right] +
\left[ \plridx(\plridx+1) - \frac{\aziidx^{2}}{1-\plrmu^{2}} \right] \plrfnc
& = & 0
\label{eqn:lgn_ass_eqn}
\end{eqnarray}
The solutions to (\ref{eqn:lgn_ass_eqn}) are associated Legendre
polynomials, also called \trmidx{spherical harmonics}.
The associated Legendre polynomial of degree~$\plridx$,
order~$\aziidx$, and argument~$\plrmu$ is denoted by 
$\lgnassplrazi(\plrmu)$.
The $\lgnassplrazi(\plrmu)$ are \trmidx{orthogonal functions}
\begin{eqnarray}
\int_{-1}^{1} \lgnassplrazi(\plrmu) 
\lgnasstwoarg{\plridx^{\prime}}{\aziidx}(\plrmu)
\,\dfr\plrmu
& = & 
\dltsubtwoarg{\plridx}{\plridx^{\prime}} 
\frac{2}{2\plridx+1} \frac{(\plridx+\aziidx)!}{(\plridx-\aziidx)!}
\label{eqn:lgn_ass_rth_dfn}
\end{eqnarray}
When $\aziidx = 0$, $\lgnassplrazi(\plrmu)$ reduces to the 
\trmidx{Legendre polynomial} of order~$\plridx$,
$\lgnplr(\plrmu)$ whose orthogonality properties follow from
(\ref{eqn:lgn_ass_rth_dfn}) and are shown in
(\ref{eqn:lgn_fnc_nrm_plr}). 
Applying suitable normalization to $\lgnassplrazi(\plrmu)$ yields
the normalized associated Legendre polynomials
$\lgnassnrmplrazi(\plrmu)$ 
\begin{eqnarray}
\lgnassnrmplrazi(\plrmu) & \equiv & 
\sqrt{\frac{2\plridx+1}{2}
\frac{(\plridx-\aziidx)!}{(\plridx+\aziidx)!}}
\lgnassplrazi(\plrmu)
\label{eqn:lgn_ass_nrm_dfn}
\end{eqnarray}
The $\lgnassnrmplrazi(\plrmu)$ are \trmidx{orthonormal} and are the
polar components of the \trmidx{spherical harmonics}
$\sphhrmthrarg{\plridx}{\aziidx}{\plr,\azi}$ (\ref{eqn:sph_hrm_dfn}).

\subsection{Spherical Harmonics}\label{sxn:sph_hrm}
Spherical harmonics are the product of the normalized associated
Legendre polynomials (\ref{eqn:lgn_ass_nrm_dfn}) and the orthonormal 
exponentials (\ref{eqn:exp_nrm_dfn})
\begin{eqnarray}
\azifncidx(\azi) & = & \frac{1}{\sqrt{2\mpi}}\me^{\mi\aziidx\azi}
\label{eqn:exp_nrm_dfn}
\end{eqnarray}
The \trmidx{spherical harmonics}
$\sphhrmthrarg{\plridx}{\aziidx}{\plr,\azi}$ of degree $\plridx$ and
order $\aziidx$ are
\begin{eqnarray}
% Arf85 p. 682 (12.147)
\sphhrmthrarg{\plridx}{\aziidx}{\plr,\azi} & \equiv & 
(-1)^{\aziidx}\sqrt{\frac{2\plridx+1}{4\mpi} 
\frac{(\plridx-\aziidx)!}{(\plridx+\aziidx)!}}
\lgnassplrazi(\cos\plr) \me^{\mi\aziidx\azi}
\label{eqn:sph_hrm_dfn}
\end{eqnarray}
The \trmidx{Condon-Shortley} phase factor $(-1)^{\aziidx}$ is a
convention which causes the sign alternation for the positive
$\aziidx$ spherical harmonics \cite[][p.~682]{Arf85}.

The spherical harmonics (\ref{eqn:sph_hrm_dfn}) are normalized such that
\begin{eqnarray}
\int_{\azi=0}^{\azi=2\mpi} \int_{\plr=0}^{\plr=\mpi} 
\sphhrmthrarg{\plridx}{\aziidx\cmpcnj}{\plr,\azi} 
\sphhrmthrarg{\plridx^{\prime}}{\aziidx^{\prime}}{\plr,\azi} 
& \equiv & 
\dltsubtwoarg{\plridx}{\plridx^{\prime}}
\dltsubtwoarg{\aziidx}{\aziidx^{\prime}}
\label{eqn:sph_hrm_dfn}
\end{eqnarray}
where $\cmpcnj$ denotes the \trmidx{complex conjugate}.

\subsection{Bessel Functions}\label{sxn:bsl}
The definitive reference for properties of Bessel functions is
\cite{Wat58}. 
Other useful references are \cite{AbS64}, p.~374 and \cite{Arf85},
p.~573.   
When the \trmidx{Helmholtz equation} is solved by separation of
variables in cylindrical or spherical coordinates
(\ref{eqn:bsl_wvnrds_sph}), the radial component satisfies
\trmidx{Bessel's equation} \cite[][p.~19]{Wat58},
\begin{eqnarray}
\bslargcpx^{2} \frac{\dfr^{2}\www}{\dfr\bslargcpx^{2}} +
\bslargcpx \frac{\dfr\www}{\dfr\bslargcpx} +
[ \bslargcpx^{2} - \bslrdrcpx^{2} ] \www
& = & 0
\label{eqn:bsl_eqn_dfn}
\end{eqnarray}

\trmdfn{Bessel functions} are solutions to~(\ref{eqn:bsl_eqn_dfn}).
The order of the functions is indicated by the parameter $\bslrdrcpx$
(not to be confused with frequency). 
The functions take a single argument~$\zzz$.
Both $\bslrdrcpx$ and $\zzz$ may be complex, e.g., 
$\zzz = \xxx + \mi\yyy$, although our particular applications depend
only on real-valued $\bslrdrcpx$.
Indeed, we require only integer and half-integer $\bslrdrcpx$.

Due to their power and range of applicability, a menagerie of
inter-related solutions to (\ref{eqn:bsl_eqn_dfn}) exist.
Fortunately, the terminology to identify and describe them is
standard.
The standard Bessel functions of the first kind, second kind,
and third kind, are denoted by $\bslJcpx(\bslargcpx)$, 
$\bslYcpx(\bslargcpx)$, and 
$\bslHonecpx(\bslargcpx)$ and $\bslHtwocpx(\bslargcpx)$, respectively.   
Bessel functions of the second kind are also called \trmidx{Neumann}
functions and notated with is $\bslNcpx(\bslargcpx)$ instead of 
$\bslYcpx(\bslargcpx)$.
Mathematical physics texts often use the Neumann function notation
$\bslNcpx(\bslargcpx)$ to avoid confusion with spherical harmonics;
whereas pure mathematical texts normally use the
$\bslYcpx(\bslargcpx)$ notation.
Bessel functions of the third kind are also called 
\trmidx{Hankel functions}.
All of these satisfy~(\ref{eqn:bsl_eqn_dfn}) and so are properly 
called Bessel functions.
Specialized versions of Bessel functions derive from the three basic
kinds. 
These specializations include modified Bessel functions, and spherical
Bessel functions.

General solutions to the Bessel equation (\ref{eqn:bsl_eqn_dfn}) are
formed from any two linearly independent Bessel functions.
For non-integer $\bslrdrcpx$,
$\bslJcpx(\bslargcpx)$ and 
$\bslthrarg{\bslJfnc}{-\bslrdrcpx}{\bslargcpx}$ 
are linearly independent solutions and so form a complete basis.
However, for integral $\bslrdrcpx$
\begin{eqnarray}
% Arf85 p. 576 (11.8) AbS64 p. 358 (9.1.5)
\bslthrarg{\bslJfnc}{-\bslrdrntg}{\bslargcpx} 
 & = & (-1)^{\bslrdrntg} \bslJntg(\bslargcpx) \quad {\mbox{Integer}\ \bslrdrntg}
\label{eqn:bsl_sym_dfn}
\end{eqnarray}
and these solutions are not linearly independent.

\trmdfn{Bessel functions of the second kind} $\bslYcpx(\bslargcpx)$
are always linearly independent of $\bslJcpx(\bslargcpx)$.
They may be defined as
\begin{eqnarray}
% Arf85 p. 596 (11.60) AbS64 p. 358 (9.1.2)
\bslYcpx(\bslargcpx) & = & 
\bslNcpx(\bslargcpx) =  
\frac{\bslJcpx(\bslargcpx)\cos(\bslrdrcpx\mpi)-\bslthrarg{\bslJfnc}{-\bslrdrcpx}{\bslargcpx}}{\sin\bslrdrcpx\mpi}
\label{eqn:bsl_Y_dfn}
\end{eqnarray}
where the RHS limiting value used for integral $\bslrdrcpx$. 
Bessel functions of the first and second kind, $\bslJcpx(\bslargcpx)$ 
and $\bslYcpx(\bslargcpx)$ together form a complete basis.
Series expansion (not shown) reveals that $\bslYntg(\bslargcpx)$
diverges logarithmically as $\bslargcpx \to 0$.
Boundary conditions which require finite solutions at the origin 
($\bslargcpx = 0$) therefore automatically exclude
$\bslYntg(\bslargcpx)$. 
Conversely, $\bslYntg(\bslargcpx)$ may appear in any solution
which does not require finite values at the origin.

\trmdfn{Bessel functions of the third kind} (Hankel functions),
$\bslHonecpx(\bslargcpx)$ and $\bslHtwocpx(\bslargcpx)$, are another
set of linearly independent solutions to Bessel's equation.
The relation between these complete bases is
\begin{subequations}
\begin{align}
\bslHonecpx(\bslargcpx) & = \bslJcpx(\bslargcpx) + \mi\bslYcpx(\bslargcpx) \\
\bslHtwocpx(\bslargcpx) & = \bslJcpx(\bslargcpx) - \mi\bslYcpx(\bslargcpx)
\label{eqn:bsl_hnk_dfn}
\end{align}
\end{subequations}

\subsubsection[Spherical Bessel Functions]{Spherical Bessel Functions}\label{sxn:bsl_sph}
The \trmidx{spherical Bessel functions} of the first and second kind
are 
\begin{subequations}
\begin{align}
% BoH83 p. 86, Arf85 p. 623
\bsljcpx(\bslargcpx) & =
\sqrt{\frac{\mpi}{2\bslargcpx}}\bslJfnc_{\bslrdrcpx+\sfrac{1}{2}}(\bslargcpx) \\
\bslycpx(\bslargcpx) & =
\sqrt{\frac{\mpi}{2\bslargcpx}}\bslYfnc_{\bslrdrcpx+\sfrac{1}{2}}(\bslargcpx) 
\label{eqn:bsl_sph_dfn}
\end{align}
\end{subequations}

In analogy to (\ref{eqn:bsl_hnk_dfn}) the spherical Bessel functions
of the third kind, or \trmidx{spherical Hankel functions} are
\begin{subequations}
\begin{align}
\bslhonecpx(\bslargcpx) & = \bsljcpx(\bslargcpx) + \mi\bslycpx(\bslargcpx) \\
\bslhtwocpx(\bslargcpx) & = \bsljcpx(\bslargcpx) - \mi\bslycpx(\bslargcpx)
\label{eqn:bsl_hnk_sph_dfn}
\end{align}
\end{subequations}

\subsubsection[Recurrence Relations]{Recurrence Relations}\label{sxn:bsl_rcr}
Bessel functions (and thus spherical Bessel functions) satisfy the
recurrence relations 
\begin{subequations}
\begin{align}
% AbS64 p. 361 (9.1.27)
\bslthrarg{\bslZfnc}{\bslrdrcpx-1}{\bslargcpx} +
\bslthrarg{\bslZfnc}{\bslrdrcpx+1}{\bslargcpx} & = 
\frac{2\bslrdrcpx}{\bslargcpx}
\bslthrarg{\bslZfnc}{\bslrdrcpx}{\bslargcpx} \\
\bslthrarg{\bslZfnc}{\bslrdrcpx-1}{\bslargcpx} -
\bslthrarg{\bslZfnc}{\bslrdrcpx+1}{\bslargcpx} & = 
2\bslqtrarg{\bslZfnc}{\bslrdrcpx}{\prime}{\bslargcpx} \\
\bslqtrarg{\bslZfnc}{\bslrdrcpx}{\prime}{\bslargcpx} & = 
\bslthrarg{\bslZfnc}{\bslrdrcpx-1}{\bslargcpx} -
\frac{\bslrdrcpx}{\bslargcpx}
\bslthrarg{\bslZfnc}{\bslrdrcpx}{\bslargcpx} \\
\bslqtrarg{\bslZfnc}{\bslrdrcpx}{\prime}{\bslargcpx} & = 
-\bslthrarg{\bslZfnc}{\bslrdrcpx+1}{\bslargcpx} +
\frac{\bslrdrcpx}{\bslargcpx}
\bslthrarg{\bslZfnc}{\bslrdrcpx}{\bslargcpx}
\label{eqn:bsl_rcr_dfn}
\end{align}
\end{subequations}

Seed values for recurrence relations (\ref{eqn:bsl_rcr_dfn}) may
be obtained from  
\begin{subequations}
\begin{align}
% AbS64 p. 361 (9.1.28)
\bslqtrarg{\bslJfnc}{0}{\prime}{\bslargcpx} & = 
-\bslthrarg{\bslJfnc}{1}{\bslargcpx} \\
\bslqtrarg{\bslYfnc}{0}{\prime}{\bslargcpx} & = 
-\bslthrarg{\bslYfnc}{1}{\bslargcpx}
\label{eqn:bsl_rcr_ntl_dfn}
\end{align}
\end{subequations}

\subsubsection[Power Series Representation]{Power Series Representation}\label{sxn:bsl_pwr}
The \trmidx{power series} for regular Bessel functions of the first
kind is 
\begin{eqnarray}
% AbS64 p. 360 (9.1.10)
\bslJcpxofz & = & 
({\tfrac{1}{2}}\bslargcpx)^{\bslrdrcpx} \sum_{\kkk=0}^{\infty}
\frac{(-\frac{1}{4}\bslargcpx^{2})^{\kkk}}{\kkk!\, 
\gmmfnc(\bslrdrcpx + \kkk + 1)}
\label{eqn:bslJ_srs_dfn}
\end{eqnarray}
\trmidx[Bessel functions, modified]{Modified Bessel functions} of the 
first kind are defined in terms ordinary Bessel functions as
\begin{eqnarray}
% Lio92 p. 51 (2.4.2a)
\bslIcpxofz & = & \mi^{-\bslrdrcpx} \bslJcpx(\mi\bslargcpx)
\label{eqn:bslI_dfn}
\end{eqnarray}
The power series for the modified Bessel function of the first kind,
$\bslIcpxofz$ differs from (\ref{eqn:bslJ_srs_dfn}) due to the absence
of the negative sign
\begin{eqnarray}
% AbS64 p. 375 (9.6.10), GrR65 p. 961 (8.445)
\bslIcpxofz & = & 
({\textstyle\frac{1}{2}}\bslargcpx)^{\bslrdrcpx} \sum_{\kkk=0}^{\infty}
\frac{(\frac{1}{4}\bslargcpx^{2})^{\kkk}}{\kkk!\, \gmmfnc(\bslrdrcpx + \kkk + 1)}
\label{eqn:bslI_srs_dfn}
\end{eqnarray}
Of particular interest to radiative transfer are $\bslIfnc_{0}(\bslargcpx)$ 
and  $\bslIfnc_{1}(\bslargcpx)$ which appear in the solution to the
mean beam absorptance (\ref{eqn:lr_dfn}) and equivalent width of an
isolated Lorentz line.
\begin{subequations}
% GrR65 p. 961 (8.447)
\label{eqn:bslI01_dfn}
\begin{align}
\label{eqn:bslI0_dfn}
\bslIfnc_{0}(\bslargcpx) & = 
\sum_{\kkk=0}^{\infty}
\frac{({\textstyle\frac{1}{2}}\bslargcpx)^{2\kkk}}{(\kkk!)^{2}} \\
\label{eqn:bslI1_dfn}
\bslIfnc_{1}(\bslargcpx) & = 
\sum_{\kkk=0}^{\infty}
\frac{({\textstyle\frac{1}{2}}\bslargcpx)^{2\kkk+1}}{\kkk!\,(\kkk+1)!}
\end{align}
\end{subequations}
For $\bslargcpx \ll 1$, (\ref{eqn:bslI01_dfn}) becomes
\begin{subequations}
% GrR65 p. 961 (8.447)
\label{eqn:bslI01_lmt_dfn}
\begin{align}
\bslIfnc_{0}(\bslargcpx \ll 1) & \approx 1 \\
\bslIfnc_{1}(\bslargcpx \ll 1) & \approx \bslargcpx/2
\end{align}
\end{subequations}

\subsubsection[Asymptotic Values]{Asymptotic Values}\label{sxn:bsl_asm}
The asymptotic behavior of functions can often be ascertained from
their power series representation.
Bessel functions obey the following limits as $\bslargcpx \to 0$:
\begin{subequations}
\begin{align}
% AbS64 p. 360 (9.1.7)
\bslthrarg{\bslZfnc}{\bslrdrcpx}{\bslargcpx} & \to 
(\tfrac{1}{2}\bslargcpx)^{\bslrdrcpx}/\gmmfnc(\bslrdrcpx+1) 
\quad (\bslrdrcpx \neq -1, -2, -3 \ldots) \\
% AbS64 p. 360 (9.1.8)
\bslthrarg{\bslYfnc}{0}{\bslargcpx} & \to (2/\mpi)\ln\bslargcpx \\
% AbS64 p. 360 (9.1.9)
\bslthrarg{\bslYfnc}{\bslrdrcpx}{\bslargcpx} & \to 
\frac{\gmmfnc(\bslrdrcpx)}{\mpi(\tfrac{1}{2}\bslargcpx)^{\bslrdrcpx}}
\label{eqn:bsl_asm_zro_dfn}
\end{align}
\end{subequations}

\subsection{Gaussian Quadrature}\label{sxn:gss}
% Good derivation in Arf85 App. 2, p. 969
The accuracy of automatic integration of analytic formulae by
computers rests on three criteria: 
\begin{enumerate*}
\item Smoothness of function
\item Number of quadrature points
\item Placement of quadrature points
\end{enumerate*}
For many problems the functional form of the integrand is given.
We are usually free to change, however, the number and placement of
quadrature points.
For many applications, the optimal number and location of quadrature
points are related through \trmdfn{Gaussian quadrature}.

Gaussian quadrature points $\xxx_{\gssidx}$ have the property that
\begin{equation}
\int_{\aaa}^{\bbb} \fff(\xxx) \www(\xxx) \,\dfr\xxx \approx
\sum_{\gssidx = 1}^{\nnn} \AAA_{\gssidx} \fff(\xxx_{\gssidx})  
\label{eqn:gss_gnr_dfn}
\end{equation}
is \textit{exact} if $\fff(\xxx)$ is any polynomial of degree 
$\le 2\nnn - 1$. 
The values of the $\xxx_{\gssidx}$ are determined by the form of the weighting
function $w(\xxx)$ and the interval boundaries $[\aaa,\bbb]$.
In fact the $\xxx_{\gssidx}$ are the roots of the orthogonal polynomial with
weighting function $\www(\xxx)$ over the interval $[\aaa,\bbb]$ \cite[see,
e.g.,][p. 969]{Arf85}. 
A common case in radiative transfer has 
$\www(\xxx) = 1$, $[\aaa,\bbb] = [-1,1]$  
\begin{equation}
\int_{-1}^{1} \fff(\xxx) \,\dfr\xxx \approx \sum_{\gssidx =
1}^{\nnn} \AAA_{\gssidx} \fff(\xxx_{\gssidx})  
\label{eqn:gss_dfn}
\end{equation}
For this case, the $\xxx_{\gssidx}$ are the $\plridx$ roots of the
\trmidx{Legendre polynomial} of degree $\plridx$, $\lgnplr(\xxx)$,  
\footnote{When $\www(\xxx) \ne 1$, the Gaussian quadrature points are
  the roots of other orthogonal polynomials, e.g., Chebyshev,
  Laguerre, Hermite.} and the \trmidx{Gaussian weights} $\AAA_{\gssidx}$
are defined by the derivative of $\lgnplr(\xxx)$
\begin{eqnarray}
\xxx_{\gssidx} = \xxx \in [0,1]\ |\ \lgnplr(\xxx_{\gssidx}) = 0 \\
\AAA_{\gssidx} = \frac{2}{(1 - \xxx_{\gssidx}^{2})[\lgnplr^{\prime}(\xxx_{\gssidx})]^{2}}
\label{eqn:wgt_dfn}
\end{eqnarray}

Radiative transfer problems often use Gaussian weights in the angular
integration. 
Depending on whether we are concerned with net or hemispherical
quantities the angular integration will have different intervals
\begin{eqnarray}
\label{eqn:gss_fll_dfn}
\int_{-1}^{1} \fff(\xxx) \www(\xxx) \,\dfr\xxx \approx \sum_{\gssidx =
1}^{\plridx} \AAA_{\gssidx} \fff(\xxx_{\gssidx}) \\ 
\label{eqn:gss_hlf_dfn}
\int_{0}^{1} \fff(\yyy) \www(\yyy) \,\dfr\yyy \approx \sum_{\gssidx =
1}^{\plridx} \BBB_{\gssidx} \fff(\yyy_{\gssidx})  
\end{eqnarray}
Equation~(\ref{eqn:gss_fll_dfn}) is known as \trmdfn{full-range
Gaussian quadrature} and (\ref{eqn:gss_hlf_dfn}) as \trmdfn{half-range
Gaussian quadrature}. 

The half-range points and weights are intimately related to the
full-range points and weights.
The interval $\yyy \in [0,1]$ is a simple linear transformation
of $\xxx \in [-1,1]$
\begin{eqnarray}
\label{eqn:map_fwd}
\yyy = \frac{\xxx + 1}{2} \\
\label{eqn:map_rvr}
\xxx = 2\yyy - 1 
\end{eqnarray}
The half-range points and weights may be obtained by applying the
forward transformation (\ref{eqn:map_fwd}) to
(\ref{eqn:gss_fll_dfn}). 
Thus
\begin{equation}
\yyy_{\gssidx} = \frac{\xxx_{\gssidx} + 1}{2}
\label{eqn:pnt_map}
\end{equation}

The weights $\AAA_{\gssidx}$ depend only on $\plridx$ (fxm: I think) and are
independent of $\fff(\xxx)$ and $\www(\xxx)$. 
For Legendre polynomials, the $\AAA_{\gssidx}$ and $\BBB_{\gssidx}$ sum to the size of
the interval
\begin{eqnarray}
\label{eqn:fll_wgt_dfn}
\sum_{\gssidx = 1}^{\plridx} \AAA_{\gssidx} = 2 \\
\label{eqn:hlf_wgt_dfn}
\sum_{\gssidx = 1}^{\plridx} \BBB_{\gssidx} = 1
\end{eqnarray}

A common technique is to apply half of the full range quadrature 
points and weights to half of the full range interval, $[0,1]$.
In this case the quadrature is no longer Gaussian (i.e., not exact for 
polynomials of degree $< 2\plridx$), but the quadrature points are
often quite optimal.

Table~\ref{tbl:gss1} lists the first five Legendre polynomials, their 
roots, the associated full Gaussian angles, and their Gaussian
weights.  
\begin{table}
\begin{minipage}{\hsize} % Minipage necessary for footnotes KoD95 p. 110 (4.10.4)
\renewcommand{\footnoterule}{\rule{\hsize}{0.0cm}\vspace{-0.0cm}} % KoD95 p. 111
\begin{center}
\caption[Full-range Gaussian quadrature]{\textbf{Full-range Gaussian
quadrature}% 
\footnote{The $\plrmu_{\gssidx}$ and $\AAA_{\gssidx}$ are only supplied for $\gssidx
\le \frac{\plridx+2}{2}$.
The $\plrmu_{\gssidx}$ are antisymmetric, and the $\AAA_{\gssidx}$ are symmetric,
about $\gssidx = \plridx/2$.
Thus $\plrmu_{\plridx-\gssidx} = -\plrmu_{\gssidx}$ and $\AAA_{\plridx-\gssidx} =
\AAA_{\gssidx}$.}% 
\label{tbl:gss1}}
\vspace{\cpthdrhlnskp}
\begin{tabular}{rlllll}
\hline \rule{0.0ex}{\hlntblhdrskp}% 
$\plridx,\gssidx$ & Polynomial & Root & Decimal & Angle & Weight \\[0.0ex]
& $\lgnplr(\plrmu)$ & $\plrmu_{\gssidx}$ & $\plrmu_{\gssidx}$ & $\plr_{\gssidx}$ & $\AAA_{\gssidx}$ \\[0.0ex]
\hline \rule{0.0ex}{\hlntblntrskp}%
$1,1$ & $\plrmu$ & $0$ & $0.0$ & $90.0\dgr$ & $2.0$ \\[1.0ex]
$2,1$ & $\frac{1}{2}(3 \plrmu^{2} - 1)$ & ${1/\sqrt{3}}$ & $0.577350$ & $54.7356\dgr$ & $1.0$ \\[1.0ex]
$3,1$ & $\frac{1}{2}(5 \plrmu^{3} - 3\plrmu)$ & $0$ & $0.0$ & $90.0\dgr$ & $0.888889$ \\[1.0ex]
$3,2$ & & $\sqrt{\frac{3}{5}}$ & $0.774597$ & $39.2321\dgr$ & $0.555556$ \\[1.0ex]
$4,1$ & $\frac{1}{8}(35\plrmu^{4} - 30\plrmu^{2}+3)$ & 
$\left( \frac{15 - 2\sqrt{30}}{35} \right)^{1/2}$ & 
$0.339981$ & $70.1243\dgr$ & $0.652145$ \\[1.0ex]
$4,2$ & & 
$\left( \frac{15 + 2\sqrt{30}}{35} \right)^{1/2}$ & 
$0.861136$ & $30.5556\dgr$ & $0.347855$ \\[1.0ex]
\hline
\end{tabular}
\end{center}
\end{minipage}
\end{table}
\cite{AbS64}, p.~917, list Gaussian quadrature points and weights for
$\plridx \le 32$.

Table~\ref{tbl:gss2} summarizes the same information needed for
half-range Gaussian quadrature.
\begin{table}
\begin{minipage}{\hsize} % Minipage necessary for footnotes KoD95 p. 110 (4.10.4)
\renewcommand{\footnoterule}{\rule{\hsize}{0.0cm}\vspace{-0.0cm}} % KoD95 p. 111
\begin{center}
\caption[Half-range Gaussian quadrature]{\textbf{Half-range Gaussian
quadrature}% 
\footnote{The symmetries differ from Table~\ref{tbl:gss1}.
The polynomials, roots and weights need to be transformed from their
full-range counterparts.}%
\label{tbl:gss2}}   
\vspace{\cpthdrhlnskp}
\begin{tabular}{rlllll}
\hline \rule{0.0ex}{\hlntblhdrskp}% 
$\plridx,\gssidx$ & Polynomial & Root & Decimal & Angle & Weight \\[0.0ex]
& $\lgnplr(\plrmu)$ & $\plrmu_{\gssidx}$ & $\plrmu_{\gssidx}$ &
$\plr_{\gssidx}$ & $\AAA_{\gssidx} $ \\[0.0ex]
\hline \rule{0.0ex}{\hlntblntrskp}%
$1,1$ & $\plrmu$ & $0$ & $0.0$ & $90.0\dgr$ & 2.0 \\[1.0ex]
$2,1$ & $\frac{1}{2}(3\plrmu^{2}-1)$ & ${+1/\sqrt{3}}$ & $+0.57735$ & $+54.7356\dgr$ & 1.0 \\[1.0ex]
$3,1$ & $\frac{1}{2}(5\plrmu^{3} - 3\plrmu)$ & & & & \\[1.0ex]
$4,1$ & $\frac{1}{8}(35\plrmu^{4} - 30\plrmu^{2}+3)$ & & & & \\[1.0ex]
\hline
\end{tabular}
\end{center}
\end{minipage}
\end{table}

\subsection{Gauss-Lobatto Quadrature}\label{sxn:qdrlbb}
\trmidx{Lobatto quadrature}, is a type of 
\trmidx{Gaussian quadrature} that prescribes abscissae at the
integration endpoints. 
The Lobatto quadrature rule of order $\lbbnbr$ is
\begin{eqnarray}
% Mec63 p. 237 (1)
\label{eqn:qdr_lbb_dfn}
\int_{-1}^{+1} \fnc(\xxx) \,\dfr\xxx & = &
\sum_{\lbbidx = 1}^{\lbbidx = \lbbnbr} \wgtlbbidx \fnc(\xxx_{\lbbidx}) \\
& = &
\wgtlbb_{1} \fnc(-1) + 
\sum_{\lbbidx = 1}^{\lbbidx = \lbbnbr} \wgtlbbidx \fnc(\xxx_{\lbbidx}) +
\wgtlbb_{\lbbnbr} \fnc(1)
\end{eqnarray}
Whereas Gaussian quadrature is exact for polynomials of degrees 
$\le 2\lbbnbr-1$, Lobatto quadrature fixes two quadrature points
and so is exact only for polynomials of degrees $\le 2\lbbnbr-3$. 
Nevertheless, the placement of abscissae can give Lobatto quadrature 
a higher effective accuracy for certain classes of functions such
as asymmetric phase functions \citep{Wis771}.

The free (non-prescribed) Lobatto abscissae $\xxx_{\lbbidx}$ for
$\lbbidx = 2,\ldots,\lbbnbr-1$ are the $\lbbnbr-2$ zeros of the
derivative of the Legendre polynomial of degree $\lbbnbr-1$ 
\begin{eqnarray}
% Mec63 p. 237 (2)
\label{eqn:qdr_lbb_abc_dfn}
\lgnfnc_{\lbbnbr-1}^{\prime}(\xxx_{\lbbidx}) & = & 0
\end{eqnarray}
(By comparison, the abscissae in Gauss-Legendre quadrature 
are Legendre polynomial roots).

The interior Lobatto weights $\wgtlbbidx$ satisfy
\begin{eqnarray}
% Mec63 p. 237 (2)
\label{eqn:qdr_lbb_wgt_dfn}
\wgtlbbidx & = & \frac{2}
{\lbbnbr(\lbbnbr-1)[\lgnfnc_{\lbbnbr-1}(\xxx_{\lbbidx})]^{2}}
\end{eqnarray}
while the endpoint weights are
\begin{eqnarray}
% Mec63 p. 237 (2)
\label{eqn:qdr_lbb_wgt_end_dfn}
\wgtlbb_{1} & = & \wgtlbb_{\lbbnbr} =  
\frac{2}{\lbbnbr(\lbbnbr-1)}
\end{eqnarray}

\citet{Mic63} describes numerical approaches to determining Lobatto
quadrature abscissae endpoints and weights.
This procedure converges remarkably quickly and smoothly.
However, the traditional quadrature variable $\plru = \cos(\plr)$
and range $[-1,1]$ produce abscissae that a forward-clustered 
in $\cos(\plr)$ and thus angles that are relatively evenly distributed
(rather than forward-clustered) in~$\plr$.

\citet{Wis771} puts forward a solution to this problem.
He changes the quadrature variable and domain to $\plr$ and
$[0,\mpi]$, respectively.
Implementing this procedure is relatively straightforward.
The Lobatto weights $\wgtlbbidx$ must be normalized on the new 
integration domain
\begin{eqnarray}
% Mec63 p. 237 (2)
\label{eqn:qdr_lbb_mdf_wgt_dfn}
\wgtlbbidx & = & \frac{\mpi}
{\lbbnbr(\lbbnbr-1)[\lgnfnc_{\lbbnbr-1}(\xxx_{\lbbidx})]^{2}}
\end{eqnarray}
while the endpoint weights are
\begin{eqnarray}
% Mec63 p. 237 (2)
\label{eqn:qdr_lbb_mdf_wgt_end_dfn}
\wgtlbb_{1} & = & \wgtlbb_{\lbbnbr} =  
\frac{\mpi}{\lbbnbr(\lbbnbr-1)}
\end{eqnarray}

\subsection{Exponential Integrals}\label{sxn:xpn}
% NB: This section copied from GoY89 p. 475
The \trmdfn{exponential integrals} $\xpnnxxx$ are defined by
\begin{equation}
\xpnnxxx = \int_{1}^{\infty} \ttt^{-n} \me^{-\ttt \xxx} \,\dfr\ttt
\label{eqn:xpn_dfn}
\end{equation}
and by the behavior of $\xpnnxxx$ at $\xxx = 0$
\begin{eqnarray}
\xpn_{1}(0) & = & \infty \nonumber \\
\xpn_n(0) & = & 1/(n-1) \nonumber
\end{eqnarray}
A definition equivalent to (\ref{eqn:xpn_dfn}) is obtained by making
the change of variable $\www = \ttt^{-1}$.
This maps $\ttt \in [1,\infty)$ into $\www \in [1,0]$ 
\begin{eqnarray}
\www & = & \ttt^{-1} \nonumber \\
\ttt & = & \www^{-1} \nonumber \\
\dfr\ttt & = & -\www^{-2} \,\dfr\www \nonumber \\
\dfr\www & = & -\ttt^{-2} \,\dfr\ttt 
\label{eqn:xpn_cov}
\end{eqnarray}
Substituting (\ref{eqn:xpn_cov}) into (\ref{eqn:xpn_dfn}) we obtain
\begin{eqnarray}
\xpnnxxx & = & \int_{1}^{0} (\www^{-1})^{-n} \me^{-\xxx/\www} (-\www^{-2}) \,\dfr\www \nonumber \\
& = & \int_{0}^{1} \www^n \www^{-2} \me^{-\xxx/\www} \,\dfr\www \nonumber \\
& = & \int_{0}^{1} \www^{n-2} \me^{-\xxx/\www} \,\dfr\www
\label{eqn:xpn_dfn_2}
\end{eqnarray}

Exponential integrals satisfy two important recurrence relations
\begin{subequations}
\label{eqn:xpn_rcr}
\begin{align}
\label{eqn:xpn_rcr_1}
-\xpnnprm(\xxx) & = -\xpn_{\nnn-1}(\xxx) \\
\label{eqn:xpn_rcr_2}
\nnn\xpn_{(\nnn+1)}(\xxx) & = \me^{-\xxx} - \xxx \xpnnxxx
\end{align}
\end{subequations}

The angular integral of many radiometric quantities may often be 
expressed in terms of the first few exponential integrals.
\begin{subequations}
\label{eqn:xpn_lst}
\begin{align}
\label{eqn:xpn_n_dfn}
\xpn_{\nnn}(\tau) & = \int_{0}^{1} \plrmu^{\nnn-2} \me^{-\tau/\plrmu} \,\dfr\plrmu \\ 
\label{eqn:xpn_1_dfn}
\xpn_{1}(\tau) & = \int_{0}^{1} \plrmu^{-1}  \me^{-\tau/\plrmu} \,\dfr\plrmu \\ 
\label{eqn:xpn_2_dfn}
\xpn_{2}(\tau) & = \int_{0}^{1}              \me^{-\tau/\plrmu} \,\dfr\plrmu \\
\label{eqn:xpn_3_dfn}
\xpn_{3}(\tau) & = \int_{0}^{1} \plrmu       \me^{-\tau/\plrmu} \,\dfr\plrmu
\end{align}
\end{subequations}

% Bibliography
%\renewcommand\refname{\normalsize Publications}
\bibliographystyle{agu04}
\bibliography{bib}
\clearpage
\label{sxn:idx}
\printindex % Requires makeidx KoD95 p. 221
\addcontentsline{toc}{section}{Index}

\csznote{
% Transfer required figures to local machine
for fl_stb in \
erbe_b_8589_01_SWCF \
erbe_b_8589_07_SWCF \
erbe_b_sld012d_8589_01_x_SWCF \
erbe_b_sld012d_8589_07_x_SWCF \
erbe_b_8589_01_LWCF \
erbe_b_8589_07_LWCF \
erbe_b_sld012d_8589_01_x_LWCF \
erbe_b_sld012d_8589_07_x_LWCF \
erbe_b_8589_SOLIN \
erbe_b_8589_FSUT \
erbe_b_8589_FSNT \
erbe_b_8589_FLUS \
erbe_b_8589_GCLD \
erbe_b_8589_FLNT \
erbe_b_8589_SWCF \
erbe_b_8589_LWCF \
erbe_b_8589_NCF \
erbe_b_yavg_10S10N_8589_0160_FLNT \
erbe_b_yavg_10S10N_8589_0160_LWCF \
erbe_b_yavg_10S10N_8589_0160_SWCF \
j_NO2_arese_19951011 \
j_NO2_rlt_arese_19951011 \
j_NO2_act_flx \
j_NO2_abs_xsx \
j_NO2_qnt_yld \
sld012d_yavg_10S10N_8589_0160_TS1 \
; do 
scp 'dust.ess.uci.edu:${DATA}/ps/'${fl_stb}'.eps' ${DATA}/fgr/rt
epstopdf ${DATA}/fgr/rt/${fl_stb}.eps
done
} % end csznote
% $: re-balance syntax highlighting

\end{document}

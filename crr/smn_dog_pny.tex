% $Id$ -*-LaTeX-*-

% Purpose: Dog and Pony show

% Subject: 

% URL: http://dust.ess.uci.edu/crr/smn_dog_pny.pdf

% Create (see also end of file):
% cd ~/crr;make -W smn_dog_pny.tex smn_dog_pny.pdf;cd -

% Distribute:
% scp ${DATA}/ps/smn_dog_pny.pdf dust.ess.uci.edu:/var/www/html/smn

\documentclass[12pt]{article}

% Standard packages
\usepackage{ifpdf} % Define \ifpdf
\ifpdf % PDFLaTeX
\usepackage[pdftex]{graphicx} % Defines \includegraphics*
\pdfcompresslevel=9
%\usepackage{thumbpdf} % Generate thumbnails
\usepackage{epstopdf} % Convert .eps, if found, to .pdf when required
\graphicspath{/data/zender/fgr/ess_phz} % Help epstopdf find .eps figures to convert
\else % !PDFLaTeX
\usepackage{graphicx} % Defines \includegraphics*
\fi % !PDFLaTeX
\usepackage{array} % Table and array extensions, e.g., column formatting
%\usepackage[usenames]{color} % usenames allows, e.g., ``ForestGreen''
\usepackage{colortbl} % Colored columns in tables, required by pdfscreen
\usepackage{makeidx} % Index keyword processor: \printindex and \see
\usepackage{mdwlist} % Compact list formats \itemize*, \enumerate*
%\usepackage[numbers,sort]{natbib} % \cite commands from aguplus
\usepackage[sort]{natbib} % \cite commands from aguplus
\usepackage{times} % Postscript Times-Roman font KoD99 p. 375
%\usepackage{tocbibind} % Add Bibliography and Index to Table of Contents
\usepackage{xspace} % Unknown

% pdfscreen uses hyperref to set margins for screen mode
% pdfscreen calls hyperref internally if hyperref is not already invoked
% pdfscreen also loads packages graphicx, color, calc, and comment
% print         Print version
% screen        Screen version
% panelright    Navigation panel on RHS
% paneltoc      Table of Contents in panel
% sectionbreak  Introduces pagebreak before section
% code          Commands producing fancy verbatim-like effects
%\usepackage[screen,code,panelright,sectionbreak]{pdfscreen}
%\usepackage[screen,code]{pdfscreen}
\usepackage[screen,code]{pdfscreen}
%\usepackage[print]{pdfscreen}

% Personal packages
\usepackage{csz} % Library of personal definitions (\ifpdf...)
\usepackage{abc} % Alphabet as three letter macros
\usepackage{dmn} % Dimensional units
\usepackage{chm} % Chemistry
\usepackage{dyn} % Fluid dynamics
\usepackage{aer} % Aerosol physics
\usepackage{rt} % Commands specific to radiative transfer
\usepackage{psd} % Particle size distributions

% Commands which must be executed in preamble

% Commands specific to this file
% 1. Primary commands
\newcommand{\pdfscreen}{\texttt{\small\color{section1}pdfscreen}\xspace} % Pretty format for pdfscreen string
\newcommand{\scq}{SC$^{4}$} % [sng] Southern California Climate Change Consortium

% Float placement
% NB: Placement of figures is very sensitive to \textfraction
\renewcommand\textfraction{0.0} % Minimum fraction of page that is text
\setcounter{totalnumber}{73} % Maximum number of floats per page
\setcounter{topnumber}{73} % Maximum number of floats at top of page
\setcounter{dbltopnumber}{73} % Maximum number of floats at top of two-column page
\setcounter{bottomnumber}{73} % Maximum number of floats at bottom of page
\renewcommand\topfraction{1.0} % Maximum fraction of top of page occupied by floats
\renewcommand\dbltopfraction{1.0} % Maximum fraction of top of two-column page occupied by floats
\renewcommand\bottomfraction{1.0} % Maximum fraction of bottom of page occupied by floats
\renewcommand\floatpagefraction{1.0} % Fraction of float page filled with floats
\renewcommand\dblfloatpagefraction{1.0} % Fraction of double column float page filled with floats

% Screen geometry
\begin{screen}
% These commands are all defined by pdfscreen
%\margins{0.65in}{0.65in}{0.65in}{0.65in} % Left, right, top, bottom margins
\margins{0.40in}{0.40in}{0.40in}{0.40in} % Left, right, top, bottom margins
%\emblema{logoLUC.png} % Location of graphic file for pdfscreen navigation panel
\screensize{6.25in}{8in} % Screen dimensions of PDF output
\urlid{http://www.ess.uci.edu/~zender} % Homepage button in navigation panel points here
\changeoverlay % Cycle through default overlays, repeating every tenth section
\paneloverlay{but.pdf} % Overlay for navigation panel
%\overlay{ipcc_ar3_rad_frc} % Overlay for screen area
\def\pfill{\vskip6pt}
\end{screen}

% Commands performed only in print section
\begin{print}
\notesname{Notes:} % String appearing in notes ovals adjacent to slides
% Redefine section formatting in print mode
\makeatletter
\def\@seccntformat#1{\llap{\scshape\color{section\thesection@level}\csname the#1\endcsname.\hspace*{6pt}}}
\makeatother
\end{print}

\hypersetup{ % A command provided by \hyperref
pdftitle={Zender Group Research}, % Title given to acroread window displaying this file
pdfsubject={Zender Group Research},
pdfauthor={Charlie Zender},
pdfkeywords={snw dst aer crr Zender Group Research},
pdfpagemode={FullScreen}, % Starts in full screen mode, hit 'Esc' to escape
pdfmenubar=true % Allow access to reader's menubar
} % end \hypersetup

\begin{document} % End preamble

% Title for screen mode
\begin{screen}
\title{\color{section0}\Huge Zender Group Research}
\end{screen}

% Title for print mode
\begin{print}
\title{\Huge\texttt{Zender Group Research}}
\end{print}

\author{
\color{section1}
\href{http://www.ess.uci.edu/~zender}{Charlie~Zender}
{\href{mailto:zender@uci.edu}{\color{section1}\texttt{<zender@uci.edu>}}}\\ 
\href{http://www.ess.uci.edu}{Department of Earth System Science}\\
\href{http://www.uci.edu}{University of California, Irvine}\\
Presented to:\\
\href{http://eee.uci.edu/05f/42132}{ESS 191: Introduction to Research in Earth System Science}\\
Built: \today\\
\\
Presentation PDF on the Web at:\\
\url{http://dust.ess.uci.edu/smn/smn_dog_pny.pdf}
} % end author

\pagedissolve{Split /D 2 /Dm /H /M /O} % pagedissolve options in pdfscreen manual p. 9

\date{} % Empty braces turns off date
\maketitle
\begin{screen}
\vfill
\end{screen}
\clearpage

\begin{abstract}
\Large
\setlength{\baselineskip}{12.0pt} % 1.234 X 11pt
\noindent Zender Group Research
\end{abstract}
\clearpage

% Do not use \tableofcontents command in document when paneltoc is option
\begin{print}
\tableofcontents
\end{print}
\begin{screen}
\vfill
\end{screen}

\Large
\section[Outline]{Outline}
\begin{enumerate*}
\item Mineral dust aerosol---its distribution, source and sink
  mechanisms, direct and indirect radiative forcing, and impact on
  tropospheric processes (Q.~Han, C.~Luo)

\item Radiation and climate---closing the surface shortwave energy
  budget, the effect of aerosols and clouds on climate (J.~Talamantes) 

\item Surface energy exchange---understanding non-linear energy
  transfer and constituent fluxes between the surface and lower
  atmosphere, and its effect on convection and climate (S.~Capps) 

\item Snow-albedo feedback---representing complex interactions
  between snow-covered surfaces and climate, including snowpack aging,
  reflectivity, absorptance, and melt (M.~Flanner, C.~Luo)

\item Distributed Data Reduction and Analysis---efficient processing
  of TB-scale datasets (e.g., from GCMs) stored within and among high
  performance computing clusters (M.~Brown, D.~Wang, H.~Mangalam)
\end{enumerate*}
\clearpage

\Large
\section[Mineral Dust Aerosol]{Mineral Dust Aerosol}
\begin{enumerate*}
\item Mobilization, Transport, and Deposition
\begin{enumerate*}
\item Where, when, what sizes, and what composition of dust is mobilized?
\item Constraining dust emission using ocean deposition measurements
\end{enumerate*}
\item Erodibility: Susceptibility to wind erosion
\begin{enumerate*}
\item How do soil characteristics, geomorphology, and hydrology affect erodibility?
\item How do Drought and Inundatation affect erodibility?
\end{enumerate*}
\item Health
\begin{enumerate*}
\item Predicting onset of human/ecosystem epidemics due to dust-borne
  disease vectors 
\end{enumerate*}
\end{enumerate*}
\clearpage

\Large
\begin{figure*}
\centering % \centering uses less vertical space than center-environment
% ${HOME}/idl/gcm.pro:gcm_xy_bch,pll=4,img=0,fld_top=1,lbl_typ='auto',prn=0
\includegraphics[width=0.5\hsize]{/data/zender/fgr/ppr_ZBN03/dstmch90_clm_DSTSFMBL}%
\includegraphics[width=0.5\hsize]{/data/zender/fgr/ppr_ZBN03/dstmch90_clm_DSTSFDPS}%

\includegraphics[width=0.5\hsize]{/data/zender/fgr/ppr_ZBN03/dstmch90_clm_DSTSFDRY}%
\includegraphics[width=0.5\hsize]{/data/zender/fgr/ppr_ZBN03/dstmch90_clm_DSTSFPCP}%
\caption{
Predicted annual mean dust source and sink fluxes in \ugxmSs\ for
(a)~mobilization, (b)~total deposition, (c)~dry deposition, 
(d)~wet deposition \cite[][]{ZBN03}.
\label{fgr:sf}}
\end{figure*}
\clearpage

\Large
\begin{figure}
\includegraphics*[height=6.5in,angle=90]{/data/zender/fgr/ppr_ZNT03/lubbock_mbl_bsn_fct}\vfill
\caption{
Soil spatial erodibility constraints $\rdbfct$ for four hypotheses: 
(a)~Uniform, (b)~Topographic, 
(c)~Geomorphic, (d)~Hydrologic \cite[][]{ZNT03}.
\label{fgr:mbl_bsn_fct}}
\end{figure}
\clearpage

\Large
\begin{figure}
\includegraphics[width=\hsize,clip=true,trim=1.45in 7.25in 1.5in 1.0in]{/data/zender/fgr/ppr_ZeK05/rdb_csn_AOD_pcp_NDVI_wnd_rgn}%
\caption{
Normalized seasonal cycle of atmospheric dust (black),
precipitation (blue), \NDVI\ (green), and surface wind speed (red)
over (a)~Eastern Sahel, (b)~Tarim Basin \cite[][]{ZeK05}.
%, (c)~Saudi Arabia, and (d)~China's Loess Plateau.   
\label{fgr:csn}}
\end{figure}
\clearpage

\Large
\begin{figure}
\centering % \centering uses less vertical space than center-environment
\includegraphics[width=0.9\hsize]{/data/zender/fgr/ppr_ZeT04/ccd_1960_2002_ncd_BFL}%
\caption{
Annual incidence $\NNN$\,[\nbrxyrfz] (solid line) 
and total number of reported cases $\NNN_{0}$\,[\nbrxyr] (dashed line)
of valley fever in Kern County from 1960--2002.
Bars show two standard deviations of each year's monthly incidence
statistics projected to annual rates \cite[][]{ZeT04}.
\label{fgr:ncd}}
\end{figure}
\clearpage

\section[Radiation And Climate]{Radiation And Climate}
\Large
\begin{enumerate*}
\item Radiative Forcing
\begin{enumerate*}
\item How do natural natural (dust, volcanic, biomass burning)
  aerosols affect climate?
\item How much solar radiation do clouds, aerosols, and trace gases absorb?
\end{enumerate*}
\item Microphysical Radiative Interactions
\begin{enumerate*}
\item What effects do particle composition, shape, and size have on scattering?
\item Do absorbing inclusions (soot, dust) significantly alter cloud properties?
\end{enumerate*}
\end{enumerate*}
\clearpage

\Large
\begin{figure*}
\centering
\csznote{
pdftk A=/data/zender/ppr/ppr_AMW03.pdf cat A3 output ${DATA}/fgr/ess_atm/ppr_AMW03_fgr3.pdf
pdf2ps ${DATA}/fgr/ess_atm/ppr_AMW03_fgr3.pdf ${DATA}/fgr/ess_atm/ppr_AMW03_fgr3.ps
ps2epsi ${DATA}/fgr/ess_atm/ppr_AMW03_fgr3.ps ${DATA}/fgr/ess_atm/ppr_AMW03_fgr3.eps
} % end csznote
\includegraphics[height=0.9\vsize,clip=true,trim=4.5in 2.0in 0.5in 6.0in]{/data/zender/fgr/ess_atm/ppr_AMW03_fgr3}%
\caption{
Observed and simulated 20th century temperature anomaly attributed to natural and
anthropogenic forcing \cite[][Figure~3]{AMW03}. 
\textbf{Solar forcing exceeded anthropogenic forcing until $\sim$1970!}
\label{fgr:AMW03_fgr3}}
\end{figure*}
\clearpage

\Large
\begin{figure*}
% /bin/cp ${DATA}/ps/swnb2_abs_spc_atm.eps ${DATA}/fgr/ess_atm/swnb2_abs_spc_atm.eps
\includegraphics[width=1.0\hsize]{/data/zender/fgr/ess_atm/swnb2_abs_spc_atm}%
\caption{
(a)~Atmospheric absorptance of sunlight in clear and cloudy sky conditions
\cite[][]{ZBP97}. 
\label{fgr:abs_spc_atm}}
\end{figure*}
\clearpage

\section[Surface Energy Exchange]{Surface Energy Exchange}
\Large
\begin{enumerate*}
\item Turbulent Energy Transport
\begin{enumerate*}
\item Do unresolved wind speed fluctuations alter mean climate?
\item How much dust do (sub-gridscale) dust devils entrain?
\end{enumerate*}
\item Drag Partitioning
\begin{enumerate*}
\item Does vegetation significantly alter heat exchange over adjacent bare ground?
\item Are models too dry because they do not account for this?
\end{enumerate*}
\end{enumerate*}
\clearpage

\Large
\begin{figure}
\centering
\includegraphics[height=0.9\vsize,clip=true,trim=2.0in 1.5in 2.0in 2.15in]{/data/zender/fgr/scapps/sh_bareground}
\caption{
Bareground sensible heat flux [\wxmS] dependence on sub-gridscale (SGS) winds.
(a)~Mean winds, (b)~Including SGS winds, (c)~Difference.
\label{fgr:shflx}}
\end{figure}
\clearpage

\section[Snow-Albedo Feedback]{Snow-Albedo Feedback}
\Large
\begin{enumerate*}
\item Snowpack Optical Properties
\begin{enumerate*}
\item How far does solar energy penetrate in snowpacks?
\item How do temperature, temperature-gradient, and
  time-since-snowfall alter snowpack brightness/absorption?
\end{enumerate*}
\item Dirty Snow
\begin{enumerate*}
\item What impact does ``dirty'' (e.g., sooty, dusty) snow have on climate?
\item Does dirty snow influence sea-ice extent?
\end{enumerate*}
\end{enumerate*}
\clearpage

\Large
\begin{figure*}
\includegraphics[clip=true,trim=0.5in 3.0in 0.0in 3.75in]{/data/zender/fgr/ppr_FlZ05/flx_abs_col_clr}%
\caption{
Solar absorption profiles prescribed by CLM and predicted by SNICAR \cite[][]{FlZ05}.
\label{fgr:rfl_snw_snicar}}
\end{figure*}
\clearpage

\Large
\begin{figure}
\includegraphics[width=0.5\hsize]{/data/zender/fgr/ppr_FlZ05/snowdp_1}%
\includegraphics[width=0.5\hsize]{/data/zender/fgr/ppr_FlZ05/snowdp_2}
\caption{Annual cycle of observed and simulated snow depth over the
  Tibetan Plateau: (Left) Coupled experiments. (Right) Off-line,
  land-only experiments \cite[][]{FlZ05}.}
\label{fgr:snowdp}
\end{figure}
\clearpage

\Large
\begin{figure}
\centering
\includegraphics[height=0.90\vsize,clip=true,trim=1.0in 0.25in 1.0in 0.25in]{/data/zender/fgr/snicar/snicar_soot_cnc_alb}
\caption{Dependence of broadband solar snowpack albedo on (upper panel)
  snow effective radius [\um] and (lower panel) soot concentration
  [\ngxg] \cite[][]{FlZ05,FlZ06}.}
\label{fgr:rfl_snw_soot}
\end{figure}
\clearpage

\section[Distributed Data Reduction And Analysis]{Distributed Data Reduction And Analysis}
\Large
\clearpage

\Large
\begin{figure}
\includegraphics[width=0.5\hsize,clip=true,trim=1.0in 2.5in 1.55in 3.5in]{/data/zender/fgr/ddra/IPCC_multiple_mdls_tempchange_fin}%
\includegraphics[width=0.5\hsize,clip=true,trim=1.0in 2.5in 1.55in 3.5in]{/data/zender/fgr/ddra/IPCC_multiple_mdls_tempchange_Cal_fin}%
\caption{
Predicted Global (left) and California (right) annual-mean temperature
from 2000--2099 under SRESA1B 720\,\ppm\ \COd\ stabilization scenario. 
Temperature scales differ.
\label{fgr:ccsm_tm_tas}}
\end{figure}
\clearpage

\Large
\begin{figure}
\includegraphics[width=\hsize,clip=true,trim=0.0in 3.0in 0.0in 3.0in]{/data/zender/ddra/hiperwall_ipcc.jpg}%
\caption{
HIPerWall displaying ensemble of IPCC AR4 Model results (Photo: Chris Knox)
\label{fgr:hiperwall_ipcc}}
\end{figure}
\clearpage

\section[References]{References}
% Bibliography
%\renewcommand\refname{\normalsize Publications}
\bibliographystyle{jas}
\bibliography{bib}
\vfill

\csznote{
% Transfer required figures to local machine
mkdir -p ${DATA}/fgr/ess_phz
for fl in \
/data/zender/pix/ipcc_ccsm_ddra.pdf \
; do 
scp ashes.ess.uci.edu:${fl} ${DATA}/fgr/ess_phz
printf "Copying ${fl}...\n"
done
} % end csznote

\csznote{
% Usage: Place usage here at end of file so comment character % not needed
cd ~/ess_phz;pdflatex smn_dog_pny.tex;cd -
cd ~/ess_phz;make -W smn_dog_pny.tex smn_dog_pny.dvi smn_dog_pny.ps smn_dog_pny.pdf;cd -
cd ~/ess_phz;pdflatex smn_dog_pny.tex;thumbpdf smn_dog_pny;pdflatex smn_dog_pny.tex;cd -
cd ~/ess_phz;texcln smn_dog_pny;make smn_dog_pny.pdf;bibtex smn_dog_pny;makeindex smn_dog_pny;make smn_dog_pny.pdf;bibtex smn_dog_pny;makeindex smn_dog_pny;make smn_dog_pny.pdf;cd -
cd ~/ess_phz;make -W smn_dog_pny.tex smn_dog_pny.pdf;cd -

scp ${DATA}/ps/smn_dog_pny.pdf dust.ess.uci.edu:/var/www/html/smn/smn_dog_pny.pdf
} % end csznote

% $: rebalance syntax highlighting

\end{document}

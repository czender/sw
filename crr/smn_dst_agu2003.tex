% $Id$

% Purpose: Seminar on mineral dust Ca measurements and modeling

% URL: http://dust.ess.uci.edu/tmp/smn_dst_agu2003.pdf

% Usage: 
% cd ${HOME}/crr;make -W smn_dst_agu2003.tex smn_dst_agu2003.ps;ps2pdf ${DATA}/ps/smn_dst_agu2003.ps ${DATA}/ps/smn_dst_agu2003.pdf;cd -
% scp ${DATA}/ps/smn_dst_agu2003.pdf ${DATA}/ps/smn_dst_agu2003.ps dust.ess.uci.edu:/var/www/html/tmp

\documentclass[final,dvips]{foils}

% Standard packages
\usepackage{amsmath} % \subequations, \eqref, \align
\usepackage{array} % Table and array extensions, e.g., column formatting
\usepackage[usenames]{color} % usenames allows, e.g., ``ForestGreen''
\usepackage{graphicx} % Defines \includegraphics*
\usepackage{longtable} % Multi-page tables, e.g., acronyms and symbols
\usepackage{mdwlist} % Compact list formats \itemize*, \enumerate*
\usepackage{natbib} % \cite commands from aguplus
\usepackage[figuresright]{rotating} % Allows sideways figures and tables
%\usepackage{times} % Times/Roman font, aguplus STRONGLY recommends this for the camera-ready version!!
\usepackage{url} % Typeset URLs and e-mail addresses

% Personal packages
\usepackage{csz} % Library of personal definitions
\usepackage{abc} % Alphabet as three letter macros
\usepackage{dmn} % Dimensional units
\usepackage{chm} % Chemistry
\usepackage{dyn} % Fluid dynamics
\usepackage{aer} % Aerosol physics
\usepackage{rt} % Radiative transfer
\usepackage{psd} % Particle size distributions
\usepackage{smn} % Seminars
% Usage: % Usage: % Usage: \input{jgr_abb} % AGU-sanctioned journal title abbreviations

\def\aapgb{{\it Amer. Assoc. Petroleum Geologists Bull.}}
\def\adg{{\it Adv. Geophys.}}
\def\ajs{{\it Amer. J. Sci.}}
\def\amb{{\it Ambio}}
\def\amgb{{\it Arch. Meteorol. Geophys. Bioclimatl.}}
\def\ang{{\it Ann. Glaciol.}}
\def\angeo{{\it Ann. Geophys.}}
\def\apo{{\it Appl. Opt.}}
\def\areps{{\it Ann. Rev. Earth Planet. Sci.}}
\def\asr{{\it Adv. Space Res.}}
\def\ate{{\it Atmos. Environ.}}
\def\atf{{\it Atmosfera}}
\def\atms{{\it ACM Trans. Math Software}}
\def\ato{{\it Atmos. Ocean}}
\def\atr{{\it Atmos. Res.}}
\def\gbc{{\it Global Biogeochem. Cycles}} % csz
\def\blm{{\it Boundary-Layer Meteorol.}} % csz 
\def\bpa{{\it Beitr. Phys. Atmosph.}}
\def\bams{{\it Bull. Am. Meteorol. Soc.}}
\def\clc{{\it Clim. Change}}
\def\cld{{\it Clim. Dyn.}}
\def\com{{\it Computing}}
\def\dao{{\it Dyn. Atmos. Oceans}}
\def\dsr{{\it Deep-Sea Res.}}
\def\esr{{\it Earth Sci. Revs.}}
\def\gec{{\it Geosci. Canada}}
\def\gei{{\it Geofis. Int.}}
\def\gej{{\it Geogr. J.}}
\def\gem{{\it Geophys. Monogr.}}
\def\geo{{\it Geology}}
\def\grl{{\it Geophys. Res. Lett.}}
\def\ieeec{{\it IEEE Computer}}
\def\ijna{{\it IMA J. Numer. Anal.}}
\def\ijnmf{{\it Int. J. Num. Meteorol. Fl.}}
\def\jac{{\it J. Atmos. Chem.}}
\def\jacm{{\it J. Assoc. Comput. Mach.}}
\def\jam{{\it J. Appl. Meteorol.}}
\def\jas{{\it J. Atmos. Sci.}}
\def\jatp{{\it J. Atmos. Terr. Phys.}}
\def\jcam{{\it J. Climate Appl. Meteorol.}}
\def\jchp{{\it J. Chem Phys.}}
\def\jcis{{\it J. Coll. I. Sci.}}
\def\jcl{{\it J. Clim.}}
\def\jcp{{\it J. Comput. Phys.}}
\def\jfm{{\it J. Fluid Mech.}}
\def\jgl{{\it J. Glaciol.}}
\def\jgr{{\it J. Geophys. Res.}}
\def\jgs{{\it J. Geol. Soc. London}}
\def\jme{{\it J. Meteorol.}}
\def\jmr{{\it J. Marine Res.}}
\def\jmsj{{\it J. Meteorol. Soc. Jpn.}}
\def\josa{{\it J. Opt. Soc. A}}
\def\jpo{{\it J. Phys. Oceanogr.}}
\def\jqsrt{{\it J. Quant. Spectrosc. Radiat. Transfer}}
\def\jpca{{\it J. Phys. Chem. A}}
\def\lnc{{\it Lett. Nuov. C}}
\def\mac{{\it Math. Comp.}}
\def\map{{\it Meteorol. Atmos. Physics.}}
\def\mem{{\it Meteorol. Mag.}}
\def\mnras{{\it Mon. Not. Roy. Astron. Soc.}} 
\def\mwr{{\it Mon. Weather Rev.}} 
\def\nat{{\it Nature}}
\def\pac{{\it Parallel Computing}}
\def\pag{{\it Pure Appl. Geophys.}}
\def\pal{{\it Paleoceanography}}
\def\pht{{\it Physics Today}}
\def\pieee{{\it Proc. IEEE}}
\def\pla{{\it Phys. Lett. A}}
\def\ppp{{\it Paleogeogr. Paleoclim. Paleoecol.}}
\def\pra{{\it Phys. Res. A}}
\def\prd{{\it Phys. Rev. D}}
\def\prl{{\it Phys. Rev. L}}
\def\pss{{\it Planet. Space Sci.}}
\def\ptrsl{{\it Phil. Trans. R. Soc. Lond.}}
\def\qjrms{{\it Q. J. R. Meteorol. Soc.}}
\def\qres{{\it Quat. Res.}}
\def\qsr{{\it Quatern. Sci. Rev.}}
\def\reg{{\it Rev. Geophys.}}
\def\rgsp{{\it Revs. Geophys. Space Phys.}}
\def\rpp{{\it Rep. Prog. Phys.}}
\def\sca{{\it Sci. Amer.}}
\def\sci{{\it Science}}
\def\sjna{{\it SIAM J. Numer. Anal.}}
\def\sjssc{{\it SIAM J. Sci. Stat. Comput.}}
\def\tac{{\it Theor. Appl. Climatl.}}
\def\tel{{\it Tellus}}
\def\wea{{\it Weather}}

%SIAM Review: (Society for Industrial and Applied Mathematics)
%       J. on Computing
%       J. on Control and Optimization
%       J. on Algebraic and Discrete Methods
%       J. on Numerical Analysis 
%       J. on Scientific and Statistical Computing



 % AGU-sanctioned journal title abbreviations

\def\aapgb{{\it Amer. Assoc. Petroleum Geologists Bull.}}
\def\adg{{\it Adv. Geophys.}}
\def\ajs{{\it Amer. J. Sci.}}
\def\amb{{\it Ambio}}
\def\amgb{{\it Arch. Meteorol. Geophys. Bioclimatl.}}
\def\ang{{\it Ann. Glaciol.}}
\def\angeo{{\it Ann. Geophys.}}
\def\apo{{\it Appl. Opt.}}
\def\areps{{\it Ann. Rev. Earth Planet. Sci.}}
\def\asr{{\it Adv. Space Res.}}
\def\ate{{\it Atmos. Environ.}}
\def\atf{{\it Atmosfera}}
\def\atms{{\it ACM Trans. Math Software}}
\def\ato{{\it Atmos. Ocean}}
\def\atr{{\it Atmos. Res.}}
\def\gbc{{\it Global Biogeochem. Cycles}} % csz
\def\blm{{\it Boundary-Layer Meteorol.}} % csz 
\def\bpa{{\it Beitr. Phys. Atmosph.}}
\def\bams{{\it Bull. Am. Meteorol. Soc.}}
\def\clc{{\it Clim. Change}}
\def\cld{{\it Clim. Dyn.}}
\def\com{{\it Computing}}
\def\dao{{\it Dyn. Atmos. Oceans}}
\def\dsr{{\it Deep-Sea Res.}}
\def\esr{{\it Earth Sci. Revs.}}
\def\gec{{\it Geosci. Canada}}
\def\gei{{\it Geofis. Int.}}
\def\gej{{\it Geogr. J.}}
\def\gem{{\it Geophys. Monogr.}}
\def\geo{{\it Geology}}
\def\grl{{\it Geophys. Res. Lett.}}
\def\ieeec{{\it IEEE Computer}}
\def\ijna{{\it IMA J. Numer. Anal.}}
\def\ijnmf{{\it Int. J. Num. Meteorol. Fl.}}
\def\jac{{\it J. Atmos. Chem.}}
\def\jacm{{\it J. Assoc. Comput. Mach.}}
\def\jam{{\it J. Appl. Meteorol.}}
\def\jas{{\it J. Atmos. Sci.}}
\def\jatp{{\it J. Atmos. Terr. Phys.}}
\def\jcam{{\it J. Climate Appl. Meteorol.}}
\def\jchp{{\it J. Chem Phys.}}
\def\jcis{{\it J. Coll. I. Sci.}}
\def\jcl{{\it J. Clim.}}
\def\jcp{{\it J. Comput. Phys.}}
\def\jfm{{\it J. Fluid Mech.}}
\def\jgl{{\it J. Glaciol.}}
\def\jgr{{\it J. Geophys. Res.}}
\def\jgs{{\it J. Geol. Soc. London}}
\def\jme{{\it J. Meteorol.}}
\def\jmr{{\it J. Marine Res.}}
\def\jmsj{{\it J. Meteorol. Soc. Jpn.}}
\def\josa{{\it J. Opt. Soc. A}}
\def\jpo{{\it J. Phys. Oceanogr.}}
\def\jqsrt{{\it J. Quant. Spectrosc. Radiat. Transfer}}
\def\jpca{{\it J. Phys. Chem. A}}
\def\lnc{{\it Lett. Nuov. C}}
\def\mac{{\it Math. Comp.}}
\def\map{{\it Meteorol. Atmos. Physics.}}
\def\mem{{\it Meteorol. Mag.}}
\def\mnras{{\it Mon. Not. Roy. Astron. Soc.}} 
\def\mwr{{\it Mon. Weather Rev.}} 
\def\nat{{\it Nature}}
\def\pac{{\it Parallel Computing}}
\def\pag{{\it Pure Appl. Geophys.}}
\def\pal{{\it Paleoceanography}}
\def\pht{{\it Physics Today}}
\def\pieee{{\it Proc. IEEE}}
\def\pla{{\it Phys. Lett. A}}
\def\ppp{{\it Paleogeogr. Paleoclim. Paleoecol.}}
\def\pra{{\it Phys. Res. A}}
\def\prd{{\it Phys. Rev. D}}
\def\prl{{\it Phys. Rev. L}}
\def\pss{{\it Planet. Space Sci.}}
\def\ptrsl{{\it Phil. Trans. R. Soc. Lond.}}
\def\qjrms{{\it Q. J. R. Meteorol. Soc.}}
\def\qres{{\it Quat. Res.}}
\def\qsr{{\it Quatern. Sci. Rev.}}
\def\reg{{\it Rev. Geophys.}}
\def\rgsp{{\it Revs. Geophys. Space Phys.}}
\def\rpp{{\it Rep. Prog. Phys.}}
\def\sca{{\it Sci. Amer.}}
\def\sci{{\it Science}}
\def\sjna{{\it SIAM J. Numer. Anal.}}
\def\sjssc{{\it SIAM J. Sci. Stat. Comput.}}
\def\tac{{\it Theor. Appl. Climatl.}}
\def\tel{{\it Tellus}}
\def\wea{{\it Weather}}

%SIAM Review: (Society for Industrial and Applied Mathematics)
%       J. on Computing
%       J. on Control and Optimization
%       J. on Algebraic and Discrete Methods
%       J. on Numerical Analysis 
%       J. on Scientific and Statistical Computing



 % AGU-sanctioned journal title abbreviations

\def\aapgb{{\it Amer. Assoc. Petroleum Geologists Bull.}}
\def\adg{{\it Adv. Geophys.}}
\def\ajs{{\it Amer. J. Sci.}}
\def\amb{{\it Ambio}}
\def\amgb{{\it Arch. Meteorol. Geophys. Bioclimatl.}}
\def\ang{{\it Ann. Glaciol.}}
\def\angeo{{\it Ann. Geophys.}}
\def\apo{{\it Appl. Opt.}}
\def\areps{{\it Ann. Rev. Earth Planet. Sci.}}
\def\asr{{\it Adv. Space Res.}}
\def\ate{{\it Atmos. Environ.}}
\def\atf{{\it Atmosfera}}
\def\atms{{\it ACM Trans. Math Software}}
\def\ato{{\it Atmos. Ocean}}
\def\atr{{\it Atmos. Res.}}
\def\gbc{{\it Global Biogeochem. Cycles}} % csz
\def\blm{{\it Boundary-Layer Meteorol.}} % csz 
\def\bpa{{\it Beitr. Phys. Atmosph.}}
\def\bams{{\it Bull. Am. Meteorol. Soc.}}
\def\clc{{\it Clim. Change}}
\def\cld{{\it Clim. Dyn.}}
\def\com{{\it Computing}}
\def\dao{{\it Dyn. Atmos. Oceans}}
\def\dsr{{\it Deep-Sea Res.}}
\def\esr{{\it Earth Sci. Revs.}}
\def\gec{{\it Geosci. Canada}}
\def\gei{{\it Geofis. Int.}}
\def\gej{{\it Geogr. J.}}
\def\gem{{\it Geophys. Monogr.}}
\def\geo{{\it Geology}}
\def\grl{{\it Geophys. Res. Lett.}}
\def\ieeec{{\it IEEE Computer}}
\def\ijna{{\it IMA J. Numer. Anal.}}
\def\ijnmf{{\it Int. J. Num. Meteorol. Fl.}}
\def\jac{{\it J. Atmos. Chem.}}
\def\jacm{{\it J. Assoc. Comput. Mach.}}
\def\jam{{\it J. Appl. Meteorol.}}
\def\jas{{\it J. Atmos. Sci.}}
\def\jatp{{\it J. Atmos. Terr. Phys.}}
\def\jcam{{\it J. Climate Appl. Meteorol.}}
\def\jchp{{\it J. Chem Phys.}}
\def\jcis{{\it J. Coll. I. Sci.}}
\def\jcl{{\it J. Clim.}}
\def\jcp{{\it J. Comput. Phys.}}
\def\jfm{{\it J. Fluid Mech.}}
\def\jgl{{\it J. Glaciol.}}
\def\jgr{{\it J. Geophys. Res.}}
\def\jgs{{\it J. Geol. Soc. London}}
\def\jme{{\it J. Meteorol.}}
\def\jmr{{\it J. Marine Res.}}
\def\jmsj{{\it J. Meteorol. Soc. Jpn.}}
\def\josa{{\it J. Opt. Soc. A}}
\def\jpo{{\it J. Phys. Oceanogr.}}
\def\jqsrt{{\it J. Quant. Spectrosc. Radiat. Transfer}}
\def\jpca{{\it J. Phys. Chem. A}}
\def\lnc{{\it Lett. Nuov. C}}
\def\mac{{\it Math. Comp.}}
\def\map{{\it Meteorol. Atmos. Physics.}}
\def\mem{{\it Meteorol. Mag.}}
\def\mnras{{\it Mon. Not. Roy. Astron. Soc.}} 
\def\mwr{{\it Mon. Weather Rev.}} 
\def\nat{{\it Nature}}
\def\pac{{\it Parallel Computing}}
\def\pag{{\it Pure Appl. Geophys.}}
\def\pal{{\it Paleoceanography}}
\def\pht{{\it Physics Today}}
\def\pieee{{\it Proc. IEEE}}
\def\pla{{\it Phys. Lett. A}}
\def\ppp{{\it Paleogeogr. Paleoclim. Paleoecol.}}
\def\pra{{\it Phys. Res. A}}
\def\prd{{\it Phys. Rev. D}}
\def\prl{{\it Phys. Rev. L}}
\def\pss{{\it Planet. Space Sci.}}
\def\ptrsl{{\it Phil. Trans. R. Soc. Lond.}}
\def\qjrms{{\it Q. J. R. Meteorol. Soc.}}
\def\qres{{\it Quat. Res.}}
\def\qsr{{\it Quatern. Sci. Rev.}}
\def\reg{{\it Rev. Geophys.}}
\def\rgsp{{\it Revs. Geophys. Space Phys.}}
\def\rpp{{\it Rep. Prog. Phys.}}
\def\sca{{\it Sci. Amer.}}
\def\sci{{\it Science}}
\def\sjna{{\it SIAM J. Numer. Anal.}}
\def\sjssc{{\it SIAM J. Sci. Stat. Comput.}}
\def\tac{{\it Theor. Appl. Climatl.}}
\def\tel{{\it Tellus}}
\def\wea{{\it Weather}}

%SIAM Review: (Society for Industrial and Applied Mathematics)
%       J. on Computing
%       J. on Control and Optimization
%       J. on Algebraic and Discrete Methods
%       J. on Numerical Analysis 
%       J. on Scientific and Statistical Computing



 % JGR-sanctioned journal abbreviations

% Margins
\topmargin 0in \headheight 0pt \headsep 0pt
\textheight 9in \textwidth 6.5in
\oddsidemargin 0in \evensidemargin 0in
\marginparwidth 0pt \marginparsep 0pt
\footskip 0pt
\footnotesep=0pt

% Commands specific to this file
\newcommand{\bluestar}{\textcolor{blue}{$^{*}$}}
\newcommand{\greenstar}{\textcolor{green}{$^{*}$}}
\newcommand{\redstar}{\textcolor{red}{$^{*}$}}

% 1. Fundamental commands
\newcommand{\egv}{\ensuremath{\lambda}} % [frc] Eigenvalue
\newcommand{\eof}{\ensuremath{\mathrm{EOF}}} % [frc] Empirical Orthogonal Function
\newcommand{\eofidx}{\ensuremath{i}} % [idx] Empirical Orthogonal Function term index
\newcommand{\eofnbr}{\ensuremath{N}} % [nbr] Number of Empirical Orthogonal Function terms
\newcommand{\prncmp}{\ensuremath{\mathrm{PC}}} % [frc] Principal Component
\newcommand{\vrn}{\ensuremath{v}} % [frc] Variance

% 2. Derived commands
\newcommand{\egvtrm}{\ensuremath{\egv_{\eofidx}}} % [frc] i'th Eigenvalue
\newcommand{\eofone}{\ensuremath{\eof_{1}}} % [frc] First Empirical Orthogonal Function
\newcommand{\eoftwo}{\ensuremath{\eof_{2}}} % [frc] Second Empirical Orthogonal Function
\newcommand{\eofnbrprm}{\ensuremath{\eofnbr^{\prime}}} % [nbr] Number of Empirical Orthogonal Function terms
\newcommand{\eoftrm}{\ensuremath{\eof_{\eofidx}}} % [frc] i'th Empirical Orthogonal Function
\newcommand{\prncmpone}{\ensuremath{\prncmp_{1}}} % [frc] First Principal Component
\newcommand{\prncmptwo}{\ensuremath{\prncmp_{2}}} % [frc] Second Principal Component
\newcommand{\prncmptrm}{\ensuremath{\prncmp_{\eofidx}}} % [frc] i'th Principal Component

% FoilTeX macros
\MyLogo{} % Appears bottom left corner every page
\rightfooter{} % Pagenumber is the default
\leftheader{} % Appears upper left
\rightheader{} % Appears upper right

\begin{document}
\raggedright % Default is justified

\rotatefoilhead{\LARGE \textcolor{blue}{
Mineral dust aerosol and tropospheric ozone: 
Sensitivity to season, species, uptake rate, and chemical composition}}\vspace{-0.5in}\LARGE
\begin{center}
Charlie Zender$^{1}$ and Huisheng Bian$^{1,2}$\\
$^{1}$Dept.\ of Earth System Science,\\ University of California, Irvine\\
$^{2}$Goddard Earth Sciences and Technology Center, NASA GSFC, Maryland\\
\bigskip
\bigskip
\bigskip
%European Geophysical Society, Nice, April~26, 2004\\
EGS, Nice, April~26, 2004\\
{\large Thanks: Vicki Grassian, Michael Prather}
\normalsize
\end{center}

\rotatefoilhead{\bgl
\Large\textcolor{blue}{\hfill Background \hfill}}\vspace{-0.75in}\large
Dust accounts for more surface area than any other aerosol\\
Oxidant budgets may be sensitive to dust \cite[]{DCZ96,ZhC99,GCS01}\\
\begin{itemize*}
\item Up to 10\% \Ot, 50\% \NOt\ removed by dust!
\end{itemize*}
Problems with ``early'' studies:
\begin{enumerate*}
\item Uptake rate uncertainties (\textcolor{blue}{reduced?}, \textcolor{red}{still ${\cal O}(10^{3})$ for \HNOt})
\item Uncertainties not propagated to~\dlthchOt\ (\textcolor{blue}{quantified})
\item Most active \Ot\ removal paths unknown (\textcolor{blue}{determined})
\item Uncertain relative roles of photochemical, heterogeneous forcing (\textcolor{blue}{quantified})
\item No coeval dust, oxidant measurements in ``natural'' settings (\textcolor{blue}{much improved})
\item Simplistic or no mineralogical effects (\textcolor{red}{unchanged})
\item Inadequate representation of aging, surface saturation (\textcolor{red}{unchanged})
\end{enumerate*}

\rotatefoilhead{\bgl
\Large\textcolor{blue}{\hfill Recent Work \hfill}}\vspace{-0.75in}\large
Subsequently, 
\begin{enumerate*}
\item Uptake rate measurements on ideal and authentic dust samples 
  \cite{USP01,MUG02,HaC01a,HaC03a,HaC03b}
\item ACE Asia, MINATROC, PRIDE, SHADE, TRACE-P
\item Global model \Ot\ sensitivity studies by
  \cite{MJL02,BiZ03,BiZ04} and \cite{BBS04}  
\begin{itemize*}
\item Effects of season, surface albedo, precursor proximity, uptake
path, uptake rate, chemical composition 
\end{itemize*}
\item Use of Inorganic Thermodynamic Equilibrium Models (ITEMs) \cite[]{MDKJ02,RoD041}
\end{enumerate*}

\rotatefoilhead{\bgl
\Large\textcolor{blue}{\hfill Modeling Strategy \hfill}}\vspace{-0.5in}\large
\begin{enumerate*}
\item Dust Entrainment And Deposition (DEAD) model
  (\url{http://dust.ess.uci.edu/dead}) simulations of 1990--1999  
\item UCI CTM quantifies range tropospheric~\Ot\ based on range of
  measured $\mssuptcff$ 
\item Empirical Orthogonal Function (EOF) and Principal
  Component Analysis (PCA)
\end{enumerate*}

\foilhead{
\Large\textcolor{blue}{\hfill MOZAIC Comparison: Abidjan\hfill}}\vspace{-0.5in}\large
\enlargethispage*{1in} 
\begin{figure}
\begin{center}
\includegraphics*[width=0.6\hsize]{fgbw/model-obs}%
\end{center}
\caption{January \Ot\ vertical profile over Abidjan ($5$\,\dgrn,
$4$\,\dgrw) from 1994--1999 MOZAIC observations \cite[]{MJL02}.
\csznote{
Error bars indicate observed standard deviation based on 28~observations.
Model results without mineral dust (thin dashes); with mineral dust
photolysis forcing only (thin dash-dot line); with both photolysis and
heterogeneous base-value uptake coefficients (thick dashes) and
a new set of uptake coefficients (see text) (thick dash-dot line).
}} 
\label{fgr:model-obs}
\end{figure}

\foilhead{
\Large\textcolor{blue}{\hfill Dust Forcing \Ot: Photolysis\hfill}}\vspace{-0.5in}\large
\enlargethispage*{1in} 
\begin{figure}
\centering
\includegraphics*[width=1.0\hsize]{/data/zender/fgr/ppr_BiZ03/fgbw/paperphoo3}%
\caption{Photolysis rate forcing of \Ot~[ppb] by dust in January and
July at three atmospheric layers. 
Global annual mean \Ot\ \textcolor{red}{increase}: 0.23\%.
\label{fgr:phooh}}
\end{figure}

\foilhead{
\Large\textcolor{blue}{\hfill Dust Forcing \Ot: Heterogeneous\hfill}}\vspace{-0.5in}\large
\enlargethispage*{1in} 
\begin{figure}
\centering
\includegraphics*[width=1.0\hsize]{/data/zender/fgr/ppr_BiZ03/fgcolor/paperheto3}%
\caption{Decrease in \Ot~[ppb] due to heterogeneous reactions on
dust in January and July at three atmospheric layers.
Global annual mean \Ot\ \textcolor{blue}{decrease}: 0.9\% (assumes $\mssuptcffHNOt = 0.001$).
\label{fgr:heto3}}
\end{figure}

\rotatefoilhead{\bgl
\Large\textcolor{blue}{\hfill Results~I: Hetero.\ vs.\ Photo.\ \hfill}}\vspace{-0.75in}\large
\begin{enumerate*}
\item Heterogeneous uptake much more important than Photochemistry
\item Photochemical forcing strongly sensitive to precursors,
  surface-albedo, season, hemisphere
\item Non-linear coupling causes $< 20$\% of dust \Ot\ forcing
\item Polar \Ot\ changes dominated by transport of perturbed air
\item Global annual mean \Ot\ decrease: 0.7\% (assumes $\mssuptcffHNOt = 0.001$)
\item Climatological regional \Ot\ decreases up to $10\times$ more
\end{enumerate*}
Uncertainty range in $\HNOt$ uptake rate is $10^{-5} < \mssuptcffHNOt < 0.2$.
If we propagate this to $\dlthchOt$ \ldots

\foilhead{}
%\Large\textcolor{blue}{\hfill Sensitivity of Tropospheric \Ot\ to \hfill}\vspace{-0.5in}\large
\enlargethispage*{1in} 
\begin{figure}
\centering
\includegraphics*[width=\hsize,height=\vsize]{ucictm_O3_gmm_HNO3_sns_bw}\vfill
%\includegraphics*[width=\hsize,height=9.0in]{ucictm_O3_gmm_HNO3_sns_blu}\vfill
%\caption{
%Change in tropospheric \Ot\,[\%] due to heterogeneous chemistry and
%photochemisty effects of dust for 
%(a)~$\mssuptcffHNOt = 0.1$, (b)~$\mssuptcffHNOt = 0.001$, and
%(c)~Difference: (a)~$-$~(b).
%\label{fgr:sns_mssuptcff}}
\end{figure}

\foilhead{
\Large\textcolor{blue}{\hfill \Ot\ Sensitivity to Species\hfill}}\vspace{-0.5in}\large
\enlargethispage*{1in} 
\begin{figure}
\begin{center}
\includegraphics*[width=0.6\hsize]{/data/zender/fgr/ppr_BiZ04/fgcolor/vheto3-cl.eps}%
\end{center}
\caption{Zonal mean \eofone\ of \dlthchXOt\ variances due to
  irreversible uptake of seven reactive species on mineral dust (left).
\csznote{
  Percent variances along \eofone\ (\ref{eqn:var_pct_dfn}) are
indicated.
}
  Correlation $\crrcff$ between \eofone\ of \dlthchOt\ and \dlthchXOt, 
  and correlation between \prncmpone\ of \dlthchOt\ and \dlthchXOt are shown.}
\label{fgr:vheto3}
\end{figure}

\foilhead{}
\enlargethispage*{1in} 
\begin{figure}
\begin{center}
\includegraphics*[width=1.0\hsize]{/data/zender/fgr/ppr_BiZ04/fgbw/surface}%
\end{center}
\caption{
Primary modes of surface \Ot\ reduction due to irreversible uptake by
mineral dust of (a)~All eight gases, (b)~\Ot\ only, (c)~\HNOt\ only,
and (d)~\NOt\ only.}
\label{fgr:surface}
\end{figure}

\rotatefoilhead{\bgl
\Large\textcolor{blue}{\hfill Results~II: Uptake Pathways, Uncertainty \hfill}}\vspace{-0.75in}\large
\begin{enumerate*}
\item Indirect mechanism (\HNOt\ removal) more important than direct
  \Ot\ uptake 
\begin{enumerate*}
\item Direct \Ot\ uptake \textit{may} exceed \HNOt-induced \Ot\ uptake
  (especially in Northern Spring) if 
$\mssuptcffOt > 5.0 \times 10^{-5}$ and 
$\mssuptcffHNOt < 1.1 \times 10^{-3}$.
\item \NOt~is also important in NH summer 
\end{enumerate*}
\item Uptake rate range of $10^{-5} < \mssuptcffHNOt < 0.1$ leads to
  \dlthchOt\ of $0.5$--$5.2$\% 
\begin{enumerate*}
\item Flawed models exaggerate \dlthchOt\ uncertainty
\end{enumerate*}
\end{enumerate*}

\rotatefoilhead{\bgl
\Large\textcolor{blue}{\hfill Future Prospects \hfill}}\vspace{-0.75in}\large
Reducing these order-of-magnitude uncertainties in \dlthchOt\ requires   
correct model representation of global dust composition, deliquescence,
and aging.
\begin{enumerate*}
\item Measurements: Target $\mssuptcffHNOt$ ($\mssuptcffNOt$, $\mssuptcffOt$)
\begin{itemize*}
\item Are surface area corrections ($\mssuptcff \rightarrow \mssuptcffBET$) necessary?
\item Characterize $\mssuptcffHNOt$ dependence on \RH\ and \CaCOt\ \cite[]{KGL03}
\item Systematize $\mssuptcff$ mixing and aging rules
\end{itemize*}
\item Models: Incorporate accurate source mineralogy
\begin{itemize*}
\item Remote sensing measurements of heterogeneous source soil
  mineralogy \cite[]{TSG02} 
\item $\mssuptcff$ mixing, aging, saturation, RH, \CaCOt\ rules
\item ITEMs with mixed-phase, internal, external mixing
\end{itemize*}
\end{enumerate*}

\rotatefoilhead{\Large \textcolor{blue}{
Detection of the geographic distribution of source soil mineralogy in
long range transported aerosol from {\Ca} measurements and simulations
in the {Dust Entrainment And Deposition (DEAD)} model}}\vspace{-0.5in}\LARGE
\begin{center}
Charlie Zender, Huisheng Bian, Sarah Bortz\\
Dept.\ of Earth System Science, UC Irvine\\
\bigskip
\bigskip
\bigskip
AGU, San Francisco, Dec.~10, 2003\\
{\large Thanks:\\ Rich Arimoto, Joe Prospero, Phil Rasch, Dennis Savoie}
\normalsize
\end{center}

\rotatefoilhead{\bgl
\Large\textcolor{blue}{\hfill Silly \Ca\ Factoids \hfill}}\vspace{-0.75in}\large
\vspace{-0.5in}
\begin{enumerate*}
\item Crustal fraction of \Ca\ is $\sim 4.1$\% and highly variable % \cite[]{CRC95}
\item Mean continental \CaCOt\ topsoil content is $\sim 2.5$\% \cite[]{Sch99}
\item Dust particles are 3--5\% \CaCOt\ \cite[]{Pye87}
%\item \CaCOt\ affects soil alkalinity \cite[]{Pye87}
\item \CaCOt\ larger fraction of silt than clay-sized particles \cite[]{CSB99}
\item \CaCOt\ affects soil aggregation and $\wndfrcthr$ (Alfaro, 2003)
\item \CaCOt\ highly correlated ($\crrcff \approx 0.8$) with \Al\ in
  China (Arimoto, 2003)
\end{enumerate*}

\rotatefoilhead{\bgl
\Large\textcolor{blue}{\hfill Why study \Ca\ in dust? \hfill}}\vspace{-0.75in}\large
\begin{enumerate*}
\item \CaCOt\ content of dust is 3--5\% \cite[]{Pye87}, highly variable (up to 20\%)
\item \CaCOt\ buffers aerosol \pH\ (important for soluble \Fe)
\item \CaCOt\ \textcolor{ForestGreen}{limits \HNOt} and \SOd\ uptake on dust \cite[]{DCZ96}
\item \CaCOt\ \textcolor{ForestGreen}{increases hygroscopicity}
\cite[]{KGL03} and likely \HNOt\ uptake (important for \Ot\ formation)
\item Disparity in measured uptake coefficients of \HNOt\
  ($\mssuptcffHNOt$) on authentic dust is $\sim 10^{3}$
  \cite[e.g.,][]{HaC01a,Gra02}
\item \Ca\ \textcolor{ForestGreen}{``accelerated aging''} may explain
  $\mssuptcffHNOt$ disparity, reduce uncertainty in dust-induced
  removal of $1$--$5$\% of tropospheric \Ot\ \cite[]{BiZ04}  
\end{enumerate*}

\rotatefoilhead{\bgl
\Large\textcolor{blue}{\hfill Objectives \hfill}}\vspace{-0.75in}\large
\begin{enumerate*}
\item Do downwind station measurements show spatial signature of \Ca\  
  in soil? 
\item Is \Ca\ useful tracer for dust source region?
\item How best to estimate \Ca\ in dust for chemistry studies?
\end{enumerate*}

\rotatefoilhead{\bgl
\Large\textcolor{blue}{\hfill Modeling Strategy \hfill}}\vspace{-0.5in}\large
Use MATCH global tracer model to simulate climatological \CaCOt\
concentration in two Dust Entrainment And Deposition (DEAD) model   
%(\url{http://dust.ess.uci.edu/dead}) 
simulations of 1990--1992:
\begin{enumerate*}
\item Spatially Uniform: $\msscncCa = 0.04 \times \msscncdst$
\item Spatially Heterogeneous: \Ca\ from IGBP \CaCOt\ database
  \cite[]{CaS98,Sch99}  
\end{enumerate*}
Caveat: Model does not (yet) account for insoluble~\Ca,
anthropogenic~\Ca\ (cement?), or sea salt~\Ca.
\par
UCI CTM used to quantify range of effects on tropospheric~\Ot.


\rotatefoilhead{\bgl
\Large\textcolor{blue}{\hfill Data Evaluation \hfill}}\vspace{-0.5in}\large
Evaluate simulated against measured and inferred \Ca:
\begin{enumerate*}
% NAA = Neutron Activation Analysis
% AAS = Atomic Absorption Spectrometry
% IC = Ion Chromatography
\item Total \Ca\ measured by Neutron Activation Analysis
  \cite[]{ADR95,ADS96} of daily samples at Barbados (1984--1992),
  Bermuda (1988--1997),  Iza\~{n}a (1989--1997), Mace Head
  (1989--1994) (mostly AEROCE data)
\item Soluble \Ca\ measured by atomic absorption spectrometry
  of biweekly samples at Ogori, Tsushima, and Seoul (Wakamatsu et~al.,
  1996) (1990--1992),  
\item Infer \Ca\ as 4\% of dust from Arimoto \Al\ measurements
\item Infer \Ca\ as 4\% of dust from U.~Miami ash measurements
\end{enumerate*}

\foilhead{
\Large\textcolor{blue}{\hfill Soil \CaCOt\ Content \hfill}}\vspace{-0.5in}\large
\enlargethispage*{1in} 
\begin{figure}
\centering
% ${HOME}/idl/gcm.pro:gcm_xy_bch,pll=4,img=0,fld_top=1,pnl_lbl='!5(a)',prn=0
\includegraphics*[width=\hsize]{map_clm_mss_frc_CaCO3}\vfill
%\includegraphics*[width=\hsize]{dstmch90_clm_DSTMPC}\vfill
% ${HOME}/idl/gcm.pro:gcm_xy_bch,pll=4,img=0,fld_top=1,pnl_lbl='!5(b)',prn=0
%\includegraphics*[width=\hsize]{/data/zender/ps/dstmch90_clm_DSTMPC}\vfill
\includegraphics*[width=\hsize]{/data/zender/ps/dstmch90_clm_DSTSFDPS}\vfill
\caption{
(a)~Mass fraction of \CaCOt\ [\%] in surface soil \cite[]{Sch99}.
%(b)~Column mass burden of dust [\mgxmS] \cite[]{ZBN03}.
(b)~Mass deposition flux of dust [\ugxmSs] \cite[]{ZBN03}.
\label{fgr:mss_frc_CaCO3}}
\end{figure}

\csznote{
export caseid=dstmch90;export yr_sng='clm';
ncap -O -s "DSTSFDPSCA=DSTSFDPS*0.04" -s "DSTSFDPSCA@units='kg m-2 s-1'" -s "DSTMPCCA=DSTMPC*0.04" -s "DSTMPCCA@units='kg m-2'" ${DATA}/${caseid}/${caseid}_${yr_sng}.nc ${DATA}/${caseid}/${caseid}_${yr_sng}.nc

export caseid=dstmchx3;export yr_sng='clm';
ncap -O -s "DSTSFDPSCA=DSTSFDPS*0.40" -s "DSTSFDPSCA@units='kg m-2 s-1'" -s "DSTMPCCA=DSTMPC*0.40" -s "DSTMPCCA@units='kg m-2'" ${DATA}/${caseid}/${caseid}_${yr_sng}.nc ${DATA}/${caseid}/${caseid}_${yr_sng}.nc
} % end csznote

\foilhead{
\Large\textcolor{blue}{\hfill \Ca\ deposition\hfill}}\vspace{-0.5in}\large
\enlargethispage*{1in} 
\begin{figure}
\centering
% ${HOME}/idl/gcm.pro:gcm_xy_bch,pll=4,img=0,fld_top=1,pnl_lbl='!5(b)',prn=0
\includegraphics*[width=\hsize]{/data/zender/ps/dstmch90_clm_DSTSFDPSCA}\vfill
% ${HOME}/idl/gcm.pro:gcm_xy_bch,pll=4,img=0,fld_top=1,pnl_lbl='!5(b)',prn=0
\includegraphics*[width=\hsize]{/data/zender/ps/dstmchx3_clm_DSTSFDPSCA}\vfill
\caption{
Mass deposition flux of \Ca\ [\ngxmSs] for (a)~Uniform (4\%) and
(b)~Hetergeneous (IGBP-based) mixing ratios in dust.
\label{fgr:DSTSFDPSCA}}
\end{figure}

\foilhead{
\Large\textcolor{blue}{\hfill Atmospheric \Ca\ Burden\hfill}}\vspace{-0.5in}\large
\enlargethispage*{1in} 
\begin{figure}
\centering
% ${HOME}/idl/gcm.pro:gcm_xy_bch,pll=4,img=0,fld_top=1,pnl_lbl='!5(b)',prn=0
\includegraphics*[width=\hsize]{/data/zender/ps/dstmch90_clm_DSTMPCCA}\vfill
% ${HOME}/idl/gcm.pro:gcm_xy_bch,pll=4,img=0,fld_top=1,pnl_lbl='!5(b)',prn=0
\includegraphics*[width=\hsize]{/data/zender/ps/dstmchx3_clm_DSTMPCCA}\vfill
\caption{
Column mass burden of \Ca\ [\mgxmS] for (a)~Uniform (4\%) and 
(b)~Hetergeneous (IGBP-based) mixing ratios in dust.
\label{fgr:DSTMPC}}
\end{figure}

\foilhead{
\Large\textcolor{blue}{\hfill Near-surface \Ca\ at Barbados \hfill}}\vspace{-0.5in}\large
\begin{figure}
\centering
% ncl < ${HOME}/caco3/caco3_Ca_clm_Brb.ncl;pepsi ${DATA}/ps/caco3_Ca_clm_Brb.eps
\includegraphics*[width=\hsize]{caco3_Ca_clm_Brb}\vfill
%\caption{
%(a)~Near-surface \Ca\ mass concentration [\ugxmC] 
%\label{fgr:Ca_clm_brb}}
\end{figure}

\foilhead{
\Large\textcolor{blue}{\hfill Near-Surface \Ca\ at Iza\~{n}a \hfill}}\vspace{-0.5in}\large
\begin{figure}
\centering
\includegraphics*[width=\hsize]{caco3_Ca_clm_Izn}\vfill
\end{figure}

\foilhead{
\Large\textcolor{blue}{\hfill Near-Surface \Ca\ at Bermuda \hfill}}\vspace{-0.5in}\large
\begin{figure}
\centering
\includegraphics*[width=\hsize]{caco3_Ca_clm_Brm}\vfill
\end{figure}

\foilhead{
\Large\textcolor{blue}{\hfill Near-Surface \Ca\ at Tsushima \hfill}}\vspace{-0.5in}\large
\begin{figure}
\centering
\includegraphics*[width=\hsize]{caco3_Ca_clm_Tsu}\vfill
\end{figure}

\rotatefoilhead{\bgl
\Large\textcolor{blue}{\hfill \Ca\ Budget \hfill}}\vspace{-0.5in}\large
\begin{table*}
\begin{minipage}{\hsize}
\renewcommand{\footnoterule}{\rule{\hsize}{0.0cm}\vspace{-0.0cm}} % KoD95 p. 111
\centering % \centering uses less vertical space than center-environment
\caption[Global \Ca\ Budget]{
Global \Ca\ emission/deposition mass flux [\Tgxyr] and atmospheric
burden [Tg] for Uniform (4\%) and Heterogeneous (IGBP) \Ca\ mixing
ratios in dust.  
Percentages are relative to total dust emission and burden,
respectively. 
\label{tbl:bdg}}
\vspace{\cpthdrhlnskp}
\begin{tabular}{l *{2}{>{$}r<{$}}}
\hline \rule{0.0ex}{\hlntblhdrskp}%
& \mbox{Emission} & \mbox{Burden Tg} \\[0.0ex]
& \mbox{\Tgxyr\ (\%)} & \mbox{Tg (\%)} \\[0.0ex]
\hline \rule{0.0ex}{\hlntblntrskp}%
& & \\[-2.0ex]
Uniform \Ca\ & 60 (4.0\%) & 0.68 (4.0\%) \\[0.0ex]
IGBP \Ca\    & 25 (1.7\%) & 0.25 (1.5\%) \\[0.0ex]
\hline
\end{tabular}
\end{minipage}
\end{table*} % end tbl:bdg
\csznote{
export caseid=dstmchx3;export yr_sng='clm';
export caseid=dstmch90;export yr_sng='clm';
ncap -O -D 1 -s "DSTSFDPS=DSTSFGRV+DSTSFPCP+DSTSFTRB" -s "DSTSFDRY=DSTSFGRV+DSTSFTRB" ${DATA}/${caseid}/${caseid}_${yr_sng}.nc ${DATA}/${caseid}/${caseid}_${yr_sng}.nc
ncwa -O -a lat,lon -w gw ${DATA}/${caseid}/${caseid}_${yr_sng}.nc ${DATA}/${caseid}/${caseid}_${yr_sng}_xy.nc
ncap -O -s "DSTMPCTG=DSTMPC*4.0*3.141*(6.370e+06)^2/(1.0e12/1.0e3)" -s "DSTSFDPSTGXYR@units='Tg'" -s "DSTSFDPSTGXYR=DSTSFDPS*4.0*3.141*(6.370e+06)^2*86400*365/(1.0e12/1.0e3)" -s "DSTSFDPSTGXYR@units='Tg yr-1'" -s "DSTSFMBLTGXYR=DSTSFMBL*4.0*3.141*(6.370e+06)^2*86400*365/(1.0e12/1.0e3)" -s "DSTSFMBLTGXYR@units='Tg yr-1'" ${DATA}/${caseid}/${caseid}_${yr_sng}_xy.nc ${DATA}/${caseid}/${caseid}_${yr_sng}_xy.nc
ncks -C -v DSTMPCTG,DSTSFDPSTGXYR,DSTSFMBLTGXYR -H -u ${DATA}/${caseid}/${caseid}_${yr_sng}_xy.nc
} % end csznote

\rotatefoilhead{\bgl
\Large\textcolor{blue}{\hfill \CaCOt\ in Dust: Impact on Air Quality \hfill}}\vspace{-0.75in}\large
\cite{KGL03} observed in atmospheric conditions
\begin{rxnarray}
% fxm: These reactions are numbered with DCZ96 prefixes in aer.ps
\CaCOt + 2\HNOt & \yields & \CaNOtd + \HdCOt \\ % KGL03 p. 48-1 (R1)
\label{rxn:CaCO3+2HNO3}
\CaNOtd + \HdCOt & \yields & \CaNOtd + \COd + \HdO \\ % KGL03 p. 48-1 (R1), HaC01a p. 3102 (1)
\label{rxn:CaNO32+H2CO3}
\CaNOtd\sldprn + n\cdot\HdO & \yields & \CaNOtd\lqdprn % KGL03 p. 48-1 (R2)
\label{rxn:CaNO32+nH2O}
\end{rxnarray}
Phase change in (\ref{rxn:CaNO32+nH2O}) may explain the observed
high reactivity of \HNOt\ and dust, $\mssuptcffHNOt \sim 0.1$
(Hanisch and Crowley, 2001, 2003).
At this rate \textcolor{ForestGreen}{irreversible \HNOt\ uptake
  occurs as fast on authentic dust as on cloud droplets}

\foilhead{}
%\Large\textcolor{blue}{\hfill Effect of Dust on Tropospheric \Ot\ \hfill}}\vspace{-0.5in}\large
\enlargethispage*{1in} 
\begin{figure}
\centering
\includegraphics*[width=\hsize,height=\vsize]{ucictm_O3_gmm_HNO3_sns_bw}\vfill
%\includegraphics*[width=\hsize,height=9.0in]{ucictm_O3_gmm_HNO3_sns_blu}\vfill
%\caption{
%Change in tropospheric \Ot\,[\%] due to heterogeneous chemistry and
%photochemisty effects of dust for 
%$(a)~$\mssuptcffHNOt = 0.1$, (b)~$\mssuptcffHNOt = 0.001$, and
%(c)~Difference: (a)~$-$~(b).
%\label{fgr:Ca_clm_brb}}
\end{figure}

\rotatefoilhead{\bgl
\Large\textcolor{blue}{\hfill Conclusions \hfill}}\vspace{-0.5in}\large
\begin{enumerate*}
\item Uniform \Ca\ mixing better predicts near-surface \Ca\
  concentration at Iza\~{n}a, Bermuda, Tsushima. 
  Heterogeneous \Ca\ mixing does better at Barbados.
\item Incorporating size-dependent \Ca\ fractionation \cite[]{CSB99}
  worsens bias (not shown)
\item Consistent with underprediction of large particle transport
  \cite[]{Gin03,CTH03,GrZ04} 
\item Consistent with up to $3 \times$ \Ca-enrichment in sandblasted
  particles (dust). Possible strategy for chemical models.
\item Uniform (4\%) \Ca\ distribution with high (observed)
  $\mssuptcffHNOt$ removes $\sim 5$\% of tropospheric~\Ot.
  Heterogeneous \Ca\ distribution with low (observed)
  $\mssuptcffHNOt$ removes $\sim 1$\% of tropospheric~\Ot.
\end{enumerate*}

\csznote{
% Transfer required figures to local machine
for fl_stb in \
caco3_Ca_clm_Brb \
caco3_Ca_clm_Brm \
caco3_Ca_clm_Izn \
caco3_Ca_clm_Tsu \
ucictm_O3_gmm_HNO3_sns_bw \
ucictm_O3_gmm_HNO3_sns_blu \
map_clm_mss_frc_CaCO3 \
dstmch90_clm_DSTSFDPS \
dstmch90_clm_DSTSFDPSCA \
dstmchx3_clm_DSTSFDPSCA \
dstmch90_clm_DSTMPC \
dstmch90_clm_DSTMPCCA \
dstmchx3_clm_DSTMPCCA \
; do 
scp 'ashes.ess.uci.edu:${DATA}/ps/'${fl_stb}'.eps' ${DATA}/ps
epstopdf ${DATA}/ps/${fl_stb}.eps
done
} % end csznote
% $: rebalance syntax highlighting

\rotatefoilhead{%\bgl
\Large\textcolor{blue}{\hfill References\hfill}}\vspace{-0.5in}\large
% Bibliography
%\renewcommand\refname{\normalsize Publications}
\bibliographystyle{jas}
\bibliography{bib}

\csznote{
************************************************************************
Begin Letters from Savoie
************************************************************************
Sarah and Charlie:  My responses are included below. _ Dennis

At 09-06-2002 11:54 -0800, bortz@uci.edu wrote:

>         Although I have already contacted you before, I'll introduce myself
> >again. My name is Sarah Bortz and I am a graduate student of Dr. Charlie
> >Zender's in the Earth System Science Department at the University of
> >California at Irvine.  My summer research project has involved studying
> >the calcium content of mineral dust at Bermuda, mostly by comparing two
> >observational data sets (hereafter abbreviated RSMAS [from Dennis
> >Savoie] and NMSU [from Richard Arimoto]) and output from a chemical
> >transport model (labeled dstmch93 on the attached plots).
> >       I have plotted both the climatologies and the monthly averages for the
> >specific time period of September 1990 to June 1991 (the sampling period
> >covered by the RSMAS data) of total dust, total calcium, and non-seasalt
> >calcium; along with the ratios of non-seasalt calcium to total calcium,
> >non-seasalt calcium to total dust, and total calcium to total dust.
> >Since the RSMAS data contained values for both total calcium and
> >non-seasalt calcium, I assumed that the rest of the calcium was present
> >as seasalt calcium (ie...... seasalt calcium concentration = total
> >calcium concentration - non-seasalt calcium),
> >
> >1)Is it reasonable to assume that the concentration of seasalt calcium
> >can be calculated by subtracting non-seasalt calcium from total
> >calcium?

Yes, this is a reasonable assumption.  In fact, the determination of
monsea-salt calcium was actually calculated from total calcium minus
seasalt calcium which I calculated as Na+ times 0.03825 (the Ca/Na
ratio in bulk seawater). 

> >2)Can there really be that much calcium present in seasalt?

If there is enough seasalt, there certainly can be.  In fact, the
seasalt concentrations at Bermuda are not that high compared to some
other areas of the oceans. 

> >3)Are there (other) ways to measure or calculate seasalt calcium
> >concentrations and to account for the extra calcium?

I know of no better way to estimate seasalt calcium than from the
concentration of water-soluble Na+. 

> >and since the NMSU data only contained trace element concentrations, I
> >calculated the total dust concentrations using the assumption that
> >globally, aluminum composed 8% of total dust.
> >4) Was that a reasonable assumption as well?

That assumption is as good as any and is widely used.  There is a
potentially significant problem in heavily polluted areas with a
significant amount of flyash since the major element composition of
flyash is very similar to that of average continental crust.  The
trace element concentrations might help to sort this out but the
results are not necessarily unambiguous. 

> >Finally, I compared all of the observational data for total calcium to
> >model values calculated as 3% of the total dust concentrations (another
> >global assumption).

This is a much poorer assumtion as the calcite and/or gypsum contant
of continental dust varies widely from one location to another.
Saharan dust collected over the Mediterranean Sea has much more
calcium than that collected over the tropical North Atlantic.
Glaccum's measurements of calcite + gypsum yields a dust calcium
concentration of about 2 at Barbados and 3-4 at Sal Island.  These
results agree fairly well with my estimates based on measurements of
nss calcium (from Na+) and mineral dust. 

> >       As you can probably tell from the attached plots, the RSMAS total
> >calcium concentration peaks in March and October, while the NMSU total
> >calcium concentration shows the highest peak in late summer/early fall,
> >with a smaller peak in the spring.  Our chemical transport model is able
> >to reproduce the primary NMSU peak fairly well but misses the secondary
> >peak, and does not correlate well with the RSMAS total calcium data.
> >Specifically, our model predicts the highest concentrations of dust
> >during the three months of unavailable RSMAS data. Additionally, when
> >the ratio of total calcium to total dust is highest, the ratio of
> >non-seasalt calcium to total calcium is lowest. According to plots of
> >the NMSU data, the ratio of climatological averages of total calcium to
> >(calculated) total dust is greater than one during January.
> >5)How can I explain this?

You need to look at more than the ratios.  A high ratio of total
calcium to total dust may be indicative of either a very high sea-salt
concentration (i.e. high sea-salt) or of a very low dust concentration
or both.  A low ratio would result from the reverse situation.  If any
of the concentrations are very low, there may be a problem with
uncertainties in the concentrations that are strongly magnified in the
ratios. 

> >6)Lastly, exactly what techniques were used to measure both the total
> >dust, total calcium (RSMAS and NMSU) and non-seasalt calcium
> >concentrations (RSMAS)?

This brings up a problem that Rich and I and others discussed
extensively during several AEROCE meetings.  The NMSU calcium and
sodium concentrations are made with instrumental neutron activation
analysis and should therefore measure total sodium and total calcium
regardless of whether there is any insoluble component.  My analyses
are of the water-soluble Na and Ca components only.  In both cases,
nss Ca was estimated as Ca minus 0.03825 times Na.  My analyses of Na+
at Izana (virtually no sea-salt) indicate that Saharan dust contains
virtually no water-soluble sodium.  Hence the water-soluble sodium
should be solely a reflection of sea-salt sodium; that may not be the
case with Na from NAA if dust contains sodium that is not
water-soluble.  There is also a posssibility that not all of the Ca in
Saharan dust is water-soluble. 
Total dust at Bermuda was estimated from the neutron activation
analysis of Al in the manner that you have stated previously.. 

> I would appreciate any information you can provide to help me resolve my
> questions .
> Regards,
> Sarah E. Bortz
************************************************************************
End Letters from Savoie
************************************************************************

************************************************************************
Begin Letters from Arimoto
************************************************************************
Folks: Just got back from a trip to Taiwan and I apologize for not
responding earlier.  However, I think Dennis has answered all of the
questions in about the same way I would have. 

The INAA data show that during  the high dust season at Bermuda, about
10\% or a little more of the monthly mean aerosol Na can be ascribed to
crustal sources rather than sea salt, but in general the
contribution of mineral dust to Na is quite small.  Again, be
careful about soluble vs. total Na, Ca, etc. 

There are some minor differences in the various references giving the
elemental composition of seawater, but the Ca to Na ratio shouldn't
vary by more then a few percent among them.  Also, at one point, there
was some discussion of possible enrichments of Ca due to bubble
scavenging and microlayer effects, but I think this too was resolved
as being relatively unimportant (I'm not sure that anything was
actually published on this after the effects of dust on the marine
aerosol were recognized.  My recollection is that some of the data we
collected for SEAREX showed a small, i.e., few percent, enrichment of
Ca in sea salt over seawater ratios, but I wasn't sure that the
analyses were accurate enough to make much of this.) 

There also are slight differences in the Ca and Al contents of crustal
material in the various compilations of data.  I usually use the
Taylor and McLennan data from the mid 90s.  As Dennis noted, you might
want to look at the Ca data with special attention to source region as
this element is more variable than some others. 

Rich
************************************************************************
At 03:03 PM 12/19/2002 -0800, you wrote:

> Hi Rich,

> We have mixed results. Our hypotheses is that there should be
> some significant spatial variation in the Ca:Al ratio because Ca
> is more variable in source soils than Al is.

OK. I was/still am doing a simple assessment of soluble Ca (as
measured by IC) vs. Al.  For Gosan, the correlation is very good (also
true for the other alkali/alkaline earths we determine. 

> For a given
> station we would not expect much temporal variation in Ca:Al.
> When we account for the Ca variation in source soils
> (and leave Al ~ 8%) in the model, we get better agreement with
> your Ca observations at some stations in the Atlantic, but not all.
> Ca simulations at Barbados improve, no clear winner at Mace Head, and
> worsen at Izana and Bermuda.
> We seem to have an anthropogenic Ca signal at Mace Head and Tsushima.
> But I'm not aware of any anthropogenic Ca sources, are you?

Cement.

The Ca/Al ratio at Zhenbeitai, was about 1 (which is about the same as
in Chinese loess).  At Gosan, the Ca/Al ratio is a little lower, about
0.6, but I'm still working up the data so these numbers might change a
little. 

Thanks for the info.

My best for '03.

Rich
************************************************************************
End Letters from Arimoto
************************************************************************

} % csznote

\end{document}


% $Id$

% Purpose: Seminar on mineral dust modeling

% Usage: 
% cd ${HOME}/crr;make -W smn_dst_mdl.tex smn_dst_mdl.ps;cd -

\documentclass[final,dvips]{foils}

% Standard packages
\usepackage{amsmath} % \subequations, \eqref, \align
\usepackage{array} % Table and array extensions, e.g., column formatting
\usepackage[usenames]{color} % usenames allows, e.g., ``ForestGreen''
\usepackage{graphicx} % Defines \includegraphics*
\usepackage{longtable} % Multi-page tables, e.g., acronyms and symbols
\usepackage{mdwlist} % Compact list formats \itemize*, \enumerate*
\usepackage{natbib} % \cite commands from aguplus
\usepackage[figuresright]{rotating} % Allows sideways figures and tables
%\usepackage{times} % Times/Roman font, aguplus STRONGLY recommends this for the camera-ready version!!
\usepackage{url} % Typeset URLs and e-mail addresses

% Personal packages
\usepackage{csz} % Library of personal definitions
\usepackage{abc} % Alphabet as three letter macros
\usepackage{dmn} % Dimensional units
\usepackage{chm} % Chemistry
\usepackage{dyn} % Fluid dynamics
\usepackage{aer} % Aerosol physics
\usepackage{rt} % Radiative transfer
\usepackage{psd} % Particle size distributions
\usepackage{smn} % Seminars
\input{jgr_abb} % JGR-sanctioned journal abbreviations

% Margins
\topmargin 0in \headheight 0pt \headsep 0pt
\textheight 9in \textwidth 6.5in
\oddsidemargin 0in \evensidemargin 0in
\marginparwidth 0pt \marginparsep 0pt
\footskip 0pt
\footnotesep=0pt

% Commands specific to this file
\newcommand{\bluestar}{\textcolor{blue}{$^{*}$}}
\newcommand{\greenstar}{\textcolor{green}{$^{*}$}}
\newcommand{\redstar}{\textcolor{red}{$^{*}$}}

% FoilTeX macros
\MyLogo{} % Appears bottom left corner every page
\rightfooter{} % Pagenumber is the default
\leftheader{} % Appears upper left
\rightheader{} % Appears upper right

\begin{document}
\raggedright % Default is justified

\rotatefoilhead{\huge \textcolor{blue}{The Atmospheric Physics of Climate Change}}\vspace{-0.5in}\LARGE
\begin{center}
Charles S. Zender\\
Deptartment of Earth System Science\\
University of California, Irvine\\
\bigskip
\bigskip
\bigskip
Summer Science Program\\
Ojai, CA, July 17, 2004\\
\bigskip
\normalsize
\end{center}

\rotatefoilhead{\huge \textcolor{blue}{
Global Patterns of Wind Erosion and Dust Emission in
the Present Climate}}\vspace{-0.5in}\LARGE
\begin{center}
Charles S. Zender\\
Dept.\ of Earth System Science, UC Irvine\\
\bigskip
\bigskip
\bigskip
Universitat Polytecnica de Catalunya, March 26, 2004\\
\bigskip
{\large Thanks: H.~Bian, A.~Grini, M.~Flanner, D.~Newman}
\normalsize
\end{center}

\rotatefoilhead{\huge \textcolor{blue}{
The Influence of Wind Erosion in Past, Present, and Future
Climates}}\vspace{-0.5in}\LARGE 
\begin{center}
Charles S. Zender\\
Dept.\ of Earth System Science, UC Irvine\\
\bigskip
\bigskip
\bigskip
Institut de Ci\`{e}ncia i~Tecnologia Ambientals\\
Universitat Autonoma de Barcelona, \\
March 30, 2004\\
\bigskip
{\large Thanks: H.~Bian, A.~Grini, M.~Flanner, D.~Newman}
\normalsize
\end{center}

\foilhead{\bgp
\Large\textcolor{blue}{\hfill Photolysis Forcing \hfill}}\vspace{-0.5in}\large
\begin{figure}
\centering
\includegraphics*[width=1.0\hsize]{/data/zender/fgr/ppr_BiZ03/fgbw/phot}%
\caption{Simulated dust-induced change in $\prc$-values.
Point~A ($10\dgr${}N, $15\dgr${}E) in this study is solid line,
and from simulation of \cite{MJL02} is dotted line.
Point~B ($18\dgr${}N, $5\dgr${}W) is dashed line.
Point~C ($14\dgr${}N, $32\dgr${}W) is dash-dot line.}
\end{figure}

\foilhead{\bgp
\Large\textcolor{blue}{\hfill \Ot\ Distribution \hfill}}\vspace{-0.5in}\large
\begin{figure}
\centering
\includegraphics*[width=1.0\hsize]{/data/zender/fgr/ppr_BiZ03/fgbw/o3}%
\caption{Climatological \Ot~[ppb] in January and July at three
  atmospheric layers simulated using gas phase chemistry alone.} 
\end{figure}

\foilhead{\bgp
\Large\textcolor{blue}{\hfill Photochemical Forcing of \Ot\ by Dust \hfill}}\vspace{-0.5in}\large
\begin{figure}
\centering
\includegraphics*[width=1.0\hsize]{/data/zender/fgr/ppr_BiZ03/fgbw/paperphoo3}%
\caption{Photolysis rate forcing of \Ot~[ppb] by dust in January and
July at three atmospheric layers.}
\end{figure}

\foilhead{\bgp
\Large\textcolor{blue}{\hfill Heterogeneous Forcing of \Ot\ by Dust \hfill}}\vspace{-0.5in}\large
\begin{figure}
\centering
\includegraphics*[width=1.0\hsize]{/data/zender/fgr/ppr_BiZ03/fgbw/heto3}%
\caption{Decrease in \Ot~[ppb] due to heterogeneous reactions on
dust in January and July at three atmospheric layers.}
\end{figure}

\rotatefoilhead{\huge \textcolor{blue}{
LGM dust distribution and radiative forcing: Sensitivity to vegetation
reconstruction}}\vspace{-0.5in}\LARGE
\begin{center}
Charles S. Zender, Mark Flanner, and Jonathan Adams\\
Dept.\ of Earth System Science, UC Irvine\\
\bigskip
\bigskip
\bigskip
INQUA, Reno, Nevada, July 23--30, 2003\\
\bigskip
{\large Thanks: S.~Levis}
\normalsize
\end{center}

\rotatefoilhead{\bgl
\Large\textcolor{blue}{\hfill Background \& Objectives \hfill}}\vspace{-0.75in}\large
\textit{Background}: 
\vspace{-0.5in}\begin{itemize*}
\item Previous research shows vegetation change required to reproduce    
observed LGM:Present deposition ratios in high latitudes
\cite[][]{MKH99,WTH02}
\item Reconstructions of LGM vegetation differ significantly 
in semi-arid regions \cite[][]{Cro95,AdF97} 
\end{itemize*}
\textit{Objectives}: 
\vspace{-0.5in}\begin{itemize*}
\item What are dust burden and climate forcing uncertainties due to  
  vegetation reconstruction?
\item How do they compare to LGM GHG, Solar climate forcing?
\item May dust simulation/evaluation inform LGM vegetation reconstruction?
\end{itemize*}

\rotatefoilhead{\bgl
\Large\textcolor{blue}{\hfill Strategy \hfill}}\vspace{-0.5in}\large
\textit{Strategy}:
Use present day and end-member $\vai$ reconstructions in mineral Dust
Entrainment And Deposition (DEAD) model
(\url{http://dust.ess.uci.edu/dead}) embedded in GCM.\\ 
\begin{itemize*}
\item Present Day: Satellite-derived $\vai$ \cite[]{KMC99,DZD03}
\item Crowley: Reconstructed $\vai$ \cite[]{Cro95}
\item Adams: Reconstructed $\vai$ \cite[]{AdF97}
\end{itemize*}
Evaluate present, LGM deposition results with DIRTMAP \cite[]{KoH01}.

\foilhead{\bgp
\Large\textcolor{blue}{\hfill Definitions \hfill}}\vspace{-0.5in}\large
\begin{eqnarray}
\flxmssvrtdstj & = & 
\mblfrc \erdfct \hrzvrtprpfct \flxmsshrzslt 
\sum_{\binidx=1}^{\srcnbr} \mssovrsrcsnk \nonumber \\
\label{eqn:flx_mss_vrt_dst_ttl}
\mblfrc & = & (1 - \lkfrc - \wtlfrc)(1 - \snwfrc)(1 - \vgtfrc) \nonumber \\
\label{eqn:mbl_frc}
\vgtfrc & = & \min[1.0,\min(\textcolor{ForestGreen}{\vai},\vaithr)/\vaithr] \nonumber
\label{eqn:vgt_frc}
\end{eqnarray}
$\flxmssvrtdstj$\,[\kgxmSs]: Vertical dust flux\\
$\mblfrc$: Fraction of ground that is erodible\\
$\erdfct$: Erodibility factor\\
$\hrzvrtprpfct$\,[\xm]: Sandblasting mass efficiency\\
$\flxmsshrzslt$\,[\kgxms]: Saltation mass flux\\ 
$\mssovrsrcsnk$: Source mode $\srcidx$ overlap with dust bin $\snkidx$\\
$\lkfrc$, $\snwfrc$, $\vgtfrc$, $\wtlfrc$: Constraint by
lakes, snow, vegetation, wetlands\\   
\textcolor{ForestGreen}{$\vai$\,[\mSxmS]: Vegetation Area Index}\\
$\vaithr = 0.3$\,\mSxmS: Threshold VAI\\

\foilhead{\bgp
\Large\textcolor{blue}{\hfill Vegetation Forcing \hfill}}\vspace{-0.5in}\large
\begin{figure}
\centering
% gcm_xy_bch,pll=4,img=0,fld_top=1,lbl_typ='auto',prn=1
\includegraphics*[width=0.7\hsize]{dstccm97_clm_VAI_DST}\vfill
\includegraphics*[width=0.7\hsize]{dstccm96_clm_VAI_DST}\vfill
\includegraphics*[width=0.7\hsize]{dstccm95_clm_VAI_DST}\vfill
\caption{
Vegetation Area Index $\vai$\,[\mSxmS] for
(a)~Present Day, (b)~Crowley LGM, (c)~Adams LGM.
\label{fgr:DSTSFMBL}}
\end{figure}

\foilhead{\bgp
\Large\textcolor{blue}{\hfill Present Day Deposition\hfill}}\vspace{-0.5in}\large
\begin{figure}
\centering
\includegraphics*[width=\hsize]{dirtmap_dstccm97_DSTSFDPS}\vfill
\caption{
Present day DIRTMAP2 \cite[]{KoH01} ($x$-axis) vs.\ simulated
($y$-axis) mass deposition flux $\flxmssvrtdst$\,[\gxmSyr].
\label{fgr:DSTSFMBL}}
\end{figure}

\rotatefoilhead{\bgl
\Large\textcolor{blue}{\hfill LGM Dust \& Loess Deposition\hfill}}\vspace{-0.5in}\large
\begin{figure}
\centering
\includegraphics*[width=0.5\hsize]{dirtmap_dstccm96_DSTSFDPS}%
\includegraphics*[width=0.5\hsize]{dirtmap_dstccm95_DSTSFDPS}
\caption{
LGM DIRTMAP2 \cite[]{KoH01} ($x$-axis) vs.\ simulated
($y$-axis) mass deposition flux $\flxmssvrtdst$\,[\gxmSyr]
for two vegetation reconstructions (a)~Crowley and (b)~Adams. 
\label{fgr:DSTSFMBL}}
\end{figure}

\rotatefoilhead{\bgl
\Large\textcolor{blue}{\hfill LGM:Present Deposition Ratio\hfill}}\vspace{-0.5in}\large
\begin{figure}
\centering
\includegraphics*[width=0.5\hsize]{dirtmap_dstccm96_DSTSFDPS_rat}%
\includegraphics*[width=0.5\hsize]{dirtmap_dstccm95_DSTSFDPS_rat}
\caption{
Ratio of LGM to present mass deposition flux from DIRTMAP2
\cite[]{KoH01} ($x$-axis) and two vegetation reconstructions 
($y$-axis) (a)~Crowley and (b)~Adams. 
\label{fgr:DSTSFMBL}}
\end{figure}

\rotatefoilhead{\bgl
\Large\textcolor{blue}{\hfill Present Day Seasonal Optical Depth \hfill}}\vspace{-0.5in}\large
\begin{figure}
\centering
% ${HOME}/idl/gcm.pro:gcm_xy_bch,img=0,pll=4,csn=1,lbl_typ='auto',prn=0
\includegraphics[width=0.5\hsize]{dstccm97_clm_1202_DSTODXC}%
\includegraphics[width=0.5\hsize]{dstccm97_clm_0608_DSTODXC}%

\includegraphics[width=0.5\hsize]{dstccm97_clm_0305_DSTODXC}%
\includegraphics[width=0.5\hsize]{dstccm97_clm_0911_DSTODXC}%
\caption{
Predicted seasonal mean dust optical depth at 0.63\,\um\ for (a)~DJF, 
(b)~MAM, (c)~JJA, and (d)~SON.
\label{fgr:dstodxc_csn}}
\end{figure}

\foilhead{\bgp
\Large\textcolor{blue}{\hfill LGM Annual Mean Optical Depth \hfill}}\vspace{-0.5in}\large
\begin{figure}
\centering
% ${HOME}/idl/gcm.pro:gcm_xy_bch,img=0,pll=4,fld_top=1,lbl_typ='auto',prn=0
\includegraphics[width=0.7\hsize]{dstccm97_clm_DSTODXC}\vfill
\includegraphics[width=0.7\hsize]{dstccm96_clm_DSTODXC}\vfill
\includegraphics[width=0.7\hsize]{dstccm95_clm_DSTODXC}
\caption{
Predicted annual mean dust optical depth at 0.63\,\um\ for
(a)~Present Day, (b)~Crowley LGM, (c)~Adams LGM.
\label{fgr:dstodxc}}
\end{figure}

\foilhead{\bgp
\Large\textcolor{blue}{\hfill Solar Radiative Forcing \& Response \hfill}}\vspace{-0.5in}\large
\begin{figure}
\centering
% ${HOME}/idl/gcm.pro:gcm_xy_bch,img=0,pll=4,fld_top=1,pnl_lbl='!5(a)',prn=0
\includegraphics[width=0.7\hsize]{dstccm94_clm_FSNT}\vfill
% ${HOME}/idl/gcm.pro:gcm_xy_bch,img=0,pll=4,fld_top=1,pnl_lbl='!5(b)',prn=0
\includegraphics[width=0.7\hsize]{dstccm94_clm_FSNTFRC}\vfill
% ${HOME}/idl/gcm.pro:gcm_xy_bch,img=0,pll=4,fld_top=1,pnl_lbl='!5(c)',prn=0,diff=1
\includegraphics[width=0.7\hsize]{dstccm97_clm_dstccm94_clm_FSNT}\vfill
\caption{
Predicted annual mean present day 
(a)~TOA net solar flux $\flxnetswtoa$, 
(b)~Dust forcing of $\flxnetswtoa$, 
(c)~Response of $\flxnetswtoa$ to dust.
\label{fgr:dstodxc}}
\end{figure}

\foilhead{\bgp
\Large\textcolor{blue}{\hfill Total Radiative Forcing \& Response \hfill}}\vspace{-0.5in}\large
\begin{figure}
\centering
% ${HOME}/idl/gcm.pro:gcm_xy_bch,img=0,pll=4,fld_top=1,pnl_lbl='!5(a)',prn=0
\includegraphics[width=0.7\hsize]{dstccm94_clm_FTNT}\vfill
% ${HOME}/idl/gcm.pro:gcm_xy_bch,img=0,pll=4,fld_top=1,pnl_lbl='!5(b)',prn=0
\includegraphics[width=0.7\hsize]{dstccm94_clm_FTNTFRC}\vfill
% ${HOME}/idl/gcm.pro:gcm_xy_bch,img=0,pll=4,fld_top=1,pnl_lbl='!5(c)',prn=0,diff=1
\includegraphics[width=0.7\hsize]{dstccm97_clm_dstccm94_clm_FTNT}\vfill
\caption{
Predicted annual mean present day 
(a)~TOA net radiative flux $\flxnetrdntoa$, 
(b)~Dust forcing of $\flxnetrdntoa$, 
(c)~Response of $\flxnetrdntoa$ to dust.
\label{fgr:dstodxc}}
\end{figure}

\foilhead{\bgp
\Large\textcolor{blue}{\hfill Sensitivity of $\flxnetswtoa$ to Vegetation \hfill}}\vspace{-0.5in}\large
\begin{figure}
\centering
% ${HOME}/idl/gcm.pro:gcm_xy_bch,img=0,pll=4,fld_top=1,lbl_typ='auto',prn=0
\includegraphics[width=0.7\hsize]{dstccm94_clm_FSNTFRC}\vfill
\includegraphics[width=0.7\hsize]{dstccm93_clm_FSNTFRC}\vfill
\includegraphics[width=0.7\hsize]{dstccm92_clm_FSNTFRC}
\caption{
Predicted annual mean TOA net solar flux forcing by dust
$\dltflxnetswtoa$ for three vegetation scenarios:
(a)~Present Day, (b)~Crowley LGM, (c)~Adams LGM.
\label{fgr:dstodxc}}
\end{figure}

\rotatefoilhead{}
\begin{table*}
\begin{minipage}{\hsize}
\renewcommand{\footnoterule}{\rule{\hsize}{0.0cm}\vspace{-0.0cm}} % KoD95 p. 111
\centering % \centering uses less vertical space than center-environment
\caption[Evaluation of Simulations vs.\ DIRTMAP2]{\textbf{Evaluation of Simulations vs.\ DIRTMAP2}
\label{tbl:stt}}
\vspace{\cpthdrhlnskp}
\begin{tabular}{l *{4}{>{$}r<{$}}}
\hline \rule{0.0ex}{\hlntblhdrskp}%
\mbox{Correlation} & \mbox{Present} & \mbox{Crowley} & \mbox{Adams} & \\[0.0ex]
\hline \rule{0.0ex}{\hlntblntrskp}%
& & & & \\[-2.0ex]
\multicolumn{5}{l}{Correlation of simulated deposition flux
  $\flxmssvrtdst$ with DIRTMAP} \\[0.0ex]
% data from dstccm95,96,97,94
Linear $\crrcff$ % \footnote{Correlation coefficient}
& 0.31 & 0.44 & \color{red}{0.47} & \\[0.0ex]
Logarithmic $\crrcff$ % \footnote{Correlation coefficient}
& 0.88 & 0.87 & \color{red}{0.89} & \\[0.0ex]
\multicolumn{5}{l}{Slope of simulated deposition flux vs.\ DIRTMAP} \\[0.0ex]
% data from dstccm95,96,97,94
Linear $\crrcff$ % \footnote{Correlation coefficient}
& 0.09 & \color{red}{0.05} & 0.04 & \\[0.0ex]
Logarithmic $\crrcff$ % \footnote{Correlation coefficient}
& 0.65 & \color{red}{0.67} & 0.60 & \\[0.0ex]
\multicolumn{5}{l}{Correlation of simulated LGM:Present flux ratio with DIRTMAP} \\[0.0ex]
% data from dstccm95,96,97,94
Linear $\crrcff$ % \footnote{Correlation coefficient}
& \mbox{NA} & 0.08 & \color{red}{0.21} & \\[0.0ex]
Logarithmic $\crrcff$ % \footnote{Correlation coefficient}
& \mbox{NA} & 0.14 & \color{red}{0.19} & \\[0.0ex]
\hline
\end{tabular}
\end{minipage}
\end{table*} % end tbl:stt

\rotatefoilhead{}
\begin{table*}
\begin{minipage}{\hsize}
\renewcommand{\footnoterule}{\rule{\hsize}{0.0cm}\vspace{-0.0cm}} % KoD95 p. 111
\centering % \centering uses less vertical space than center-environment
\caption[Sensitivity of Budget, Forcing, to Vegetation]{\textbf{Sensitivity of Budget, Forcing to Vegetation}
%\caption[Sensitivity of Budget, Forcing, Response, to Vegetation]{\textbf{Sensitivity of Budget, Forcing, Response, to Vegetation}
\label{tbl:sns}}
\vspace{\cpthdrhlnskp}
\begin{tabular}{l *{4}{>{$}r<{$}}}
\hline \rule{0.0ex}{\hlntblhdrskp}%
\mbox{Quantity} & \mbox{Present} & \mbox{Crowley} & \mbox{Adams} & \mbox{Range} \\[0.0ex]
\hline \rule{0.0ex}{\hlntblntrskp}%
% data from dstccm95,96,97,94
Emission\,[\Tgxyr] % \footnote{Emissions of $\dmt < 10$\,\um\ particles in \Tgxyr}
& 1.8 & 3.6 & 6.4 & \pm 2.8 \\[0.0ex]
Lifetime\,[day] % \footnote{Turnover time in days}
& 3.6 & 3.9 & 3.3 & \pm 0.6 \\[0.0ex]
Burden\,[\Tg] % \footnote{Atmospheric burden in Tg}
& 17 & 38 & 57 & \pm 19 \\[0.0ex]
\tauext % \footnote{Optical depth at 0.63\,\um}
& 0.030 & 0.065 & 0.096 & \pm 0.03 \\[0.0ex]
\dltflxnetrdnsfc\,[\wxmS] % \footnote{Net radiative flux forcing at surface}
& -1.8 & -3.9 & -5.7 & \pm 1.8 \\[0.0ex]
\dltflxnetrdnatm\,[\wxmS] % \footnote{Net radiative flux forcing of atmosphere}
& 1.5 & 3.3 & 5.0 & \pm 1.7 \\[0.0ex]
\dltflxnetrdntoa\,[\wxmS] % \footnote{Net radiative flux forcing at TOA}
& -0.3 & -0.5 & -0.7 & \pm 0.2 \\[0.0ex]
\csznote{
\ensuremath{\delta\flxnetrdnsfc}\,[\wxmS] % \footnote{Net radiative flux response at surface}
& -1.5 & -3.0 & -5.0 & \pm 2.0 \\[0.0ex]
\ensuremath{\delta\flxnetrdnatm}\,[\wxmS] % \footnote{Net radiative flux response of atmosphere}
& 1.0 & 2.3 & 3.8 & \pm 1.5 \\[0.0ex]
\ensuremath{\delta\flxnetrdntoa}\,[\wxmS] % \footnote{Net radiative flux response at TOA}
& -0.4 & -0.7 & -1.1 & \pm 0.4 \\[0.0ex]
} % csznote
\hline
\end{tabular}
\end{minipage}
\end{table*} % end tbl:sns

\rotatefoilhead{\bgl
\Large\textcolor{blue}{\hfill Conclusions \hfill}}\vspace{-0.5in}\large
\begin{itemize*}
\item Uncertainty due to LGM vegetation reconstruction:
  Emissions $2.8$\,\Tgxyr\ ($\sim 60\%$), 
  Dust burden $19$\,\Tg\ ($\sim 40\%$), 
  Surface radiative forcing $1.8$\,\wxmS\ ($\sim 40\%$),
  TOA radiative forcing $0.2$\,\wxmS\ ($\sim 30\%$)
\item Surface radiative forcing uncertainty comparable to LGM GHG
  forcing of $2.5$\,\wxmS\ \cite[]{SLO03} 
\item Uncertainties in size distribution, transport, and scavenging 
  preclude definitive conclusions about source regions.
  Preliminary results suggest less Asian vegetation improves
  simulation of Asian and N.~Pacific dust deposition.
\end{itemize*}

\rotatefoilhead{\huge \textcolor{blue}{
Spatial Heterogeneity in Aeolian Erodibility: Uniform, Topographic, Geomorphic, and Hydrologic Hypotheses}}\vspace{-0.5in}\LARGE  
\begin{center}
Charles S. Zender and David J. Newman\\
Dept.\ of Earth System Science, UC Irvine\\
\bigskip
\bigskip
\bigskip
University of Oslo, January 16, 2003\\
\bigskip
{\large Thanks: J.~Famiglietti, P.~Ginoux, S.~Levis, D.~Savoie, O.~Torres}
\normalsize
\end{center}

\rotatefoilhead{\bgl
\Large\textcolor{blue}{\hfill Objectives \hfill}}\vspace{-0.5in}\large
\textit{Objectives}: 
\begin{itemize*}
\item Demonstrate/understand spatial heterogeneity of soil erodibility
\item Constrain natural sources of dust, increase confidence 
in identifying anthropogenic emissions 
\end{itemize*}

\rotatefoilhead{\bgl
\Large\textcolor{blue}{\hfill Definitions \hfill}}\vspace{-0.5in}\large
\textit{Definitions}:
Soil aeolian erodibility $\erdfct$ is efficiency with which soil
produces dust for given meteorological forcing.
$\erdfct$ is dimensionless factor appearing in formulation of dust
mobilization mass flux $\flxmssvrtdstj$\,[\kgxmSs] 
%\cite[]{ZBN03}
\begin{eqnarray}
\flxmssvrtdstj & = & 
\mblfrc \erdfct \hrzvrtprpfct \flxmsshrzslt 
\sum_{\binidx=1}^{\srcnbr} \mssovrsrcsnk 
\label{eqn:flx_mss_vrt_dst_ttl}
\end{eqnarray}
where $\mblfrc$ is fraction of ground that is erodible,
$\hrzvrtprpfct$\,[\xm] is sandblasting mass efficiency,
$\flxmsshrzslt$\,[\kgxms] is saltation mass flux, and
$\mssovrsrcsnk$ is mass fraction of source mode $\srcidx$
carried in transport bin $\snkidx$.
Visit \url{http://dust.ess.uci.edu/dead} for more details.

\rotatefoilhead{\bgl
\Large\textcolor{blue}{\hfill Strategy \hfill}}\vspace{-0.5in}\large
\textit{Strategy}:
Compare performance of four different $\erdfct$
hypotheses in mineral Dust Entrainment And Deposition (DEAD) model
\begin{itemize*}
\item Uniform: globally uniform $\erdfct$. 
  Emissions determined by instantaneous local meteorology, vegetation,
  soil moisture.   
\item Topographic: $\erdfct$ characterized by relative elevation in basin \cite[]{GCT01} 
\item Geomorphic: $\erdfct$ characterized by upstream area.
\item Hydrologic: $\erdfct$ characterized by present day surface runoff
\end{itemize*}

\rotatefoilhead{\bgl
\Large\textcolor{blue}{\hfill Geomorphic Basins \hfill}}\vspace{-0.5in}\large
\begin{figure}[!h]
\centering % \centering uses less vertical space than center-environment
\includegraphics*[width=\hsize,angle=0]{bds_bsn_T62}\vfill
\caption{
Distinct geomorphic basins are determined by backfilling DEM algorithm 
\cite[]{JeD88}, modified to preserve true internal drainage basins
\cite[]{Asa00}. 
\label{fgr:bds_bsn_T62}}
\end{figure}

\rotatefoilhead{\bgl
\Large\textcolor{blue}{\hfill Soil Erodibility \hfill}}\vspace{-0.5in}\large
\begin{figure}[!h]
\centering % \centering uses less vertical space than center-environment
\includegraphics*[height=\hsize,angle=90]{bds_erd_fct_T62}\vfill
\caption{
Soil erodibility $\erdfct$ for four hypotheses:
(a)~Uniform, (b)~Topographic, 
(c)~Geomorphic, (d)~Hydrologic.
\label{fgr:mbl_bsn_fct}}
\end{figure}

\rotatefoilhead{\bgl
\Large\textcolor{blue}{\hfill Methods \hfill}}\vspace{-0.5in}\large
Four three-year DEAD simulations driven by NCEP 1990--1992 meteorology  
\csznote{\cite[]{Kal96}} in MATCH CTM \cite[]{RME97} 
(T62/L28, $\Delta \tm = 40$~min.).
Objectively evaluate hypotheses using three independent datasets: 
\begin{itemize*}
\item U.~Miami seasonal \& annual mean dust concentration
\item Annual mean deposition at 11 stations \cite[]{GCT01}
\item Spatial correlation with TOMS AAI/AOD \cite[]{TBH98}
\end{itemize*}
For DEAD description and evaluation of 1990s climatology:
\texttt{\normalsize{ftp://dust.ess.uci.edu/pub/zender/ppr/ppr\_ZBN03.pdf}}

\rotatefoilhead{\bgl
\Large\textcolor{blue}{\hfill Mobilization \hfill}}\vspace{-0.5in}\large
\begin{figure}
\centering % \centering uses less vertical space than center-environment
\includegraphics*[height=\hsize,angle=90]{lubbock_DSTSFMBL}\vfill
\caption{
Predicted annual mean mass emissions flux [\ugxmSs] for
(a)~Uniform $\erdfct$, (b)~Topographic $\erdfct$,
(c)~Geomorphic $\erdfct$, (d)~Hydrologic $\erdfct$.
Scale is logarithmic.
\label{fgr:DSTSFMBL}}
\end{figure}

\rotatefoilhead{\bgl
\Large\textcolor{blue}{\hfill Column Mass Path \hfill}}\vspace{-0.5in}\large
\begin{figure}
\centering % \centering uses less vertical space than center-environment
\includegraphics*[height=\hsize,angle=90]{lubbock_DSTMPC_dff}\vfill
\caption{
Predicted annual mean dust burden [\mgxmS] for
(a)~Uniform $\erdfct$, and change relative to (a) due to
(b)~Topographic $\erdfct$, (c)~Geomorphic $\erdfct$,
(d)~Hydrologic $\erdfct$. 
Scale is non-linear.
\label{plt:DSTMPC}}
\end{figure}

\rotatefoilhead{\bgl
\Large\textcolor{blue}{\hfill Deposition \hfill}}\vspace{-0.5in}\large
\begin{figure}
\centering % \centering uses less vertical space than center-environment
\includegraphics*[height=\hsize,angle=90]{lubbock_DSTSFDPS}\vfill
\caption{
Predicted annual mean mass deposition flux [\ugxmSs] for
(a)~Uniform $\erdfct$, (b)~Topographic $\erdfct$,
(c)~Geomorphic $\erdfct$, (d)~Hydrologic $\erdfct$.
Scale is logarithmic.
\label{fgr:DSTSFDPS}}
\end{figure}

\rotatefoilhead{\bgl
\Large\textcolor{blue}{\hfill Spatial Correlation \hfill}}\vspace{-0.5in}\large
\begin{itemize}
\item Compare model-simulated AOD to the TOMS-observed AAI and AOD
in three important desert regions.
\item Data are (re-)gridded to T62 
\item Points with vegetation $\vai > 0.3$\,\mSxmS\ are removed to avoid
  contamination by biomass burning aerosol. 
\end{itemize}

\foilhead{\bgp
\Large\textcolor{blue}{\hfill Spatial Correlation \hfill}}\vspace{-0.5in}\large
\begin{figure}
\centering % \centering uses less vertical space than center-environment
\includegraphics*[width=0.33\hsize]{toms_clm_Afr_aer_idx_331_360}%
\includegraphics*[width=0.33\hsize]{toms_clm_EAs_aer_idx_331_360}%
\includegraphics*[width=0.33\hsize]{toms_clm_Aus_aer_idx_331_360}%

\includegraphics*[width=0.33\hsize]{dstmch1p_clm_Afr_DSTODXC}%
\includegraphics*[width=0.33\hsize]{dstmch1p_clm_EAs_DSTODXC}%
\includegraphics*[width=0.33\hsize]{dstmch1p_clm_Aus_DSTODXC}%

\includegraphics*[width=0.33\hsize]{dstmch1q_clm_Afr_DSTODXC}%
\includegraphics*[width=0.33\hsize]{dstmch1q_clm_EAs_DSTODXC}%
\includegraphics*[width=0.33\hsize]{dstmch1q_clm_Aus_DSTODXC}%

\includegraphics*[width=0.33\hsize]{dstmch1t_clm_Afr_DSTODXC}%
\includegraphics*[width=0.33\hsize]{dstmch1t_clm_EAs_DSTODXC}%
\includegraphics*[width=0.33\hsize]{dstmch1t_clm_Aus_DSTODXC}%

\includegraphics*[width=0.33\hsize]{dstmch1r_clm_Afr_DSTODXC}%
\includegraphics*[width=0.33\hsize]{dstmch1r_clm_EAs_DSTODXC}%
\includegraphics*[width=0.33\hsize]{dstmch1r_clm_Aus_DSTODXC}%
\caption{
Mean 1980--2001 TOMS AAI and predicted dust AOD for 1990--1992 in the
Saharan desert and Saudi Arabian peninsula, East Asia, and Australia.
(a)~TOMS AAI, (b)~Uniform $\erdfct$, (c)~Topographic $\erdfct$, 
(d)~Geomorphic $\erdfct$, (e)~Hydrologic $\erdfct$.
$\crrcff$ is the spatial correlation coefficient between the
experiment and TOMS. 
\label{fgr:stt_odx}}
\end{figure}

\rotatefoilhead{\bgl
\Large\textcolor{blue}{\hfill Spatial Correlations \hfill}}\vspace{-0.5in}\large
\begin{table*}
\begin{minipage}{\hsize}
\renewcommand{\footnoterule}{\rule{\hsize}{0.0cm}\vspace{-0.0cm}} % KoD95 p. 111
\centering % \centering uses less vertical space than center-environment
\caption[Spatial Correlation of Model AOD with TOMS AAI and
AOD]{\textbf{Spatial Correlation of Model AOD with TOMS AAI and AOD}%
%\footnote{Number $\NNN$ of valid T62 gridpoints is 290, 77, and 59 for
%  Saharan, East Asian, and Australian regions, respectively.
%Highest $\crrcff$ for TOMS AAI in each region is boldfaced.}
\label{tbl:stt_odx}}%
\vspace{\cpthdrhlnskp}
\begin{tabular}{l *{9}{>{$}r<{$}}}
\hline \rule{0.0ex}{\hlntblhdrskp}%
& \multicolumn{2}{c}{Sahara} & & \multicolumn{2}{c}{East Asia} & & \multicolumn{2}{c}{Australia} & \\[0.0ex] 
\cline{2-3}\cline{5-6}\cline{8-9}
Hypothesis & \mathrm{AAI} & \mathrm{AOD} & & \mathrm{AAI} & \mathrm{AOD}& & \mathrm{AAI} & \mathrm{AOD} & \\[0.5ex]
\hline \rule{0.0ex}{\hlntblntrskp}%
% Data from dstmch1q-dstmch1t
Uniform & 0.55 & 0.18 & & 0.81 & 0.78 & & 0.12 & 0.27 & \\[0.5ex]
Topographic & \mathbf{0.69} & 0.34 & & 0.76 & 0.70 & & 0.64 & 0.55 & \\[0.5ex]
Geomorphic & 0.65 & 0.37 & & \mathbf{0.91} & 0.77 & & 0.57 & 0.32 & \\[0.5ex]
Hydrologic & 0.59 & 0.43 & & 0.71 & 0.67 & & \mathbf{0.70} & 0.51 & \\[0.5ex]
\hline
\end{tabular}
\end{minipage}
\end{table*} % end tbl:stt_odx

\rotatefoilhead{\bgl
\Large\textcolor{blue}{\hfill Budgets \hfill}}\vspace{-0.5in}\large
Compare annual mean dust budget of control to three experiments. 
Mass path is prescribed, so lifecycle varies with proximity of
sources to regions of wet deposition. 
Note 15\% change in emissions and turnover time.
\begin{table*}
\begin{minipage}{\hsize}
\renewcommand{\footnoterule}{\rule{\hsize}{0.0cm}\vspace{-0.0cm}} % KoD95 p. 111
\caption[Climatological Budget Statistics]{\textbf{Climatological Budget Statistics}
\label{tbl:bdg}}
\vspace{\cpthdrhlnskp}
\begin{tabular}{l *{4}{>{$}r<{$}}}
\hline \rule{0.0ex}{\hlntblhdrskp}%
Quantity & \mbox{Uniform} & \mbox{Topo} & \mbox{Geo} & \mbox{Hydro} \\[0.0ex]
\hline \rule{0.0ex}{\hlntblntrskp}%
% data from dstmch1q--dstmch1t
Burden%\footnote{Prescribed atmospheric burden in Tg}
& 19.3   & 19.3   & 19.3   & 19.3   \\[0.5ex]
Emission%\footnote{Emissions of $\dmt < 10$\,\um\ particles in \Tgxyr}
& \color{Red}{1438}   & 1535   & \color{Red}{1638}   & 1632   \\[0.5ex]
Lifetime%\footnote{Turnover time in days}
& \color{Red}{5.0} & 4.7    & \color{Red}{4.4} & 4.4    \\[0.5ex]
Global $\tau$%\footnote{Optical depth at 0.63\,\um}
& 0.033  & 0.034  & 0.034  & 0.034  \\[0.5ex]
\hline
\end{tabular}
\end{minipage}
\end{table*} % end tbl:bdg

\rotatefoilhead{\bgl
\Large\textcolor{blue}{\hfill Hypothesis Evaluation with Station
  Observations: \hfill}}\vspace{-0.5in}\large
\begin{figure}
\centering % \centering uses less vertical space than center-environment
\includegraphics[width=0.5\hsize]{cnc_lubbock}%
\includegraphics[width=0.5\hsize]{dep_lubbock}
\caption{
(a) Predicted and observed climatological mean surface concentration of
dust in \ugxmC\ at 19 U.~Miami stations.
Dashed lines indicate factor of two disparity.
(b). Predicted and observed climatological mean dust deposition flux 
in \gxmSyr\ at 11 stations compiled by \cite{GCT01}.
Axes are logarithmic.

Large inter-model spread (\%) occurs for stations in and
downwind of source regions, including Sal Island (57\%) and Cape
Point (2500\%), Kaashidhoo (50\%), Jeju (310\%) and Okinawa (280\%).
Surface observations of concentration and deposition show minimal
inter-model spread at most non-dusty stations. 
\label{fgr:cnc_sct}}
\end{figure}

\rotatefoilhead{\bgl
\Large\textcolor{blue}{\hfill Linear and Logarithmic Statistics \hfill}}\vspace{-0.5in}\large
Linear statistics preferentially weight dusty stations, so we compute
logarithmic statistics too.
Absolute and relative RMS biases $\rmsabs$ and $\rmsrlt$,
respectively, computed using
\begin{eqnarray}
\rmsabs = \sqrt{ \frac{1}{\NNN} 
\sum_{\iii = 1}^{\NNN} 
( \xxx_{\iii} - \yyy_{\iii} )^{2} } \\
\rmsrlt = \sqrt{ \frac{1}{\NNN} 
\sum_{\iii = 1}^{\NNN} 
\left( \frac{\xxx_{\iii} - \yyy_{\iii}}{\xxx_{\iii}} \right)^{2} }
\label{eqn:rms_abs}
%\label{eqn:rms_rlt}
\end{eqnarray}

\rotatefoilhead{
\Large\textcolor{blue}{\hfill Statistics \hfill}}\vspace{-0.5in}\large
Geomorphic Hypothesis best explains station observations:
\begin{scriptsize}
\begin{table*}
\begin{minipage}{\hsize}
\renewcommand{\footnoterule}{\rule{\hsize}{0.0cm}\vspace{-0.0cm}} % KoD95 p. 111
\centering % \centering uses less vertical space than center-environment
\caption[Annual Mean Concentration at 19 U.~Miami
Stations]{\textbf{Biases in Annual Mean Concentration
at 19 U.~Miami stations}%
%\footnote{Value of best metric in each category is boldfaced.}
\label{tbl:stt_cnc}}
\vspace{\cpthdrhlnskp}
\begin{tabular}{l *{12}{>{$}r<{$}}}
\hline \rule{0.0ex}{\hlntblhdrskp}%
& \multicolumn{5}{c}{Linear} & &
\multicolumn{5}{c}{Logarithmic} \\[0.0ex] 
\cline{2-6}\cline{8-12}
Hypothesis & \crrcff & \mmm & \bbb & \rmsabs & \rmsrlt & & \crrcff & \mmm & \bbb & \rmsabs & \rmsrlt \\[0.5ex]
\hline \rule{0.0ex}{\hlntblntrskp}%
% Data from dstmch1p-dstmch1t
Uniform & 0.91   & 2.04   & -4.01  & 16.06  & 0.88   & & 0.84   & \mathbf{1.00}   & -0.11  & 0.52   & 2.12    \\[0.5ex]
Topographic  & 0.96   & 1.58   & -2.30  & 8.60   & 0.85   & & 0.78   & 1.03   & -0.20  & 0.68   & 2.80    \\[0.5ex]
Geomorphic   & 0.94   & 1.74   & -2.96  & 11.28  & 0.83   & & \mathbf{0.91}   & 0.93   & \mathbf{0.00}  & \mathbf{0.33}   & \mathbf{1.35}    \\[0.5ex]
Hydrologic & \mathbf{0.98}   & \mathbf{1.47}   & \mathbf{-1.97}  & \mathbf{6.71}   & \mathbf{0.81}   & & 0.81   & 1.03   & -0.18  & 0.60   & 2.46    \\[0.5ex]
\hline
\end{tabular}
\end{minipage}
%} % end csznote

%\csznote{
%Table~\ref{tbl:stt_dps} summarizes the statistical correlations of
%each experiment with the 11 stations in the observed climatology
%shown in Figure~\ref{fgr:dps_sct}. 
\begin{minipage}{\hsize}
\renewcommand{\footnoterule}{\rule{\hsize}{0.0cm}\vspace{-0.0cm}} % KoD95 p. 111
\centering % \centering uses less vertical space than center-environment
\caption[Biases in Simulation of Annual Mean Deposition at 11 Stations]{\textbf{Biases in Simulation of Annual Mean Deposition at 11 stations}%
%\footnote{Value of best metric in each category is boldfaced.}
\label{tbl:stt_dps}}
\vspace{\cpthdrhlnskp}
\begin{tabular}{l *{12}{>{$}r<{$}}}
\hline \rule{0.0ex}{\hlntblhdrskp}%
& \multicolumn{5}{c}{Linear} & &
\multicolumn{5}{c}{Logarithmic} \\[0.0ex] 
\cline{2-6}\cline{8-12}
Hypothesis & \crrcff & \mmm & \bbb & \rmsabs & \rmsrlt & & \crrcff & \mmm & \bbb & \rmsabs & \rmsrlt \\[0.5ex]
\hline \rule{0.0ex}{\hlntblntrskp}%
% Data from dstmch1q-dstmch1t
Uniform & 0.97   & 0.07   & 2.06   & 125.34 & 0.84   & & 0.93   & 0.77   & -0.15  & 0.45   & 0.78    \\[0.5ex]
Topographic  & \mathbf{1.00}   & 0.21   & 1.35   & 107.52 & 0.74   & & 0.95   & 0.85   & \mathbf{-0.14}  & 0.37   & 0.73    \\[0.5ex]
Geomorphic   & \mathbf{1.00}   & \mathbf{0.42}   & \mathbf{0.26}   & \mathbf{78.66}  & \mathbf{0.66}   & & \mathbf{0.96}   & \mathbf{0.90}   & \mathbf{-0.14}  & \mathbf{0.32}   & \mathbf{0.71}    \\[0.5ex]
Hydrologic & \mathbf{1.00}   & 0.13   & 0.97   & 117.44 & 0.68   & & 0.94   & 0.78   & -0.21  & 0.45   & 0.78    \\[0.5ex]
\hline
\end{tabular}
\end{minipage}
\end{table*}
\end{scriptsize}
%} % end csznote

\rotatefoilhead{\bgl
\Large\textcolor{blue}{\hfill Conclusions \hfill}}\vspace{-0.5in}\large
\begin{itemize*}
\item All three spatially varying erodibility hypotheses produce
significantly better agreement with station and satellite data than 
the null (Uniform) hypothesis.
\item The Geomorphic erodibility hypothesis performs best overall,
but results vary by region and by metric.
\item These results support the hypothesis that dust emission ``hot spots''
exist in regions where alluvial sediments have accumulated and may be
disturbed \cite[]{PGT02}. 
\item Our physically-based erodibility hypotheses help explain dust
observations in some regions, particularly East Asia, and can 
be used to help discriminate between natural and anthropogenic soil 
emissions.  
\end{itemize*}

\rotatefoilhead{\bgl
\Large\textcolor{blue}{\hfill Future Work \hfill}}\vspace{-0.5in}\large
\textit{Linking erodibility $\erdfct$ to regional
  geomorphology/hydrology helps constrain natural sources of dust, and
  thus increases confidence in anthropogenic emissions identified using residual methods}.
\begin{itemize*}
\item Examine Hydrologic hypothesis with wet period (mid-Holocene) runoff
\item Use TOMS/MODIS/AMSR-E to determine empirical $\erdfct$, subtract
  natural $\erdfct$ to obtain anthropogenic $\erdfct$ 
\item Apply to regional forecasts of Columbia Plateau, Aral Sea
\end{itemize*}

\foilhead{\vspace{0.25in}\Large\textcolor{blue}{\hfill
Performance with Area $\erdfct$ \hfill}}\vspace{-0.5in}
\begin{figure*}
\centering % \centering uses less vertical space than center-environment
% ${HOME}/idl/gcm.pro:gcm_tv_bch,rsmas=1,psym=[8,0],shd=1,stp=1,bw=1,info=1,prn=0
\includegraphics*[width=0.33\hsize]{rsmas_dstmch90_clm_0112_Brb_cnc_mss_dst}%
\includegraphics*[width=0.33\hsize]{rsmas_dstmch90_clm_0112_Mmi_cnc_mss_dst}%
\includegraphics*[width=0.33\hsize]{rsmas_dstmch90_clm_0112_Brm_cnc_mss_dst}%

\includegraphics*[width=0.33\hsize]{rsmas_dstmch90_clm_0112_Izn_cnc_mss_dst}%
\includegraphics*[width=0.33\hsize]{rsmas_dstmch90_clm_0112_SlI_cnc_mss_dst}%
\includegraphics*[width=0.33\hsize]{rsmas_dstmch90_clm_0112_McH_cnc_mss_dst}%

\includegraphics*[width=0.33\hsize]{rsmas_dstmch90_clm_0112_KGI_cnc_mss_dst}%
\includegraphics*[width=0.33\hsize]{rsmas_dstmch90_clm_0112_CGH_cnc_mss_dst}%
\includegraphics*[width=0.33\hsize]{rsmas_dstmch90_clm_0112_Kaa_cnc_mss_dst}%

\includegraphics*[width=0.33\hsize]{rsmas_dstmch90_clm_0112_Jej_cnc_mss_dst}%
\includegraphics*[width=0.33\hsize]{rsmas_dstmch90_clm_0112_Okn_cnc_mss_dst}%
\includegraphics*[width=0.33\hsize]{rsmas_dstmch90_clm_0112_Enw_cnc_mss_dst}%

\includegraphics*[width=0.33\hsize]{rsmas_dstmch90_clm_0112_Mdw_cnc_mss_dst}%
%\includegraphics*[width=0.33\hsize]{rsmas_dstmch90_clm_0112_Oah_cnc_mss_dst}%
\includegraphics*[width=0.33\hsize]{rsmas_dstmch90_clm_0112_Nau_cnc_mss_dst}%
\includegraphics*[width=0.33\hsize]{rsmas_dstmch90_clm_0112_AmS_cnc_mss_dst}%

\includegraphics*[width=0.33\hsize]{rsmas_dstmch90_clm_0112_NeC_cnc_mss_dst}%
\includegraphics*[width=0.33\hsize]{rsmas_dstmch90_clm_0112_Nrf_cnc_mss_dst}%
\includegraphics*[width=0.33\hsize]{rsmas_dstmch90_clm_0112_CGr_cnc_mss_dst}%
\caption{
Predicted (line) and observed (circles) monthly mean surface
dust concentration (\ugxmC) at U.~Miami stations.
\label{fgr:rsmas}}
\end{figure*}

\rotatefoilhead{\huge \textcolor{blue}{
Understanding the Global Distribution and Radiative Forcing of Mineral
Dust Aerosol}}\vspace{-0.5in}\LARGE 
\begin{center}
Charles S. Zender\\
Dept.\ of Earth System Science, UC Irvine\\
\bigskip
\bigskip
\bigskip
Dept. of Earth Sciences Seminar\\
University of Southern California, December 3, 2001\\
\bigskip
{\large Thanks: P.~Ginoux, S.~Levis, P.~Rasch, D.~Savoie}
\normalsize
\end{center}

\rotatefoilhead{\bgl
\Large\textcolor{blue}{\hfill Problems in Assessing Dust Radiative Forcing \hfill}}\vspace{-0.5in}\large
\begin{itemize*}
\item Distribution (size, space, time)
\begin{itemize*}
\item Elements of variability: wind speed, vegetative, topographic,
hydrologic, pedalogic constraints
\end{itemize*}
\item Optical Properties of Dry Component
\begin{itemize*}
\item Dust is inhomogeneous mixture of minerals, salts, OM
\end{itemize*}
\item Composition (multi-component aerosols)
\begin{itemize*}
\item Atmospheric processing (``aging'') may coat, deliquesce dust
\end{itemize*}
\item Indirect Effects
\begin{itemize*}
\item Dust particles can serve as CCN/IN
\end{itemize*}
\end{itemize*}
Summarized in \jgr\ special issue August 2001 \cite[]{SWB01}

\rotatefoilhead{\bgl
\Large\textcolor{blue}{\hfill Distribution: Observations \hfill}}\vspace{-0.5in}\large
\begin{itemize}
\item Size: \bluestar Filters \cite[]{MSI00}, \bluestar AERONET \cite[]{Hol98}
\item Composition: Filters, Mass Spec (ATOFMS)
\item Space: \bluestar TOMS \cite[]{HBT97},
\bluestar AVHRR \cite[]{SIS97}, Meteosat \cite[]{KPR99} 
\item Vertical: LIDAR (LITE, ACE Asia)
\item Plume Evolution: ACE Asia
\end{itemize}
\bluestar Presented in this talk

\rotatefoilhead{\bgl\Large\textcolor{blue}{\hfill Modeling Strategies \hfill}}\large\vspace{-0.5in}
\fcolorbox{SkyBlue}{white}{\parbox{\textwidth}{
\begin{table}
\centering % \centering uses less vertical space than center-environment
\begin{tabular}{l >{\raggedright}p{20.0em}<{} r}
Model & Purpose & \\[4pt]
\tableline
\hline \rule{0.0ex}{\hlntblntrskp}%
Box & \bluestar Test entrainment theory vs.\ wind-tunnel & \\[4pt]
& Lagrangian trajectories, MCA optical properties & \\[4pt]
1-D & \bluestar Radiative Forcing, Radiance Inversion & \\[4pt]
RTM & Event Simulation, SS Processes & \\[4pt]
& Air Pollution Modeling & \\[4pt]
GTM & \bluestar Event Simulation, \bluestar Source/Sink Budgets & \\[4pt] 
& \bluestar Assimilation, Offline radiative forcing, BGC & \\[4pt]
AGCM & \bluestar Radiative Forcing, \bluestar Atm.\ Feedback & \\[4pt]
CSM & Climate feedback, Coupled Biogeochemistry & \\[4pt]
\end{tabular}
\caption{\bluestar Shown today\label{tbl:mdl}}
\end{table}
}} % end \fcolorbox
\clearpage

\rotatefoilhead{\bgl\Large\textcolor{blue}{\hfill Global Mean Dust Budget Statistics \hfill}}\large\vspace{-0.5in}
\fcolorbox{SkyBlue}{white}{\parbox{\textwidth}{
\begin{table}
\centering % \centering uses less vertical space than center-environment
\begin{tabular}{lr *{4}{>{$}r<{$}}} % KoD95 p. 94 describes '*' notation
\hline \rule{0.0ex}{\hlntblhdrskp}% 
Quantity & Units & \mbox{Andreae} & \mbox{Tegen} &
\mbox{GOCART}\footnote{For size range $\dmt \in [0.2,12]$\,\um} & \mbox{DEAD} \\[0.0ex]
\hline \rule{0.0ex}{\hlntblntrskp}%
% Numbers from 5 years of dstmch90 simulation
% ncks -C -H -v DSTMPC,DSTODXC ${DATA}/${caseid}/${caseid}_clm_xy.nc
Emission & MT yr$^{-1}$ & 1500 & 1222 & 1814 & 1403 \\[0.5ex]
Lifetime & days & 4 & 4 & 7.1 & 4.3 \\[0.5ex]
Burden & Tg & 8.4 & 18.8 & 35.9 & 16.1 \\[0.5ex]
$\tau$ Ocean & & - & - & & 0.018 \\[0.5ex]
$\tau$ Land & & - & - & & 0.047 \\[0.5ex]
$\tau$ Global & & 0.023 & 0.033 & & 0.028 \\[0.5ex]
\hline
\end{tabular}
\end{table}
}} % end \fcolorbox
\clearpage

\rotatefoilhead{\bgl\Large\textcolor{blue}{\hfill Size Dependence of
Dust Removal Processes \hfill}}\large\vspace{-0.5in}
\fcolorbox{SkyBlue}{white}{\parbox{\textwidth}{
\begin{table}
\centering % \centering uses less vertical space than center-environment
\begin{tabular}{l *{4}{>{$}r<{$}}}
Size & \mbox{Grv} & \mbox{Dry} & \mbox{Wet} & \mbox{Total} \\[4pt]
\tableline
\hline \rule{0.0ex}{\hlntblntrskp}%
% Numbers from dstmch26 simulation of 1998
$0.1$--$1.0$\,\um & 580 & 291 & 14 & 13 \\[4pt]
$1.0$--$2.5$\,\um & 119 & 88 & 13 & 12 \\[4pt]
$2.5$--$5.0$\,\um & 28 & 5.7 & 8.5 & 3.4 \\[4pt]
$5.0$--$10.0$\,\um & 6.3 & 1.2 & 6.5 & 1.0 \\[4pt]
\hline
$0.1$--$10.0$\,\um & 66 & 18 & 12 & 7 \\[4pt]
% % Numbers from dstmch57 simulation of 1998
% $0.1$--$1.0$\,\um & 1054 & 262 & 17 & 16 \\[4pt]
% $1.0$--$2.5$\,\um & 201 & 109 & 15 & 13 \\[4pt]
% $2.5$--$5.0$\,\um & 33 & 25 & 8 & 6 \\[4pt]
% $5.0$--$10.0$\,\um & 7 & 7 & 7 & 3 \\[4pt]
% \hline
% $0.1$--$10.0$\,\um & 58 & 43 & 12 & 9 \\[4pt]
\end{tabular}
\caption{Global mean turnover time (days) by removal process
\label{tbl:tau}}
\end{table}
}} % end \fcolorbox
Scavenging accounts for 61\% of deposition.\\
Dry processes account for 39\%.
\clearpage

\rotatefoilhead{\bgl\Large\textcolor{blue}{\hfill Size Dependence of
Burden and Optical Depth \hfill}}\large\vspace{-0.5in}
\fcolorbox{SkyBlue}{white}{\parbox{\textwidth}{
Fractional contribution to global mean dust burden
($29$\,\mgxmS), optical depth $\tau$ ($0.032$) by size bin: \\[4pt]  
\begin{table}
\centering % \centering uses less vertical space than center-environment
\begin{tabular}{l *{3}{>{$}r<{$}}}
Size & \mbox{Burden} & \tau & \\[4pt]
\tableline
\hline \rule{0.0ex}{\hlntblntrskp}%
% Numbers from dstmch26 simulation of 1998
$0.1$--$1.0$\,\um & 0.17 & 0.44 & \\[4pt]
$1.0$--$2.5$\,\um & 0.65 & 0.50 & \\[4pt]
$2.5$--$5.0$\,\um & 0.16 & 0.06 & \\[4pt]
$5.0$--$10.0$\,\um & 0.02 & 0.01 & \\[4pt]
\hline
\end{tabular}
%\caption{Fractional Contribution to Burden and Optical Depth by Size Bin
%\label{tbl:tau}}
\end{table}
}} % end \fcolorbox
\clearpage

\rotatefoilhead{\vspace{0.25in}\Large\textcolor{blue}{\hfill AVHRR
Seasonal AOD 0.63\,\um, 1998 \hfill}}\vspace{-0.5in}
%\enlargethispage*{1in} 
\begin{figure*}
\centering % \centering uses less vertical space than center-environment
\includegraphics*[width=0.5\hsize]{avhrr_1998_1202_DSTODXC}%
\includegraphics*[width=0.5\hsize]{avhrr_1998_0608_DSTODXC}%

\includegraphics*[width=0.5\hsize]{avhrr_1998_0305_DSTODXC}%
\includegraphics*[width=0.5\hsize]{avhrr_1998_0911_DSTODXC}%
\end{figure*}

\rotatefoilhead{\vspace{0.25in}\Large\textcolor{blue}{\hfill TOMS
Seasonal AAI 331--360~nm, 1998 \hfill}}\vspace{-0.5in} 
%\enlargethispage*{1in} 
\begin{figure*}
\centering % \centering uses less vertical space than center-environment
\includegraphics*[width=0.5\hsize]{toms_1998_1202_aer_idx_331_360}%
\includegraphics*[width=0.5\hsize]{toms_1998_0608_aer_idx_331_360}%

\includegraphics*[width=0.5\hsize]{toms_1998_0305_aer_idx_331_360}%
\includegraphics*[width=0.5\hsize]{toms_1998_0911_aer_idx_331_360}%
\end{figure*}

\rotatefoilhead{\vspace{0.25in}\Large\textcolor{blue}{\hfill MATCH
Seasonal Dust AOD 0.63\,\um, 1998 \hfill}}\vspace{-0.5in} 
%\enlargethispage*{1in} 
\begin{figure*}
\centering % \centering uses less vertical space than center-environment
\includegraphics*[width=0.5\hsize]{dstmch26_1998_1202_DSTODXC}%
\includegraphics*[width=0.5\hsize]{dstmch26_1998_0608_DSTODXC}%

\includegraphics*[width=0.5\hsize]{dstmch26_1998_0305_DSTODXC}%
\includegraphics*[width=0.5\hsize]{dstmch26_1998_0911_DSTODXC}%
\end{figure*}

\foilhead{
\vspace{0.0in}\large\textcolor{blue}{\hfill CCM Dust SW Forcing 1998 \hfill}}\vspace{-0.25in}
\enlargethispage*{1in} 
\begin{figure*}
\centering % \centering uses less vertical space than center-environment
\includegraphics[height=0.3\vsize]{dstccm45_1998_FSNTFRC}\vfill
\includegraphics[height=0.3\vsize]{dstccm45_1998_FSATFRC}\vfill
\includegraphics[height=0.3\vsize]{dstccm45_1998_FSDSFRC}\vfill
\end{figure*}

\foilhead{
\vspace{0.25in}\Large\textcolor{blue}{\hfill Dust at Last Glacial Maximum\hfill}}\vspace{-0.5in}
\enlargethispage*{1in} 
\begin{figure*}
\centering % \centering uses less vertical space than center-environment
\includegraphics[height=0.5\vsize]{fdst10tc_1998_DSTODXC}\vfill
\includegraphics[height=0.5\vsize]{fdst10ja_1998_DSTODXC}\vfill
\end{figure*}

\foilhead{
\vspace{0.25in}\Large\textcolor{blue}{\hfill Transport Model vs.\ GCM\hfill}}\vspace{-0.5in}
\enlargethispage*{1in} 
\begin{figure*}
\centering % \centering uses less vertical space than center-environment
\includegraphics[height=0.5\vsize]{dstmch26_1998_DSTODXC}\vfill
\includegraphics[height=0.5\vsize]{dstccm45_1998_DSTODXC}\vfill
\end{figure*}

\foilhead{
\vspace{0.25in}\Large\textcolor{blue}{\hfill Solar Forcing \hfill}}\vspace{-0.5in}
\enlargethispage*{1in} 
\begin{figure*}
\centering % \centering uses less vertical space than center-environment
\includegraphics[height=0.5\vsize]{dstccm45_1998_x_QRSFRC_glb}\vfill
\includegraphics[height=0.5\vsize]{dstccm45_1998_x_QRSFRC_SAM}\vfill % Saharan Air Mass 80W--20W
\end{figure*}

\rotatefoilhead{\bgl\Large\textcolor{blue}{\hfill Dust model in NCAR
MATCH, CCM, LSM\hfill}}\large\vspace{-0.5in}
\begin{itemize}
\item Source: $F \propto u_{*}^3$, basins, LAI, VWC sfc., snow
\item Dry deposition: turbulent and gravitational deposition
\item Wet deposition: nucleation, collision, convective, stratiform
\item Transported size distribution: SGS bins, 4 std.\ in [0,10]\,\um
\item Source size dist.: 3 background modes of Shettle (1984)
\item Soil texture, mineralogy: IGBP DIS
\item Radiative properties: Saharan dust, Mie theory
\end{itemize}

\rotatefoilhead{\bgl\Large\textcolor{blue}{\hfill Size Dependence of
Something \hfill}}\large\vspace{-0.5in}
\fcolorbox{SkyBlue}{white}{\parbox{\textwidth}{
\begin{table}
\centering % \centering uses less vertical space than center-environment
\begin{tabular}{l *{2}{>{$}r<{$}}}
Size & \mbox{Field1} & \mbox{Field2} \\[4pt]
& \mbox{\mgxmS} & \\[4pt]
\tableline
\hline \rule{0.0ex}{\hlntblntrskp}%
% Numbers from dstmch26 simulation of 1998
$0.1$--$1.0$\,\um & & \\[4pt]
$1.0$--$2.5$\,\um & & \\[4pt]
$2.5$--$5.0$\,\um & & \\[4pt]
$5.0$--$10.0$\,\um & & \\[4pt]
\hline
$0.1$--$10.0$\,\um & & \\[4pt]
\end{tabular}
\caption{Global mean turnover time (days) by removal process
\label{tbl:tau2}}
\end{table}
}} % end \fcolorbox
\clearpage

\rotatefoilhead{\bgl
\Large\textcolor{blue}{\hfill Mineral Dust \& Solar Absorption \hfill}}\vspace{-0.5in}\large
\begin{itemize}
\item Models may underpredict atmospheric solar absorption by 20\,\wxmS
\item Does mineral dust absorption ameliorate GCM atmospheric absorption bias?
\item Globally averaged, do clouds enhance or reduce solar absorption by dust?
\item Why such large uncertainty in forcing estimates like IPCC?
\end{itemize}

\rotatefoilhead{
\Large\textcolor{blue}{\hfill Refractive Indices by Station \hfill}}\vspace{-0.5in}\large
\enlargethispage*{1in} 
\begin{figure*}
\centering % \centering uses less vertical space than center-environment
\includegraphics[width=1.0\hsize]{idx_rfr_gph}%
\end{figure*}

\rotatefoilhead{
\Large\textcolor{blue}{\hfill Mass-Independent Dust Optical Properties \hfill}}\vspace{-0.5in}\large
\enlargethispage*{1in} 
% Cool trick: Overlay figures in LaTeX by using picture environment
% and placing % after everything to get rid of any whitespace
\begin{picture}(0,0)(0,0)%
\put(0.0,-5){\includegraphics[width=4.0in]{aer_saharan_dust_01_SWNB_ss_alb}}%
\end{picture}%
\begin{picture}(0,0)(0,0)%
\put(0.0,-5){\includegraphics[width=4.0in]{aer_afghan_dust_01_SWNB_ss_alb}}%
\end{picture}%
\begin{picture}(0,0)(0,0)%
\put(0.0,-5){\includegraphics[width=4.0in]{aer_aeronet_Bnz_01_1000_ss_alb}}%
\end{picture}%

\begin{picture}(0,0)(0,0)%
\put(5.0,-5){\includegraphics[width=4.0in]{aer_saharan_dust_01_SWNB_asm_prm}}%
\end{picture}%
\begin{picture}(0,0)(0,0)%
\put(5.0,-5){\includegraphics[width=4.0in]{aer_afghan_dust_01_SWNB_asm_prm}}%
\end{picture}%
\begin{picture}(0,0)(0,0)%
\put(5.0,-5){\includegraphics[width=4.0in]{aer_aeronet_Bnz_01_1000_asm_prm}}%
\end{picture}%

\rotatefoilhead{
\Large\textcolor{blue}{\hfill Mass-Specific Dust Optical Properties \hfill}}\vspace{-0.5in}\large
\enlargethispage*{1in} 
\begin{figure*}
\centering % \centering uses less vertical space than center-environment
\includegraphics[width=0.5\hsize]{aer_saharan_dust_01_SWNB_aer_ext_cff_mss}%
\includegraphics[width=0.5\hsize]{aer_afghan_dust_01_SWNB_aer_ext_cff_mss}
\end{figure*}

\rotatefoilhead{\bgl\Large\textcolor{blue}{\hfill Mineral Dust Solar Forcing \hfill}}\large\vspace{-0.5in}
\fcolorbox{SkyBlue}{white}{\parbox{\textwidth}{
\begin{table}
\centering % \centering uses less vertical space than center-environment
\begin{tabular}{lll}
\tableline
Metric & \multicolumn{2}{c}{SW Forcing} \\[0pt]
& Clear Sky & All-Sky \\[0pt]
\hline\hline
& & \\[-10pt]
% Results from four year mean of dstccm44
% Optical Depth (0.63\,\um) & $0.032$ \\[0pt]
Atmospheric Absorption & $1.9$ & $1.9$ \\[0pt]
Surface Insolation & $$ & $-2.3$ \\[5pt]
Surface Absorption & $-3.1$ & $-2.1$ \\[0pt]
Sfc.\ $+$ Atm.\ Absorption & $-1.2$ & $-0.5$ \\[0pt]
Net Flux at Tropopause & $-1.2$ & $-0.5$ \\[0pt] % TLF96 get 
\tableline
\end{tabular}
\caption{Global annual mean solar forcing (\wxmS) by dust
modeled with Saharan Dust optical properties \cite[]{Pat81}.
Atmospheric dust may be \textit{up to 10 time less absorptive} than
this! 
\label{tbl:frc}}   
\end{table}
}} % end \fcolorbox

\rotatefoilhead{\bgl
\Large\textcolor{blue}{\hfill New Mobilization Parameterization \hfill}}\vspace{-0.5in}\large
\begin{itemize}
\item Couples land surface properties to source mechanisms
\item Global, mechanistic model based on microphysical understanding
\begin{itemize}
\item Based on work by Balkanski, Bergametti, F\'ecan, Gillette,
Iversen, Marticorena, Raupach, Schulz, Shao, White
\item Sensitivities consistent with lab and field measurements
\end{itemize}
\item Hooks for surface chemistry
\item Easier to incorporate future dust research products
\end{itemize}

\rotatefoilhead{\bgl
\Large\textcolor{blue}{\hfill Size Distribution \hfill}}\vspace{-0.5in}\large
\begin{itemize}
\item Four transport bins: $[0.1$--$1.0],[1.0$--$2.5],[2.5$--$5.0],[5.0$--$10.0]$\,\um
\begin{itemize}
\item Transport bins have sub-grid distribution properties
\item Deposition acts on mass-weighted mean size of bin
\end{itemize}
\item Spatially invariant tri-modal source distribution
\begin{itemize}
\item Lognormal ``background desert modes'' \cite[]{She84,SBG98}
% Mass median diameters:
\item $(\dmtvma$\,\um,$\gsd) = (0.01, 1.89), (2.5, 2.0), (42, 2.13)$
\item Mass fractions 0.00026, 0.781, 0.219
\end{itemize}
\item Matrix maps source distributions to transport bins: 87\% of
vertical mass flux is transported (i.e., $\dmt < 10$\,\um)
\end{itemize}

\rotatefoilhead{\bgl
\Large\textcolor{blue}{\hfill Mobilization \hfill}}\vspace{-0.5in}\large
Sub-gridscale surface type: bare ground erodible surface
\begin{itemize}
\item Bare ground with LAI $< 0.2$\,\mSxmS, no snow cover
\item Globally uniform $\rghmmn = 0.01$~cm
\item ``Smooth'' roughness length $\rghmmnsmt = 0.003$~cm
\item Drag partition efficient fraction $f_e = 0.8$ \cite[]{MaB95}
\item Iterative solution for $\wndfrc$, $\mnolng$ \cite[]{Bon96}
\item $\wndfrc$ adjusted for ``Owen's effect'' \cite[]{GMB98}
\end{itemize}

\rotatefoilhead{\bgl
\Large\textcolor{blue}{\hfill Saltation Mass Flux $\flxmsshrzslt$\hfill}}\vspace{-1.0in}\large
\begin{eqnarray}
\wndfrcrat & = & \wndfrcthr / \wndfrc \nonumber \\
\flxmsshrzslt & = & { \cstslt \sfcfrcslt \dnsatm \wndfrc^3 \over \grv } 
( 1 - \wndfrcrat ) ( 1 + \wndfrcrat )^2
\label{eqn:flx_mss_hrz_slt_MaB95_1}
\end{eqnarray}
%\emph{Definitions:}
\begin{table}
\begin{tabular}{ >{$}r<{$} l}
\cstslt & Empirically determined saltation constant (2.61) \\[0.0ex]
\sfcfrcslt & Fractional area susceptible to erosion \\[0.0ex]
\grv & Gravity \\[0.0ex]
\dnsatm & Air density \\[0.0ex]
\flxmsshrzslt & Vertically integrated horizontal mass flux \\[0.0ex]
\wndfrcrat & Friction velocity ratio ($\wndfrcrat \le 1$) \\[0.0ex]
\wndfrc & Wind friction velocity \\[0.0ex]
\wndfrcthr & Threshold wind friction velocity \\[0.0ex]
\end{tabular}
\end{table}

\rotatefoilhead{\bgl
\Large\textcolor{blue}{\hfill Threshold Friction Velocity $\wndfrcthr$\hfill}}\vspace{-0.5in}\large
\cite{IvW82} show $\wndfrcthr = $
\begin{equation}
\left[
{ 0.1666681 \dnsaer \grv \dmt \over -1 + 1.928 \BBB^{0.0922} }
\left( 1 + {6 \times 10^{-7} \over \dnsaer \grv \dmt^{2.5} } \right)
\right]^{1/2}
\dnsatm^{-1/2}
\end{equation}
for $0.03 \le \BBB \le 10$, and, for $\BBB > 10$, $\wndfrcthr = $ 
\begin{equation}
\left[
0.0144 \dnsaer \grv \dmt ( 1 - 0.0858 e^{ -0.0617 ( \BBB - 10 ) } ) 
\left( 1 + {6 \times 10^{-7} \over \dnsaer \grv \dmt^{2.5} } \right)
\right]^{1/2}
\dnsatm^{-1/2}
\label{eqn:wnd_frc_thr_obs_2}
\end{equation}

\rotatefoilhead{\bgl
\Large\textcolor{blue}{\hfill Threshold Friction Velocity, continued\hfill}}\vspace{-0.5in}\large
\cite{MaB95} show $\rynnbrfrcthr$ or $\BBB$ is
\begin{eqnarray}
% IvW82 p. 111 variable ``B''
\rynnbrfrcthr & = & \wndfrcthr \dmt / \vscknm \\
% [frc] "B" MaB95 p. 16417 (5)
& \approx & 0.38 + 1331.0 (100.0 \dmt )^{1.56}
\end{eqnarray}
\emph{Definitions:}
\begin{table}
\begin{tabular}{ >{$}r<{$} l}
\dmt & Particle diameter \\[0.5ex]
\dnsaer & Particle density \\[0.5ex]
\BBB, \rynnbrfrcthr & Threshold friction Reynolds number \\[0.5ex]
\grv & Gravitational acceleration \\[0.5ex]
\vscknm & Kinematic viscocity \\[0.5ex]
\wndfrcthr & Threshold wind friction velocity \\[0.5ex]
\end{tabular}
\end{table}

\rotatefoilhead{\bgl
\Large\textcolor{blue}{\hfill Threshold Friction Velocity, continued\hfill}}\vspace{-0.5in}\large
Adjust threshold velocity for inhibition by moisture and for drag partition 
\begin{eqnarray}
\wndfrcthr(\vwc;\rghmmn,\rghmmnsmt) & = & \wndfrcthr(0;\rghmmnsmt)
\frcthrwet (\vwc) \frcthrdrg (\rghmmn,\rghmmnsmt)
%\frcthrwet (\vwc) & = & { \wndfrcthr(\vwc) \over \wndfrcthr(\vwc = 0) }
%= { \wndfrcthrwet \over \wndfrcthrdry } \\
%\frcthrdrg (\rghmmn,\rghmmnsmt) & = & { \wndfrcthr(\rghmmn,\rghmmnsmt)
%\over \wndfrcthr(\rghmmnsmt) }
\label{eqn:wnd_frc_thr_wet_dfn}
\end{eqnarray}
\emph{Definitions:}
\begin{table}
\begin{tabular}{ >{$}r<{$} l}
\frcthrdrg & Threshold increase due to drag partition \\[0.5ex]
\frcthrwet & Threshold increase due to moisture \\[0.5ex]
\rghmmn & Roughness length of erodible surface \\[0.5ex]
\rghmmnsmt & ``Smooth'' roughness length \\[0.5ex]
\vwc & Volumetric water content \\[0.5ex]
\end{tabular}
\end{table}

\rotatefoilhead{\bgl
\Large\textcolor{blue}{\hfill Threshold Increase due to Drag Partition $\frcthrdrg$\hfill}}\vspace{-0.5in}\large
\cite{MaB95} show
\begin{eqnarray}
\wndfrcfshfrc & = &
 1.0 - { \log(\rghmmn/\rghmmnsmt) \over
\log [ 0.35 (0.1/\rghmmnsmt)^{0.8} ] } \\
\frcthrdrg & = & \wndfrcfshfrc^{-1}
\label{eqn:wnd_frc_fsh_frc_dfn}
% MaB95 p. 16420, GMB98 p. 6207
\end{eqnarray}
\emph{Definitions:}
\begin{table}
\begin{tabular}{ >{$}r<{$} l}
\wndfrcfshfrc & Efficient fraction of wind friction velocity \\[0.5ex]
\frcthrdrg & Threshold increase due to drag partition \\[0.5ex]
\rghmmn & Roughness length of erodible surface \\[0.5ex]
\rghmmnsmt & ``Smooth'' roughness length \\[0.5ex]
\end{tabular}
\end{table}

\rotatefoilhead{\bgl
\Large\textcolor{blue}{\hfill Threshold Increase due to Moisture $\frcthrwet$\hfill}}\vspace{-0.5in}\large
\cite{FMB99} show
\begin{eqnarray}
\label{eqn:vwc_thr_dfn}
\vwcthr & = & 0.17 \mssfrccly + 0.14 \times \mssfrccly^2 \\
\label{eqn:frc_thr_wet_dfn}
% FMB99 p. 155 (15)
\frcthrwet & = & \left\{
\begin{array}{r@{\quad:\quad}l}
1 & \vwc \le \vwcthr \\
\sqrt{ 1 + 1.21 [ 100 ( \vwc - \vwcthr ) ]^{0.68} } & \vwc > \vwcthr
\end{array} \right.
\label{eqn:frc_thr_wet_FMB99}
\end{eqnarray}
\emph{Definitions:}
\begin{table}
\begin{tabular}{ >{$}r<{$} l}
\frcthrwet & Threshold increase due to moisture \\[0.5ex]
\mssfrccly & Soil clay content \\[0.5ex]
\vwc & Volumetric water content \\[0.5ex]
\vwcthr & Threshold volumetric water content \\[0.5ex]
\end{tabular}
\end{table}

% \foilhead{\vspace{0.25in}\Large\textcolor{blue}{\hfill Surface VWC\hfill}}\vspace{-0.5in} 
% %\enlargethispage*{1in} 
% \begin{figure*}
% \centering % \centering uses less vertical space than center-environment
% \includegraphics*[width=0.5\hsize]{map_8589_05_vwc_sfc}%
% \includegraphics*[width=0.5\hsize]{dstccm28_8589_05_VWC_SFC}%

% \includegraphics*[width=0.5\hsize]{map_8589_06_vwc_sfc}%
% \includegraphics*[width=0.5\hsize]{dstccm28_8589_06_VWC_SFC}%

% \includegraphics*[width=0.5\hsize]{map_8589_07_vwc_sfc}%
% \includegraphics*[width=0.5\hsize]{dstccm28_8589_07_VWC_SFC}%
% \end{figure*}

\rotatefoilhead{\bgl
\Large\textcolor{blue}{\hfill Vertical Mass Flux \hfill}}\vspace{-0.5in}\large
For $\mssfrccly < 0.20$, \cite{MaB95} show 
\begin{eqnarray}
% MaB95 p. 16423 (47)
\hrzvrtprpfct & \equiv & \flxmssvrtdst / \flxmsshrzslt \nonumber \\ 
& = & 100 \exp [ ( 13.4 \mssfrccly - 6.0 ) \ln 10 ]
\label{eqn:hrz_vrt_prp_fct_MaB95_2}
\end{eqnarray}
\emph{Definitions:}
\begin{table}
\begin{tabular}{ >{$}r<{$} l}
\hrzvrtprpfct & Ratio of vertical to horizontal mass fluxes \\[0.5ex]
\flxmssvrtdst & Vertical mass flux of suspended dust \\[0.5ex]
\mssfrccly & Soil clay content \\[0.5ex]
\flxmsshrzslt & Vertically integrated horizontal saltation mass flux \\[0.5ex]
\end{tabular}
\end{table}

\rotatefoilhead{\bgl
\Large\textcolor{blue}{\hfill Dry Deposition Parameterization \hfill}}\vspace{-0.5in}\large
\begin{eqnarray}
\flxdps & = & - \vlcdps \cnc \\ % SeP97 p. 958 (19.1)
\label{eqn:flx_dps_dfn}
\vlcdps & = & \vlctrb + \vlcgrv \nonumber \\
\label{eqn:vlc_dps_dfn}
& = & \frac{1}{\rssaer + \rsslmn + \rssaer \rsslmn \vlcgrv } + \vlcgrv \\
\label{eqn:rss_dps_dfn}
\rsslmn & = & \left\{
\begin{array}{ >{\displaystyle}l<{} @{\quad:\quad}l}
\frac{ 1}{\wndfrc ( \shmnbr^{-2/3} + 10^{-3/\stknbr} ) } & \mbox{Solid surfaces} \\
\frac{ 1}{\wndfrc ( \shmnbr^{-1/2} + 10^{-3/\stknbr} ) } & \mbox{Liquid surfaces}
\end{array} \right.
\label{eqn:rss_lmn_dfn}
\end{eqnarray}

\rotatefoilhead{
\Large\textcolor{blue}{\hfill Size Sensitivity of Dry Deposition \hfill}}\vspace{-0.5in}\large
\begin{figure*}
\centering % \centering uses less vertical space than center-environment
\includegraphics[width=1.0\hsize]{vlc_dps_aer}%
\end{figure*}

\rotatefoilhead{
\Large\textcolor{blue}{\hfill Size Sensitivity of Collision Efficiency \hfill}}\vspace{-0.5in}\large
\begin{figure*}
\centering % \centering uses less vertical space than center-environment
\includegraphics[width=1.0\hsize]{psd_cll_fsh}%
\end{figure*}

\rotatefoilhead{\bgl\Large\textcolor{blue}{\hfill IGBP DIS Soil Mineralogy \hfill}}\large\vspace{-0.5in}
\enlargethispage*{1in} 
\begin{itemize}
\item Exchangeable Aluminum
\item Organic Carbon
\item \CaCOt
\item Exchangeable potassium
\item Nitrogen
\item Exchangeable sodium 
\item Extractable phosphorous
\end{itemize}

\rotatefoilhead{\bgl\Large\textcolor{blue}{\hfill BGC Directions \hfill}}\large\vspace{-0.5in} 
\enlargethispage*{1in} 
\begin{itemize}
\item Publish dust work this millenium
\item ``Effects of Land Use on Climate'': Integrate dust and LSM 
\item Budgets of particulate \C, \N, \Pu\ (Mahowald) 
\item Examine soil alkalinity (e.g., \CaCOt) effects on chemistry
\end{itemize}

\rotatefoilhead{\bgl\Large\textcolor{blue}{\hfill Objectives \hfill}}\large\vspace{-0.5in} 
\enlargethispage*{1in} 
\begin{itemize}
\item Evaluate model climatology over Tropical North Atlantic
\item Demonstrate sensitivity of African Dust plume to 
uncertain model parameters 
\begin{itemize}
\item Sensitivity of Size distribution to Collection Efficiency
\item Sensitivity of Radiative Forcing to Size Distribution
% Change in size distribution between Izana and Barbados
% Radiative forcing efficiency per unit mass/optical depth at Izana vs. Barbados
\end{itemize}
\end{itemize}

% AGU
\rotatefoilhead{\bgl\Large\textcolor{blue}{\hfill Conclusions \hfill}}\large\vspace{-0.5in} 
\enlargethispage*{1in} 
\begin{itemize}
\item Microphysical approach: realistic mean/variability over TNA
\item Modeled African dust plume:
\begin{itemize}
\item Sensitive to washout through collection efficiency
\item Impacts longitudinal gradient of radiative forcing
\end{itemize}
\item Field measurements required to evaluate:
\begin{itemize}
\item Barbados dust size distribution cutoff at 2--3\,\um\ 
\item Significant change in radiative forcing per unit mass/optical
depth during transport (e.g., Iza\~na to Barbados)
\end{itemize}
\end{itemize}

\rotatefoilhead{\bgl\Large\textcolor{blue}{\hfill Conclusions \hfill}}\large\vspace{-0.5in} 
\enlargethispage*{1in} 
\begin{itemize*}
\item Microphysical approach to mobilization is promising in both 
CTMs and GCMs. 
Both require improved vegetation, soil (or rock!) texture, hydrology, winds
%GCMs require improved snow cover, Easterly waves.
\item Distribution looks reasonable without only natural sources
\item Dust absorbs $< 1.8$\,\wxmS\ solar radiation. 
This improves agreement between observed and modeled atmospheric
energy budgets (``enhanced absorption'').
\item Regionally, low clouds enhance dust forcing while
high clouds reduce it but these effects compensate in mean.
\item Uncertainty in total forcing more due to optical properties
(including composition) than to spatio-temporal distribution. 
%\item May require global datasets of sub-gridscale properties
%$E$, $\mssfrccly$, $\rghmmn$ 
\item Present day feedbacks on atmosphere are small
\end{itemize*}

% BGC
\rotatefoilhead{\bgl\Large\textcolor{blue}{\hfill Conclusions \hfill}}\large\vspace{-0.5in} 
\enlargethispage*{1in} 
\begin{itemize}
\item Microphysical approach yields realistic mean/variability
\item Mean and variability on daily--annual timescales 
\item Need better surface properties and wet deposition
\begin{itemize}
\item Surface type $n(D)$, LAI, $\mssfrccly$, $\rghmmn$ 
\end{itemize}
\item Preliminary:
\item Terrigenous \CaCOt\ variation negligible at ODP658
\item \CaCOt\ is regional, not linear with total dust
\end{itemize}

% UCI
\rotatefoilhead{\bgl\Large\textcolor{blue}{\hfill Conclusions \hfill}}\large\vspace{-0.5in} 
\enlargethispage*{1in} 
\begin{itemize}
\item Microphysical approach to global mobilization works
\item Solutions sensitive to size-dependent microphysics 
\item Need better surface properties and wet deposition
\begin{itemize}
\item Surface type $n(D)$, LAI, $\mssfrccly$, $\rghmmn$, mineralogy,
optical properties
\end{itemize}
\item Mineral and nutrient deposition is regional, not linear  
with total dust 
%\item Terrigenous \CaCOt\ variation negligible at ODP658
\end{itemize}

\rotatefoilhead{\bgl\Large\textcolor{blue}{\hfill Mineral Dust
Radiative Forcing at Barbados \hfill}}\large\vspace{-0.5in}
\fcolorbox{SkyBlue}{white}{\parbox{\textwidth}{
\begin{table}
\centering % \centering uses less vertical space than center-environment
\begin{tabular}{ l >{$}r<{$} >{$}r<{$} }
\tableline
Metric & \mbox{Forcing} & \mbox{$\Delta$ Forcing}\\[0pt]
\hline\hline
& & \\[-10pt]
Atmospheric Absorption & & \\[0pt]
(same, but for clear sky) & () & \\[0pt]
Surface Insolation & & \\[5pt]
Surface Absorption & & \\[0pt]
Sfc.\ $+$ Atm.\ Absorption & & \\[0pt]
Net Flux at Tropopause & & \\[0pt] % TLF96 get 
\tableline
\end{tabular}
\caption{July mean solar forcing (\wxmS) by dust at Barbados
\label{tbl:foo}}   
\end{table}
}} % end \fcolorbox

\rotatefoilhead{\bgl\Large\textcolor{blue}{\hfill Dust
Radiative Forcing Gradient \hfill}}\large\vspace{-0.5in}
\fcolorbox{SkyBlue}{white}{\parbox{\textwidth}{
\begin{table}
\centering % \centering uses less vertical space than center-environment
\begin{tabular}{ lrr }
\tableline
Metric & Forcing & $\Delta$ Forcing \\[0pt]
& Gradient & Gradient \\[0pt]
\hline\hline
& & \\[-10pt]
Surface Insolation & 11 & -7 \\[5pt]
Sfc.\ $+$ Atm.\ Absorption & 5.2 & -3.8 \\[0pt]
\tableline
\end{tabular}
\caption{Change in July mean gradient of solar forcing (\wxmS) by
mineral dust from Africa to Barbados at $15^\circ$N due to accounting
for size dependent scavenging.
\label{tbl:grd}}
\end{table}
}} % end \fcolorbox

\rotatefoilhead{\bgl\Large\textcolor{blue}{\hfill Dust
Radiative Forcing Gradient \hfill}}\large\vspace{-0.5in}
\fcolorbox{SkyBlue}{white}{\parbox{\textwidth}{
\begin{table}
\centering % \centering uses less vertical space than center-environment
\begin{tabular}{ lrr }
\tableline
Metric & Forcing & $\Delta$ Forcing \\[0pt]
& Gradient & Gradient \\[0pt]
\hline\hline
& & \\[-10pt]
Surface Insolation & 45 & -7 \\[5pt]
Sfc.\ $+$ Atm.\ Absorption & 19 & -3.8 \\[0pt]
\tableline
\end{tabular}
\caption{Change in July mean gradient of solar forcing (\wxmS) by
mineral dust from Africa to Barbados at $15^\circ$N due to accounting
for size dependent scavenging.
\label{tbl:grd2}}
\end{table}
}} % end \fcolorbox

\rotatefoilhead{\huge \textcolor{blue}{Re-examining radiative forcing
by clouds, aerosols, and gases}}\vspace{-0.5in}\LARGE
\begin{center}
Charles S. Zender\\
Department of Earth System Science, UCI\\
\bigskip
\bigskip
\bigskip
Department of Atmospheric Science, UCLA \\
April 25, 2001\\
{\large Thanks to:\\ Bill Collins, Susan Solomon, Francisco Valero, Shaocai Yu}
\normalsize
\end{center}

\rotatefoilhead{\huge \textcolor{blue}{Aeolian Mineral Dust in the
Climate System: Observations, Models, and
Implications}}\vspace{-0.5in}\LARGE 
\begin{center}
Charles S. Zender\\
Department of Earth System Science, UCI\\
\bigskip
\bigskip
\bigskip
Department of Earth Sciences, UCSC \\
February 21, 2001\\
{\large Thanks to:\\ Paul Ginoux, Natalie Mahowald, Phil Rasch, Dennis Savoie}
\normalsize
\end{center}

\rotatefoilhead{\huge \textcolor{blue}{Modeling the Global
Distribution of Mineral Dust}}\vspace{-0.5in}\LARGE
\begin{center}
Charles S. Zender\\
Department of Earth System Science, UCI\\
\bigskip
\bigskip
\bigskip
Bren School of Env.~Sci.~\&~Mgmt. \\
UCSB, January 19, 2001\\
{\large Thanks to:\\ Phil Rasch, Paul Ginoux, Dennis Savoie, Natalie Mahowald}
\normalsize
\end{center}

\rotatefoilhead{\huge \textcolor{blue}{Implications of the Size-Resolved
Distribution of African Mineral Dust Over the Tropical North Atlantic}}\vspace{-0.5in}\LARGE
\begin{center}
Charles S. Zender\par
\medskip
%Department of Earth System Science\\
University of California, Irvine\\
AGU, San Francisco, Dec 15, 2000\\
{\large Thanks to:\\ Phil Rasch, Paul Ginoux, Dennis Savoie, Natalie Mahowald}
\normalsize
\end{center}

\rotatefoilhead{\Huge \textcolor{blue}{Mineral Dust Aerosol
over the N. Atlantic and Indian Oceans}}\vspace{-0.5in}\LARGE
\begin{center}
Charles S. Zender\par
\medskip
University of California, Irvine\\
Biogeochemistry Working Group Meeting\\
Boulder CO, June 28, 2000\\
{\large Thanks to:\\ Brian Eaton, Paul Ginoux, Natalie Mahowald, Dennis Savoie}
\normalsize
\end{center}

\rotatefoilhead{\huge \textcolor{blue}{
Understanding the Global Distribution and Radiative Forcing of Mineral
Dust Aerosol}}\vspace{-0.5in}\LARGE 
\begin{center}
Charles S. Zender\\
Dept.\ of Earth System Science, UC Irvine\\
\bigskip
\bigskip
\bigskip
Env.\ Engr. Sci./Global Env. Science Seminar
CalTech, November 14, 2001\\
\bigskip
{\large Thanks: P.~Ginoux, N.~Mahowald, P.~Rasch, D.~Savoie}
\normalsize
\end{center}

\rotatefoilhead{\bgl
\Large\textcolor{blue}{\hfill Outline \hfill}}\vspace{-0.5in}\large
\begin{itemize}
\item Observed Distribution
\item Physical processes
\item Radiative Forcing
\item Biogeochemical Implications
\end{itemize}

\rotatefoilhead{\bgl\Large\textcolor{blue}{\hfill Conclusions \hfill}}\large\vspace{-0.5in} 
\enlargethispage*{1in} 
\begin{itemize}
\item Models underestimate atmospheric absorption $\sim 20$\,\wxmS
\item Some missing absorption is correlated with clouds
\item Absorbing aersols:
\begin{itemize}
\item Mineral Dust absorbs $\sim$1.8\,\wxmS 
\item Carbon: SE US $\sim 5$\,\wxmS, Indian Ocean $\sim 50$\,\wxmS
\end{itemize}
\item Absorbing gases:
\begin{itemize}
\item \OdX\ absorbs $\sim 1$\,\wxmS
\end{itemize}
\item No smoking gun! $\sim 17$\,\wxmS\ remains unexplained!
\begin{itemize}
\item Future: dirty (multi-component) aerosols, \NOd, \HdO
\end{itemize}
\end{itemize}

\rotatefoilhead{\bgl
\Large\textcolor{blue}{\hfill Future Directions/Opportunities \hfill}}\vspace{-0.5in}\large
Interaction of Aerosols and Climate
\begin{enumerate}
\item Natural Aerosol distributions
\begin{itemize}
\item Natural emissions controls: climate, veg., soil, $\mu$-physics
\item Controls, time evolution of silt, dune, loess systems (WSU, UCSB)
\item Aerosol hindcasting, forecasting (NCAR)
\end{itemize}
\item Radiative Forcing of dust, volcanic (current) sea-salt
\begin{itemize}
\item Multi-component aerosols: Optics, Mineralogy
\item Natural aerosol forcing present/paleo climates (NCAR, UCSC?)
\end{itemize}
\item Atmospheric Chemistry
\begin{itemize}
\item Photochemical forcing of aerosol (UCI)
\item Heterogeneous uptake of gases \HOx, \NOx
\end{itemize}
\item Biogeochemistry (UCSB, NCAR):
\begin{itemize}
\item Nutrient transport by Aeolian dust
\item \Fe\ and \Pu\ limitation
\end{itemize}
\item ``Trace'' gases: \NOd, in-cloud \HdO, \HdOHdO
\begin{itemize}
\item Improve representation and reduce absorption bias
\end{itemize}
\end{enumerate}

\csznote{
% Transfer required figures to local machine
for fl_stb in \
aer_aeronet_Bnz_01_1000_asm_prm \
aer_aeronet_Bnz_01_1000_ss_alb \
aer_afghan_dust_01_SWNB_aer_ext_cff_mss \
aer_afghan_dust_01_SWNB_asm_prm \
aer_afghan_dust_01_SWNB_ss_alb \
aer_afghan_dust_01_SWNB_ss_alb \
aer_saharan_dust_01_SWNB_aer_ext_cff_mss \
aer_saharan_dust_01_SWNB_asm_prm \
aer_saharan_dust_01_SWNB_asm_prm \
aer_saharan_dust_01_SWNB_ss_alb \
avhrr_1998_0305_DSTODXC \
avhrr_1998_0305_DSTODXC \
avhrr_1998_0608_DSTODXC \
avhrr_1998_0911_DSTODXC \
avhrr_1998_1202_DSTODXC \
avhrr_1998_1202_DSTODXC \
bds_bsn_T62 \
bds_erd_fct_T62 \
cnc_lubbock \
dstccm45_1998_DSTODXC \
dstccm45_1998_FSATFRC \
dstccm45_1998_FSDSFRC \
dstccm45_1998_FSNTFRC \
dstccm45_1998_FSNTFRC \
dstccm45_1998_x_QRSFRC_SAM \
dstccm45_1998_x_QRSFRC_glb \
dstmch1p_clm_Afr_DSTODXC \
dstmch1p_clm_Aus_DSTODXC \
dstmch1q_clm_EAs_DSTODXC \
dstmch1r_clm_EAs_DSTODXC \
dstmch1t_clm_Afr_DSTODXC \
dstmch1t_clm_Aus_DSTODXC \
dstmch26_1998_0305_DSTODXC \
dstmch26_1998_0305_DSTODXC \
dstmch26_1998_0608_DSTODXC \
dstmch26_1998_0911_DSTODXC \
dstmch26_1998_1202_DSTODXC \
dstmch26_1998_1202_DSTODXC \
dstmch26_1998_DSTODXC \
fdst10ja_1998_DSTODXC \
fdst10tc_1998_DSTODXC \
idx_rfr_gph \
lubbock_DSTMPC_dff \
lubbock_DSTSFDPS \
lubbock_DSTSFMBL \
psd_cll_fsh \
rsmas_dstmch90_clm_0112_AmS_cnc_mss_dst \
rsmas_dstmch90_clm_0112_Brb_cnc_mss_dst \
rsmas_dstmch90_clm_0112_Brm_cnc_mss_dst \
rsmas_dstmch90_clm_0112_CGH_cnc_mss_dst \
rsmas_dstmch90_clm_0112_CGr_cnc_mss_dst \
rsmas_dstmch90_clm_0112_Enw_cnc_mss_dst \
rsmas_dstmch90_clm_0112_Izn_cnc_mss_dst \
rsmas_dstmch90_clm_0112_Jej_cnc_mss_dst \
rsmas_dstmch90_clm_0112_KGI_cnc_mss_dst \
rsmas_dstmch90_clm_0112_Kaa_cnc_mss_dst \
rsmas_dstmch90_clm_0112_McH_cnc_mss_dst \
rsmas_dstmch90_clm_0112_Mdw_cnc_mss_dst \
rsmas_dstmch90_clm_0112_Mmi_cnc_mss_dst \
rsmas_dstmch90_clm_0112_Nau_cnc_mss_dst \
rsmas_dstmch90_clm_0112_NeC_cnc_mss_dst \
rsmas_dstmch90_clm_0112_Nrf_cnc_mss_dst \
rsmas_dstmch90_clm_0112_Okn_cnc_mss_dst \
rsmas_dstmch90_clm_0112_SlI_cnc_mss_dst \
toms_1998_0305_aer_idx_331_360 \
toms_1998_0305_aer_idx_331_360 \
toms_1998_0608_aer_idx_331_360 \
toms_1998_0911_aer_idx_331_360 \
toms_1998_1202_aer_idx_331_360 \
toms_1998_1202_aer_idx_331_360 \
toms_clm_Afr_aer_idx_331_360 \
toms_clm_EAs_aer_idx_331_360 \
vlc_dps_aer \
dstccm97_clm_VAI_DST \
dstccm96_clm_VAI_DST \
dstccm95_clm_VAI_DST \
dirtmap_dstccm97_DSTSFDPS \
dirtmap_dstccm96_DSTSFDPS \
dirtmap_dstccm95_DSTSFDPS \
dirtmap_dstccm96_DSTSFDPS_rat \
dirtmap_dstccm95_DSTSFDPS_rat \
dstccm97_clm_1202_DSTODXC \
dstccm97_clm_0608_DSTODXC \
dstccm97_clm_0305_DSTODXC \
dstccm97_clm_0911_DSTODXC \
dstccm97_clm_DSTODXC \
dstccm96_clm_DSTODXC \
dstccm95_clm_DSTODXC \
dstccm94_clm_FSNT \
dstccm94_clm_FSNTFRC \
dstccm97_clm_dstccm94_clm_FSNT \
dstccm94_clm_FTNT \
dstccm94_clm_FTNTFRC \
dstccm97_clm_dstccm94_clm_FTNT \
dstccm94_clm_FSNTFRC \
dstccm93_clm_FSNTFRC \
dstccm92_clm_FSNTFRC \
; do 
scp 'dust.ess.uci.edu:${DATA}/ps/'${fl_stb}'.eps' ${DATA}/ps
epstopdf ${DATA}/ps/${fl_stb}.eps
done
} % end csznote
% $: rebalance syntax highlighting

\rotatefoilhead{%\bgl
\Large\textcolor{blue}{\hfill References\hfill}}\vspace{-0.5in}\large
% Bibliography
%\renewcommand\refname{\normalsize Publications}
\bibliographystyle{jas}
\bibliography{bib}

\end{document}

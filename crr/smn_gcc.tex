% $Id$ -*-LaTeX-*-

% Purpose: Seminar on global climate change 
% Presented to: Dartmouth club 20020315, ESS 20E 20020605, Old folks,
% PDFscreen example: http://www.cgd.ucar.edu/stats/LATEX

% Usage: 
% cd ~/crr;pdflatex smn_gcc.tex;cd -
% cd ~/crr;make -W smn_gcc.tex smn_gcc.dvi smn_gcc.ps smn_gcc.pdf;cd -
% cd ~/crr;pdflatex smn_gcc.tex;thumbpdf smn_gcc;pdflatex smn_gcc.tex;cd -
% cd ~/crr;make -W smn_gcc.tex smn_gcc.pdf;cd -

\documentclass[12pt]{article}

% Standard packages
\usepackage{ifpdf} % Define \ifpdf
\ifpdf % PDFLaTeX
\usepackage[pdftex]{graphicx} % Defines \includegraphics*
\pdfcompresslevel=9
%\usepackage{thumbpdf} % Generate thumbnails
\usepackage{epstopdf} % Convert .eps, if found, to .pdf when required
\graphicspath{/data/zender/fgr/esmf} % Help epstopdf find .eps figures to convert
\else % !PDFLaTeX
\usepackage{graphicx} % Defines \includegraphics*
\fi % !PDFLaTeX
\usepackage{array} % Table and array extensions, e.g., column formatting
\usepackage[usenames]{color} % usenames allows, e.g., ``ForestGreen''
\usepackage{colortbl} % Colored columns in tables, required by pdfscreen
\usepackage{makeidx} % Index keyword processor: \printindex and \see
\usepackage{mdwlist} % Compact list formats \itemize*, \enumerate*
\usepackage[numbers,sort]{natbib} % \cite commands from aguplus
\usepackage{times} % Postscript Times-Roman font KoD99 p. 375
%\usepackage{tocbibind} % Add Bibliography and Index to Table of Contents
\usepackage{xspace} % Unknown

% pdfscreen uses hyperref to set margins for screen mode
% pdfscreen calls hyperref internally if hyperref is not already invoked
% pdfscreen also loads packages graphicx, color, calc, and comment
% print         Print version
% screen        Screen version
% panelright    Navigation panel on RHS
% paneltoc      Table of Contents in panel
% sectionbreak  Introduces pagebreak before section
% code          Commands producing fancy verbatim-like effects
%\usepackage[screen,code,panelright,paneltoc,sectionbreak]{pdfscreen}
\usepackage[screen,code,sectionbreak]{pdfscreen}
%\usepackage[print]{pdfscreen}

% Personal packages
\usepackage{csz} % Library of personal definitions (\ifpdf...)
\usepackage{abc} % Alphabet as three letter macros
\usepackage{dmn} % Dimensional units
\usepackage{chm} % Chemistry
\usepackage{dyn} % Fluid dynamics
\usepackage{aer} % Aerosol physics
\usepackage{rt} % Commands specific to radiative transfer

% Commands which must be executed in preamble

% Commands specific to this file
\newcommand{\pdfscreen}{\texttt{\small\color{section1}pdfscreen}\xspace} % Pretty format for pdfscreen string
\newcommand{\bluestar}{\textcolor{blue}{$^{*}$}}
\newcommand{\greenstar}{\textcolor{green}{$^{*}$}}
\newcommand{\redstar}{\textcolor{red}{$^{*}$}}

% Float placement
% NB: Placement of figures is very sensitive to \textfraction
\renewcommand\textfraction{0.0} % Minimum fraction of page that is text
\setcounter{totalnumber}{73} % Maximum number of floats per page
\setcounter{topnumber}{73} % Maximum number of floats at top of page
\setcounter{dbltopnumber}{73} % Maximum number of floats at top of two-column page
\setcounter{bottomnumber}{73} % Maximum number of floats at bottom of page
\renewcommand\topfraction{1.0} % Maximum fraction of top of page occupied by floats
\renewcommand\dbltopfraction{1.0} % Maximum fraction of top of two-column page occupied by floats
\renewcommand\bottomfraction{1.0} % Maximum fraction of bottom of page occupied by floats
\renewcommand\floatpagefraction{1.0} % Fraction of float page filled with floats
\renewcommand\dblfloatpagefraction{1.0} % Fraction of double column float page filled with floats

% Screen geometry
\begin{screen}
% These commands are all defined by pdfscreen
\margins{0.65in}{0.65in}{0.65in}{0.65in} % Left, right, top, bottom margins
%\emblema{logoLUC.png} % Location of graphic file for pdfscreen navigation panel
\screensize{6.25in}{8in} % Screen dimensions of PDF output
\urlid{http://www.ess.uci.edu/~zender} % Homepage button in navigation panel points here
\changeoverlay % Cycle through default overlays, repeating every tenth section
\paneloverlay{but.pdf} % Overlay for navigation panel
%\overlay{/data/zender/ps/ipcc_rad_frc.pdf} % Overlay for screen area
\def\pfill{\vskip6pt}
\end{screen}

% Commands performed only in print section
\begin{print}
\notesname{Notes:} % String appearing in notes ovals adjacent to slides
% Redefine section formatting in print mode
\makeatletter
\def\@seccntformat#1{\llap{\scshape\color{section\thesection@level}\csname the#1\endcsname.\hspace*{6pt}}}
\makeatother
\end{print}

\hypersetup{ % A command provided by \hyperref
pdftitle={21st Century Climate Change: A Southern Californian Perspective}, % Title given to acroread window displaying this file
pdfsubject={21st Century Climate Change: A Southern Californian Perspective},
pdfauthor={Charlie Zender},
pdfkeywords={gcc},
pdfpagemode={FullScreen}, % Starts in full screen mode, hit 'Esc' to escape
pdfmenubar=true % Allow access to reader's menubar
} % end \hypersetup

\begin{document} % End preamble

% Title for screen mode
\begin{screen}
\title{\color{section0}\Huge 21st Century Climate Change: A Southern Californian Perspective}
\end{screen}

% Title for print mode
\begin{print}
\title{\Huge\texttt{21st Century Climate Change: A Southern Californian Perspective}}
%\title{\Huge\texttt{Causes, Predictions, Impacts, Adaptation, Vulnerability}}
\end{print}

\author{
\color{section1}\Large
\href{http://www.ess.uci.edu/~zender}{Charles~S.~Zender}
{\small\href{mailto:zender@uci.edu}{\color{section1}\texttt{<zender@uci.edu>}}}\\
\href{http://www.ess.uci.edu}{Department of Earth System Science}\\
\href{http://www.uci.edu}{University of California, Irvine}\\
Presented to:\\
\href{http://unex.uci.edu/all/}{UC Irvine Academy for Lifelong Learning}, February 27, 2003\\
%\href{http://www.alum.dartmouth.org/clubs/orange/index.html}{Dartmouth Alumni of Orange County}, March 13, 2002\\
%\href{http://eee.uci.edu/02s/42020}{ESS20E: The Atmosphere}, 
%\href{http://eee.uci.edu/02s/42020/lct/ess_atm_lct_20_pst.pdf}{Lecture 20}\\
} % end author

\pagedissolve{Split /D 2 /Dm /H /M /O} % pagedissolve options in pdfscreen manual p. 9

\date{} % Empty braces turns off date
\maketitle
\begin{screen}
\vfill
\end{screen}

\begin{abstract}
\noindent Scientific observations and theories give us a basis
for understanding and predicting climate change.
We explore some predicted impacts of climate change that are of
particular relevance to Southern California.
Appreciating these impacts helps us to identify our potential
vulnerabilities, and to plan to adapt.
\end{abstract}

% Do not use \tableofcontents command in document when paneltoc is option
\begin{print}
\tableofcontents
\end{print}
\begin{screen}
\vfill
\end{screen}

\Large
\section[Outline]{Outline}
\begin{itemize*}
\item Earth System Science 
%\item Causes: Natural or Man-made?
\item Global Perspective 
\item Impacts: A Southern Californian Perspective
\begin{itemize*}
\item Temperature
\item Water
\item Sea Level Rise
\item Weather-Related Insurance
\end{itemize*}
\item Vulnerability: Who is at risk?
\item Adaptation: How can we adjust?
\begin{itemize*}
\item The Good (for us) News 
\end{itemize*}
\item Future Prospects
\end{itemize*}
\vfill

\section[Earth System Science]{Earth System Science Department at UCI}
Our Department \ldots
\begin{itemize*}
\item Dedicated to understanding climate on scale of human lifetime
\item Founded 1989 by Cicerone (UCI Chancellor)
\item Influential---top ranked in citations per professor
\item 17 Faculty (and growing!)
\item Croul Hall opened September 2003
\end{itemize*}
Scientific goals: Carbon Cycle and Climate System \ldots
\begin{itemize*}
\item Demographic, Economic models $\rightarrow$ (GHG)
Emissions scenarios $\rightarrow$ \textcolor{green}{Climate Models} $\rightarrow$
\textcolor{green}{Forecasts} $\rightarrow$ Impacts $\rightarrow$ Adaptation
\item Climate models simulate all relevant physical processes (we hope!)
\item ``Hindcasts'' of 20th century \textcolor{green}{\textit{climate}} are rather good
\item Involved in national efforts to improve models, measurements
\end{itemize*}
\vfill

\section[IPCC Scenarios]{A Global Perspective: IPCC Scenarios}
\begin{table}[h]
\begin{center}
\begin{tabular}{ *{7}{>{}c<{}} }
Date & Population & GDP & Surface \Ot & \COd & $\Delta T$\greenstar & $\Delta\mbox{MSL}$\greenstar \\[4pt]
& (billions) & $10^{12}$ US\$ yr$^{-1}$ & ppm & ppm & \dgrf & (inches) \\[4pt]
\tableline
\hline \rule{0.0ex}{\hlntblntrskp}%
1990 & 5.3 & 21 & -- & 354 & 0 & 0 \\[4pt]
2000 & 6.1--6.2 & 25--28 & 40 & 367 & 0.36 & 2 \\[4pt]
2050 & 8.4--11.3 & 59--187 & $\sim$60 & 463--623 & \textcolor{red}{1.4--4.7} & \textcolor{red}{2--13} \\[4pt]
2100 & 7.0--15.1 & 197--550 & $>$70 & 478--1099 & 2.5--10.5 & 3.5--35 \\[4pt]
\end{tabular}
\end{center}
\caption{\greenstar Discussed today\label{tbl:ipcc}}
\end{table}
\begin{itemize*}
\item IPCC is consensus report by $>$1000 top scientists
\item Large ranges in forecast values due to \ldots
\begin{itemize*}
\item Emissions scenarios (1990 level, business as usual, China \ldots)
\item Intrinsic uncertainty (natural variability, chaos)
\item Model differences
\end{itemize*}
\end{itemize*}
\clearpage

\section[Temperature]{Temperature}
\begin{itemize*}
\item Past century: N. America \textcolor{red}{warmed} $\sim$1.3\,\dgrf
\item Next century: Likely \textcolor{red}{warming} by 1.8--5.4\,\dgrf (best case) to 6.3--12.6\,\dgrf
\begin{itemize*}
\item Very likely \textcolor{red}{increasing} minimum temperatures
\begin{itemize*}
\item Fewer \textcolor{blue}{cold} days, \textcolor{blue}{frost} days
\item Less heating required in Winter
\end{itemize*}
\item Very likely \textcolor{red}{increasing} maximum temperatures
\begin{itemize*}
\item More \textcolor{red}{hot} days, \textcolor{red}{heat} waves
\item More cooling required in Summer
\end{itemize*}
\item Frequency of (currently) rare events will increase
\begin{itemize*}
\item Hottest year in 100 years was 1995. Soon once in 10 years!
\end{itemize*}
\end{itemize*}
\item El Ni\~{n}o/La Ni\~{n}a more pervasive
\begin{itemize*}
\item El Ni\~{n}o brings strong winter storms to Southland
\end{itemize*}
\end{itemize*}
\vfill

\section[Water]{Water}
\begin{itemize*}
\item Condensation in warmer climate
\begin{itemize*}
\item More winter rainfall--less snowfall
\item Less snowpack $\rightarrow$ Earlier seasonal peak in runoff
\item Possible declines in groundwater recharge--water supply
\item Increased frequency of intense precipitation--flash floods, sediments
\end{itemize*}
\item Warmer lakes and streams
\begin{itemize*}
\item Warmwater species benefit, but reduced lake levels
\item Damage to coldwater species (trout, salmon)
\end{itemize*}
\item How to adapt?
\begin{itemize*}
\item Improve management of water storage capacity
\item Voluntary water transfers between users
\item Pressure to transfer irrigation water $\rightarrow$ cities
\item Move to crops with lower water demand
\item Ski earlier, use snow machines
\end{itemize*}
\end{itemize*}
\vfill

\section[Sea Level Rise]{Sea Level Rise}
\begin{itemize*}
\item Present: Sea levels are \textcolor{red}{rising} at 1--2\,\mm\ per year
\item Next Century: Total \textcolor{red}{Rise} of 4--36\,in
\begin{itemize*}
\item Causes
\begin{itemize*}
\item Thermal expansion, glacier melt, sea-ice thinning
\item Greenhouse gases \textit{accelerate} thermal expansion\ldots
\end{itemize*}
\item Consequences
\begin{itemize*}
\item Shelf slope of $\sim$3\% $\rightarrow$ 1\,ft vertical rise engulfs 30\,ft of shoreline
\item Increased erosion
\item Storm surges
\item Siltation
\end{itemize*}
\item Developments along exposed coast especially vulnerable
\begin{itemize*}
\item Southern California less vulnerable than SE US
\item Louisiana, Florida, Texas, North Carolina most vulnerable
\end{itemize*}
\end{itemize*}
\end{itemize*}
\vfill

\section[Weather-Related Insurance]{Weather-Related Insurance}
\begin{itemize*}
\item Insurance spreads weather-related risk across society
\item Sensitive to many weather-related phenomena
\begin{itemize*}
\item Storms: Hurricanes, Tornados, Winds, Surges \ldots
\item Other: Fire, Frost, Flood, Landslide, Drought, Subsidence, \ldots
\end{itemize*}
\item Insurance industry may be vulnerable
\begin{itemize*}
\item Weather-related losses increased three times faster than
property/casualty premiums from 1985--1999 
\item 80--90\% of natural disaster losses are weather-related
\end{itemize*}
\item How to adapt?
\begin{itemize*}
\item Reduce coverage (e.g., no flood, crop insurance)
\item Increased demand for publically funded compensation
\end{itemize*}
\end{itemize*}
\vfill

\section[Future Work]{Future Work}
\begin{itemize*}
\item Forecast/Hindcast Atmospheric Soil Dust
\begin{itemize*}
\item Radiative forcing (like greenhouse gases)
\item Iron fertilization of marine ecosystems (\COd\ sink)
\item Phosphate fertilization of terrestrial ecosystems (\COd\ sink)
\item Role in cloud formation and structure
\end{itemize*}
\end{itemize*}
\vfill

\section{The Scientific Consensus on Global Warming}
\begin{center}
Charles S. Zender\\
Deptartment of Earth System Science\\
University of California, Irvine\\
\bigskip
\bigskip
\bigskip
Unitarian Universalist Church of Orange County\\
Costa Mesa, CA, August~18, 2004\\
\bigskip
\normalsize
\end{center}

\section{The Scientific Consensus on Global Warming}
\begin{itemize*}
\item Is Earth's climate changing significantly?
\item Did Humans cause the recent changes?
\item How is climate change expected to affect Earth globally?
\item How is climate change expected to affect Southern California?
\item Is global warming irreversible? 
\item How do scientists answer global warming skeptics?   
\end{itemize*}

\section{Definitions}
\begin{itemize*}
\item \textcolor{green}{Climate}: Average patterns of
  temperature, rainfall, wind\ldots 
\begin{itemize*}
\item Averages taken over 1000's of miles, many years
\item Climate is intrinsically noisy (butterfly effect)
\end{itemize*}
\item \textcolor{green}{Climate change}: Statistically significant change in climate
\begin{itemize*}
\item Natural climate change occurs independent of humans
\item Detectable \textcolor{green}{anthropogenic} (man-made)
  climate change must exceed the envelop of natural variability 
\end{itemize*}
\item \textcolor{green}{Global Change}: Includes
  \textcolor{red}{Global Warming} (temperature change)
\begin{itemize*}
\item Stratospheric Ozone Depletion, Acid Rain, De-forestation 
\item Impacts biodiversity, human health, agriculture\ldots
\end{itemize*}
\end{itemize*}

\section{Is Climate Changing Significantly?}
Satellite sensors and thermometers measure temperature directly.
Temperature \textcolor{green}{proxies} from tree rings, pollen,
ice cores, and sedimentary rock extend the temperature record in time.
\begin{itemize*}
\item Earth's climate has changed dramatically more in the past
  century than in the past 10,000 years  
\item Humans are driving Greenhouse Gas (GHG) concentrations to levels
  not reached since the Eocene (50,000,000\,\yr\ ago)
\end{itemize*} 
The Intergovernmental Panel on Climate Change (IPCC), a UN and
government appointed group of $\sim 1000$ experts in climate science,
agree that these GHG increases will very likely accelerate climate
change over the next century and beyond.

\section{Are Humans Changing Climate?}
Climate is a statistic so most appropriate evidence that humans cause
climate change is statistical (i.e.,
circumstantial)---\textcolor{blue}{there is no ``smoking gun''}.   
\begin{itemize*}
\item Legal standards for convictions in court
\begin{itemize*} 
\item Civil: ``preponderance of evidence''
\item Criminal: ``beyond a reasonable doubt'' 
\end{itemize*}
\item Scientists have demanding standards for evidence too!
\begin{itemize*}
\item Statistical significance (e.g., 95\% confidence),
  reproducibility, peer review  
\item Cautious attribution of climate change means it is
  \textcolor{blue}{much likelier that scientists attribute too little 
  than too much climate change to humans}
\end{itemize*}
\end{itemize*}

\section{Crime of the Century: Is Humanity Guilty?}
Scientists agree that most warming of last 50\,\yr\ is likely due to
humans.
How would they prosecute this allegation?
\begin{itemize*}
\item \textcolor{green}{Motive}: None! No one wants to change Earth's climate
\item \textcolor{green}{Opportunity}: Industrial, consumer, population
  pressures $\rightarrow$ increase fuel, land use $\rightarrow$
  unregulated GHG increases
\item \textcolor{green}{Method}: Physical law dictates that
  \textcolor{red}{GHGs increase temperature} 
\item \textcolor{green}{Weapon}: Greenhouse Gas (GHG), aerosol emission, land-use
\item \textcolor{green}{Evidence} humans are responsible:
\begin{itemize*}
 \item \textcolor{blue}{Anthropogenic GHGs and aerosols leave
  predictable ``fingerprints''} (diurnal temperature range decrease,
  accelerated polar warming, precipitation frequency/intensity)
\item Computer models accurately hindcast/predict recent change
  (response to eruptions, interannual warming/cooling)
\end{itemize*}
\end{itemize*}

\section{What Global Climate Changes to Expect?}
Scientists agree that \textcolor{blue}{the following changes will very
  likely occur} in the 21st century:
\begin{itemize*}
\item Global temperature \textcolor{red}{increases} (relative to
  1960--1990) by $1.8\mbox{--}5.5$\,\dgrf\ (low) to
  $2.7\mbox{--}8.1$\,\dgrf\ (high) by 2085   
\item Sea level rises $4\mbox{--}36$\,inches
\item Disappearing glaciers (ice sheets take longer)
\item Increased: minimum and maximum temperatures; summer drying and drought;
rainfall intensity; tropical cyclone (hurricane) winds, precipitation
\end{itemize*}

\section{Climate Changes to Expect in California}
A recent study examined two future climate change scenarios in two
climate prediction models.
The ``low'' and ``high'' emissions increase scenarios represent
agressive emissions reductions and business as usual, respectively. 
\begin{itemize*}
\item Summer temperature \textcolor{red}{increases} in CA (relative to 1960--1990): 
$3\mbox{--}3.5$\,\dgrf\ (low) to $2.5\mbox{--}5.5$\,\dgrf\ (high) by 2035,
$4.5\mbox{--}8$\,\dgrf\ (low) to $7.5\mbox{--}15$\,\dgrf\ (high) by 2085.
\item By the 2090s, heatwave days and excess heat-related deaths in LA
  will increase by $2\mbox{--}3$ times (low) to $5\mbox{--}7$ times (high). 
\item Warmer Winter storms and earlier snowmelt stress reservoirs
\item Spring Sierra snowpack decrease by $25\mbox{--}40$\% by 2035,
  $30\mbox{--}90$\% by 2085 
\item Alpine forest reduction and desert expansion
\end{itemize*}

\section{Is Climate Change Irreversible?}
Earth's Climate system takes many years to respond to forcings:
\begin{itemize*}
\item \textcolor{blue}{Our pollution outlives us}---GHG lifetimes: \CHq~(10\,\yr),
  \COd~(150\,\yr), \NdO~(300\,\yr) CFCs~(10000\,\yr)
\item Ice sheets bases take millenia to feel atmospheric temperature 
\item Oceans store/release \COd, heat in 1000--5000\,\yr\ cycles
\end{itemize*}
Our actions decades ago determined current climate change.
\textcolor{blue}{Restricing GHG emissions now will avert the worst  
  impacts in coming decades and centuries.}    
Individuals and societies will \textcolor{blue}{lead and act now}
to limit climate change, or 
\textcolor{red}{leave our children to deal with our mess}.

\section{What About Skeptics' Arguments?}
Climate change skeptics \textcolor{blue}{agree} that global surface
temperature \textcolor{red}{increased} $1\,\dgrf$ since \textit{c.}~1900.
They \textcolor{blue}{dispute} attributing this change to
anthropogenic forcing (GHGs, aerosols, \ldots).
\begin{itemize*}
\item Theories and models of natural climate change (Solar, Volcanic)
  \textit{explain} much of pre-anthropogenic climate change.
\item Natural climate change \textit{does not explain} recent
  (since $\sim$1900) climate change patterns
\item Greenhouse Gases and Aerosols, together with Natural forcing, 
  \textit{do explain} recent climate change patterns
\item Precautionary Principle: ``Better safe than sorry''
\end{itemize*}

\section{Where to Learn More}
\begin{itemize*}
\item My Website: \url{http://www.ess.uci.edu/~zender}
\item IPCC Website: \url{http://www.ipcc.ch}
\item UCS/ESA Report on Climate Change in California:
  \url{http://www.ucsusa.org/climatechange/california.html}
\item Hayhoe et~al.\ paper \url{http://www.pnas.org/cgi/doi/10.1073/pnas.0404500101}
\end{itemize*}

\csznote{
% Usage: Place usage here at end of file so comment character % not needed
cd ~/crr;pdflatex smn_gcc.tex;cd -
cd ~/crr;make -W smn_gcc.tex smn_gcc.dvi smn_gcc.ps smn_gcc.pdf;cd -
cd ~/crr;pdflatex smn_gcc.tex;thumbpdf smn_gcc;pdflatex smn_gcc.tex;cd -
cd ~/crr;texcln smn_gcc;make smn_gcc.pdf;bibtex smn_gcc;makeindex smn_gcc;make smn_gcc.pdf;bibtex smn_gcc;makeindex smn_gcc;make smn_gcc.pdf;cd -
cd ~/crr;make -W smn_gcc.tex smn_gcc.pdf;cd -

scp ${DATA}/ps/smn_gcc.pdf dust.ess.uci.edu:/var/www/html/smn/smn_gcc.pdf
} % end csznote

% $: rebalance syntax highlighting

\end{document}


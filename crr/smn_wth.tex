% $Id$ -*-LaTeX-*-

% Purpose: Seminar on weather
% Presented to: Vista Verde Elementary School, 20070222

% URL: 
% http://dust.ess.uci.edu/smn/smn_wth.pdf
% http://dust.ess.uci.edu/smn/smn_wth_vv_200702.pdf

% Create (see also end of file):
% cd ~/crr;make -W smn_wth.tex smn_wth.pdf;cd -

% Synchronize:
% rsync dust.ess.uci.edu:/data/zender/fgr/smn_wth ${DATA}/fgr
% rsync /data/zender/fgr/smn_wth dust.ess.uci.edu:/data/zender/fgr

% Distribute:
% scp ${DATA}/ps/smn_wth.pdf dust.ess.uci.edu:/var/www/html/smn/smn_wth.pdf
% scp ${DATA}/ps/smn_wth.pdf dust.ess.uci.edu:/var/www/html/smn/smn_wth_vv_200702.pdf

\documentclass[12pt]{article}

% Standard packages
\usepackage{ifpdf} % Define \ifpdf
\ifpdf % PDFLaTeX
\usepackage[pdftex]{graphicx} % Defines \includegraphics*
\pdfcompresslevel=9
%\usepackage{thumbpdf} % Generate thumbnails
\usepackage{epstopdf} % Convert .eps, if found, to .pdf when required
\graphicspath{/data/zender/fgr/smn_wth} % Help epstopdf find .eps figures to convert
\else % !PDFLaTeX
\usepackage{graphicx} % Defines \includegraphics*
\fi % !PDFLaTeX
\usepackage{array} % Table and array extensions, e.g., column formatting
%\usepackage[usenames]{color} % usenames allows, e.g., ``ForestGreen''
\usepackage{colortbl} % Colored columns in tables, required by pdfscreen
\usepackage[dayofweek]{datetime} % \xxivtime, \ordinal
\usepackage{makeidx} % Index keyword processor: \printindex and \see
\usepackage{mdwlist} % Compact list formats \itemize*, \enumerate*
%\usepackage[numbers,sort]{natbib} % \cite commands from aguplus
\usepackage[sort]{natbib} % \cite commands from aguplus
\usepackage{times} % Postscript Times-Roman font KoD99 p. 375
%\usepackage{tocbibind} % Add Bibliography and Index to Table of Contents
\usepackage{xspace} % Unknown

% pdfscreen uses hyperref to set margins for screen mode
% pdfscreen calls hyperref internally if hyperref is not already invoked
% pdfscreen also loads packages graphicx, color, calc, and comment
% print         Print version
% screen        Screen version
% panelright    Navigation panel on RHS
% paneltoc      Table of Contents in panel
% sectionbreak  Introduces pagebreak before section
% code          Commands producing fancy verbatim-like effects
%\usepackage[screen,code,panelright,sectionbreak]{pdfscreen}
\usepackage[screen,code]{pdfscreen}
%\usepackage[print]{pdfscreen}

% Personal packages
\usepackage{csz} % Library of personal definitions (\ifpdf...)
\usepackage{abc} % Alphabet as three letter macros
\usepackage{dmn} % Dimensional units
\usepackage{chm} % Chemistry
\usepackage{dyn} % Fluid dynamics
\usepackage{aer} % Aerosol physics
\usepackage{rt} % Commands specific to radiative transfer
\usepackage{psd} % Particle size distributions
\usepackage{smn} % Seminar configuration for pdfscreen

% Commands which must be executed in preamble

% Commands specific to this file
% 1. Primary commands
\hypersetup{ % A command provided by \hyperref
pdftitle={Weather}, % Title given to acroread window displaying this file
pdfsubject={Weather},
pdfauthor={Charlie Zender},
pdfkeywords={outreach physics weather},
%pdfpagemode={FullScreen}, % Starts in full screen mode, hit 'Esc' to escape
pdfmenubar=true % Allow access to reader's menubar
} % end \hypersetup

\begin{document} % End preamble

\def\smnttl{Pretty, Nasty, Weather}

\begin{screen}
\title{\color{section0}\Huge \smnttl} % Title for screen mode
\end{screen}
\begin{print} 
\title{\Huge\texttt{\smnttl}} % Title for print mode
\end{print}

\author{\LARGE
Olivia's Dad (Charlie)
{\href{mailto:zender@uci.edu}{\color{section1}\texttt{<zender@uci.edu>}}}\\ 
\href{http://www.ess.uci.edu}{Department of Earth System Science}\\
University of California, Irvine\\
\\
Presented to:\\
Vista Verde Elementary School First Grade Assembly\\
Built: \shortdate\today, \xxivtime\\
\\
(Web:~\url{http://dust.ess.uci.edu/smn/smn_wth_vv_200702.pdf})
} % end author

\pagedissolve{Split /D 2 /Dm /H /M /O} % pagedissolve options in pdfscreen manual p. 9

\Large
\begin{figure*}
\centering\vspace{-0.25in}
% fgr='pix_rth_blue_marble_july_8km';cd ${DATA}/fgr/smn_wth;/bin/rm -f ${fgr}.*;wget --output-document=${fgr}.jpg http://earthobservatory.nasa.gov/Newsroom/BlueMarble/images_bmng/8km/world.topo.bathy.200407.3x5400x2700.jpg
\includegraphics[width=\hsize,angle=0,clip=true,trim=0.0in 0.0in 0.0in 0.0in]{/data/zender/fgr/smn_wth/pix_rth_blue_marble_july_8km}%
\caption{
``Blue Marble''---Earth without clouds in July
(\href{http://earthobservatory.nasa.gov/Newsroom/BlueMarble/images_bmng/8km/world.topo.bathy.200407.3x5400x2700.jpg}{earthobservatory.nasa.gov}).
\label{fgr:pix_rth_blue_marble_july_8km}}
\end{figure*}
\clearpage

\date{} % Empty braces turns off date
\maketitle
\begin{screen}
\vfill
\end{screen}
\clearpage

\csznote{
\begin{abstract}
\Large
\setlength{\baselineskip}{12.0pt} % 1.234 X 11pt
\noindent \smnttl
\end{abstract}
\clearpage
} % end csznote

% Do not use \tableofcontents command in document when paneltoc is option
\begin{print}
\tableofcontents
\end{print}
\begin{screen}
\vfill
\end{screen}

\csznote{ % Example text slide
\Large
\section[Foo]{Foo}
Hi
\clearpage
} % end csznote

\Large
\section[Outline]{Outline}
\begin{itemize*}
\item Environmental as Art: Turbulence, Landforms, Rainbows
\item Clouds
\item Precipitation: Rain, Snow, Hail
\item Severe Weather: Tornados, Hurricanes
\item Peaceful weather
\item Extreme dryness
\item Extreme heat
\item Extreme cold
\item Climate change
\end{itemize*}
\clearpage

\Large
\begin{figure*}
\centering\vspace{-0.25in}
% fgr='pix_cld_krm_vrt_AlI';cd ${DATA}/fgr/smn_wth;/bin/rm -f ${fgr}.*;wget --output-document=${fgr}.jpg http://earthasart.gsfc.nasa.gov/images/von_kar_hires.jpg
\includegraphics[height=1.0\vsize,angle=0,clip=true,trim=0.0in 0.5in 0.0in 1.0in]{/data/zender/fgr/smn_wth/pix_cld_krm_vrt_AlI}%
\caption{
Von Karman street clouds over the Aleutian Islands
(\href{http://earthasart.gsfc.nasa.gov/images/von_kar_hires.jpg}{earthasart.gsfc.nasa.gov}).
\label{fgr:pix_cld_krm_vrt_AlI}}
\end{figure*}
\clearpage

\Large
\begin{figure*}
\centering\vspace{-0.25in}
% fgr='pix_dsr_namib';cd ${DATA}/fgr/smn_wth;/bin/rm -f ${fgr}.*;wget --output-document=${fgr}.jpg http://earthasart.gsfc.nasa.gov/images/namib_hires.jpg
\includegraphics[height=1.0\vsize,angle=0,clip=true,trim=0.0in 0.0in 0.0in 0.0in]{/data/zender/fgr/smn_wth/pix_dsr_namib}%
\caption{
Namibian desert
(\href{http://earthasart.gsfc.nasa.gov/images/namib_hires.jpg}{earthasart.gsfc.nasa.gov}).
\label{fgr:pix_dsr_namib}}
\end{figure*}
\clearpage

\Large
\begin{figure*}
\centering\vspace{-0.25in}
% fgr='pix_rnb_dsr';cd ${DATA}/fgr/smn_wth;/bin/rm -f ${fgr}.*;wget --output-document=${fgr}.jpg http://www.weatherwise.org/pgallery04/2004c.jpg
\includegraphics[height=1.0\vsize,angle=0,clip=true,trim=0.0in 0.0in 0.0in 0.0in]{/data/zender/fgr/smn_wth/pix_rnb_dsr}%
\caption{
Rainbow in White Sands desert
(\href{http://www.weatherwise.org/pgallery04/2004c.gif}{www.weatherwise.org}).
\label{fgr:pix_rnb_dsr}}
\end{figure*}
\clearpage

\Large
\begin{figure*}
\centering\vspace{-0.25in}
% fgr='pix_rnb_qdr';cd ${DATA}/fgr/smn_wth;/bin/rm -f ${fgr}.*;wget --output-document=${fgr}.gif http://www.weatherwise.org/pgallery06/2006b.gif;giftopnm ${fgr}.gif > ${fgr}.pnm;pnmtojpeg ${fgr}.pnm > ${fgr}.jpg;/bin/rm -f ${fgr}.gif ${fgr}.pnm
\includegraphics[height=1.0\vsize,angle=0,clip=true,trim=0.0in 0.0in 0.0in 0.0in]{/data/zender/fgr/smn_wth/pix_rnb_qdr}%
\caption{
Double double-rainbows
(\href{http://www.weatherwise.org/pgallery06/2006b.gif}{www.weatherwise.org}).
\label{fgr:pix_rnb_qdr}}
\end{figure*}
\clearpage

\Large
\begin{figure*}
\centering\vspace{-0.25in}
% fgr='pix_cld_mmt_brd';cd ${DATA}/fgr/smn_wth;/bin/rm -f ${fgr}.*;wget --output-document=${fgr}.gif http://www.weatherwise.org/pgallery06/2006l.gif;giftopnm ${fgr}.gif > ${fgr}.pnm;pnmtojpeg ${fgr}.pnm > ${fgr}.jpg;/bin/rm -f ${fgr}.gif ${fgr}.pnm
\includegraphics[width=\hsize,angle=0,clip=true,trim=0.0in 0.0in 0.0in 0.0in]{/data/zender/fgr/smn_wth/pix_cld_mmt_brd}%
\caption{
Mammatus clouds
(\href{http://www.weatherwise.org/pgallery06/2006l.gif}{www.weatherwise.org}).
\label{fgr:pix_cld_mmt_brd}}
\end{figure*}
\clearpage

\Large
\begin{figure*}
\centering\vspace{-0.25in}
% fgr='pix_cld_mmt_sns';cd ${DATA}/fgr/smn_wth;/bin/rm -f ${fgr}.*;wget --output-document=${fgr}.jpg http://www.weatherwise.org/pgallery04/2004b.jpg
\includegraphics[height=1.0\vsize,angle=0,clip=true,trim=0.0in 0.0in 0.0in 0.0in]{/data/zender/fgr/smn_wth/pix_cld_mmt_sns}%
\caption{
Mammatus clouds at sunset
(\href{http://www.weatherwise.org/pgallery04/2004b.gif}{www.weatherwise.org}).
\label{fgr:pix_cld_mmt_sns}}
\end{figure*}
\clearpage

\Large
\begin{figure*}
\centering\vspace{-0.25in}
% fgr='pix_cld_lgt';cd ${DATA}/fgr/smn_wth;/bin/rm -f ${fgr}.*;wget --output-document=${fgr}.gif http://www.weatherwise.org/pgallery06/2006a.gif;giftopnm ${fgr}.gif > ${fgr}.pnm;pnmtojpeg ${fgr}.pnm > ${fgr}.jpg;/bin/rm -f ${fgr}.gif ${fgr}.pnm
\includegraphics[width=\hsize,angle=0,clip=true,trim=0.0in 0.0in 0.0in 0.0in]{/data/zender/fgr/smn_wth/pix_cld_lgt}%
\caption{
Mammatus clouds with lightning
(\href{http://www.weatherwise.org/pgallery06/2006a.gif}{www.weatherwise.org}).
\label{fgr:pix_cld_lgt}}
\end{figure*}
\clearpage

\Large
\begin{figure*}
\centering\vspace{-0.25in}
% fgr='pix_cherrapunji_sgn';cd ${DATA}/fgr/smn_wth;/bin/rm -f ${fgr}.*;wget --output-document=${fgr}.jpg http://upload.wikimedia.org/wikipedia/commons/f/ff/Cherrapunji.jpg
\includegraphics[height=0.97\vsize,angle=0,clip=true,trim=0.0in 0.5in 0.0in 0.0in]{/data/zender/fgr/smn_wth/pix_cherrapunji_sgn}%
\caption{
Cherrapunji, India receives on average 450\,in.\ (38\,\ft) of rain per year.
In 1861, it received 904\,in.\ (80\,\ft); 366\,in. (31\,\ft) fell in July alone.
(\href{http://upload.wikimedia.org/wikipedia/commons/f/ff/Cherrapunji.jpg}{wikimedia.org}).
\label{fgr:pix_cherrapunji_sgn}}
\end{figure*}
\clearpage

\Large
\begin{figure*}
\centering\vspace{-0.25in}
% fgr='pix_snw_flk_car';cd ${DATA}/fgr/smn_wth;/bin/rm -f ${fgr}.*;wget --output-document=${fgr}.gif http://www.weatherwise.org/pgallery06/2006p.gif;giftopnm ${fgr}.gif > ${fgr}.pnm;pnmtojpeg ${fgr}.pnm > ${fgr}.jpg;/bin/rm -f ${fgr}.gif ${fgr}.pnm
\includegraphics[height=1.0\vsize,angle=0,clip=true,trim=0.0in 0.0in 0.0in 0.0in]{/data/zender/fgr/smn_wth/pix_snw_flk_car}%
\caption{
Snowflakes on a car windshield
(\href{http://www.weatherwise.org/pgallery06/2006p.gif}{www.weatherwise.org}).
\label{fgr:pix_snw_flk_car}}
\end{figure*}
\clearpage

\Large
\begin{figure*}
\centering\vspace{-0.25in}
% fgr='pix_hail_hand';cd ${DATA}/fgr/smn_wth;/bin/rm -f ${fgr}.*;wget --output-document=${fgr}.jpg http://www.chaseday.com/PHOTOSHP/hail/hailhand3.JPG
\includegraphics[width=\hsize,angle=0,clip=true,trim=0.0in 0.0in 0.0in 0.0in]{/data/zender/fgr/smn_wth/pix_hail_hand}%
\caption{
Large hailstone from a tornadic supercell northeast of Breckenridge, Texas.
The record hailstone, 7\,in. diameter, 18.75\,in. circumference, fell
in Aurora, Nebraska on June~22, 2002.
(\href{http://www.chaseday.com/PHOTOSHP/hail/hailhand3.JPG}{www.chaseday.com}).
\label{fgr:pix_hail_hand}}
\end{figure*}
\clearpage

\Large
\begin{figure*}
\centering\vspace{-0.25in}
% fgr='pix_cld_acu_sns';cd ${DATA}/fgr/smn_wth;/bin/rm -f ${fgr}.*;wget --output-document=${fgr}.gif http://www.weatherwise.org/pgallery06/2006c.gif;giftopnm ${fgr}.gif > ${fgr}.pnm;pnmtojpeg ${fgr}.pnm > ${fgr}.jpg;/bin/rm -f ${fgr}.gif ${fgr}.pnm
\includegraphics[width=\hsize,angle=0,clip=true,trim=0.0in 0.0in 0.0in 0.0in]{/data/zender/fgr/smn_wth/pix_cld_acu_sns}%
\caption{
Altocumulus at sunset
(\href{http://www.weatherwise.org/pgallery06/2006c.gif}{www.weatherwise.org}).
\label{fgr:pix_cld_acu_sns}}
\end{figure*}
\clearpage

\Large
\begin{figure*}
\centering\vspace{-0.25in}
% fgr='pix_cld_spr_cll';cd ${DATA}/fgr/smn_wth;/bin/rm -f ${fgr}.*;wget --output-document=${fgr}.jpg http://www.weatherwise.org/pgallery05/2005f.jpg
\includegraphics[height=1.0\vsize,angle=0,clip=true,trim=0.0in 0.0in 0.0in 0.0in]{/data/zender/fgr/smn_wth/pix_cld_spr_cll}%
\caption{
Supercell thunderstorm
(\href{http://www.weatherwise.org/pgallery05/2005f.gif}{www.weatherwise.org}).
\label{fgr:pix_cld_spr_cll}}
\end{figure*}
\clearpage

\Large
\begin{figure*}
\centering\vspace{-0.25in}
% fgr='pix_trn_OK_19990503';cd ${DATA}/fgr/smn_wth;/bin/rm -f ${fgr}.*;wget --output-document=${fgr}.jpg http://upload.wikimedia.org/wikipedia/en/1/1a/Dszpics1.jpg
\includegraphics[width=\hsize,angle=0,clip=true,trim=0.0in 0.0in 0.0in 0.0in]{/data/zender/fgr/smn_wth/pix_trn_OK_19990503}%
\caption{
Tornado in Oklahoma May~3, 1999
(\href{http://upload.wikimedia.org/wikipedia/en/1/1a/Dszpics1.jpg}{wikimedia.org}).
\label{fgr:pix_trn_OK_19990503}}
\end{figure*}
\clearpage

\Large
\begin{figure*}
\centering\vspace{-0.25in}
% fgr='pix_trn_dmg_f5';cd ${DATA}/fgr/smn_wth;/bin/rm -f ${fgr}.*;wget --output-document=${fgr}.jpg http://upload.wikimedia.org/wikipedia/commons/d/df/F5_tornado_damage_example.jpg
\includegraphics[width=\hsize,angle=0,clip=true,trim=0.0in 0.0in 0.0in 0.0in]{/data/zender/fgr/smn_wth/pix_trn_dmg_f5}%
\caption{
Winds in a tornado near Oklahoma City May~3, 1999 reached
318\,mph (509\,\kmxhr).
Winds reached 231\,mph (370\,\kmxhr) at Mount Washington, New
Hampshire on April~12, 1934.
(\href{http://upload.wikimedia.org/wikipedia/commons/d/df/F5_tornado_damage_example.jpg}{wikimedia.org}).
\label{fgr:pix_trn_dmg_f5}}
\end{figure*}
\clearpage

\Large
\begin{figure*}
\centering\vspace{-0.25in}
% fgr='pix_hrr_katrina_20050828';cd ${DATA}/fgr/smn_wth;/bin/rm -f ${fgr}.*;wget --output-document=${fgr}.jpg http://www.osei.noaa.gov/Events/Tropical/Gulf_Mexico/2005/TRCkatrina240_N5.jpg
\includegraphics[height=1.0\vsize,angle=0,clip=true,trim=0.0in 0.0in 0.0in 0.0in]{/data/zender/fgr/smn_wth/pix_hrr_katrina_20050828}%
\caption{
Hurricane Katrina at Category~5 status with winds 160\,mph
(258\,\kmxhr), August~28, 2005.
(\href{http://www.osei.noaa.gov/Events/Tropical/Gulf_Mexico/2005/TRCkatrina240_N5.jpg}{www.osei.noaa.gov}).
\label{fgr:pix_hrr_katrina_20050828}}
\end{figure*}
\clearpage

\Large
\begin{figure*}
\centering\vspace{-0.25in}
% fgr='pix_cld_lnt_pr';cd ${DATA}/fgr/smn_wth;/bin/rm -f ${fgr}.*;wget --output-document=${fgr}.jpg http://www.weatherwise.org/pgallery04/2004a.jpg
\includegraphics[width=\hsize,angle=0,clip=true,trim=0.0in 0.0in 0.0in 0.0in]{/data/zender/fgr/smn_wth/pix_cld_lnt_pr}%
\caption{
Lenticular clouds in Puerto Rico
(\href{http://www.weatherwise.org/pgallery04/2004a.gif}{www.weatherwise.org}).
\label{fgr:pix_cld_lnt_pr}}
\end{figure*}
\clearpage

\Large
\begin{figure*}
\centering\vspace{-0.25in}
% fgr='pix_fog_rdn_lni';cd ${DATA}/fgr/smn_wth;/bin/rm -f ${fgr}.*;wget --output-document=${fgr}.jpg http://www.weatherwise.org/pgallery03/ww2000a.jpg
\includegraphics[width=\hsize,angle=0,clip=true,trim=0.0in 0.0in 0.0in 0.0in]{/data/zender/fgr/smn_wth/pix_fog_rdn_lni}%
\caption{
Radiation fog off Loon Island
(\href{http://www.weatherwise.org/pgallery03/ww2000a.jpg}{www.weatherwise.org}).
\label{fgr:pix_fog_rdn_lni}}
\end{figure*}
\clearpage

\Large
\begin{figure*}
\centering\vspace{-0.25in}
% fgr='pix_dsr_atacama_boulders';cd ${DATA}/fgr/smn_wth;/bin/rm -f ${fgr}.*;wget --output-document=${fgr}.jpg http://www.atacamaphoto.com/atacama/atacama18.jpg
\includegraphics[width=\hsize,angle=0,clip=true,trim=0.0in 0.0in 0.0in 0.0in]{/data/zender/fgr/smn_wth/pix_dsr_atacama_boulders}%
\caption{
Boulders in the Atacama Desert, Peru
(\href{http://www.atacamaphoto.com/atacama/atacama18.jpg}{www.atacamaphoto.com}).
\label{fgr:pix_dsr_atacama_boulders}}
\end{figure*}
\clearpage

\Large
\begin{figure*}
\centering\vspace{-0.25in}
% fgr='pix_dsr_atacama_llareta';cd ${DATA}/fgr/smn_wth;/bin/rm -f ${fgr}.*;wget --output-document=${fgr}.jpg http://www.atacamaphoto.com/atacama/atacama03.jpg
\includegraphics[width=\hsize,angle=0,clip=true,trim=0.0in 0.0in 0.0in 0.0in]{/data/zender/fgr/smn_wth/pix_dsr_atacama_llareta}%
\caption{
Llareta (Azorella compacta), an extremophile flower, at Salar de Huasco, Peru
(\href{http://www.atacamaphoto.com/atacama/atacama03.jpg}{www.atacamaphoto.com}).
\label{fgr:pix_dsr_atacama_llareta}}
\end{figure*}
\clearpage

\Large
\begin{figure*}
\centering\vspace{-0.25in}
% fgr='pix_dst_cld_az';cd ${DATA}/fgr/smn_wth;/bin/rm -f ${fgr}.*;wget --output-document=${fgr}.jpg http://www.weatherwise.org/pgallery04/2004e.jpg
\includegraphics[width=\hsize,angle=0,clip=true,trim=0.0in 0.0in 0.0in 0.0in]{/data/zender/fgr/smn_wth/pix_dst_cld_az}%
\caption{
Dust storm in Arizona
(\href{http://www.weatherwise.org/pgallery04/2004e.gif}{www.weatherwise.org}).
\label{fgr:pix_dst_cld_az}}
\end{figure*}
\clearpage

\Large
\begin{figure*}
\centering\vspace{-0.25in}
% fgr='pix_dst_snt_ana_20020220';cd ${DATA}/fgr/smn_wth;/bin/rm -f ${fgr}.*;wget --output-document=${fgr}.jpg http://www.jpl.nasa.gov/images/earth/usa/misr_022002_browse.jpg
\includegraphics[height=\vsize,angle=0,clip=true,trim=0.0in 0.0in 0.0in 0.0in]{/data/zender/fgr/smn_wth/pix_dst_snt_ana_20020220}%
\caption{
Santa Ana winds carry desert dust over Vista Verde Elementary School,
February~20, 2002.
(\href{http://www.jpl.nasa.gov/images/earth/usa/misr_022002_browse.jpg}{www.jpl.nasa.gov}).
\label{fgr:pix_dst_snt_ana_20020220}}
\end{figure*}
\clearpage

\Large
\begin{figure*}
\centering\vspace{-0.25in}
% fgr='pix_dsr_death_valley_racetrack';cd ${DATA}/fgr/smn_wth;/bin/rm -f ${fgr}.*;wget --output-document=${fgr}.jpg http://k47.pbase.com/u44/zeroscan/upload/28652293.CRW_4695copy.jpg
\includegraphics[width=\hsize,angle=0,clip=true,trim=0.0in 0.0in 0.0in 0.0in]{/data/zender/fgr/smn_wth/pix_dsr_death_valley_racetrack}%
\caption{
Death Valley, California reached 134\,\dgrf\ (56.6\,\dgrc) July~10, 1913. 
El~Azizia (Al~'Aziziyah), Libya reached 135.9\,\dgrf\ (57.7\,\dgrc) September~13, 1922.
(\href{http://k47.pbase.com/u44/zeroscan/upload/28652293.CRW_4695copy.jpg}{k47.pbase.com}).
\label{fgr:pix_dsr_death_valley_racetrack}}
\end{figure*}
\clearpage

\Large
\begin{figure*}
\centering\vspace{-0.25in}
% fgr='pix_antarctica_halo';cd ${DATA}/fgr/smn_wth;/bin/rm -f ${fgr}.*;wget --output-document=${fgr}.jpg http://www.coolantarctica.com/gallery2/weather/images/weather_1000_solar_halo1.jpg
\includegraphics[width=\hsize,angle=0,clip=true,trim=0.0in 0.0in 0.0in 0.0in]{/data/zender/fgr/smn_wth/pix_antarctica_halo}%
\caption{
The Russian Base Vostok in Antarctica reached $-129$\,\dgrf\ ($-89$\,\dgrc) in 1983
(\href{http://www.coolantarctica.com/gallery2/weather/images/weather_1000_solar_halo1.jpg}{www.coolantarctica.com}).
\label{fgr:pix_antarctica_halo}}
\end{figure*}
\clearpage

\Large
\begin{figure*}
\centering\vspace{-0.25in}
% fgr='glc_mlt_20020606';cd ${DATA}/fgr/smn_wth;/bin/rm -f ${fgr}.*;wget --output-document=${fgr}.jpg http://fxm
\includegraphics[height=1.0\vsize,angle=0,clip=true,trim=0.0in 0.0in 0.0in 0.0in]{/home/zender/ppr_ZFM07/fgr/glc_mlt_20020606_Grn.jpg}%
\caption{
Meltwater stream flowing into a large moulin in the ablation zone (area below the equilibrium line) of the Greenland ice sheet. Photo Courtesy: Roger J. Braithwaite, University of Manchester, UK.
\label{fgr:glc_mlt_20020606}}
\end{figure*}
\clearpage

\Large
\begin{figure*}
\centering\vspace{-0.25in}
% fgr='pix_plr_br_mlt';cd ${DATA}/fgr/smn_wth;/bin/rm -f ${fgr}.*;wget --output-document=${fgr}.jpg http://img.dailymail.co.uk/i/pix/2007/02_1/polarbrsDM010207_468x762.jpg
\includegraphics[height=1.0\vsize,angle=0,clip=true,trim=0.0in 0.0in 0.0in 0.0in]{/data/zender/fgr/smn_wth/pix_plr_br_mlt}%
\caption{
Polar bears stranded on melting ice
(\href{http://www.ec.gc.ca/EnviroZine}{www.ec.gc.ca/EnviroZine}). 
Photo \copyright{2006} Envirozine.
\label{fgr:pix_plr_br_mlt}}
\end{figure*}
\clearpage

\Large
\begin{figure*}
\centering\vspace{-0.25in}
% fgr='pix_earthrise';cd ${DATA}/fgr/smn_wth;/bin/rm -f ${fgr}.*;wget --output-document=${fgr}.jpg http://dayton.hq.nasa.gov/IMAGES/LARGE/GPN-2001-000009.jpg
\includegraphics[height=0.95\hsize,angle=270,clip=true,trim=0.0in 0.0in 0.0in 0.0in]{/data/zender/fgr/smn_wth/pix_earthrise}%
\caption{
Earthrise from Apollo~8, December~29, 1968
(\href{http://dayton.hq.nasa.gov/IMAGES/LARGE/GPN-2001-000009.jpg}{NASA}).
\label{fgr:pix_earthrise}}
\end{figure*}
\clearpage

\csznote{
\Large
\begin{figure*}
\centering\vspace{-0.25in}
% fgr='foo';cd ${DATA}/fgr/smn_wth;/bin/rm -f ${fgr}.*;wget --output-document=${fgr}.jpg url
\includegraphics[width=\hsize,angle=0,clip=true,trim=0.0in 0.0in 0.0in 0.0in]{/data/zender/fgr/smn_wth/foo}%
\caption{
Hi
(\href{url}{url}).
\label{fgr:foo}}
\end{figure*}
\clearpage
} % end csznote

\section[Learning More]{Where to Learn More}
\begin{itemize*}
\item \href{http://visibleearth.nasa.gov}{NASA Visible Earth}: Cool pictures
\item \href{http://www.eo.ucar.edu}{NCAR Education}: K--12  Weather/Climate education
\item \href{http://www.weather.com}{The Weather Channel}: Breaking weather
\item \href{http://www.ess.uci.edu/~zender}{Zender Group Website}: Our climate research
\end{itemize*}

\csznote{
\section[References]{References}
% Bibliography
%\renewcommand\refname{\normalsize Publications}
\bibliographystyle{jas}
\bibliography{bib}
\vfill
} % end csznote

\csznote{
% Usage: Place usage here at end of file so comment character % not needed
cd ~/crr;pdflatex smn_wth.tex;cd -
cd ~/crr;make -W smn_wth.tex smn_wth.dvi smn_wth.ps smn_wth.pdf;cd -
cd ~/crr;pdflatex smn_wth.tex;thumbpdf smn_wth;pdflatex smn_wth.tex;cd -
cd ~/crr;texcln smn_wth;make smn_wth.pdf;bibtex smn_wth;makeindex smn_wth;make smn_wth.pdf;bibtex smn_wth;makeindex smn_wth;make smn_wth.pdf;cd -
cd ~/crr;make -W smn_wth.tex smn_wth.pdf;cd -

scp ${DATA}/ps/smn_wth.pdf dust.ess.uci.edu:/var/www/html/smn/smn_wth.pdf
cp ${DATA}/ps/smn_wth.pdf /media/USB20FD
} % end csznote

% $: rebalance syntax highlighting

\end{document}

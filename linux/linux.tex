% $Id$ -*-LaTeX-*-

% Purpose: Notes on Linux

% Copyright (c) 1998--2017, Charles S. Zender
% Permission is granted to copy, distribute and/or modify this document
% under the terms of the GNU Free Documentation License (GFDL), Version 1.3
% or any later version published by the Free Software Foundation;
% with no Invariant Sections, no Front-Cover Texts, and no Back-Cover Texts.
% GFDL: http://www.gnu.org/copyleft/fdl.html

% The original author of this software, Charlie Zender, wants to improve
% it with your suggestions, contributions, bug-reports, and patches.
% Charlie Zender <zender at uci dot edu>
% Department of Earth System Science
% University of California at Irvine
% Irvine, CA 92697-3100

% URL: http://dust.ess.uci.edu/doc/linux/linux.pdf

% Usage (see also end of file):
% cd ~/sw/linux;make linux.pdf;cd -

\documentclass[12pt,twoside]{article}

% Standard packages
\usepackage{ifpdf} % Define \ifpdf
\ifpdf % We are running PDFLaTeX
\usepackage[pdftex]{graphicx} % Defines \includegraphics*
\pdfcompresslevel=9
\usepackage{thumbpdf} % Generate thumbnails
\else % We are not running PDFLaTeX
\usepackage{graphicx} % Defines \includegraphics*
\fi % endif PDFLaTeX
\usepackage{array} % Table and array extensions, e.g., column formatting
\usepackage[dayofweek]{datetime} % \xxivtime, \ordinal
\usepackage{natbib} % \cite commands from aguplus
\usepackage{makeidx} % Index keyword processor: \printindex and \see
\usepackage{mdwlist} % Compact list formats \itemize*, \enumerate*
\usepackage{times} % Postscript Times-Roman font KoD99 p. 375
\usepackage{tocbibind} % Add Bibliography and Index to Table of Contents
\usepackage{url} % Typeset URLs and e-mail addresses
% fxm: 20001028 /usr/share/texmf/tex/latex/base/showidx.sty breaks hyperref
%\usepackage{showidx} % Print index entries as marginal notes

% hyperref is last package since it redefines other packages' commands
% hyperref options, assumed true unless =false is specified:
% backref       List citing sections after bibliography entries
% baseurl       Make all URLs in document relative to this
% bookmarksopen Unknown
% breaklinks    Wrap links onto newlines
% colorlinks    Use colored text for links, not boxes
% hyperindex    Link index to text
% plainpages=false Suppress warnings caused by duplicate page numbers
% pdftex        Conform to pdftex conventions
% Colors used when colorlinks=true:
% linkcolor     Color for normal internal links
% anchorcolor   Color for anchor text
% citecolor     Color for bibliographic citations in text
% filecolor     Color for URLs which open local files
% menucolor     Color for Acrobat menu items
% pagecolor     Color for links to other pages
% urlcolor      Color for linked URLs
\ifpdf % We are running PDFLaTeX
\usepackage[backref,breaklinks,colorlinks,citecolor=blue,linkcolor=blue,urlcolor=blue,hyperindex,plainpages=false,pdftex]{hyperref} % Hyper-references
\pdfcompresslevel=9
\else % We are not running PDFLaTeX
\usepackage[backref,breaklinks,colorlinks=false,hyperindex,plainpages=false]{hyperref} % Hyper-references
\fi % endif PDFLaTeX

% Personal packages
\usepackage{csz} % Library of personal definitions
\usepackage{abc} % Alphabet as three letter macros
\usepackage{dmn} % Dimensional units

% Commands which must be executed in preamble
\makeglossary % Glossary described on KoD95 p. 221
\makeindex % Index described on KoD95 p. 220

% Commands specific to this file

% Margins
\topmargin -24pt \headheight 12pt \headsep 12pt
\textheight 9in \textwidth 6.5in
\oddsidemargin 0in \evensidemargin 0in
\marginparwidth 0pt \marginparsep 0pt
\footskip 24pt
\footnotesep=0pt

\begin{document}

\begin{center}
Online: \url{http://dust.ess.uci.edu/doc} \hfill Updated: \shortdate\today, \xxivtime\\
\bigskip
{\Large \textbf{Linux Cheat Sheet}}\\
\bigskip
by Charlie Zender\\
University of California at Irvine\\
\end{center}
Department of Earth System Science \hfill zender@uci.edu\\
University of California \hfill Voice: (949)\thinspace 824-2987\\
Irvine, CA~~92697-3100 \hfill Fax: (949)\thinspace 824-3256

\pagenumbering{roman}
\setcounter{page}{1}
\pagestyle{headings}
\thispagestyle{empty}
%\onecolumn
\tableofcontents
\listoftables
\pagenumbering{arabic}
\setcounter{page}{1}
%\markleft{Linux Cheat Sheet}
%\markright{}
\thispagestyle{empty}

\clearpage
\section{CCSM: Community Climate System Model}
\begin{verbatim}
# Preliminary system modification
sudo ln -s /usr/bin/make /usr/bin/gmake
aptitude install netcdf mpich mpi-doc # Debian GNU/Linux
\end{verbatim}
\subsection{CAM: Community Atmosphere Model}
\subsection{CLM: Community Land Model}
Make Linux workstation capable of running CLM offline:
\begin{verbatim}
cd ~
# Download and unpack CLM 2.1 source code into your home directory
http://www.cgd.ucar.edu:8080/accept/license?action=fillOut&file_id=7
# Use CVS to check-out model if you have an NCAR login:
cvs -d :ext:zender@goldhill.cgd.ucar.edu:/fs/cgd/csm/models/CVS.REPOS \
co -r clm2_deva_52 -d clm2_deva_52 clm2

# Create model run space
sudo mkdir /ptmp
sudo chmod 777 /ptmp
if [ -n "${LOGNAME}" ]; then export LOGNAME=${USER}; fi
mkdir /ptmp/${LOGNAME}

# Create and populate boundary condition directories
export CSMDATA_NCAR=/fs/cgd/csm/inputdata/lnd/clm2
export CSMDATA_UCI=/datashare/csm/inputdata/lnd/clm2
export CSMDATA_UCI=/ptmp/${USER}/inputdata/lnd/clm2
export CSMDATA_UCI=${HOME}/tmp/inputdata/lnd/clm2
if [ ! -d ${CSMDATA_UCI} ]; then sudo mkdir -p ${CSMDATA_UCI}; fi
data_drc_fll_lst="NCEPDATA" # Directories to be copied in full
# Directories to selectively copy
data_drc_prt_lst="inidata_2.1/cam srfdata/cam pftdata rawdata rtmdata"
# Create all boundary condition directories
for drc in ${data_drc_fll_lst} ${data_drc_prt_lst}; do
  mkdir -p ${CSMDATA_UCI}/${drc}
done
# Populate entire directories
for drc in ${data_drc_fll_lst}; do
  scp -r "babyblue.ucar.edu:${CSMDATA_NCAR}/${drc}/*" ${CSMDATA_UCI}/${drc}
done # done populating directories
# Populate selected directories with selected files
for fl in \
inidata_2.1/cam/clmi_0000-09-01_64x128_T42_c021125.nc \
pftdata/pft-physiology \
rawdata/mksrf_glacier.nc \
rawdata/mksrf_lai.nc \
rawdata/mksrf_lanwat.nc \
rawdata/mksrf_navyoro_20min.nc \
rawdata/mksrf_pft.nc \
rawdata/mksrf_soicol_clm2.nc \
rawdata/mksrf_soitex.10level.nc \
rawdata/mksrf_urban.nc \
rtmdata/rdirc.05 \
srfdata/cam/clms_64x128_T42_c020514.nc \
; do
if [ ! -f ${CSMDATA_UCI}/${fl} ] ; then 
  scp babyblue.ucar.edu:${CSMDATA_NCAR}/${fl} ${CSMDATA_UCI}/${fl}
fi # endif
done

export USER_FC=lf95
cd ~/clm;
clm.sh
\end{verbatim}
\clearpage

\section{ESS~200B: Earth System Physics}
\begin{verbatim}
Atmospheric Physics and Radiation

Course Description:
Physical processes which mediate the transformation of energy, heat, 
and momentum in the Climate system. Topics include hydrostatics,
convection, radiative forcing and climate feedbacks.

Here are topics for the 5 Week ESS Atm. Phys/Radiation mini-course.
Each bullet is one week, and is subdivided into two parts:

1. Hydrostatics, Thermodynamics
2. Buoyancy/Convection, Radiance/Absorption
3. Flux/Emission, Scattering/Actinic Flux
4. Clouds/Direct Effects, Aerosols/Indirect Effects
5. Earth's Energy Budget, T<-->Climate interactions

The text has been chosen to dovetail with other courses,
``Global Physical Climatology'' by Hartmann.
Each day's topic (Tues./Thurs. structure) comprises the key 
points enumerated below, which are to be illustrated by theoretical
and practical examples from the course text and from current research.

Hydrostatics:
1. Mass, density, pressure
2. Ideal gas law
3. Hydrostatic equilibrium
4. Dry lapse rate
5. Realistic T profiles

Thermodynamics:
1. Work, Expansion compression
2. Heat, heat capacity
3. Adiabatic vs. Diabatic
4. Saturation
5. Condensation, Evaporation
6. Latent heat
7. Moist lapse rate

Earth's Energy Budget:
1. Planetary Radiative Equilibrium
2. Energy Transformations
3. Dry, Moist Static Energy
4. Gravitational Potential Energy
5. Available Energy
6. Global Mean Energy Storage

Climate Models:
1. Planetary Radiative Equilibrium
2. Boltzmann's Law, Effective Temperature
3. Solar and Infrared Radiative Budgets
4. Greenhouse Effect
5. Forcing, Response, Sensitivity
6. 0-D, 1-D Climate models
7. Multiple Climate Equilibria

Buoyancy/Convection:
1. Dry/Moist static energy
2. Stability/Instability
3. Convectively Available Potential Energy
4. Simple convection: Adiabatic adjustment
5. Convective redistribution of heat, vapor, tracers

Flux, Emission:
1. Solid angles/Radiance/Isotropy
2. Blackbody radiation 
3. Radiance-Flux relationships
4. Boltzmann's law
5. Radiative equilibrium T structure of grey atmosphere

Absorption/Extinction:
1. Molecular cross-sections/spectra
2. Absorption: Vibration, rotation bands
3. Extinction
4. Beer's law
5. Explaining the observed spectrum of the clear atmosphere
6. GHGs and the atmospheric window

Scattering/actinic flux:
1. Reflection
2. Direct vs. diffuse radiation
3. Particle-light interaction
4. Size parameters, Mie theory
5. Photochemical radiation
6. PAR, NDVI definitions

Clouds:
1. Formation, Lifecycle
2. Cloud climatology (cloud classification taught in Hydrology)
3. Particle size distributions
4. Radiative properties
5. Liquid/ice distinctions
6. Direct radiative forcing: LW, SW, Net

Aerosols:
1. Formation, Lifecycle
2. Global sources
3. Sinks: scavenging, dry deposition
4. Global distribution & radiative forcing
7. Vertical flux/heating profiles for realistic atmospheres

Indirect Effects:
1. Daisyworld?
2. Indirect forcing via clouds
3. Surface/Atmosphere partitioning of absorption
4. Global energy budget, including latent, sensible heat
5. Meridional heat transport (atm, ocn)
6. Nutrient transport?

T<-->Climate interactions, Physical Climate Change
1. Coupling of troposphere/stratosphere change
2. Coupling of troposphere/land surface change
3. Water vapor feedback
4. Effects of altered lapse rates
5. Thermal expansion of oceans
6. Estimating RF of GHG increases
7. Observed indirect effects

Notes on course design:

Topics which might belong in other courses which
I assume will be covered here (and thus might be duplicated):
1. Buoyancy/convection (maybe in Boundary layer)

Topics which might belong here but which I assume
will be covered in other courses (and thus might fall through 
the cracks):
1. Ice/albedo feedback (Hydrology cryosphere)
2. Mass continuity equation (Dynamics)
3. Advection (Dynamics or Geosci. Model)
4. Diffusion (Dynamics or Geosci. Model)
5. Box models, feedbacks, residence times (Hydrology)
6. Aerosol composition, nutrients, pH (Chemistry)
7. Gibbs/Helmholtz energy (Terr BGC)

Topics which might belong here but which I assume
will not be covered in any first year unit due to time contraints.
These topics should be available in an advanced graduate course.
1. Formulation of RT Equation
2. Solution for pure absorbing atmosphere with reflecting lower boundary
3. Definition of single scattering, asymmetry
4. Radiative-convective equilibrium?
\end{verbatim}
\clearpage

\subsection{iPod}\label{sxn:ipod}
The primary reference for using the Apple \trmidx{iPod} with Linux is 
\url{http://www.cavecanen.org/linux/ipod}.
Two Linux applications for managing iPod music are \cmdidx{GNUpod} and 
\cmdidx{gtkpod}.

\subsection{Cameras}\label{sxn:cmr}
If your Linux uses modules, load the following modules:
Support for USB. Also select either UHCI (Intel PIIX4, VIA, ...)
support, UHCI Alternate Driver (JE) support or OHCI-HCD (Compaq,
iMacs, OPTi, SiS, ALi, ...) support.  
\begin{verbatim}
usbcore.o
usb-uhci.o, uhci.o or usb-ohci.o
\end{verbatim}
In particular, \trmidx{CompactFlash} relies on the \trmidx{USB}
driver. 

\subsubsection{CompactFlash}\label{sxn:cf}
To transfer files from your Camera's CompactFlash to disk, first
mount the CompactFlash card as a \cmdidx{vfat} file system:
\begin{verbatim}
mount -t vfat /dev/hde1 /cf # CompactFlash, root-only
\end{verbatim}
This CompactFlash is enumerated as an \trmidx{IDE} drive here, because
it is inserted into the system on the \trmidx{PCMCIA} bus (in a PCMCIA
adaptor).
CompactFlash to USB adaptors are at least as common.
The USB sub-system uses SCSI drivers, so CompactFlash drives will be
enumerated as SCSI drives (e.g., \flidx{/dev/sda1}) when mounted
through a USB port.

By default, CompactFlash devices require root priveleges---they are
not user-writable unless opened with the appropriate mask:
\begin{verbatim}
mount -t vfat -o umask=022 /dev/hde1 /cf # CompactFlash, users R/W
\end{verbatim}
In this mode users may read and write files but can not actually
own files or directories.

The \cmdprn{vfat} \trmidx{filesystem} does not support multiple user
identities and priveleges.
However, the device may be mounted with a particular \trmidx{UID} and
\trmidx{GID} which will allow full user read-write access.
\begin{verbatim}
sudo mount -t vfat -o umask=022,uid=3555,gid=2400 /dev/hde1 /cf # CompactFlash, user zender
\end{verbatim}
This is the recommended way to mount CompactFlash used for backup
and file transfer purposes. 

\subsubsection{Memory Stick}\label{sxn:memstick}
A Memory Stick, aka a \trmidx{USB Flash Drive},  is very similar to
CompactFlash device. 
As always the Flash Memory HOWTO is very useful.

To quickly backup and copy material to my black attache memory stick,
use  
\begin{verbatim}
sudo mkdir -p /media/USB20FD/${HOME} /media/USB20FD/${DATA}
sudo chmod 755 /media/USB20FD/${HOME} /media/USB20FD/${DATA}
sudo chown zender /media/USB20FD/${HOME} /media/USB20FD/${DATA}
sudo chgrp cgdcsm /media/USB20FD/${HOME} /media/USB20FD/${DATA}
/bin/cp -p ${HOME}/jrn/TODO /media/USB20FD
/bin/cp -p ${DATA}/ppr/ppr_DSL08.pdf /media/USB20FD
/bin/cp -p -r ${DATA}/annecy /media/USB20FD/data/zender
/bin/cp -p -r ${DATA}/lyon /media/USB20FD/data/zender
/bin/cp -p -r ${DATA}/vienna /media/USB20FD/data/zender
\end{verbatim}
To quickly backup and copy material from \mchprn{virga} to my grey 
Attache memory stick, use  
\begin{verbatim}
sudo mkdir -p /media/disk/${HOME} /media/disk/${DATA}
sudo chmod 755 /media/disk/${HOME} /media/disk/${DATA}
sudo chown zender /media/disk/${HOME} /media/disk/${DATA}
sudo chgrp cgdcsm /media/disk/${HOME} /media/disk/${DATA}
/bin/cp -p ${HOME}/jrn/TODO /media/disk
/bin/cp -p ${DATA}/ppr/ppr_DSL08.pdf /media/disk
/bin/cp -p -r ${DATA}/annecy /media/disk/data/zender
/bin/cp -p -r ${DATA}/lyon /media/disk/data/zender
/bin/cp -p -r ${DATA}/vienna /media/disk/data/zender
\end{verbatim}
To synchronize transient and sensitive data directories
on two sides of a firewall, copy data to an obscure intermediate
machine accessible from both sides of the firewall:
\begin{verbatim}
# To repository:
rsync /data/zender/dt grele.ess.uci.edu:/data/zender
rsync /data/zender/dvr grele.ess.uci.edu:/data/zender
rsync /data/zender/ess grele.ess.uci.edu:/data/zender
rsync /data/zender/hire grele.ess.uci.edu:/data/zender
rsync /data/zender/mny grele.ess.uci.edu:/data/zender
rsync /data/zender/prp grele.ess.uci.edu:/data/zender
rsync /data/zender/rvw grele.ess.uci.edu:/data/zender
rsync /data/zender/scidac grele.ess.uci.edu:/data/zender
rsync /data/zender/trv grele.ess.uci.edu:/data/zender

rsync /data/zender/bc dust.ess.uci.edu:/data/zender
rsync /data/zender/bc_health dust.ess.uci.edu:/data/zender
rsync /data/zender/esdswg dust.ess.uci.edu:/data/zender
rsync /data/zender/ess_cnt dust.ess.uci.edu:/data/zender
rsync /data/zender/ess_gpc dust.ess.uci.edu:/data/zender
rsync /data/zender/fgr dust.ess.uci.edu:/data/zender
rsync /data/zender/harvard dust.ess.uci.edu:/data/zender
rsync /data/zender/house dust.ess.uci.edu:/data/zender
rsync /data/zender/iusd dust.ess.uci.edu:/data/zender
rsync /data/zender/kdz dust.ess.uci.edu:/data/zender
rsync /data/zender/prp_msb dust.ess.uci.edu:/data/zender
rsync /data/zender/ppr dust.ess.uci.edu:/data/zender
rsync /data/zender/reu dust.ess.uci.edu:/data/zender
rsync /data/zender/smn dust.ess.uci.edu:/data/zender
rsync /data/zender/stephenson dust.ess.uci.edu:/data/zender
rsync /data/zender/wenshanw dust.ess.uci.edu:/data/zender

# From repository:
rsync grele.ess.uci.edu:/data/zender/dt /data/zender
rsync grele.ess.uci.edu:/data/zender/dvr /data/zender
rsync grele.ess.uci.edu:/data/zender/ess /data/zender
rsync grele.ess.uci.edu:/data/zender/hire /data/zender
rsync grele.ess.uci.edu:/data/zender/mny /data/zender
rsync grele.ess.uci.edu:/data/zender/prp /data/zender
rsync grele.ess.uci.edu:/data/zender/rvw /data/zender
rsync grele.ess.uci.edu:/data/zender/scidac /data/zender
rsync grele.ess.uci.edu:/data/zender/trv /data/zender

rsync dust.ess.uci.edu:/data/zender/bc /data/zender
rsync dust.ess.uci.edu:/data/zender/bc_health /data/zender
rsync dust.ess.uci.edu:/data/zender/esdswg /data/zender
rsync dust.ess.uci.edu:/data/zender/ess_cnt /data/zender
rsync dust.ess.uci.edu:/data/zender/ess_gpc /data/zender
rsync dust.ess.uci.edu:/data/zender/fgr /data/zender
rsync dust.ess.uci.edu:/data/zender/harvard /data/zender
rsync dust.ess.uci.edu:/data/zender/house /data/zender
rsync dust.ess.uci.edu:/data/zender/iusd /data/zender
rsync dust.ess.uci.edu:/data/zender/kdz /data/zender
rsync dust.ess.uci.edu:/data/zender/ppr /data/zender
rsync dust.ess.uci.edu:/data/zender/reu /data/zender
rsync dust.ess.uci.edu:/data/zender/smn /data/zender
rsync dust.ess.uci.edu:/data/zender/stephenson /data/zender
rsync dust.ess.uci.edu:/data/zender/wenshanw /data/zender
\end{verbatim}

In the old days, to mount a \trmidx{Memory Stick} one did this: 
\begin{verbatim}
# Install necessary sysfs utilities
aptitude install hotplug sysfsutils udev
# Create a mountpoint for sysfs
cd /
mkdir sys
mount -t sysfs none /sys
# Mount USB Memory Stick
sudo mkdir -m 777 /memstick
sudo mount -t vfat -o umask=022,uid=3555,gid=2400 /dev/sda1 /memstick # Ashes
sudo mount -t vfat -o umask=022,uid=3555,gid=2400 /dev/sdb1 /memstick # Virga
\end{verbatim}
For Linux 2.6 kernels, make sure to install \trmidx{udev}.

The Memory stick with a USB adaptor appears automatically in the
\flidx{/sys} filesystem as a \trmidx{SCSI} drive:
\begin{verbatim}
zender@elnino:~$ more /sys/block/sda/device/model 
IntelligentStick
zender@elnino:~$ more /sys/block/sda/device/vendor
I-Stick2
\end{verbatim}
It must still be mounted.
See the \cmdidx{rsync} description in Section~\ref{sxn:rsync} for
examples of how to back up to CompactFlash and Memory Stick media.
I bought a PNY~Technologies ``Attache'' 4~GB USB 2.0 Flash drive on 20060729.
I bought a second PNY~Technologies ``Attache'' 4~GB USB 2.0 Flash drive on 20071006.

\section{X and XFree86}\label{sxn:x}

As of \trmidx{Ubuntu} 9.04, \trmidx{Jaunty} Jackalope, the 
\kbdidx{Ctrl+Alt+Backspace} no longer restarts the \trmprn{X} server.
\begin{verbatim}
sudo aptitude install dontzap
sudo dontzap --disable # Ctrl+Alt+Backspace restarts the xserver
sudo dontzap --enable # Ctrl+Alt+Backspace does nothing
\end{verbatim}

\trmidx{X} tends to have large memory leaks. 
Sometimes the problems are due to the X~server, and sometimes the
problems are due to programs which never free graphics space requested
by the X~server. 
The \cmdidx{xrestop} monitors resources consumed by the X~server.

\subsection{X and XFree86: Configuration}\label{sxn:x_cnf}
A configuration tool has been built directly into the \trmidx{XFree86}
server accessible with \cmdprn{XFree86 -configure}.
\cmdprn{XFree86 -configure} generated a perfectly working
\flidx{XF86Config} file for \flidx{lanina}, and is the method I now 
recommend. 
XFree86~4.0.1 also has a different and new configuration tool,
\cmdidx{xf86cfg}. 
\cmdprn{xf86cfg} did not generate a working/optimal configuration file
for \flprn{lanina}, so I do not recommend this method.
Both these methods should generate fairly good configuration files
which must be stored as \flidx{XF86Config-4}, e.g.,
\flprn{/etc/X11/XF86Config-4}. 
Note the extra "-4", which indicates suitability for \cmdidx{XFree86-4.x}.
Version~4 uses dynamically loaded modules for each particular chipset 
so there is no need to symbolically link the \trmidx{X Window System}
server (e.g., \flprn{/etc/X11/X}) to anything. 

XFree86 version 3.3.x uses the file \flprn{/etc/X11/XF86Config},
which may be generated by \cmdidx{Xconfigurator}:
\begin{verbatim}
Xconfigurator
cp /etc/X11/XF86Config ${HOME}/linux
cp ${HOME}/linux/XF86Config /etc/X11
\end{verbatim}
In XFree86 version~3 the correct \trmidx{X~server} should be
dynamically linked to the generic server file, e.g., 
\begin{verbatim}
ln -sf /usr/X11R6/bin/XF86_I128 /etc/X11/X
\end{verbatim}

The \trmidx{Google} \cmdidx{googleearth} application requires
correctly functioning \trmidx{OpenGL} drivers.
The \flidx{xorg.conf} file instructs X~to load these drivers with
\begin{verbatim}
Section "Module"
        Load  "glx"
        Load  "GLcore"
EndSection
\end{verbatim}
It is important that \trmidx{glx} precede \trmidx{GLcore}.
The \cmdidx{glxinfo} command reports the OpenGL driver status.

Use \cmdidx{import} to save an X~window to an image format
\begin{verbatim}
import san_antonio.gif
import -quality 100 san_antonio.png
\end{verbatim}

Tweaking the default \flidx{XF86Config} file is often necessary for
special performance.
The following tweaks to \cmdidx{swcursor} and \cmdidx{XkbOptions} are
useful 
\begin{enumerate*}
\item Software cursor to allow large cursors.
In \verb'Section "Device"' add
\begin{verbatim}
# csz++
# 19991005: "swcursor" required for 96 x 94 pixel ~/.lightning cursor
   Option "swcursor"
# csz--
\end{verbatim}
\item Swap positions of capslock and control keys.
In the \cmdidx{InputDevice} section for the keyboard add
\begin{verbatim}
#csz++
# Swap positions of capslock and control keys
   XkbOptions  "ctrl:swapcaps"
#csz++
\end{verbatim}
In the \cmdprn{InputDevice} section for the keyboard add
\begin{verbatim}
#csz++
# Make pressing both touchpad buttons at same time emulate middle
# button on 3 button mice
        Option      "Emulate3Buttons" "on"
#csz++
\end{verbatim}
Adjust the speed of the \cmdidx{touchpad} mouse using the 
\cmdidx{MinSpeed}, \cmdidx{MaxSpeed}, and \cmdidx{AccelFactor},
options in the \cmdprn{Touchpad} section
\begin{verbatim}
Section "InputDevice"
Identifier "Synaptics Touchpad"
Driver "synaptics"
Option "SendCoreEvents" "true"
Option "Device" "/dev/psaux"
Option "Protocol" "auto-dev"
Option "HorizScrollDelta" "0"
Option "MinSpeed" "0.5"
Option "MaxSpeed" "0.7"
Option "AccelFactor" "0.0350"
EndSection
\end{verbatim}
Another way to do this is to add 
\begin{verbatim}
Option "SHMConfig" "on"
\end{verbatim}
and then 
\begin{verbatim}
aptitude install ksynaptics
\end{verbatim}
\end{enumerate*}

Make sure the 100-dpi fonts are installed or else many \cmdidx{xterm}
and \cmdidx{emacs} fonts will appear chunky. 

Font management and installation is described by the \trmidx{Linux
Documentation Project} (\trmidx{LDP}) at
\url{http://www.linuxdoc.org/HOWTO/Font-HOWTO.html}. 
X can be made to use any font desired, but the required 
procedures to do so are arcane.
Here is a skeleton outline:
\begin{verbatim}
ps ax | grep xfs # Check for running font server xfs
xset -q # Check font path
xset fp+ unix/:port_number # Add font server xfs port to font path
xset fp rehash # Rehash font path
sudo /etc/rc.d/init.d/xfs restart # Restart font server xfs
end{verbatim}
None of this should be necessary on a \trmidx{RedHat} GNU/Linux machine,
since RedHat has an ``automatic'' way of serving fonts.
Instead, what should be done on all machines, is to list
in order of preference the valid font directories in the \verb'Files'
section of the \flidx{XF86Config-4} file.
\begin{verbatim}
Section "Files"
        FontPath "/usr/X11R6/lib/X11/fonts/local/"
        FontPath "/usr/X11R6/lib/X11/fonts/misc/"
        FontPath "/usr/X11R6/lib/X11/fonts/75dpi/:unscaled"
        FontPath "/usr/X11R6/lib/X11/fonts/100dpi/:unscaled"
        FontPath "/usr/X11R6/lib/X11/fonts/Type1/"
        FontPath "/usr/X11R6/lib/X11/fonts/CID/"
        FontPath "/usr/X11R6/lib/X11/fonts/Speedo/"
        FontPath "/usr/X11R6/lib/X11/fonts/75dpi/"
        FontPath "/usr/X11R6/lib/X11/fonts/100dpi/"
        FontPath "/usr/share/AbiSuite/fonts/"
        FontPath "/wnd/windows/fonts/"
EndSection
\end{verbatim}
Also, make sure the X-server loads the modules to handle 
\cmdprn{freetype} and Type~1 (Adobe) fonts.
This is accomplished by having 
\begin{verbatim}
Section "Module"
        Load  "freetype"
        Load  "type1"
EndSection
\end{verbatim}
Notice that the font server can use the MS Windows fonts directly from
the Windows fonts directory, \flprn{/wnd/windows/fonts}.
However, the fonts in this directory first must be prepared for use
by running a few commands in that directory:
\begin{verbatim}
cd /wnd/windows/fonts
ttmkfdir -o fonts.scale
mkfontdir
\end{verbatim}
Also, make sure all the fonts are world-readable.

When fooling with X~configurations it is convenient to have a clean
way to shut down and restart all X~processes.
This can be done by initializing the system to a new
\trmidx{runlevel}.  
Most Unices start multi-user mode and all network
applications \textit{except} the \cmdidx{X~server} in runlevel~3.
Runlevel~5 starts the same processes as runlevel~3 and starts
the X~server as well.
Thus initializing a system to a different runlevel (usually 3 or~5)
is a clean way of starting an stopping X~processes.
The \cmdidx{telinit} command serves this purpose.
Shut down all X~processes with \cmdprn{telinit 3}.
Restart   all X~processes with \cmdprn{telinit 5}.
Set the default system runlevel in \flidx{/etc/inittab}.
This is what determines whether \trmidx{X} automatically starts on
reboots. 

Some computers have terrible default X~settings which may be corrected
by use of \cmdidx{xset}, usually done in \flidx{~/.xinitrc}. 
For example, lanina has \acr{DPMS} turned off by default, and very
slow key repeat settings so its \flprn{~/.xinitrc} contains 
\begin{verbatim}
xset q # Show current settings
xset +dpms # Enable DPMS
xset dpms 300 600 1200 # Seconds until Standby, Suspend, Off
xset mouse 4 4 # Mouse acceleration and threshold
\end{verbatim}
The keyboard repeat rate may be changed with the \cmdidx{kbdrate}
command. 
The \cmdprn{-r} options sets the repeat rate in characters per second.
The maximum allowed rate, 30 cps, gives a nice response.
\begin{verbatim}
kbdrate -r 30 # Set keyboard repeat rate to 30 cps
\end{verbatim}
The default mouse movement rate is also susceptible to fubaration.

\cmdidx{xvidtune} generates modelines which center the picture on the
monitor. 

The \cmdidx{XAPPLRESDIR} contains X~defaults for various programs.
\begin{verbatim}
mkdir ${DATA}/app-defaults
export XAPPLRESDIR="${X11}/lib/X11/app-defaults:${DATA}/app-defaults"
\end{verbatim}
% $: rebalance syntax highlighting
These defaults may be over-ridden on a per-user basis by implementing
X resources in the \flidx{.Xdefaults} file.

\subsection{XFree86: NVidia Graphics drivers}
Source code drivers that support all of the advanced features of
theubiquitous \trmidx{NVidia} graphics cards are unavilable.
Two type of drivers are available.
First, the XFree86 project \cmdidx{nv} driver is \trmidx{open source}
and works well with all \trmidx{2D} drawing commands.
Second, NVidia supplies binary-only \cmdidx{nvidia} drivers which 
support full \trmidx{3D} acceleration and \trmidx{OpenGL} graphics. 
The \cmdprn{nvidia} kernel modules are closed source, prone to errors,
and must be re-installed after the kernel is recompiled.
Since there is no easy mechanism to retain multiple \cmdprn{nvidia}
modules for multiple kernels on one machine (e.g., for testing), the
constraints of the closed-source modules are bothersome.

Since these drivers are modules, installing them requires kernel
recompilation which, in turn, requires \trmidx{kernel headers}.
\begin{verbatim}
dpkg --list '*686*' 
apt-cache search kernel | grep 2.6.15 | grep 686

# Ubuntu Breezy Badger
uname_r=`uname -r`
sudo aptitude install linux-image-${uname_r} linux-restricted-modules-${uname_r}
sudo aptitude remove linux-image-${uname_r} linux-headers-${uname_r} linux-restricted-modules-${uname_r} linux-ubuntu-modules-${uname_r}
\end{verbatim}
Note that recent Debian kernel packages require the 
\trmidx{initial ram-disk} (\flidx{initrd}) to be specified in the 
GRUB \flidx{menu.lst}.

\begin{verbatim}
aptitude install libncurses5-dev
cd ${DATA}/zender/tmp
svn checkout http://powertop.googlecode.com/svn/trunk/ powertop
cd ${DATA}/tmp/powertop
make
sudo make install
\end{verbatim}

Linux experts re-compile their own kernals frequently.
GNU/Linux will not boot into \trmidx{X} until the correct X~server
module is available for the current kernel.
Download NVidia drivers from
\url{http://www.nvidia.com/object/linux.html}. 
Integrate the NVidia drivers into these kernels by hand:
\begin{verbatim}
wget http://download.nvidia.com/XFree86/Linux-x86/1.0-7174/NVIDIA-Linux-x86-1.0-7174-pkg1.run
cd /usr/src; sh NVIDIA-Linux-x86-1.0-4496-pkg2.run
cd /usr/src; sh NVIDIA-Linux-x86-1.0-5336-pkg2.run
cd /usr/src; sh NVIDIA-Linux-x86-1.0-6111-pkg1.run
cd /usr/src; sh NVIDIA-Linux-x86-1.0-7174-pkg1.run
\end{verbatim}
When using the closed source NVidia module driver, make sure that
\flidx{XF86Config-4} references \flidx{nvidia} not the open-source
XFree86 driver~\flidx{nv}.

The \trmidx{X.org} and \trmidx{XFree86} X~servers write output from
the initialization process to \flidx{/var/log/Xorg.0.log} and
\flidx{/var/log/XFree86.0.log}, respectively.

Ubuntu Breezy Badger packages for NVIDIA cards:
\begin{verbatim}
sudo aptitude install linux-restricted-modules-2.6.15-16-686 # Non-free Linux 2.6.12 modules on PPro/Celeron/PII/PIII/PIV
sudo aptitude install nvidia-glx nvidia-settings avm-fritz-firmware # NVIDIA binary XFree86 4.x/X.Org driver
sudo aptitude install nvidia-glx-180 # NVIDIA binary drivers
# 
sudo nvidia-xconfig
\end{verbatim}

\subsection{Performance Profiling}\label{sxn:prf}\label{sxn:oprofile}

\mchidx{clay} is set up to do performance profiling by both
\cmdidx{oprofile} and \trmidx{PAPI} with \trmidx{HPCToolkit}.  
This required a kernel patch and recompile.  
There is one module for each approach:
\flidx{oprofile} for oprofile and \flidx{perfctr} for any software
that requires the PAPI modifications (specifically, \cmdidx{hpcrun}
from the HPCToolkit).
There are other performance tools that use the PAPI API (\cmdidx{Tau}, 
for one), but so far we only have HPCToolkit installed. 
As a side note, most \trmidx{Ubuntu} kernels have the
\cmdprn{oprofile} module available, but we are not aware of any with
the more exciting \cmdprn{perfctr} patch applied and the
\cmdprn{perfctr} module available.  

The Oprofile homepage is \url{http://oprofile.sourceforge.net/docs} 
Oprofile first requires
\begin{verbatim}
sudo modprobe oprofile
\end{verbatim}
Second, initialize the \cmdidx{oprofiled} daemon and start it
collecting info. 
This command depends on your exact hardware/software configuration.
\begin{verbatim}
opcontrol --vmlinux=/path/to/vmlinux # Normal generic
opcontrol --no-vmlinux # No vmlinux exists
\end{verbatim}
The path must lead to the \emph{uncompressed} linux ELF executable,
not to typical \flidx{vmlinuz} compressed boot sector that is
installed in the \flidx{/boot} directory.
For \mchprn{clay.ess.uci.edu}:
\begin{verbatim}
sudo opcontrol --vmlinux=/usr/src/linux-2.6.11/vmlinux
# Explicitly say one CPU:
sudo opcontrol --separate=none --vmlinux=/usr/src/linux-2.6.11/vmlinux
# --separate=cpu reports counts on both CPUs
sudo opcontrol --separate=cpu --vmlinux=/usr/src/linux-2.6.11/vmlinux
# Once separated, must explicitly shut off for succeeding runs
sudo opcontrol --start
# Next line kills daemon, removes small amount of overhead:
sudo opcontrol --shutdown  
# Remove profile module (needed for hpcrun)
opcontrol --deinit
\end{verbatim}
In order for hpcrun to work, \cmdidx{perfctr} module must be
\cmdidx{modprobe}-loaded and \flidx{/dev/perfctr} must be mode~644.
When ready to collect info, do a \cmdidx{sudo ls} to initialize the
timeout on the \cmdprn{sudo} command so later commands do not ask for 
passwords.
For \cmdidx{ncbo}, assuming ncbo has been compiled with the
\kbdidx{-g} option, 
\begin{verbatim}
sudo opcontrol --reset # Reset counters
ncbo -O --op_typ='-' -p ${DATA}/nco_bm \
gcm_T85.nc gcm_T85_00.nc ${DATA}/nco_bm/ipcc.diff.nc
opreport --exclude-dependent --demangle=smart --symbols > \
oprofile.report.full.ncbo
\end{verbatim}

The output is a text file that gives the time spent in each function. 
The \trmidx{poll\_idle} time is that time which the CPU(s) has spent
doing nothing, i.e., idling. 
For a lightly loaded dual-CPU machine, you would expect to obtain
about 50\% in \trmprn{poll\_idle} running a single serial job. 

To use \trmidx{HPCToolkit}, make sure that \cmdprn{oprofile} is
not loaded, and load the \cmdprn{perfctr} module.
\begin{verbatim}
lsmod | grep oprofile
sudo opcontrol --deinit # Unload oprofile if loaded
modprobe perfctr # Load perfctr module
\end{verbatim}
The \trmidx{PAPI} API has access to necessary hardware counters once 
the \cmdprn{perfctr} has been loaded.
After this, profiling is relatively straightforward.
To profile a command, prefix it with \cmdidx{hpcrun}, e.g.,
\begin{verbatim}
hpcrun [options] -- ncbo -O --op_typ='-' -p ${DATA}/nco_bm  \
gcm_T85.nc gcm_T85_00.nc ${DATA}/nco_bm/ipcc.diff.nc
\end{verbatim}
The \cmdprn{hpcrun} options are typically a set of hardware counters
you want to access during the run. 
\begin{verbatim}
% hpcrun -L | grep Yes # List available hpcrun options
PAPI_L2_DCM     Yes     Level 2 data cache misses ()
PAPI_L2_ICM     Yes     Level 2 instruction cache misses ()
PAPI_FPU_IDL    Yes     Cycles floating point units are idle ()
PAPI_TLB_DM     Yes     Data translation lookaside buffer misses ()
PAPI_TLB_IM     Yes     Instruction translation lookaside buffer misses ()
PAPI_L1_LDM     Yes     Level 1 load misses ()
PAPI_L1_STM     Yes     Level 1 store misses ()
PAPI_L2_LDM     Yes     Level 2 load misses ()
PAPI_L2_STM     Yes     Level 2 store misses ()
PAPI_STL_ICY    Yes     Cycles with no instruction issue ()
PAPI_HW_INT     Yes     Hardware interrupts ()
PAPI_BR_TKN     Yes     Conditional branch instructions taken ()
PAPI_BR_MSP     Yes     Conditional branch instructions mispredicted ()
PAPI_TOT_INS    Yes     Instructions completed ()
PAPI_FP_INS     Yes     Floating point instructions ()
PAPI_BR_INS     Yes     Branch instructions ()
PAPI_VEC_INS    Yes     Vector/SIMD instructions ()
PAPI_RES_STL    Yes     Cycles stalled on any resource ()
PAPI_TOT_CYC    Yes     Total cycles ()
PAPI_L2_DCH     Yes     Level 2 data cache hits ()
PAPI_L1_DCA     Yes     Level 1 data cache accesses ()
PAPI_L2_DCR     Yes     Level 2 data cache reads ()
PAPI_L2_DCW     Yes     Level 2 data cache writes ()
PAPI_L2_ICH     Yes     Level 2 instruction cache hits ()
PAPI_L1_ICA     Yes     Level 1 instruction cache accesses ()
PAPI_L1_ICR     Yes     Level 1 instruction cache reads ()
PAPI_FML_INS    Yes     Floating point multiply instructions ()
PAPI_FAD_INS    Yes     Floating point add instructions ()
PAPI_FP_OPS     Yes     Floating point operations ()
\end{verbatim}
These options are requested as follows
\begin{verbatim}
hpcrun -e PAPI_TOT_CYC:32767 -e PAPI_FP_OPS:32767 -e PAPI_FP_INS:32767 \
-e PAPI_HW_INT:32767 -e PAPI_L2_DCM:32767 -- <command_to_profile>
\end{verbatim}
\cmdprn{hpcrun} profiles \emph{everything} that results.
For example, \cmdprn{command\_to\_profile} is a shell command, then
\cmdprn{hpcrun} profiles every subcommand in the shell, and gives each
its own output file in the form of: 
\flprn{app\_name.PAPI\_TOT\_CYC.clay.ess.uci.edu.10137.0}.

Process \cmdprn{hpcrun} output files into something usable with
\cmdidx{hpcquick}.  
\cmdprn{hpcquick} is a \trmidx{Perl} script that calls some other
hpctools to generate the XML database (in its own subdirectory) that
\cmdidx{hpcviewer} needs. 
\begin{verbatim}
# src_location      hpct DB file to process
hpcquick -I src/nco -P ncwa.PAPI_TOT_CYC.clay.ess.uci.edu.10137.0
# View results with Java hpcviewer
hpcviewer # Open the './hpcquick.dbxxx/hpcquick.hpcviewer' file
\end{verbatim}

To use \trmidx{oprofile} on \mchprn{clay} requires a re-compiled
kernel with the profiling switches enabled.
Once re-compiled, the kernel may require a new \trmidx{Nvidia} driver.
This may require un-installing and \trmidx{purging} packages, e.g.,
\begin{verbatim}
aptitude remove --purge  nvidia-glx
\end{verbatim}
If the package is not purged, then \flidx{/etc/init.d/nvidia-glx},
which runs at each startup, may wipe out the \trmidx{TLS} links that
the new Nvidia driver needs. 

\subsection{Virtual Private Network: VPN}
Use a \trmidx{Virtual Private Network} (\trmidx{VPN}) to
obtain a UCI net address from an off-campus computer.
\begin{verbatim}
sudo /etc/init.d/vpnclient_init start # Load kernel module
sudo vpnclient connect UCI # Start VPN connection
sudo vpnclient connect UCIFull # Start VPN connection--full mode
\end{verbatim}
\trmidx{NACS} uses allows browsers to request a VPN by
logging into the UCI VPN Proxy at \mchidx{https://vpn.nacs.uci.edu}. 

\subsection{XFree86: Security}
\trmidx{X}~supports a number of security measures, not all of them
 helpful.
By default, the \trmidx{Debian} distribution prohibits X~connections
connections from any processes not started by the current desktop
 owner. 
Apparently this is enforced through so-called \trmidx{TCP-forwarding}.
TCP-forwarding may be prohibited with the \cmdidx{-nolisten tcp}
 argument to the X~server.
To permit other users access to your desktop display, remove this
 argument from appropriate file(s).
Debian starts the X~server from \flidx{xserverrc}
(\flidx{/etc/X11/xinit/xserverrc}).
The \flidx{gdm.conf} file (\flidx{/etc/gdm/gdm.conf}) may also need to
 be modified.
\begin{verbatim}
/usr/bin/X11/X -dpi 100 -nolisten tcp
\end{verbatim}
Another way of allowing access to your X~desktop is to authorize
access from all clients using the \cmdidx{xhost} command:
\begin{verbatim}
xhost + # Allow connections from any server
\end{verbatim}
This may be risky unless the desktop sits behind a firewall or is
offline.

\subsection{XFree86: Multiple sessions}
Default X~display is accessible as Ctl-Alt-F7 on \trmidx{RedHat} Linux 
For a different number of bitplanes on default display, use, e.g., 
\cmdidx{startx}
\begin{verbatim}
startx -- -depth 24
\end{verbatim}
To create a second display at, say, 24\,bits-per-pixel (\index{bpp}),
accessible as Ctl-Alt-F8, use 
\begin{verbatim}
startx -- -depth 24 :1
\end{verbatim}
To create a second display using a test \flidx{XF86Config} file, use
\begin{verbatim}
startx -- -xf86config linux/etc/X11/XF86Config-4.elnino :1
\end{verbatim}
For secutrity reasons, the path to the \flprn{XF86Config} file must be
a \trmidx{relative path}, not an \trmidx{absolute path}.

Table~\ref{tbl:key} summarizes the keystroke commands related to
X~Windows. 
\begin{table}
\begin{minipage}{\hsize} % Minipage necessary for footnotes KoD95 p. 110 (4.10.4)
\renewcommand{\footnoterule}{\rule{\hsize}{0.0cm}\vspace{-0.0cm}} % KoD95 p. 111
\begin{center}
\caption[Funky Keystrokes]{\textbf{Funky Keystrokes}% 
\footnote{Use 'em or lose 'em}%  
\label{tbl:key}}
\vspace{\cpthdrhlnskp}
% fxm:\raggedright does not work here, must be between two other columns?
\begin{tabular}{ >{\ttfamily}l<{} l }
\hline \rule{0.0ex}{\hlntblhdrskp}% 
\textrm{Key} & Description \\[0.0ex]
\hline \rule{0.0ex}{\hlntblntrskp}%
Ctl-Alt-F1--F5 & Virtual consoles \\[0.1ex]
Ctl-Alt-F7 & X~display :0.0 \\[0.1ex]
Ctl-Alt-F8 & X~display :0.1 \\[0.1ex]
Ctl-Alt-+\footnote{This \textit{must} be the plus key (\kbdidx{+}) on
  the numeric keypad. On most laptops this requires a \textit{four key combination, since the numeric keypad requires a key to activate.}} & Change X~Resolution \\[0.1ex]
Ctl-Alt--\footnote{This \textit{must} be the minus key (\kbdidx{-}) on the numeric keypad} & Change X~Resolution \\[0.1ex]
% & \\[0.5ex]
\hline
\end{tabular}
\end{center}
\end{minipage}
\end{table} % end tbl:key

Accessing virtual consoles and starting multiple
sessions is the same in XFree86~4.x as XFree86~3.x.
However, the specification of the color depth has changed from
\cmdidx{-bpp} to \cmdidx{-depth}. 
To start~X with 24~bit color the command is
\begin{verbatim}
startx -- -depth 24
\end{verbatim}
\begin{sloppy}
Display managers such as \cmdidx{xdm}, \cmdidx{kdm}, and \cmdidx{gdm}
start~X themselves, and require that non-default options be passed
directly to the X~server in the configuration file, e.g.,
\flidx{/etc/X11/gdm/gdm.conf}, or \flidx{/etc/kde/kdm/kdmrc}.
\end{sloppy}
Before configuring display managers individually, however, it is wise
to consider setting many parameters in the system-wide X~configuration 
file, \flidx{XF86Config-4}.
For example, adding \cmdidx{DefaultColorDepth 24} to
\verb'Section "Screen"' should cause the X~server to always run with
24 bit-planes.

\subsection{LCD Projectors}\label{sxn:lcd}
\trmidx{LCD Projectors} interact with X in strange ways. 
The ESS Department projector works best at resolution fxm.
The IGPP projector works best at resolution fxm.

\subsection{Desktop}\label{sxn:dsk}
The \trmidx{X~server} automatically starts at the specified system
\trmidx{runlevel}.
The default runlevel is specified in \flidx{/etc/inittab}, and is
OS-dependent. 
In a given runlevel (\trmidx{RedHat} uses runlevel~5 for this,
\trmidx{Debian} uses runlevel~2), X~automatically starts the
\trmidx{display manager} specified in
\flidx{/etc/X11/default-display-manager}. 
The most common choices are \trmidx{GNOME}, \trmidx{KDE}, and
\trmidx{XDM}.
These choices invoke the display managers \cmdidx{gdm}, \cmdidx{kdm},
and \cmdidx{xdm}, respectively. 
These display managers are capable of starting any desktop on the
system (as determined in their configuration settings).
Naturally they default to their appropriate desktops, e.g.,
\cmdprn{kdm} starts the \trmidx{K~desktop environment}.
This default may be over-ridden by
\flidx{/etc/X11/default-display-manager}.
\begin{verbatim}
$ cat /etc/X11/default-display-manager 
/usr/bin/kdm
\end{verbatim}
Installing and removing desktops:
\begin{verbatim}
sudo aptitude purge kubuntu-desktop # Delete .debs from /var/cache/apt
sudo aptitude install kubuntu-desktop # Install .debs
sudo aptitude remove kubuntu-desktop # Uninstall .debs

sudo aptitude purge gnome-desktop # Delete .debs from /var/cache/apt
sudo aptitude install gnome-desktop # Install .debs
sudo aptitude remove gnome-desktop # Uninstall .debs
\end{verbatim}
On 20100430 I tried to upgrade Kubuntu from Karmic to Lucid.
The upgrade failed spectacularly on \mchidx{givre.ess.uci.edu}.
In particular, the \trmidx{X} window system would work fine at the
login manager (both \trmidx{kdm} and \trmidx{gdm}) stage, but
neither would boot into the full desktop.
Forced to use Vista/Cygwin from 20100430-20100524!
Suggestions to fix included: 
\begin{verbatim}
# Problem was with libGL.so.1 which isn't supposed to exist in lucid
# Bypass kdm:
startx /usr/bin/startkde # Start KDE desktop (e.g., plasma)
xinit startkde # Put in strange old widgets (uses .xinitrc too?)
startx /usr/bin/gnome-session # Start GNOME desktop (e.g., nautilus)

# Generate .xession-errors file for debugging
20100521:
zender@givre:/data/zender/ppr$ more ~/.xsession-errors
/etc/gdm/Xsession: Beginning session setup...
stty: entrée standard: Ioctl() inappropré pour un périphérique
stty: entrée standard: Ioctl() inappropré pour un périphérique
n'est pas un « tty »
[: 159: n'est: unexpected operator
-f /etc/xprofile
.: 36: Can't open /etc/xprofile

# Rename KDE environment
mv ~/.kde/share/config ~/foo
sudo aptitude install plasma-desktop

kdm.log  
Xorg.0.log
Xorg.failsafe.log

# NVidia driver de-installs old libraries and installs newer ones
wget http://us.download.nvidia.com/XFree86/Linux-x86_64/195.36.24/NVIDIA-Linux-x86_64-195.36.24-pkg2.run
chmod ugo+x NVIDIA-Linux-x86_64-195.36.24-pkg2.run
sudo ./NVIDIA-Linux-x86_64-195.36.24-pkg2.run
\end{verbatim}

\subsection{JPEG}\label{sxn:JPEG}
Processing \trmidx{JPEG} files produced by \trmidx{IDL}.
\begin{verbatim}
for fl_stb in `ls *.jpg | perl -p -e 's/$*\.jpg/$1/g;'` ; do
        echo "Converting ${fl_stb}.jpg to ${fl_stb}.pnm..."
        jpegtopnm ${fl_stb}.jpg > ${fl_stb}.pnm
        echo "Converting ${fl_stb}.pnm to ${fl_stb}.png..."
        pnmtopng ${fl_stb}.pnm > ${fl_stb}.png
        echo "Converting ${fl_stb}.pnm to ${fl_stb}.png..."
        pnmtoppm ${fl_stb}.pnm > ${fl_stb}.ppm
done
\end{verbatim}
% end section on X and XFree86

\section{Disk Management}\label{sxn:dsk}

\subsection{RAID}\label{sxn:raid}
\trmdfn{RAID} is Redundant Array of Independent Disks.
Reference: \url{http://en.tldp.org/HOWTO/Software-RAID-HOWTO.html}

\subsubsection{Hardware}\label{sxn:raid_hrd}
(Will Kitto helped with this setup)

The RAID on biogenic consists of $4 \times 150$~GB \trmidx{Maxtor}
disks. 
There are two Promise disk controller cards installed in
\mchidx{biogenic}, with two disks connected to each card.
The jumpers on each disk were set appropriately.  
All disks are "master" not "slave", so the disks have the following
IDs: 

\begin{verbatim}
Card which drive
---- ----- -----
1    m1    hde
    s1    hdf (does not exist)
    m2    hdg
    s2    hdh (does not exist)
2    m1    hdi
    s1    hdj (does not exist)
    m2    hdk
    s2    hdl (does not exist)
\end{verbatim}

\subsubsection{Getting Promise cards to work}\label{sxn:raid_prm}
The \trmidx{Promise cards} were not supported by the kernel (that
comes with RH7.2).
A patch was found to support the Promise chipset, and a new kernel was 
made, with all the RAID modules included.  
I do not know if RH8.0 has support for the Promise cards, but I would
guess yes. 

\subsubsection{Software}\label{sxn:raid_sw}
The devices were setup to run software \trmidx{RAID-5} (i.e. not RAID done in 
hardware).
This means that of the 4~disks, only the capacity of~3 (i.e., 450~GB)
is available for users (the other 150~GB are for redundancy).
I followed the instructions that you'd see in any typical RAID howto
(e.g. as per the reference).
The \flidx{raidtab} was setup 
\begin{verbatim}
% more /etc/raidtab
raiddev /dev/md0
   raid-level            5
   nr-raid-disks         4
   nr-spare-disks        0
   persistent-superblock 1
   parity-algorithm      left-symmetric
   chunk-size            32
   device                /dev/hde1
   raid-disk             0
   device                /dev/hdg1
   raid-disk             1
   device                /dev/hdi1
   raid-disk             2
   device                /dev/hdk1
   raid-disk             3
\end{verbatim}

\begin{verbatim}
% mkraid /dev/md0
% mke2fs -v -j -b 4096 -R stride=8 /dev/md0
\end{verbatim}

\begin{verbatim}
/dev/md0 /raid ext3 defaults 1 2
\end{verbatim}

To check the status of the RAID, view \flidx{/proc/mdstat}:
\begin{verbatim}
% more /proc/mdstat
Personalities : [linear] [raid0] [raid1] [raid5]
read_ahead 1024 sectors
md0 : active raid5 hde1[0](F) hdk1[3] hdi1[2] hdg1[1]
     480238656 blocks level 5, 32k chunk, algorithm 2 [4/3] [_UUU]
unused devices: <none>
\end{verbatim}

I have had a disk failure a couple of times (a RAID failure fixable
with \cmdidx{raidhotadd}, not an actual problem with the disk).
Issuing the \cmdidx{raidhotadd} command fixed it (first view
\flprn{/proc/mdstat} to see which disk failed):
\begin{verbatim}
% raidhotadd /dev/md0 /dev/hde1
\end{verbatim}

Bootup and shutdown: It does not appear that any of the RAID commands 
(\cmdidx{raidstart} or \cmdidx{raidstop}) appear in any of the startup
or shutdown scripts, but it all seems to come up and down properly.

RAID commands:
\cmdidx{raid0run}, \cmdidx{raidhotgenerateerror}, \cmdidx{raidstart},
\cmdidx{raidhotadd}, \cmdidx{raidhotremove}, \cmdidx{raidstop} 

\subsection{Servers: Sand}\label{sxn:sand.ess.uci.edu}
Modifications 3.29.05 by hjm
\begin{verbatim}
# 200 gb disk from lanina moved to sand on IDE bus 1
# new hoary ubuntu distro loaded on 200gb disk as::
    Filesystem           1K-blocks      Used Available Use% Mounted on
    /dev/hda1             10482044   2048572   8433472  20% /
    tmpfs                  1786416         0   1786416   0% /dev/shm
    /dev/hda6            105294788   5897208  99397580   6% /data
    /dev/hda3             73278252   2128912  71149340   3% /home
    /dev                  10482044   2048572   8433472  20% /.dev
    none                      5120      2836      2284  56% /dev

# hjm - 6.7.05 - 200GB IDE disk moved from lanina failed SMART test, replaced with 
# new spare on same IDE bus; Kubuntu (ubuntu + KDE) installed.
# there is currently no floppy on sand
#/dev/fd0        /media/floppy0  auto    rw,user,noauto  0       0
\end{verbatim}

\subsection{Formatting Disks}\label{sxn:fmt}
Ocassionally disks are added or need to be replaced.
There is a lot one can know about \trmidx{formatting disks}.
Fortunately, a few commands suffice for most situations.
The venerable \cmdidx{fdisk} family of commands is important.
The recommended formatting command is \cmdidx{cfdisk}, a
\trmidx{curses}-based disk partition table manipulator.

The command to build a Linux filesystem is \cmdidx{mkfs}.
\begin{verbatim}
cfdisk /dev/hdc # Build partition table on /dev/hdc
mkfs -t ext3 /dev/hdc1 # Build an ext3 filesystem on /dev/hdc
\end{verbatim}
Once the disk is partitioned (with \cmdidx{cfdisk}) and formatted 
with (\cmdidx{mkfs}), add an entry to the \flidx{/etc/fstab}.
The entry should be for, e.g., \texttt{/dev/hdc1} (the partition) 
not \texttt{/dev/hdc} (the disk).
Make sure the mount point exists as a directory, e.g., \flidx{/data},
before attempting to mount it.

In May 2007, I formatted a 100\,GB replacement disk for
\mchprn{elnino} as \texttt{ext3} using the partition name
\flidx{hdc1}. 
This seemed to work fine for a few reboots until one day in June
the \flidx{hdc1} device could not be found.
I found the other disk devices were using \flidx{sda1}-style
terminology presumably because they are using SATA drivers.
After changing the \flidx{/etc/fstab} entry from \flidx{hdc1} to
\flidx{sdb1} the new drive mounted again.
There was no data loss.
 

\subsection{Creating LVM}\label{sxn:lvm_mk}
\trmdfn{LVM} is Logical Volume Management.
Reference: \url{http://tldp.org/HOWTO/LVM-HOWTO/}

\begin{verbatim}
fdisk /dev/hde1
fdisk /dev/hdi1
vgscan -v
pvcreate -v /dev/hde1
pvcreate -v /dev/hdi1
vgcreate -v -s 16 dust_vg /dev/hde1 /dev/hdi1
lvcreate -L 305000 -n dust_lv dust_vg # Creats /dev/dust_vg/dust_lv
mke2fs -j /dev/dust_vg/dust_lv
vgscan # Creates /etc/lvmtab and /etc/lvmtab.d/* and /etc/lvmconf/*
\end{verbatim}

vg commands:
\begin{verbatim}
vg_annotate vgcfgbackup vgck vgexport vgmerge vgremove vgsplit
vg_cachegen vgcfgrestore vgcreate vgextend vgmknodes vgrename
vga_reset vgchange vgdisplay vgimport vgreduce vgscan
\end{verbatim}

Entry in \flidx{/etc/fstab}:
\begin{verbatim}
/dev/dust_vg/dust_lv /data ext3 exec,dev,suid,rw 1 2
\end{verbatim}

Bootup and shutown: It does not appear that any of the LVM commands 
appear in any of the startup or shutdown scripts.
However, \mchprn{dust} seems to have a perpetual problem of shutting
down, because it hangs on the \cmdidx{umount} command (recall that all
the computers mount \flidx{/home} from \mchprn{dust}, so it is tricky
for this disk not to be busy).
Typically upon shutdown, I am forced to just power-off at the point
that \mchprn{dust} gets stuck.
Upon boot-up, everything usually comes up fine after the disk is
checked. 

\subsection{LVM on ashes}\label{sxn:lvm_ashes}
We installed LVM managing some partitions on
\mchidx{ashes.ess.uci.edu}. 
\begin{verbatim}
lvscan
lvremove /dev/data/lvol1
lvcreate 
lvcreate -L 5g /dev/data # Create space for /usr/local
lvcreate -L 5g /dev/data # Create space for /home
lvcreate -L 42g /dev/data # Failed due to not enough space
lvcreate -L 39g /dev/data # Create space for /data
mkdir /mnt/home
/sbin/telinit 1
mkdir /mnt/data
mkdir /mnt/local
cd /mnt
mkreiserfs /dev/data/lvol1
mkreiserfs /dev/data/lvol2
mkreiserfs /dev/data/lvol3
ls /usr/local
mount /dev/data/lvol1 /mnt/home
mv /home/* /mnt/home
ls /mnt/home
ls /home
mount /dev/data/lvol2 /mnt/local
mv /usr/local/* /mnt/local
ls /mnt/local
ls /usr/local
umount home local
vi /etc/fstab
mkdir /data
mount /data
more /etc/fstab
tune2fs -j /dev/hda1 # Turn ext2 partition into ext3
tune2fs -j /dev/hda6 # Turn ext2 partition into ext3
tune2fs -j /dev/hda7 # Turn ext2 partition into ext3
more /etc/fstab
tune2fs -j /dev/hda5 # Turn ext2 partition into ext3
emacs -nw /etc/fstab # Turn /home, /usr/local, /boot, /data into reiserfs
shutdown -r now
\end{verbatim}

\subsection{Knoppix}\label{sxn:knp}
\trmidx{Knoppix} is a Linux distribution useful for previewing
hardware and for rescuing downed computers.
\trmprn{Knoppix} offers many boot time options.
Specify these at the \texttt{boot:} prompt.
\begin{verbatim}
knoppix lang=us # Change language from default (German) to English
knoppix 2 # Boot to run-level 2
knoppix screen=1600x1200 # Specify screen resolution
\end{verbatim}
\trmprn{Knoppix} is very useful for restoring corrupted systems.
Typically this requires mounting the disks on the filesystem.
\trmprn{Knoppix} does this automatically by double-clicking on the
disk icon (alternatively, on may use the \cmdidx{mount} command).
Then the disk must be set to enable-writing.
\trmprn{Knoppix} makes this available through the mouse menu.
Finally it is usually necessary to change from user \texttt{knoppix}
to user \cmdidx{root} before doing any serious repairs.
The command sequence to fix a bad sector on a \trmidx{reiserfs}
partition on \mchidx{ashes} was:
\begin{verbatim}
su root
reiserfsck /dev/hda1
\end{verbatim}
Assuming corruption is found and the disk,
may want to \trmidx{unmount} the disk and attempt to rebuild the
filesystem tree
\begin{verbatim}
umount /dev/hda1
reiserfsck --rebuild-tree /dev/hda1
\end{verbatim}
Many \trmidx{ATA}-3 and later ATA, \trmidx{IDE} and \trmidx{SCSI}-3
hard drives include Self-Monitoring, Analysis and Reporting Technology
(\trmidx{SMART}) utilities.
The \cmdidx{smartmontools} package uses these capabilities to monitor
and tests for disk problems. 
The command line program is \cmdidx{smartctl}, and the \trmidx{daemon}
is \cmdidx{/etc/init.d/smartmontools} which uses the configuration in
\begin{verbatim}
/etc/init.d/smartmontools start
/usr/sbin/smartctl --all /dev/hda1
smartctl -i /dev/hda # Print disk information
smartctl -H /dev/hda # Print health status
smartctl -l error /dev/hda # Print error log from test
smartctl -t offline /dev/hda # Perform offline test (then check error log)
smartctl -t short /dev/hda # Perform short disk test
smartctl -t long /dev/hda # Perform long disk test
smartctl -l selftest /dev/hda # Check selftest error log
\end{verbatim}

The \cmdidx{lm-sensors} package complements the SMART utilities.
\begin{verbatim}
sudo aptitude install lm-sensors # Install lm-sensors
# Create devices on the local system
cat > ~/mkdev.sh << EOF
#!/bin/bash
# number of devices to create (max: 256)
NUMBER=32
# Device owner and group
OUSER=root
OGROUP=root
# Set device mode
MODE=600
# Do not run script if devfs is used
if [ -r /proc/mounts ] ; then
if grep -q "/dev devfs" /proc/mounts ; then
echo "You do not need to run this script as your system uses devfs."
exit;
fi
fi

i=0;
while [ $i -lt $NUMBER ] ; do
echo /dev/i2c-$i
mknod -m $MODE /dev/i2c-$i c 89 $i || exit
chown "$OUSER:$OGROUP" /dev/i2c-$i || exit
i=$[$i + 1]
done
EOF

chmod 755 ~/mkdev.sh # Make mkdev.sh executable
sudo ~/mkdev.sh

\end{verbatim}

\subsection{Mirroring Disks}\label{sxn:mrr}
Chad Cantwell used hardware RAID~1 to join two Maxtor 150\,\GB\ disks
on \mchprn{dust} as \flprn{/home} on \mchidx{dirt.ess.uci.edu}. 
\trmidx{Mirroring disks} to other disks is a smart strategy for
creating \trmidx{high-availability storage}.
The RAID status is in \flidx{/proc/mdstat}.
\flidx{mdstat} contains a percentage done while setting up,
and then the final RAID status afterwards.
This is configured via \flidx{/etc/raidtab}.

\subsection{Garbage collection}\label{sxn:grb}
Through time most disks accumulate unwanted files such as
\flidx{core} dumps, intermediate files, and obsolete files.
Unless care is taken, a substantial portion of free disk space may be 
used by unwanted files.
The following short script uses \cmdidx{find} to locate and sort
in order of descending size the largest 100~files within the
\verb'${drc}' hierarchy.
\begin{verbatim}
for drc in ${DATA} ${HOME}; do
  find ${drc} -type f -exec ls -l {} \; | sort -r -n -k 5 | head -100 > ${drc}/bgf_foo.txt
done
\end{verbatim}
The results are stored in \flprn{/tmp/bgf\_foo.txt}.
Find files that contain \verb'foo' in their names:
\begin{verbatim}
find . -name '*html'
/tmp/bgf_foo.txt
\end{verbatim}

\subsection{Disk corruption}\label{sxn:fsck}
Disks die occasionally and it is prudent to gain familiarity with
disk recovery strategies (although these are no substitute for 
making regular off-site backups!).
Disk blocks that are unwritable are known as \trmdfn{bad blocks}.
When disks develop bad blocks they may still function for quite a
while, but restarting the system may become tedious as manual
\cmdidx{fsck}'s become mandatory. 
Specific disk repair commands are filesystem-dependent.
The default Linux filesystem is \trmdfn{ext2}, which is supported by a
package of utilities known as \cmdidx{e2fsprogs}.
The \cmdidx{dumpe2fs} command displays useful information about a
device, such as the block size, chunk size, etc.
The \cmdidx{e2fsck} command is used to monitor and repair ext2
filesystems. 
\begin{verbatim}
dumpe2fs -b /dev/hda7 # Print bad block list
umount /dev/hda7
e2fsck -c /dev/hda7 # Write bad blocks to list
e2fsck -p /dev/hda7 # Automatically preen filesystem without prompting
\end{verbatim}
Ext2 filesystems support the notion of a bad block list, i.e., a list
of blocks it will not attempt to write to.
The recommended procedure to create such a list is to first identify
the device, e.g., /dev/hda7.
Then logout of any windows accessing that device and unmount it.
This may be tricky if the partition is used by the \cmdidx{umount}
itself (or a dynamic library upon which the executable depends,
\verb'libc', for example).
If this occurs, one can presumably boot from a rescue floppy and
execute \verb'umount' from there.
Then run \verb'e2fsck -c' on the device.

A highly recommended rescue floppy known as ``Tom's root boot'' is
available from \url{http://www.toms.net/rb/home.html}.

hjm++ 1.23.06
updatedb (on sand).
'locate' is a very useful tool on many *ix systems, however it can 
overwhelm you with output, especially if it is used on a system that has 
filesystems and backup directories.  It is useful to grep the output to 
for the lead path or remove backup directories from being included in 
the locate db, by entering them in the /etc/updatedb.conf.  This has been 
done on sand to avoid cluttering the output with repeated entries from
the backup mirrors.
hjm-- 

\subsection{Renaming Disks}\label{sxn:rnm}
Renaming disk drives, partitions, and USB sticks is hard under Linux. 
Instructions are given at
\href{https://help.ubuntu.com/community/RenameUSBDrive}.
This technique relies on the \trmidx{GPartEd} partition manager,
\cmdidx{gparted}, which in turn relies on \cmdidx{dosfstools},
\cmdidx{mtools}, and \cmdidx{ntfstools}.

% end section on Disk Management

\section{Software}\label{sxn:sw}
\subsection{netCDF Browsers}\label{sxn:ncbrws}
There are at last four freely available \trmidx{netCDF} file
browsers: \cmdidx{ncview}, \cmdidx{ncBrowse}, \cmdidx{NCVweb}, and
\cmdidx{panoply}.  

\subsubsection{ncBrowse}\label{sxn:ncBrowse}
\href{http://plover.wrc.noaa.gov/java/ncBrowse}{ncBrowse}
\cmdprn{ncBrowse} is a \trmidx{Java} application.

\subsubsection{panoply}\label{sxn:panoply}
\href{http://www.giss.nasa.gov/tools/panoply}{panoply}
It functionality is somewhat limited in that logarithmic axes are not
allowed. 

\subsubsection{ncview}\label{sxn:ncview}
\href{http://meteora.ucsd.edu}{ncview}
Dave Pierson's excellent \flidx{ncview} software must be installed by
hand. 
Since it relies on UDUNITS, it is important to build that as well.
\begin{verbatim}
sudo scp /usr/local/bin/ncview dust.ess.uci.edu:/usr/local/bin
sudo scp -r /usr/local/lib/ncview dust.ess.uci.edu:/usr/local/lib
\end{verbatim}
\cmdprn{ncview} allows logarithmic transformation of coordinates.

\subsubsection{NCVweb}\label{sxn:NCVweb}
The \href{http://arm.mrcsb.com}{NCVweb} package is designed
specifially for viewing netCDF files produced by the 
\trmidx{Atmospheric Radiation Measurement} (\trmidx{ARM}) Program. 
\trmidx{NCVweb} is for web-based visualization.
It uses \trmidx{NCO} for aggregating data.
\begin{verbatim}
fxm
\end{verbatim}

\subsection{Python}\label{sxn:python}
The \trmidx{netCDF} interface to \trmidx{Python} is 
Python has at least 2 actively maintained netCDF interfaces:
\url{http://met-www.cit.cornell.edu/noon/ncmodule.html} by Bill Noon, and
\url{http://starship.python.net/crew/hinsen/netcdf.html} by Konrad Hinsen.

\subsection{Mail}\label{sxn:mail}
My Linux servers run \cmdidx{sendmail}, or the \cmdidx{postfix}
interface to \cmdprn{sendmail}.
This program, like other \trmidx{daemons}, can be restarted using
\verb'killall -HUP sendmail'.
To start the program, use the \trmidx{RedHat} \cmdidx{control-panel}.
The outgoing \trmidx{SMTP} server is set in the file
\flprn{/etc/sendmail.cf} or \flprn{/etc/mail/sendmail.cf} in the line
beginning with \verb'DS'.
Mail that has not yet been delivered is stored in the
\flprn{/var/spool/mqueue} directory.
Incoming mailbox (e.g., \flprn{/var/spool/mail/zender}) must be known 
to \cmdidx{Emacs}. 

There is a good example of how to set up \cmdidx{postfix} for
\trmidx{Ubuntu} machines at
\url{http://www.ubuntulinux.org/wiki/DialupEmailHowto}.
This example works for \mchprn{ashes.ess.uci.edu}.

All too often, mail does not get delivered.
Thus it is very important to keep a file copy of all messages before
entrusting them to the \trmidx{Mail Transfer Agent}, (\trmidx{MTA}).
At the very least, CC yourself on all mail you send.
If it does not reach you, it did not reach its intended recipient.
Undelivered mail is stored in \flidx{/var/spool/mqueue-client}.
Undelivered messages have non-zero sizes.

Occasionally it is necessary to move mail between machines.
Mail folders which are in standard format can be simply joined
together using the \cmdidx{cat} command.
As of September 2002, the correct UCI POP server to use for incoming
mail is \flidx{pop.uci.edu}.
The \trmidx{SMTP server} to use for outgoing mail depends on the
Internet address of the client.
In general, setting SMTP server to 
\mchidx{localhost.localdomain} works fine.
This requires correctly configuring a \trmidx{Mail Transport Agent}
(\trmidx{MTA}), such as \trmidx{Postfix}, first. 
When configuring the SMTP client (e.g., \cmdidx{thunderbird}) delivery
protocol, a good option to select is (\trmidx{TLS}), in particular,
the ''TLS, if available'' option.

Clients directly connected to UCI may use \flidx{smtp.uci.edu}.  
See \url{http://www.nacs.uci.edu/computing/e4e.html} for additional
UCI details.
Clients at remote locations have two basic options for for sending
mail from the remote location, port forwarding or changing mail
servers.
Port forwarding works from any remote location.
Being portable, it is recommended for all short-term trips where
there is no advantage to being on a local mail server.
The goal of port-forwarding is to redirect traffic on the local port 
to the hostport on the remote host.
This forwarding is done via encrypted tunnel so the remote\_client must
have trusted access to the remote server.
This command forwards 2025 traffic on \mchidx{localhost} 
to port~25 on the SMTP server.
The tunneling is done on a trusted remoted client.
\begin{verbatim}
ssh -L localport:remote_server:server_port remote_client
ssh -L 2025:smtp.uci.edu:25 sand.ess.uci.edu
# Access PBS cluster firewalled httpd 
# First ssh to pbs, then open local browswer to http://localhost:2026
ssh -L 2026:pbs.ess.uci.edu:80 pbs.ess.uci.edu
\end{verbatim}
The tunnel must remain open for this port-forwarding to work
(do not close the window).

Another example is to route around firewalls so that, for example,
the clientpc (a laptop) behind a firewall can access software
repositories (such as the Ubuntu archives) that would otherwise be
inaccessible.
\begin{verbatim}
# clientpc is virga
# hostpc is sand
# yourport is ssh port = 22
ssh -p 22 -X zender@dust.ess.uci.edu
ssh -p 22 -D 9999 -X zender@dust.ess.uci.edu
# First ssh to sand, then open local browswer to http://localhost:2026
ssh -L 2026:dust.ess.uci.edu:80 dust.ess.uci.edu

ssh -L localport:remote_server:server_port remote_client
ssh -L 2025:smtp.uci.edu:25 dust.ess.uci.edu
# Access PBS cluster firewalled httpd 
# First ssh to pbs, then open local browswer to http://localhost:2026
ssh -L 2026:pbs.ess.uci.edu:80 pbs.ess.uci.edu
\end{verbatim}

A second option is to login to an authorized SMTP server.
All non-UCI SMTP servers must use port~587.
Clients directly connected to \trmidx{wsu.edu} should use
\flidx{mail.wsu.edu} (WSU blocks mail routed through
\mchprn{localhost.localdomain} SMTP servers).
Clients directly connected to \trmidx{Cox.net} should use
\flidx{smtp.west.cox.net}.
Clients directly connected to \trmidx{cgd.ucar.edu}, or to
\trmidx{wireless.ucar.edu} should use \flidx{mailhub.cgd.ucar.edu}.
Clients directly connected to \trmidx{greenspeedisp.net} should use
\flidx{mail.greenspeedisp.net}.

Setting the SMTP server in \trmidx{Mozilla} is non-intuitive.
Select the \cmdprn{Outgoing Server (SMTP)} setting at the
\textit{bottom} of the accounts menu presented by 
\cmdprn{Edit | Mail and Newsgroups Account Settings}.

\trmidx{Emacs} \trmidx{rmail} mode provides an excellent environment
for editing and sending mail. 
\trmidx{rmail} understands your \flidx{.mailrc} file and, moreover, 
works with all features of Emacs.
Remember to change the Emacs \trmidx{SMTP} server when roaming.
\begin{verbatim}
(setq smtpmail-default-smtp-server "smtp.uci.edu") 
\end{verbatim}

Sometimes it is useful to combine or juggle rmail and
\trmidx{Mozilla}/\trmidx{Thunderbird} mail folders:
\begin{verbatim}
cd ${HOME}
tar cvzf ${DATA}/tmp/mail.tar.gz ./mail
scp ${DATA}/tmp/mail.tar.gz dust.ess.uci.edu:
tar xvzf mail.tar.gz
rm ${HOME}/mail/*

# Synchronize mail backup repository (dust) to main repository (givre) 
rsync /home/zender/.mozilla-thunderbird/w97e02d0.default/Mail/pop.uci.edu dust.ess.uci.edu:mail
rsync /home/zender/.mozilla-thunderbird/w97e02d0.default/Mail/Local\ Folders dust.ess.uci.edu:mail

# Copy mail backup repository (dust) to local backup (neige or givre) 
rsync dust.ess.uci.edu:mail ~

# Synchronize remote mail backup repository with local backup (neige or givre) 
cd;rsync mail snow.engin.umich.edu:

# Synchronize current client (ashes,elnino,sand) with main repository (givre) or backup repository (sand)
#pth_src='givre.ess.uci.edu:/home/zender/.mozilla-thunderbird/w97e02d0.default' # givre
pth_src='sand.ess.uci.edu:/home/zender/.mozilla-thunderbird/1h6xplh6.default' # sand
#pth_dst=${HOME}/.mozilla-thunderbird/lsymr6b9.default # ashes
pth_dst=${HOME}/.mozilla-thunderbird/ivj7otox.default # neige
#pth_dst=${HOME}/.mozilla-thunderbird/1h6xplh6.default # sand
#pth_dst=${HOME}/.mozilla-thunderbird/w97e02d0.default # givre
#pth_dst=${HOME}/.thunderbird/ikalbq96.default # roulee
rsync ${pth_src}/Mail/pop.uci.edu ${pth_dst}/Mail
rsync ${pth_src}'/Mail/Local\ Folders' ${pth_dst}/Mail
\end{verbatim}

My user preferences file, \flidx{prefs.js}, was corrupted on 20060222.
Upgrading \cmdidx{thunderbird} while it is running is usually safe.
However, this major upgrade (to version 1.5) appeared to corrupt it.
See \url{http://www.mozilla.org/unix/customizing.html#prefs}
Create a new profile with \cmdidx{mozilla-thunderbird -profilemanager}.
To see all your \cmdidx{firefox} settings, view the URL
\url{about:config} in a Firefox browser.
\begin{verbatim}
act_src='prznnk5x.zender' # elnino
act_dst='w97e02d0.default' # givre
pth_src=${HOME}/.mozilla-thunderbird/${act_src}
pth_dst=${HOME}/.mozilla-thunderbird/${act_dst}
/bin/cp -p -f ${pth_src}/*.mab ${pth_dst}
/bin/cp -p -f ${pth_src}/cert8.db ${pth_dst}
/bin/cp -p -f ${pth_src}/key3.db ${pth_dst}
/bin/cp -p -f ${pth_src}/63760765.s ${pth_dst}
/bin/cp -p -f ${pth_src}/mimeTypes.rdf ${pth_dst}
/bin/cp -p -f ${pth_src}/training.dat ${pth_dst}
/bin/cp -p -f ${pth_src}/virtualFolders.dat ${pth_dst}
/bin/cp -p -f ${pth_src}/Mail/pop.uci.edu/* ${pth_dst}/Mail/pop.uci.edu
/bin/cp -p -f ${pth_src}/Mail/Local\ Folders/* ${pth_dst}/Mail/Local\ Folders
\end{verbatim}

Useful \trmprn{Thunderbird} add-ons:
\trmidx{Enigmail} (encryption/decryption), 
\trmidx{LookOut} (reads \texttt{winmail.dat} attachments),
Useful \trmprn{Firefox} add-ons:
\trmidx{Adblock Plus} (ABP), 
\trmidx{Zotero} (bibliography/citation manager).

\subsection{Cygwin}
Installing \trmidx{Cygwin} requires a few additional commands.
Be sure to install \cmdidx{sudo}.
\begin{verbatim}
# Get LaTeX working
rsync dust.ess.uci.edu:sw/tex/cls ~/sw/tex
scp -r 'dust.ess.uci.edu:${TEXMFDIR}/tex/latex/datetime' .;sudo mv datetime ${TEXMFDIR}/tex/latex # datetime.sty
scp -r 'dust.ess.uci.edu:${TEXMFDIR}/tex/latex/fmtcount' .;sudo mv fmtcount ${TEXMFDIR}/tex/latex # fmtcount.sty
scp -r 'dust.ess.uci.edu:${TEXMFDIR}/tex/latex/ifdraft' .;sudo mv ifdraft ${TEXMFDIR}/tex/latex # ifdraft.sty
scp -r 'dust.ess.uci.edu:${TEXMFDIR}/tex/latex/revnum' .;sudo mv revnum ${TEXMFDIR}/tex/latex # revnum.sty
scp -r 'dust.ess.uci.edu:${TEXMFDIR}/tex/latex/tipa' .;sudo mv tipa ${TEXMFDIR}/tex/latex # tipa.sty
scp -r 'dust.ess.uci.edu:${TEXMFDIR}/tex/latex/tocbibind' .;sudo mv tocbibind ${TEXMFDIR}/tex/latex # tocbibind.sty
scp dust.ess.uci.edu:tex-site.el /usr/share/emacs/site-lisp/tex-site.el # Helps AucTeX
# Use '/cygdrive/c' prefix instead of 'C:'
ls /cygdrive/c/
\end{verbatim}

Cygwin supports \trmidx{SSH} and can run an \trmidx{ssh-server}, as
described at \url{http://www.petri.co.il/setup-ssh-server-vista.htm}:
\begin{enumerate*}
\item Right-click the icon on your Cygwin icon on your desktop and chose Run as Administrator.
\item To configure your SSH Server (Open SSH) type ssh-host-config. Cygwin will then ask ``Should privilege separation be used?'' Answer Yes.
\item Cygwin will then ask ``Should this script create a local user 'sshd' on this machine?'' Answer Yes.
\item Cygwin will then ask ``Do you want to install sshd as service?'' Answer Yes.
\item Cygwin will then ask ``Should this script create a new local account 'ssh\_server' which has the required privileges?'' Answer No.
\item Cygwin will then ask ``Which value should the environment variable CYGWIN have when sshd starts?'' Answer 'ntsec tty'. Congratulations the setup of your SSH Server is complete.
\item Start your SSH Server by typing 'net start sshd' and hit Enter
\item Note: it is a generally considered an unsafe practice to allow SSH access on protocol 1. You can block protocol access by editing \flprn{C:/cygwin/etc/defaults/etc/sshd\_config}
\end{enumerate*}

Start the Cygwin \trmidx{X~server} with \cmdidx{startxwin}.
Then, to copy from cygwin to Windows
\begin{enumerate*}
\item use the mouse to select (highlight) the text you want to copy
\item the highlighted text is now on the clipboard
\item if you want to paste that selection into word or notepad then use CTRL-V on the windows side
\end{enumerate*}

To copy from Windows to cygwin
\begin{enumerate*}
\item use the mouse to select (highlight) the text you want to copy
\item press CTRL-C to copy selection into clipboard
\item go into your xterm and press the middle mouse button to paste. If you don't have a
3 button mouse use the -emulate3buttons option in the startxwin.bat file. Then, to paste, all you'd have to do is press the two mouse buttons simultaneously.
\end{enumerate*}

To copy from cygwin to cygwin
\begin{enumerate*}
\item use the mouse to select (highlight) the text you want to copy
\item that puts the selection into the clipboard
\item place the cursor where you want to paste the text
\item press the middle mouse button (or the two mouse buttons simultaneously if using the -emulate3buttons option)
\end{enumerate*}

\subsection{Windows and NTFS}
Accessing \trmidx{Microsoft Windows} partitions is possible on
\trmidx{dual-boot} systems.
To mount a Windows Vista \trmidx{NTFS} filesystem,
\begin{verbatim}
aptitude install ntfs-3g ntfs-config # ntfs-3g necessary for NTFS write support
mkdir /wnd # Prepare mount-point
mount -t ntfs-3g /dev/sda3 /wnd # Mount NTFS partition at mountpoint
umount /wnd # Unmount NTFS partition
sudo ntfs-config & # GUI method to mount/unmount NTFS partitions
scp givre.ess.uci.edu:cygwin.neige.tar.gz /media/OS/cygwin/home/zender # Usage example
cp ~/crr/crr.txt /media/OS/cygwin/home/zender/crr # Linux->Windows
\end{verbatim}

\subsection{Macintosh OS~X}
\trmidx{OS~X} is based on \trmidx{FreeBSD}.
The OS~X equivalent to \cmdidx{ldd} is \cmdidx{otool}
\begin{verbatim}
otool -L `which ncks`
\end{verbatim}

\subsubsection{Fink}
\trmidx{Fink} is a \trmidx{Debian}-based system for installing
packages on OS~X.
Fink comes as a ``disk image'' file with a \cmdidx{.dmg} suffix.
Clicking with the pointer on a disk image file causes a sequence of
actions to occur: the file is mounted, the contents appear as icons,
etc.
The same effect may be achieved from the \trmidx{command line} with
the \cmdidx{hdiutil} and \cmdidx{installer} commands.
\begin{verbatim}
man hdiutil
man installer
# OS X VNC URL: http://netmath.math.uiuc.edu/VNC-osx.htm, http://www.realvnc.com
hdiutil attach OSXvnc-0.6.dmg # Mount disk image file
cd '/Volumes/OSXvnc 0.6' # Look at mounted image
cp -r OSXvnc.app /Applications # Copy package to software directory
cd '/Volumes/OSXvnc 0.6/OSXvnc.app/Contents/MacOS' # Directory with executables
/Applications/OSXvnc.app/Contents/MacOS/OSXvnc & # Run VNC
# Window pops up on Macintosh, prompts to start VNC server
# Recommended password is grape
hdiutil unmount '/Volumes/OSXvnc 0.6' # Unmount disk image file
hdiutil unmount /dev/disk2 # Unmount disk image file
# To view from Linux, use
vncviewer c-67-169-127-154.client.comcast.net:1 # IP:display

# Fink URL: http://fink.sf.net
hdiutil attach Fink-0.6.1-Installer.dmg # Mount disk image file
installer -pkginfo -pkg '/Volumes/Fink 0.6.1 Installer/Fink 0.6.1 \
Installer.pkg' -target /sw # Needs a .pkg package # Query package info
sudo installer -pkg '/Volumes/Fink 0.6.1 Installer/Fink 0.6.1 Installer.pkg' \
-target /sw # Install package, requires sudo password
hdiutil unmount '/Volumes/Fink 0.6.1 Installer' # Unmount disk image file
hdiutil unmount /dev/disk3 # Unmount disk image file
\end{verbatim}
% end section on Software

\section{Programming}\label{sxn:prg}
\subsection{Operator precedence}\label{sxn:pcd}
The \trmidx{operator precedence} of most languages follows that
of the \trmidx{C}~language. 
Table~\ref{tbl:pcd} summarizes operator precedence of~C.
Precedence decrease from top to bottom (the first line has highest precedence).
\begin{table}
\begin{minipage}{\hsize} % Minipage necessary for footnotes KoD95 p. 110 (4.10.4)
\renewcommand{\footnoterule}{\rule{\hsize}{0.0cm}\vspace{-0.0cm}} % KoD95 p. 111
\begin{center}
\caption[PCD Commands]{\textbf{C/\cxx Operator Precedence}% 
%\footnote{\emph{Source:} }%  
\footnote{Precedence decreases from top to bottom (first line has
highest precedence)}%  
\label{tbl:pcd}}
\vspace{\cpthdrhlnskp}
% fxm:\raggedright does not work here, must be between two other columns?
%\begin{tabular}{ >{\ttfamily}l<{} >{\raggedright}p{20.0em}<{} }
\begin{tabular}{ >{\ttfamily}l<{} l l }
\hline \rule{0.0ex}{\hlntblhdrskp}% 
\textrm{Operator} & Description & Associativity \\[0.0ex]
\hline \rule{0.0ex}{\hlntblntrskp}%
\verb'::a' & Unary scope resolution & Left to right \\[0.1ex]
\verb'a::b' & Binary scope resolution & Left to right \\[0.1ex]
\verb'a++' & Post-increment & Left to right \\[0.1ex]
\verb'a--' & Post-decrement & Left to right \\[0.1ex]
\verb'a()' & Function call & Left to right \\[0.1ex]
\verb'a[b]' & Array element & Left to right \\[0.1ex]
\verb'a->b' & Pointer to structure member & Left to right \\[0.1ex]
\verb'.' & Structure or Union member & Left to right \\[0.1ex]
%\footnote{In this line, \verb'+', \verb'-' and \verb'*' are the unary forms.}%
\verb'++a' & Pre-increment & Right to left \\[0.1ex]
\verb'--a' & Pre-decrement & Right to left \\[0.1ex]
\verb'!a' & Logical NOT & Right to left \\[0.1ex]
\verb'~a' & Bitwise NOT & Right to left \\[0.1ex]
\verb'-a' & Unary minus & Right to left \\[0.1ex]
\verb'+a' & Unary plus & Right to left \\[0.1ex]
\verb'&a' & Address & Right to left \\[0.1ex]
\verb'*a' & Indirection & Right to left \\[0.1ex]
\verb'sizeof(a)' & Size in bytes & Right to left \\[0.1ex]
\verb'new' & Allocate memory & Left to right \\[0.1ex]
\verb'delete' & De-allocate memory & Left to right \\[0.1ex]
\verb'(type)' & Typecast \& all \cxx\ cast operators & Left to right \\[0.1ex]
\verb'.*' & Pointer to member (objects) & Left to right \\[0.1ex]
\verb'->*' & Pointer to member (pointers) & Left to right \\[0.1ex]
\verb'*' & Multiply & Left to right \\[0.1ex]
\verb'/' & Divide & Left to right \\[0.1ex]
\verb'%' & Remainder & Left to right \\[0.1ex]
\verb'+' & Add & Left to right \\[0.1ex]
\verb'-' & Add & Left to right \\[0.1ex]
\verb'<<' & Left shift & Left to right \\[0.1ex]
\verb'>>' & Right shift & Left to right \\[0.1ex]
\verb'<' & Less than & Left to right \\[0.1ex]
\verb'<=' & Less than or equal to & Left to right \\[0.1ex]
\verb'>' & Greater than & Left to right \\[0.1ex]
\verb'>=' & Greater than or equal to & Left to right \\[0.1ex]
\verb'==' & Equal to & Left to right \\[0.1ex]
\verb'!=' & Not equal to & Left to right \\[0.1ex]
\verb'&' & Bitwise AND & Left to right \\[0.1ex]
\verb'^' & Bitwise XOR & Left to right \\[0.1ex]
\verb'|' & Bitwise OR & Left to right \\[0.1ex]
\verb'&&' & Logical AND & Left to right \\[0.1ex]
\verb'||' & Logical OR & Left to right \\[0.1ex]
\verb'?:' & Conditional & Right to left \\[0.1ex]
\verb'=' & Assignment & Right to left \\[0.1ex]
\verb'+= -= *= /= %= &= ^= |= <<= >>=' & Compound assignment & Right to left \\[0.1ex]
\verb',' & Comma & Left to right \\[0.1ex]
% & & \\[0.5ex]
\hline
\end{tabular}
\end{center}
\end{minipage}
\end{table} % end tbl:pcd

\subsection{Regular expressions}\label{sxn:rx}
The alphanumeric pattern specifying a group of strings is called a
\trmidx{regular expression}.
Special characters are \verb'$', \verb'^', \verb'.', \verb'*',
\verb'+', \verb'?', \verb'[', \verb']', and \verb'\'.
\begin{itemize}
\item ``\verb'.''' Matches any character except newline % .
\item ``\verb'*''' Match smallest possible preceding regular expression as many times as possible % *
\item ``\verb'+''' Match preceding regular expression at least once % +
\item ``\verb'?''' Match preceding regular expression once or not at all % ?
\item ``\verb'[ ... ]''' Character set % []
\item ``\verb'[^ ... ]''' Complemented character set % [^]
\item ``\verb'^''' Match beginning of line % ^
\item ``\verb'$''' Match end of line % $
\item ``\verb'\''' Quote special characters % \
\item ``\verb'\w''' Matches any word constituent character % \w
\item ``\verb'\W''' Matches any non-word constituent character % \W
\end{itemize}
In the replace expression, \verb'\&' stands for the match found for
the whole regular expression, and ``\verb'\N''' stands for the match
to the $N$th occurence of the ``\verb'\( ... \)''' pair.
A floating point number match is \cite[][p.~128]{Fri97}
\begin{verbatim}
-?([0-9]+(\.[0-9]*)?|\.[0-9]+)
\end{verbatim}
The ``\verb'?''' makes the negative sign optional.
The regular expressions used for lexically recognizing doubles
in \cmdidx{ncap} and in \cmdidx{ncgen}, respectively, are
\begin{verbatim}
[0-9]*\.[0-9]*([eE][+-]?[0-9]+)?[LlDd]?|[0-9]*([eE][+-]?[0-9]+)[LlDd]?
[+-]?[0-9]*\.[0-9]*{exp}?[LlDd]?|[+-]?[0-9]*{exp}[LlDd]?
\end{verbatim}

Using regular expressions in \cmdidx{Emacs} is both a pleasure and a
problem. 
Some example regular expression search and replaces that work, and
what they do
Special literal characters that must always be protected (preceded by
backslashes) in \cmdidx{Emacs} buffers include the asterisk
\kbdprn{*} and the square brackets \kbdprn{[} and \kbdprn{]}.
Literal characters that do not require protection include braces
and parentheses.
NB: typing directly into the \cmdidx{Emacs} command buffer requires
and extra protection (backslash) relative to reading in the command
from a macro file, e.g., \flidx{my\_macros.el}.
NB: Use either \cmdidx{query-replace-regexp} or
\cmdidx{tags-query-replace}.
There is no \cmdidx{tags-query-replace-regexp} command.
\begin{verbatim}
; Replace "{"cll_mth",no_argument,0,0}, /* [flg] Foo */" by "{"no_cll_mth",no_argument,0,0}, /* [flg] Foo */"
; (tags-query-replace "{"cll_mth",no_argument,0,0}, /\\* \\[flg\\] Foo \\*/" "{"no_cll_mth",no_argument,0,0}, /* [flg] Foo */" nil)
; Replace "strtol(optarg,(char **)NULL" by "strtol(optarg,&sng_cnv_rcd"
; (tags-query-replace "strtol(optarg,(char \\*\\*)NULL" "strtol(optarg,&sng_cnv_rcd" nil)
; Replace "!foo" by "! foo"
; (query-replace-regexp "!\(\w\)" "! \1" nil nil nil)
; Replace "character foo*80" by "character(80)::foo"
; (query-replace-regexp "character \(.*\)\*\([0-9]*\) " "character(\2)::\1 " nil)
; Replace "end do     !" by "end do !"
; (query-replace-regexp "end \(do\|if\) *!" "end \1 !" nil)
; Replace "foo   !" by "foo !" as long as "!" is not followed by "="
; (query-replace-regexp "\([^ !\t\n]\) + +! " "\1 ! " nil)
; Replace floating point number X.Y with floating point number X.Yf, e.g., "1.23" by "1.23f"
; (query-replace-regexp "\([+-]?[0-9]*\.[0-9]*[LlDd]?\)" "\1f" nil)
; Replace floating point number X.Y with floating point number $X.Y$, e.g., "-1.23" by "$-1.23$"
; (query-replace-regexp "\([+-]?[0-9]*\.[0-9]*[LlDd]?\)" "$\1$" nil)
; Replace "command{\foo}{bar}" by "command{\foo}{\ensuremath{bar}}"
; (query-replace-regexp "command{\\.*}{.*}" "command{\\1}{\ensuremath{\2}}" nil)
\end{verbatim}

One often composes text in an \cmdprn{Emacs} buffer, then copies and
pastes that buffer into another program, e.g., a browser or mail
program.
Browser and mail windows often have column-width limits imposed, so it
is helpful to compose with the same column-width in \cmdprn{Emacs}.
The command to set the maximum buffer width is
\cmdidx{set-fill-column}. 
The argument to this command is the number of columns.
Entering the argument to \cmdprn{Emacs} commands is non-trivial.
The example of setting the number of columns to~80 show this:
\kbdidx{ESC}~\kbdidx{80}~\kbdidx{ESC}~\kbdidx{x}~\kbdidx{set-fill-column}~\kbdidx{RET}.
This command might be written in \cmdprn{Emacs} short-hand as
\begin{verbatim}
M-x set-fill-column
\end{verbatim}

\cmdidx{Bash} supports a wide variety of pattern operators for 
shell \trmidx{filename expansion}, aka \trmidx{globbing}.
These globbing operations filter all files and directories present
through a filter including one or more \trmidx{wildcard characters}.
\begin{verbatim}
for fl in `ls dst25_8589_??.nc` ; do
  mv $fl ${fl/8589_/clm} # Change '8589_' to 'clm' in filenames
done
for fl in `ls *khus*.jpg` ; do
  mv $fl ${fl/khus/kuhs} # Change 'khus' to 'kuhs' in filenames
done
for fl in `ls *Johnson,_Jack*` ; do
  mv $fl ${fl/Johnson,_Jack/Jack_Johnson}
done
# On machine with temporary mail, e.g., ashes:
cd ~/mail
for fl in `/bin/ls` ; do mv ${fl} ${fl}.foo ; done
# On machine with archived mail, e.g., elnino:
cd ~/mail
scp 'ashes.ess.uci.edu:mail/*.foo' ~/mail
for fl in `/bin/ls *.foo` ; do 
cat ${fl/.foo/} ${fl} > ${fl/.foo/.new}
mv -f ${fl/.foo/.new} ${fl}
done
\end{verbatim}
I also have a custom \trmidx{Perl} script, \flidx{fl\_rnm.pl}, for
renaming files.

The most common shell pattern matching operators, such as \kbdidx{*}
and \kbdidx{?}, have minimal functionality as regular expressions.
The \trmidx{Bash} shell supports extended regular expressions in
filename globbing via the shell option \cmdidx{extglob}.
Activate this functionality using the \cmdidx{shopt} command:
\begin{verbatim}
shopt -s extglob
\end{verbatim}

\subsection{Bash}\label{sxn:bash}
Use \cmdidx{hash} to \trmidx{rehash} commands in \trmidx{Bash}, e.g.,   
\cmdprn{hash -r}.
This is equivalent to \cmdidx{rehash} or \cmdidx{reset} in
\cmdidx{csh}. 
Systemwide defaults are set in \flprn{/etc/ssh/ssh\_config}.

The Bash shell supports a wide variety of built-in commands, command
line editing, job control, and history features.
\begin{verbatim}
!-1:s/foo/bar/ # Repeat previous command, substitute bar for foo
!!:s/foo/bar/  # Repeat previous command, substitute bar for foo
^foo^bar^      # Repeat previous command, substitute bar for foo
!foo:s/foo/bar/ # Substitute bar for foo in most recent command containing foo
\end{verbatim}

\subsection{OpenOffice}\label{sxn:ooo}
Start OpenOffice in different language:
\begin{verbatim}
LANG=fr_FR.utf8 oowriter prp_sdci.odt &
LANG=en_GB.utf8 oowriter prp_sdci.odt &
\end{verbatim}
Can also set language in drop-down menu (Tools->Language).
To upgrade in one step, copy and paste this in terminal:
\begin{verbatim}
echo -e 'echo "#PPA openoffice-pkgs
deb http://ppa.launchpad.net/openoffice-pkgs/ppa/ubuntu $(lsb_release -sc) main" | sudo tee -a /etc/apt/sources.list.d/ppa.list
gpg --keyserver keyserver.ubuntu.com --recv 247d1cff
gpg --export --armor 247d1cff | sudo apt-key add -
sudo aptitude update
sudo aptitude -y safe-upgrade
sudo aptitude -y dist-upgrade
' > ~/oooupgrade.sh
sudo sh ~/oooupgrade.sh && rm ~/oooupgrade.sh
\end{verbatim}

\subsection{Internationalization I18N}\label{sxn:i18n}
Create a local directory structure to hold \verb'*.mo' (``machine
object'') files created from \verb'*.po' (``portable object'') by
\verb'gettext' machinery. 
One directory is needed for each language.
\begin{verbatim}
mkdir -p ${HOME}/share/locale/es/LC_MESSAGES
mkdir -p ${HOME}/share/locale/fr/LC_MESSAGES
\end{verbatim}
% $: rebalance syntax highlighting
System-wide translation database is under \verb'/usr/share/locale'.
The installation of \verb'gettext()' is its own documentation.
Examine \verb'/usr/share/gettext/intl' to see how its done.

Emacs \verb'po-mode.el' should be loaded whenever \verb'*.po' files
are loaded.
% end section on Programming

\clearpage
\section{Files}\label{sxn:fl}

\subsection{Date conventions}\label{sxn:cnv}
This section describes conventions for naming files from geophysical
models. 
We adopt the usual convention that \kbdidx{DD} is a two digit sequence
to indicate the day of month $\kbdidx{DD} \in [01, 02, \ldots, 31]$,
\kbdidx{MM} is a two digit sequence to indicate the month of year 
$\kbdidx{MM} \in [01, 02, \ldots, 12]$, and \kbdidx{YYYY} is the four
digit Common Era year.
It is often useful to select files based on their date convention.
In such cases it is useful to have regular expressions
(cf.~\S\ref{sxn:rx}) for each date convention:
Date components like \kbdprn{MM} and \kbdprn{YYYY} are not just
keyboard inputs, they are also variables, since useful information
may be derived from them. 
For example, the number of years $\NNN$ in a file containing
data from years \varidx{YYYY} and \varidx{ZZZZ} is 
$\NNN = \mbox{\varprn{ZZZZ}}-\mbox{\varprn{YYYY}}+1$.

Valid replacements for \kbdprn{DD} are \verb'[0123][0-9]'.
Valid replacements for \kbdprn{MM} are \verb'[01][0-9]'.
Valid replacements for \kbdprn{YYYY} are \verb'[0-9][0-9][0-9][0-9]'.

Climatological values are abrreviated \verb'clm', which is best
interpreted as ``all of the available data present when the command 
was run''.
Syntactically, \kbdprn{clm} works equivalently to \kbdprn{YYYY}.
However, \kbdprn{clm} is only three characters, while \kbdprn{YYYY} is 
four characters.        
This distinction helps reduce errors when commands use globbing
to do the right thing, e.g.,
\begin{verbatim}
ncra caseid_????01.nc caseid_clm01.nc # OK globbing
ncra caseid_[0-9][0-9][0-9][0-9]01.nc caseid_clm01.nc # Better
ncra caseid_[0-9]{4}01.nc caseid_clm01.nc # Preferred
\end{verbatim}

\begin{enumerate}
\item \verb'${caseid}_YYYY' Annual mean
\item \verb'${caseid}_YYYYMM' Monthly mean
\item \verb'${caseid}_YYYYMMDD' Daily mean
\item \verb'${caseid}_YYYYMMDDHH' Hourly mean
\item \verb'${caseid}_YYYYMMDDHHMM' Minute mean
\item \verb'${caseid}_YYYYMMDDHHMMSS' Second mean

\item \verb'${caseid}_clm' Climatological mean
\item \verb'${caseid}_clmMM' Climatological monthly mean

\item \verb'${caseid}_YYYYZZZZ_MM' Ensemble mean of month \varprn{MM}
from years \varprn{YYYY} through \varprn{ZZZZ} (one record)
\item \verb'${caseid}_YYYYZZZZ_0112' Ensemble mean of seasonal cycle sampled
from years \varprn{YYYY} through \varprn{ZZZZ} (twelve records)
\item \verb'${caseid}_YYYY_ZZZZ_MM' Timeseries of month \varprn{MM} from years \varprn{YYYY}
through \varprn{ZZZZ} ($\mbox{\varprn{ZZZZ}} - \mbox{\varprn{YYYY}} + 1$ records)
\item \verb'${caseid}_YYYY_ZZZZ_MMNN' Timeseries of month \varprn{MM}
  through month \varprn{NN} means from years \varprn{YYYY} through \varprn{ZZZZ} 
($\mbox{\varprn{ZZZZ}} - \mbox{\varprn{YYYY}} + 1$ records)
\item \verb'${caseid}_YYYY_ZZZZ_0305' Timeseries of Springtime means from
years \varprn{YYYY} through \varprn{ZZZZ} 
($\mbox{\varprn{ZZZZ}} - \mbox{\varprn{YYYY}} + 1$ records) 

\item \verb'${caseid}_clm_0112' Twelve month seasonal cycle
\item \verb'${caseid}_YYYY_0112' Annual mean seasonal cycle (12
records). 
Note this is an exception to the general rule. 
It is unambiguous, however, because the mean of all twelve months
is simple representable by the \verb'${caseid}_YYYY' convention
\item \verb'${caseid}_YYYY_MMNN' Mean of months \varprn{MM} through \varprn{NN} (1 record)
\item \verb'${caseid}_YYYY_0305' Springtime mean (1 record)
\item \verb'${caseid}_YYYY_ZZZZ' Annual mean timeseries (multiple records)
\item \verb'${caseid}_YYYYMM_ZZZZNN' Monthly mean timeseries from \varprn{YYYYMM} to
\varprn{ZZZZNN}, inclusive (multiple records)
\item \verb'${caseid}_YYYYMMDD_ZZZZNNEE' Daily mean timeseries from \varprn{YYYYMMDD}
to \varprn{ZZZZNNEE}, inclusive (multiple records)

\item \verb'${caseid}_YYYY_ZZZZ_t' Mean of annual mean timeseries from
\varprn{YYYY} to \varprn{ZZZZ}, inclusive (single record)
\item \verb'${caseid}_YYYYMM_ZZZZNN_t' Mean of monthly mean timeseries from
\varprn{YYYYMM} to \varprn{ZZZZNN}, inclusive (single record)
\item \verb'${caseid}_YYYYMMDD_ZZZZNNEE_t' Mean of daily mean timeseries from
\varprn{YYYYMMDD} to \varprn{ZZZZNNEE}, inclusive (single record)
\end{enumerate}
% end section on Files

\clearpage

\section{System Administration}\label{sxn:sys}
\subsection{Backups}\label{sxn:bck}
Simple but effective system \trmidx{backups} are made each night.
In February, 2009, Daniel Wang designed these backups to copy the
\flidx{/home} and \flidx{/data/www} directories of
\mchidx{pbs.ess.uci.edu} the \flidx{/data2/pbs\_backup} directory of  
\mchidx{pbs1.ess.uci.edu}. 
The backups are controlled by a \trmprn{Cron} script called
\flidx{/etc/cron.daily/backupToPbs1}.

From Daniel's 20090213 e-mail:
\begin{verbatim}
Regarding backups:  If you notice that /data (aka LABEL=pbs-data or 
/dev/sda1) is bad, you should disable the backup script in
"/etc/cron.daily/backupToPbs1" by moving it outside /etc/cron.daily or
inserting an "exit" command near the top of it.  Please use whatever
method you feel would be more obvious, sensible, and less
forgettable/mysterious.  You wouldn't want the system to syncronize the
backup with bad (perhaps zeroed-out) data.

Regarding the nfs mount in /etc/fstab:  This mounts a directory on pbs1 in
/root/backup-on-pbs1 .  The backup script applies rsync from /data/www and
/home over to this mount path.  This was the simplest way I could think
of, that didn't involve creating passwordless public-key ssh logins
(dangerous) or exposing passwords and allowing true root-privileges on
pbs1.
\end{verbatim}

The \trmidx{Cron} system performs specified actions at regular intervals.
The \cmdidx{cron} program checks for actions to perform once per minute.
Currently, all backups are handled by the \flidx{crontab} file
of the \trmidx{root user}.
Use the \cmdidx{crontab} program to alter any user's \flprn{crontab}.
\begin{verbatim}
crontab -e
sudo crontab -u root -e
\end{verbatim}
The \cmdprn{crontab} program stores each user's crontab in
\flidx{/var/spool/cron}. 
For the root user, this is \flidx{/var/spool/cron/root} and
\flidx{/var/spool/cron/crontabs/root} for \trmidx{RedHat} and
\trmidx{Debian} \trmidx{GNU/Linux}, respectively.
Currently this system backup \flidx{crontab} file is
\begin{verbatim}
15  2 * * *  /usr/local/bin/bck_home.sh zender
45  2 * * *  /usr/local/bin/bck_home.sh bian
15  3 * * *  /usr/local/bin/bck_home.sh alfgr
45  3 * * *  /usr/local/bin/bck_home.sh mflanner
15  4 * * *  /usr/local/bin/bck_home.sh kwon
45  4 * * *  /usr/local/bin/bck_home.sh strombrg
15  5 * * *  /usr/local/bin/bck_home.sh lopez
45  5 * * *  /usr/local/bin/bck_home.sh jtalaman
15  6 * * *  /usr/local/bin/bck_home.sh okin
45  6 * * *  /usr/local/bin/bck_home.sh pajarola
30  5 * * *  /usr/local/bin/bck_data.sh
35  5 * * *  /usr/local/bin/bck_etc.sh
40  5 * * *  /usr/local/bin/bck_var.sh
37  3 * * *  /usr/bin/rdate -s time.nist.gov
38  3 * * *  /sbin/hwclock --systohc
\end{verbatim}
The first column is the minute of each hour for an action.
The second column is the hour of each day for an action.
The third, fourth, and fifth columns are the day of month, 
month of year, and year for the corresponding actions. 
Asterisks denote all values for the corresponding field.
Always backup data to at least two physically separate locations!

My personal, user-specific \flprn{crontab} file is
\begin{verbatim}
30  2 * * *  /home/zender/sh/bck_zender.sh dly
\end{verbatim}
Note that the backup commands themselves are only valid when specified
with fully qualified paths. 
This security feature of Cron helps prevent malicious files from being
inadvertently executed. 

Backup binaries from /usr/local/bin (e.g., \cmdidx{ddd}, \cmdidx{netscape}).
The script \flidx{bck.pl} is devoted to this, and works for
\trmidx{ZIP disk}s and \trmidx{LS120 drive}s too.
``Dot'' files (e.g., \flprn{.cshrc}, \flprn{.mailrc},
\flprn{.netscape} directory) should be relatively safe as they are
frequently archived by CVS. 
The most important files to backup thus become files in the mail
directories. 

\subsection{Mounting devices}\label{sxn:mnt}
Make sure CDROM is in fstab, mounting it as \verb'/dev/cdrom' is OK,
mounting it as /dev/hdc is also common. 
\begin{verbatim}
ln -s /mnt/cdrom /cdrom
ln -s /mnt/floppy /flp
\end{verbatim}
Occasionally, such as when rescuing lost systems, it is helpful to
mount disk partitions from a command line shell, e.g.,
\begin{verbatim}
mount /dev/hda7 /home
mount /dev/hda8 /data
\end{verbatim}

\section{rsync}\label{sxn:rsync}
\cmdidx{rsync} is \textit{the} program to synchronize non-archived
files among remote machines.
Files and directories which are not controlled by a source code
control system are notoriously hard to keep synchronized.
Examples are my directory of \trmidx{PDF} journal papers, \LaTeX\ class
files, and web directories.
The general syntax of \cmdidx{rsync} is \smpprn{rsync src dst}.
A common mistake is to use the same path depth in \cmdprn{src} and
\cmdprn{dst} arguments when normally the \cmdprn{src} path should be
one level deeper than the \cmdprn{dst} path,
\begin{verbatim}
rsync source:/drc1/drc2/drc3 destination:/drc1/drc2
rsync source:/drc1/drc2/drc3/ destination:/drc1/drc2/drc3
\end{verbatim}
The previous two commands are equivalent and the first form is preferred. 

Specifying \smpprn{-av} is usually recommended for recursive
synchronization without modifying file attributes.
Hence the typical commands to keep machines in sync are
\begin{verbatim}
# Rsync synchronization methods
# -a (equivalent to -rlptgoD): Copy recursively, preserve file modes & times 
# -v: Verbose
rsync -av /data/zender/cccac dust.ess.uci.edu:/data/zender
rsync -av /data/zender/ppr dust.ess.uci.edu:/data/zender
\end{verbatim}

Often backups are made to \trmidx{CompactFlash} or 
\trmidx{Memory Stick} devices, described in Sections~\ref{sxn:cf} and
\ref{sxn:memstick}.
\begin{verbatim}
# Backup home directory to compact flash
/bin/cp -r /home/zender /cf # Backup for dummies
# Rsync backup methods
# -a (equivalent to -rlptgoD): Copy recursively, preserve file modes & times 
# -v: Verbose
rsync -av /home/zender /cf
# Exclude backup files from being copied
rsync -av --exclude '*~' /home/zender /cf
\end{verbatim}

There are different types of back-ups. 
Often a primary source location contains all important information,
and is the first repository to remove deprecated files.
Back-ups of this primary repository, i.e., secondary repositories,
should delete these deprecated files.
\begin{verbatim}
# Delete files on receiving side that are not on sending side
rsync -av --delete-excluded /data/zender/snd dust.ess.uci.edu:/data/zender # Synchronize
rsync -av --delete-excluded dust.ess.uci.edu:/data/zender/snd /data/zender # Synchronize
rsync -av --delete-excluded --cvs-exclude --exclude '*~' /home/zender /cf # Backup 
# Backup to memory stick
rsync -av --delete-excluded --cvs-exclude --exclude='*~' /home/zender /memstick # Backup 
rsync -av --delete-excluded --cvs-exclude \
--exclude='*~' \
--exclude='*.o' \
--exclude='*.so' \
--exclude='*.a' \
--exclude='*.mod' \
--exclude='.[a-zA-Z]*' \
/home/zender /media/USB20FD/ # Backup 
\end{verbatim}

\subsection{Slink}
NCAR CGD uses a tool called \verb'slink' which makes numerous
separate software installations appear as one whole tree.
\begin{verbatim}
cd /data/zender/gsl-0.7
./configure --prefix=/contrib/gsl-0.7
make
make install
cd /contrib
newgrp contrib
co -l slink.conf
ci slink.conf
./slink
exit
\end{verbatim}

\subsection{Documentation}\label{sxn:doc}
GNU/Linux documentation is a little scattered.
\trmidx{RedHat} and \trmidx{Debian} GNU/Linux install program-specific
documentation in \verb'/usr/share/doc'.

\subsection{\trmidx{PPP}}\label{sxn:ppp}
Modem is \flidx{/dev/ttyS0}, which should be linked to \flidx{/dev/modem}. 
\flprn{/dev/modem} must be usable by all.
This should be set with control-panel.
Must get correct permissions and ownerships on various files and directories:
Line speeds, etc. have not changed from RH5 defaults
Permissions are a major security issue!
\begin{verbatim}
cd ${HOME}/linux/usr/local/bin
sudo cp ppp-go ppp-off ppp-on-dialer /usr/local/bin # Or /usr/local/sbin
cd ${HOME}/linux/etc/ppp
sudo cp options chat* *secrets /etc/ppp
\end{verbatim}
% $: rebalance syntax highlighting
Some security configurations may require that '\cmdidx{pppd} be run with
the \verb'setuid=root' bit set, and some distributions ship
\verb'pppd' with \verb'mode=644'.
In this case, use, e.g., \verb'chmod +s /usr/sbin/pppd'.

\subsection{Batch Queues}\label{sxn:q}
Two batch queue handlers are in use at NCAR and UCI, 
the \cmddfn{NQS} Network Queueing System and the AIX
\cmddfn{LoadLeveler} system.
LoadLeveler is described at
\url{http://www.scd.ucar.edu/docs/blackforest/batch.html}. 
The commands to submit jobs in these systems are
\cmddfn{qsub} and \cmddfn{llsubmit}, respectively.
The commands to query jobs in these systems are
\cmddfn{qstat} and \cmddfn{llq}, respectively.
The commands to cancel jobs in these systems are
\cmddfn{qdel -k} and \cmddfn{llcancel}, respectively.
\begin{verbatim}
qstat -a -h ute # Check batch queues on ute from utefe
qdel -h ute 12978.ute # Delete jobs on ute from utefe
\end{verbatim}

\subsection{Remote shell service rcp, rsh, ssh, telnet...}\label{sxn:dmn}
Make sure \flidx{.rhosts} is installed and NOT group/other writable!
Turn on \cmdidx{sshd}, \cmdidx{telnetd}, \cmdidx{httpd} in, at least,
run levels 3--5. 
Turn on services in runlevel 2 if they should be running even when
the \trmidx{X}-server is not.

\subsection{Root}\label{sxn:root}
Change shell to tcsh using control-panel
\begin{verbatim}
chsh -s /bin/tcsh root
\end{verbatim}
Install abbreviations to root's .cshrc file, e.g.,
\begin{verbatim}
alias m  'less'
alias h  'history'
alias csrc 'source ~/.tcshrc'
alias cd 'cd \!*; set prompt=${cwd}" ROOT"#" "'
alias dir 'ls -lga'
setenv PATH "/usr/sbin:/sbin:/bin:/usr/bin:/usr/local/bin:$PATH"
\end{verbatim}

\subsection[\LaTeX]{\LaTeX}\label{sxn:ltx}
All \LaTeX\ notes are contained in \flidx{ltx.tex} and \flidx{ltx.ps}.

\subsection{Library}\label{sxn:lbr}
Often unresolved external links are reported by a loader and the
required library must be located.
A useful procedure to follow is to change to the library directory
and use the \cmdidx{nm} command to search for the missing subroutine
\begin{verbatim}
sbr=foo
cd /usr/lib
for lib in lib*.a ; do 
#  printf "Searching ${lib} for ${sbr}...\n"
   nm ${lib} | grep ${sbr}
done # end loop over lib
\end{verbatim}

\subsection{Info}\label{sxn:info}
After installing packages in \verb'/usr/local' which install info in \verb'/usr/local/info', 
you need to update \verb'/usr/local/info/dir'

\subsection{Networking}\label{sxn:ntw}
\cmdidx{nslookup} returns information about the machine on the
Internet with a given name or \trmidx{IP address}.
Use control-panel to set network parameters such as 
\trmidx{hostname} in \flidx{/etc/hosts}, e.g.,
\begin{verbatim}
128.117.91.216  lanina.ppp.ucar.edu lanina
\end{verbatim}
and the \trmidx{nameserver} in \flidx{/etc/resolv.conf}:
\begin{verbatim}
nameserver 128.117.24.2  bearmtn-e0
nameserver 128.117.24.3  greenmtn-e0
nameserver 128.117.64.22 NCAR PPP router
nameserver 128.200.1.201 csi.ns.nts.uci.edu
nameserver 128.200.192.202 cpl2.ns.nts.uci.edu
\end{verbatim}

\subsection{C development}\label{sxn:c}
\cmdidx{kernel-headers} and \cmdidx{glibc-devel} are required to
develop \trmidx{C language} programs, but not installed by default 
\begin{verbatim}
rpm -ivh /mnt/cdrom/RedHat/RPMS/glibc-devel-2.0.5c-10.i386.rpm
rpm -ivh /mnt/cdrom/RedHat/RPMS/kernel-headers-2.0.32-2.i386.rpm
\end{verbatim}

\subsection{Time and TimeZone}\label{sxn:tm}
Use \verb'--utc' option to indicate that HW clock is kept in
\trmidx{coordinated universal time} or \trmidx{UTC}, which is
virtually identical to \trmidx{Greenwich mean time} or \trmidx{GMT}. 
This option is set during the installation of GNU/Linux.
\begin{verbatim}
/sbin/hwclock --debug
/sbin/hwclock --test --set --date="08/13/98 04:38:00" # Tests the command
/sbin/hwclock --set --date="04/09/00 15:41:00"
\end{verbatim}
A better option seems to be using \cmdidx{rdate} to set the system clock
and then \cmdidx{hwclock} to set the hardware clock to the system clock.
\begin{verbatim}
sudo /usr/sbin/rdate -s time.nist.gov
sudo /usr/sbin/rdate -s ntp.ucsd.edu
sudo /sbin/hwclock --systohc
\end{verbatim}
Note that \mchidx{time.nist.gov} may refuse connections, so consider
alternatives like \mchidx{ntp.ucsd.edu}.
Inserting this command in \flprn{/etc/rc.d/rc.local} ensures time is
set correctly on each reboot (assuming machine is on network at boot
time). 
I think this resets the hardware clock, and not just the system time.

The preferred solution is to use \trmidx{NTP}, the 
\trmidx{network time protocol}.
The NTP homepage is 
\url{http://www.eecis.udel.edu/~ntp}. 
and the NTP FAQ is
\url{http://www.eecis.udel.edu/~ntp/ntpfaq/NTP-a-faq.htm}. 
As of about 1999, few Stratum~1 timeservers will accept
synchronization requests from Stratum~3 machines.
Stratum~3 machines should synchronize with any publically accessible 
Stratum~2 server, listed, for example, at
\url{http://www.eecis.udel.edu/~mills/ntp/clock2.htm}.
I chose server \verb'132.239.254.49 = ntp.ucsd.edu'.
Alternatively, the \cmdidx{ntpdate} command works just like
\cmdidx{rdate}, and can be used when the NTP daemon itself is
unavailable. 

Changing the system timezone is described at
\url{http://www.wikihow.com/Change-the-Timezone-in-Linux}. 
\begin{verbatim}
mv /etc/localtime /etc/localtime-old # Backup old TZ file
ln -s /usr/share/zoneinfo/Europe/Amsterdam /etc/localtime # Set new TZ
/usr/bin/rdate -s time.nist.gov # Update current system time
# Set ZONE entry (e.g. "America/Los_Angeles") in /etc/sysconfig/clock
/sbin/hwclock --systohc # Set hardware clock
\end{verbatim}

\subsection{Mediawiki}\label{sxn:mediawiki}
\begin{verbatim}
aptitude install mediawiki
http://127.0.1.1
/var/www/index.html
\end{verbatim}

\subsection{Perl}\label{sxn:perl}
\begin{verbatim}
ln -s /usr/bin/perl /usr/local/bin/perl
\end{verbatim}

\subsection{Links}\label{sxn:ln2}
Whenever /home is re-installed, many links must be recreated. 
Links may be listed with
\begin{verbatim}
ls --recursive --classify | grep "@"
\end{verbatim}
For machines at NCAR, it is best to store source code in
\verb'/fs/cgd/home0/zender' and create links to \verb'${HOME}'
so that the larger object files and executables will not 
consume the expensive space on the central fileserver.
\begin{verbatim}
cd ${HOME}
mkdir -p lib/${PVM_ARCH}
mkdir -p obj/${PVM_ARCH}
mkdir -p bin/${PVM_ARCH}
mkdir include
ln -s /fs/cgd/home0/zender/sh bin/sh
ln -s /fs/cgd/data0/zender/match match
ln -s /fs/cgd/data0/zender/match_dst/dst aer
ln -s /fs/cgd/home0/zender/aca aca
ln -s /fs/cgd/home0/zender/c c
ln -s /fs/cgd/home0/zender/c++ c++
ln -s /fs/cgd/home0/zender/ck ck
ln -s /fs/cgd/home0/zender/dot dot
ln -s /fs/cgd/home0/zender/f f
ln -s /fs/cgd/home0/zender/fsf fsf
ln -s /fs/cgd/home0/zender/idx_rfr idx_rfr
ln -s /fs/cgd/home0/zender/mk mk
ln -s /fs/cgd/home0/zender/nco nco
ln -s /fs/cgd/home0/zender/perl perl
\end{verbatim}

\subsection{Install /etc/initscript to boost stacksize so CCM can run}\label{sxn:etc}

\subsection{Math Libraries}\label{sxn:mth}
Compile \flidx{libspecfun.a}.
Normally, the double precision version of \flprn{libspecfun.a},
located in \verb'src.dp' should be built and used.
However, on Alpha chips using f90, this results in \cmdprn{gamma\_} 
being un-defined.
In this case, build the single precision source (and hence function
names) using double precision flags:
\begin{verbatim}
cd specfun/src.sp
setenv FFLAGS "-r8 -i4"
make -e -f Makefile.unix
cp ../libspecfun.a $MY_LIB_DIR
\end{verbatim}
% $: rebalance syntax highlighting

\subsection{Hardware description of Zender group computers:}\label{sxn:hw}
Dell Support\\
1-800-624-9896 (general)\\
General Technical Support 1-800-234-1490 x 69080 (general) \\
Premier Support (neige) 1-866-876-3355 \\
General Technical Support for Precision Workstations: 1-800-945-3355
http://premiersupport.dell.com

For all machines:
Netmask = 255.255.255.0\\
Default gateway (subnet 14) = 128.200.14.1\\ Croul Hall
Default gateway (subnet 24) = 128.200.24.1\\ Rowland Hall (old)
Default gateway (subnet 93) = 128.200.93.1\\ Rowland Hall (new)
Default gateway (subnet 185) = 128.195.185.1\\ CalIT2
Default gateway (subnet 185) = 128.200.197.1\\ Engineering Gateway
Broadcast (subnet 14) = 128.200.14.255\\
Primary nameserver (UCI) = 128.200.1.201\\
Secondary nameserver (UCI) = 128.200.192.202\\
UCI News server = news.service.uci.edu\\
UCI POP server (newer, should work) = pop.uci.edu\\
UCI IMAP server = imap.uci.edu\\
UCI SMTP server = smtp.uci.edu\\

ashes.ess.uci.edu = 128.200.14.90 \\
Dell Inspiron 8500 \\
Arrived 20030408 \\
Service Tag C2PNM21 \\
Express Service Code 26285412457 \\
UCI Property \#: 039003448 \\
Pentium IV 2.4 GHz 512 KB L2 Cache\\
Wireless card is TrueMobile1400\\
MAC address: 00:90:4B:B2:09:86 (wireless internal chip TrueMobile1400) \\
MAC address: 00:0b:db:17:83:0d (wired) \\
aptitude install bcmwl5driverloader \\
Broadcom wireless card: http://www.linuxant.com/drivers\_bcmwl/bcmwl5/downloads-license.php \\
Installed driver Broadcom 01/09/2003, 3.10.39.0 \\
ndiswrapper driver for Broadcom wireless chip installed 20061216 as aper
http://ubuntuforums.org/showthread.php?t=25683
/cdrom/Setup/I8500/bcmwl5.inf /cdrom/Setup/bcmwl5.sys \\
License Key:  87-A7-39-A4-18-C0 \\
email address: zender@uci.edu \\
Registered as eth1 \\
hda: FUJITSU MHS2060AT, ATA DISK drive 60 GB\\
hdc: HL-DT-STCD-RW/DVD-ROM GCC-4240N, ATAPI CD/DVD-ROM drive\\
hdc: ATAPI 24X DVD-ROM CD-R/RW drive, 2048kB Cache, UDMA(33)
Intel 810 + AC97 Audio, version 0.24, 04:35:38 May  6 2003\\
Broadcom 4401 Ethernet Driver bcm4400 ver. 2.0.0 (03/25/03)\\
eth0: Broadcom BCM4401 100Base-T found at mem faffe000, IRQ 11, node addr 000bdb17830d\\
0: nvidia: loading NVIDIA Linux x86 nvidia.o Kernel Module  1.0-4349  Thu Mar 27 19:00:02 PST 2003\\
AC97 modem device forced to iobase\_0=0xb400, iobase\_1=0xb080, irq=11\\
http://www.linuxvoodoo.com/store/index.php/cPath/45\_66
Bought netgear card from Fry's on 20041211 for \$45-\$25=\$20
NetGear WG511 802.11B/G
D-Link Wireless Cardbus NIC 802.11 g 108Mbps
MAC address: 00:09:5B:E8:C4:E1 (NetGear WG511 802.11B/G) \\
20050818: Bad internal disk, problem report filed with Dell
https://wiki.ubuntu.com/HardwareSupportMachinesLaptopsDell suggests pci=noacpi,acpi=noirq
Orinoco lucent silver MAC address: 000F668EB6D3
% end ashes.ess.uci.edu

biogenic.ess.uci.edu = 128.200.14.73\\
Dell Precision 530\\
Arrived Jan 10, 2002\\
Serial number: HJKZ411\\
Service Tag HJKZ411\\
Express Service Code 38189387557\\
UCI Property \#: 019003703\\
Intel(R) Xeon(TM) CPU 1.50GHz
% end biogenic.ess.uci.edu

hp5850.ess.uci.edu = 128.200.15.157:\\
Hewlett Packard 5850 Color inkjet printer in Croul 1101\\

silt.ess.uci.edu = 128.200.14.160:\\
clay.ess.uci.edu = 128.200.14.154:\\
UCI property number (for both as Los Alamos Cluster:) 059000194
Los Alamos Computers (LAC) order \#014214
Both are dual opterons on a Tyan
Tyan S2885ANRF motherboard (onboard firewire) 
Dual AMD Opteron 246 (2.0 GHz, 1M L2 cache) 
2G PC3200 registered ECC DDR RAM (1 has 2GB of PC2100, one has 4GB of PC3200)
nVidia Corporation NV34 [GeForce FX 5200] 128MB RAM w/ video w/DVI,
3x 250GB WD SD series SATA disks WDC WD2500SD-01K
hdc: SONY DVD RW DW-D26A, ATAPI CD/DVD-ROM drive
3.5 inch floppy drive 
Onboard gigabit NIC 
Multimedia audio controller: Advanced Micro Devices [AMD] AMD-8111 AC97 Audio (rev 03)
Four port USB 2.0 PCI card 
Logitech Z560 400W speakers (4 satellites + sub) 
Chenbro SR10403 enclosure (3 case fans) 
Enermax 460W power supply (quiet and dependable) 
disks are mostly in raid5 config:
Filesystem           1K-blocks      Used Available Use% Mounted on
/dev/md1              15496084   1864536  12844376  13% /
tmpfs                  1028532         0   1028532   0% /dev/shm
/dev/md0                100954     14318     81424  15% /boot
/dev/md2             462259168     34468 438743308   1% /home
none                      5120      2708      2412  53% /dev
where the md devices are setup like this:
% cat /etc/mdadm/mdadm.conf
DEVICE partitions
ARRAY /dev/md2 level=raid5 num-devices=3 UUID=b190d39b:cad75d67:7abb3ee1:8c71f882
   devices=/dev/sda8,/dev/sdb8,/dev/sdc8
ARRAY /dev/md1 level=raid5 num-devices=3 UUID=ba4fb7ff:85a95d9c:988a6647:1d9d8f8c
   devices=/dev/sda6,/dev/sdb6,/dev/sdc6
ARRAY /dev/md0 level=raid1 num-devices=2 UUID=c8d86633:d20e14e3:f9025448:de67a792
   devices=/dev/sda5,/dev/sdb5
% end silt.ess.uci.edu, clay.ess.uci.edu
Monitors are: Viewsonic VP201b 20" 
LCD: Model number VLCDS26064-2W
Serial numbers: A21050401846, A21050401861
Connections to Cisco 3550 first floor switch are 1 Gb s-1 to jacks
75~(sand), 48~(silt), and 56~(clay). 
These occupy ports 1--3 of the total 10~jacks.
In switch closet, blue panel goes to station, yellow goes to switch.

dirt.ess.uci.edu = 128.200.14.25:\\
Dell Precision 610\\
Shipped 19990902 from Dell Computer on UCI PO\\
System Service Tag 4R5EJ\\
Express Service Code 7985179\\
Dual 500 MHz Pentium III Xeon with 512 KB Level 2 cache\\
1 GB RAM\\
Primary SCSI controller for hard disks: Adaptec AIC-7890 Ultra2/Wide LVD controller (Adaptec 2940 UW-equivalent)\\
Two 36 GB SCSI disks: QUANTUM Model: ATLAS 10K 36WLS\\
Secondary SCSI controller for CDROM: Adaptec AIC-7880 internal Ultra/Narrow and\\
external Ultra/Wide (Adaptec 2940 UW-equivalent)\\
SCSI CDROM NEC Model: CD-ROM DRIVE:466\\
Audio: System-board-integrated 16-bit Crystal CS4237B audio controller
chip which emulates Sound Blaster Pro card from Creative Laboratories, Inc. 
NIC: 3Com 3C905b-TX Wakeup On LAN-capable (uses a 3Com 3C918v2 ASIC)\\
24" Dell UltraScan P1690\\
1920x1200 @ 60 Hz, 75.0 kHz hsync\\
See http://support.dell.com/oti/monitors/P1690/En/specs.htm\\
Dell warranties monitor syncs at 1920x1200 resolution with hsync = 95 kHz, vsync = 76 Hz, dotclock = 245.5 MHz, horizontal/vertical sync polarity = -/-\\
Video controller: Diamond Viper 770D AGP PCI video adapter with 16Mb of SGRAM\\
This controller is based on the NVidia RIVA TNT2 chipset and uses the \verb'XF86_SVGA' driver\\
IOmega 250 MB ZIP drive\\
% end dirt.ess.uci.edu

elnino.ess.uci.edu = 128.200.14.97 \\
Dell Precision Mobile Workstation M50 \$4400\\
Arrived 20030129 \\
Service Tag 4RPK921 \\
Express Service Code 103-826-513-53 \\
UCI Property \#: 039003413 \\
1 GB RAM\\
15" UXGA\\
IEEE 1394\\
elnino inherited haze's second battery, which is a 66 Whr JP-04M778-42016-2CR-1656\\
24x CD R/W 8x DVD ROM\\
nVidia, Quadro?4 500 GoGL, 64MB, VGA\\
Intel Mobile Intel(R) Pentium(R) 4 - M CPU 2.20GHz stepping 07 512B L2 Cache\\
hda: IC25T060ATCS05-0, ATA DISK drive\\
hdb: Samsung CD-RW/DVD-ROM SN-324B, ATAPI CD/DVD-ROM drive\\
NB: hdb requires ide-scsi driver
hdb mounts as sr0
hdc: HITACHI\_DK23EB-40, ATA DISK drive\\
hde: SanDisk SDCFB-1024, CFA DISK drive
Floppy drive(s): fd0 is 1.44M\\
Attached scsi CD-ROM sr0 at scsi0, channel 0, id 0, lun 0\\
sr0: scsi3-mmc drive: 0x/24x writer cd/rw xa/form2 cdda tray\\
agpgart: Detected Intel i845 chipset
Intel 810 + AC97 Audio, version 0.21, 09:15:48 Aug 14 2002\\
eth0: 3c59x
eth1: Looks like a Lucent/Agere firmware version 8.10\\
Uses hermes, orinoco, and orinoco\_cs modules
MAC address: 00:08:74:E4:EC:3F (wired) \\
MAC address: 00:02:2D:85:5C:3E (wireless) \\
Firewire IEEE 1394 uses ohci1394: 
Linksys Wireless G Broadband Router:\\
S/N CDF80E406886
MAC 0013107D321C
Ownership ID 4HFFS4BW
Device ID 5ZRJG2FV
ZyXel AG200 Wireless 802.11a/b/g USB adaptor (Based on Atheros chipset)
S/N S510500149
MAC 00A0C5B810C7
http://www.zyxel.com/product
Cleaned fans 20070917 using on-line service manual
Found with search for "Dell Precision M50 Service Manual"
http://support.dell.com/support/edocs/systems/wsm50/en\_sm/index.htm
elnino's internal 60 GB disk drive died ~20080120
This was an IBM travelstar Model: IC25T060ATCS05-0 5400 RPM
Received replacement 80 GB disk drive died in 20080205
This was from Samsung 
% end elnino.ess.uci.edu

Samsung LN52A750 52-Inch 1080p DLNA LCD HDTV with RED Touch of Color
\$1891/1988
Full 1080p HD Resolution with 120Hz Auto Motion Plus
Full Wrap Touch of Color Bezel
ISP Addressable w/1GB internal memory
RSS-Infolink
Wheel Key Remote
SRS Surround Sound
Four HDMI version 1.3 connectors (3 rear/1 side) (probably HDMI Type A, single link)
One set component video (R,G,B) inputs
Two S-video inputs (one rear, one side)
Two composite inputs (one rear, one side) (Y, Pb, Pr) 
One USB 2.0 (Wiselink)
One PC Inpuy (D-Sub)
Two RF antenna inputs 
HDMI-CEC
Alternative names for CEC are Anynet (Samsung), Aquos Link (Sharp), BRAVIA Theatre Sync (Sony), Kuro Link (Pioneer), CE-Link and Regza Link (Toshiba), RIHD (Remote Interactive over HDMI) (Onkyo), Simplink (LG), HDAVI Control, EZ-Sync and VIERA Link (Panasonic), EasyLink (Philips), and NetCommand for HDMI (Mitsubishi)
DVI-HDMI Adapters are lossless and relatively inexpensive
% end samsung TV

esmf.ess.uci.edu = 128.200.197.165
% end esmf.ess.uci.edu

Sony DSCF828 Digital Camera \\
SanDisk ImageMate USB 2.0 Reader/Writer for CompactFlash Type I \& II
Model SDDR-91 
Part Number 20-90-00091
1 year warranty
SanDisk CompactFlash 1.0 GB www.sandisk.com/registration
5 year warranty
Sony Camera Case  LCS-VA3
% end Sony DSCF828 Digital Camera

haze.ess.uci.edu: 128.200.14.62\\
Dell Inspiron 8000\\
Dell order \#: 609532437\\
UCI is Dell customer \#: 6774301\\
Laptop is UCI PO \#: 0119L03200561\\
Shipped on 2001/07/31\\
System Service Tag JGP4R01\\
Express Service Code 423-685-352-17\\
Touchpad is PS/2 compatible\\
Two 59 WHR LI-ION batteries\\
Pentium III 1 GHz\\
512 MB SDRAM\\
48 GB Hard drive\\
Fixed Internal CD-RW/DVD combination drive\\
100 MB ZIP drive, interchangeable with 3.5'' floppy drive\\
15'' UXGA display\\
Video controller: NVIDIA GeForce 2 Go Video, w/ 32 MB DDR, AGP 4X\\
Grey Microsoft USB mouse with 3 buttons\\
Audio controller: ESS Maestro 3\\
Xircom RealPort 10/100 + 56K Ethernet/Modem Combo PC Card, Type 3\\
PCMCIA \verb'tulip_cb' driver:\\
\begin{verbatim}
/etc/pcmcia/config
device "tulip"
  class "network" module "cb_enabler", "tulip"
card "Xircom RealPort2 10/100 Fast Ethernet"
  version "Xircom", "*", "R2E-100BTX"
\end{verbatim}
Bought extra battery from http://www.computergiants.com
Attached to printer
% end haze.ess.uci.edu

hp5180.zendernet = fxm \\
HP Photosmart C5180 All-in-One Printer-Scanner-Copier
Purchased from Carrefour Echirolles 20070905
Takes 6-pack ink cartridges type part number 363
Hangs off zendernet router
MAC address~\#: 001a4b954ae8 \\
Serial Number: MY6CRQ217F04MK \\
Firmware Version: R0631MxNxxN0 \\
Service ID: 17249 \\
Problem printing is:
Network host '192.168.1.2' is busy; will retry in 30 seconds

ARRIS SURFboard SB6183 DOCSIS 3.0 Cable Modem\\
Purchased Amazon 20160627 \$91.00\\

Noria Neighbors' (Tony \& Donna network):
SSID=Belladonna
pw=
Speedtest.net from Ruby's room: 135 Mbps, 24 Mbps

Cox Internet Premier: Up to 150 Mbps download, 10 MBps upload Cost: \$65/mo for 12 months
My Account: 7601020171508 

ASUS RT-AC1200 Dual Band wireless router on Noria\\
Purchased Amazon 20160627 \$64.02\\
MAC Address: 203D669FD36F
Join network and login to http://router.asus.com
Custom Configuration:
adm_logyn,adm_pw: admin, pw_lw
SSID(2.4GHz)=honaunau
SSID(5.0GHz)=nakili
pw: hawaiian fish
Speed:
Speedtest.net 20160722 2.4 GHz from aerosol in Garage (ping,download,upload)=14ms,82 Mbps, 15 Mbps
Speedtest.net 20160724 2.4 GHz from aerosol in Ruby's room (ping,download,upload)=14ms,15 Mbps, 13 Mbps
Speedtest.net 20160724 2.4 GHz from aerosol in Ruby's room (ping,download,upload)=11ms,18 Mbps, 11 Mbps
Speedtest.net 20160726 2.4 GHz from aerosol in Ruby's room (ping,download,upload)=10ms,27 Mbps, 12 Mbps
Speedtest.net 20160726 2.4 GHz from aerosol in Ruby's room (ping,download,upload)=09ms,22 Mbps, 09 Mbps
Speedtest.net 20160724 5.0 GHz from aerosol in Ruby's room (ping,download,upload)=16ms,5.8 Mbps,2.4 Mbps
Speedtest.net 20160726 2.4 GHz from aerosol in Ruby's room, extender in Olivia's (ping,download,upload)=13ms,28 Mbps, 9.5 Mbps
Speedtest.net 20160726 5.0 GHz from aerosol in Ruby's room, extender in Olivia's (ping,download,upload)=11ms,40 Mbps, 5.5 Mbps
Factory Configuration:
adm_logyn,adm_pw: admin, admin
SSID(2.4G)=RT-AC1200_78_2G
SSID(5.0G)=RT-AC1200_78_5G
usr_pw: hundred_6955

TP-Link AC750 WiFi Range Extender Model RE200 on Noria\\
Purchased Amazon 20160724 \$30.00\\
Serial Number: 2165587003498
MAC Address: 18a6f7c600d8
Default SSID (2.4 GHz): TP-LINK_Extender_2.4Ghz
Default SSID (5.0 GHz): TP-LINK_Extender_5Ghz
Default access:
http://tplinkrepeater.net
adm_logyn,adm_pw: admin, admin

Netgear wireless router in SMU\\
WGR614v7
Serial~\#: 1JS2767T02DC9 \\
MAC address~\#: 001B2F5B6BFC \\
Default access: http://www.routerlogin.net \\
ESSID: zendernet
Router settings date-stamped and saved to virga: 
\begin{verbatim}
\${DATA}/tmp/netgear_wgr614v7_20070916.cfg
\${DATA}/tmp/netgear_wgr614v7_20070923.cfg
cp /home/zender/Desktop/netgear.cfg ${DATA}/tmp/netgear_wgr614v7_20070923.cfg
\end{verbatim}
Came with WG111v2 USB 802.11g card:
MAC address~\#: 00184DFF1D2C \\
S/N: 1AC174BP09583 \\

Comprehensive Static IP list for Zender computers:
\begin{verbatim}
# Obsolete
ashes.ess.uci.edu = 128.200.14.90
biogenic.ess.uci.edu    128.200.93.67
biogenic.ess.uci.edu = 128.200.14.73
dirt.ess.uci.edu        128.200.93.68
dirt.ess.uci.edu = 128.200.14.25 (née dust)
elnino.ess.uci.edu = 128.200.14.97
elnino.ess.uci.edu = 128.200.14.97
esmf.ess.uci.edu = 128.200.197.165
flyash.ess.uci.edu: none (née lanina)
haze.ess.uci.edu: 128.200.14.62
hp4600.ess.uci.edu 128.200.14.123
ipcc.ess.uci.edu 128.195.185.75
lanina.zender.org 128.200.14.80
lanina.zender.org 128.200.14.80
sand.ess.uci.edu        128.200.93.65
sand.ess.uci.edu = 128.200.14.132
seasalt.ess.uci.edu = 128.200.14.39
silt.ess.uci.edu   128.200.14.160
soot.ess.uci.edu        128.200.93.70
soot.ess.uci.edu = 128.200.14.98
virga.ess.uci.edu = 128.200.14.189
zendergradprnt.ess.uci.edu 128.200.93.72

# Croul, Engineering Gateway, CalIT2
givre.ess.uci.edu 128.200.14.205
hp4700.ess.uci.edu 128.200.14.122 # (née zendergradprnt)
neige.ess.uci.edu = 128.200.14.122
pbs.ess.uci.edu = 128.195.185.76 (née ipcc)

# Formerly in Rowland Hall 240R/240U
clay.ess.uci.edu   128.200.14.154
glace.ess.uci.edu  128.200.14.253
grele.ess.uci.edu  128.200.14.254
tephra.ess.uci.edu 128.200.14.142

# Moving into Rowland Hall 240B (20160601):
spore.ess.uci.edu  128.200.125.77
\end{verbatim}

Comprehensive Wireless MAC list for zendernet router:
\begin{verbatim}
airlink_awlc3026_pcmcia 00032F36D551
ashes_trumobile1400 00904BB20986
elnino 00022D855C3E
linksys_dlink_pcmcia fxm
netgear_wg111v2_usb 00184DFF1D2C
netgear_wg511_pcmcia 00095BE8C4E1
zyxel_a200_pcmcia 00A0C5B810C7
hp5180_printer 001a4b954ae8 # wired

gel_tmobile_g1 001841d47948 cellphone
ashes_trumobile1400 00904BB20986 .1.75 robynn ashes
ashes_orinoco_silver_pcmcia 000F668EB6D3 .1.75 robynn ashes
flyash_orinoco_silver_pcmcia 00022D0262B6 ruby flyash
neige_ipw4965 001DE0289E6D .1.77 charlie neige
givre_ipw5300 00216a294e10 .1.78 charlie givre
virga_ipw3945 0013020A7CE5 .1.76 olivia virga
\end{verbatim}

zendernet router reserved IPs:
\begin{verbatim}
192.168.1.2=hp5180
192.168.1.3=virga
192.168.1.4=elnino
192.168.1.5=neige (ipw4965)
192.168.1.6=ashes (trumobile1400 broadcom)
192.168.1.7=ashes (netgear-wg511_pcmcia)
192.168.1.8=neige (broadcom netXtreme 57xx Gigabit Controller)
192.168.1.9=givre (ipw5300)
192.168.1.= (orinoco_silver_pcmcia)
\end{verbatim}

dell5130.ess.uci.edu = 128.200.14.fxm \\
Dell 5130cdn Color Laser Printer
Dell-supplied driver:
Dell-5130cdn-Color-Laser-1.2-1.noarch.rpm
scp dust.ess.uci.edu:/data/zender/tmp/Dell\_5130cdn.ppd \${DATA}/tmp

hp4700.ess.uci.edu = 128.200.14.122 \\
256 MB DIMM RAM
32 MB flash-RAM
Serial number: JP4LB04005
Model \#: J7949E
UCI Property \#: 
Manufactured 200703

hp4600.ess.uci.edu = 128.200.14.123 \\
COLOR LASERJET 4600DN 17PPM 96MB PAR ENET 2-EIO PS3 PCL6/5C \\
Purchased from GST. Inc.
17707 Valley View Ave.
Cerritos CA 90703-7004
Arrived Jan 23, 2003 \\
Serial number: JPBKB18664 \\
UCI Property \#: 039003414 \\
17 ppm clr\\ 
600x600 DPI\\
96 MB RAM\\
10 GB disk\\
81.0180 EIO 1 ERROR messages:
If you have a 615N/J6057A card and you get EIO errors, open a ticket
for your free replacement. You have a bad card.
1-800-HPINVENT.
\begin{verbatim}
https://forums1.itrc.hp.com/service/forums/questionanswer.do?threadId=718390&admit=-682735245+1140135512205+28353475
My card is an HP615N/J6057A card 
S/N: SG2B303C8E
Case #1: 3213917944
Case #2: 3213977696
CSO#: 2689174068000
Called second time on 20060223 switched from port 2 to port 3
\end{verbatim}
% end hp4600.ess.uci.edu

Green Laser Pointer
Originally with three Vinnic L1154 batteries
http://www.batterymart.com/battery.mv?p=BAT-LR44
has a picture of these batteries and gives their technical specifications as 
Volts: 1.5, mAh: 120., Chemistry: Alkaline
Dimensions: 11.6 X 5.4 (Dia x Ht. MM)
The bumpy (negative) side faces into the pointer, against the spring
The flat (positive) side faces out of the pointer, toward the chain
I believe the pointer came with three L1154 batteries
The pointer does not work with three Rayovac 303/357 batteries
The pointer does work with four Rayovac 303/357 batteries, but the lid will not shut
Hence, the problem seems to be that the aggregate power is not enough
The Rayovac is a Silver Oxide chemistry, like the Vinnic S1154
(but not the L1154).

ipcc.ess.uci.edu = ipcc.calit2.uci.edu = 128.195.185.75\\
pbs.ess.uci.edu = pbs.calit2.uci.edu = 128.195.185.76\\
PBS wiki is/was at http://tephra.ess.uci.edu/PBSWiki\\
tephra.ess.uci.edu = 128.200.14.142\\

givre.ess.uci.edu = 128.200.14.205\\
20111024: AC adapter Case \# 844556976, Dispatch \# 149612654, 877-671-3355 x. 4340003
20111102: Received new AC adapter fedex tracking \# 468367331253 1-949-552-1717 1-800-gofedex not kevin mccain
UCI PO: 0119 L 03 905131
Dell Precision M6400
64-bit architecture uses EM64T instruction set
Ship date 20090226
Received at UCI 20090228
Dell Service Tag: C0S28J1
Express Service Code: 26168521645
5 year basics limited warranty plus 5 year NBD on-site service
NBD support until 20100226, then ...
NBD support until 20140226
MAC address: 00:22:19:DD:C6:8C (wired broadcom netXtreme 57xx) \\
MAC address: 00:21:6a:29:4e:10 (wireless ipw5300AGN) \\
Intel Core 2 Extreme Edition Quad Core QX9300, 2.53 GHz, 12MB L2 Cache, 1066 FSB, Penryn
RAM: 8 GB DDR3-1066MHz SDRAM, 4 DIMMS @ 2 GB
RAM: Dual In-Line Memory Module, 2GB 1067MHZ, 256X64, 8K, 200
Display: 17" Wide Screen WXGA+ LCD panel with microphone
Display: Liquid Crystal Display, 17WX, RGBE2E, Silver M640
Monitor: 2 @ Dell 24" UltraSharp 2408FPW Widescreen VGA/DVI/DP
Monitor: one VGA, 15 pin HD D-Sub (HD-15)
Monitor: one HDMI, 19-pin HDMI Type A
Monitor: two DVI, DVI-D dual-link
Monitor: one DP, 20-pin DisplayPort
Monitor: USB, CF, multi-card inputs
Monitor: one SVideo 4-pin mini-DIN
Monitor: one audio line-ouput
Monitor: one component video input (r,g,b RCA)
Monitor: one composite video input (RCA)
Monitor: one Hi-Speed (USB 2.0) upstream, 4-pin USB Type B
Monitor: one subwoofer output
Graphics: NVidia Quadro FX 3700M, 1 GB GDDR3 local frame buffer, 51.2 GBps memory bandwidth, http://www.nvidia.com/object/product\_quadro\_fx\_3700\_m\_us.html
Graphics: WQXGA bandwidth = 2560*1600* 60 Hz * 4 Bpp = 983040000 Bps = 983 MBps
Graphics: CUDA Parallel Processor Cores: 128
csz: intrepid version 180.29 (MIT)
csz: intrepid version 177.82 (default)
Video: one 15-hole connector (VGA)
Video: one dual-mode DisplayPort (DP) connector 
Power: Alternating Current External adapter, 210W, Lead Free Delta, World Wide
Keyboard: 101 key, United States English, E-series Number Pad Dual Point Backlit
Battery, Primary, 85WHR, 9 Lithium, Dynapack InternationalTechnology Corp, Slide
Media: Blu-ray Rewritable, double speed, 9.5, Panasonic
Storage: 2 @ 500GB hard drives, free-fall sensor, 7200 RPM, 2.5, SGT-HOLL
Bluetooth: Dell Wireless 410 Bluetooth 2.1 + EDR Module with UWB
Camera: Integrated 2.0 MP webcam
Wireless LAN: Intel WiFi Link 5300 802.11a/g/n draft
Network: Broadcom 10/100/1000 Ethernet LAN
PC Card: Ricoh R5C847 Supports one Type I or Type II Card
ExpressCard: Supports 34 mm and 54 mm ExpressCards
Smart Card: Supports ISO-7816-3 Type I and Type II and ISO-7816-12
Mini-Card: One dedicated half Mini-Card slot for WLAN
Mini-Card: One dedicated full Mini-Card slot for WWAN (mobile broadband)
Mini-Card: One dedicated full Mini-Card slot for WPAN (Bluetooth or ultra wideband (UWB)))
CardBus/PCIMIA slot: Supports Type I/II cards
USB: two 4-pin USB 2.0-compliant connectors
USB: one 4-pin USB 2.0-compliant PowerShare connector
USB: one eSATA/USB 2.0-compliant connector
IEEE 1394 (firewire): powered 6-pin
Memory Card Reader: 8-in-1 memory card reader supports: 
SD, SDIO, MMC, XD, MS, MS-Pro, Mini-MMC, MMC+, Hi-density SD, Hi-capacity SD
Keyboard: Bluetooth Dell BT: 00:07:61:A2:1D:39 PIN: 0000 (20100204: no longer used)
Keyboard: Logitech cordless USB LX310 Laser (20100204)
Mouse: Logitech cordless USB LX310 Laser (20100204)
Mouse: Bluetooth Dell BT Travel Mouse: 00:07:61:9B:DA:2B PIN: 0000

givre port replicator: \$199
Manufacturer Part\#: R537F
Dell Part\#: 430-3312
E-Quote number: 1009444033278
E/Port Plus 210W Port Replicator for Dell Precision Mobile WorkStation M6400
one serial RS-232 9-pin D-Sub (DB-9) 
one parallel 25-pin D-Sub (DB-25) 
one keyboard generic 6-pin mini-DIN (PS/2 style) 
one mouse generic 6-pin mini-DIN (PS/2 style) 
two USB 4-pin USB Type A 
one network RJ-45 
one VGA 15-pin HD D-Sub (HD-15) 
two DVI-Digital 24-pin DVI (hardware compatible with dual-link, but most Dell laptops only pass single-link signals through replicator to GPU)
two DP 20-pin DisplayPort
one audio
NVidia settings lists connections from port replicator as:
DFP0: Connection link: Dual
DFP0: Native/Best Fit/Frontend/Backend Resolution: fxm
DFP0: Signal: LVDS
DFP3/DFP4: Connection link: 4 lanes @ 2.70 Gbps
DFP3/DFP4: Native/Best Fit/Frontend/Backend Resolution: 2560x1600
DFP3/DFP4: Signal: DisplayPort

Two Dell DELL UltraSharp 3008WFP 30-inch Widescreen Flat Panel 
2560x1600 30" LCD Monitors
370 cd/m² brightness and 3000:1 contrast ratio
TrueColor Technology producing 117% of the NTSC color gamut
8-ms response time (grey-to-grey)
Monitor: one VGA, 15 pin HD D-Sub (HD-15)
Monitor: one HDMI, 19-pin HDMI Type A
Monitor: two DVI, DVI-D dual-link
Monitor: one DP, 20-pin DisplayPort
Monitor: four Hi-Speed (USB 2.0) downstream, 4-pin USB Type A
Monitor: CF, multi-card inputs
Monitor: one SVideo 4-pin mini-DIN
Monitor: one audio line-ouput
Monitor: one component video input (r,g,b RCA)
Monitor: one composite video input (RCA)
Monitor: one Hi-Speed (USB 2.0) upstream, 4-pin USB Type B
Monitor: one subwoofer output
Monitor: 250 W active, 2 W stand-by/sleep

Brother PT-1650 Cable Labeler
Received 20090608 from newegg.com
S/N: M8G652379

Epson PowerLite Home Cinema 6500 UB

glace.ess.uci.edu = 128.200.93.87\\ Xianwei's computer
grele.ess.uci.edu = 128.200.93.88\\ Bob's computer

neige.ess.uci.edu = 128.200.14.122\\
UCI PO: 0119 L 03 803727
Dell Precision M6300
64-bit architecture uses EM64T instruction set
Received at UCI/SMU 20071221/20071228
Dell Service Tag: JB1S8F1
Complete Care until 20111213
Gold support until 20111213
Express Service Code: 42026989213
4 year limited warranty plus 4 year NBD on-site service and complete care
Gold technical support issues:
20090111 Case \#: 631107288 AC Power supply. Spoke to Trevor
Intel Core2 Extreme CPU X7900 2.8 GHz 800 MHz FSB\\
17" UltraSharp Wide Screen WUXGA\\
Bluetooth 2.1 with EDR\\
24x CD-RW/DVD burner, 8x DVD+/-RW, DVD-ROM, Blu-Ray Re-writable\\
4 GB RAM
NVidia Quadro FX 1600M 512 MB
Windows Vista Business Product ID: 89576-OEM-7332141-00054
Intel Wireless WiFi 4965AGN
Broadcom NetXtreme 57xx Gigabit Controller
MAC address: 001C231F2730 (wired broadcom) \\
MAC address: 001DE0289E6D (wireless ipw4965AGN) \\
One FireWire IEEE 1394 connector
Six USB 2.0 connectors
One DVI-D (Single-Link) connector 
One VGA connector
One S-video connector
D-Family Advanced Port Replicator
One 54mm Express Card slot, supports both 54mm and 34mm I/O
Integrated Smart Card Reader
UPEK fingerprint reader
Sound controller problems:
\begin{verbatim}
https://wiki.ubuntu.com/Gutsy_Intel_HD_Audio_Controller
\end{verbatim}
Get video working:
nvidia-glx-new
UPEK Fingerprint Reader \\
\begin{verbatim}
http://thinkfinger.sourceforge.net
http://packages.ubuntu.com/hardy/admin/libpam-thinkfinger
aptitude install ntfs-3g
Windows Vista NTFS: mkdir /wnd;mount -t ntfs-3g /dev/sda3 /wnd
\end{verbatim}

neige port replicator: \$159-\$269
Manufacturer Part \# A0610225 Video card Radeon 64MB graphic Half-height 
Manufacturer Part \# PD891 D/Dock 64MB graphic supports half-height PCI
DELL D/Dock Expansion Station for Latitude D Series
DELL D/Dock Expansion Station for Precision M6300
For Precision M6300: Manufacturer Part\# : PD891 | Dell Part\# : 310-8112
For XPS M1710      : Manufacturer Part\# : PD891 | Dell Part\# : 310-8556
The PCI expansion slot in the DELL D/Dock Expansion Station 
supports one "half-height"/"low-profile" card. 
Only cards labeled as "half-height" or "low-profile" will fit.
NVIDIA Quadro FX 370 Low Profile
Drives digital displays at resolutions up to 1920 x 1200 at 60Hz and
analog displays at resolutions up to 2040 x 1536 at 85Hz. 
Display Connectors 	DMS-59
Single-Link DVI 	2 (with supplied adapter cable)
VGA 	2 (with supplied adapters)
or nVIDIA GeForce 7300

Epson PowerLite Home Cinema (EPHC) 6500UB
EPHC 1080 UB was top-rated by CR in 2008
http://www.projectorreviews.com/epson/home-cinema-6500ub/index.php
December 2008: \$2,999, \$2,929.54 at Amazon, \$3,059.99 at Dell
Brightness: 1600 Lumens, Contrast: 75000:1, Technology: 0.7" 3 LCD
Lamp Life: 4000 Hours, Audible Noise: 31 dB, Warranty: 2 Years
Purchased from Dell: 20090315
Received at UCI: 20090421
Two year technical support: 20090421--20110420
Epson Computers PrivateLine technical support: (800) 637-7661 PYN 8789402
Same as Dell-sold...
Manufacturer Part\# V11H292020
Dell Part\# A2402327
UNSPSC 4511160900
Interfaces:
Two HDMI 1.3a
One RCA (Composite)
One 1x3 RCA (Component)
One Mini DIN (S-video)
One D-Sub 15 pin (Analog RGB)
One D-Sub 9 pin (RS-232c)
Input Signals:
NTSC/ NTSC4.43/ PAL/ M-PAL/ N-PAL/ PAL60/ SECAM/ 480i/ 576i/ 480p/ 576p/ 720p/ 1080i/ 1080p
Video I/O
HDMI
HDTV
Component video
S-video
Composite video
NTSC/ PAL/ SECAM
Analog RGB (PC only)

chromebook = noname \\
Samsung Chromebook (Wi-Fi, 11.6 inch)
Received 20130322
Cost \$249
Personal use, not UCI property (actually I teach with it)
Built-in dual band Wi-Fi 802.11 a/b/g/n
Screen Resolution	1366 x 768
Processor	1.7 GHz Exynos 5000 Series
RAM	2 GB DDR3L SDRAM
Memory Speed	1333.00
Hard Drive	16 GB eMMC Solid State Drive
Graphics Coprocessor	Integrated Graphics
Wireless Type	802.11 a/b/g/n
Number of USB 2.0 Ports	1
Number of USB 3.0 Ports	1
HDMI Port

katabatic.ess.uci.edu = 128.200.14.42 = katabatic
UCI PO:fxm
Used by Matt Laffin
Apple MacBook Pro 15-inch Gray Space
Model A1707
Received at UCI 20171012
Apple Serial \#: C02VH1AGHTDF
AppleCare Protection Plan Agreement \#: 1006777628
15.4 inch MacBook Pro
2 TB Flash
3.1 GHz
16 GB RAM
MAC address: 8c85905415df (aka AirPort ID)
Bluetooth ID: 8c859058e3a7
Mac OS X 10.12.6 High Sierra
Retina display: 15.4-inch diagonal
IPS technology 2880x1800 resolution
RAM: 16 GB DDR3L 1600MHz
Disk: 2 TB PCIe-based flash storage
AMD Radeon Pro 560 4096 MB Intel HD Graphics 630 1536 MB
Simultaneously supports full native resolution on the built-in display at millions of colors and:
Up to two displays with 5120-by-2880 resolution at 60Hz at over a billion colors
Up to four displays with 4096-by-2304 resolution at 60Hz at millions of colors
Up to four displays with 3840-by-2160 resolution at 60Hz at over a billion colors
HDMI output: fxm
CPU: 3.1GHz Quad-core Intel Core i7
USB: none
Thunderbolt: Four Thunderbolt 3 Ports
Bluetooth 4.2
802.11ac Wi-Fi wireless networking; IEEE 802.11a/b/g/n compatible
Force Touch trackpad
76.0-watt-hour lithium-polymer battery
87W USB-C Power Adapter

rigb.ess.uci.edu = 128.200.125.98 = rigb
UCI PO:fxm
Used by Wenshan Wang
Apple MacBook Pro 15-inch Gray Space
Model A1707
Received at UCI 20171012
Apple Serial \#: C02VL29FHTDF
AppleCare Protection Plan Agreement \#: 1006880919
15.4 inch MacBook Pro
2 TB Flash
3.1 GHz
16 GB RAM
MAC address: 58:ef:68:7e:a2:64 (aka AirPort ID)
Bluetooth ID: 8C:85:90:6D:19:31
Mac OS X 10.12.6 High Sierra
Retina display: 15.4-inch diagonal
IPS technology 2880x1800 resolution
RAM: 16 GB DDR3L 1600MHz
Disk: 2 TB PCIe-based flash storage
AMD Radeon Pro 560 4096 MB Intel HD Graphics 630 1536 MB
Simultaneously supports full native resolution on the built-in display at millions of colors and:
Up to two displays with 5120-by-2880 resolution at 60Hz at over a billion colors
Up to four displays with 4096-by-2304 resolution at 60Hz at millions of colors
Up to four displays with 3840-by-2160 resolution at 60Hz at over a billion colors
HDMI output: none
CPU: 3.1GHz Quad-core Intel Core i7
USB: none
Thunderbolt: Four Thunderbolt 3 Ports
Bluetooth 4.2
802.11ac Wi-Fi wireless networking; IEEE 802.11a/b/g/n compatible
Force Touch trackpad
76.0-watt-hour lithium-polymer battery
87W USB-C Power Adapter

Sagar's machine
psagar.ess.uci.edu = 128.200.125.76
Inspiron 3847
Cost: \$896.39 including tax

skyglow.ess.uci.edu = 128.200.125.81 (formerly 128.200.14.6) = skyglow
Now used by Cheng Dang
Dell Precision 5510 (for short)
Full name: Dell Mobile Precision 15 5000 series 5510
UCI PO: 
Received at UCI 20160921
Dell Serial \#: 
Dell Service Tag: D6HCNC2
Express Service Code: 28690111010
32 GB RAM
1 TB flash
CPU 
\$fxm
MAC address:  (WiFi)
MAC address: 28f10e4d3c73 (Wired)
RAM: 32 GB 
Disk:  TB PCIe SSD
WiFi: fxm
USBC
Power: 130W

Monitor: Dell P4317Q
Diagonal: 42.5" (aka Dell 43 Multi-Client Monitor)
Resolution: 3840x2160 (= 4 x 1920x1080 = 4kUltra HD 4K2)
Dual 8W speakers
Ports: DP1.2, mDP, dual HDMI 1.4, VGA, four-port USB3.0

frazil.ess.uci.edu = 128.200.14.5 = frazil
Dell Precision 5510 (for short)
Full name: Dell Mobile Precision 15 5000 series 5510
UCI PO: 
Received at UCI 20160301
Dell Serial \#: 
Dell Service Tag: J160N72 (new)
Dell Service Tag: 2WW7M72 (old)
Express Service Code: 6342585374
20160307: Called to report inability to boot to BIOS
20160315: Replaced with new system
Case #: 926347479
Phone: 1-800-822-8965
\$3069
MAC address: 525400602dae (WiFi)
MAC address: 9cebe8309cb6 (Wired)
Bluetooth ID: 
Display: 15.6'' UltraSharp FHD IPS (1920x1080) 
RAM: 32 GB 
Disk: 1 TB PCIe SSD
WiFi: Intel WiFi Link 8260 2x2 80211ac+BT 4.1
USBC
Power: 130W

firn.ess.uci.edu = 128.200.14.25 = firn
UCI PO: 
Apple Macbook Pro Retina Display
Model A1398
Received at UCI 20150702
Apple Serial \#: C02PX49RG8WM
AppleCare Protection Plan Agreement \#: 4306813717 expires 20180625
15.4 inch MacBook Pro
1 TB Flash
2.6 GHz
16 GB RAM
MAC address: a0999b06ae49 (aka AirPort ID)
Bluetooth ID: 
Mac OS X 10.10 Yosemite
Retina display: 15.4-inch diagonal
IPS technology 2880x1800 resolution
RAM: 16 GB DDR3L 1600MHz
Disk: 1 TB PCIe-based flash storage
AMD Radeon R9 M370X 2048 MB
Simultaneously supports full native resolution on built-in display and up to 2560x1600 on up to two external displays
HDMI output: support 1080p at 60 Hz, 3840x2160 at 30 Hz, 4096x2160 at 24 Hz
CPU: 2.8 GHz quad-core Intel Core i7 with 6 MB shared L3 cache
USB: 2 USB3 ports
Thunderbolt: 2 Thunderbolt2 ports
SDXC Card slot
Bluetooth 4.0
WiFi 802.11a/b/g/n (NB: Yes, /n)
85 W MagSafe 2 power port
Force Touch trackpad
99.5 W-hr lithium-polymer battery
IOGear.com USB 3.0 GigaLinq S/N: OU56USG4N00092 (purchased ~\$22 UCI Bookstore 20151002)
NB: After each OS upgrade, obtain/reinstall IOGear driver from
https://www.iogear.com/support/dm/driver/GUC3100
This requires a restart after install...

aerosol.ess.uci.edu = 128.200.14.160 = aerosol
UCI PO: 
Apple Macbook Pro Retina Display
Model A1398
Received at UCI 20140528
Apple Serial \#: C02MQ1HGFD59
AppleCare Protection Plan Agreement \#: 6212819251 expired 20150528
15.4 inch MacBook Pro
1 TB Flash
2.6 GHz
16 GB RAM
MAC address: 3c15c2d43010 (aka AirPort ID)
Bluetooth ID: 3c15c2d43011
Mac OS X 10.9 Mavericks
Retina display: 15.4-inch diagonal
IPS technology 2880x1800 resolution
RAM: 16 GB DDR3L 1600MHz
Disk: 1 TB PCIe-based flash storage
NVIDIA GeForce GT 750M with 2 GB GDDR5 memory
Simultaneously supports full native resolution on built-in display and up to 2560x1600 on up to two external displays
HDMI output: support 1080p at 60 Hz, 3840x2160 at 30 Hz, 4096x2160 at 24 Hz
CPU: 2.6 GHz quad-core Intel Core i7 with 6 MB shared L3 cache
USB: 2 USB3 ports
Thunderbolt: 2 Thunderbolt2 ports
SDXC Card slot
Bluetooth 4.0
WiFi 802.11a/b/g 
MagSafe 2 power port

roulee.ess.uci.edu = 128.200.14.fxm = roulee = roulée \\
UCI PO: 0119 L 03 115364
Dell Precision M6500
64-bit architecture uses EM64T instruction set
Ship date 20110322
Received at UCI 20110325
Repaired for overheating 20140312
Dell Service Tag: FL97YP1
Express Service Code: 33937012837
5 year basics limited warranty plus 5 year NBD on-site service
NBD support until , then ...
NBD support until 
MAC address: 64:80:99:09:61:D0 (Wireless Intel Centrino Advanced-n 6250 AGN) \\
MAC address: 14:FE:B5:9B:8C:90 (Wired Broadcom NetXtreme 57xx Gigabit Controller) \\
MAC address: 90:00:4E:E0:D5:D4 (Bluetooth) \\
MAC address: 00:1D:E1:43:65:E4 (Intel Centrino WiMAX 6250) \\
Windows 7 Version: Professional
Windows 7 Homegroup: pw: ih1Ha6fJ8h
Intel Core 2 Extreme Edition Quad Core QX9300, 2.53 GHz, 12MB L2 Cache, 1066 FSB, Penryn
RAM: 8 GB DDR3-1066MHz SDRAM, 4 DIMMS @ 2 GB
RAM: Dual In-Line Memory Module, 2GB 1067MHZ, 256X64, 8K, 200
Display: 17" Wide Screen WXGA+ LCD panel with microphone
Display: Liquid Crystal Display, 17WX, RGBE2E, Silver M640
Graphics: NVidia Quadro FX 3700M, 1 GB GDDR3 local frame buffer, 51.2 GBps memory bandwidth, http://www.nvidia.com/obje[ct/product\_quadro\_fx\_3700\_m\_us.html
Graphics: WQXGA bandwidth = 2560*1600* 60 Hz * 4 Bpp = 983040000 Bps = 983 MBps
Graphics: CUDA Parallel Processor Cores: 128
csz: intrepid version 180.29 (MIT)
csz: intrepid version 177.82 (default)
Video: one 15-hole connector (VGA)
Video: one dual-mode DisplayPort (DP) connector 
Power: Alternating Current External adapter, 210W, Lead Free Delta, World Wide
Keyboard: 101 key, United States English, E-series Number Pad Dual Point Backlit
Battery, Primary, 85WHR, 9 Lithium, Dynapack InternationalTechnology Corp, Slide
Media: Blu-ray Rewritable, double speed, 9.5, Panasonic
Storage: 2 @ 500GB hard drives, free-fall sensor, 7200 RPM, 2.5, SGT-HOLL
Bluetooth: Dell Wireless 410 Bluetooth 2.1 + EDR Module with UWB
Camera: Integrated 2.0 MP webcam
Wireless LAN: Intel WiFi Link 5300 802.11a/g/n draft
Network: Broadcom 10/100/1000 Ethernet LAN
PC Card: Ricoh R5C847 Supports one Type I or Type II Card
ExpressCard: Supports 34 mm and 54 mm ExpressCards
Smart Card: Supports ISO-7816-3 Type I and Type II and ISO-7816-12
Mini-Card: One dedicated half Mini-Card slot for WLAN
Mini-Card: One dedicated full Mini-Card slot for WWAN (mobile broadband)
Mini-Card: One dedicated full Mini-Card slot for WPAN (Bluetooth or ultra wideband (UWB)))
CardBus/PCIMIA slot: Supports Type I/II cards
USB: two 4-pin USB 2.0-compliant connectors
USB: one 4-pin USB 2.0-compliant PowerShare connector
USB: one eSATA/USB 2.0-compliant connector
IEEE 1394 (firewire): powered 6-pin
Memory Card Reader: 8-in-1 memory card reader supports: 
SD, SDIO, MMC, XD, MS, MS-Pro, Mini-MMC, MMC+, Hi-density SD, Hi-capacity SD
Keyboard: Bluetooth Dell BT: 00:07:61:A2:1D:39 PIN: 0000 (20100204: no longer used)
Keyboard: Logitech cordless USB LX310 Laser (20100204)
Mouse: Logitech cordless USB LX310 Laser (20100204)
Mouse: Bluetooth Dell BT Travel Mouse: 00:07:61:9B:DA:2B PIN: 0000

roulée port replicator: \$199
Manufacturer Part\#: R537F
Dell Part\#: 430-3312
E-Quote number: 1009444033278
E/Port Plus 210W Port Replicator for Dell Precision Mobile WorkStation M6500
one serial RS-232 9-pin D-Sub (DB-9) 
one parallel 25-pin D-Sub (DB-25) 
one keyboard generic 6-pin mini-DIN (PS/2 style) 
one mouse generic 6-pin mini-DIN (PS/2 style) 
two USB 4-pin USB Type A 
one network RJ-45 
one VGA 15-pin HD D-Sub (HD-15) 
two DVI-Digital 24-pin DVI (hardware compatible with dual-link, but most Dell laptops only pass single-link signals through replicator to GPU)
two DP 20-pin DisplayPort
one audio
NVidia settings lists connections from port replicator as:
DFP0: Connection link: Dual
DFP0: Native/Best Fit/Frontend/Backend Resolution: fxm
DFP0: Signal: LVDS
DFP3/DFP4: Connection link: 4 lanes @ 2.70 Gbps
DFP3/DFP4: Native/Best Fit/Frontend/Backend Resolution: 2560x1600
DFP3/DFP4: Signal: DisplayPort

Two Dell DELL UltraSharp U3011 30-inch Widescreen Flat Panel 
20120701: These monitors, unlike the 3008 WFPs used on givre,
do not automatically resync once put into sleep mode.
Based on suggestion on Dell board, changed monitor settings
to disable DDC/CI (Menu->Other Setting->DDC/CI) on 20120701.
2560x1600 30" LCD Monitors
Active matrix, thin-film transistor (TFT), liquid crystal display (LCD).
370 cd/m² brightness and 3000:1 contrast ratio
TrueColor Technology producing 117% of the NTSC color gamut
8-ms response time (grey-to-grey)
Monitor: one VGA, 15 pin HD D-Sub (HD-15)
Monitor: one HDMI, 19-pin HDMI Type A
Monitor: two DVI, DVI-D dual-link
Monitor: one DP, 20-pin DisplayPort
Monitor: four Hi-Speed (USB 2.0) downstream, 4-pin USB Type A
Monitor: CF, multi-card inputs
Monitor: one SVideo 4-pin mini-DIN
Monitor: one audio line-ouput
Monitor: one component video input (r,g,b RCA)
Monitor: one composite video input (RCA)
Monitor: one Hi-Speed (USB 2.0) upstream, 4-pin USB Type B
Monitor: one subwoofer output
Monitor: 250 W active, 2 W stand-by/sleep

\begin{verbatim}
# Dell site for BIOS updates
# http://linux.dell.com/wiki/index.php/Repository/firmware
# deb http://linux.dell.com/repo hardy dell-software
# http://direct2dell.com/one2one/archive/category/1021.aspx
# Four steps to firmware updates
# RPMs:
sudo wget -q -O - http://linux.dell.com/repo/community/bootstrap.cgi | bash # Set-up repos
sudo wget -q -O - http://linux.dell.com/repo/firmware/bootstrap.cgi | bash # Set-up repos
sudo dnf -y install firmware-addon-dell # Install firmware tools
sudo dnf -y install $(bootstrap_firmware) # Install BIOS update
sudo update_firmware
# .debs:
sudo wget -q -O - http://linux.dell.com/repo/firmware/bootstrap.cgi | bash
sudo aptitude install firmware-tools firmware-addon-dell
sudo aptitude install $(bootstrap_firmware -a)
sudo update_firmware
# Messages like "Couldn't find any package pci-firmware-*" can be safely ignored
# They are working as designed.
\end{verbatim}

virga.ess.uci.edu = 128.200.14.189 \\
20090306: Warranty/Battery Issue (888) 242-0938 \\
\url{James_Burns_Jr@Dell.com}
1-866-876-3355 x43-40997
Case Number: 636212602
\url{JoAnn_Soliz@Dell.com}
Dell Inspiron 9400 \\
Received at UCI 20060221 \\
Dell Service Tag: F1N0J91 \\
Express Service Code: 327-508-573-33 \\
4 year limited warranty plus 4 year NBD on-site service and complete care\\
Gold technical support\\
Intel Core Duo Processor 2 GHz/667 MHz FSB\\
17" UltraSharp Wide Screen UXGA WUXGA\\
8x CD/DVD burner (DVD+/-RW) with double layer DVD+R write capability\\
2 GB RAM
Logitech MX1000 Laser Cordless mouse: 29.1 ROM GP
Logitech Resolution 800 DPI, 5.8 MP s-1
Logitech S/N: LZB533350679
MAC address: 00:14:22:EF:61:8E (wired) \\
MAC address: 00:13:02:0A:7C:E5 (wireless ipw3945) \\
Bluetooth Dell Wireless 350 Bluetooth Internal (2.0+ enhanced data rate)\\
Bluespoon AX2 (lost in 2009)
Logitech QuickCam Pro for notebooks
IEEE 1394 \\
scsi0 : ata\_piix Vendor: ATA Model: Hitachi HTS72101  Rev: MCZO
scsi1 : ata\_piix Vendor: \_NEC Model: DVD+-RW ND-6650A  Rev: 102C
SCSI device sda: 192426570 512-byte hdwr sectors (98522 MB)
1 PCI Express card slot
Zero pcmcia slots
Video: 256 MB Nvidia GeForce Go 7800
eth0: Broadcom 4400 10/100BaseT Ethernet 00:14:22:ef:61:8e
Intel PRO/Wireless 3945 80211a/g minicard\\
\# Sound: http://wiki.ubuntu.com/DebuggingSoundProblems 
\# Turn off internal speakers when headphones plugged in
\# echo options snd-hda-intel model=ref | sudo tee -a /etc/modprobe.d/alsa-base
\begin{verbatim}
# Ubuntu wireless cards:
https://wiki.ubuntu.com/HardwareSupportComponentsWirelessNetworkCards
# This helpful site:
http://www.thinkwiki.org/wiki/Intel_PRO/Wireless_3945ABG_Mini-PCI_Express_Adapter
# mentions that the correct driver is available from sourceforge at
http://ipw3945.sourceforge.net/
# The upshot is that the ipw3945 driver is expected to be available Q1 2006
# When complete, Intel/Linux wireless drivers are available at
http://support.intel.com/support/notebook/sb/CS-006408.htm
# Thinkwiki used to (20060215) recommend using the ipw2200 driver
# Here's what happens when you try that:
http://www.thinkwiki.org/wiki/Ipw2200#Installation_on_Debian
aptitude remove linux-headers-2.6.15-18-686 linux-headers-2.6.15-18
aptitude install linux-headers-2.6.15-20-686
sudo aptitude install ieee80211-source 
sudo module-assistant -t build ieee80211-source 
sudo aptitude install ipw2200-source 
sudo module-assistant -t build ipw2200-source 
# Install firmware
cd ${DATA}/tmp
sudo tar xzvf ipw3945-linux-1.0.0.tgz
cd ${DATA}/tmp/intel-ipw3945-1.0.0/
 -C /lib/hotplug/firmware
modprobe ipw2200 
iwconfig 
lsmod | grep ipw
# Module loads but eth1 does not show up in ifconfig
\end{verbatim}

flyash.ess.uci.edu (originally named lanina):\\
usually dhcp-14116.ess.uci.edu, dhcp-14118.ess.uci.edu (wired) or dhcp-086159.mobile.uci.edu,dhcp-086183.mobile.uci.edu (wireless) \\
Dell Inspiron 7500R\\
Dell order \#: 320036478\\
UCI is Dell customer \#: 6774301\\
Laptop is UCI PO \#: 0120L03002256\\
Shipped on 20000111\\
System Service Tag Y20Z8\\
Express Service Code 572-015-24\\
Touchpad is PS/2 compatible\\
Pentium mobile III 500 MHz\\
512 MB 100 MHz RAM\\
25 GB Hard disk\\
120 MB Super disk, aka LS120 (mounted as /mnt/ls120 = /dev/hdc)\\
Display is 15.0" SXGA+ active matrix color \\
Video controller: ATI Mobility P, 64 bit, AGP 2X w/ 8 MB RAM (\verb'ati' X.org server)\\
Controller supports 32 bit color but LCD display limited to 18 bit\\
Infrared IrDA 1.1 port\\
Logitech first mouse 3 button PS/2 compatible\\
Audio controller: ESS Technology Maestro-2e (Sound Blaster Pro-compatible)\\
udev:DEVPATH=/bus/pci/drivers/ES1968 (ESS Maestro)
udev:UDEV  [1170546089.233339] add@/module/snd\_es1968
udev:PHYSDEVDRIVER=ES1968 (ESS Maestro)
http://occy.net/taxonomy/term/6?from=20:
"In order to get sound working on my Mom's Dell Inspiron 8200, using
Ubuntu Linux, I had to do the following:"
\begin{verbatim}
 # edit /etc/hotplug/blacklist
  add snd_intel8x0m
  edit /boot/grub/menu.lst
  add # kopt=root=/dev/hda1 ro acpi_irq_isa=7
#  make SURE you leave in the # above
  run update-grub
\end{verbatim}
Wireless: Lucent technologies Orinoco silver card
Orinoco MAC address: 00:02:2D:09:82:E2
Xircom 32bit Cardbus Ethernet 10/100 + Modem 56 (aka CBEM56G 1.03) \\
Xircom MAC address: 00:10:A4:08:12:31
PCMCIA \verb'tulip_cb' driver:\\
\begin{verbatim}
/etc/pcmcia/config
device "tulip_cb"
        class "network" module "cb_enabler", "tulip_cb"
card "Xircom CardBus 10/100 Ethernet + 56K Modem"
        version "Xircom", "*", "CBEM56G"
        bind "tulip_cb" to 0, "serial_cb" to 1
\end{verbatim}
CBEM56G ethernet cable: CABLE-ETH122
CBEM56G modem cable: CABLE-MOD444
PCI controller: Texas Instruments PCI 1225 Cardbus\\
Airlink AWLC3026 802.11b/g card purchased 20061216 from Fry's:
MAC 00:03:2F:36:D5:51
\begin{verbatim}
hda: IBM-DARA-225000, ATA DISK drive\\
hdc: LS-120 SLIM3 00 UHD Floppy, ATAPI FLOPPY drive\\
hdd: TorISAN DVD-ROM DRD-U624, ATAPI CDROM drive\\
hda: IBM-DARA-225000, 24207MB w/418kB Cache, CHS=3278/240/63\\
hdd: ATAPI 0X DVD-ROM drive, 256kB Cache\\
\end{verbatim}
Video modes supported:\\
\begin{verbatim}
720x400 16 colors @ 70 Hz text mode
1280x1024 8,16,24,32 bit @ 100 Hz
1400x1050 8,16,24 bit @ 100 Hz
1400x1050 32 bit @ 85 Hz
1600x1200 8,16,24 bit @ 100 Hz
1600x1200 32 bit @ 85 Hz
\end{verbatim}
Do horizontal and vertical refresh rates have any meaning for TFT/LCD
displays?
Installation programs ask for these variables and the following
settings are reported to work for the i7500 at 1400x1050: hsync range
31.5 -82, vrefresh 40-110. 
For haze the horizontal refresh should be set at 28--90, while the
vertical should be set at 40--110.
\textbf{NB: ATI RAGE card at 1400x1050 resolution does not work on
Fedora/Ubuntu unless magic option ``vga=792'' is added to
kernel boot line, e.g., in grub.conf or menu.lst}
% end flyash.ess.uci.edu

lanina.zender.org = 128.200.14.80\\
Compaq Presario 5240\\
Purchased 19990312 from CompUSA in Boulder ~\$1100\\
AMD K6-2 3d Processor at 400 MHz\\
Serial number 1X91CFDH8662\\
10 GB Hard disk\\
hjm:03-03-05 added 200GB disk, added ProMEPIS (Debian) OS, in following partitions:
hda1-10GB -OS, hda2-1GB swap, hda3- /home (~78GB) hda4 -/data(spare)
64 MB RAM\\
hjm: 03-03-05 upgraded to 256MB (with simms from home)
120 MB Super disk, aka LS120 (mounted as /mnt/ls120 = /dev/hdd)\\
\begin{verbatim}
hda: WDC AC310000B, 9541MB w/512kB Cache, CHS=1292/240/63
hda: WDC AC310000B, ATA DISK drive
hdc: LTN382, ATAPI CDROM drive
hdc: ATAPI 40X CD-ROM drive, 120kB Cache
hdd: LS-120 COSM 05 UHD Floppy, ATAPI FLOPPY drive
\end{verbatim}
Floppy drive(s): fd0 is 1.44M\\
Rockwell HCF 56 kbps DataFax modem on PCI bus COM2 (/dev/ttyS1)\\
3COM-US Robotics 56 kbps winmodem on ISA bus COM1 (/dev/ttyS0) IRQ4 \\
NIC: PCI Fast ethernet DEC 21143-based controller\\
Graphics: Rage LT PRO AGP 2X (\verb'XF86_Mach64' server)\\
USB\\
PCI disk controllers\\
\begin{verbatim}
# Graphics: Rage LT PRO AGP 2X
# 19" Monitor purchased separately 19990901 from CompUSA ~\$350:
# Horizontal scan frequency 30--95 KHz
# Vertical scan frequency 47--150 Hz
# Supports 1024x768 @ 68.6 KHz / 85 Hz non-interlaced
# Supports 1280x1024 @ 80 KHz / 75 Hz non-interlaced
# Supports 1600x1200 @ 93.7 KHz / 75 Hz non-interlaced
\end{verbatim}
\begin{verbatim}
# Computed following modeline settings using above info and
# http://www.dkfz-heidelberg.de/spec/linux/modeline/
# V-freq: 85.00 Hz  // h-freq: 68.79 KHz
Modeline "1024x768" 97.40  1024 1072 1192 1416   768  768  771  809 
# V-freq: 75.00 Hz  // h-freq: 80.42 KHz
Modeline "1280x1024" 151.83  1280 1360 1544 1888  1024 1024 1027 1072 
# V-freq: 75.00 Hz  // h-freq: 94.24 KHz
Modeline "1600x1200" 242.01  1600 1728 2024 2568  1200 1200 1204 1256 
\end{verbatim}
% end lanina.ess.uci.edu

sand.ess.uci.edu = 128.200.14.132 \\
Western Scientific \$4000\\
Chassis Serial number 1012003577\\
Arrived 20040221 \\
Service Tag fxm \\
Express Service Code fxm \\
UCI Property \#: 049003617 \\
2 GB RAM\\
Two AMD Opteron(tm) Processor 244s at 1.8GHz\\
Bought without monitor\\
Now using Dell UltraSharp monitor same as biogenic\\
Network card \flidx{sk98lin}
Logitech cordless mouse/keyboard: Navigator Duo
Graphics Card: ASUS V9520 Series CPU Graphics card powered by NVIDIA
GeForce FX 5200 GPU. Supports AGP 8x
Motherboard: IWILL DK8X:
Eight sockets for up to 16 GB RAM 
Use 184-pin Registered PC2700/2100 ECC DDR memory modules
Two 32-bit/33 MHz PCI slots
One 64 bit/66 MHz PCI slot
Two PCI-X slots
One AGP Pro 8X slot
hda: DVD: Plextor Model PX-708A Internal 40X ATAPI DVD+-R/RW CD-R/RW drive
hdc: DV-516D 0106 ATAPI 48X CD/DVD-ROM drive
Tech Support (800) 443-6699 
Hank Vu (800) 443-6699 x. 211
hank@wsm.com

2.28.05
mods 3.29.05 by hjm
200 gb disk from lanina moved to sand on IDE bus 1
new hoary ubuntu distro loaded on 200gb disk as::
    Filesystem           1K-blocks      Used Available Use% Mounted on
    /dev/hda1             10482044   2048572   8433472  20% /
    tmpfs                  1786416         0   1786416   0% /dev/shm
    /dev/hda6            105294788   5897208  99397580   6% /data
    /dev/hda3             73278252   2128912  71149340   3% /home
    /dev                  10482044   2048572   8433472  20% /.dev
    none                      5120      2836      2284  56% /dev

sand was upgraded to 4GB ram, of which 3.5 is available to the system - 
a little mentioned weirdness of Opteron systems is that the 0.5 GB of RAM
just below 4 GB disappears into the memory-mapping black hole that AMD 
inherited from INtel.

After the SW raid experiment, sand has 3ware Escalade HW raid card driving 
eight identical Western Digital 2500jd drives in a RAID~5 config
yielding ~1.7TB usable.   
It looks to the system like a single SCSI disk.

5.13.05 - disk failure on sand's RAID5 partition - detected on reboot in 
dmesg, not by log or email as expected.  At this point, the data was still 
intact and SHOULD have been backed up to another system, but since it was HW 
RAID5 AND it was 3ware controller (known for reliability (HA!) and robustness 
(HA!)) AND this was acting as the backup for other systems (which were still 
OK) AND the data was ~200GB at this point, I thought it was ok to go ahead. 
MISTAKE! 

First thing was to find the problem of why we hadn't been informed of the 
failure beforehand.  
The controller ( a 3ware Escalade 86506-8port driving 8x250GB identical WD 
disks) has, like most such cards, a BIOS-based utility for setting up the 
RAID which actually worked pretty well, except that unlike SW RAID, you can't 
use the raid immediately in degraded mode (while it's building the checksum 
info across the raid) - you have to let it sit there for hours (it's a 1.6 TB 
array) while it checksums the entire array (even tho there's nothing on it to 
begin with).  That done, it looks like a giant scsi disk to the OS - so far 
so great.

3ware also comes with a web-frontend utility called 3dm and a commandline 
utility called tw\_cli.  When I had installed the 3dm, I had gone thru the 
installation script, checked that there were no error messages, checked that 
I got an email verification and then forgot about it - altho thinking about 
it - I must have gotten the email from the script, not the app.  I did not 
check that the web server interface was working as I didn't think I'd ever 
use it.  MISTAKE.

Now I DID need to talk to the controller and the 3dm/tw\_cli were the only 
things that could while the OS was running.  THIS is one of the downsides of 
a hardware RAID - you're stuck with the tools that the vendor gives you.  
Since I was running on a 64-bit SMP Linux (Ubuntu), dual opteron, the 
installation bash script ran fine, but the monitoring daemon silently failed 
(32 bit code and I was running a 64-bit-only OS).  So nothing was hearing the 
controller screaming that a disk had died and the RAID was now running in 
degraded mode.  (as noted above, the only thing that let us know this was an 
entry in dmesg on a reboot.)

After verifying that this software was in fact incompatible with the OS, I 
tried to find an upgrade that WOULD let me talk to the controller.  I figured 
that 3ware being a vendor of high-end hardware, my kind of machine would be 
among their main targets.  And I was right - BUT ... Trying to find the 
software that was compatible with my system was an exercise in frustration - 
3ware's web site is walled off from google's bots (like almost all corporate 
sites) and since 3ware is relatively high end hardware, there are not a lot 
of messages on the linux BBSs about such failures and how to deal with them.  
So after a couple hours of browsing I had to go back to the 3ware site and 
deal with their oh-so cool web design that doesn't show URLs in any way 
different than regular text.  The text only shows up as a hyperlink if you 
mouse over it.  I noticed this and then had to mouse over entire pages of 
text, line by line to search for likely hyperlinks.  

The one that finally took me to the page I needed was buried in a paragraph 
that I almost overlooked.  Turns out the SW does exist, but is NOT specified 
for the controller I have (8506-8) but  the 9000-series controller (which is 
noted in the fine print as being backwards compatible with the 8000-series).  
ALso, it's not 'released SW' , it's being 'In Engineering Phase'.  To make a 
long story shorter, I ended up downloading and trying several versions of 
software until I finally stumbled over the right software - the 64bit 
versions of the 3dm2 and cli for the 9000-series controllers.  This installed 
OK and apparently ran.  The web interface software however, while it started 
up and presented an optimistic login screen, gave no indication of what the 
passwords should be or where to go to set them.  After looking in the config 
file (/etc/3dm2/3dm2.conf) only to find encrypted passwords, I then wandered 
around the 3ware web site trying to find documentation about how to set or 
even find out what the initial passwords were. There were no docs or help 
files or README's with teh software (it's '3ware' for those of you going thru 
the same hell; you can change them via the web interface when you finally get 
in.)

Re the passwords - nothing - or at least nothing I could find in about an 
hour's searching.  I finally decided to look in the installation script - 
bingo.  The passwords are set and encrypted into the config file from there.  
SO after setting them to what I wanted, FINALLY I was able to log into the 
web interface and talk to the controller. And in fact after being able to log 
in, the help file DOES tell you what what the password is and how to change 
it.

Actually the tw\_cli app also works, but it's pretty ugly (altho give them 
credit - they DID make 2 linux-specific clients).  The one that I needed was 
the 3ware 9000 series 3DM2 Linux64-bit one - helpfully, on the web page I 
eventually found: (http://3ware.com/support/downloadpageeng.asp?SNO=4), both 
the 32 bit and 64 bit one are named the SAME THING.

So here I am, talking to the 3ware controller via the web interface and while 
it's not fantastic, it's really not bad.  And one disk has been marked bad.  
So now I have to replace the bad disk.  I'm just about to bring the system 
down to do this, when I realize the disks are sitting in the expensive 
hotswap cages we bought for this specific purpose, so (after unmounting the 
filesystem) I take a deep breath, and pop the offending disk. ... ... nothing 
happens - the system doesn't freeze or explode or anything - it looks like it 
has actually worked - and the 3dm2 interface shows that the bad disk is now 
gone.  GREAT!  I quickly replace the disk with a spare and slam it back in 
again - and there it is on the web interface.  Now isolated all by itself. 

 Now - how to go about adding it back to the RAID?  The Web interface is a bit 
dodgy on how to go about adding this disk back into the array. And the help 
pages are not particularly helpful; the Maintenance help page sort of 
obliquely refers to this scenario, but certainly doesn't give any specific 
step-by step instructions.  You'd have thought that since one of the primary 
reasons for buying such an expensive controller is to be able to replace a RAID5 
disk on the fly, they might have a specific  mention of such an eventuality.   
The way I did it is to add the disk to a new 'UNIT' and then add that UNIT to 
the previously defined RAID5 UNIT and request that the new combined unit be 
rebuilt.  That seemed to work and the controller went about integrating the 
new disk into the raid 5 array.  Again, it was not possible to mount the 
array and use it while it was being integrated, like you can do with SW RAID 
under linux.  This took several hours, and in the end, it FAILED.  That was 
the just about the last straw.  After spending \$ and time (=\$) on 
this escapade (that's what the Escalade series SHOULD be named), the 
thing fails to rebuild the array.  (But at least it now reports via 
email that it has failed.) So now what???  

The filesystem was a reiserfs to begin with.  As a last resort, I try to 
rescue the thing with a fsck.reiserfs.  After reading the dire warning about 
this being the last thing you should try, I give it the --rebuild-tree option 
and go home.  This being 2TB of disk, it takes a while.  Later that night I 
see it's completed and try to mount it.  To my astonishment it mounts.  I do 
a 'df' - hmm - that's not good - only 3% of disk used.  There was 11% when I 
started (the raid had only been running a short while). I'm not at all happy 
to see that it the only directory on the partition is ... lost+found.  This 
dir contains the rubble of what used to be about 200GB of expensive and 
carefully groomed earth-sensing and atmospheric data. %$^%$^$^%$^!^&$!^%

So go ahead - ask me - Am I happy that I spent the extra $ to buy a hardware 
raid card rather than two $30 4port sata controllers and using SW RAID?  

I probably couldn't have done all this disk hot-swapping with a non-HW RAID 
card, but the cost of a reboot for most of us is not that big a deal.  That 
said, I'm not sure of the total complexity that doing such a thing under SW 
RAID would have entailed. To do this with 2x 4 port controllers would have 
required additional complexity and I'm not sure it can be done easily with 
mdadm.  And it is possible that I did something wrong in the 3ware rebuild - 
I'll be sending this narrative back to them as well.

As a postscript to this, I should also mention that while most Linux server 
vendors sell 3ware cards, at least one (Los Alamos Computers) suggests SW 
raid as being both significantly cheaper and faster.  They suggest the 
Promise SATA TX4 for about $70.  Newegg has the supported-in-kernel Silicon 
Image chipsetted Syba 4 port card for $30. If you remember my previous posts, 
I was surprised to find SW RAID to be a bit (10-20%) faster.  I now think 
that I probably should have tried the SW RAID on a full 2TB array.

Well, you makes your choices and you takes your chances.

Currently, the 3dm2 SW is running and should be restarted with a reboot 
(commands to re-start are in /etc/init.d/local.  Will test to be sure.
The 3dm2 web interface can be gotten at sand:888 and the password for 
user (read-only) is '3ware'.  
The admin password is about a coyote-resistant neighbor's cat.

6.7.05 - noticed that the 3dm2 daemon was reverting to original
settings on each reboot.  
It hadn't written a config dir (/etc/3dm2.3dm2.conf) and so was losing  
the config. 
It also wasn't complaining that it couldn't do it, to stdout/err, to dmesg 
nor to syslog.

rsync backups:  There is an rsync cron job in /etc/crond.daily/rsync.home2data
that will rsync the /home dir (where the web site is, the local dir tree, 
an all user dirs, as well as some additional stuff) to the /data/home dir.  
It's stored uncompressed, so we can squeeze a bit more data out of it if we 
want.  The rsync is initiallized and tested and will write an entry into the 
syslog.  excluded dirs/files can be entered in the file: 
/etc/rsync/sand.home2data.exclude, one per line.
6.7.05 - also rsyncing the web site to soot so it can be used as a backup server 
in case sand explodes again.  Note that it is being owned as'hjm' so it may have 
to be recursively chowned on soot, tho I think it will work fine

There is a /etc/init.d/local file that is responsible for starting a variety 
of local services - the license manager, the nco-bm benchmark server, makeing 
sure apache starts, mounting the /data dir, etc.  It should be updated on a 
change with:
\begin{verbatim}
$ update-rc.d local defaults 80

hjm++ 9.20.05
sand's syslog and dmesg ring was filling up with ethernet errors. While you can get a finer 
degree of control by editing /etc/syslog.conf, the easiest way to stop the vast majority 
of such errors like:
Sep 20 10:14:56 sand kernel: DROPPED IN=eth0 OUT= MAC=ff:ff:ff:ff:ff:ff:00:36:81:f5:3d:81:08:00 SRC=128.200.14.81 DST=128.200.14.255 LEN=96 TOS=0x00 PREC=0x00 TTL=64 ID=0 DF PROTO=UDP SPT=137 DPT=137 LEN=76
Sep 20 10:14:56 sand kernel: DROPPED IN=eth0 OUT= MAC=ff:ff:ff:ff:ff:ff:08:00:46:c3:03:83:08:00 SRC=128.200.14.130 DST=128.200.14.255 LEN=78 TOS=0x00 PREC=0x00 TTL=128 ID=828 PROTO=UDP SPT=137 DPT=137 LEN=58
Sep 20 10:14:57 sand kernel: DROPPED IN=eth0 OUT= MAC=ff:ff:ff:ff:ff:ff:00:36:81:f5:3d:81:08:00 SRC=128.200.14.81 DST=128.200.14.255 LEN=218 TOS=0x00 PREC=0x00 TTL=64 ID=0 DF PROTO=UDP SPT=138 DPT=138 LEN=198

is to stop the logging using guarddog (with the "Logging" panel, surprisingly enough)
That stops 99% of the lines.
hjm--

flexlm license manager running on sand
    If the license manager screws up (infrequent, but not unknown):
    test the status:
    % /usr/local/rsi/idl/bin/lmstat

    # if it needs to be restarted, bring it down smoothly with:
    % /usr/local/rsi/idl/bin/lmdown -q
    # (you can't just kill the flexlm daemon and restart it as it sets up lock
    # files all over the place)

    #After it comes down, restart it via:

    su -c "/usr/local/rsi/idl/bin/lmgd" hjm
    or
    su -c "/usr/local/rsi/idl/bin/lmgd" cluo

    # (the manager has to run as a regular user, not as root)
        
\end{verbatim}
installed gnome as well for Dan.  Just as well - kde has some issues with the 
kernel 2.6.11-1 SMP installed.
extras include:
libxml2-dev libxml2-doc libxml2-utils
bonnie++
xosview
tkdiff
kompare
kdesdk-doc-html
ssh apcupsd
guarddog
vncserver
synaptic
ddd ddd-doc pydb  glibc-doc gnuplot
libqt3-dev tree nedit joe 
gnome 

sand is now sitting behiind a pretty restrictive firewall so if things don't work, 
that may be the reason.  In fact, it was preventing the nco-bm server from getting 
data on udp port 29659 until guarddog was changed to address that issue.

also running ubuntu kernel 2.6.11-1 smp sort of successfully, but it kills the usb 
connectivity so apcupsd doesn't see teh ups.  And then it did...  ANd then it didn't.
Then it did.  Then it didn't.  Still to be resolved.

07-13-05 hjm - Now runnning 2.6.10-5-smp-k8 more or less successfully, een with the USB
also upgraded all the KDE stuff, so now it appears to be running more stably.
printing hint for remote use:
kcmshell printmgr  
will bring up the KDE print manager in full.

DODS server on sand.  The DODS server on sand is not a server at all, but 
just a series of cgi scripts that are active as soon as the apache server
comes up. DODS urls are constructed as:
\begin{verbatim}
http://sand.ess.uci.edu/cgi-bin/dods/nph-dods/dodsdata/in.nc.dds? 
       server name              ^^^  ^^^^^^^^ ^^^^^^^^ ^^^^^ ^^^
                                |    |         |       |     |
                                |    |         |       |     file name ext is used
name of the subdir* in the cgi -+    |         |       |     to define action from 
tree where the DODS cgi are          |         |       |     DODS
kept                                 |         |       |
                                     |         |       data file name      
                                     |         subdir* rooted from htmlroot where the 
                                     |         actual files live
                                     name of the cgi script that is called
\end{verbatim}
*this is the same across zender DODS servers
% end sand.ess.uci.edu

seasalt.ess.uci.edu = 128.200.14.39:\\
Dell Dimension T Minitower 800 MHz PIII\\
Arrived 20000615\\
System Service Tag DS8J10B \\
Express Service Code 300 055 479 95 \\
Intel Pentium III (Coppermine) 800 MHz\\
256 MB RAM\\
40 GB Ultra ATA 7200 RPM with ATA 66 controller\\
NIC: 3Com EtherLink 10/100 PCI PCI For Complete PC Management (3c905c-TX)\\
Microsoft PS/2 Mouse (Intellipoint)\\
Promise Technology Inc. Ultra66 IDE Controller\\
Intel 82371AB/EB PCI Bus Master IDE Controller\\
120 MB Super disk, aka LS120\\
Sony 8X/4X/32X CD-RW drive\\
\begin{verbatim}
# Graphics: NVIDIA TNT2 M64 4xAGP with 32MB RAM 
07-12-05 hjm
modified the xorg.conf to read:
Section "Device"
        Identifier      "NVIDIA Corporation NV5M64 [RIVA TNT2 Model 64/Model 64 Pro]"
        Driver          "nvidia"
        BusID           "PCI:1:0:0"
        Option "BackingStore" "True" <--- allows covered windows to refresh correctly
        Option "SaveUnders" "True"   <--- ditto
        Option "RenderAccel" "true"  <--- allows better HW acceleration.
 EndSection

# 19" Dell M990 purchased with computer
# See http://support.dell.com/docs/monitors/m990/En/Index.htm
# Video controller: NVIDIA TNT2 M64 4xAGP with 32MB RAM
# Uses XF86_SVGA driver
# Horizontal scan frequency 30--96 kHz (automatic)
# Vertical scan frequency 50--160 Hz (automatic)
# Optimal preset resolution 1024x768 at 75 Hz or 85 Hz 
# Highest preset resolution 1600x1200 at 75 Hz
# Highest addressable resolution 1600x1200 at 75 Hz
# Preset modes:
# Supports 1024x768 @ 68.7 kHz 85 Hz +/+
# Supports 1280x1024 @ 80.0 kHz 75 Hz +/+
# Supports 1280x1024 @ 91.1 kHz 85 Hz +/+
# Supports 1600x1200 @ 93.7 kHz 75 Hz +/+ 
\end{verbatim}
\begin{verbatim}
# Computed following modeline settings using above info and
# http://www.dkfz-heidelberg.de/spec/linux/modeline/
# V-freq: 85.00 Hz  // h-freq: 68.79 KHz
Modeline "1024x768" 97.40  1024 1072 1192 1416   768  768  771  809 +HSync +VSync
# V-freq: 80.00 Hz  // h-freq: 86.05 KHz
Modeline "1280x1024" 167.97  1280 1368 1576 1952  1024 1024 1027 1075 +HSync +VSync
# V-freq: 85.00 Hz  // h-freq: 91.72 KHz
Modeline "1280x1024" 185.64  1280 1376 1600 2024  1024 1024 1028 1079 +HSync +VSync
# V-freq: 75.00 Hz  // h-freq: 94.24 KHz
Modeline "1600x1200" 242.01  1600 1728 2024 2568  1200 1200 1204 1256 +HSync +VSync 
\end{verbatim}
Linux 2.2.x kernels do not recognize the ATA66 controller, but patches
and workarounds are available.
HHPT366 HOWTO at \url{http://www.csie.ntu.edu.tw/%7Eb6506063/hpt366/}
UDMA Mini-HOWTO \url{http://www.linuxdoc.org}
My solution was to plug the hard drive directly into the 
IDE controller on the motherboard and then install linux,
i.e., I bypassed the ATA66 controller. 
% end seasalt.ess.uci.edu

soot.ess.uci.edu = 128.200.14.98\\
Dell Precision Workstation 650n Minitower\\
Arrived Jan 23, 2003\\
Service Tag GC9L921\\
Express Service Code 35570219545\\
UCI Property \#: 039003412\\
Dual Xeon 2.8GHz\\
2 GB RAM\\
hjm: 4.11.05 - added "/etc/init.d/httpd restart" to /etc/rc.local to 
restart web server on reboot.
\begin{verbatim}
# hjm - new 200 GB disk added (4.14.05)-on same IDE bus as 1st; suboptimal, 
# but physically tricky to take over other IDE bus.  If very slow, will 
# try to move it.
# 6.1.05 - moved 2nd disk to other controller channel - much faster.
1024 $ df
Filesystem           1K-blocks      Used Available Use% Mounted on
/dev/mapper/VolGroup00-LogVol00
                     113177264  94615204  12812916  89% /
/dev/hda2               101105     30766     65118  33% /boot
none                   1037348         0   1037348   0% /dev/shm
/dev/hdc2            191440744  73245868 118194876  39% /data3

6.7.05 - installed Intel icc compiler/debugger on soot in /opt.
% end soot.ess.uci.

\subsection{SuSE Linux}\label{sxn:suse}
The Zender group server, \mchidx{sand.ess.uci.edu}, runs
\trmidx{SuSE Linux} Professional version~9.0.
SuSE offers phone support at 1-510-628-3385 Monday--Friday from
9:00~AM--3:00~PM PST.
SuSE email support is at \url{support@suse.com}.
Our registration login is csz and the password is pw\_lw.
Our SuSE software serial number is 48205940904172.
Yast Online Update location should be ftp://ftp.suse.com/pub/suse.
Configuring printers with SuSE is non-trivial, but good technical 
details are given at \mchidx{portal.suse.com}.
Search the support database for \trmidx{CUPS}.

\subsection{Browsers}\label{sxn:moz}
UCI libraries make many \trmidx{journal subscriptions} available
for free on the campus network or from remote locations provided that
browser \trmidx{proxy} settings are correctly configured.
\begin{verbatim}
http://www.lib.uci.edu/services/how/connect.html # Describes proxies
http://www.nacs.uci.edu/help/proxy/ # UCI proxy instructions
\end{verbatim}
Change \verb'Preferences->Advanced->Proxies->Automatic proxy'
configuration to
\begin{verbatim}
http://proxy.uci.edu/proxy_config.pac # Use for automatic proxy configuration 
\end{verbatim}

\subsubsection{Firefox}\label{sxn:moz}
Disable popup windows by adding the following line to \flidx{user.js}:
\begin{verbatim}
user_pref("dom.disable_open_during_load", true);
\end{verbatim}
Do not modify \flidx{prefs.js} directly since it is automatically
overwritten on shutdown and thus custom changes are not persistent.

\subsection{Create krein disk structure}\label{sxn:krein}
\begin{verbatim}
mkdir -p /data/zender/ZENDER
mkdir -p /dks9d2s0/BIAN/match
mkdir -p /dks9d3s0/ZENDER/match
mkdir -p /dks9d4s0/DSS
mkdir -p /dks9d5s0/ZENDER/csm
ln -s /dks9d2s0/BIAN/match /data/zender/BIAN/match 
ln -s /dks9d3s0/ZENDER/match /data/zender/ZENDER/match 
ln -s /dks9d4s0/DSS /data/zender/DSS
ln -s /dks9d5s0/ZENDER/csm /data/zender/ZENDER/csm 
\end{verbatim}

\subsection{NFS export}\label{sxn:nfs_xpt}
One may use the \trmidx{network file system} (\acr{NFS}) to
mount or export directories across the network.
To export local disks to a host, edit the \flidx{exports} file, e.g.,
\flprn{/etc/exports}.

If remote host needs root access (root on remote machine has root 
privilage on the disks), add the hostname behind the \verb'root='
chain.  Otherwise add the host behind the \verb'-access=' chain.  
If not sure which one you want then add the system to \verb'-access='.
Note that the host separator is \verb':'.

Details of managing setup are described in excellent NFS-HOWTO
\url{http://www.linux.org/docs/ldp/howto/NFS-HOWTO}.
It is easy to export the disks on an already running filesystem.
First make sure that the \trmidx{NFS daemon}, \trmidx{nfsd}, is running.
Then make sure the \trmidx{NFS lock daemon}, \cmdidx{nfslock}, is running.
\cmdidx{exportfs}:
\begin{verbatim}
sudo /etc/init.d/nfs restart
/usr/sbin/exportfs -rv
\end{verbatim}
By default, NFS exports filesystems with a feature called
\trmidx{root-squashing} enabled.
With root-squashing enabled, processes do not maintain \trmidx{root}
priveleges when writing to an NFS-exported partition on a different
machine. 
This is a useful security measure, but plays hell with poor man's
\cmdidx{cron} scripts that backup files across machines. 
To override root-squashing, use the \cmdidx{no\_root\_squash} option
in the \flidx{exports} file.

To export \acr{NFS} filesystems securely, update \flidx{hosts.allow} 
and \flidx{hosts.deny} as appropriate. 
In particular, \flprn{hosts.allow} must allow access to all machines
that request \trmidx{INET} services from a machine with
\acr{NFS}-mounted home directories.
To restrict INET services to specific hosts use, e.g.,
\begin{verbatim}
$ cat /etc/hosts.allow
portmap: 128.200.14.25
\end{verbatim}
To allow INET services to all hosts use, e.g.,
\begin{verbatim}
$ cat /etc/hosts.allow
portmap:ALL
\end{verbatim}
To restrict INET services to all hosts use, e.g.,
\begin{verbatim}
$ cat /etc/hosts.allow
portmap: 128.200.14.25
portmap: 128.200.14.62
portmap: 128.200.14.39
portmap: 128.200.86.234
\end{verbatim}
This may prevent errors revealed by \cmdidx{gconf-sanity-check-1}:
\begin{verbatim}
$ gconf-sanity-check-1
\end{verbatim}

\subsection{NFS mount these disks on local machine:}\label{sxn:nfs_mnt}
Here are the mountpoints to use on \flprn{lanina} when mounting the
central \verb'fileserver' partitions at NCAR:
\begin{verbatim}
sudo mkdir -p /ncar/fs/cgd/home0
sudo mkdir -p /ncar/fs/cgd/data0
sudo mkdir -p /ncar/fs/cgd/csm
\end{verbatim}
Once these mountpoints exist, the machine is on the NCAR network,
and the mount options are listed in the \flprn{/etc/fstab} file,
the partitions may be mounted with the following commands:
\begin{verbatim}
sudo mount fileserver-n8.ucar.edu:/fs/cgd/home0
sudo mount fileserver-n8.ucar.edu:/fs/cgd/data0
sudo mount fileserver-n8.ucar.edu:/fs/cgd/csm
\end{verbatim}
Note that the partitions should not be automatically mounted at
boot time since usually the laptop is not directly on the NCAR
network.
The \cmdidx{noauto} option to \cmdidx{mount} accomplishes this.
Thus the appropriate \flidx{fstab} entries are
\begin{verbatim}
fileserver-n8.ucar.edu:/fs/cgd/home0    /ncar/fs/cgd/home0   nfs     exec,dev,suid,rw,noauto 0 0
fileserver-n8.ucar.edu:/fs/cgd/data0    /ncar/fs/cgd/data0   nfs     exec,dev,suid,rw,noauto 0 0
fileserver-n8.ucar.edu:/fs/cgd/csm      /ncar/fs/cgd/csm     nfs     exec,dev,suid,rw,noauto 0 0
\end{verbatim}
When disconnecting the laptop from the network, it is best to 
\cmdidx{umount} these partitions so that the NFS-related daemons do
not waste time looking for them when the laptop is turned on again
somewhere off the network.
\begin{verbatim}
sudo umount fileserver-n8.ucar.edu:/fs/cgd/home0
sudo umount fileserver-n8.ucar.edu:/fs/cgd/data0
sudo umount fileserver-n8.ucar.edu:/fs/cgd/csm
\end{verbatim}

Here are the mountpoints on \verb'krein':
\begin{verbatim}
mount krein.math.uci.edu:/dks9d2s0 at mountpoint /dks9d2s0
mount krein.math.uci.edu:/dks9d3s0 at mountpoint /dks9d3s0
mount krein.math.uci.edu:/dks9d4s0 at mountpoint /dks9d4s0
mount krein.math.uci.edu:/dks9d5s0 at mountpoint /dks9d5s0
\end{verbatim}
Create mountpoints on local machine (e.g., \verb'dust.ess.uci.edu'):
\begin{verbatim}
mkdir /dks9d2s0
mkdir /dks9d3s0
mkdir /dks9d4s0
mkdir /dks9d5s0
\end{verbatim}
Add \cmdidx{mount} commands to \flidx{/etc/fstab}:
\begin{verbatim}
krein.math.uci.edu:/dks9d2s0    /dks9d2s0       nfs      exec,dev,suid,rw 1 1
krein.math.uci.edu:/dks9d3s0    /dks9d3s0       nfs      exec,dev,suid,rw 1 1
krein.math.uci.edu:/dks9d4s0    /dks9d4s0       nfs      exec,dev,suid,rw 1 1
krein.math.uci.edu:/dks9d5s0    /dks9d5s0       nfs      exec,dev,suid,rw 1 1
\end{verbatim}
Mount these partitions interactively the first time:
\begin{verbatim}
mount krein.math.uci.edu:/dks9d2s0 /dks9d2s0
mount krein.math.uci.edu:/dks9d3s0 /dks9d3s0
mount krein.math.uci.edu:/dks9d4s0 /dks9d4s0
mount krein.math.uci.edu:/dks9d5s0 /dks9d5s0
\end{verbatim}
Create softlinks on local machine (e.g., \verb'dust.ess.uci.edu') so
these directories may be accessed identically on \cmdidx{krein} and on
local machine 
\begin{enumerate*}
\item Disk \verb'dks9d2s0' is for match runs \flprn{/BIAN/match}, 
\item Disk \verb'dks9d3s0' is for match runs \flprn{/ZENDER/match}
\item Disk \verb'dks9d4s0' is for NCEP data \flprn{/DSS}
\item Disk \verb'dks9d5s0' is for boundary data \flprn{${DATA}/data},
diagnostics \flprn{${DATA}/dgn}, \flprn{${DATA}/aca},
\flprn{${DATA}/map}, and csm runs \flprn{/ZENDER/csm} and all run  
directories \flprn{/data/zender/match}, \flprn{/data/zender/csm}
\end{enumerate*}
\begin{verbatim}
mkdir -p /data/zender/ZENDER
mkdir -p /dks9d2s0/BIAN/match
mkdir -p /dks9d3s0/ZENDER/match
mkdir -p /dks9d4s0/DSS
mkdir -p /dks9d5s0/ZENDER/csm
mkdir -p /dks9d5s0/aca
mkdir -p /dks9d5s0/csm
mkdir -p /dks9d5s0/data
mkdir -p /dks9d5s0/dgn
mkdir -p /dks9d5s0/match
mkdir -p /dks9d5s0/tmp
ln -s /dks9d2s0/BIAN/match /data/zender/BIAN/match 
ln -s /dks9d3s0/ZENDER/match /data/zender/ZENDER/match 
ln -s /dks9d4s0/DSS /data/zender/DSS
ln -s /dks9d5s0/ZENDER/csm /data/zender/ZENDER/csm 
ln -s /dks9d5s0/aca /data/zender/aca
ln -s /dks9d5s0/csm /data/zender/csm
ln -s /dks9d5s0/data /data/zender/data
ln -s /dks9d5s0/dgn /data/zender/dgn
ln -s /dks9d5s0/match /data/zender/match
ln -s /dks9d5s0/tmp /data/zender/tmp
\end{verbatim}

\subsection{Install sudo}\label{sxn:sudo}
As root, install \cmdidx{sudo} 
\begin{verbatim}
scp dust.ess.uci.edu:/etc/sudoers /etc/sudoers
scp dust.ess.uci.edu:/usr/bin/sudo /usr/bin/sudo
\end{verbatim}
The \cmdprn{sudo} command has some subtle problems.
First, \cmdprn{sudo} does not alter \cmdidx{HOME}.
To execute commands with \cmdidx{HOME} changed to the target user,
i.e., root, invoke \cmdprn{sudo} with -H
\flidx{/root} but does not change 
\begin{verbatim}
sudo -H foo # Change ${HOME} to /root
\end{verbatim}
Second, \cmdprn{sudo} carries the user's environment with it.
Often it is desirable to see a command executed exactly as it would be
if it were being execture from a root login shell.
\begin{verbatim}
# Initialize $HOME, $SHELL, $USER, $LOGNAME, and $PATH
# Un-set all other variables
sudo -i foo # Run command as in initial login shell
\end{verbatim}
Some GNU/Linux distributions, such as \trmidx{Ubuntu}
\footnote{From the Ubuntu website:
``Ubuntu'' is an ancient African word, meaning ``humanity to others''.  
Ubuntu also means ``I am what I am because of who we all are''.
Ubuntu means ``towards humanity'' in Bemba.}, 
do not enable the root account by default.
Instead, Ubuntu expects the first user created to use \cmdprn{sudo}
to perform administrative tasks.
In such situations, it may be useful to create a shell with root
privileges
\begin{verbatim}
sudo -s # Create shell with root priveleges
\end{verbatim}
Another option is to enable the root accounta
\begin{verbatim}
sudo passwd root # Enable root account 
sudo passwd -l root # Lock root account after enabling it
\end{verbatim}

hjm - 08-18-05
I've had a continuing problem with sand periodically complaining about my .ICEauthority and .Xauthority files being screwed up and then refusing to let me log in on the console.  This may not affect you two too much as you probably never use GUI tools to admin the system, but I do and this is the reason.

Since sand has no root user, it manipulates the user's .Xauthority to enable X to open displays.  That changes the permission of the file and thus prohibits a console login because xdm needs to write to that file.

So if you use an X app as root, you will need to explicitly change the permissions on these files before you will be able to log in.

I've added the following into my .alias file which fixes (but does not solve) the problem:

alias chice="sudo chown hjm.cgdcsm /home/hjm/.ICEauthority /home/hjm/.Xauthority"


\subsection{Install IDL and PGI}\label{sxn:idl}
\trmidx{Intel} recently purchased \trmidx{Kai} software and is now
distributing their compilers for Linux.
\begin{verbatim}
http://www.intel.com/software/products/compilers/flin/noncom.htm
http://www.coyotegulch.com/hpc/intel_1st_look.html
\end{verbatim}
Select the ``non-commercial unsupported software'' for Linux, not the 
free evaluation software.
The Intel compilers are \trmidx{OpenMP}-compliant.
The Fortran and \cxx\ compiler work fine on \trmidx{RedHat}~7.2. 
Only the OpenMP features of the \cxx\ compiler seems not to work.
Serial numbers of \cmdidx{icc} and \cmdidx{ifc} are
\verb'1110-19809410' and \verb'1130-70846464', respectively.
Intel compiler support questions go to \url{pto.support@intel.com}.

We bought \trmidx{Lahey} 6.1 Pro fortran compiler serial number LP072528.
Technical support is \url{support@lahey.com}.

My PGI PINs are 109584 (dust), 502421 (seasalt), and 508553 (sand).
Download releases from PGI website at
\url{http://www.pgroup.com/downloads}.
Generate permanent licenses keys using personalized account on
the PGI website at \url{http://www.pgroup.com/pgilogin.htm}
Username \verb'pn109584' and Password \verb'Mall!21' (dust).
Username \verb'pn502421' and Password \verb'Teen:28' (seasalt),
Username \verb'pn508553' and Password \verb'Jane&40' (sand).
Updating PGI license keys. 
To learn the FLEXlm-style hostid of the system, execute
\begin{verbatim}
${PGI}/linux86/bin/lmutil lmhostid
${PGI}/linux86-64/6.0/bin/lmutil lmhostid
\end{verbatim}
For lanina, this results in \verb'0010a4081231'
For seasalt, this results in \verb'0001031c5c7f'
For seasalt, this results in \verb'00d0680399a4'
PGI-style hostids do not require a license daemon, but are locked to
the username that installed the compilers, and to the hardware
configuration in existence when the compilers were installed. 
The PGI-style hostid can be found by running the command 
\begin{verbatim}
${PGI}/linux86/bin/pghostid
unset LM_LICENSE_FILE
\end{verbatim}
The host~ID must match that in the license file.

To copy files from dust to other machines, e.g., \mchprn{lanina}:
\begin{verbatim}
tar cvzf /data/zender/tmp/pgi.tar.gz ./pgi
tar cvzf /data/zender/tmp/rsi.tar.gz ./rsi
sudo scp dust.ess.uci.edu:/data/zender/tmp/pgi.tar.gz /usr/local
sudo scp dust.ess.uci.edu:/data/zender/tmp/rsi.tar.gz /usr/local
cd /usr/local
sudo tar xvzf pgi.tar.gz
sudo tar xvzf rsi.tar.gz
sudo ln -s ./rsi/idl_5.6 ./rsi/idl
/bin/rm ./rsi/idl_linux.tar.gz
\end{verbatim}

Add the following to \flidx{.bashrc}:
\begin{verbatim}
# Licenses for IDL and PGI are handled by FlexLM
# FlexLM is started in /etc/rc.d/rc.local
# Running lmgrd as root is not recommended but is expedient
# IDL lmgrd starts vendor-specific license daemons for both IDL and PGI
# This is because contents of PGI license file are in IDL license file
# Users must have LM_LICENSE_FILE environment variable point to both license files
case ${HOST} in
    dust* )
        export LM_LICENSE_FILE="${IDL_DIR}/../license/license.dat:${PGI}/license.dat"
    ;; # endif using licenses directly on license server
    dakine* | lanina* | seasalt* | dhcp* | ras* | cgd85* )
# Randerson's server
        export LM_LICENSE_FILE="${PGI}/license.dat:1700@ess1.ess.uci.edu"
# Zender's server
        export LM_LICENSE_FILE="${PGI}/license.dat:1700@dust.ess.uci.edu"
    ;; # endif using network licenses
esac # endcase ${HOST}
\end{verbatim}
% $: rebalance syntax highlighting
PGI Fortran may not run on \flprn{lanina} unless \verb'LM_LICENSE_FILE'
is undefined with \verb'unset LM_LICENSE_FILE'.
This is because any licenses specified in \verb'LM_LICENSE_FILE'
appear to override searching for PGI-style node-locked licenses first.
Since \verb'LM_LICENSE_FILE' must be defined to allow network access
to the IDL software, this means it will be difficult if not impossible
to get PGI and IDL both working on \flprn{lanina} at the same time.

To remove a checked out license use the \cmdidx{lmremove} option to
the \cmdidx{lmutil} command.
\begin{verbatim}
lmutil lmremove [-c license_file_list] feature user user_host display
lmutil lmremove idl zender lanina.ess.uci.edu /dev/pts/0
\end{verbatim}
The argument to \cmdprn{lmremove} were obtained from the
\cmdidx{lmstat -a} command, which is now obsolete.
Instead give sub-commands arguments directly to \cmdidx{lmutil}, e.g.,
\begin{verbatim}
lmutil lmstat
\end{verbatim}

\subsubsection{PGI on sand}\label{sxn:pgi_sand}
So that any system user may run the software, we create a
\trmidx{FlexLM} license daemon \flidx{/etc/init.d/pgi-lmgrd}.
A corresponding user, \cmdidx{flexlm}, runs this daemon. 
\begin{verbatim}
/usr/local/pgi-5.1-3 # Western Scientific installation
/usr/local/src/ # Source tarball
/etc/init.d/pgi-lmgrd # daemon to start FlexLM
/usr/local/pgi-5.1-3/license.dat # License
/var/tmp/flexlm.log # License daemon error messages 
\end{verbatim}
The server's \trmidx{hostname} configuration is very important.
When the hostname changes then update the license file accordingly. 

\subsection{PCMCIA}\label{sxn:pcmcia}
\begin{verbatim}
cd /usr/src
tar xvzf pcmcia-cs-3_x_x.tar.gz 
cd /usr/src/linux/pcmcia-cs-3.x.x
make config
make all 
make install
\end{verbatim}
Customize in \flidx{/etc/pcmcia} if neccessary.
For \trmidx{RedHat} systems, read Section 2.5.2 (p. 12) of PCMCIA HOWTO\\
As recommended, delete (or comment out) contents of default\\
\flidx{/etc/pcmcia/network.opts}\\
and replace with script given in HOWTO\\
This fix enables ethernet connections on power-up\\
Alternatively, I modified \flidx{/etc/pcmcia/network} with two
suggestions from PCMCIA Xircom Hypermail list, but, as stated in the
HOWTO,\\ 
RedHat may not actually run this script.\\
JWZ has similiar card and uses three commands \\
\begin{verbatim}
usernetctl eth0 down
/etc/rc.d/init.d/pcmcia restart
usernetctl eth0 up
\end{verbatim}

\subsection{Building new kernel}\label{sxn:krn}
Before building kernel make sure default compiler is known to build
stable kernels.
For \trmidx{RedHat}~7.0 systems use \cmdidx{kgcc} not \cmdidx{gcc}.
New kernel (2.2.18+) Makefiles do this automatically, but just in case
do this as root \verb'export CC=/usr/bin/kgcc'. 
For guaranteed kernel stability, \trmidx{GCC} version 2.95 is recommended.
\verb'export CC=/usr/bin/gcc-2.95'. 
Finally, the \verb'System.map' file must be copied along with the
kernel.  
\begin{verbatim}
cd /usr/src/linux
# Choose _one_ of the following:
cp cnf_fl /usr/src/linux/.config # Use archived config file
make clean (remove old object files)
make oldconfig # Use .config file, or 
make config # Create/modify .config file (text), or
make menuconfig # Create/modify .config file (curses), or
make xconfig # Create/modify .config file (Xwindows)
make mrproper # Remove .config (CAREFUL!!!), clean up disk
# Enable: APM,ACPI,ISA,LVM,UDF,USB
# DHCP requires CONFIG_PACKET and CONFIG_FILTER
# USB requires fxm
# DVD and CD-ROM requires UDF filesystem
# Following steps are mandatory
make dep # Dependency check
make # Create bzImage (NB: bzImage, not zImage)
make modules
make modules_install
make install # Copy bzImage to boot. Rename it intelligently.
# This automagically does the following three steps
cp /usr/src/kernel-source-2.4.23/arch/i386/boot/bzImage /boot/vmlinuz-2.4.23
cp /usr/src/linux/arch/i386/boot/bzImage /boot/vmlinuz-2.4.23
cp /usr/src/linux/System.map /boot/System.map-2.4.23
make clean (remove old object files)
rm /boot/vmlinuz
rm /boot/System.map
ln -s /boot/vmlinuz-2.4.23 /boot/vmlinuz
ln -s /boot/System.map-2.4.23 /boot/System.map
/sbin/lilo
shutdown -r now
# Exit X the rebuild NVidia driver
cd /usr/src;sh NVIDIA-Linux-x86-1.0-4496-pkg2.run
# Get 2.6 kernel patches for NVidia drivers
http://www.minion.de/files
\end{verbatim}
I like the \verb'make xconfig' method.
It is very clean and allows storing and retrieving configuration files.
By default, the configuration file is saved as \flidx{.config} in the
top-level make directory.
Save a visible version of the configuration file as, e.g.,
\verb'/home/zender/linux/usr/src/linux/config.lanina.2.4.23.20010322'
or \verb'config.lanina.2.4.23.20010322' and then
\verb'cp /usr/src/linux/config.lanina.2.4.23.20010322 /home/zender/linux/usr/src/linux'
Specific configuration options which are not the default must be set.
\begin{enumerate}
\item Block Devices: \verb'CONFIG_BLK_DEV_IDEFLOPPY' for LS-120/Iomega Zip support.
\item Sound: \verb'CONFIG_SOUND_MAESTRO' for Lanina soundcard
\item Networking Options: \verb'' for IEEE 1394/Firewire
\item IrDA (infrared) support: \verb'CONFIG_IRDA' 
\item USB support: Various. Important for futurre goodies like mouse, joystick...
\item Character devices: \verb'CONFIG_PRINTER'
\item Filesystems: \verb'CONFIG_FAT_FS', \verb'CONFIG_VFAT_FS' for MS
Windows filesystem support
\end{enumerate}

The \cmdidx{sysctl} command allows one to print (and set) kernel
parameters at runtime.
The \trmidx{Procfs} file system which displays \flidx{/proc/sys} is
required for \cmdidx{sysctl} to work.
\begin{verbatim}
sysctl -a # Print all kernel configuration values in /proc/sys
\end{verbatim}

\subsection{LILO configuration}\label{sxn:lilo}
The \trmidx{LILO} (linux loader) system is a venerable method of
loading Linux which has lately been superceded by \trmidx{GRUB}.
One problem with the LILO method is that the \cmdidx{lilo} command
must be run \textit{after} installing a kernel and \textit{prior to}
rebooting the machine. 
If it is necessary to rescue a machine that boots with LILO, one must
construct rather complex \cmdprn{lilo} commands to synchronize the
system. 
\begin{verbatim}
# Find major, minor device numbers on other RedHat machine
ls -l /dev/sda1
ls -l /dev/sda3
# Create devices with correct numbers
mknod /dev/sda1 b 8 1
mknod /dev/sda3 b 8 3
# Create mountpoint directory
mkdir /mnt/root
# Mount disks with root and kernel images
mount /dev/sda3 /mnt/root
mount /dev/sda1 /mnt/root/boot
# Change root directory to simplify paths for lilo
chroot /mnt/root # Change root from RedHat rescue to disk
chroot /mnt/sdb3 /sh # Change root from Knoppix CDROM to disk
cd /;ls # Verify we've changed root
/sbin/lilo # Simplest form
# Unless chroot succeeded, lilo may need following arguments
# -b: Location of boot device
/sbin/lilo -b /mnt/sdb1 # 
# -C: Location of configuration file (default /etc/lilo.conf)
/sbin/lilo -C /mnt/sdb3/etc/lilo.conf
# -i: Location of bootloader (default /boot/boot.b)
/sbin/lilo -i /mnt/sdb1/boot.b 
\end{verbatim}
When updating the kernel, make the old, working image available as a
backup.
\begin{verbatim}
image=/boot/vmlinuz-2.4.23
        label=linux
image=/boot/vmlinuz-2.2.16-22
        label=linux-2.2.16-22
\end{verbatim}
\flidx{lilo.conf} must specify the \trmidx{linear} keyword
to boot off a \trmidx{SCSI} disk (e.g., \mchidx{dirt.ess.uci.edu}).

\subsection{Restarting daemons}\label{sxn:dmn2}
Send HUP (hangup) signal to process:
\begin{verbatim}
kill -HUP pid
\end{verbatim}

On newer systems that use \trmidx{systemd}, restart services
\begin{verbatim}
sudo systemctl restart NetworkManager.service
sudo systemctl restart sshd.service
\end{verbatim}

On older systems, restart \trmidx{daemon} manually using :
\begin{verbatim}
/etc/rc.d/init.d/inet restart
/etc/rc.d/init.d/network restart
\end{verbatim}
Use alias:
\begin{verbatim}
restart inet
\end{verbatim}

\subsection{Monitor Kernel Startup}\label{sxn:krn2}
Trace with the \trmidx{kernel} startup with \cmdidx{dmesg}.
This command prints \flidx{/var/log/dmesg}.
\begin{verbatim}
dmesg > foo
\end{verbatim}

\subsection{Strip downloaded text files of DOS linefeeds}\label{sxn:dos}
There are three different systems used for representing the end of 
the line in text files. 
MS \trmidx{Windows} uses ASCII \cmdidx{CR/LF}, \cmdidx{Macintosh} uses 
\cmdidx{CR}, and \trmidx{Unix} uses \cmdidx{LF} as end-of-line
characters.  
To convert from Windows to Unix, use the \cmdidx{tr} command to strip
the file of the excess carriage-returns:
\begin{verbatim}
# Strip excess carriage returns from Windows file
tr -d '\015' < fl_in > fl_out
\end{verbatim}
To convert from Unix to Windows, insert carriage-returns in front of
line-feeds:
\begin{verbatim}
tr '\012' '\012\015' < fl_in > fl_out
tr '\012' '\012\015' < ~/job/cv.txt > ~/cv.txt
\end{verbatim}
\trmidx{FTP} servers attempt to handle these translations
automatically when text mode is selected.

The end-of-line convention is (part of what is) known in
\cmdidx{Emacs} as the \trmidx{encoding}.
Files written with the DOS-encoding convention may be reset to the
Unix convention within Emacs using
\begin{verbatim}
C-x RET f undecided-unix
\end{verbatim}

\subsection{Installing Debian}\label{sxn:dbn_nst}
Useful mailing list \url{debian-user@lists.debian.org}
\url{http://www.debian.org/MailingLists}

A few packages cause many warnings when upgrading Debian. 
These include \trmidx{kbuildsycoca} and \trmidx{kio}.
What causes these warnings?
\begin{verbatim}
kio (KService*): WARNING: The desktop entry file Utilities/kfilereplacepart.desktop has Type=Service but is located under "apps" instead of "services"
kbuildsycoca: WARNING: 'kfile_koffice.desktop' specifies undefined mimetype/servicetype 'application/vnd.kde.kword'
\end{verbatim}

\begin{verbatim}
cd
git clone git@github.com:czender/acd.git
git clone git@github.com:czender/prv.git
git clone git@github.com:czender/pnp.git
git clone git@github.com:czender/rsr.git
git clone git@github.com:czender/sw.git
git clone git@github.com:nco/nco.git
git clone git@github.com:jaws/jaws.git
git clone git@github.com:diwg/diwg.git
git clone git@github.com:diwg/cf2.git
git clone git@github.com:minxu74/conv_remap2
git clone git@github.com:ACME-Climate/PreAndPostProcessingScripts
git clone git@github.com:Unidata/EC-netCDF-CF
mkdir -p ~/terraref
cd ~/terraref
git clone git@github.com:terraref/extractors-environmental.git
git clone git@github.com:terraref/extractors-hyperspectral.git
git clone git@github.com:terraref/documentation.git
cd
ln -s sw/mk mk
ln -s sw/sh sh
cd ~/sh
ln -s ~/terraref/extractors-hyperspectral/hyperspectral/hyperspectral_workflow.sh hyperspectral_workflow.sh
cd
# sudo codesign -s 'Charlie Zender' `which gdb`
\end{verbatim}

\begin{verbatim}
# Begin Mac OS X installation procedure on aerosol.ess.uci.edu 20140529
# Install Xcode
# Install MacPorts
# NB: -t activates "trace mode" to conceal /usr/local from ports build system
# This increases likelihood package will build without interference from other compilers, etc.

sudo port install cvs
sudo port install subversion
sudo port install coreutils # ls=gls, cp=gcp
sudo port install gtime # time=gtime

# Software development
sudo port install emacs
sudo port install emacs-app
sudo port install auctex
sudo port install aspell
sudo port install aspell-dict-en
# Document-handling
sudo port install pdftk # Proposal generation
# LaTeX packages
sudo port install texlive # 
sudo port install texlive-latex-extra # 
sudo port install texlive-fonts-extra # ifsym, elvish, yfonts
sudo port install pdftk

# port list texlive-* and not installed
sudo port install auctex
# NCO
sudo port install antlr
sudo port install libdap
sudo port install gsl
sudo port install netcdf
sudo port install rx
sudo port install udunits2
#sudo port install nco

# ncdismember
sudo port install py27-numpy
sudo port install py27-netcdf4
sudo port install py27-lxml

# Utilities
sudo port install netcdf-cxx # needed by TempestRemap
sudo port install netcdf-fortran
sudo port install ncview
sudo port install ncarg
sudo port install p7zip # 7z compression decompression

sudo port install gmake
sudo port install -t ddd # Installation sudo

# NB: Activating real gcc, g++, and gfortran on MacPorts requires something like:
# https://stackoverflow.com/questions/8361002/how-to-use-the-gcc-installed-in-macports
# port select --list gcc
# sudo port select --set gcc mp-gcc7

sudo port selfupdate
sudo port upgrade outdated
sudo port uninstall nco
sudo port upgrade nco

cd ~/anaconda/bin
conda update libnetcdf
conda update nco
# 20180418 ocefpaf suggests this, which activates an "environment":
conda create --name NCO --channel conda-forge nco
conda activate NCO  # if your conda is old you may need to source activate NCO instead.

# Webserver
# Ensure port 80 is open
sudo apachectl start # Start web server
sudo apachectl stop # Stop web server
sudo apachectl restart # Restart web server after configuration changes
\end{verbatim}
\begin{verbatim}
# Begin Fedora installation procedure on roulee.ess.uci.edu 20130607
# Language support
#sudo dnf groupinstall french-support -y
sudo dnf langinstall fr_FR -y
sudo dnf langinstall es_ES -y
# Monitor stuff for NVidia cards
sudo dnf install --nogpgcheck http://download1.rpmfusion.org/free/fedora/rpmfusion-free-release-$(rpm -E %fedora).noarch.rpm
sudo dnf install --nogpgcheck http://download1.rpmfusion.org/nonfree/fedora/rpmfusion-nonfree-release-$(rpm -E %fedora).noarch.rpm
lspci |grep -i VGA
sudo dnf install kmod-nvidia -y
sudo dnf install nvidia-settings -y
# Proprietary filesystems
sudo dnf install exfat-utils fuse-exfat # Large USB drives often in exFAT
# Bluetooth
sudo dnf install bluez-hid2hci
# Generic OS-stuff
sudo dnf install tcsh # Needed for csh (NCL too)
# Software development
sudo dnf install emacs -y
sudo dnf install emacs-gettext -y # PO-mode
sudo dnf install aspell -y
sudo dnf install aspell-en -y
sudo dnf install man-pages -y
sudo dnf install kernel-devel -y
sudo ln -s /usr/src/kernels/3.9.6-200.fc18.x86_64 /usr/src/linux # Update <version>.<release>-<arch> as necessary
sudo dnf install ddd -y
# Install debugging information used by DDD?
# sudo debuginfo-install nco # Do not do this without further testing
# sudo dnf remove *-debuginfo # This removes all -debuginfo RPMs
# Version control
sudo dnf install svn -y
sudo dnf install cvs -y
# Chromium
wget http://repos.fedorapeople.org/repos/spot/chromium-stable/fedora-chromium-stable.repo .
sudo mv fedora-chromium-stable.repo /etc/dnf.repos.d/
sudo dnf install chromium -y
sudo dnf install http://linuxdownload.adobe.com/adobe-release/adobe-release-x86_64-1.0-1.noarch.rpm -y
sudo rpm --import /etc/pki/rpm-gpg/RPM-GPG-KEY-adobe-linux
sudo dnf install flash-plugin -y
sudo ln -s /usr/lib64/mozilla/plugins/libflashplayer.so /usr/lib64/chromium-browser/plugins/libflashplayer.so
sudo dnf install icedtea-web # Java plug-in
# Swap ctrl and capslock using tweak-tool:. Also set focus follows mouse (FFM).
sudo dnf install gnome-tweak-tool
gnome-tweak-tool # Swap ctrl<->capslock
gnome-tweak-tool # Set focus follows mouse (FFM)
gnome-tweak-tool # Decorate titlebar with all buttons
# auto-raise and raise delay must be set with commandline:
gconftool-2 --type boolean --set /apps/metacity/general/auto_raise true
gconftool-2 --type string --set /apps/metacity/general/auto_raise_delay 600
# Remember sessions
gnome_session-properties # Choose Options tab, toggle switch
# Always show logout button
sudo dnf install dconf-editor
dconf-editor # Org->gnome->metacity
dconf-editor # Org->gnome->gnome-session
dconf-editor # Org->gnome->desktop->applications->interface
# KDE
sudo dnf install @kde-desktop
sudo dnf install okular -y 
sudo dnf install okular -y
sudo dnf install ksnapshot -y
# Document-handling
sudo dnf install pdftk # Proposal generation
# LaTeX packages
sudo dnf install texlive-algorithmic -y # algorithmic environment
sudo dnf install texlive-datetime -y # \xxivtime, \ordinal
sudo dnf install texlive-was -y # upgreek Upright Greek letters [Euler,Symbol,Symbolsmallscale]
sudo dnf install texlive-draftcopy -y # Blaze ``Draft'' across [none,first,all,etc.] pages
sudo dnf install texlive-lettrine -y # Drop first letter of paragraph, \lettrine
sudo dnf install texlive-epigraph -y # Pithy quotes at section starts, \epigraph
sudo dnf install texlive-lineno -y # Line numbering \linenumbers, \nolinenumbers, \pagewiselinenumbers
sudo dnf install texlive-srcltx -y # Enable dvi <-> tex searches fxm: not working
sudo dnf install texlive-subfigure -y # \subfigure for panel labeling within figure environment
sudo dnf install texlive-tocbibind -y # Add Bibliography and Index to Table of Contents
sudo dnf install texlive-ulem -y # Strikeout text
sudo dnf install texlive-ifsym -y # Various symbols: Weather
sudo dnf install texlive-doublestroke -y # Doublestroke \mathds font
sudo dnf install texlive-yfonts -y # Yannis Haralambous Initials (yinit) fonts?
sudo dnf install texlive-gothic -y # Contains Yannis Haralambous Initials (yinit) fonts
sudo dnf install texlive-endfloat -y # Floats at end for manuscript submittal
sudo dnf install texlive-floatflt -y # Wrap text around floats
sudo dnf install texlive-epstopdf -y # Convert .eps to .pdf
sudo dnf install texlive-latex-recommended -y # PDFLaTeX ...
sudo dnf install texlive-luatex -y # LuaTeX ...
sudo dnf install texlive-etaremune -y # Reverse enumerate, supercedes revnum
sudo dnf install texlive-comment -y # commend environment
sudo dnf install emacs-auctex -y # Preview, AUCTeX
# Package to build RPMs
sudo dnf install @development-tools -y #
sudo dnf install fedora-packager -y #
sudo dnf install rpmlint -y # 
sudo dnf install automake autoconf libtool -y #
# NCO
sudo dnf install antlr antlr-C++ -y # ANTLR
sudo dnf install curl-devel libxml2-devel -y # DAP-prereqs
sudo dnf install libdap libdap-devel -y # DAP
sudo dnf install bison flex gcc gcc-c++ -y # GNU toolchain
sudo dnf install gsl gsl-devel -y # GSL
sudo dnf install netcdf netcdf-devel netcdf-fortran -y # netCDF
sudo dnf install librx librx-devel -y # RX
sudo dnf install texinfo-tex # TeXInfo, makeinfo
sudo dnf install udunits2 udunits2-devel -y # UDUnits
sudy dnf install protobuf -y # ncStream compiler for C++/Java/Python bindings
sudy dnf install protobuf-c -y # ncStream compiler for C-bindings
sudy dnf install protobuf-compiler -y # ncStream compiler
# Science
sudo dnf install hdf -y # HDF4
sudo dnf install hdf-devel -y # HDF4 headers needed by, e.g., netCDF --enable-hdf4
sudo dnf install hdfview -y # HDF5 viewer
sudo dnf install ncl -y # ncl, ncl_convert2nc
sudo dnf install ncl-examples -y # ng4ex
sudo dnf install ncview
sudo dnf install numpy -y # CFchecker
sudo dnf install netcdf4-python -y # CFchecker
sudo dnf install cfunits -y # CFchecker hand-install
sudo dnf install python-lxml -y # CFchecker
sudo dnf install valgrind -y # Development
sudo dnf install netcdf-cxx netcdf-cxx-devel -y # Tempest
# MPI
sudo dnf install openmpi -y # mpirun
sudo dnf install openmpi-devel -y # mpicc
# General OS stuff
sudo dnf install libreoffice
sudo dnf install lm_sensors
sudo dnf install qtwebkit lsb # Prereqs for google-musicmanager
# sudo rpm -i google-musicmanager-beta_current_x86_64.rpm # Obtain from Google Play
# Support for non-Free A/V codecs provided by rpmfusion.org
su -c 'rpm -Uvh http://download1.rpmfusion.org/free/fedora/rpmfusion-free-release-stable.noarch.rpm'
su -c 'rpm -Uvh http://download1.rpmfusion.org/nonfree/fedora/rpmfusion-nonfree-release-stable.noarch.rpm'
sudo dnf install gstreamer-plugins-bad gstreamer-plugins-ugly gstreamer-ffmpeg
sudo dnf install vlc
sudo dnf install gstreamer1-libav
# Netflix installation relies on the PostInstallerF tool which installs most commonly requested non-free packages
sudo dnf install http://sourceforge.net/projects/postinstaller/files/fedora/releases/19/x86_64/updates/postinstallerf-1.2.1-1.fc19.noarch.rpm 
sudo dnf -y install http://sourceforge.net/projects/postinstaller/files/fedora/releases/19/x86_64/updates/wine-silverligh-1.6-3.fc19.x86_64.rpm
sudo dnf -y install http://sourceforge.net/projects/postinstaller/files/fedora/releases/19/x86_64/updates/netflix-desktop-0.7.0-6.fc19.noarch.rpm
# Acroread installation (necessary for newer PDF forms, installs 75 additional FC19 packages for FC20 installation)
wget http://ardownload.adobe.com/pub/adobe/reader/unix/9.x/9.5.5/enu/AdbeRdr9.5.5-1_i486linux_enu.rpm

# Following task bar features activated at http://extensions.gnome.org
# Suspend/Resume button
# Weather
# Icons on task bar
# System Monitor on task bar
# Sensors
# Window options
# Reset hostname in Fedora:
hostnamectl set-hostname --static frazil.ess.uci.edu
hostnamectl set-hostname --transient frazil.ess.uci.edu
hostnamectl set-hostname --pretty frazil.ess.uci.edu
\end{verbatim}
\begin{verbatim}
# Following procedure was tested on ashes.ess.uci.edu 20030508, neige 20080103, givre 20090302, roulee 20110331
# Bring linux.tex over first to revise procedure as needed
scp dust.ess.uci.edu:/home/zender/linux/linux.tex ~
# Upgrade base installation
aptitude update; aptitude dist-upgrade 
# Install packages which make installations easier
sudo aptitude install apt-file cvs sudo subversion openssh-server
sudo aptitude install kubuntu-desktop
scp dust.ess.uci.edu:/etc/sudoers /etc/sudoers
sudo groupadd -g 2400 cgdcsm # Create group cgdcsm
sudo useradd -D -g cgdcsm # Make group cgdcsm default
sudo usermod -g cgdcsm -u 3555 zender # Change UID and default GID of user
sudo usermod -a -G www-data zender # Add user to supplementary group
sudo usermod -s /bin/bash zender # Change user's shell to bash
# If NFS-mounting home directory from dust, do this:
# sudo usermod -d /dhome/zender zender 
# Log out then log back in so UID, GID is consistent
chgrp -R cgdcsm /home/zender
export CVSROOT=':ext:zender@dust.ess.uci.edu:/home/zender/cvs'
export SVNROOT='svn+ssh://dust.ess.uci.edu/data/home/zender/svn/trunk'
export CVSUMASK=002
export CVS_RSH=ssh
sudo slocate -u
svn checkout ${SVNROOT}/dot ${SVNROOT}/elisp ${SVNROOT}/linux ${SVNROOT}/sh # Retrieve dot files, emacs, Linux Cheatsheet
# cvs co -kk dot elisp linux sh # Retrieve dot files, emacs, Linux Cheatsheet
# Use KDE/GNOME control center to left-hand mouse, auto-raise 
# xmodmap ~/sw/dot/xmodmaprc # Swap ctl and caps-lock # deprecated
export PVM_ARCH=`~/sh/pvmgetarch`
cd ~;mkdir -p bin include lib obj share/locale/es/LC_MESSAGES share/locale/fr/LC_MESSAGES
# Ubuntu: First activate 'universe' sources in sources.list
sudo aptitude install firefox # Install browser
sudo aptitude install auctex preview-latex # Install Emacs add-ons
sudo ln -s /usr/bin/mimencode /usr/local/bin/mmencode # Rmail uses mmencode
emacs & # Edit dot files to contain new system name
cd dot;make dot;. bashrc;cd # Install dot files, load custom bash settings
ssh-keygen -t rsa # Generate RSA key
ssh-keygen -t dsa # Generate DSA key
ssh-vulnkey -a # Check keys for vulnerability
cd ~/dot;cp authorized_keys known_hosts ~/.ssh # Setup SSH
chmod 700 ~/.ssh && chmod 600 ~/.ssh/authorized_keys
# Dell-specific packages http://linux.dell.com/wiki/index.php/Repository/firmware
sudo wget -q -O - http://linux.dell.com/repo/firmware/bootstrap.cgi | bash
sudo aptitude install firmware-addon-dell
sudo aptitude install $(bootstrap_firmware -a)
# Edit known_hosts and authorized_keys
# Insert ~/.ssh/id_dsa.pub, ~/.ssh/id_rsa.pub into authorized_keys
# SSH to remote machine, copy value from ~/.ssh/known_hosts* on remote
# machine into known_hosts*. Or take values directly from /etc/ssh/ssh_host_*_key.pub
cd ~/dot;make ssh;cd
# Install GPG keys from
# https://help.ubuntu.com/community/GnuPrivacyGuardHowto
# On trusted machine, do
gpg --list-keys
gpg --list-secret-keys
cd;gpg -ao foo_-public.key --export ${GPGKEY}
cd;gpg -ao foo_-private.key --export-secret-keys ${GPGKEY}
scp foo_*.key dust.ess.uci.edu:
# On new machine, do
scp 'dust.ess.uci.edu:foo*.key' ~
cd;gpg --import foo_-public.key
cd;gpg --import foo_-private.key
/bin/rm ~/foo_*
# Transition away from SHA1 to SHA256 as per
# http://www.debian-administration.org/users/dkg/weblog/48
cat >> ~/.gnupg/gpg.conf << EOF
personal-digest-preferences SHA256
cert-digest-algo SHA256
EOF
# Directories for personal machines (superset of below)
# cvs -d :ext:zender@dust.ess.uci.edu:/home/zender/cvs co -kk \
for drc in \
aca aeroce aeronet afgl anl anv arese avhrr bxm c c++ cld clm cns crm crr \
dead dmr dst esmf \
ess ess_acc ess_atm ess_bnd ess_ccc ess_ccp ess_cnt ess_gng ess_gpc ess_hyd ess_lsp ess_phz ess_prc ess_rdn ess_sco \
f fsf grd hdf hire icr idea idl idx_rfr igpp job jrn lsm ltr \
map match matlab mie mk mny ncep ncl perl phd poetry pr \
prp prp_arl prp_IGPP06 prp_access prp_acme prp_ans prp_axs prp_esdr prp_gpe prp_ids prp_sdci prp_si2 \
rvw sdn slr_spc tex time toms trv uci www \
;do svn co ${SVNROOT}/${drc};done
# Directories for computational machines
for drc in aca afgl anl c c++ cld clm crr dead dst f fsf icr idx_rfr map \
mie mk ncl perl slr_spc time \
;do cd;svn co ${SVNROOT}/${drc};done
# Papers
for drc in \
ppr_BiZ03 ppr_BiZ04 ppr_CaZ08 ppr_FlZ05 ppr_FlZ06 ppr_FZR07 ppr_GrZ04 ppr_HZM12 ppr_ZeK05 \
ppr_ZeM07 ppr_Zen08 ppr_Zen12 ppr_ZFM07 ppr_ZGD09 ppr_ZKT12 ppr_ZMT04 ppr_ZeT06b ppr_ZeT06 sltsbl \
;do cd;svn co ${SVNROOT}/${drc};done
scp -r givre.ess.uci.edu:ppr_ZFM07/fgr ~/ppr_ZFM07
cvs -d :ext:charlesz@hox.uio.no:/mn/hox/d1/alfgr/CVS co -kk ppr_GMZ05 
cd ~/tex;scp dust.ess.uci.edu:tex/mflanner.bib .
# Scott's papers
cd;rsync givre.ess.uci.edu:ppr_CaZ09 .
cd;rsync givre.ess.uci.edu:ppr_CaZ10 .
# Other Proposals
cvs -z3 -d :ext:zender@nco.cvs.sourceforge.net:/cvsroot/nco co -kk nco
svn checkout https://swamp.googlecode.com/svn/trunk/ swamp --username charlie.zender
sudo mkdir /data;cd /data;sudo mkdir zender;sudo chown zender zender;sudo chgrp cgdcsm zender
cd /data/zender;mkdir -p mie ps tmp # Create data directories that may be empty
# Disk management and formatting
sudo aptitude install dosfstools gparted mtools 
# Hardware monitoring 
sudo aptitude install gkrellm hddtemp powertop xosview
# Language
sudo aptitude install language-pack-fr language-pack-es language-pack-en
# Reboot-less Kernel Upgrades with KSplice
sudo aptitude install python-pycurl kerneloops python-yaml gdebi ksplice
# Security
sudo aptitude install ksplice
sudo aptitude install enigmail
sudo aptitude install chkrootkit clamav ksplice
sudo chkrootkit
# Packages for building NCO .debs
sudo aptitude install antlr autoconf autoconf-doc automake \
 binutils-multiarch bison ccache \
 debhelper debootstrap devscripts dput debian-policy dupload \
 figlet fakeroot flex gnuplot hardening-check libantlr-dev libtool lintian \
 pbuilder reportbug wdiff
sudo aptitude install manpages-dev glibc-doc
# sudo aptitude install gfortran-4.1 # Needed for NCL/NCAR Graphics? (but may break HDF5/netcdf4 installation which prefers gfortran to g95)
# MPI: Choose one MPI distribution: OpenMPI (recommended), MPICH2, MPICH
sudo aptitude install openmpi-bin openmpi-doc openmpi-dev
# Install GSL
sudo aptitude install gsl-bin libgsl0-dev gsl-doc-pdf
# Install netcdf by hand (uses Fortran90 build, finishes C++ build)
sudo aptitude install antlr libantlr-dev # ANTLR
sudo aptitude install libcurl4-gnutls-dev libexpat1-dev libxml2-dev # DAP-prereqs
sudo aptitude install bison flex gcc g++ # GNU toolchain
sudo aptitude install gsl-bin libgsl0-dev # GSL
sudo aptitude install doxygen libnetcdf6 libnetcdf-dev netcdf-bin # netCDF and DAP
sudo aptitude install udunits libudunits2-0 libudunits2-dev # UDUnits
# netcdf-perl needed by aca HITRAN programs
sudo aptitude install libpdl-netcdf-perl libxml-simple-perl
# Install HDF4
sudo aptitude install libhdf4-dev # For netCDF
sudo aptitude install hdf4-tools 
sudo aptitude install libhdfeos-dev
# Install HDF5
sudo aptitude install hdf5-tools libhdf5-serial-dev libhe5-hdfeos-dev
sudo aptitude install h5utils
# Install CFchecker
sudo aptitude install python-numpy # CFchecker
sudo aptitude install python-netcdf # CFchecker
sudo aptitude install netcdf4-python # CFchecker
sudo aptitude install cfunits # CFchecker hand-install
sudo aptitude install python-lxml # CFchecker
# Install nr by hand (needed by fff)
# Install specfun (needed by fff)
cd ~/c;make makdep # Build makdep only
cd ~/nco/bld;make OPTS=D allinone # Build NCO module (needed by mie)
cd ~/nco/src/nco_c++;make -f Makefile.old all;cd ~ # Build libnco_c++ (needed by c++)
cd ~/c++;make OMP=N # Build c++ module (including getopt_bsd)
cd ~/c;make # Build c module
cd ~/mie;make OMP=N # Build mie module
cd ~/f;make OMP=N # Build f module
cd ~/dead;make # Build dead module
cd ~/aca;make OMP=N # Build aca module
# Copy important data directories wholesale
for drc in aca arese avhrr data dst fgr hitran icr lsmdata map mny mtrdst no2 ppr prp rsmas specfun tex toms tomsaod tuv wiscombe ; do
 rsync -av givre.ess.uci.edu:/data/zender/${drc} ${DATA}
 printf "Copied dust.ess.uci.edu:/data/zender/${drc}\n"
done
# Copy personal data directories wholesale
for drc in bc_health clean dstmch90 dstmtr dvr esh ess ess_atm ess_bnd ess_ccp ess_cnt ess_gcm ess_gng ess_gpc ess_hyd ess_phz ess_prc ess_sco fr frnd harvard hire hlt ipcc iusd kdz lbaop lgl mny mtrdst opac pix slf trv uci ; do
 rsync -av givre.ess.uci.edu:/data/zender/${drc} ${DATA}
 printf "Copied givre.ess.uci.edu:/data/zender/${drc}\n"
done
# Mount /dev/sg0 as /cdrom to enable CD-burning (done in fstab)
# Install packages not on default Debian unstable install 
# Graphics, X-related
sudo aptitude install \
dontzap xrestop xsane
sudo dontzap --disable # Ctrl+Alt+Backspace restarts the xserver
# Printing, web, A/V-related
sudo aptitude install \
cupsys cupsys-client cupsys-driver-gimpprint \
gimp foomatic-filters-ppds foomatic-filters imagemagick \
kaffe sane traceroute vncserver xvncviewer
# Home Printer Canon PIXMA MG8120
dtm;sudo scp Dell_5130cdn.ppd /usr/share/cups/model
dtm;sudo dpkg -i cnijfilter*.deb
# Debugging
sudo aptitude install \
ddd ddd-doc gdb gfortran graphviz valgrind valgrind-callgrind \
graphviz kcachegrind kcachegrind-converters \
# TeX, LaTeX fundamentals
sudo aptitude install texlive-bibtex-extra texlive-fonts-recommended texlive-fonts-extra texlive-latex-extra texlive-extra-utils # LaTeX
# TeX, document-related
sudo aptitude install antiword bibtool bibtex2html cmatrix-xfont dvipng emacs-intl-fonts gawk gsfonts-x11 info latex2html mpage netpbm pdftk perl-doc preview-latex-style prosper texi2html texinfo tipa txt2html ttf-bitstream-vera ttf-freefont tth untex xfonts-100dpi 
# OpenOffice hack required on virga 20080923
sudo aptitude install openoffice.org-java-common
# Packages R requires:
sudo aptitude install r-base r-cran-gplots
# Packages NCL requires:
sudo aptitude install csh
# Install medibuntu for video codecs
sudo wget http://www.medibuntu.org/sources.list.d/intrepid.list
sudo aptitude install medibuntu-keyring
# Now can install acroread, googleearth, codecs
sudo aptitude install googleearth-package
sudo aptitude install acroread acroread-l10n-fr
sudo aptitude install w64codecs
sudo aptitude install mplayer
sudo aptitude install xine-ui
# KDE applications
sudo aptitude install amarok konsole okular
# Video, sound-related: 
sudo aptitude install mpg321 vorbis-tools # Audio (re-)sampling
sudo aptitude install ubuntu-restricted-extras # includes flashplugin-nonfree, sun java plugins
sudo aptitude install regionset kubuntu-restricted-extras
sudo aptitude install libdvdcss2 # from medibuntu repository
# 20070819: NB: Flash differs from Shockwave. There is no Linux Shockwave player.
sudo aptitude install flashplugin-installer
# 20100505: Required steps for Flash on 64-bit systems:
sudo aptitude purge flashplugin-nonfree flashplugin-installer
sudo add-apt-repository ppa:sevenmachines/flash
sudo aptitude install update
sudo aptitude install flashplugin64-installer
# Download and install http://labs.adobe.com/downloads/flashplayer10.html
dtm;scp 'dust.ess.uci.edu:/data/zender/tmp/libflash*' .
tar xvzf libflashplayer-10.0.22.87.linux-x86_64.so.tar.gz
sudo cp libflashplayer.so /usr/lib/mozilla/plugins
# Install TightVNC for webcasts
sudo aptitude install tightvnc-java tightvncserver xtightvncviewer
xine # Watch DVDs
# Get LaTeX working
rsync dust.ess.uci.edu:tex/cls ~/tex
scp -r 'dust.ess.uci.edu:${TEXMFDIR}/tex/latex/datetime' .;sudo mv datetime ${TEXMFDIR}/tex/latex # datetime.sty
scp -r 'dust.ess.uci.edu:${TEXMFDIR}/tex/latex/fmtcount' .;sudo mv fmtcount ${TEXMFDIR}/tex/latex # fmtcount.sty
scp -r 'dust.ess.uci.edu:${TEXMFDIR}/tex/latex/floatflt' .;sudo mv floatflt ${TEXMFDIR}/tex/latex # floatflt.sty
scp -r 'dust.ess.uci.edu:${TEXMFDIR}/tex/latex/ifdraft' .;sudo mv ifdraft ${TEXMFDIR}/tex/latex # ifdraft.sty
scp -r 'dust.ess.uci.edu:${TEXMFDIR}/tex/latex/revnum' .;sudo mv revnum ${TEXMFDIR}/tex/latex # revnum.sty
scp -r 'dust.ess.uci.edu:${TEXMFDIR}/tex/latex/tocbibind' .;sudo mv tocbibind ${TEXMFDIR}/tex/latex # tocbibind.sty
scp 'dust.ess.uci.edu:/data/zender/ps/erbe_b_sld012d_8589_0[17]_x_[LS]WCF.*' 'dust.ess.uci.edu:/data/zender/ps/erbe_b_8589_0[17]_[LS]WCF.*' 'dust.ess.uci.edu:/data/zender/ps/erbe_b_8589_SOLIN.*' ${DATA}/ps # ltx.tex, rt.tex
scp dust.ess.uci.edu:/data/zender/ps/uci_fax.pdf dust.ess.uci.edu:/data/zender/ps/uci_loa.pdf dust.ess.uci.edu:/data/zender/ps/uci_mlg.xls dust.ess.uci.edu:/data/zender/ps/uci_ntr.pdf dust.ess.uci.edu:/data/zender/ps/uci_po.pdf dust.ess.uci.edu:/data/zender/ps/uci_rmb.pdf dust.ess.uci.edu:/data/zender/ps/uci_trv.pdf dust.ess.uci.edu:/data/zender/ps/sgn_csz.eps ${DATA}/ps # Forms
rsync -r dust.ess.uci.edu:tex/bst ~/tex
cd ~/tex;sudo mkdir ${TEXMFDIR}/bibtex/bst/csz;sudo cp bst/* ${TEXMFDIR}/bibtex/bst/csz
# sudo chmod a+w /var/cache/fonts/ls-R
# sudo chmod 666 /dev/dsp # Sound device mode 660 causes permissions errors
# Make sure /cdrom is mount point for /dev/sg0 in /etc/fstab
# Get hardware tools (USB mouse, reiserfs utilities)
sudo aptitude install gpm lm-sensors smartmontools # Get narrower font
# Create links to directories in ${DATA}
sudo mkdir -p /fs/cgd/home0;sudo ln -s /home/zender /fs/cgd/home0/zender
sudo mkdir -p /fs/cgd/data0;sudo ln -s /data/zender /fs/cgd/data0/zender
mkdir -p ${DATA}/csm/inputdata/lnd/clm2 ${DATA}/csm/input
sudo mkdir -p /fs/cgd;sudo ln -s /data/zender/csm /fs/cgd/csm
cd ${DATA}/csm/inputdata/lnd/clm2;scp -r dust.ess.uci.edu:${DATA}/csm/inputdata/lnd/clm2/rawdata .
cd ${DATA}/csm/input;scp -r dust.ess.uci.edu:${DATA}/csm/input/atm dust.ess.uci.edu:${DATA}/csm/input/lnd .
sudo mkdir -p /datashare/inputdata/csm
sudo ln -s ${DATA}/csm/inputdata/lnd /datashare/inputdata/csm/lnd
# Remove automatic gdm/kdm/xdm
cd /etc/init.d;mkdir unused;mv gdm unused
# Add Ubuntu PPAs for Google Chromium, wicd
# deb http://ppa.launchpad.net/chromium-daily/ppa/ubuntu jaunty main 
# deb-src http://ppa.launchpad.net/chromium-daily/ppa/ubuntu jaunty main 
sudo apt-key adv --recv-keys --keyserver keyserver.ubuntu.com 4E5E17B5
# Install ccache as default compiler
sudo cp /usr/bin/ccache /usr/local/bin
sudo ln -s /usr/local/bin/ccache /usr/local/bin/gcc
sudo ln -s /usr/local/bin/ccache /usr/local/bin/g++
sudo ln -s /usr/local/bin/ccache /usr/local/bin/icc
sudo ln -s /usr/local/bin/ccache /usr/local/bin/icpc
\end{verbatim}

Commands to set up a basic scientific computing environment
\begin{verbatim}
mkdir -p bin
/bin/cp -r ~zender/sh ~
export PVM_ARCH=`~/sh/pvmgetarch`
mkdir -p ~/bin/${PVM_ARCH} ~/include ~/lib/${PVM_ARCH} ~/obj/${PVM_ARCH}
cp ~zender/bin/${PVM_ARCH}/makdep ~/bin/${PVM_ARCH}
cp ~zender/sh/pvmgetarch ~/sh
mv ~/.bashrc ~/.bashrc.old
cp ~zender/.bashrc ~
\end{verbatim}

\subsection{SWAMP}\label{sxn:swamp}
\begin{verbatim}
# Prevent copying all data from this and all subdirectories
zender@esmf04m:/data/zender/sncpi10$ cat > .swamp-filter
- *
ctrl-D
# Prevent copying of first 20-years of data
zender@esmf04m:/data/zender/sncpi10$ cat > .swamp-filter
- *snc*.c?m2.h0.00[01]?-*.nc
ctrl-D
\end{verbatim}
Files end up on \mchidx{pbs.ess.uci.edu} in local directory
\flidx{/misc/bulk/swamp}.
Files exported from ESMF land in 
\flidx{/misc/bulk/swamp/esmf/zender}.
SWAMP scripts root files at the machine name level, e.g.,
SWAMP sees \flidx{/misc/bulk/swamp/esmf/zender/sncpd10} as
\flidx{/esmf/zender/sncpd10}.

\subsection{Recreating Server Files}\label{sxn:srv}
Server port to leave open when registering over UCInet.
\begin{verbatim}
29659 # NCO
1700 # PGI, IDL license servers
7143 # PathScale licence server on IPCC
\end{verbatim}
When the system goes down and takes the \flidx{/etc} partition with
it, the system services must be re-installed from scratch.
\begin{verbatim}
#sudo mv /home/backup/etc/passwd /etc/passwd
ls /home/backup

# Reset SSH keys to ease work on remote machine 

# Reinstall sudo permissions
sudo scp biogenic.ess.uci.edu:/etc/sudoers dust.ess.uci.edu:/etc/sudoers

# Restore accounts (RedHat Linux: add -n to turn off creation of group named after user
sudo groupadd -g 2400 cgdcsm # Create group cgdcsm
sudo groupadd -g 1965 esmfadm # Create group esmfadm
sudo useradd -c 'Charlie Zender' -d /home/zender -g cgdcsm -m -p '$1$charlie$VJFpvyBLLZzEvLkainaFW/' -u 3555 zender
sudo useradd -c 'Mark Flanner' -d /home/mflanner -g cgdcsm -m -p '$1$charlie$VJFpvyBLLZzEvLkainaFW/' -u 3563 mflanner
sudo useradd -c 'Eun Young Kwon' -d /home/kwon -g cgdcsm -m -p '$1$charlie$VJFpvyBLLZzEvLkainaFW/' -u 3564 kwon
sudo useradd -c 'Huisheng Bian' -d /home/bian -g cgdcsm -m -p '$1$charlie$VJFpvyBLLZzEvLkainaFW/' -u 3008 bian
sudo useradd -c 'Alf Grini' -d /home/alfgr -g cgdcsm -m -p '$1$charlie$VJFpvyBLLZzEvLkainaFW/' -u 3012 alfgr
sudo useradd -c 'Dave Newman' -d /home/newman -g cgdcsm -m -p '$1$charlie$VJFpvyBLLZzEvLkainaFW/' -u 3009 newman
sudo useradd -c 'Sarah Bortz' -d /home/bortz -g cgdcsm -m -p '$1$charlie$VJFpvyBLLZzEvLkainaFW/' -u 3557 bortz
sudo useradd -c 'Renato Pajarola' -d /home/pajarola -g cgdcsm -m -p '$1$charlie$VJFpvyBLLZzEvLkainaFW/' -u 3561 pajarola
sudo useradd -c 'Jorge Talamantes' -d /home/jtalaman -g cgdcsm -m -p '$1$charlie$VJFpvyBLLZzEvLkainaFW/' -u 3566 jtalaman
sudo useradd -c 'Dan Stromberg' -d /home/strombrg -g esmfadm -G users -m -p '$1$charlie$VJFpvyBLLZzEvLkainaFW/' -u 3570 strombrg
sudo useradd -c 'Francisco Lopez' -d /home/lopez -g esmfadm -G users -m -p '$1$charlie$VJFpvyBLLZzEvLkainaFW/' -u 3571 lopez
sudo useradd -c 'Greg Okin' -d /home/okin -g cgdcsm -m -p '$1$charlie$VJFpvyBLLZzEvLkainaFW/' -u 3568 okin
sudo useradd -c 'Harry Mangalam' -d /home/hjm -g cgdcsm -m -p '$1$charlie$VJFpvyBLLZzEvLkainaFW/' -u 3567 hjm
sudo useradd -c 'Test User' -d /home/test -g cgdcsm -m -p '$1$charlie$VJFpvyBLLZzEvLkainaFW/' -u 3569 test
sudo useradd -c 'ESMF System Administrator' -d /home/esmfadm -g cgdcsm -m -p '$1$charlie$VJFpvyBLLZzEvLkainaFW/' -u 3565 esmfadm
sudo useradd -c 'CVS' -M -g cgdcsm -u 3562 cvspub
sudo useradd -c 'ESS 211 Class Account' -d /home/ess211 -g cgdcsm -m -p '$1$charlie$VJFpvyBLLZzEvLkainaFW/' -u 3006 ess211
sudo useradd -c 'Hsun-Ying Kao' -d /home/hkao -g cgdcsm -m -p '$1$charlie$VJFpvyBLLZzEvLkainaFW/' -u 3572 hkao
sudo useradd -c 'Gayathri Venkitachalam' -d /home/gvenkita -g cgdcsm -m -p '$1$charlie$VJFpvyBLLZzEvLkainaFW/' -u 3573 gvenkita
sudo useradd -c 'Chao Luo' -d /home/chaoluo -g cgdcsm -m -p '$1$charlie$VJFpvyBLLZzEvLkainaFW/' -u 3574 chaoluo
sudo useradd -c 'Qin Han' -d /home/qhan -g cgdcsm -m -p '$1$charlie$VJFpvyBLLZzEvLkainaFW/' -u 3575 qhan
sudo useradd -c 'Scott Capps' -d /home/scapps -g cgdcsm -m -p '$1$charlie$VJFpvyBLLZzEvLkainaFW/' -u 3576 scapps
# Oops, got to change Robynn's UID
sudo useradd -c 'Robynn Zender' -d /home/robynn -g cgdcsm -m -p '$1$charlie$VJFpvyBLLZzEvLkainaFW/' -u 3577 robynn
sudo useradd -c 'Mike Brown' -d /home/brownmc -g cgdcsm -m -p '$1$charlie$VJFpvyBLLZzEvLkainaFW/' -u 3577 brownmc
sudo useradd -c 'Danielle Johnson' -d /home/daniellj -g cgdcsm -m -p '$1$charlie$VJFpvyBLLZzEvLkainaFW/' -u 3578 daniellj
sudo useradd -c 'Mike Tosca' -d /home/mtosca -g cgdcsm -m -p '$1$charlie$VJFpvyBLLZzEvLkainaFW/' -u 3579 mtosca
sudo useradd -c 'Henry Butowsky' -d /home/hbutowsk -g cgdcsm -m -p '$1$charlie$VJFpvyBLLZzEvLkainaFW/' -u 3580 hbutowsk
sudo useradd -c 'Olivia Zender' -d /home/olivia -g cgdcsm -m -p '$1$charlie$VJFpvyBLLZzEvLkainaFW/' -u 3581 olivia
sudo useradd -c 'Daniel Wang' -d /home/wangd -g cgdcsm -m -p '$1$charlie$VJFpvyBLLZzEvLkainaFW/' -u 3582 wangd
sudo useradd -c 'Xianwei Wang' -d /home/xianweiw -g cgdcsm -m -p '$1$charlie$VJFpvyBLLZzEvLkainaFW/' -u 3583 xianweiw
sudo useradd -c 'Bob Allen' -d /home/rjallen -g cgdcsm -m -p '$1$charlie$VJFpvyBLLZzEvLkainaFW/' -u 3584 rjallen
sudo useradd -c 'Guest User' -d /home/guest -g cgdcsm -m -p '$1$charlie$VJFpvyBLLZzEvLkainaFW/' -u 3585 guest
sudo useradd -c 'Alex Krolewski' -d /home/krolewski -g cgdcsm -m -p '$1$charlie$VJFpvyBLLZzEvLkainaFW/' -u 3586 krolewski
sudo useradd -c 'Pedro Vicente' -d /home/pvicente -g cgdcsm -m -p '$1$charlie$VJFpvyBLLZzEvLkainaFW/' -u 3587 pvicente
sudo useradd -c 'Wenshan Wang' -d /home/wenshanw -g cgdcsm -m -p '$1$charlie$VJFpvyBLLZzEvLkainaFW/' -u 3588 wenshanw
sudo useradd -c 'Dingying Wei' -d /home/dywei -g cgdcsm -m -p '$1$charlie$VJFpvyBLLZzEvLkainaFW/' -u 3589 dywei
sudo useradd -c 'Christine Smit' -d /home/csmit -g cgdcsm -m -p '$1$charlie$VJFpvyBLLZzEvLkainaFW/' -u 3590 csmit
sudo useradd -c 'Feng Ding' -d /home/fding -g nco -m -p '$1$charlie$VJFpvyBLLZzEvLkainaFW/' -u 3591 fding
sudo useradd -c 'Sagar Parajuli' -d /home/parajuli -g nco -m -p '$1$charlie$VJFpvyBLLZzEvLkainaFW/' -u 3592 parajuli
sudo useradd -c 'Zach Wolff' -d /home/zwolff -g cgdcsm -m -p '$1$charlie$VJFpvyBLLZzEvLkainaFW/' -u 3593 zwolff
sudo useradd -c 'Cheng Dang' -d /home/cdang5 -g cgdcsm -m -p '$1$charlie$VJFpvyBLLZzEvLkainaFW/' -u 3594 cdang5

# Create work directories and change permissions 
for usr in krolewski; do
  sudo chmod 755 /home/${usr}
  sudo mkdir /data/${usr};cd /data;sudo chown ${usr} ${usr};sudo chgrp cgdcsm ${usr}
  sudo mkdir /var/www/html/${usr};cd /var/www/html;sudo chown ${usr} ${usr};sudo chgrp cgdcsm ${usr}
done

# Patch RPMs

# Restore /usr/local (includes NCAR graphics, Lahey fortran, PGI, RSI)
cd /usr;tar cvzf local.tar.gz ./local
sudo scp local.tar.gz dust.ess.uci.edu:/usr
cd /usr;tar xvzf local.tar.gz

# Restore CVS
sudo mkdir /var/lock/cvs
sudo chmod 777 /var/lock/cvs
sudo scp ~/linux/etc/xinetd.d/cvspserver /etc/xinetd.d
sudo /etc/rc.d/init.d/xinetd restart

# Restore WWW
#sudo /bin/rm -r /var/www/html
#sudo scp -r /var/www/html dust.ess.uci.edu:/var/www
sudo scp ~/linux/etc/apache/httpd.conf dust.ess.uci.edu:/etc/httpd/conf/httpd.conf
sudo chown -R zender *
sudo chgrp -R cgdcsm *
sudo /etc/init.d/httpd restart
# Rebuild contents: doc, ppr, facts, bxm, nco, 

# Restore BXM
cd ~/bxm;sudo make bxm;cd -
cd ~/bxm;sudo make bxm_aer;cd -
sudo mkdir -p /var/ftp/dead
sudo chown apache /var/ftp/dead
sudo chgrp apache /var/ftp/dead

# Restore NFS Crossmounts
sudo /etc/init.d/nfs restart
sudo /usr/sbin/exportfs -rv
sudo mount -a

# Restore Scientific computing
sudo rpm -i gsl-1.4-5mdk.src.rpm
netCDF

# Restore FTP
cd /var/ftp/pub
sudo mkdir zender
sudo chown zender zender
sudo chgrp cgdcsm zender
\end{verbatim}

\subsection{i8500s}\label{sxn:i8500}
Inspiron~8500
\begin{verbatim}
Use the xev program to find the keycodes for the volume keys. 
In my case they turned out to be 174 (lower) and 176 (raise).  
Ran the commands xmodmap -e "keycode 174 = XF86AudioLowerVolume" and
xmodmap -e "keycode 176 = XF86AudioRaiseVolume" to map the buttons to
the right key symbols.  
Configured the Sawfish window manager to run the command aumix -c-5 on
lower and aumix -c+5 on raise.  
Tried it, and it worked. 
Put the commands into .xsession to make this permanent. 
\end{verbatim}

\subsection{Images}\label{sxn:img}
Images refers to all still images, including \trmidx{JPEG},
\trmidx{tiff}, etc. 
The best program to use for viewing images depends on your intent.
For image processing, use \cmdidx{gimp}.
For slideshows of raster images, use \cmdidx{gimp}.
\subsubsection{Gimp}\label{sxn:gimp}
The \cmdidx{gimp} program is excellent for viewing and editing images.
However, it has now slideshow capabilities.

\subsubsection{KView}\label{sxn:kview}
\subsubsection{Kuickshow}\label{sxn:kuickshow}
Kuickshow is a very fine tool with excellent preview and slieshow
modes.
In preview mode, just click on a filename and keep hitting 
\kbdidx{Pg Dn} to see new pictures.
The slideshow mode has an appropriate delay betweeen slides.
Remember to de-activate power-saving and screen-blanking before
expecting the slideshow to continue unattended.

\subsection{CDs}\label{sxn:cd}
\trmidx{Compact Disks} (\trmidx{CDs}) store up to 700\,\MB.
Formatting and writing data to the CD, known as \trmidx{burning}, is
not always straightforward with Linux.
The program \cmdidx{cdbakeoven} has an intuitive interface.
It provides continuous real-time status reports during burning, and
sometimes works when K3b fails.
On \mchidx{sand.ess.uci.edu}, K3b tends to fail yet
\cmdprn{cdbakeoven} works. 
On \mchidx{elnino.ess.uci.edu}, K3b tends to work and 
\cmdprn{cdbakeoven} fails. 

The \trmidx{KDE} CD/DVD-burning utility is called \trmidx{K3b}.
It automatically loads when it detects a blank CD in a writable drive.
However, K3b does not have a completion meter and just hangs when it
fails to burn CDs (at least under \trmidx{SuSE}.  
This is annoying because one does not know whether the program is
working, and how long until completion.

\subsection{DVDs}\label{sxn:dvd}
I use \cmdidx{xine} and \cmdidx{mplayer} to play \trmidx{DVDs}.
Due to legal concerns, few Linux distributions automatically install
\trmidx{DeCSS}, the pre-requisite library for decrypting DVDs.
Multimedia software works with \trmidx{media resource locators}
(\trmidx{MRL}s). 
MRLs are similar to \trmidx{URL}s with media-like protocols.
Most audio/video players do not allow direct control of volume,
perhaps because the computer speakers are a shared resource.
Hence it is wise to start the audio controls before the video player.
\trmidx{Kmix} is a fine audio controller to use.

\subsubsection{MPlayer}\label{sxn:mplayer}
\trmidx{MPlayer} accepts many of the same commands that work with
\trmidx{Xine}: 
\begin{verbatim}
kmix & # Volume control
mplayer dvd://1 # Quickstart DVD playing
mplayer dvd://1 -dvdangle 2 # Multiangle DVD playing
mplayer http://mplayer.hq/example.avi # Stream from HTTP
mplayer rtsp://server.example.com/streamName # Stream using RTSP
mencoder "mf://*.jpg" -mf fps=25 -o output.avi -ovc divx4 # Encode all *.jpg files
mencoder  -tv  driver=v4l:width=640:height=480  tv:// -o tv.avi -ovc raw # Encode from tuner
\end{verbatim}
Mplayer has difficulty with full-screen mode.

\subsubsection{dvdrip}\label{sxn:dvdrip}
\begin{verbatim}
sudo aptitude install lame
sudo aptitude install dvdrip vcdimager cdrdao subtitleripper
sudo aptitude install rar unrar
sudo ln -fs /usr/bin/rar /usr/bin/rar-2.80
rsync /data/zender/snd dust.ess.uci.edu:/data/zender
cd /data/zender;rsync dust.ess.uci.edu:/data/zender/snd .
\end{verbatim}

\subsubsection{Xine}\label{sxn:xine}
\trmidx{Xine} accepts many of the same commands that work with
\trmidx{Mplayer}: 
\begin{verbatim}
kmix & # Volume control
xine dvd://1 # Quickstart DVD playing
xine dvd://1 -dvdangle 2 # Multiangle DVD playing
xine http://mplayer.hq/example.avi # Stream from HTTP
xine rtsp://server.example.com/streamName # Stream using RTSP
\end{verbatim}
Xine provides trouble-free operation in full-screen mode.

\begin{verbatim}
/usr/share/doc/libdvdread2/examples/install-css.sh # 
aptitude install xine-ui
\end{verbatim}
Make sure the DVD is in the drive an mounted
\begin{verbatim}
sudo ln -s -f /dev/sr0 /dev/cdrom # ashes
sudo ln -s -f /dev/sr0 /dev/dvd # ashes
sudo mount -t iso9660 -o ro,user,noauto,unhide /dev/cdrom /cdrom
sudo mount -t udf -o ro,user,noauto,unhide /dev/dvd /dvd 
sudo mount -t udf -o ro,user,noauto,unhide /dev/dvd /dvd 
# Xine mounts DVDs like this on elnino:
sudo mount -t udf -o ro,noexec,nosuid,nodev /dev/scd0 /cdrom
\end{verbatim}
\cmdprn{xine} generates lots of warning messages: 

Adobe's \cmdprn{Flash} is a popular format for video.
\begin{verbatim}
# Remove Ubuntu Flash modules
sudo aptitude remove libflash-mozplugin libflashsupport flashplugin-nonfree
# Install Adobe's flash player beta release
tar xvzf flashplayer10_install_linux_081108.tar.gz
cd install_flash_player_10_linux
# Only works on x86 (not x86_64!)
\end{verbatim}

Run \cmdprn{xine} as \trmidx{root} until permissions are straightened out.  
\begin{verbatim}
libdvdcss error: failed opening raw device, continuing
main: probing <aadxr3> video output plugin
load_plugins: failed to load video output plugin <aadxr3>
main: probing <dxr3> video output plugin
load_plugins: failed to load video output plugin <dxr3>
main: probing <xv> video output plugin
main: probing <alsa> audio output plugin
load_plugins: failed to load audio output plugin <alsa>
main: probing <oss> audio output plugin
xine_interface: unknown param 10
xine_interface: unknown param 10
xine_interface: unknown param 10
xine_interface: unknown param 10
vo_scale: invalid ratio, using 4:3
vo_scale: unknown aspect ratio (0) in stream => using 4:3
load_plugins: plugin dxr3-spudec failed to instantiate itself.
libspudec:init_plugin called
liba52:No accelerated IMDCT transform found
\end{verbatim}

\subsection{Network}\label{sxn:ntw2}
Linksys Etherfast 10/100 hub\\
\\
netstat -rn: BuH98 p. 177\\
Destination 0.0.0.0 is default destination of all datagrams not explicitly\\
routed elsewhere\\
Flags: U = Route is Up or Usable, G = Route uses a gateway, H = Route targets a host\\

Sometimes connectivity to a host or a subnet can be lost when a static
route gets stuck in the routing table.
This has happened twice in the past when system administrators log
into \cmdprn{lanina} and manipulate the routing table.
In such cases, the static route must be manually removed from the
routing table as follows.
Assuming \cmdprn{128.200.14.0} is unreachable,
\begin{verbatim}
zender@lanina:~$ /sbin/route -n
Kernel IP routing table
Destination     Gateway         Genmask         Flags Metric Ref    Use Iface
128.200.14.0    0.0.0.0         255.255.255.0   U     0      0        0 eth0
zender@lanina:~$ sudo route del -net 128.200.14.0 netmask 255.255.255.0 eth0
\end{verbatim}

\begin{verbatim}
zender@dust:~/dot$ netstat -rn:
Kernel IP routing table
Destination     Gateway         Genmask         Flags   MSS Window  irtt Iface
128.200.14.25   0.0.0.0         255.255.255.255 UH        0 0          0 eth0
128.200.14.0    0.0.0.0         255.255.255.0   U         0 0          0 eth0
127.0.0.0       0.0.0.0         255.0.0.0       U         0 0          0 lo
0.0.0.0         128.200.14.1    0.0.0.0         UG        0 0          0 eth0

zender@lanina:~$ netstat -rn (lanina at UCI):
Kernel IP routing table
Destination     Gateway         Genmask         Flags   MSS Window  irtt Iface
128.200.14.0    0.0.0.0         255.255.255.0   U         0 0          0 eth0
192.168.82.0    0.0.0.0         255.255.255.0   U         0 0          0 vmnet1
127.0.0.0       0.0.0.0         255.0.0.0       U         0 0          0 lo
0.0.0.0         128.200.14.1    0.0.0.0         UG        0 0          0 eth0

zender@lanina:~/dot$ netstat -rn (with dakine as PPP gateway to UCI)
Kernel IP routing table
Destination     Gateway         Genmask         Flags   MSS Window  irtt Iface
192.168.1.0     0.0.0.0         255.255.255.0   U         0 0          0 eth0
127.0.0.0       0.0.0.0         255.0.0.0       U         0 0          0 lo
0.0.0.0         192.168.1.1     0.0.0.0         UG        0 0          0 eth0

zender@dakine:~/dot$ netstat -rn (with dakine as PPP gateway to UCI)
Kernel IP routing table
Destination     Gateway         Genmask         Flags   MSS Window  irtt Iface
192.168.1.1     0.0.0.0         255.255.255.255 UH        0 0          0 eth0
128.195.187.2   0.0.0.0         255.255.255.255 UH        0 0          0 ppp0
192.168.1.0     0.0.0.0         255.255.255.0   U         0 0          0 eth0
127.0.0.0       0.0.0.0         255.0.0.0       U         0 0          0 lo
0.0.0.0         128.195.187.2   0.0.0.0         UG        0 0          0 ppp0   

zender@lanina:~/dot$ netstat -rn (with lanina as PPP gateway to UCI)
Kernel IP routing table
Destination     Gateway         Genmask         Flags   MSS Window  irtt Iface
128.195.186.2   0.0.0.0         255.255.255.255 UH        0 0          0 ppp0
192.168.1.0     0.0.0.0         255.255.255.0   U         0 0          0 eth0
127.0.0.0       0.0.0.0         255.0.0.0       U         0 0          0 lo
0.0.0.0         128.195.186.2   0.0.0.0         UG        0 0          0 ppp0

zender@dakine:~/dot$ netstat -rn (with lanina as PPP gateway to UCI)
Kernel IP routing table
Destination     Gateway         Genmask         Flags   MSS Window  irtt Iface
192.168.1.1     0.0.0.0         255.255.255.255 UH        0 0          0 eth0
192.168.1.0     0.0.0.0         255.255.255.0   U         0 0          0 eth0
127.0.0.0       0.0.0.0         255.0.0.0       U         0 0          0 lo
0.0.0.0         192.168.1.2     0.0.0.0         UG        0 0          0 eth0

zender@lanina:~$ netstat -rn (from NCAR as dhcp25.cgd.ucar.edu)
Kernel IP routing table
Destination     Gateway         Genmask         Flags   MSS Window  irtt Iface
128.117.0.0     0.0.0.0         255.255.0.0     U         0 0          0 eth0
127.0.0.0       0.0.0.0         255.0.0.0       U         0 0          0 lo
0.0.0.0         128.117.22.251  0.0.0.0         UG        0 0          0 eth0

zender@lanina:~/dot$ netstat -rn (with lanina as PPP gateway to NCAR)
Kernel IP routing table
Destination     Gateway         Genmask         Flags   MSS Window  irtt Iface
128.117.64.227  0.0.0.0         255.255.255.255 UH        0 0          0 ppp0
127.0.0.0       0.0.0.0         255.0.0.0       U         0 0          0 lo
0.0.0.0         128.117.64.227  0.0.0.0         UG        0 0          0 ppp0
\end{verbatim}

ifconfig -a:
\begin{verbatim}
zender@ashes:~$ ifconfig -a (at UCI, wired working wireless not quite)
dummy0    Link encap:Ethernet  HWaddr 00:00:00:00:00:00
          BROADCAST NOARP  MTU:1500  Metric:1
          RX packets:0 errors:0 dropped:0 overruns:0 frame:0
          TX packets:0 errors:0 dropped:0 overruns:0 carrier:0
          collisions:0
          RX bytes:0 (0.0 b)  TX bytes:0 (0.0 b)

eth0      Link encap:Ethernet  HWaddr 00:0B:DB:17:83:0D
          inet addr:128.200.14.90  Bcast:128.200.14.255  Mask:255.255.255.0
          UP BROADCAST RUNNING MULTICAST  MTU:1500  Metric:1
          RX packets:9379086 errors:0 dropped:0 overruns:0 frame:0
          TX packets:1543592 errors:0 dropped:0 overruns:0 carrier:0
          collisions:0
          RX bytes:1840313132 (1.7 GiB)  TX bytes:1113652359 (1.0 GiB)

eth1      Link encap:Ethernet  HWaddr 00:90:4B:B2:09:86
          BROADCAST MULTICAST  MTU:1500  Metric:1
          RX packets:0 errors:0 dropped:0 overruns:0 frame:0
          TX packets:0 errors:0 dropped:0 overruns:0 carrier:0
          collisions:0
          RX bytes:0 (0.0 b)  TX bytes:0 (0.0 b)

lo        Link encap:Local Loopback
          inet addr:127.0.0.1  Mask:255.0.0.0
          UP LOOPBACK RUNNING  MTU:16436  Metric:1
          RX packets:13954461 errors:0 dropped:0 overruns:0 frame:0
          TX packets:13954461 errors:0 dropped:0 overruns:0 carrier:0
          collisions:0
          RX bytes:1488519751 (1.3 GiB)  TX bytes:1488519751 (1.3 GiB)
zender@dust:~/dot$ ifconfig -a
eth0      Link encap:Ethernet  HWaddr 00:C0:4F:60:73:8D
          inet addr:128.200.14.25  Bcast:128.200.14.255  Mask:255.255.255.0
          UP BROADCAST RUNNING MULTICAST  MTU:1500  Metric:1
          RX packets:5406688 errors:2103 dropped:0 overruns:0 frame:3179
          TX packets:1398865 errors:0 dropped:0 overruns:0 carrier:24
          collisions:62451 txqueuelen:100
          Interrupt:17 Base address:0xdc00
 
lo        Link encap:Local Loopback
          inet addr:127.0.0.1  Mask:255.0.0.0
          UP LOOPBACK RUNNING  MTU:3924  Metric:1
          RX packets:59429 errors:0 dropped:0 overruns:0 frame:0
          TX packets:59429 errors:0 dropped:0 overruns:0 carrier:0
          collisions:0 txqueuelen:0                                                     
zender@lanina:~$ ifconfig -a (from UCI)
eth0      Link encap:Ethernet  HWaddr 00:10:A4:08:12:31  
          inet addr:192.168.1.2  Bcast:192.168.1.255  Mask:255.255.255.0
          UP BROADCAST RUNNING MULTICAST  MTU:1500  Metric:1
          RX packets:674 errors:0 dropped:0 overruns:0 frame:0
          TX packets:541 errors:2 dropped:0 overruns:0 carrier:2
          collisions:73 txqueuelen:100 
          Interrupt:11 Base address:0x200 

eth0:1    Link encap:Ethernet  HWaddr 00:10:A4:08:12:31  
          inet addr:128.200.14.80  Bcast:128.200.255.255  Mask:255.255.0.0
          UP BROADCAST RUNNING MULTICAST  MTU:1500  Metric:1
          Interrupt:11 Base address:0x200 

lo        Link encap:Local Loopback  
          inet addr:127.0.0.1  Mask:255.0.0.0
          UP LOOPBACK RUNNING  MTU:3924  Metric:1
          RX packets:45 errors:0 dropped:0 overruns:0 frame:0
          TX packets:45 errors:0 dropped:0 overruns:0 carrier:0
          collisions:0 txqueuelen:0 

zender@lanina:~/dot$ ifconfig -a
eth0      Link encap:Ethernet  HWaddr 00:10:A4:08:12:31
          inet addr:192.168.1.2  Bcast:192.168.1.255  Mask:255.255.255.0
          UP BROADCAST RUNNING MULTICAST  MTU:1500  Metric:1
          RX packets:897 errors:0 dropped:0 overruns:0 frame:0
          TX packets:1029 errors:0 dropped:0 overruns:0 carrier:0
          collisions:0 txqueuelen:100
          Interrupt:11 Base address:0x200
 
eth0:1    Link encap:Ethernet  HWaddr 00:10:A4:08:12:31
          inet addr:128.200.14.80  Bcast:128.200.255.255  Mask:255.255.0.0
          UP BROADCAST RUNNING MULTICAST  MTU:1500  Metric:1
          Interrupt:11 Base address:0x200
 
lo        Link encap:Local Loopback
          inet addr:127.0.0.1  Mask:255.0.0.0
          UP LOOPBACK RUNNING  MTU:3924  Metric:1
          RX packets:2982 errors:0 dropped:0 overruns:0 frame:0
          TX packets:2982 errors:0 dropped:0 overruns:0 carrier:0
          collisions:0 txqueuelen:0

zender@dakine:~$ ifconfig -a
eth0      Link encap:Ethernet  HWaddr 00:40:D0:0B:E9:E4
          inet addr:192.168.1.1  Bcast:192.168.1.255  Mask:255.255.255.0
          UP BROADCAST RUNNING MULTICAST  MTU:1500  Metric:1
          RX packets:2290 errors:0 dropped:0 overruns:0 frame:0
          TX packets:1273 errors:0 dropped:0 overruns:0 carrier:0
          collisions:0 txqueuelen:100
          Interrupt:10 Base address:0x2000
 
lo        Link encap:Local Loopback
          inet addr:127.0.0.1  Mask:255.0.0.0
          UP LOOPBACK RUNNING  MTU:3924  Metric:1
          RX packets:222 errors:0 dropped:0 overruns:0 frame:0
          TX packets:222 errors:0 dropped:0 overruns:0 carrier:0
          collisions:0 txqueuelen:0
 
ppp0      Link encap:Point-to-Point Protocol
          inet addr:128.195.187.37  P-t-P:128.195.187.2  Mask:255.255.255.255
          UP POINTOPOINT RUNNING NOARP MULTICAST  MTU:1500  Metric:1
          RX packets:216 errors:1 dropped:0 overruns:0 frame:1
          TX packets:221 errors:0 dropped:0 overruns:0 carrier:0
          collisions:0 txqueuelen:10

zender@lanina:~$ ifconfig -a (from NCAR as dhcp25.cgd.ucar.edu)
eth0      Link encap:Ethernet  HWaddr 00:10:A4:08:12:31  
          inet addr:128.117.22.248  Bcast:128.117.255.255  Mask:255.255.0.0
          UP BROADCAST RUNNING MULTICAST  MTU:1500  Metric:1
          RX packets:9198 errors:0 dropped:0 overruns:0 frame:0
          TX packets:529 errors:0 dropped:0 overruns:0 carrier:0
          collisions:28 txqueuelen:100 
          Interrupt:11 Base address:0x200 

lo        Link encap:Local Loopback  
          inet addr:127.0.0.1  Mask:255.0.0.0
          UP LOOPBACK RUNNING  MTU:3924  Metric:1
          RX packets:468 errors:0 dropped:0 overruns:0 frame:0
          TX packets:468 errors:0 dropped:0 overruns:0 carrier:0
          collisions:0 txqueuelen:0 

zender@lanina:~/dot$ ifconfig -a (with lanina as PPP gateway to NCAR)
eth0      Link encap:Ethernet  HWaddr 00:10:A4:08:12:31  
          inet addr:192.168.1.2  Bcast:192.168.1.255  Mask:255.255.255.0
          BROADCAST MULTICAST  MTU:1500  Metric:1
          RX packets:0 errors:0 dropped:0 overruns:0 frame:0
          TX packets:0 errors:12 dropped:0 overruns:0 carrier:12
          collisions:0 txqueuelen:100 
          Interrupt:11 Base address:0x200 

lo        Link encap:Local Loopback  
          inet addr:127.0.0.1  Mask:255.0.0.0
          UP LOOPBACK RUNNING  MTU:3924  Metric:1
          RX packets:120 errors:0 dropped:0 overruns:0 frame:0
          TX packets:120 errors:0 dropped:0 overruns:0 carrier:0
          collisions:0 txqueuelen:0 

ppp0      Link encap:Point-to-Point Protocol  
          inet addr:128.117.68.69  P-t-P:128.117.64.227  Mask:255.255.255.255
          UP POINTOPOINT RUNNING NOARP MULTICAST  MTU:1500  Metric:1
          RX packets:621 errors:4 dropped:0 overruns:0 frame:4
          TX packets:650 errors:0 dropped:0 overruns:0 carrier:0
          collisions:0 txqueuelen:10 

zender@haze: ifconfig -a (from UCI)
eth0      Link encap:Ethernet  HWaddr 00:10:A4:A0:9D:9D
          inet addr:128.200.14.62  Bcast:128.200.14.255  Mask:255.255.255.0
          UP BROADCAST RUNNING MULTICAST  MTU:1500  Metric:1
          RX packets:618084 errors:0 dropped:0 overruns:0 frame:0
          TX packets:416759 errors:0 dropped:0 overruns:0 carrier:0
          collisions:0 txqueuelen:100
          Interrupt:11 Base address:0xe000
 
lo        Link encap:Local Loopback
          inet addr:127.0.0.1  Mask:255.0.0.0
          UP LOOPBACK RUNNING  MTU:16436  Metric:1
          RX packets:28 errors:0 dropped:0 overruns:0 frame:0
          TX packets:28 errors:0 dropped:0 overruns:0 carrier:0
          collisions:0 txqueuelen:0
\end{verbatim}

\subsection{Kernel Modules}\label{sxn:mdl}
Modules may be listed with the \cmdidx{lsmod} command,
inserted with \cmdidx{insmod}, probed with \cmdidx{modprobe}, or
removed with \cmdidx{rmmod}.
\begin{verbatim}
lsmod # List loaded modules
insmod foo # Install loadable kernel module
modprobe ohci1394
modprobe raw1394
\end{verbatim}
Occasionally it is necessary to insert standalone kernel modules.
\begin{verbatim}
# First bring down interface to avoid 'interface busy' errors
ifconfig eth0 down
# Remove current module from kernel
rmmod tulip
# Insert new module into kernel
insmod /home/zender/tulip.o
# Examine messages
tail -10 /var/log/messages
\end{verbatim}

\subsection{Change network topology}\label{sxn:ntw3}
Linux Network (BuH98) p. 138, Linux Bible p. 374
Following scenarios include possibly superfluous commands to shut down
\flidx{eth0:1}.
If \flprn{eth0:1} is not shutdown and is running, route table may be
bad.
Machine which runs \cmdidx{PPP} (calls ISP) tries to set default route
to ppp0 interface.
If default route (0.0.0.0) on \cmdprn{PPP} host is preset to eth0,
\cmdprn{PPP} fails doing this 
Hence unset and preset eth0 default route on \cmdprn{PPP} host before
ydialing 

On \trmidx{RedHat} systems, the key networking information is kept in
\flidx{/etc/sysconfig/network} and in
\flidx{/etc/sysconfig/network-scripts/ifcfg-eth0}. 
Things such as \cmdidx{GATEWAY}, \cmdidx{NETMASK}, must be set
correctly in these files.
These files are easiest to change by hand.

\begin{enumerate*}
\item dakine is \cmdprn{PPP} dialin host and gateway for home LAN
Execute following commands on lanina:
\begin{verbatim}
# Delete existing Ethernet interface(s), if any
/sbin/ifconfig eth0 down
/sbin/ifconfig eth0:1 down 
/sbin/ifconfig eth0 lanina.zender.org # Connect IP address with Ethernet interface
# Add entry to kernel routing table which directs all datagrams to any host on network 192.168.1.0 to Ethernet interface fxm: gives error: SIOCADDRT: Invalid argument
/sbin/route add -net 192.168.1.0
# Route to gateway host all datagrams bound for hosts outside intranet
# fxm: "metric 1" argument appears necessary
/sbin/route add default gw 192.168.1.1 metric 1
# Execute following commands on dakine
/sbin/ifconfig eth0 down
/sbin/ifconfig eth0:1 down 
/sbin/ifconfig eth0 dakine.zender.org # Connect IP address with Ethernet interface
# Ensure default route interface is not preset to eth0 before dialing
/sbin/route del default gw 192.168.1.1 metric 1
ppp-go 
/etc/ppp/chain start # Turn on IP masquerading
\end{verbatim}

\item lanina is \cmdprn{PPP} dialin host and gateway for home LAN
\begin{verbatim}
# Execute following commands on lanina
# Delete existing Ethernet interface(s), if any
hostname lanina.zender.org
hostname -i -v
/sbin/ifconfig eth0 down 
/sbin/ifconfig eth0:1 down 
# Ensure default route interface is not preset to eth0 before dialing
/sbin/route del default gw 192.168.1.1 metric 1
ppp-go 
/sbin/ifconfig eth0 lanina.zender.org # Connect IP address with Ethernet interface
/etc/ppp/chain start # Turn on IP masquerading
# Execute following commands on dakine
# Point dakine to lanina gateway
/sbin/ifconfig eth0 down 
/sbin/ifconfig eth0 dakine.zender.org
# Route to gateway host all datagrams bound for hosts outside intranet
# fxm: "metric 1" argument appears necessary
/sbin/route add default gw 192.168.1.2 metric 1
\end{verbatim}
These commands plus some other goodies (clock synchronization) are
executed by the \verb'lanina.sh' script.

\item lanina is fixed IP (lanina.ess.uci.edu) running Linux
\begin{verbatim}
# Execute following commands on lanina
# This topology setup works, but lanina does not resolve names on local network:
# ssh dust.ess.uci.edu works but ssh dust does not
hostname lanina.ess.uci.edu
hostname -i -v
/sbin/ifconfig eth0 down
/sbin/ifconfig eth0:1 down 
#/sbin/ifconfig eth0 lanina.zender.org # Connect IP address with Ethernet interface
/sbin/ifconfig eth0 inet 128.200.14.80 # Connect IP address with Ethernet interface
/sbin/ifconfig eth0 netmask 255.255.0.0 broadcast 128.200.14.255
# Creating eth0:1 puts potentially troublesome entries in route table
# These entries may cause problems when machine is again connected to home network
#/sbin/ifconfig eth0:1 lanina.ess.uci.edu # Connect IP address with Ethernet interface
# Route to gateway host all datagrams bound for hosts outside intranet
# fxm: "metric 1" argument appears necessary
/sbin/route add default gw 128.200.14.1 metric 1
\end{verbatim}

\item \mchidx{elnino} is wireless on Harry's Earthlink.net DSL network.
\mchprn{elnino} uses \flidx{eth0} for wired connection and
\flidx{eth1} for wireless.

\item lanina is wireless on Harry's Cox DSL network
Make sure \flidx{/etc/resolv.conf} contains the \mchprn{cox.net} nameserver: 
\begin{verbatim}
nameserver 68.4.16.25
nameserver 68.2.16.30
nameserver 68.5.16.30
\end{verbatim}
\mchidx{lanina} is 192.168.1.73 on Harry's network.
\mchidx{elnino} is 192.168.1.74 on Harry's network.
\mchidx{ashes} is 192.168.1.75 on Harry's network.
Harry's router is 192.168.1.1.
Following lines switch lanina to work on wireless network:
\begin{verbatim}
/sbin/ifconfig eth0 down 
sleep 1
/sbin/ifconfig eth0 add address 192.168.1.73 add netmask 255.255.255.0 
/sbin/route add default gw  192.168.1.1 
\end{verbatim}
These commands plus some other goodies (clock synchronization) are
executed by the \cmdidx{harry.sh} script.
\end{enumerate*}

Harry's DSL server, \mchidx{moon}, is assigned a ``permanent''
\trmidx{IP address} by \mchprn{cox.net}. 
Since I masquerade as \mchprn{moon}, it may be useful to try
to display \trmidx{X}~programs from other machines on my laptop.
\begin{verbatim}
# As of 20050504:
# Hiperstore = 128.195.169.30
export DISPLAY='ip68-109-196-84.oc.oc.cox.net'
# Cliff's house in Lyons as of 20041221:
65.112.206.3
# Harry's Cox DSL
ssh ip68-109-196-84.oc.oc.cox.net # (to moon from outside)
# LGGE Static DHCP
ssh lgge-pc194.lgge.obs.ujf-grenoble.fr # 192.168.102.194 (to virga static IP from inside)
ssh lgge-pc194.obs.ujf-grenoble.fr # 192.168.102.194 (to virga static IP from outside)
ssh lgge-asa50.obs.ujf-grenoble.fr # (virga appears as this from outside)
ssh lgge-pc239.lgge.obs.ujf-grenoble.fr # 192.168.102.239 (Michel Sacchettini's ordinateur)
# SMU ADSL
ssh 62.147.226.50 # (neige appears as this from outside)
# UCI wireless for ashes as of 20050411:
/sbin/route del default gw 192.168.33.1 # Delete gateway
/sbin/route add default gw 192.168.0.1 # Add gateway
kwifimanager # Convenient GUI for WiFI configuration
\end{verbatim}
\mchprn{moon} is visible to the internet, but mine is not since all
IP packets are masquerading as Harry's machine.
Hence two consecutive \trmidx{SSH}'s are required unless \mchprn{moon}
is contacted on \trmidx{port}~261.
This port forwards \trmprn{SSH} requests directly to \mchprn{elnino}'s
DHCP address of \mchprn{192.168.1.226}
\begin{verbatim}
ssh -p 262 68.109.196.84 # Forward to 192.168.1.73:22 (to haze from outside)
ssh -p 263 68.109.196.84 # Forward to 192.168.1.74:22 (to elnino from outside)
ssh -p 264 68.109.196.84 # Forward to 192.168.1.75:22 (to ashes from outside)
ssh -p 265 68.109.196.84 # Forward to 192.168.1.76:22 (to virga from outside)
ssh -p 267 68.109.196.84 # Forward to 192.168.1.77:22 (to neige from outside)
ssh -p 268 68.109.196.84 # Forward to 192.168.1.78:22 (to givre from outside)
ssh -p 262 192.168.1.1 # Forward to 192.168.1.73:22 (to haze from inside)
ssh -p 263 192.168.1.1 # Forward to 192.168.1.74:22 (to elnino from inside)
ssh -p 264 192.168.1.1 # Forward to 192.168.1.75:22 (to ashes from inside)
ssh -p 265 192.168.1.1 # Forward to 192.168.1.76:22 (to virga from inside)
ssh -p 267 192.168.1.1 # Forward to 192.168.1.77:22 (to neige from inside)
ssh -p 268 192.168.1.1 # Forward to 192.168.1.78:22 (to givre from inside)
\end{verbatim}

\mchprn{bonk}'s network sometimes gets into strange states which
dramatically reduce network speed.
If you suspect this might be the case, log into \mchprn{bonk} and do a
\begin{verbatim}
/sbin/ifconfig eth1
\end{verbatim}
If it shows errors, alert Harry.
The fix is often simply a quick \cmdidx{ifdown}/\cmdidx{ifup} cycle.
NCAR standard for routers on each subnet is \mchidx{128.117.xxx.251}.
CGD router is \mchidx{128.117.22.251} or \mchidx{128.117.24.251}.
The default gateway should be the router
The default nameserver (in \flidx{resolv.conf}) should be
\mchidx{bearmtn} = \mchidx{128.117.24.2}. 

\subsection{Amazon Kindle}\label{sxn:knd}
\begin{verbatim}
kiki
Amazon Kindle 3G works Globally, 6" Display with E-Ink Pearl 
Received 20110106
1 Year warranty
Support: www.amazon.com/kindlesupport
E-Mail : kindle-cs-support@amazon.com
Phone: 1-866-321-8851
MAC: 28:ef:01:8d:63:7c
Serial Number: B006A0A00527ECA5
Version: Kindle 3.0.2 (529960105)
Linux 2.6.10
CPU Freescale 532 MHz, ARM-11
4 GB/3.3 GB (Kindle DX) internal flash memory
Display: 6 in diagonal, 3.6 in (91 mm) × 4.8 in (122 mm)
Display: 600 × 800 pixels or 0.48 megapixels, 167 ppi density
Display: 16-level grayscale (Kindle 2 and 3)
Input: USB 2.0 port (micro-B connector)
Input: 3.5 mm stereo headphone jack
Input: AC power adapter jack
Sound: built-in speaker
Connectivity: Amazon Whispernet EVDO/CDMA AnyDATA wireless modem 802.11bg WiFi
Weight: 8.5 oz
Size: 7.5 × 4.8 × 0.34 in (190 × 123 × 8.51 mm) 

# From http://www.amazon.com/gp/help/customer/display.html?nodeId=200505520&#email
"Kindle makes it easy to take your personal documents with you. Each
Kindle has a unique and customizable e-mail address you can set on
your Manage Your Kindle page. You can send unprotected Microsoft Word,
PDF, HTML, TXT, RTF, JPEG, GIF, PNG, BMP, PRC and MOBI files to your
Kindle e-mail address. 

To access your personal documents on your Kindle, send attachments to
your Kindle's e-mail address ("name"@free.kindle.com). The files will
be sent to your Kindle via the Wi-Fi connection and also to the e-mail
address associated with your Amazon.com account at no charge. 

To have your PDF documents converted to Kindle format so you can take
advantage of functionality such as variable font size, annotation,
Text-to-Speech, etc., type "Convert" in the subject of the e-mail when
you submit your personal document to "name"@free.kindle.com. 
Image-heavy PDF files are presented in landscape orientation and don't
work with devices that have auto-rotation, so those will be delivered
in the Kindle format.  You can also transfer personal documents to
your Kindle at no charge using your USB connection. 

In general, send personal documents to your "name"@free.kindle.com
address to wirelessly transfer personal documents to your Kindle over
Wi-Fi as well as to the e-mail address associated with your Amazon.com
account at no charge. 

If you are not able to connect your Kindle via Wi-Fi, send your
documents to your "name"@kindle.com address. The files will be sent to
your Kindle over Wi-Fi if available. If Wi-Fi is not available, the
files will be sent via 3G for a small fee."
\next{verbatim}

\subsection{Sony Cybershot Digital Camera}\label{sxn:cyclops}
\begin{verbatim}
cyclops
Sony Cybershot Digital Camera
Model: DSC-F828, aka the Sony F828
Purchased 2004 before Spain trip
Mfd. Jan, 2004
8 MP image
7x zoom
Zeiss lens
4 Color super HAD CCD
# 20110202 mounting camera is difficult with Ubuntu, works fine with Cygwin
# http://personal.bgsu.edu/~inathan/camera.html

\end{verbatim}

\subsection{Kestrel 4500 w/Bluetooth}\label{sxn:kestrel}
\begin{verbatim}
http://www.amazon.com/Kestrel-Weather-Wireless-Bluetooth-Technology/dp/B0038YQ03Q/ref=sr_1_4?s=sporting-goods&ie=UTF8&qid=1341364153&sr=1-4&keywords=kestrel+4500
http://www.allkestrel.com/Kestrel-4500-BLUETOOTH-Wind-Vane-Tripod-Case-Impeller-BUNDLE-_p_181.html
Cost: $380
Received 20120727 from Amazon.com
S/N: 657271
\end{verbatim}

\subsection{Google/Samsung Nexus S Cell Phone on T-Mobile}\label{sxn:namaste}
\begin{verbatim}

Ruby's iPhone SE
Given by RLZ+CSZ ~20160829
IMEI 89011201000457846929
SIM Card: 
Move from Android: 197369
Order Number: 3731255
Zip Code: 92617
Credo Account Name: Robynn Zender
Credo Password: dakine
Credo Account #: 2109662
Phone #: 9498727505
There is another account name field where RLZ uses her phone number 9493316593 as login
Apple Case #: 100014136491

Ruby's iPhone 4S
Given by Michelle ~20150320
AT&T Unlock request # 2014211552 on 20150322
AT&T Unlock request # 2016400880 on 20150514
Please call 800.331.0500, or dial 611 from your AT&T wireless phone. 
IMEI 013042003116214
Old Phone # 310-951-2534
T-Mobile RSZ account: 9492479870 pw_rby (RSZ)
T-Mobile PIN 2005
Last4 MAZ SSN: 2759

New New Pixel Pixie
Google Pixel XL black
128 GB Flash Drive
Received 20171003, replacement for broken-screen on original Pixie
IMEI: 352693082738101
ICCID: 89011201004007326853
Phone #: 949-891-2429
WiFi MAC Address: ac:37:43:de:0f:08
Bluetooth address: 40:4e:36:42:36:37
Serial number: HT76T0201325
Device Protection: mydeviceprotect.com \$99
Name: Charles S. Zender
Email: CHARLIE.ZENDER@GMAIL.COM
Contract No.: 382328987
Coverage Start Date: 04/28/2017
Coverage End Date: 04/28/2019
To file a claim, visit mydeviceprotect.com or call
1-877-406-4372. We're here to help 24 hours a day, 7 days a week.

New Pixel Pixie
Google Pixel XL very silver
128 GB Flash Drive
Received 20170504
Phone cover (\$30)
Cost new 
IMEI: 352693081912780 (both before and after project fi transition on 20150827)
Phone #: 949-891-2429
WiFi MAC Address: 40:4e:36:1f:23:21
Bluetooth address: 40:4e:36:06:09:bd
Serial number: HT74C0207325
Device Protection: mydeviceprotect.com \$99
Name: Charles S. Zender
Email: CHARLIE.ZENDER@GMAIL.COM
Contract No.: 382328987
Coverage Start Date: 04/28/2017
Coverage End Date: 04/28/2019
Device IMEI or Serial Number: 352693081912780
To file a claim, visit mydeviceprotect.com or call
1-877-406-4372. We're here to help 24 hours a day, 7 days a week.

Great-Great-Great-Grandson of Namaste:
Motorola Nexus 6
32 GB Flash Drive
CaseMate phone cover (\$20)
Cost new \$650+tax=\$701.91 at T-Mobile on Culver (cc_spb)
Received 20150414
IMEI: 355455061181613 (both before and after project fi transition on 20150827)
Phone #: 949-891-2429 (Old T-Mobile number was 949-310-4794)
WiFi MAC Address: 44:80:eb:f0:1b:3a
Bluetooth address: 44:80:eb:f0:1b:39
Serial number: ZX1G22H4XL

Great-Great-Grandson of Namaste:
LG Nexus 5
Cost new \$399+shipping+tax=~\$450 on Google Play
Received 20131127
IMEI: 358239054505161
WiFi MAC Address: 8c:3a:e3:4c:43:57
Bluetooth address: BC:F5:AC:7C:51:01
Serial number: 03b0ad15437f538e
External speaker broke ~201404
20140827: Replaced under warrantee with reconditioned LG Nexus 5:
IMEI: 358239053916872
WiFi MAC Address: 8c3ae3468cf7
Bluetooth address: bcf5ac759e84
Serial number: 03caba9e0947894f
20150401: Microphone stopped working
20150408: Nexus screen stopped working on flight to NYC
https://support.google.com/store/contact/play_c2c_android?rd=1

Olivia bought LG 769 aka Optimus L9 cost new at T-Mobile \$250
Phone died 201404 (due to delayed effects of getting wet?)
Olivia bought LG 769 aka Optimus L9 used from Amazon 20140410 for \$100

Great-Great-Grandson of Namaste (given to Ruby 20131127):
LG Optimus L7
Cost new \$165 on Amazon
Received from Amazon 20131009
EDGE network only? Goof on my part?
LG Model p705 aka 
Actually Optimus L7 refers to both LG Model p700 and p705
My model number is p705 is this an Optimus L7 II? or what?
IMEI: 356536053659812
IMEI SV: 00
Serial number: 
2G Radio Support (MHz): 850, 900, 1800, 1900 MHz GSM//EDGE "Quad-Band" voice
3G Radio support (MHz): 900, 2100 MHz GPRS Class 12/EDGE Class 12/HSDPA Speed 7.2
4G Radio support (MHz): None
First released 201207
Android OS 4.1.2 Jellybean
CPU 1 Ghz Qualcomm Snapdragon MSM7227A
GPU Adreno 200
RAM 512 MBa
Storage 4GB eMMC (2.4 GB user-available)
MicroSD: supports up to 32 GB
Battery 1700 mAh
Display 4.3-inch (480x800) IPS LCD, ~217 ppi
Rear camera 5.0 MP LED flash
Front camera VGA (640x480)
Wi-Fi 802.11 abgn
Bluetooth 3.0 + A2DP
http://www.lg.com/au/mobile-phones/lg-Optimus-L7-P705/technical-specifications
https://en.wikipedia.org/wiki/LG_Optimus_L7
http://support.t-mobile.com/thread/38279?start=0&tstart=0
T-mobile says: "Our 3G network requires two bands in order to
function, 1700MHz and 2100MHz. If both are not present, access will
not be open to the network. In your case you only have the 2100MHz
band so the device will only connect to the 2G network."
This is exactly what happens on Namasté. Dang!
Jives with this chart: https://en.wikipedia.org/wiki/Cellular_frequencies
The 900/2100 bands are the European standard 3G bands. 

Stores pictures in /storage/extSdCard/DCIM

Grandson of Namaste:
Drowned at sea 20131005 while swimming at Crystal Cove
Samsung Galaxy S III (S3) on T-Mobile
Samsung Model #: SGH-T999L
Received as replacement from T-Mobile 20130905 after speakers failed on previous phone
Model is SGH-T999L slightly better than original SGH-T999: It supports LTE
SIM number (T-Mobile USA) 
IMEI number 355880053161333 (obtained by dialing *#06#)
MAC Address: 34:23:ba:3d:c7:f1
Bluetooth address: c0:65:99:80:da:6b
Serial number: r21d794671l
2G Radio Support (MHz): 850, 900, 1800, 1900 MHz GSM/GPRS/EDGE
3G Radio support (MHz): 850, 1700 (AWS/Band IV), 1900, 2100 MHz UMTS/HSPA+/DC-HSPA+
4G Radio support (MHz): 700 (band 17), 1700 (band 4) MHz
Max network speed: 42 Mbit/s DC-HSPA+ and 100 Mbit/s LTE

Son of Namaste:
Samsung Galaxy S III (S3) on T-Mobile
Samsung Model #: SGH-T999
T-Mobile voicemail pw: pyn_mdm
Purchased at T-Mobile 20120905
SIM number (T-Mobile USA) 
IMEI number 353022052928683 (obtained by dialing *#06#)
SIM unlock request 5426756
Rate plan = $64.97/mo for 500 min voice + 5 GB data + unlimited text + $7.99 premium handset protection bundle (worthless?)
UC Institutional discount of 15\% to  : \$.98
Installment pay on monthly bill: Additional $20/mo for 20 months
Total monthly bill 
MAC Address: 5c:0a:5b:ee:1f:a3
Bluetooth address: 1c:66:aa:e6:c0:48
Serial number: r21c81megjm
Billing cycle: starts 2nd of each month
2G Radio Support (MHz): 850, 900, 1800, 1900 MHz GSM/GPRS/EDGE
3G Radio support (MHz): 850, 1700 (AWS/Band IV), 1900, 2100 MHz UMTS/HSPA+/DC-HSPA+
4G Radio support (MHz): None
Max network speed: 42 Mbit/s DC-HSPA+
HSPA connection categories are defined at
http://en.wikipedia.org/wiki/High-Speed_Downlink_Packet_Access
Reception, as determined from Settings->About Device->Status->Mobile network type: 
Croul office usually connects to HSPA: 11, sometimes to EDGE: 2 or to HSPAP: 15
Home  office often connects to HSPA: 11, sometimes to HSPAP: 15 fxm
Status bar reads "4G" in both cases
Support forum at http://support.t-mobile.com/community/phones_data_devices/android/samsung_galaxy_s_iii
Test speeds by visiting www.mobilespeedtest.com
Results from Home with 1 MB test on phone, 2 MB test on laptop, Haggis I 201210:
Namaste T-mobile 2614-2632 kbps, latency 0.278 s
Neige   Cox      5948      kbps, latency 0.219 s
Roulee  Cox      1417-2500 kbps, latency 0.197-0.198 s
Re-test Home  with 1 MB test on phone, 2 MB test on laptop, Haggis II 201307:
Roulee  Cox      4376-4495 kbps, latency fxm s
Results from Croul with 1 MB test on phone, 3 MB test on laptop:
Namaste T-mobile  667-7339 kbps, latency 0.357-0.3999 s
Givre   UCI       6383     kbps, latency 0.201 s
Unlock procedure received on request from T-Mobile 20120909:
Unlock Code:  68520370
0. Please be advised a non T-Mobile SIM card MUST be used to unlock this device.
1. Insert a NON T-Mobile SIM \& turn Device on
2. The Device will automatically prompt you to enter code into an empty field.
3. Click/Tap in the empty field above the "Unlock \& Dismiss" options (this will display a key pad on the screen).
4. Using the key pad input the Unlock Code provided by T-Mobile.
5. Press the Back button to remove the keyboard from screen.
6. Click/Tap the "Unlock" option; handset should display "Network Unlock Successful".
Capabilities:
SoC: Qualcomm Snapdragon S4 MSM8960 (
CPU: 1.5 GHz dual-core Krait (NA and JP versions)
GPU: Adreno 225 (NA and JP versions)
RAM: 2 GB (NA, KR, AU, and JP versions)
Storage: 16 GB flash
External Storage: 64 GB micro SDXC
Battery: 2,100 mAh, 7.98 Wh, 3.8 V, Internal rechargeable li-ion, user replaceable
Data inputs: Multi-touch capacitive touchscreen, 3 push buttons, aGPS,
  GLONASS, Barometer, Gyroscope, Accelerometer, Digital compass
Display: 4.8 in (120 mm) diagonal with 16:9 aspect ratio widescreen
  HD Super AMOLED touchscreen, 1280×720 pixels (306 ppi) and RGBG-Matrix (PenTile)
  Contrast ratio: infinite (nominal) / 3.419:1 (sunlight), 16M colours
Rear camera: 8.0 megapixels back-side illuminated sensor, LED flash
  HD video (1080p) at 30 frames/s, Aperture f/2.6, Autofocus, 
  Zero shutter lag, Simultaneous HD video and image recording,
  Smile and face detection, Image stabilization
Front camera: 1.9 megapixels, Zero shutter lag, HD video (720p) at 30 frames/s
Compatible media formats:
  Audio MP3, AMR-NB/WB, AAC/AAC+/eAAC+, WMA, OGG Vorbis, FLAC, AC-3, apt-X
  Video MPEG-4, H.264, H.263, DivX, DivX3.11, VC-1, VP8, WMV7/8, Sorenson Spark
Ringtones \& notifications: Vibration, MP3, and WAV
Connectivity: 3.5 millimetres (0.14 in) TRRS
  Bluetooth 4.0, Wi-Fi (802.11 a/b/g/n), Wi-Fi Direct, 
  Micro-USB On-The-Go with MHL link, NFC, DLNA, Samsung Kies

Original Namaste:
T-Mobile Nexus S by Samsung
Monthly charges: \$72.98
Samsung Model #: GT-I9020FSTTMB
T-Mobile voicemail pw: pyn_mdm
Purchased at BestBuy Mobile 20110118
SIM number (T-Mobile USA) 8901260432491127503 
SIM number (BSNL India)   fxm
IMEI number 355031040057743
Phone is already ``unlocked''
Rate plan = $50/mo for 1500 min voice + $20/mo unlimited data + $10/mo unlimited text = $80/mo
MAC Address: B4:07:F9:E0:8F:83
Billing cycle: starts the 2nd of each month

Samsung support: +1 855-EZ2NEXUSS (+1 855-392-6398)
T-Mobile International Customer Care (free!): 1-505-998-3793
\end{verbatim}

\subsection{Paywalls}\label{sxn:pyw}
The New York Times (NYT) paywall may be worked-arouned using
instructions found at
\url{http://euri.ca/2011/03/get-around-new-york-times-20-article-limit} 
which instruct users to visit the \texttt{NYClean} web 
\begin{verbatim}
javascript:(function()%20{var%20s=document.createElement('script');s.setAttribute('src','http://toys.euri.ca/nyt.js');document.getElementsByTagName('head')[0].appendChild(s);})();
\end{verbatim}
The Los Angeles Times (LAT) paywall may be worked-arouned using
instructions found at
\url{http://euri.ca/2012/05/bookmarklet-for-the-la-times-paywall}
which instruct users to visit the \texttt{LATClean} web 
\begin{verbatim}
jQuery('.dialog-overlay').hide();
jQuery('.no-close').hide();
jQuery('body').css('overflow', 'scroll');
jQuery('html').css('overflow', 'scroll');
window.scroll=function(a,b){};
\end{verbatim}

\subsection{T-Mobile G1 ``Gphone'' Cell Phone}\label{sxn:g1}
\begin{verbatim}
gel (.ess.uci.edu)
T-Mobile G1
Received at UCI 20090126
Order number 119157060
SIM number 08901260750026442281
IMEI number 358279015293176
Dispatch #: 146677593 (called Damon 20110721) 
Unlock Code for this IMEI: 62471777
Rate plan MY FAVES 300 = $40/mo voice + $25 month data and $12 Govt discount = $55/mo
MAC Address: 001841d47948 
Original contract T-mobile Phone Number 1-949-231-7002 (until 20110314)
New T-mobile pay-as-you-go Phone Number 1-949-302-3146 (since 20110314)
Olivia's Google Voice Number            1-209-560-6548 (560-OLIV in Sutter Creek, CA)

# Resetting phone
Press and hole Home+Power to turn-on
Once ``caution'' icon appears, type Alt-L
Execute factory reset/wipe-data option
Execute re-boot option
Enter new user information (i.e., new Google Gmail account)
NB: 20110314 Google Voice call forwarding does not work with T-Mobile
prepaid phone plans. Results in ``Connection problem or invalid MMI code''

# Power connector
G1 has female modified USB->Mini-B connector with _one_ beveled end
Formally this is known as an "Enhanced mini-USB connector" (EMU)
"Fits" standard USB->Micro-B connectors but does not charge from them

Connectivity 	Wi-Fi (802.11b/g), Bluetooth 2.0+EDR, ExtUSB, A-GPS
Quad band GSM 850 900 1800 1900 MHz GPRS/EDGE
Dual band UMTS 1700 2100 MHz HSDPA/HSUPA (US/Europe) (7.2/2 Mbit/s)

MyFaves and Google Voice:
 "Your five numbers must be US domestic numbers and must not include
... customers' own numbers; and single numbers allowing access to 500
or more persons." 

# Very nice emacs/Android examples
http://riddell.us/tutorial/android/android.html
# Installing Android SDK
# http://www.howtoforge.com/installing-google-android-sdk1.0-on-ubuntu8.04-desktop
sudo aptitude install ia32-libs
sudo aptitude install sun-java6-bin 
# Android SDK 1.0 requires Eclipse 3.3 or 3.4
# http://www.eclipse.org/downloads
sudo aptitude install eclipse
# Learn Eclipse IDE at http://eclipsetutorial.sourceforge.net/index.html
# Console emulator
telnet localhost 5554 # Open console on emulator

# Connectbot SSH usage
1. launch connect bot 
2. menu > manage keys 
3. menu > generate (go through the keygen process) 
4. press and hold on your key in the list 
5. copy signature 
6. log in to your ssh host using normal password 
7. edit .ssh/authorized keys remotely 
8. menu > paste 
8b. (fix line breaks if necessary) 
9. save, logout, disconnect 
10. connect again. you should get in without being asked for a password

# Contact Owner
Use the ``Contact Owner'' application by Paranoid Android
to place your name and address on the screen lock page.
However, this application is unsafe on HTC builds.

# Emacs on Android
http://www.emacswiki.org/emacs/EmacsOnAndroid
http://www.myhangoutonline.com/g1/g1tutorials

# Roaming
Airplane mode means no network connections are possible.
Placing T-Mobile G1 in airplane mode while abroad works fairly well.
I did this on Caribbean Cruise and in Norway/China and Brazil.
If phone loses all power while in airplane mode, it will start-up in
airplane mode next time powered on. This is good. 
Same thing for manual power-downs---it restarts in airplane mode.

Unhappy T-Mobile G1 user writes:
"Absolutely, positively NEVER leave your G1 phone on roaming while
you're in a foreign country. Never EVER leave your phone and connect
via WI-FI while in a foreign country. I made the mistake of leaving my
phone ON when I landed in Mexico. After two weeks, I checked my
T-Mobile online at a local Internet Cafe and was STUNNED at the $950 bill.
T-Mobile also charges you for incoming calls while you're in that
foreign country whether you answer or not -- so every call in is at
the Roaming rate. ~$3 a call.
Take my advice. Leave the G1 at home."

# Billing
Google Voice calls to T-Mobile G1 appear from "Blocked Number" on bill
Phone minutes reset on the 11th of each month

# Stereo Bluetooth headphones
Motorola S705 Soundpilot Stereo Headset, mfg. c. 2006
Supports A2DP and AVRCP and Bluetooth v1.1 and v1.2 phones
NB: BT-1.2 is old, c. 2003.
BT specification in 2009 is version 3.0.
Passkey 0000

# Mono Bluetooth phone earpiece
Motorola H700 (gift from Mom) headset microphone, mfg. c. 2006
EMU (mini USB) connector
Type SYN1311B, i.e., "B"-version with ~30 ft range, works with CDMA 
950 mAh Lithium Standard Cell Phone Battery

Motorola customer service database says of both the H700 and s705,
"Can I replace the headset battery?"
"The battery is embedded in the headset and cannot be replaced."

Replace battery on H700:
"I used a small 'brand new' standard straight and one sided razor
blade. There is an even slit all around the two pieces of the ear
piece. On the bottom side, where the boom side of the microphone is
attached is where you put the sharp side of the razor and slide it in
between the slit sides. I wiggled it a little bit, then slide the
razor around to the edge portion that is slightly thicker and twisted
the razor slightly back and forth within that slit. It took less than
30 seconds and the unit split open with no visible damage, or
marks. At this point Do *NOT* Force the unit completely open!!! There
is a small ring at the Mini-USB port, which surrounds that plug. To
get it out so the unit will open completely, you need to take a small
flat head screw driver and insert it on the inside portion of that
ring. You can see a small notch on the inside of it. Push under the
ring where the notch is and push it forward, outward from the unit and
it will 'pop out'. That's it. The rest is self explanatory I think. Be
really careful, as the tiny wires will break if you push, twist, or
pull on them!" 
\end{verbatim}

\trmidx{Sipdroid} instructions (Free \trmidx{SIP} calls)
\begin{enumerate}
\item Install sipagent
\item User (use your gizmo login id name)
\item Password (use your gizmo password)
\item domain/server: proxy01@sipphone.com
\item Lock wifi and/or enable 3G
\item Install googlevoice callback
\item Enter your gizmo number (747-XXX-XXXX)
\item Make sure it says gizmo on dropdown
\item Go to Google Voice and enter your Gizmo number in the settings as forwarding number to receive your incoming calls
\item Setup Google Voice callback to be default to make outgoing calls
\item There you have it, free incoming and outgoing phone calls
\end{enumerate}

\subsection{IISc}\label{sxn:iisc}
The \trmidx{proxy} server for accessing the web via a browser or
an e-mail client inside the firewall is 
\begin{verbatim}
# IISC-wide firewall/proxy server
iiscproxy.serc.iisc.ernet.in # Proxy server on port 3128
\end{verbatim}
which uses port~\texttt{3128}.
IISc has a draconian firewall.
Invoke \cmdidx{corkscrew} from \flidx{~/.ssh/config} 
to tunnel through the firewall:
\begin{verbatim}
# Purpose: SSH ~/.ssh/config to tunnel through firewalls
# 20110126 csz corkscrew works in Bengaluru at IISc Guest House
# 20110126 csz corkscrew does not help at IISc CAOS
# http://www.agroman.net/corkscrew
Host *
  ProxyCommand corkscrew iiscproxy.serc.iisc.ernet.in 3128 %h %p
\end{verbatim}

\subsection{LGGE}\label{sxn:lgge}
The external web homepage at the
Laboratoire de Glaciologie G\'{e}ophysique de l'Environnement 
(\trmidx{LGGE}) in Grenoble, France is 
\url{http://www-lgge.ujf-grenoble.fr/eng}.
\trmprn{LGGE} is cordoned off by a firewall.
The outside-facing LGGE SSH server is
\mchidx{triolet.obs.ujf-grenoble.fr}. 
The two-step hop to log into \mchidx{ashes} from outside the LGGE
firewall is  
\begin{verbatim}
ssh triolet.obs.ujf-grenoble.fr
ssh lgge-pc194 # Virga static IP
ssh 192.168.102.194 # Virga static IP
ssh 192.168.102.10 # Virga dynamic DHCP (or .10, .19, ...)
ftp ftp # Triolet is also ftp server: login to ftp as zender
ftp ftp # Triolet is also ftp server: login to ftp as anonymous
cd pub/depot/zender
ftp://ftp/pub/depot/zender # Inside IP
ftp://ftp.lgge.obs.ujf-grenoble.fr/pub/depot/zender/ # Outside IP
\end{verbatim}
The automatic \trmidx{proxy} configuration file for accessing the web
via a browser inside the firewall is
\begin{verbatim}
http://proxy.ujf-grenoble.fr/auto-proxy.pac
\end{verbatim}

The internal \trmprn{LGGE} homepage is 
\url{http://www} (yes, you read that correctly).
The \trmprn{LGGE} homepage for Linux is:
\url{http://lgge-pc240}
and for printer configuration is
\url{http://lgge-pc240:631/printers}.
My USA/UCI printer configuration file is in
\flidx{/etc/cups/cupsd.conf.bak}.

\trmprn{LGGE} supports NFS-mounting the Linux partitions
on \mchidx{lgge-pc240}:
\begin{verbatim}
ls /mnt
mkdir /mnt/lgge-pc240
mount lgge-pc240:/usr/local /mnt/lgge-pc240
ls -l /mnt/lgge-pc240/bin/ncbo
/mnt/lgge-pc240/bin/ncbo -r
\end{verbatim}

Pour imprimer à partir du poste linux:
imprimante Noir et Blanc
HP laserjet 4250 au 1er étage couloir (format papier A4
recto/verso-duplex): 192.168.101.39 
Photocopieuse Canon 3570 deuxieme étage salle reprographie: (format A4
et A3 recto/verso-duplex): 192.168.101.36 

Imprimante couleur:
Dell 3110CN deuxieme étage salle reprographie: (format A4
recto/verso-duplex): 192.168.101.42 

\subsection{Router}\label{sxn:rtr}
Whistlernet routers: \trmidx{Linksys Wireless-G router WRT54G}
Install \trmidx{HyperWRT} Linux router distribution from 
\url{http://www/hyperwrt.org}.
\begin{enumerate}
\item Connect ethernet of laptop to router
\item Log into router. Login,password when shipped is admin,admin
\item Destination is http://192.168.1.1
\item Router Name is zender-WRT54G
\item Local Address is 192.168.1.5
\item Internet Address is 192.168.1.6
\item Subnet Mask is 255.255.255.0
\item Enable DHCP server
\item Starting IP address: 192.168.1.160
\item Ending IP address: 192.168.1.179
\item Static DNS: 128.200.1.201, 64.4.16.25, 
\item Router password: dakinenet
\item Wireless Network Name (SSID): zendernet
\item Setup to use MAC-based filtering
\item Startup script for each router to bond to eachother:
\begin{verbatim}
/usr/sbin/wl lazywds 0
# Give Mangonet MAC to zendernet
/usr/sbin/wl ws 00:0F:66:8E:B6:D2
\end{verbatim}
\end{enumerate}

\subsection{Sound}\label{sxn:snd}
Audio can be one of the most pleasant features of a computer,
or its most annoying.
Usually Linux does recognize the special keys many keyboard
manufacturers have added to automatically control audio features.
In public places such as planes, meetings, and libraries, it is often
very desirable to turn off not just the sound server, but all audio
beeps. 
For this purpose, the \cmdidx{xset} command does very nicely:
\begin{verbatim}
xset b 0 # Turn off sound
xset b off # Turn off sound
# Set beep parameters: % maximum volume, pitch (Hz), duration (ms)
xset b 20 440 50
\end{verbatim}

\url{http://aries17.uwaterloo.ca/~dmg/brick} suggests adding
\begin{verbatim}
alias sound maestro
options sb io=0x220 irq=5 dmg=1
\end{verbatim}
to \flprn{/etc/conf.modules} for Dell Inspiron 7500.

\begin{verbatim}
# Convert MPEG to AVI
# http://ubuntuforums.org/showthread.php?t=456035
mencoder dvd://1 -dvd-device /dev/dvd -alang English \
-info srcform="DVD ripped by acidrip.sf.net" \
-oac mp3lame -lameopts abr:br=128 \ # Try 
-ovc lavc -lavcopts vcodec=mpeg4:vhq:v4mv:vqmin=2:vbitrate=1000 \
-vf pp=de,crop=0:0:0:0,scale=480:-2 \
-o '/data/zender/vid/dolphin.avi'
\end{verbatim}

\begin{verbatim}
# Convert Ogg to MP3
# Use SoundConverter GUI, or...
# Method fails on strange characters and whitespace in filenames
for fl in *.ogg;do oggdec -o - "$fl"| lame -h -V 4 --vbr-new - "$(basename "$fl" .ogg).mp3";done
# Convert MP3 to Ogg
mpg321 foo.mp3 -w - | oggenc -o foo.ogg -
# Convert Ogg to WAV
oggdec -o foo.wav foo.ogg
# Specify high-quality encoding averaging 256 kbps (but still VBR)
oggenc foo.wav -b 256 -o out.ogg
# All at once:
mpg321 foo.mp3 -w - | oggenc -b 96 -o foo_096kbps.ogg -
# find | perl solves problem when whitespace gets split into different files:
find . -maxdepth 1 -name '* *' | perl -e 'while(<STDIN>){chop;$fl_in=$_;s/ /_/g;$fl_out=$_;printf STDOUT ("mv %s %s\n",$fl_in,$fl_out);`mv "$fl_in" $fl_out`}'
for fl in `ls *.mp3`; do echo "$fl";done
for fl in `ls *.mp3`; do mpg321 ${fl} -w - | oggenc -b 96 -o ${fl/mp3/ogg} -;/bin/rm ${fl};done
# Convert an entire 256 kbi/s MP3 discography to 128 kbi/s Ogg, keeping filenames
find . -maxdepth 2 -name '*mp3' | perl -e 'while(<STDIN>){chop;$fl_in=$_;s/mp3/ogg/g;$fl_out=$_;printf STDOUT ("mpg321 %s -w - | oggenc -b 128 -o %s -\n",$fl_in,$fl_out);`mpg321 '$fl_in' -w - | oggenc -b 128 -o '$fl_out' -`}'
\end{verbatim}

Sometimes the sound device is controlled by processes that
do not play well with others.
One symptom of this is when \cmdidx{amarok} complains that
the sound device ``is unavailable''.
When this occurs, try to kill the errant processes revealed
to be controlling the sound device.
To find those processes, follow these steps:
\begin{enumerate}
\item Open the system monitor: System->Administration->System Monitor
\item Open in the menu: Monitor->Search for open files
\item In the text box, enter: /dev/snd/* then click on Find
\end{enumerate}
On 20080325, \cmdidx{kpdf}, of all things, was responsible for
stealing the sound outputs.

\subsection{dakine = msw, lanina = linux}\label{sxn:smb}
List dakine msw shares accessible via Samba on lanina
lanina: smbclient -L dakine
Work interactively in FTP-like environment with msw shares
lanina: smbclient \verb'\\dakine\home'
Ability to browse Linux from Windows is determined by /etc/smb.conf:
\begin{verbatim}
   hosts allow = 192.168.1. 128.200.14. 127. 128.117.24.
[printers]
   browseable = yes
 [] 
\end{verbatim}
Specifically, you must set up encrypted passwords by using the line
"encrypt passwords = yes" in your smb.conf file, and you must create an
appropriate smbpasswd file

\subsection{DHCP}\label{sxn:dhcp}
\trmidx{Dynamic Host Configuration Protocol} allows computers to
configure as an Internet node without a static \trmidx{IP address}.
DHCP clients receive their IP address from the DHCP server.
The network configuration files must be altered to enable DHCP.
On \trmidx{Debian} systems, this is configured in
\flidx{/etc/network/interfaces} with a line for the appropriate
interface. 
For example, to configure \cmdidx{eth0} (usually the wired
\trmidx{ethernet} connection) for DHCP, use
\begin{verbatim}
iface eth0 inet dhcp
\end{verbatim}

The \cmdidx{pump} command may work when \trmidx{DHCP} does not.
\begin{verbatim}
/sbin/pump -i eth0 # 
/sbin/pump -i eth1 --status # Inquire settings 
\end{verbatim}
Note, however, that \cmdprn{pump} may corrupt the \trmidx{ARP} table
(Section~\ref{sxn:arp}).

\subsection{Wireless}\label{sxn:wrl}
Wireless is a PITA.
If the wireless connection managing software (e.g., \cmdidx{wicd},
\cmdidx{network-manager}) does not automatically connect to the
network, give up or expect to spend many hours debugging the problem.

For starters, make sure the WiFi indicator light is working.
On \mchidx{neige}, this light is next to the blue Bluetooth light.
On most Dell laptops, it is possible to inadvertently turn-off WiFi 
by accidentally hitting the Function-F2 keyboard combination.

One annoying aspect of wireless is the detail mismatch between
information that hotspots (e.g., hotels) provide, and the information
that the management software needs.
There are at least ten different security conventions in use:
\begin{enumerate*}
\item Wired Equivalent Privacy (\trmidx{WEP});
\item WPA with (passphrase);
\item WPA 1/2 (pre-shared key);
\item WEP (Hex 0-9,A-F);
\item WEP (passphrase);
\item WEP (shared/restricted);
\item LEAP with WEP (needs Username and Password); 
\item TTLS with WEP (needs Identity, Password, Authentication);
\item EAP-FAST (needs Username, Password and Path to PAC File);
\item PEAP with GTC (needs Identity and Password);
\item PEAP with TKIP/MSCHAPV2 (needs Identity, Password, Path to CA Cert);
\item EAP-TLS (needs Identity, Private Key, Private Key Password, Path to CA
  Cert, Path to Client Cert)
\end{enumerate*}
\begin{verbatim}
# Virga procedure: 
# ifdown eth1;iwconfig eth1 essid "UCInet Mobile Access";ifup eth1
# ifdown eth1;iwconfig eth1 essid "House of Haggis";ifup eth1
# Neige procedure: 
# ifdown wlan0;iwconfig wlan0 essid "House of Haggis";ifup wlan0
# ifdown wlan0;iwconfig wlan0 essid "House of Haggis" ap 00:0F:66:8E:B6:D3;ifup wlan0 # Irvine mangonet
# Ashes procedure:
# iwconfig eth2 rate 54M # Increase speed on internal broadcom chip
# iwconfig eth2 rate 5.5M # Increase speed on internal broadcom chip
# Alphabetical by city name
iwlist wlan0 scan # List available hotspots
iwconfig wlan0 essid any key off # Reset wireless interface
iwconfig eth1 essid 313a2a ap 00:02:2D:31:3A:2A # Albuquerque taylornet
iwconfig eth1 s:13a2a # Albuquerque taylornet
iwconfig eth1 essid 2WIRE823 s:2449922649 # Newport Beach Sharinet 20140605
iwconfig eth1 essid Jambo key s:jambojambojambo # Anchorage Lynnnet 20140513
iwconfig eth1 essid ESDSWG_1 key s:loews123 # Annapolis ESDSWG
iwconfig eth1 essid sid_ciencieswl # Barcelona UAB ICTA
iwconfig eth2 essid IAMAS-2005 # Beijing IAMAS
iwconfig wlan0 essid iiscwlan key s:atmos54321 # Bengaluru IISC
iwconfig eth1 essid UCAR # Boulder NCAR/UCAR (heartbeat on wireless.ucar.edu)
iwconfig eth1 essid UCAR ap 00:1B:2A:64:F1:50 # Boulder NCAR Director's Conference Room
iwconfig eth1 essid UCAR ap 00:1B:2A:64:F3:40 # Boulder NCAR Chapman Room
iwconfig eth1 essid UCAR ap 00:1B:2A:64:F3:60 # Boulder CMS offices
iwconfig wlan0 essid UCAR ap 00:19:2F:AD:E6:80 # Boulder UCAR Center Green 
iwconfig eth1 ap 00:18:F8:FA:BD:46 # Greenbelt Marriott EOS Aerosol Cloud Review
iwconfig wlan0 essid ibahn_conference # Greenbelt Marriott EOS MAP CAR Review
iwconfig wlan0 essid ibahn_conference # Greenbelt Marriott EOS MAP CAR Review
iwconfig eth1 essid COMINN key s:wificomfort38 # Grenoble Comfort Inn
iwconfig eth1 essid U1-OSUG-LGGE ap 00:16:9D:73:D0:60 # Grenoble LGGE
iwconfig eth1 ap 00:0F:24:F1:89:B0 # Irvine 1103 Croul
iwconfig eth1 ap 00:0F:24:F1:89:40 # Irvine 3101 Croul
iwconfig eth1 ap 00:0F:24:F1:87:10 # Irvine 3228 Croul
iwconfig eth1 ap 00:0F:24:F1:86:90 # Irvine 3103 Croul
iwconfig eth1 essid "UCInet Mobile Access" # Irvine UCInet access points
iwconfig eth1 essid panera # Irvine Panera
iwconfig eth1 ap 00:0F:66:8E:B6:D3 # Irvine mangonet
iwconfig eth1 essid "House of Haggis" ap 00:0F:66:8E:B6:D3 # Irvine mangonet
iwconfig eth1 essid "Hale_La'_Aloa" key 6efd8c1bee # Kona
iwconfig wlan0 essid "Free Public WiFi" ap F6:92:50:17:0C:50 # Los Angeles LAX Gate 71 Terminal 7 (United)
iwconfig wlan0 essid tmobile ap 00:12:DA:94:48:D0 # Los Angeles LAX Gate 71 Terminal 7 (United)
iwconfig eth1 essid linksys ap 00:18:39:87:E1:60 # Lyons cliffnet
iwconfig wlan0 essid linksys_rufenach ap 00:18:39:87:E1:60 # Lyons cliffnet 20081012
iwconfig wlan0 essid linksys_rufenach ap 00:12:17:F6:7E:2F # Lyons cliffnet 20101003
iwconfig wlan0 essid myqwest5253 pw:  FF566275FFAB4378D5B3F9E680 # Lyons cliffnet 20110813
iwconfig eth1 essid "Blue Tile Beach House" key s:beach # Maui
iwconfig eth1 # New York City lizzienet "Home Network" "welcomeguest"
iwconfig eth1 essid "casa caribena downstairs" login lessard pw angels # Tankah Casa Caribeña
iwconfig eth1 essid "casa caribena upstairs" login lessard pw angels # Tankah Casa Caribeña
iwconfig eth1 essid Hyatt login hyattregency password newportbeach # Newport Beach Hyatt Regency
iwconfig eth1 essid Guest # New London
iwconfig eth1 essid NICEXPO # Nice EGU
iwconfig eth1 essid EGU2004-1 # Nice EGU 192.168.33.181
iwconfig eth1 essid UCOP logyn "ucop guest" password welcome # Oakland UCOP
iwconfig eth1 ap 00:02:72:59:D9:0A key 1122334455 # Oslo Carlton Hotel
iwconfig eth1 essid IUGG-R # Perugia IAMAS Garden iuggpg4015, X/cp+a4C
ifconfig eth0 key s:guest_cpt # Pune IITM (guest, guest_cpt)
iwconfig eth1 essid 2WIRE467 # Sacramento curtisnet
iwconfig eth1 essid "Free Public WiFi" ap 02:18:DE:04:D0:48 # San Antonio airport (SAT)
iwconfig eth1 essid WiFi4public ap 00:90:0E:00:32:C1 # San Antonio airport (SAT)
iwconfig eth1 essid "Public Access Wifi" ap 00:0D:67:00:5E:61 # San Antonio Menger Hotel
iwconfig wlan0 essid linksys ap 00:18:39:87:E1:5D # San Miguel de Allende cliffnet
iwconfig eth1 essid ATT160 key 4445728512 # Sutter Creek lindanet
iwconfig wlan0 essid any key off # Reset wireless interface
iwconfig wlan0 essid "Amalie Hotell" ap 00:40:05:5E:F0:F5 # Tromsø Hotel Amalie
iwconfig wlan0 essid any key off # Vienna Arcotel Kaiserwasser
iwconfig wlan0 essid ACV2 ap 00:17:65:8B:99:C4 # Vienna EGU
iwconfig wlan0 essid ACV2 # Vienna EGU
iwconfig wlan0 essid "Wireless Vienna Airport" ap 00:07:85:B3:D4:C4 # Vienna airport
sudo cp ~/linux/etc/network/interfaces.minimal /etc/network/interfaces
/etc/init.d/networking restart
iwconfig wlan0 essid CNRS key 4a7935d567 # Annecy IGAC 4A7935D567
\end{verbatim}
The Intel \cmdidx{ipw3945} driver does not automatically associate
with the best quality network available, e.g., from multiple UCInet
access points. 
You may configure a wireless interface to try access points in a
certain order specified in file \flidx{/etc/wpa\_supplicant.conf},
e.g., 
\begin{verbatim}
% cat /etc/wpa_supplicant.conf
ctrl_interface=/var/run/wpa_supplicant
network={
        ssid="name"
        psk="pass"
}
% cat /etc/network/interfaces 
allow-hotplug eth1
iface eth1 inet dhcp
wpa-conf /etc/wpa_supplicant.conf
\end{verbatim}

\subsection{ARP Address Resolution Protocol}\label{sxn:arp}
The \trmdfn{Address Resolution Protocol} (\trmidx{ARP}) table maps
ethernet addresses to \trmidx{IP address}es. 
Examine the ARP table contents (BuH98 p.~173) with
\begin{verbatim}
arp -a 
\end{verbatim}

Sometimes the ARP table contains duplicate or ill-defined entries.
For example, \cmdidx{pump} may corrupt the ARP table.
One symptom of this is when \trmidx{ssh} to a working host fails with,
e.g.:
\begin{verbatim}
ssh: connect to host ashes.ess.uci.edu port 22: No route to host
\end{verbatim}
When this occurs, the \textit{first thing to check} is that the
\cmdidx{sshd} \trmidx{daemon} is installed and running.
Many security-conscious operating systems ship by default with 
very few services running.
Another cause of this message is incorrect network configuration.
The suspect device is \flidx{eth0} or \flidx{eth1}.
Ensure that only one \trmidx{Ethernet} device is configured.
The best solution is to remove the offending entry: 
\begin{verbatim}
sudo arp --delete ashes.ess.uci.edu
\end{verbatim}
A simple but dangerous fix is to bring down (and then up) the
offending device. 
This does not work remotely since the machine will be off the
network. 

\cmdidx{ssh} and \cmdidx{scp} will also fail with the error
\begin{verbatim}
ssh: connect to host ashes.ess.uci.edu port 22: No route to host
\end{verbatim}
when the \cmdidx{sshd} server is not running.
To check whether \cmdprn{sshd} is running, use \cmdidx{ps}.
To check wheter \trmidx{port~22} is open for \trmidx{SSH} connections,
use the \cmdidx{nmap} command:
\begin{verbatim}
ps -ef | grep sshd
sudo nmap -sS -F localhost
\end{verbatim}

\subsection{Power}\label{sxn:power}
A helpful primer on reducing power usage by Linux is at
\url{http://www.lesswatts.org/tips/disks.php}.

The \trmidx{Intel} \cmdidx{powertop} command gives useful hints to
reduce laptop power usage.
Suggestions (for \mchidx{neige} and \mchidx{virga}) have included:
\begin{verbatim}
powertop # Run as root with laptop unplugged

# Following tips are annotated to tell which machines they work on

# Spin-down hard disk: This may shorten disk life
# Data transfer becomes more bursty so hard disk can power down for longer
# Do not add this to /etc/rc.local
# Works on: all
echo 5 > /proc/sys/vm/laptop_mode # Set laptop mode

# Disable unusedhttp://www.lesswatts.org/tips/disks.php bluetooth
# Do not add this to /etc/rc.local (you will forget that you did)
# Works on: all
hciconfig hci0 down ; rmmod hci_usb

# Extend time betwee dirty page flushes from 5 to 15 s
# Works on: all
echo 1500 > /proc/sys/vm/dirty_writeback_centisecs

# Disable CDROM polling
# Works on: all
hal-disable-polling --device /dev/scd0

# Reduce nvidia driver interrupts: Seems safe enough
# Add following to driver section of xorg.conf then restart X
# May not be necessary with heron
# Minimal wakeups using 3d nvidia driver is 60 Hz (once per horizontal scan)
# Works on: neige
Option "OnDemandVBlankInterrupts" "True" 

# Turn off unused wireless
# Works on: virga
echo 1 > /sys/bus/pci/devices/0000:0c:00.0/rf_kill

# Mount drives with noatime
# Works on: virga
mount -o remount,noatime /

# SATA controllers use the AHCI specification that has Aggressive Link
# Power Management (ALPM) option
# Works on: neige
echo min_power > /sys/class/scsi_host/host0/link_power_management_policy
\end{verbatim}
Applying these suggestions reduce power usage on \mchprn{virga} by
10\,W! 

\subsection{Hibernate}\label{sxn:hibernate}
Laptops these days have various power consumption options controlled
by the Linux \trmidx{Advanced Configuration and Power Interface}
(\trmidx{ACPI}). 
The states are standby, suspend, and hibernate.
The \trmidx{standby} action causes the screen to go blank.
The \trmidx{suspend} action invokes ACPI state~S3, aka
\trmidx{suspend-to-RAM}.
According to
\href{http://www.linux.com/article.pl?sid=06/05/24/1716222}{linux.com},
``everything in the system enters a low-power state except for RAM,
which consumes a small amount of power in order to retain its
contents, so that upon resuming, everything is loaded back from the
memory and all running applications are restored immediately.''
According to the article, the following should be in the
device section of \flidx{xorg.conf}:
\begin{verbatim}
       Option          "VBERestore"    "true"
\end{verbatim}
The suspend action may be invoked by, e.g., the \flidx{suspend.sh}
script in the article.

The \trmidx{hibernate} action saves the state to disk and powers down.
According to
\href{http://www.linux.com/article.pl?sid=06/05/24/1716222}{linux.com},
hibernate is also known as ACPI State S4 or \trmidx{suspend-to-disk}.
On reboot, the kernel senses that a saved state is available and
reloads it. 

Experimented with ACPI on Ubuntu~6.06 LTS on 20060604 by connecting
the actions to the lid shut button action. Results follow.
Suspend on \mchidx{ashes}: blanks screen and turns off wireless.
With \trmidx{nv} driver, lid-up restores without problems except must
manually restart wireless interface. 
With \trmidx{nvidia} driver, lid-up get that black melty look and must
manually re-boot.
Also, both \trmidx{nv} and \trmidx{nvidia} driver intermittently
(after a few hours) may reboot on ashes.
This may indicate an \flidx{xorg.conf} problem.
However \trmidx{nv} driver seems to have problems with translucency.
In fact, translucency could be a problem with \trmidx{nvidia} as well.

Hibernate on \mchidx{ashes}: With \trmidx{nv} driver, saves session to
disk and powers down. 
Session restores correctly on re-boot, except must manually restart
wireless interface. 
With \trmidx{Nvidia} driver, nothing happens---no shut down at all.

Suspend \ldots on \mchidx{virga}: works perfectly with
\cmdidx{suspend.sh}! 
Hibernate \ldots on \mchidx{virga} works perfectly after modifying
\flidx{xorg.conf} to allow session restores.
Only checked \mchprn{virga} with \trmidx{NVidia} driver, not sure if
suspend and hibernate work on \mchprn{virga} with \trmidx{nv} driver.

The \cmdidx{i8kutils} package provides a number of Inspiron-specific 
control commands including \cmdidx{i8kmon}, \cmdidx{i8kctl},
\cmdidx{i8kbuttons}. 
These appear to work when the kernel has loaded the \cmdidx{i8k}
kernel module, e.g., with 
\begin{verbatim}
insmod /lib/modules/2.6.15-23-686/kernel/drivers/char/i8k.ko
\end{verbatim}

The \cmdidx{gkrellm} utility will graphically display output from
the \cmdidx{i8kutils} package.
\begin{verbatim}
sudo aptitude install xosview hddtemp gkrellm i8kutils gkrellm-i8k
# To load module automatically add "i8k force=1" to end of /etc/modules
sudo emacs -nw /etc/modules
sudo modprobe i8k # Restart or modprobe to load module
\end{verbatim}
See
\href{http://ubuntuguide.org/wiki/Ubuntu:Feisty_ja/Hardware#How_to_monitor_CPU.2C_GPU_temperatures.2C_fan_speeds_and_voltages_.28GKrellM.29}{here}
for more details.

This thread
\url{https://launchpad.net/distros/ubuntu/+source/util-linux/+bug/66637} 
describes how to fix broken swap partitions with Ubuntu Edgy.
\begin{enumerate*}
\item Determine your swap with \cmdidx{fdisk -l}
\item Run \cmdidx{mkswap} on your swap partition and record the output \trmidx{UUID}.
\item Put this UUID into \flidx{fstab}.
\item Put \texttt{RESUME=UUID=<the-swap-partition-uuid-from-vol\_ID} into \flidx{/etc/initramfs-tools/conf.d/resume}
\item Run \cmdidx{update-initramfs -u}
\item Reboot normally after this finishes
\item Run \cmdidx{swapon -s} to check if your swap is active.
\item Run \cmdidx{ls -la /dev/disk/by-uuid/}
\end{enumerate*}
Do not change any symlinks. Just try these instructions.

\subsection{PCMCIA}\label{sxn:pcmcia2}
Get rid of annoying beep when PCMCIA is inserted
Add to /etc/sysconfig/pcmcia:
\begin{verbatim}
CARDMGR_OPTS = -q 
kill -HUP cardmgr_pid
\end{verbatim}
Use \cmdidx{cardctl} to notify system of PCMCIA card status changes.
To suspend, eject, or resume the card, use
\begin{verbatim}
cardctl suspend # Shut down, disable power for socket
cardctl eject # Notify client drivers card will be ejected, cut power
cardctl resume # Restore power to socket, re-configure for use
cardctl reset # Send reset signal to  socket
cardctl status # Display current socket status flags
cardctl ident # Display card identification information
cardctl config # Display the socket configuration
\end{verbatim}
To restart the entire pcmcia subsystem, or to remove the eth0 interface, use
\begin{verbatim}
/sbin/ifconfig eth0 down
/etc/rc.d/init.d/pcmcia restart
/sbin/ifconfig eth0 up
\end{verbatim}
To trace shell execution use, e.g.,
\begin{verbatim}
sh -x /etc/sysconfig/network-scripts/ifup ifcfg-eth0
\end{verbatim}

\subsection{core dumps}\label{sxn:core}
RH6.1 default \verb'/etc/profile' uses the \cmdidx{ulimit} command 
to limit size of core files with \cmdprn{ulimit -c 1000000}. 
This causes an error on telnet logins so comment it out.

\subsection{Printing}\label{sxn:prn}
\trmidx{RedHat} causes spool directories in \verb'/var/spool/lpd' to
be owned by root with group \cmdidx{lp}.
If this is not the case then printing may fail with permission denied errors
comp.os.linux.setup suggested adding to \verb'/etc/conf.modules'
\begin{verbatim}
alias parport_lowlevel parport_pc
\end{verbatim}
This appears to fix printer queue ``permission denied'' errors on dakine
But this is untested on lanina, which has given parport errors on
bootup in the past.

If lanina is booted up unconnected to the printer, and then the 
printer is connected later and printing is attempted, an error may
result such as 
\verb'2000-12-30-17:13:18.388'
\verb+Get_local_host: hostname 'lanina.zender.org' bad+.
When this occurs printing will fail and restarting the print daemon
will also fail.
The solution is simply to rename the machine, e.g.,
\verb'sudo hostname lanina' and then restart the printer daemon 
and then printing should succeed.

\subsubsection{CUPS}\label{sxn:cups}
\trmdfn{CUPS} is the Common Unix Printing System.
Configure CUPS by pointing a web browser to \url{http://localhost:631}.
The HP4600 network color laser printer is configured to accept IPP
as \url{http://hp4600.ess.uci.edu:631/ipp/port1}.
The HP4350 network laser printer is configured to accept IPP
\url{http://192.168.14.2:631/ipp/port1}.
The Tek850 color printer is \mchidx{tek850.ess.uci.edu} = 128.200.24.133.
The HP6840 color printer is \mchidx{hp6840.ess.uci.edu} = 128.200.24.134.
The most informative site for \trmidx{printing} on Linux is 
\url{http://www.linuxprinting.org}.
They sponsor a \href{linuxprinting.hp.general}{newsgroup} with 
an \trmidx{email gateway} for virtually all types of printer,
especially \href{hp-list@linuxprinting.org}{HP printers}.
\href{http://www.linuxprinting.org/newsportal/thread.php3?name=linuxprinting.hp.general}{Web Portal}
to \trmidx{HP printers} may be useful.
\begin{verbatim}
aptitude install cupsomatic-ppd # Install latest CUPS ppd drivers
killall -HUP cupsd # Restart CUPS daemon
/etc/init.d/cupsys restart # Restart CUPS daemon
lp ~/.plan # Print file to default printer
lp ~/.plan http://hp4600.ess.uci.edu:631/ipp/port1 # Specify printer
lp ~/.plan http://xerox.ess.uci.edu:631/ipp/port1 # Specify printer
lp ~/.plan http://tek850.ess.uci.edu:631/ipp/port1 # Specify printer
lp ~/.plan http://hp6840.ess.uci.edu:631/ipp/port1 # Specify printer
# Xerox drivers
# http://www.office.xerox.com/perl-bin/opb_drivers.pl
lp ~/.plan http://xerox.ess.uci.edu # Xerox Phaser 4500 is 128.200.24.127 
lp -o docs ~/.plan # Print printer options to printer
lp -dhp4600/hp4600t -oraw # GIMP2 uses this by default
lpr -z InputSlot=Tray1 # CUPS command line option to send to Tray 1
lpr -P XeroxPhaser4500 ~/.plan # Xerox Phaser 4500 is xerox.ess.uci.edu = 128.200.24.127 
lpr -P hp4600 ~/.plan # HP4600 is hp4600.ess.uci.edu = 
\end{verbatim}
Enabling extra printers in \trmidx{OpenOffice.org}
\begin{verbatim}
sudo ln -s /usr/lib/openoffice/program/spadmin /usr/local/bin/spadmin
PATH=${PATH}\:/usr/lib/openoffice/program
/usr/lib/openoffice/program/spadmin
# Set printer to standard input
kprinter --stdin
\end{verbatim}

From the CUPS documentation 
\href{http://www.cups.org/articles.php?L271+THow-To+Q}{here},
there are two ways to add a PPD file:
\begin{verbatim}
sudo lpadmin -p printer -E -v device-uri -i Dell_5130cdn.ppd # Use lpadmin ... OR ...
sudo cp Dell_5130cdn.ppd /usr/share/cups/model # Then restart cupsd
\end{verbatim}

The Xerox Phaser 4500 printer is set up as 
\begin{verbatim}
Xerox/Docuprint 4508
lpd://xerox.ess.uci.edu:515
Add port 515 (LPD port) to /etc/rc.firewall the same way 631 is added?
Add port 9100 (JetDirect port) to /etc/rc.firewall the same way 631 is added?
\end{verbatim}

On 20061108, ESS installed a new printer for faculty on a private network.
This requires modifying the \flidx{interfaces} file to start a new
interface solely for this printer.
This printer was changed toan HPM600 printer in 201311.
\begin{verbatim}
# HP4350 DTN printer in ESS mail room IP 192.168.14.2
# HPM600 printer in ESS mail room IP 192.168.14.2
iface eth2 inet static
      address 192.168.14.2
      broadcast 192.168.14.255
      netmask 255.255.255.0
\end{verbatim}

Printing protocols may be enabled and disabled by \trmidx{firewall}
controls on particular ports.
On \trmidx{Debian}-compatible GNU/Linux systems with 2.6 Kernels, the
firewall is built with \cmdidx{iptables}.
The iptables rules are set in \flidx{/etc/rc.firewall}.
\begin{verbatim}
iptables -L # Print IP tables configuration
iptables -F # Flush IP tables
iptables -X # Delete extra chains
iptables -P OUTPUT ACCEPT
iptables -P INPUT ACCEPT
iptables -P FORWARD ACCEPT
ip route flush cache # Flush existing routes
ip route show cache # Show existing routes
\end{verbatim}
Under \trmidx{Debian}, use the \cmdidx{guarddog} program to configure
firewalls.
At a minimum, allow these protocols: 
\trmidx{DNS}, 
\trmidx{FTP}, 
\trmidx{HTTPS}, 
\trmidx{HTTP}, 
\trmidx{IPP}, 
\trmidx{NNTP}, 
\trmidx{NTP}, 
\trmidx{POP3}, 
\trmidx{Ping}, 
\trmidx{rsync}, 
\trmidx{SMTP}, 
\trmidx{SSH}, 
\trmidx{VNC}.
Systems without these protocols set will not behave well.
License daemons on other monstrosities will usually need to be
configured by hand.
For instance, the \trmidx{IDL} license daemon wants to send and
receive \trmidx{TCP} packets on \trmidx{port}~1700.
This must be defined as a new ``User Defined'' protocol under
``Advanced Settings'' and then manually enabled in all intervening
network zones.

\subsection{Virtual Memory}\label{sxn:mem}
When compiling large programs like \verb'mie.cc', \cmdidx{g++} may fail
with an error like ``virtual memory exhausted''.
This may be due to having to small a swap partition.
To see system parameters use \cmdidx{ulimit}, e.g., \cmdprn{ulimit -a}.
According to Linus, the size of the swap partition should be twice the
amount of RAM, see \S\ref{sxn:prt}. 
Here is an untested way to use a swapfile instead of repartitioning:
\begin{verbatim}
# Make a swap file large enough to work around "virtual memory exhausted" errors
# $SIZE is number of kilobytes, $SWAP is name of swapfile
$ dd if=/dev/zero of=$SWAP bs=1024 count=$SIZE && chmod 0600 $SWAP && sync && mkswap -c $SWAP $SIZE && swapon $SWAP 
\end{verbatim}
% $: rebalance syntax highlighting

\subsection{tramp.el}\label{sxn:tramp}
\trmidx{Tramp} stands for ``Transparent Remote (file) Access, Multiple
Protocol''.
It a most excellent Emacs extension which facilitates editing
files on remote machines in a local \cmdidx{emacs} session, using
\cmdidx{ssh}, \cmdidx{rcp}, or any number of other protocols to 
transfer edits to the remote machine.
Checkout \cmdidx{tramp} from the \trmidx{SourceForge} archive:
\begin{verbatim}
cd ${DATA}
cvs -d:pserver:anonymous@cvs.sourceforge.net:/cvsroot/tramp login
cvs -d:pserver:anonymous@cvs.sourceforge.net:/cvsroot/tramp co -kk tramp
\end{verbatim}
Here are Tramp formats that appear to work:
\begin{verbatim}
/multi:dust.ess.uci.edu:greenplanet.ps.uci.edu:~/.plan
/greenplanet.ps.uci.edu:.plan
/dust.ess.uci.edu:.plan
/esmf.ess.uci.edu:.plan
/nco.sf.net:/home/groups/n/nc/nco/nco/doc/nco.texi
/scp/zender@krein.math.uci.edu:/home/ess/zender/zender/.plan
/scp/krein.math.uci.edu:/home/ess/zender/zender/.plan
/zender@krein.math.uci.edu:/home/ess/zender/zender/.plan
/krein.math.uci.edu:/home/ess/zender/zender/.plan
/multi:ssh:zender@dust.ess.uci.edu:ssh:zender@krein.math.uci.edu:~/.plan
/[multi/ssh:zender@dust.ess.uci.edu/ssh:zender@krein.math.uci.edu]~/.plan
\end{verbatim}
Yes, Tramp does expand \verb'~' correctly (i.e., on the remote machine).
This section falls out of date quickly.
Tramp filename conventions have changed at least three times in three
years. 

\subsection{RPM packages in /usr that were installed (and may need to be uninstalled or reinstalled) by hand: }\label{sxn:nst2}
\begin{verbatim}
opendx-4.1.0-1.i386.rpm
opendx-docs-4.0.10-1.i386.rpm
sgi-opengl-1.2.1-1.i386.rpm
ImageMagick-5.1.1-1.i386.rpm
ghostscript-6.01-1.i386.rpm
ghostscript-fonts-6.0-2.noarch.rpm
\end{verbatim}

\subsubsection{IPCC and PBS configuration}\label{sxn:ipcc}
The \trmidx{IPCC} and \trmidx{PBS} clusters,
\mchidx{ipcc.ess.uci.edu} and \mchidx{pbs.ess.uci.edu}, run
\trmidx{Rocks}. 
Rocks-based clusters use the Community Enterprise Operating System,
\trmidx{CentOS}, a re-packaged version version of \trmidx{RHEL}. 
\trmprn{CentOS} is an \trmidx{RPM}-based system.
Copies of \trmprn{CentOS} are kept on the system so that new RPMs
may be easily installed.
\begin{verbatim}
# Get CentOS bison locally or from the net:
/data/centos/4.3/os/x86_64/CentOS/RPMS/bison-1.875c-2.x86_64.rpm
cd ${DATA}/tmp
wget ftp://ftp.nluug.nl/pub/os/Linux/distr/CentOS/4.3/os/x86_64/CentOS/RPMS/bison-1.875c-2.x86_64.rpm
sudo rpm -ivh bison-1.875c-2.x86_64.rpm
\end{verbatim}
\trmprn{Rocks} names compute nodes in a 1-based enumerated format so a
26-compute node cluster has nodes:
\begin{verbatim}
compute-0-1, compute-0-2, compute-0-3, ... compute-0-26
\end{verbatim}
Some IPCC compute nodes are loaded by the Medium Performance Cluster
(\trmidx{MPC}). 
\trmprn{MPC} names compute nodes in a 1-based enumerated format so a
26-compute node cluster has nodes:
\begin{verbatim}
ipcc-n1, ipcc-n2, ipcc-n3, ... ipcc-n26
\end{verbatim}
Each node is, by default, accessible via \cmdidx{ssh}.
The \cmdidx{top} command shows the load on all processors in the
line labeled ``Cpu(s)'' (near line~3).
This mode presents statistics as a fraction of total available
resources for all CPUs so that full usage of 1~CPU on a dual CPU
system shows up as $\sim 50$\% user usage, and $\sim 50$\% idle.
The load average (uppermost line), on the otherhand, is shown as
a fraction of maximum load for one CPU.

Pressing \kbdidx{1} while \cmdprn{top} is running toggles the
presentation mode between a single-line ``Cpu(s)'' summary and 
multiple lines of per-CPU summaries. 
The load average in the top-most line approaches~$N$ when an
$N$-processor node is fully utilized, e.g., about~4.0 for a quad-CPU
system. 
The \cmdidx{free} and \cmdidx{xosview} commands are also useful
at monitoring system usage.

The \trmidx{PBS} batch queuing system may be interrogated with a
number of commands such as \cmdidx{pbsnodes}
\begin{verbatim}
cluster-ps # Connects to all computes nodes and does 'ps'
free-nodes # Display all free compute nodes (MPC)
job-uptime # Show load averages for all user running jobs on Torque
pbsnodes -a # Display all compute nodes and jobs assigned to them
qdel 161608 # Cancel job number 161608 in queue
qmon & # GUI for qstat-functionality for SGE
qstat -a # Display all jobs running/queued
qstat -r # Display all running jobs
qstat -r | grep ipcc # Display all running jobs
qsub -I -q opteron # Request interactive node from Opteron pool
esmfusers # display processes owned on each ESMF node
\end{verbatim}

Running (and killing) \trmidx{MPI} jobs can leave clusters with
un-released resources such as shared-memory allocations and shared
semaphores.
This unclean state may block further \trmprn{MPI} execution.
To clean up the leftover state, use 
\trmidx{Inter-Process Communication} (\trmidx{IPC}) management
commands such as \cmdidx{ipcs}:
\begin{verbatim}
# On interactive node:
ipcs -m | awk '/^ *0x/ {print $2 }' | xargs -n 50 ipcrm shm
ipcs -s | awk '/^ *0x/ {print $2 }' | xargs -n 50 ipcrm sem
# To compute-nodes:
cluster-fork "ipcs -m | awk '/^ *0x/ {print \\\$2 }' | xargs -n 50 ipcrm shm"
cluster-fork "ipcs -s | awk '/^ *0x/ {print \\\$2 }' | xargs -n 50 ipcrm sem"
\end{verbatim}
As its name implies, \cmdidx{cluster-fork} issues its argument as a
command to all cluster nodes.

\subsubsection{GPG: GNU Privacy Guard}\label{sxn:GPG}
The \trmidx{OpenPGP} standard defines the cryptographic authentication 
protocol that \cmdidx{gpg} implements.
\begin{verbatim}
# Generate a new key
gpg --gen-key

export GPGKEY='6F635D10' # GPG key ID
gpg --export -a 'Charlie Zender' # -a: Add ASCII armor
gpg --keyserver pgp.mit.edu --send-keys ${GPGKEY}
gpg --keyserver keys.gnupg.net --send-keys ${GPGKEY}
gpg --keyserver pgp.mit.edu --search-keys 'Charlie Zender'
gpg --list-public-keys
Type  Bits / KeyID   Date       User ID
pub   1024D/6F635D10 2006-11-20 Charlie Zender <zender@uci.edu>
sub   2048g/9148C6AF 2006-11-20

Key fingerprint = DBD0 E788 E13C 56A2 6C5D  2C62 CB91 49AD 6F63 5D10

-----BEGIN PGP PUBLIC KEY BLOCK-----
Version: GnuPG v1.4.3 (GNU/Linux)

mQGiBEVh1bcRBADRCbLu37Qlsu2cOjckAlAgzvHBjMYQMtdQjJOe/rv9+p6StXyM
SNp6GhpwX3b8j6r0NkT6fOqwaUh/uvxdTRR1FQJF7tHVAq0n9uY1stKFqRQC+cu+
UP74FgfEvqh5pFL8sI05ayOMO7GZ0lh54rQQAQ/l7TK29ttNDIyg2hpyWwCg7tfm
nQ3gbeqdOhC9N1EikO7jCd0EAMq2ceX8hJv4l701dpb4O5zogKYcto7BHULlUPAY
v9FYt42MsF1kH6E0XC+HAA/pJ9sfCCrWKzbRtYkZGipJnB8bx19kxUvlRtPHXSdJ
PHyckPOTSeyCAX0/YbyOWMsZYKMZf2t0VIlEIMLSjTgwpu2OY/zSyi0Lx9kr2K8y
bUdpBACXoGOSVBZdEX3z/ck5Ogan874UbTbOTxB3yw5Ry58CL52eZXaYP0v9pWL9
cCe9hAoVlY2sH5fqZnKwQpT0HeNYcCdcY7wBtVnnW/4bjeljv5MsP2tkqsT2ucYg
S4HDpfshfH0GwpUMHgDgfwXqyOUG4setLd5jfhct9UVptdZ8tLQfQ2hhcmxpZSBa
ZW5kZXIgPHplbmRlckB1Y2kuZWR1PohgBBMRAgAgBQJFYdW3AhsDBgsJCAcDAgQV
AggDBBYCAwECHgECF4AACgkQy5FJrW9jXRC60ACgkWtdilBohVmiaxTwGXgstMSF
JzUAnApKDwlVOOpTlDUw1CxpCqeAYwRmuQINBEVh1b8QCACyZg28taaSucl0vMSC
7ZB5KOZgr+SDuuEjl9p7Wwu83E6MppsZKQWyhjKJAhmjnAMYAgOexN5xSjgi50Or
Df+pPrULN49Ec90qu2z/B2O74NZA3237t4ATZnMz8l0saL7R+BjtpEKjh43SDOmd
lD9Da3aNYP04V2tU78F8Iusq3/k66Ppi+v9XZ+vLsApCgjBt0PJUbOvznt0cUYBS
M4wvUsMzMB6iffW+CfAo2gaPCsqYEbEvMN1KiM15jOi7wNFCAOtx+s8sWQ6LVIH8
vpqMNcEPIlUrH95FTuPIklnEC2sk8WFzmCNWqRBTP/iTeXF51zvz58OXg7joVoHp
oN0rAAMGB/9n/AOrRMLIv9nSzM/R25O5VBAiAU/vka+1lSABpN4H/S2F8B+7ehrK
OWEo2yuStwcu2Hl/F02OwaaNbMzK0sUGMrbgHDGd/venbIKu/F7mEWfXndP8MbXx
eT9tTjD0Oirm42JSTIQ7SA6RQg5eX0iPapCGbh+BLemv0itR10E3YBf1t86UElL6
MzSJUv8xo/lodt8FxFvM8nKSEs/CAxwYFkAUV/8I+Gdmd6ZTgwFcegNSm5V+811K
BBe2xcNkpokPH2Zfam7W7Teo1dSvF7RUXGDoEMorkXvuKGkb6rjwv4k/j8/krbNu
V/WGP7GerNFkb1JDu3fcC2CFqlpvsFfqiEkEGBECAAkFAkVh1b8CGwwACgkQy5FJ
rW9jXRAZnwCdHhbbyY0H5bszcM7cj3kXnYoEIPoAn3DXHYEmnSXBZ//FJJc0LTG8
WEy6
=YK0f
-----END PGP PUBLIC KEY BLOCK-----
\end{verbatim}
See \url{https://help.ubuntu.com/community/GnuPrivacyGuardHowto}
for GPG suggestions.

\subsubsection{Building RPMs}\label{sxn:rpmbld}
\begin{verbatim}
# HOWTO build RPMs as a user
# http://myy.helia.fi/~karte/linux/doc/rpm-build-as-user.html
sudo dnf -y install rpm-build
# /etc/rpm/macros # (c) GPL 2003 Tero.Karvinen at-sign iki.fi
%packager %(echo "$USER")
%_topdir %(echo "$HOME")/rpmbuild
%_rpmtopdir %{_topdir}/%{name}
%_builddir %{_rpmtopdir}/BUILD
%_rpmdir %{_rpmtopdir}
%_sourcedir %{_rpmtopdir}
%_specdir %{_rpmtopdir}
%_srcrpmdir %{_rpmtopdir}
%_tmppath %{_rpmtopdir}/TMP
%_buildroot %{_tmppath}/%{name}-root

# Build rpms in your home directory, without root priviledges
# Users can copy this file to $HOME/.rpmmacros to override
# settings, such as packager to "Firstname Lastname email".
# Example of typical directory and file hierarchy for
# rpm building:
#  ~/rpmbuild/
#  ~/rpmbuild/nano/ # replace nano with rpm name
#  ~/rpmbuild/nano/TMP/
#  ~/rpmbuild/nano/BUILD/ # user must create this dir
# and files (can be extracted from a source rpm with
#  rpm2cpio *.src.rpm | cpio -dvi
#   ~/rpmbuild/nano/nano-1.2.tar.gz
#   ~/rpmbuild/nano/nano.spec
# Then you can build your rpm with
#  rm -rf TMP/* BUILD/* ; rpmbuild -ba *.spec 2>&1 |tee rpmbuild.log
# For more information, see www.iki.fi/karvinen
\end{verbatim}

\subsubsection{RPM commands}\label{sxn:rpm}
The RedHat Package Manager (\trmidx{RPM}) is used to maintain sources
and binaries on a wide variety of Linux systems.
The packages, so-called RPMs, are manipulated with the \cmddfn{rpm}
command. 
\begin{verbatim}
rpm -qa | grep foo # List all installed packages and search for foo
rpm -Uhv *.rpm # Upgrade packages (even if not installed)
rpm -Fhv *.rpm # Freshen packages (only if already installed)
rpm -q --whatprovides foo # Which installed package provides file foo?
rpm -qf foo # Which installed package provides file foo?
rpm -ql foo | less # Which files does installed package foo own?
rpm -qpl foo.rpm | less # Which files will package foo.rpm install?
rpm -e foo.rpm # Uninstall package
rpm --info foo.rpm # Print descriptive information about package
rpm -qR foo.rpm # Upon what packages does foo depend?
rpm --nodeps -e foo.rpm # Uninstall package regardless of dependencies
rpm -F --replacefiles --nodeps foo*.rpm # Do not check dependencies, allow file overwrites
rpm2cpio mypackage.rpm | cpio -vid # Extract files from RPM to disk
\end{verbatim}

\subsubsection{APT commands}\label{sxn:apt}
The \trmidx{Debian} package manager is called \trmidx{APT}, Advanced
Package Tool.
Debian packages, so-called \flidx{.deb}'s (``dot debs''), are
manipulated with the \cmddfn{apt} command. 
\begin{verbatim}
apt-cache search foo # Which packages relate to subject foo?
apt-cache search XML::Simple # Which package contains XML::Simple?
apt-file update # Refresh apt-file database
apt-file search foo # Which (non-installed) package installs/provides file foo?
aptitude --help
aptitude -t unstable install foo # Install package foo from unstable
aptitude autoclean # Remove only package files that can no longer be downloaded
aptitude clean # Remove everything except lock files from /var/cache/apt
aptitude dist-upgrade # Smarter version of upgrade
aptitude install `apt-show-versions -u -b | grep unstable` # Upgrade unstable packages only
aptitude install foo # Locate, download, and install package foo
aptitude install foo http://marillat.free.fr/ # Locate, download, and install package foo
aptitude install foo=x.y # Locate, download, and install package foo version x.y
aptitude remove foo # Uninstall foo-x.y.deb
aptitude update # Resynchronize package index files from sources
aptitude upgrade # Install newest versions of all packages currently installed
aptitude upgrade foo # Install newest version of package foo
dpkg --configure foo # Configure unpacked package foo
dpkg --contents foo # Examine contents of archive
dpkg --force-help # Help on force actions
dpkg --install --force-overwrite # Install and overwrite files from one package with another (e.g., icc & ifc)
dpkg --info foo # Examine archives
dpkg --install foo # Install package foo (Use this not aptitude install)
dpkg --list 'foo*' # Locate all packages named foo*
dpkg --listfiles foo # Which files does installed package foo own?
dpkg --print-avail foo # Print details about package foo
dpkg --purge foo # Remove package foo.deb, including configuration files
dpkg -P foo # Remove package foo.deb, including configuration files
dpkg --remove foo # Remove package foo.deb, leave configuration files intact
dpkg --search 'foo*' # Which installed package owns file foo?
dpkg -S 'foo*' # Which installed package owns installed file foo?
dpkg --status foo # Print installation status of package foo
dpkg --vextract foo bar # Display filenames contained by package foo in directory bar
dpkg-reconfigure foo # Re-configure installed package foo
\end{verbatim}

\clearpage
\subsection{Required software}\label{sxn:tch2}
\trmidx{required software}
Following is a list of scientific software required for
research/teaching. 
Most of these packages are installed by default with standard and/or
``power'' GNU/Linux distributions, such as \trmidx{Debian} and
\trmidx{RedHat}. 
Since these are free (as in no-cost) software packages, it is 
easiest if they are all installed on all machines, i.e., no
differences between servers and clients.
Binaries and libraries should be installed in \flprn{/usr} when
the package is (or is indistinct from one) supplied with the base
operating system. 
Software not available in native pre-compiled package format should
be installed in \flprn{/usr/local} by default. 
This avoids pathname proliferation.   
This is true of most \trmidx{.deb}s and \trmidx{RPM}s, for example.
When any administrative intervention is required, the package is best
installed in \flprn{/usr/local}.
For this reason, it is wise to back-up \flprn{/usr/local} and
unnecessary to back-up \flprn{/usr}.
Packages which have up-to-date pre-compiled binaries for most
Linux distributions:
\begin{enumerate*}
\item \trmidx{ANTLR} 
\item \trmidx{Autoconf} 
\item \trmidx{Autoheader} 
\item \trmidx{Automake} 
\item \trmidx{Bash} 
\item \trmidx{Bison} 
\item \trmidx{CVS} 
\item \trmidx{DDD} 
\item \trmidx{Emacs}
\item \trmidx{Flex} 
\item \trmidx{GCC}
\item \trmidx{GDB}
\item \trmidx{GSL}
\item \trmidx{Gettext} 
\item \trmidx{Ghostscript}
\item \trmidx{Ghostview}
\item \trmidx{Gnuplot} 
\item \trmidx{Gzip}
\item \trmidx{Libtool} 
\item \trmidx{M4} 
\item \trmidx{Make} 
\item \trmidx{Octave} (\trmidx{Matlab} clone)
\item \trmidx{Perl} (must be executable as \flidx{/usr/bin/perl})
\item \trmidx{R} (\trmidx{S+} clone) 
\item \trmidx{Subversion}
\item \trmidx{Tar}
\item \trmidx{wget} 
\item \trmidx{ldd}
\item \trmidx{locate}
\end{enumerate*} 
Packages which may not have up-to-date pre-compiled binaries for many
Linux distributions:
\begin{enumerate*}
\item Adobe: \trmidx{Acroread}
\item Intel: Fortran~95 and C++ compilers
\item Java runtime environment
\item HDF: \trmidx{HDF} 1.8.1+
\item MPI: MPICH2
\item NCAR: NCAR Graphics and NCL
\item Sourceforge: \trmidx{NCO}
\item UCSD: \trmidx{Ncview}
\item Unidata: \trmidx{netCDF} 4.0+, \trmidx{UDUnits}
\end{enumerate*}
\clearpage

\subsection{Packages installed in /usr/local (some RPM, some *.tgz,
some proprietary):}\label{sxn:nst3}
DODS    Distributed Oceanographic Data System
dx      IBM Data explorer
hdf     Hierarchical Data Format
gnuplot Gnuplot (for DDD)
java    Java
ncBrowse ncBrowse 1.2.1
ncarg   NCAR graphics
pgi     PGI Fortran
rsi     Research Systems International Interactive Data Language (IDL)
udunits Unidata units conversion package

\subsection{NCO and patches}\label{sxn:nco}
Updating NCO with patches:
The only difficult thing about patches is remembering the switches
used to create and apply them.
\begin{verbatim}
cd ~/nco # Create patch relative to top level NCO directory
cvs diff -c > feature.patch # Create patch
cd ~/nco # Apply patch from top level NCO directory
patch -p0 < feature.patch # -p0 = Causes patch to use unmodified file names
\end{verbatim}
I have simplified remembering the last command by adding
\verb*alias patch='patch -p0'* to my \verb'.bashrc' file.
When patches go wrong they can be unapplied with 
\begin{verbatim}
patch -p0 -R < feature.patch # -R = Reverse patch
patch -p0 --reverse < feature.patch # --reverse = Reverse patch
\end{verbatim}

\subsection{Skype}\label{sxn:skype}
The UCI \trmidx{VoIP} FAQ is
\href{http://www.nacs.uci.edu/telephone/voip-faq.html}{here}. 

The \trmidx{Skype} application allows free IP-based telephony
between registered users.
The \cmdidx{skype} program package for Debian is at
\url{http://www.skype.com/download/skype/linux}.
Once registered, simply start \cmdprn{skype} from a console.

Inspiron 9400 users have found that the audio captured via microphone
(plugged into the external jack) is almost inaudible on
playback. 
One solution is use the command-line \cmdidx{alsamixer} program to 
set the ``Capture'' recording level to 75/75.
Apparently the 9400 uses the microphone input for dual purposes and
this may eventually be sorted out by software.

Ekiga is another program that uses VoIP.
My \cmdidx{ekiga} address is \url{sip:zender@ekiga.net}.
Ekiga is supposed to work with cameras like my Logitech QuickCam Pro
for Notebooks.
This camera works with \trmprn{Ubuntu} ``out-of-the-box'' if one
selects the \cmdidx{v4l2} driver.
Otherwise, compile and install the driver yourself with:
\begin{verbatim}
sudo aptitude install linux-headers-`uname -r` linux-restricted-modules-`uname -r` build-essential subversion
cd ${DATA}/tmp
svn checkout http://svn.berlios.de/svnroot/repos/linux-uvc/
cd linux-uvc/linux-uvc/trunk
make
sudo make install
# Start ekiga with debugging option
ekiga -d 1 &
# Plug in camera look at dmesg for the device listing, e.g., /dev/video1
# Point your application at that device and see if it works
# If the video device does not work, try restarting the module
sudo rmmod uvcvideo
sudo modprobe uvcvideo trace=15
dmesg | grep uvc

# New method
vncviewer -via tephra.ess.uci.edu -shared localhost:15
# Enter password when asked

# Old method
ssh -L5915:localhost:5915 zender@tephra.ess.uci.edu # Start tunnel to/on tephra
vncviewer :15 # Open VNC window on local machine
\end{verbatim}

\subsection{Installing RedHat Linux}\label{sxn:rhl}
Get following non-default packages:
Disk 1: openssh-server ElectricFence X100dpifonts
DisXk 2: sudo xsanegimp
Powertools: ddd openmotif openmotif-devel acroread dxpc octave blas lapack
Other: gsl, RealPlayer, abisuite, HDF
Do not install netcdf RPM as it is built with double-underscore versions
of all Fortran functions.

\subsubsection{Updating RedHat Linux with latest patches}\label{sxn:rhl2}
Read UCI instructions at
\url{http://www.dcslib.uci.edu/linux/index.html}.
Script is installed as \verb'/usr/local/bin/uci_dcs_lnx_pch.pl'.
Replace ``\verb'7.0''' and ``\verb'i386''' with appropriate version
and architecture information below, then run as root
The \trmidx{kernal packages} and \trmidx{C library} are stored in 
CPU-specific directories for maximum performance.
Update these directories first, e.g., \verb'i686' instead of \verb'i386'. 
Then update the rest of the packages.
\begin{verbatim}
sudo mount ftp.uci.edu:/extra/ftp/mirrors/redhat /mnt
cd /mnt/linux/updates/9/en/os/i686 # CPU-specific, e.g., kernel updates
rpm -Fhv *.rpm
cd /mnt/linux/updates/9/en/os/i386
rpm -Fhv *.rpm
rpm -Fhv /mnt/linux/7.2/en/os/i386/preview/RPMS
rpm -Fhv /mnt/linux/7.2/en/os/powertools/i386/RedHat/RPMS
# Remotely update all RedHat RPMs against UCI Redhat Mirror
rpm -Fhv 'ftp://ftp.uci.edu/mirrors/redhat/linux/updates/7.2/en/os/i386/*.rpm' 
# Remotely update all RedHat RPMs against UCI Redhat Mirror
rpm -Fhv 'ftp.uci.edu:mirrors/redhat/linux/updates/7.2/en/os/i386/*.rpm'
\end{verbatim}
When a large number of simultaneous updates are required to patch a
system, the command \cmdprn{rpm -Fhv *.rpm} may not work.
In this case, break the task down into smaller tasks for \cmdidx{rpm},
e.g., 
\begin{verbatim}
for ltr in a b c d e f g h i j k l m n o p q r s t u v w x y z A B C D \
E F G H I J K L M N O P Q R S T U V W X Y Z; do
rpm -Fhv ${ltr}*.rpm
done
\end{verbatim}
Another option is to install the \trmidx{RedHat} \trmidx{Rawhide}
distribution. 
This distribution is available from
\url{ftp://ftp.redhat.com/pub/redhat/linux/rawhide/i386}.
A useful, comprehensive list of distributions is maintained at
\url{http://distrowatch.com}.

\subsubsection{Upgrading Ubuntu Linux}\label{sxn:ubn}
Upgrade problems are usually discussed at a central site, e.g., 
\url{https://help.ubuntu.com/community/KarmicUpgrades/Kubuntu}
\begin{verbatim}
# Command line upgrade prompt (as non-priveleged user):
update-notifier-kde -u
# If this fails, try
sudo dpkg --configure -a
# Dapper to Hardy upgrade via command line:
sudo aptitude install update-manager-core
sudo do-release-upgrade
\end{verbatim}

\subsubsection{Upgrading RedHat Linux}\label{sxn:rh3}
For network upgrades and installs, you must first create a
\verb'bootnet.img' floppy to install from, and then specify
the network address of the \trmidx{RedHat} FTP repository.
\begin{verbatim}
dd if=bootnet.img of=/dev/fd0 bs=1440k
ftp://ftp.uci.edu/mirrors/redhat/linux/6.2/i386/RedHat/
ftp://ftp.uci.edu/mirrors/redhat/linux/6.2/i386/images/bootnet.img
\end{verbatim}

\subsection[Debugging]{Debugging}\label{sxn:dbg}
Debugging is an art as much as anything.
A list of debugging methods sorted by past efficacy is
\begin{enumerate}
\item Compile programs on different platforms.
One compiler may notice errors that another compiler does not report.
SGI compilers are especially good at finding errors.
\item Compile with bounds checking if possible
\item Run with \cmdprn{MALLOC\_CHECK\_=1} in environment
\item Link to memory debugging libraries like Electric Fence 
\url{ftp://ftp.perens.com/pub/ElectricFence} or
\cmdidx{dmalloc} (\url{http://dmalloc.com}).
\item Be sure that the shell has enough resources (e.g., memory) to
run the program or the crash may be especially mysterious.
This can involve using the \cmdidx{ulimit} command, e.g.,
\cmdprn{ulimit -s unlimited}.
However, this command may require special permissions to execute. 
\item When all else fails, use a symbolic debugger like
  \cmdprn{DDD}/\cmdprn{gdb} (\S\ref{sxn:ddd}).
\end{enumerate}
The \trmidx{Electric Fence} debugger works very well cith C-language
programs. 
However, Electric Fence may hinder debugging \cxx\ programs.
\cxx\ programs linked to the Electric Fence library
(\flidx{libefence.a}) may generate obscure errors within
\trmidx{DDD}/\trmidx{GDB} such as 
\verb'libpthread.so.0: cannot load shared object file: Cannot allocate memory'. 
If this occurs, simply compile the program without \flprn{-lefence}
before loading into GDB. 

Once bugs have been found and identified, consider sending a notice to
those who might have been affected by the bug.
\begin{enumerate}
\item Identify bug symptoms so users may determine whether they
were affected by it. 
It is also helpful to identify versions and/or dates of the code
releases known to be affected by the bug.
\item Identify the bug cause in plain English, i.e., passed wrong
pointer, transposed two arguments, error in equation, etc.
\item Describe extent of side-effects which bug \textit{may} cause.
Estimates of magnitude of bug, whether results were randomly or
systematically biased, which particular regions were more or less
affected, etc.
\item Complete the classification of the bug's damage by mentioning
what processes, routines, regions, or versions, are \textit{not}
affected by the bug. 
\item Determine and report on what future changes, if any, should be 
made to software management to ensure this type of bug does not occur
again. 
\item Acknowledge whether or not the bug fix is known to be solid
\item Provide actual code patch to fix bug
\end{enumerate}

Linux Journal \#87 (July, 2001, p.~82) gives helpful debugging tips.
The \cmdidx{MALLOC\_CHECK\_} environment variable is one such method.
Setting this variable and then running a faulty program will cause
the program to print some verbose error messages when the fault is
triggered. 
See the \cmdidx{man} page for \cmdidx{malloc} for more information.
\begin{verbatim}
MALLOC_CHECK_=0 ccc --tst=mmr # Heap corruption silently ignored
MALLOC_CHECK_=1 ccc --tst=mmr # Heap corruption prints diagnostics to stderr
MALLOC_CHECK_=2 ccc --tst=mmr # Heap corruption causes immediate abort()
\end{verbatim}

Many commercial tools are available to find \trmidx{memory leaks}.
\cmdidx{ccmalloc} is a free tool.
Simply link with \cmdidx{ccmalloc g++}, and then run with 
\cmdprn{MALLOC\_CHECK\_ = 1} to produce a report
\begin{verbatim}
ccmalloc g++ -o ccc ccc.o
gcc -o ncks ncks.o /usr/local/lib/ccmalloc-gcc.o -L/usr/local/lib -lccmalloc -ldl
g++ -o ccc ccc.o /usr/local/lib/ccmalloc-g++.o -L/usr/local/lib -lccmalloc -ldl
MALLOC_CHECK_=1 ccc
\end{verbatim}
\cmdprn{ccmalloc} generates a report, and the number of memory leaks
in the program is shown in the garbage column.
\url{http://ieee.uow.edu.au/~mjp16/prog/memleak.html} describes the procedure.

\subsection{Valgrind}\label{sxn:valgrind}
The most useful memory debugging tool since about 2002 has been
\cmdidx{valgrind}\footnote{The term is from Norse mythology and is
pronounced ``val-grinned''}. 
Simply precede the program invocation with \cmdprn{valgrind [options]}.
\begin{verbatim}
valgrind ccc >foo 2>&1
valgrind --leak-check=yes ccc > ~/foo 2>&1 # Print location of likely leaks
valgrind --tool=memcheck ccc > ~/foo 2>&1 
\end{verbatim}
Since there are many errors in the default GNU compilers and
libraries, these errors may be suppressed using the 
To make it easier to write suppressions, tell \cmdprn{valgrind} to
print the suppression command for each error it encounters
\begin{verbatim}
valgrind --quiet --gen-suppressions=yes ccc # Generate suppression messages 
\end{verbatim}
In this mode, \cmdprn{valgrind} queries whether to print suppression
text for each 
The default suppressed warnings are contained in
\flidx{/usr/local/lib/valgrind/default.supp}, and 
Store additional suppressions you wish to ignore in a 
\trmidx{suppresions file}, e.g., \flidx{valgrind.txt}.
Multiple suppressions files may be used by specifying them with the 
\cmdprn{--suppressions} flag: 
\begin{verbatim}
valgrind -v --suppressions=${HOME}/c++/valgrind.txt ccc >foo 2>&1
\end{verbatim}
% $: rebalance syntax highlighting

\subsection{DDD}\label{sxn:ddd}
DDD has many obscure capabilities, especially when running in GDB mode.
Table~\ref{tbl:gdb} summarizes frequently used GDB commands.
\begin{table}
\begin{minipage}{\hsize} % Minipage necessary for footnotes KoD95 p. 110 (4.10.4)
\renewcommand{\footnoterule}{\rule{\hsize}{0.0cm}\vspace{-0.0cm}} % KoD95 p. 111
\begin{center}
\caption[GDB Commands]{\textbf{GDB Commands}% 
\label{tbl:gdb}}
\vspace{\cpthdrhlnskp}
% fxm:\raggedright does not work here, must be between two other columns?
%\begin{tabular}{ >{\ttfamily}l<{} >{\raggedright}p{20.0em}<{} }
\begin{tabular}{ >{\ttfamily}l<{} l }
\hline \rule{0.0ex}{\hlntblhdrskp}% 
\textrm{Command} & Result \\[0.0ex]
\hline \rule{0.0ex}{\hlntblntrskp}%
start & Break at beginning of program (all languages) \\[0.5ex]
break nco\_omp.c:116 & Set breakpoint at line \# of specified file \\[0.5ex]
info breakpoints & Print all breakpoints \\[0.5ex]
delete 1 & Delete breakpoint \#1 \\[0.5ex]
break rt\_cls::var\_put & Break at start of function \verb'rt_cls::var_put' \\[0.5ex]
break nco\_var\_fll & Break at start of function \verb'nco_var_fll' \\[0.5ex]
break spc\_slr.cc:spc\_slr\_cls::spc\_slr\_cls & Break at start of function \verb'spc_slr_cls' of class \verb'spc_slr_cls' in file \verb'spc_slr.cc' \\[0.5ex]
print grd[0]@(grd\_nbr-0) & Print first \verb'grd_nbr' items of array \verb'grd' \\[0.5ex]
print var\_prc[idx]->val.fp[0] & Print first item of array \verb'var\_prc[idx]->val.fp' \\[0.5ex]
watch idx & Set watchpoint on \verb'idx' \\[0.5ex]
break dstmbl\_MP\_dst\_mbl\_ & Set breakpoint on G95-compiled module procedure (MP) \\[0.5ex]
print dstctl\_MP\_nstep & Print G95-compiled Module Procedure (MP) variable \\[0.5ex]
% & \\[0.5ex]
\hline
\end{tabular}
\end{center}
\end{minipage}
\end{table} % end tbl:gdb
GDB does not know the size of dynamically allocated arrays.
To print the first \verb'grd\_nbr' items of array \verb'grd', use
\begin{verbatim}
print grd[0]@(grd_nbr-0)
\end{verbatim}
The \verb'-0' is necessary to get DDD to recognize that the argument
is an integer.
This may only be neccessary when argument is type \verb'long'.
To display this array in the display window, highlight
\verb'grd[0]@(grd_nbr-0)' so that it appears in the argument line at
the top next to \verb'():'.
Then click the display button.
The required specification can be tedious, especially when indirection
and class structures are involved, e.g.,
\verb'tst_obj->flx_slr_frc_in[0]@tst_obj->wvl_nbr_in'
or, for, say, $N-3$ elements, the even more complex
\verb'(*tst_obj)->flx_slr_frc_in[0]@(tst_obj->wvl_nbr_in-3)'.

Setting conditional breakpoints can be done with the \verb'watch'
function.
For example, consider the problem of breaking inside of a loop
once the value of the counter, \verb'idx', is~37.
The GDB command is \verb'watch idx', which sets a watchpoint on
\verb'idx'.
A watchpoint is a breakpoint that is called whenever the expression
changes value.

``Stepping'' through \cxx\ code is tedious because GDB takes the long 
route through all the interface files.
Instead, set breakpoints at the start of the desired function
with, e.g., \verb'break rt_cls::var_put'.

Using \cmdprn{gdb} on \trmidx{G95} code is possible but not pretty.
Symbols are case sensitive (use lower case).
Dummy arguments are actually pointers.
Module variables have a \verb'modulename_MP_' prefix.
Module procedures have the same, plus an underscore \verb'_' suffix.

On SGI 64-bit machines, \cmdprn{dbx} and \cmdprn{gdb} do not work and one
must use the \cmdidx{cvd} debugger.
This debugger is powerful but non-intuitive. 
To view the stdout stream one must open the ``Execution View''
window. 
Also in the Views menu is the ``Variable Browser'' which does what it
says. 
Clicking on the variable names in the ``Variable Browswer'' will bring
up a nice ``Array Browser'' for arrays.
Breakpoints are set in the ``Traps'' menu in \cmdidx{cvd}.

On AIX, use the \cmdidx{xldb} debugger.
Invoke with 
\begin{verbatim}
export DISPLAY=ashes.ess.uci.edu:0.0
xldb `which swnb2` --drc_in=${DATA}/aca -D 1 -E -e 1603 -d foo.nc &
dbx -d 100 `which mie`
\end{verbatim}
% $: rebalance syntax highlighting

The \trmidx{Lahey} \cmdidx{lf95} compiler comes with the \cmdidx{fdb}
debugger. 
Invoke with 
\begin{verbatim}
fdb `which fff` &
\end{verbatim}

\subsection{Mailman}\label{sxn:mlm}
\trmidx{Mailman} is used to create and manage mailing lists.
First, run \cmdidx{mailman} to create the ``site list'':
\begin{verbatim}
newlist mailman
\end{verbatim}

\subsection{Web Servers}\label{sxn:www}
The \href{http://dust.ess.uci.edu/dead}{DEAD}
\href{http://dust.ess.uci.edu/dead/bxm_home.html}{box model}
is interactively served on the web. 
Doing so requires coordination between model scripts, the
\trmidx{HTTP} server (\trmidx{Apache}), and system permissions.
The Apache server keeps its transaction logs in
\flidx{/var/log/httpd}.
Users will get an \trmidx{Internal Server Error} if the \trmidx{CGI} 
script fails. 
\begin{verbatim}
tail /var/log/httpd/access_log # RedHat
tail /var/log/httpd/error_log # RedHat
tail /var/log/apache2/access.log # Debian
tail /var/log/apache2/error.log # Debian
\end{verbatim}
Track web server usage using \cmdidx{webalizer}.
\begin{verbatim}
scp ~/linux/etc/webalizer.conf /etc
mkdir -p /var/www/html/usage
chmod -R 755 /var/www/html/usage
webalizer
\end{verbatim}

\subsubsection{WebDAV}\label{sxn:dav}
% http://www.digital-arcanist.com/sanctum/article.php?story=20070427101250622
\trmidx{WebDAV} stands for Web-based Distributed Authoring and
Versioning.
\begin{verbatim}
aptitude install apache2 php5 libapache2-mod-php5
# Add WebDAV Modules
# Activate symlinks to mod_dav installed with Apache 2.x
cd /etc/apache2/mods-enabled
sudo ln -s ../mods-available/dav* .
# Create LockDB file for WebDAV
sudo mkdir /usr/share/apache2/var
sudo touch /usr/share/apache2/var/DAVLock
sudo chown -R www-data:www-data /usr/share/apache2/var
# Setup Authentication
sudo htpasswd -m -c /etc/apache2/.htpasswd zender
# Add Users
# Use this command to add more users later or change their passwords
# Remove the -c option or you'll truncate your password db every time! 
sudo htpasswd -m /etc/apache2/.htpasswd 
# Create the WebDAV Directory
# Make a directory in our web document root where we will run our WebDAV
sudo mkdir /var/www/html/webdav
sudo chown www-data:www-data /var/www/html/webdav
# Configure Apache
# Add definitions to user-defined config file at /etc/apache2/httpd.conf

      ## Location of the DavLock file
      DavLockDB /usr/share/apache2/var/DavLock
             
      ## Set up the webdav directory to use WebDAV and authentication
      <Directory "/var/www/html/webdav">
             	Dav On
              AuthName "WebDAV Login"
              AuthType Basic
              AuthUserFile /etc/apache2/.htpasswd
      	## Limit access for enhanced security
      	<LimitExcept GET HEAD OPTIONS POST>
                      require valid-user
      	</LimitExcept>
              Order allow,deny
              Allow from all
      </Directory>
# Restart Apache and Test
/etc/init.d/apache2 restart
\end{verbatim}

\subsubsection{Group Web Server}\label{sxn:www_grp}
The group relies on a number of non-default webserver features.
Most of these requirements can be met by small modifications of the
\flidx{apache2.conf} or \flidx{httpd.conf} files.
\begin{enumerate}
\item \cmdidx{DirectoryIndex} should include \flidx{index.html} and
  \flidx{index.shtml} 
\item \trmidx{PHP} should be enabled (although not currently used). 
\item Processing of server-side includes (\trmidx{SSI}) should be enabled.
  This handles processing of \flidx{.shtml} files.
  The \flprn{apache2.conf} handles this:
\begin{verbatim}
# ++hjm to include handling for server-parsed files (.shtml)
<IfModule mod_mime.c>
    AddType text/html .shtml
    AddHandler server-parsed .shtml
</IfModule>
\end{verbatim}
\item \cmdidx{DocumentRoot} (top directory) of webpage hierarchy
  should be \flidx{/var/www/html} 
\item DODS/OPeNDAP access must work.
  This may require enabling web-server usage of \trmidx{CGI} scripts 
  in the \flidx{cgi-bin} directory
\item Webserver must re-start automatically on reboots
\end{enumerate}

\subsubsection{ESS Web Server}\label{sxn:www_ess}
The ESS webserver is \mchidx{www.ess.uci.edu}.
It does not support remote \trmidx{SSH} access.
The ESS server, \mchidx{ess1.ess.uci.edu}, allows remote \trmidx{SSH} 
access. 
Both the webserver and the department server are virtual hosts
which appear to be hosted by the same physical hosts.
The physical hosts which support \trmidx{SSH} access include
\mchidx{swamis.ps.uci.edu} and \mchidx{lunada.ps.uci.edu}.
Physical hosts which firewall \trmidx{SSH} include
\mchidx{mavericks.ps.uci.edu}.

\subsection{FTP and firewalls}\label{sxn:ftp}
On Linux, the \cmdidx{ncftp} client may be used to circumvent firewalls.
\cmdprn{ncftp} supports most of the commands as the standard \cmdidx{ftp}
client, but many more intuitive commands as well.
Experience shows that \cmdprn{ncftp} should replace \cmdprn{ftp} in nearly
every situation.
\begin{verbatim}
/bin/rm -r /data/zender/tmp/rpm
mkdir /data/zender/tmp/rpm
cd /data/zender/tmp/rpm
ncftp ftp.uci.edu # This will circumvent NCAR's firewall
cd mirrors/redhat/updates/8.0/en/os/i386
get *rpm
set passive yes # Other firewalls may require this command
bye
sudo /usr/local/bin/uci_dcs_lnx_pch.pl /data/zender/tmp/rpm
\end{verbatim}

\subsection{Accounts}\label{sxn:act}
When GUIs go bad, accounts need to be added by hand.
The following commands work to add accounts and appropriately 
cross-mounted home directories to the cluster:
\begin{verbatim}
sudo groupmod -g 2400 cgdcsm
sudo useradd -D -g cgdcsm
sudo useradd -d /dhome/mflanner -g cgdcsm -n -u 3563 mflanner
sudo usermod -p '$1$charlie$VJFpvyBLLZzEvLkainaFW/' chaoluo
sudo usermod -L daniellj # Lock user password
sudo usermod -a -G adm,admin,audio,cdrom,cgdcsm,dialout,dip,floppy,lpadmin,plugdev,scanner,video robynn # Add groups to access devices on Ubuntu
sudo usermod -a -G sudo pvicente # Add to sudoers
\end{verbatim}
When adding an account to the home directory server itself
(\mchprn{dust.ess.uci.edu}), the default home directory should be
\flprn{/home/\$\{USER\}} rather than \flprn{/dhome/\$\{USER\}}.

The \cmdidx{userdel} command is for \trmidx{deleting user accounts}.
\begin{verbatim}
userdel foobar # Delete account
userdel -r foobar # Same and remove all files in ${HOME} and mail spool
\end{verbatim}

\subsection{NCAR}\label{sxn:ncar}
Over the years \trmidx{NCAR} has developed a number of idiosyncratic
procedures for managing user accounts.
One way to gain access to any NCAR machine is to connect
through the gatekeeper machine, \mchidx{gate.ucar.edu}
\begin{verbatim}
ssh gate.ucar.edu
\end{verbatim}
This machine will open a proxy to any other machine at \trmidx{UCAR}.

\subsection{Autotools}\label{sxn:auto}
The \trmidx{GNU} Autotools refers to an integrated set of software
development and portatibility tools including \trmidx{Libtool}.
\begin{verbatim}
cd ${DATA}
cvs -z3 -d :pserver:anoncvs@subversions.gnu.org:/cvsroot/libtool login
cvs -z3 -d :pserver:anoncvs@subversions.gnu.org:/cvsroot/libtool co -r branch-1-4 libtool
cd ${DATA}/libtool
./bootstrap;./configure --prefix=/usr/local;make;sudo make install
hash -r
cvs -z3 -d :pserver:anoncvs@subversions.gnu.org:/cvsroot/autoconf login
cvs -z3 -d :pserver:anoncvs@subversions.gnu.org:/cvsroot/autoconf co -r AUTOCONF-2_53 autoconf
cvs -z3 -d :pserver:anoncvs@subversions.gnu.org:/cvsroot/m4 login
cvs -z3 -d :pserver:anoncvs@subversions.gnu.org:/cvsroot/m4 co -r m4-1_4o m4
\end{verbatim}

\subsection{SSH}\label{sxn:ssh}
Most Unices use \trmidx{OpenSSH} from the \trmidx{FreeBSD} folks.
Systemwide defaults are set in \flidx{/etc/ssh/ssh\_config}.
The directory \flidx{.ssh} contains authorization files which make 
passwordless access possible.
An \flidx{authorized\_keys} file, if present on machine~A, contains
public keys of users generated on machines B--Z. 
These users will be allowed to log in to machine~A without entering
any password.
A \flidx{known\_hosts} file, if present, contains public host keys of  
known remote machines. 
These keys were generated on remote machines.
\begin{enumerate*}
\item \flprn{authorized\_keys} contains \trmidx{RSA1} public keys of
  authorized users gathered from the \flprn{identity.pub} files on
  remote machines. 
  These keys employ SSH protocol version~1.
\item \flprn{authorized\_keys2} contains \trmidx{RSA} and \trmidx{DSA}
  public keys of authorized users gathered from the \flprn{id\_dsa.pub}
  files on remote machines. 
  These keys employ SSH protocol version~2.
\item \flprn{known\_hosts} contains RSA1 public keys of authorized
  machines gathered from the \flprn{/etc/ssh/ssh\_host\_key.pub} files
  on remote machines. 
  These keys employ SSH protocol version~1.
\item \flprn{known\_hosts2} contains DSA public keys of authorized
  machines gathered from the \flprn{/etc/ssh/ssh\_host\_dsa\_key.pub}
  and  \flprn{/etc/ssh/ssh\_host\_rsa\_key.pub} files on remote machines. 
  These keys employ SSH protocol version~2.
\item Note that the system-wide configuration files (e.g.,
  \flprn{/etc/ssh/ssh\_config}) may disallow recognizing user-specific
  known hosts. 
\end{enumerate*}
Execute \cmdprn{ssh-keygen -t rsa1} to generate \flprn{identity} and \flprn{identity.pub} files.
Execute \cmdprn{ssh-keygen -t rsa} to generate \flprn{id\_rsa} and \flprn{id\_rsa.pub} files.
Execute \cmdprn{ssh-keygen -t dsa} to generate \flprn{id\_dsa} and \flprn{id\_dsa.pub} files. 

When upgrading or renaming a server, one may need to re-generate the
server's keys.
\begin{verbatim}
cd /etc/ssh
sudo ssh-keygen -t rsa1 -f /etc/ssh/ssh_host_key
sudo ssh-keygen -t rsa -f /etc/ssh/ssh_host_rsa_key
sudo ssh-keygen -t dsa -f /etc/ssh/ssh_host_dsa_key
\end{verbatim}

To exercise all the various keys, force \cmdidx{ssh} to use the
different algorithms:
\begin{verbatim}
ssh -1 # Use SSH version 1 protocol only
ssh -2 # Use SSH version 2 protocol only
\end{verbatim}

Recently, OpenSSH added some useful commands.
The command \cmdidx{ssh-copy-id} automates the process of installing
the \flidx{identity.pub} and \flidx{id\_[dr]sa.pub} files in a remote
machine's \flidx{authorized\_keys} and \flidx{authorized\_keys2} files. 
\begin{verbatim}
ssh-copy-id [-i [identity_file]] [user@]machine
\end{verbatim}

Some machines have multiple \trmidx{IP address}es.
\cmdidx{blackforest.ucar.edu} has four IP~addresses.
The entries in the \flprn{known\_hosts} files might conflict with one
another and lead to warnings like ``POSSIBLE DNS SPOOFING DETECTED!''
or ``REMOTE HOST IDENTIFICATION HAS CHANGED!''.
Machines with mulitple IP addresses and multiple host keys should
be entered in the \flidx{known\_hosts} files in one of two ways.
First, one may omit the optional \trmidx{FQDN} from the entry and
specify only the \trmidx{IP address}(es) associated with the host.
When SSH looks at IP addresses only, it cannot get confused.
Second, one may enter the optional FQDN with the \trmidx{IP address},
but, in this case, one must make sure that entries for each possible
IP address are present.
Otherwise the above warnings will result if one tries to use one of
the missing addresses.

Misconfiguration of the login shell, e.g., causing \verb'.bashrc' to
print, will cause \verb'scp' to fail with a \verb'lost connection'
error. 

The \trmidx{NCO} project uses the \trmidx{Sourceforge} shell and
\trmidx{CVS} servers, \mchidx{nco.sf.net} and \mchidx{nco.cvs.sf.net},
respectively. 
Set SSH keys for these machines by uploading keys to the Sourceforge
web interface at \url{https://sourceforge.net/account}. 
Uploading keys directly (e.g., with \cmdidx{scp}) to the shell server
is allowed, but discouraged.
Uploading keys directly to the Sourceforge CVS server is not allowed.

\subsection{Security}\label{sxn:scr}
It happens. 
Passwords get compromised.
When this occurs, the security environment of all computers logically
connected to the compromised account should be rebuilt
First, disable password-less logins from the affected account by
removing the \flidx{authorized\_keys} files from all machines.
This firewalls the machines while the passwords are changed.
Then search the systems for evidence of compromises.
Two packages that check for installation of \trmidx{root-kits} are
\cmdidx{rkhunter} and \cmdidx{chkrootkit}.
Theese may help get discover/eliminate \trmidx{spyware} and
\trmidx{malware} as well.
Login individually to all machines and change the passwords with
\cmdidx{usermod}: 
\begin{verbatim}
# 1. Remove password-less authorization
mch_lst='biogenic.ess.uci.edu dust.ess.uci.edu esmf.ess.uci.edu goldhill.cgd.ucar.edu sand.ess.uci.edu swamis.ps.uci.edu seasalt.ess.uci.edu soot.ess.uci.edu'
for mch in ${mch_lst}; do
    printf "Removing authorized_keys from ${mch}..."
    ssh ${mch} '/bin/rm ${HOME}/.ssh/authorized_keys*'
    printf "done\n"
done
# 2. Change passwords
for mch in ${mch_lst}; do
    ssh ${mch} sudo usermod -p '$1$salt$hashedpassword' zender
done
# 3. Check for compromises
rkhunter
# 4. Change keys
\end{verbatim}

\subsection{Hacks}\label{sxn:hck}
Elite programming (\texttt{l33t}) hacks:
\begin{verbatim}
# Compiling a program in one line
echo -en '#include <stdio.h>\nint main(void){return printf("Hello World\\n");}\n' | gcc -Wall -xc -
\end{verbatim}

\subsection{GCC}\label{sxn:gcc}
The GNU Compiler Collection, \trmidx{GCC}, is the default compiler on 
Linux systems.
To report a GCC bug, send following compile command
\begin{verbatim}
g++ -v -save-temps OPTIONS PROGRAM
\end{verbatim}
and its text output in plain text to \verb'bug-gcc@gnu.org'.
Attach (with MIME) the resulting fully preprocessed file
(\verb'*.i*').
It is OK to compress the file before attaching it.
Do not attach the assembly language file (\verb'*.s*').

\subsection{Groups}\label{sxn:grp}
Make default group \cmdidx{cgdcsm} with \trmidx{GID}~2400.
It is also helpful to have a consistent user~ID or \trmidx{UID} across 
all systems.
My CGD UID is the same as my NCAR scientist number, 3555, as can be
seen on Solaris with the \cmdprn{ypcat passwd | grep zender}.

\subsection{WINE}\label{sxn:wine}
\begin{verbatim}
cd /wnd/Program\ Files/Microsoft\ Office/Office
wine excel.exe
\end{verbatim}

\subsection{Partitioning}\label{sxn:prt}
\trmidx{Linus Torvalds} recommends setting swap partition size equal
to twice the amount of \trmidx{RAM}. 
Table~\ref{tbl:prt} shows the partitioning schemes used on various computers.
\begin{table}
\begin{minipage}{\hsize} % Minipage necessary for footnotes KoD95 p. 110 (4.10.4)
\renewcommand{\footnoterule}{\rule{\hsize}{0.0cm}\vspace{-0.0cm}} % KoD95 p. 111
\begin{center}
\caption[Partitions]{Partition sizes used on various computers}% 
\label{tbl:prt}   
\vspace{\cpthdrhlnskp}
\begin{tabular}{ r *{10}{>{$}r<{$}} } % KoD95 p. 94 describes '*' notation
\hline \rule{0.0ex}{\hlntblhdrskp}% 
Computer & \mathtt{/wnd} & \mathtt{/} & \mathtt{/usr} &
\mathtt{/usr/local} & \mathtt{/home} & \mathtt{/tmp} & \mathtt{swap} &
\mathtt{/data} & \mathtt{/var} & \mathtt{/boot} \\[0.0ex]
& \mathrm{GB} & \mathrm{GB} & \mathrm{GB} & \mathrm{GB} & \mathrm{GB}
& \mathrm{GB} & \mathrm{GB} & \mathrm{GB} & \mathrm{GB} & \mathrm{GB} \\[0.0ex] 
\hline \rule{0.0ex}{\hlntblntrskp}%
seasalt & 5 & 0.5 & 2 & 2 & 2 & 0.5 & 0.5 & 27 & & \\[0.5ex]
dust & & 10 & 20 & & 2 & & & 33 & & \\[0.5ex]
lanina & 5 & 0.1 & 2 & 2 & 3 & 0.2 & 0.4 & 11.5 & & \\[0.5ex]
dakine & & & & & & & & & & \\[0.5ex]
ashes & & & & & & & & & & \\[0.5ex]
biogenic & & & & & & & & & & \\[0.5ex]
elnino & & & & & & & & & & \\[0.5ex]
haze & & & & & & & & & & \\[0.5ex]
soot & & & & & & & & & & \\[0.5ex]
\hline
\end{tabular}
\end{center}
\end{minipage}
\end{table} % end tbl:prt
\begin{enumerate}
\item \flidx{/usr}: 4~GB may not be enough on Debian. 10~GB would be better.
\item \flidx{/boot}: 100~MB is insufficient as having up to five
  kernels is reasonable (i.e., \trmidx{UP} and \trmidx{SMP} versions
  of old and new kernels plus one extra for testing).
\item \flidx{/var}: 1~GB is too small. 
Web server may grow to 5~GB soon. 
Use \cmdidx{aptitude clean} to free space used by Debian package installation.
\item \flidx{/data}: All remaining space.
\item \flidx{/home}: 10~GB seems adequate
\end{enumerate}

% Bibliography
%\renewcommand\refname{\normalsize Publications}
%\nocite{ZeK972}
\bibliographystyle{agu}
\bibliography{bib}
\printindex % Requires makeidx KoD95 p. 221
\addcontentsline{toc}{section}{Index}

\csznote{ 
% Usage: Place usage here at end of file so comment character % not needed
cd ~/sw/linux;make linux.tex;cd -
cd ~/sw/linux;make linux.pdf;cd -
#cd ~/sw/linux;make linux.dvi;cd -
cd ~/sw/linux;make linux.ps;cd -

scp ~/sw/linux/linux.tex ~/sw/linux/linux.txt ${DATA}/ps/linux.ps ${DATA}/ps/linux.pdf ~/sw/linux/linux.dvi dust.ess.uci.edu:Sites/doc/linux

# NB: latex2html chokes on linux.tex
# latex2html -dir Sites/doc/linux linux.tex
# NB: tth chokes on linux.tex
# cd ~/sw/linux;tth -a -Llinux -p./:${TEXINPUTS}:${BIBINPUTS} < ~/sw/linux/linux.tex > linux.html
# scp linux.html dust.ess.uci.edu:Sites/doc/linux
# NB: tex4ht works well on linux.tex
cd ~/sw/linux;htlatex linux.tex
scp linux*.css linux*.html dust.ess.uci.edu:Sites/doc/linux
# NB: tex4moz works well on linux.tex
cd ~/sw/linux;/usr/share/tex4ht/mzlatex linux.tex
scp linux*.css linux*.html linux*.xml dust.ess.uci.edu:Sites/doc/linux
} % end csznote on usage

\end{document}

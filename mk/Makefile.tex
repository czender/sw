# $Id$ -*-Makefile-*-

# Purpose: Makefile for TeX/LaTeX/PDFTeX/PDFLaTeX/LuaTeX/LuaLaTeX document processing

# Usage:
# Put following line in Makefile in local source directory
# include ${HOME}/sw/mk/Makefile.tex

ifndef ${BIBTEX}
 BIBTEX := bibtex
endif
ifndef ${DVIPS}
# See below for notes on dvips and DVIPSFLAGS
 DVIPS := dvips
endif
ifndef ${GZIP}
 GZIP := gzip
endif
ifndef LATEX
# --src-specials embeds inverse search information in DVI file
# Emacs can use inverse search information search DVI files
 LATEX := latex --src-specials
endif # endif LATEX
ifndef ${LATEX2HTML}
 LATEX2HTML := latex2html
endif
ifndef ${MAKEINDEX}
 MAKEINDEX := makeindex
endif
ifndef ${PDFLATEX}
# -shell-escape enables epstopdf.sty to run epstopdf 
 PDFLATEX := pdflatex -shell-escape
endif
ifndef ${LUALATEX}
# -shell-escape enables epstopdf.sty to run epstopdf 
 LUALATEX := lualatex
endif
# http://www.cs.wisc.edu/~ghost/doc/AFPL/7.04/Ps2pdf.htm
ifndef ${PS2PDF}
 PS2PDF := 'ps2pdf -dMaxSubsetPct=100 -dCompatibilityLevel=1.2 -dSubsetFonts=true -dEmbedAllFonts=true -sAutoRotatePages=PageByPage -sColorConversionStrategy=LeaveColorUnchanged'
# An alternative posted to comp.text.tex 20040825
# PS2PDF := gs -dSAFER -dNOPAUSE -dBATCH -sDEVICE=pdfwrite -sPAPERSIZE=letter -dPDFSETTINGS=/printer -dCompatibilityLevel=1.3 -dMaxSubsetPct=100 -dSubsetFonts=true -dEmbedAllFonts=true -sOutputFile=myfile.pdf myfile.ps
endif
ifndef ${PSDRAFT}
 PSDRAFT := psdraft
endif
ifndef ${TEX}
 TEX := tex
endif
ifndef ${THB}
 THB := Y
endif
ifndef ${THUMBPDF}
 THUMBPDF := thumbpdf
endif
ifndef ${UNTEX}
 UNTEX := untex
endif
ifndef ${PDFTOTEXT}
 PDFTOTEXT := pdftotext
endif
ifndef ${OPTS}
 OPTS := 
endif

ifndef ${DVIPSFLAGS}
# Theoretically, -Ppdf causes dvips to produce nice scalable fonts rather than bitmap fonts
# My experience is that -Ppdf produces worse output (on screen) in gv
# Some European distributions default to A4 papertype
# -t letter: Set papertype to letter 
# Most documents use a combination of fonts.
# Type 1 fonts 
# Type 3 fonts.
# 
# -Ppdf: Convert everything to Type 1 fonts. Looks great in acroread, OK in ghostview.
# Type 3 fonts look great in acroread and great in ghostview.
# -G0: Suppress implicit switch activated by -Ppdf causing weird ligatures comp.text.tex 20021010
 DVIPSFLAGS := -t letter -Ppdf -G0
endif
ifeq (${OPTS},L)
 DVIPSFLAGS += -t landscape
endif

# Disable standard pattern rules
%.dvi: %.tex

# Do not delete following targets
.PRECIOUS: %.aux %.bbl %.eps %.ind

# Handle citations for bibliography
# .aux rule is long and sophisticated
%.aux: %.tex
# \\nonstopmode causes LaTeX to keep running even if errors are encountered
# It seems counterproductive to run in \\nonstopmode, however since errors are hard to notice
# Where are all these modes documented?
#	@${LATEX} \\nonstopmode \\input{$*}
	@${LATEX} \\input{$*}
# Look for citations and ensure grep never returns an error code
	@grep "^\\\\citation" *.aux > .btmp.new || true
# If citations are unchanged, do nothing, otherwise replace .btmp file to make sure Bibtex will be run
	@if ( diff .btmp.new .btmp >& /dev/null ); then rm .btmp.new; else mv .btmp.new .btmp; fi
# If numbering is unchanged, do nothing, otherwise replace .itmp file to make sure makeindex will be run
	@if [ -f $*.idx ]; then cp $*.idx .itmp.new; else touch .itmp.new; fi
	@if ( diff .itmp.new .itmp >& /dev/null ); then rm .itmp.new; else mv .itmp.new .itmp; fi

.btmp:

%.bbl: ${BIBFILES} .btmp
# Only use BibTeX if \bibliography occurs in document
# In that case, run BibTeX and recompile
# .btmp is touched to prevent useless making next time
	@if ( grep "^\\\\bibliography{" $*.tex > /dev/null ); then ${BIBTEX} $*; touch .rerun; fi
	@touch .btmp

.itmp:

%.ind: .itmp
	@if [ -f $*.idx ]; then ${MAKEINDEX} $*; touch .rerun; touch .itmp; fi

%.dvi: %.aux %.ind %.bbl
# Make sure DVI file exists; if not then recompile
	@if [ ! -f $*.dvi ]; then touch .rerun; fi
	@if [ -f .rerun ]; then rm .rerun; ${LATEX} \\input{$*}; else ${MAKE} $*.aux; fi
# While references et al. are changed then recompile
	@while ( grep Rerun $*.log > /dev/null ); do ${LATEX} \\input{$*}; done
	@if [ -f $*.ind ]; then touch $*.ind; fi

# %.dvi : %.tex
# 	-${LATEX} $*.tex
# 	-${BIBTEX} $*
# 	-${MAKEINDEX} $*

%.html : %.tex
	-mkdir ${HOME}/public_html/$*
	-${LATEX2HTML} $*.tex

## Method tested starting 20040410: Treats bibliographies intelligently, like .dvi
#%.pdf : %.aux %.ind %.bbl 
## Make sure PDF file exists; if not then recompile
#	@if [ ! -f ${DATA}/ps/$*.pdf ]; then touch .rerun; fi
#	@if [ -f .rerun ]; then rm .rerun; ${PDFLATEX} \\input{$*}; else ${MAKE} $*.aux; fi
## While references et al. are changed then recompile
#	@while ( grep Rerun $*.log > /dev/null ); do ${PDFLATEX} \\input{$*}; done
#	@if [ -f $*.ind ]; then touch $*.ind; fi

# Method used until 20040410: Is not smart about bibliographies
# In any case, using thumbpdf.sty package may obsolete running thumbpdf 
%.pdf : %.tex
	-${LUALATEX} $*.tex
	-${BIBTEX} $*
	-${MAKEINDEX} $*
	-${LUALATEX} $*.tex
	-${BIBTEX} $*
	-${MAKEINDEX} $*
	-${LUALATEX} $*.tex
	-mv -f $*.pdf ${DATA}/ps

# NSF FastLane recommends
# dvips -Ppdf -o <file>.ps <file>.dvi 
# The -Ppdf causes ugliness in ghostview
# Use ps2pdf
# %.pdf : ${DATA}/ps/%.ps
# 	-${LATEX} $*.tex
# 	-${BIBTEX} $*
# 	-${MAKEINDEX} $*
# 	${DVIPS} ${DVIPSFLAGS} $*.dvi -o ${DATA}/ps/$*.ps
# 	-${PS2PDF} ${DATA}/ps/$*.ps ${DATA}/ps/$*.pdf

%.ps : %.dvi
	${DVIPS} ${DVIPSFLAGS} $< -o ${DATA}/ps/$*.ps
#	${PSDRAFT} ${DATA}/ps/$*.ps > ${DATA}/ps/foo.ps
#	/bin/mv -f foo.ps $*.ps

# Convert Postscript seminar to Postcard mode Postscript format
%_pst.ps : %.ps
	mpage -R -4 ${DATA}/ps/$*.ps > ${DATA}/ps/$*_pst.ps

# Convert Postcard mode Postscript format to Postcard mode PDF format
%_pst.pdf : %_pst.ps
	${PS2PDF} ${DATA}/ps/$*_pst.ps ${DATA}/ps/$*_pst.pdf

%.ps.gz : %.ps
	-${GZIP} --verbose --force ${DATA}/ps/$*.ps

%.pdf.gz : %.pdf
	-${GZIP} --verbose --force ${DATA}/ps/$*.pdf

# Pattern rules to build tex files
%.tpt : %.tex
	-${THUMBPDF} ${DATA}/ps/$*.pdf
	-mv -f ${DATA}/ps/$*.tpt .

%.txt : ${DATA}/ps/%.pdf
	-${PDFTOTEXT} ${DATA}/ps/$*.pdf > $*.txt
	-mv -f ${DATA}/ps/$*.txt .

%.txt : %.tex
	-${UNTEX} $*.tex > $*.txt

# Byte-compile elisp files
%.elc: %.el
	emacs -batch -f batch-byte-compile $*.el

all: 

tags: 
	etags *.tex *.sty *.txt *.html *.shtml Makefile README ChangeLog TODO

clean: cln
cln: 
	rm -f *.aux *.bbl *.blg *.brf *.cb *.dvi *.glo *.idx *.ilg *.ind *.inx *.log *.lot *.out *.pdf *.ps *.toc *.tpt

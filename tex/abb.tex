% $Id$

% Purpose: Acronyms and abbreviations

\chapter{Acronyms and abbreviations}\label{app:abb}
%\begin{tabular}{r l}
\begin{longtable}[l]{r l}
ACC & Arctic Circumpolar Current \\
ACWP & Adiabatic Condensed Water Profile \\
ALW & Adiabatic Liquid Water \\
AMIP & Atmospheric Model Intercomparison Project \\
ANV & ``Anvil'' ice cloud parameterization \\
APE & Available Potential Energy \\
ARESE & ARM Enhanced Shortwave Experiment \\
ARM & Atmospheric Radiation Measurement Program \\
ASASX & Active Scattering Aerosol Spectrometer Probe \\
BSA & Black Sky Albedo (albedo to direct beam) \\
WSA & White Sky Albedo (albedo to diffuse radiation) \\
NBAR & Nadir-viewing BRDF-Adjusted Reflectance \\
CAPE & Convectively Available Potential Energy \\
CCM & NCAR Community Climate Model \\
CEM & Cumulus Ensemble Model \\
CEP & Central Equatorial Pacific \\
CEPEX & Central Equatorial Pacific Experiment \\
CISK & Conditional Instability of the Second Kind \\
CIMAS & Cooperative Institute for Marine and Atmospheric Studies \\
COARE & Coupled Ocean-Atmosphere Response Experiment \\
CTEI & Cloud Top Entrainment Instability \\
CWC & Condensed Water Content \\
CWP & Condensed Water Path \\
DIDECUP & DIurnal DECoUPling \\
ECHAM & ECmwf/HAMburg GCM \\
ECMWF & European Center for Medium Range Weather Forecasts \\
EEP & Eastern Equatorial Pacific \\
ENSO & El Ni\~no Southern Oscillation \\
ERBE & Earth Radiation Budget Experiment \\
FCT & Flux Corrected Transport \\
FSBR & Fractional Solar Broadband Radiometer \\
FSSP & Forward Scattering Spectrometer Probe \\
GCM & General Circulation Model \\
HPMM & Habit Preserving Microphysical (cirrus cloud) Model \\ 
ICRCCM & InterComparison of Radiation Codes in Climate Models \\
IFN & Ice Freezing Nuclei \\
IR & Infrared (radiation) \\
ISCCP & International Satellite Cloud Climatology Project \\
ITCZ & Inter-Tropical Convergence Zone \\
IWC & Ice Water Content \\
IWP & Ice Water Path \\
JJA & June-July-August \\
LDL & Lower Detection Limit \\
LHS & Left Hand Side \\
LW & Longwave (radiation) \\
LWC & Liquid Water Content \\
LWCF & Longwave Cloud Forcing \\
LWP & Liquid Water Path \\
MAM & March-April-May \\
MCC & Mesoscale Convective Complex \\
NADW & North Atlantic Deep Water \\
NCAR & National Center for Atmospheric Research \\
NH & Northern Hemisphere \\
NHS & Northern Hemisphere Summertime \\
NHW & Northern Hemisphere Wintertime \\
NTP & Northwest Tropical Pacific \\
OLR & Outgoing Longwave Radiation \\
PBL & Planetary Boundary Layer \\
PNA & Pacific North American \\
RH & Relative Humidity \\
RHS & Right Hand Side \\
RSMAS & Rosenstiel School of Marine and Atmospheric Science \\
SCM & Single Column Model \\ 
SRCM & Size Resolving Column Model \\ 
SEB & Surface Energy Budget \\
SH & Southern Hemisphere \\
SHS & Southern Hemisphere Summertime \\
SHW & Southern Hemisphere Wintertime \\
SIROS & Solar and Infrared Observing System \\
SLT & Semi-Lagrangian Transport \\
SMF & Simple Mass Flux scheme (for moist convection) \\
SOI & Southern Oscillation Index \\
SPCP & Semi-Prognostic Cirrus Parameterization \\
SRB & Surface Radiation Budget \\
SSMI & Special Sensor Microwave/Imager \\
SST & Sea Surface Temperature \\
STEP & Stratosphere-Troposphere Exchange Project \\
SW & Shortwave (radiation) \\
SWCF & Shortwave Cloud Forcing \\
TCP & Tropical Central Pacific \\
TEP & Tropical Eastern Pacific \\
TOA & Top of Atmosphere \\
TOGA & Tropical Ocean Global Atmosphere \\
TSBR & Total Solar Broadband Radiometer \\
TUT & Tropical Upper Troposphere \\
TWP & Tropical Western Pacific \\
UKMO & United Kingdom Meteorological Office \\
WEP & Western Equatorial Pacific \\
WISHE & Wind-Induced Surface Heat Exchange \\ 
WMONEX & Winter Monsoon Experiment \\
WPWP & Western Pacific Warm Pool \\
\end{longtable}
%\end{tabular}



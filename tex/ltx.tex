% $Id$ -*-LaTeX-*-

% Purpose: Template for LaTeX files

% Copyright (c) 1998--present, Charles S. Zender
% Permission is granted to copy, distribute and/or modify this document
% under the terms of the GNU Free Documentation License (GFDL), Version 1.3
% or any later version published by the Free Software Foundation;
% with no Invariant Sections, no Front-Cover Texts, and no Back-Cover Texts.
% GFDL: http://www.gnu.org/copyleft/fdl.html

% The original author of this software, Charlie Zender, seeks to improve
% it with your suggestions, contributions, bug-reports, and patches.
% Charlie Zender <zender at uci dot edu>
% Department of Earth System Science
% University of California, Irvine
% Irvine, CA 92697-3100

% URL: http://dust.ess.uci.edu/doc/ltx/ltx.pdf

% Usage (see also end of file):
% cd ~/sw/tex;make ltx.pdf;cd -

\documentclass[12pt]{article}

% Standard packages
\usepackage{ifpdf} % Define \ifpdf
\ifpdf % PDFLaTeX
\usepackage{graphicx} % Defines \includegraphics*
%\usepackage[pdftex]{graphicx} % Defines \includegraphics*
%\pdfcompresslevel=9
%\usepackage{thumbpdf} % Generate thumbnails
%\usepackage{epstopdf} % Convert .eps, if found, to .pdf when required
\graphicspath{/Users/zender/data/ps} % Help epstopdf find .eps figures to convert
\else % !PDFLaTeX
\usepackage{graphicx} % Defines \includegraphics*
\fi % !PDFLaTeX
\usepackage{amsbsy} % \boldsymbol, \pmb
\usepackage{amsmath} % \subequations, \eqref, \align
\usepackage{amstext} % \text, \intertext
\usepackage{array} % Table and array extensions, e.g., column formatting
\usepackage{bm} % Boldface Greek letters
\usepackage{color} % Colors
\usepackage[dayofweek]{datetime} % \xxivtime, \ordinal
\usepackage[first]{draftcopy} % Blaze ``Draft'' across [none,first,all,etc.] pages
%\usepackage{dropcaps} % Drop first letter of paragraph, \bigdrop
\usepackage{epigraph} % Pithy quotes at section starts, \epigraph
\usepackage{etoolbox} % \newbool, \setbool, \ifxxx
\usepackage{fancyhdr} % Headers and footers
%\usepackage[T1]{fontenc} % Used fxm
\usepackage{lettrine} % Drop first letter of paragraph, \lettrine
\usepackage{lineno} % Line numbering \linenumbers, \nolinenumbers, \pagewiselinenumbers
\usepackage{listings} % Source code printing, \lstinline, \lstset
\usepackage{longtable} % Multi-page tables, e.g., acronyms and symbols
\usepackage{lscape} % Landscape environment
\usepackage{makeidx} % Index keyword processor: \printindex and \see
\usepackage{mdwlist} % Compact list formats \itemize*, \enumerate*
\usepackage{mflogo} % METAFONT logo
\usepackage{natbib} % \cite commands from aguplus
%\usepackage{pslatex} % Use Times, Helvetica and Courier fonts wherever possible
\usepackage[inactive]{srcltx} % Enable dvi <-> tex searches fxm: not working
\usepackage{subfigure} % \subfigure for panel labeling within figure environment
%\usepackage{texnames} % Typeset AMSTeX, METAFONT, BiBTeX (poorly coded)
\usepackage[safe]{tipa} % International Phonetic Alphabet
% \usepackage{times} % Postscript Times-Roman font KoD99 p. 375 20170927: prevents non-Times fonts with LuaLaTex
\usepackage{tocbibind} % Add Bibliography and Index to Table of Contents
\usepackage{type1cm} % Type 1 fonts are scalable
%\usepackage[Euler]{upgreek} % Upright Greek letters [Euler,Symbol,Symbolsmallscale]
\usepackage[Symbol]{upgreek} % Upright Greek letters [Euler,Symbol,Symbolsmallscale]
%\usepackage[Symbolsmallscale]{upgreek} % Upright Greek letters [Euler,Symbol,Symbolsmallscale]
\usepackage{ulem} % Strikeout text
\usepackage{url} % Typeset URLs and e-mail addresses
\usepackage{verbatim} % comment environment
% \usepackage{yfonts,color} % Causes \gothfamily already defined errors
% fxm: 20001028 ${TEXMFDIR}/tex/latex/base/showidx.sty breaks hyperref
%\usepackage{showidx} % Print index entries as marginal notes

% HTML, XML discussed on KoD04 p. 457--459
% cd ~/tex;latex ltx.tex;tex4ht ltx.dvi;cd -
% NB: htlatex does better job and does not require enabling any packages
%\usepackage[html]{tex4ht} % HTML output
%\usepackage[xhtml,docbook]{tex4ht} % XML, DocBook output
%\usepackage[xhtml,tei]{tex4ht} % XML, TEI output
%\usepackage[xhtml,mathml]{tex4ht} % XML, MathML output

% Fonts
\usepackage[weather]{ifsym} % Various symbols: Weather
\usepackage{amssymb} % amsfonts \mathbb KoD99 p. 434
\usepackage{dsfont} % Doublestroke \mathds font
%\usepackage{eufrak} % \mathfrak KoD99 p. 434 # 20250821: eufrak is redundant if amsfonts is used
%\usepackage{marvosym} % Martin Vogel's Science symbols: \Aries, \Radioactivity, \Sun # 20071021: \Sun conflicts with texlive on gutsy gibbon
\usepackage{mathrsfs} % \mathscr Ralph Smith's Formal Script
\usepackage{oldgerm} % Old German: \textgoth, \textfrak, \textswab
%\usepackage{wasysym} % Science symbols: \photon, \neptune...
%\usepackage[mathscr]{eucal} % \mathscr KoD99 p. 434
%\usepackage{avant} % Postscript AvantGarde font KoD99 p. 375
%\usepackage{bookman} % Postscript Bookman font KoD99 p. 375
%\usepackage{eso-pic} % Required for Draft (\AddToShipoutPicture)
%\usepackage{helvet} % Postscript Helvetica font KoD99 p. 375
%\usepackage{latexsym} % Extra LaTeX symbols (\leadsto, \Box...)
%\usepackage{newcent} % Postscript Palatino font KoD99 p. 375
%\usepackage{palatino} % Postscript Palatino font KoD99 p. 375
%\usepackage{pxfonts} % \varmathbb font
\usepackage{textcomp} % Text companion symbols (nicer \texttrademark, \textregistered, \textonehalf)
%\usepackage{times} % Postscript Times-Roman font KoD99 p. 375
%\usepackage{txfonts} % Math symbols: \piup...

% hyperref is last package since it redefines other packages' commands
% hyperref options, assumed true unless =false is specified:
% backref       List citing sections after bibliography entries
% baseurl       Make all URLs in document relative to this
% bookmarksopen Unknown
% breaklinks    Wrap links onto newlines
% colorlinks    Use colored text for links, not boxes
% hyperindex    Link index to text
% plainpages=false Suppress warnings caused by duplicate page numbers
% pdftex        Conform to pdftex conventions
% Colors used when colorlinks=true:
% linkcolor     Color for normal internal links
% anchorcolor   Color for anchor text
% citecolor     Color for bibliographic citations in text
% filecolor     Color for URLs which open local files
% menucolor     Color for Acrobat menu items
% pagecolor     Color for links to other pages
% urlcolor      Color for linked URLs
\ifpdf % PDFLaTeX
%\usepackage[colorlinks,linkcolor=blue,urlcolor=blue]{hyperref}
\usepackage[backref,breaklinks,colorlinks,hyperindex,citecolor=blue,linkcolor=blue,urlcolor=blue,plainpages=false]{hyperref} % Hyper-references
%\pdfcompresslevel=9
\else % !PDFLaTeX
%\usepackage[colorlinks,linkcolor=blue,urlcolor=blue]{hyperref}
\usepackage[backref,breaklinks,colorlinks=false,plainpages=false]{hyperref} % Hyper-references
\fi % !PDFLaTeX

% preview-latex recommends it be last-activated package
%\usepackage[inactive,displaymath]{preview} % preview-latex equation extraction
\usepackage[showlabels,sections,floats,textmath,displaymath]{preview} % preview-latex equation extraction

% Personal packages
\usepackage{csz} % Library of personal definitions
\usepackage{abc} % Alphabet as three letter macros
\usepackage{dmn} % Dimensional units
\usepackage{chm} % Commands generic to chemistry
\usepackage{dyn} % Commands generic to fluid dynamics
\usepackage{aer} % Commands specific to aerosol physics
\usepackage{rt} % Commands specific to radiative transfer
\usepackage{psd} % Particle size distributions
\usepackage{hyp} % Hyphenation exception list
\usepackage{jrn_agu} % AGU-sanctioned journal abbreviations

% Commands which must be executed in preamble
\DeclareGraphicsRule{.gif}{eps}{}{`gif2eps #1 -} % Convert GIF files to eps
\allowdisplaybreaks[1] % AMSTeX permits eqnarray to wrap pages [1--4]
\listfiles % Print required files, versions
\makeglossary % Glossary described on KoD95 p. 221
\makeindex % Index described on KoD95 p. 220

% Commands specific to this file
% 1. Fundamental commands
% TeX-definition of CVS macros, e.g., \CVSRevision, \CVSDate
\def\CVS$#1: #2 ${\expandafter\def\csname CVS#1\endcsname{#2}} % CVS macros
\newcommand{\vrbcmdfmt}[2]{\texttt{\\#1\{}\textit{#2}\texttt{\}}} % Format verbatim commands with arguments in italics
\DeclareMathAlphabet{\mathpzc}{OT1}{pzc}{m}{it} % Adobe Zapf Chancery font \mathpzc (TeX FAQ #102)
\DeclareMathAlphabet{\mathsfsl}{OT1}{cmss}{m}{sl} % Slanted sans serif math font for matrices c.t.tex 20080515

\newcommand{\tensorb}[1]{\textsl{\textsf{\bfseries{#1}}}} % Tensors bold sans serif italic
\newcommand{\tensor}[1]{\textsl{\textsf{#1}}} % Tensors sans serif italic

% 2. Derived commands
% 3. Doubly-derived commands

% Margins
\topmargin -24pt \headheight 12pt \headsep 12pt
\textheight 9in \textwidth 6.5in
\oddsidemargin 0in \evensidemargin 0in
%\marginparwidth 0pt \marginparsep 0pt
\setlength{\marginparwidth}{1.5in} % Width of callouts of index terms and page numbers KoD95 p. 220
\setlength{\marginparsep}{12pt} % Add separation for index terms KoD95 p. 220
\footskip 24pt
\footnotesep=0pt

\begin{document} % End preamble

\ifpdf % PDFLaTeX
% 20170927: LuaLaTex does not have \pdfinfo{}
%\pdfinfo{ % A command provided by PDFLaTeX
%/Title (Template for LaTeX files)
%/Subject (Template for LaTeX files)
%/Author (Charlie Zender)
%/Keywords (ltx)
%} % end \pdfinfo
% \pdfcatalog{ % A command provided by PDFLaTeX
% /URI (http://www.ess.uci.edu/zender/ltx) % Base URL of document
% } % end \pdfcatalog
\fi % !PDFLaTeX
\hypersetup{ % A command provided by \hyperref
pdftitle={Template for LaTeX files},
pdfsubject={Template for LaTeX files},
pdfauthor={Charlie Zender},
pdfkeywords={ltx}
} % end \hypersetup

\begin{center}
Online: \url{http://dust.ess.uci.edu/doc} \hfill Built: \shortdate\today, \xxivtime\\
\bigskip
{\Large \textbf{\LaTeX\ Cheat Sheet}}\\
\bigskip
by Charlie Zender\\
University of California, Irvine\\
\end{center}
Department of Earth System Science \hfill \url{zender@uci.edu}\\
University of California \hfill Voice: (949)\thinspace 824-2987\\
Irvine, CA~~92697-3100 \hfill Fax: (949)\thinspace 824-3256

% GFDL legalities: http://www.gnu.ai.mit.edu/copyleft/fdl.html
\bigskip\noindent
Copyright \copyright\ 2000--present, Charles S. Zender\\
Permission is granted to copy, distribute and/or modify this document
under the terms of the GNU Free Documentation License, Version~1.2
or any later version published by the Free Software Foundation;
with no Invariant Sections, no Front-Cover Texts, and no Back-Cover
Texts.
The license is available online at
\url{http://www.gnu.org/copyleft/fdl.html}.

\pagenumbering{roman}
\setcounter{page}{1}
\pagestyle{headings}
\thispagestyle{empty}
%\onecolumn
\tableofcontents
\listoffigures
\listoftables
\pagenumbering{arabic}
\setcounter{page}{1}
%\markleft{Radiative Transfer}
%\markright{}
\thispagestyle{empty}

\section{Introduction}\label{sxn:ntr}
\cszepigraph{Just a box of rain---\\Wind and water---\\Believe it if you need it,\\if you don't just pass it on}{http://arts.ucsc.edu/gdead/agdl/box.html}{Box of Rain}{Robert Hunter}

\lettrine[lines=3]{\color{red}S}{omeone}
%\lettrine[lines=3]{S}{omeone}
%\bigdrop{0pt}{3}{cmr10}{S}omeone
once said that the best way to learn a language was to speak it.
This is my attempt to learn \LaTeX, one of the most challenging
and useful languages ever devised.
This section is devoted to introductory material and matters 
generic to typography.
Section~\ref{sxn:nst} describes \LaTeX\ installation and maintenance,
Section~\ref{sxn:ltx} contains hints on using \LaTeX,
Section~\ref{sxn:txt} is devoted to text typography,
Section~\ref{sxn:mth} covers formatting mathematics.

\cmdltxidx{\foo} is undefined
Incidentally, the first letter of the preceding paragraph was
``dropped'' by calling the \cmdltxidx{\lettrine} macro defined by the
\flidx{lettrine.sty} package:
\cmdltxprn{\lettrine[lines=3]{S}{omeone}}\ldots. 
%\cmdltxprn{\bigdrop{0pt}{3}{cmr10}{S}omeone}\ldots. 
A newer package capable of handling large first letters of paragraphs
is the \flidx{lettrine.sty} package.
Most of the documentation for \textsf{lettrine} appears to be in
French, however. 

\section[\LaTeX\ Installation]{\LaTeX\ Installation}\label{sxn:nst}
This section describes \LaTeX\ installation, maintenance, and
upgrades. 
Thanks to Thomas Esser for producing \cmdprn{tetex}, the \TeX\
distribution that I use.

Many Olympian \TeX\ and \LaTeX\ gurus do not like/use the RedHat
GNU/Linux installation.  
Apparently RedHat does not change the \verb'tetex' defaults from their  
European settings. 
This may cause vertical margins to change in weird ways, e.g., in
printouts but not in \cmdprn{ghostview}, or in \cmdprn{acroread} but
not in \cmdprn{xdvi}. 
When this occurs one solution is to reconfigure \TeX\ by running
\cmdprn{texconfig} as root:
\begin{verbatim}
sudo texconfig dvips paper letter
sudo texconfig xdvi us
\end{verbatim}
Also, beause of their European heritage, most Ubuntu \LaTeX\
distributions default to European papersizes.
Use \cmdidx{texconfig} to change the paper format:
\begin{verbatim}
sudo texconfig # Try this first
sudo texconfig-sys # Some report that this is necessary, too
\end{verbatim}
It has also been reported that 
It may also be necessary to change the value of \flidx{/ete/papersize} 
from ``A4'' to ``letter''.

\subsection[Personal packages]{Personal Packages}\label{sxn:prs}
The \cmdprn{TEXINPUTS} and \cmdprn{BIBINPUTS} environmental variables
determine the \TeX\ search path.
These should be set in startup files, e.g., \flidx{.bashrc}:
\begin{verbatim}
export TEXINPUTS=".:${HOME}/texmf//:${HOME}/crr//:${DATA}/ps//: \
        ${TEXMFDIR}/pdftex//:${TEXMFDIR}/tex//::" 
export BIBINPUTS=".:${HOME}/texmf//:${TEXMFDIR}/bibtex//::"
\end{verbatim}
% $: re-balance syntax highlighting
If these variables are not set then packages must be in the current
directory or the default system search paths.
The \verb'::' activates the default system search paths.
The \verb'//' indicates that directories beneath the specified
directory should be searched recursively. 
These symbolic links should not be necessary when \verb'TEXINPUTS'
contains \verb'${HOME}/tex//':
% $: re-balance syntax highlighting
\begin{verbatim}
sudo ln -s ${HOME}/tex/ncarletter.cls ${HOME}/tex/cls/ncarletter.cls
sudo ln -s ${HOME}/tex/zenletter.cls ${HOME}/tex/cls/zenletter.cls
sudo ln -s ${HOME}/tex/bib.bib ${HOME}/tex/cls/bib.bib
sudo ln -s ${HOME}/tex/csz.sty ${HOME}/tex/cls/csz.sty
sudo ln -s ${HOME}/tex/jgr_abb.tex ${HOME}/tex/cls/jgr_abb.tex
sudo ln -s ${HOME}/tex/cls ${TEXMFDIR}/tex/latex/csz
sudo ln -s ${HOME}/tex/bst ${TEXMFDIR}/bibtex/bst/csz
sudo texhash
\end{verbatim}

\subsection[Ubuntu]{Ubuntu}\label{sxn:bnt}
The command \cmdidx{kpsewhich} identifies which version of a
given file \LaTeX\ will use: 
\begin{verbatim}
% kpsewhich agu04.bst
/home/zender/texmf/bibtex/bst/csz/agu04.bst
\end{verbatim}
Personal \TeX\ updates placed in \flidx{~/texmf} will override all
other versions installed in the system.
For example, place \flprn{agu04.bst} below \flprn{~/texmf/bibtex/bst}.
On Mac OS~X, this directory is \flidx{~/Library/texmf}.
There I follow the below (Linux) prescription then do
\begin{verbatim}
cd ~/Library/
ln -s /Users/zender/texmf texmf
\end{verbatim}
Mac OS~X also has problems finding ifsym.sty, dsfont.sty, yinit.

Commands to copy/link an ``old-style'' environment to leverage this
functionality are: 
\begin{verbatim}
# Copy 3rd party packages below ~/texmf/tex/latex
mkdir -p ~/texmf/tex/latex/revnum
scp 'dust.ess.uci.edu:${DATA}/tex/revnum/revnum.sty' ~/texmf/tex/latex/revnum
mkdir -p ~/texmf/tex/latex/dchem
scp 'dust.ess.uci.edu:tex/cls/dchem.sty' ~/texmf/tex/latex/dchem

# Place links to personal packages in ~/texmf/tex/latex/csz
mkdir -p ~/texmf/tex/latex/csz
for fl in csz uci_ltr ; do
    /bin/ln -s ~/sw/tex/${fl}.sty ~/texmf/tex/latex/csz/${fl}.sty
done
cd ~/sw/crr
for fl in `ls *.sty` ; do
    /bin/ln -s ~/sw/crr/${fl} ~/texmf/tex/latex/csz/${fl}
done

# Place links to bibliography and styles in ~/texmf/bibtex/bst/csz
mkdir -p ~/texmf/bibtex/bst/csz
for fl in agu agu04 jas jqsrtcsz unsrtnat ; do
    /bin/cp ~/sw/tex/bst/${fl}.bst ~/texmf/bibtex/bst/csz/${fl}.bst
done
/bin/ln -s ~/sw/tex/bib.bib ~/texmf/bibtex/bst/csz/bib.bib
\end{verbatim}

\subsection[Hyphenation]{Hyphenation}\label{sxn:hyp}
Check to be sure \trmidx{hyphenation} works when documents look poorly hyphenated 
\trmidx{Ubuntu} teTeX upgrades have a habit of breaking hyphenation.
The other symptom (besides bad hyphenation) is the header line
``Babel <v3.8d> and hyphenation patterns for loaded''.
A list of languages should be between ``for loaded''.
Somehow this file gets corrupted:
\href{http://www.mail-archive.com/desktop-bugs@lists.ubuntu.com/msg29134.html}
These commands may help:
\begin{verbatim}
sudo mv /etc/texmf/language.d/00tetex.cnf /etc/texmf/language.d/10tetex.cnf
sudo update-language
sudo fmtutil-sys --all
\end{verbatim}

\section[Output Formats]{Output Formats}\label{sxn:out}
Using \LaTeX\ to produce PDF files is lots of fun.
There is an entire \trmidx{NSF} website devoted to this topic.

\subsection[ps2pdf]{ps2pdf}\label{sxn:ps2pdf}
The \cmdidx{ps2pdf} program converts complex \trmidx{Postscript} files
into \trmidx{PDF} files.
\cmdprn{ps2pdf} is the \trmidx{Ghostscript} replacement for
\trmidx{Adobe Distiller}.
A distiller killer, you might say.
It is highly configurable; full details are available 
\href{http://www.cs.wisc.edu/~ghost/doc/AFPL/7.04/Ps2pdf.htm}{here}.
\begin{verbatim}
ps2pdf -dMaxSubsetPct=100 -dCompatibilityLevel=1.2 -dSubsetFonts=true \
-dEmbedAllFonts=true -sAutoRotatePages=PageByPage \
-sColorConversionStrategy=LeaveColorUnchanged in.ps out.pdf
\end{verbatim}
The \cmdidx{AutoRotatePages} and \cmdidx{ColorConversionStrategy}
switches are important when converting slide presentations.
The \cmdidx{EmbedAllFonts} option tells \cmdprn{ps2pdf} to embed all 
fonts in the output file.
This makes the output file more portable.

The \trmidx{ghostscript} command to convert \trmprn{Postscript} into  
\trmprn{PDF} was posted to \texttt{comp.text.tex} on 20040825:
\begin{verbatim}
gs -dSAFER -dNOPAUSE -dBATCH -sDEVICE=pdfwrite -sPAPERSIZE=letter \
-dPDFSETTINGS=/printer -dCompatibilityLevel=1.3 -dMaxSubsetPct=100 \
-dSubsetFonts=true -dEmbedAllFonts=true -sOutputFile=foo.pdf \
foo.ps 
\end{verbatim}

With \cmdidx{teTeX}, one can tell all \cmdidx{DVI} manipulators
(including \cmdidx{pdftex}, \cmdidx{xdvi}, \cmdidx{dvips}) to 
embed fonts by using one command.
The \cmdidx{updmap} command alters the font configuration files 
of all these programs under \cmdprn{teTeX}:
\begin{verbatim}
sudo updmap --setoption pdftexDownloadBase14 true
\end{verbatim}

\subsection[a2ps]{a2ps}\label{sxn:a2ps}
The \cmdidx{a2ps} program converts text files to \trmidx{Postscript}
\begin{verbatim}
a2ps --no-header fl.txt
a2ps --no-header fl.txt -o fl.ps
a2ps --columns=80 --font-size=12 --lines-per-page=52 --no-header fl.txt -o fl.ps
\end{verbatim}

\subsection{mpage}\label{sxn:mpage}
The \cmdidx{mpage} program combines multiple pages onto one page:
\begin{verbatim}
mpage -R -8 -Phplj5l ${DATA}/ps/ess_atm_lct_01.ps
mpage -R -8 ${DATA}/ps/ess_atm_lct_03.ps > ${DATA}/tmp/lct.ps
mpage -R -4 ${DATA}/ps/ess_atm_lct_01.ps > ${DATA}/tmp/lct.ps
gv ${DATA}/tmp/lct.ps &
\end{verbatim}

\subsection[ps2epsi]{ps2epsi}\label{sxn:ps2epsi}
The \cmdidx{ps2epsi} program recomputes the minimal possible bounding
box of \trmidx{Postscript} and \trmidx{Enscapsulated postscript} files.
\begin{verbatim}
ps2epsi foo.ps foo.eps
\end{verbatim}

\subsection[epsffit]{epsffit}\label{sxn:epsffit}
The \cmdidx{epsffit} program permits rescaling bounding boxes
\trmidx{Enscapsulated postscript} files, as well as rotating them.
The following doubles the natural size of an image with original
\trmidx{bounding box} = $[0,0,141,510]$, and rotates the image
by~90\dgr\ counter-clockwise:
\begin{verbatim}
% cat foo.eps | epsffit -r 0 0 282 1020 > foo_90.eps
\end{verbatim}
\trmprn{epsffit} coordinate arguments are: [llx,lly,urx,ury] in 
Postscript units (points), i.e., 
[lower-left~$x$, lower-left~$y$, upper-right~$x$, upper-right~$y$].

\subsection[Posters]{Posters}\label{sxn:pst}
Creating conference \trmidx{posters} using \LaTeX\ is not well documented.
The \flidx{a0poster} package is designed to hold conference posters.
\cmdidx{epssplit} can break up a poster into letter (or A4) sized
pieces, which you can then put back together like a big jigsaw puzzle.
First, though, you may need to turn your poster into an \cmdidx{EPS}
file using \trmidx{ghostscript}, \cmdidx{gs}:
\begin{verbatim}
gs -q -sDEVICE=epswrite -sOutputFile=${DATA}/ps/pst_ZNT03.eps -r600 -q - < ${DATA}/ps/pst_ZNT03.ps
epssplit -o ${DATA}/ps/pst_ZNT03_mlt.ps -mar 2.5mm ${DATA}/ps/pst_ZNT03.eps
\end{verbatim}
Use \cmdidx{psresize} to resize the poster to fit onto a single page.
Default units are points, but either \verb'cm' or \verb'in' may be
specified for centimeters or inches, respectively.
\begin{verbatim}
psresize -W1106 -H905 -w2728.575 -h3334.59 ${DATA}/ps/pst_ZNT03.ps ${DATA}/ps/pst_ZNT03_ltr.ps
psresize -Wwdt_in -Hhgt_in -wwdt_out -hhgt_out ${DATA}/ps/pst_ZNT03.ps ${DATA}/ps/pst_ZNT03_ltr.ps
psresize -W60.0in -H40.0in -w11.00in -h8.5in ${DATA}/ps/pst_ZNT03.ps ${DATA}/ps/pst_ZNT03_ltr.ps
\end{verbatim}
Poster sizes should take advantage of common large format printer
sizes. 
The most common printer widths are 36 and 42~inches.
The UCI ImageWorks print shop ((949)\,824-6414, Natural Sciences~I, 
Room~2112) charges about 
$\$10 + \$12$~\xft\ for 36~inch-wide posters, and 
$\$15 + \$15$~\xft\ for 42~inch-wide posters.

\subsection[dvips]{dvips}\label{sxn:dvips}
The \cmdidx{dvips} program converts \trmidx{DVI} files to
\trmidx{Postscript} files.
It also helps generate other formats such as PDF.
\begin{verbatim}
dvips -o nco.ps nco.dvi # Convert DVI to Postscript
\end{verbatim}
One very important feature of \cmdprn{dvips} is that it may be used to
generate beautiful PDF files without having to invoke
\cmdidx{pdflatex}.
This is accomplished by setting the \cmdprn{dvips} output pipe to PDF
\begin{verbatim}
dvips -Ppdf -G0 -o nco.ps nco.dvi # Convert to Postscript intermediate, then PDF
ps2pdf -Ppdf -G0 nco.ps nco.pdf # Convert Postscript to PDF
\end{verbatim}
The resulting PDF file does not preserve the helpful PDF features,
such as hyperlinks, but it does appear as intended in PDF readers such
as \cmdidx{acroread}.
This procedure is very useful when \cmdprn{pdflatex} does not work,
e.g., for complicated files, or when the source is in some other
format, e.g., \TeXInfo\index{TeXInfo@\TeXInfo}.

Unfortunately, the \verb'-Ppdf' switch may typeset ligatures such as
``fi'' or ``ff'' incorrectly.
The solution is to add the \verb'-G0' switch, or to upgrade to more
recent versions of \cmdprn{dvips}.
It is also recommend to add \verb'-Pcmz' and \verb'-amz' switches.
This is explained in the 
\href{http://www.tex.ac.uk/cgi-bin/texfaq2html?label=charshift}{\TeX~FAQ}.  

Another feature of \cmdprn{dvips} is its ability to produce
\trmidx{EPS} (Encapsulated Postscript) files from all or parts of a
document.
\begin{verbatim}
dvips -Ppdf -G0 -E -i -o rt.ps rt.dvi # Convert to Postscript intermediate, then PDF
\end{verbatim}

\subsection[PDF\TeX]{PDF\TeX}\label{sxn:pdftex}
\href{http://www.tug.org/applications/pdftex}{PDF\TeX}, by
\nmidx{H\`{a}n Th\'{\^{e}} Th\`{a}nh}, is a micro-typographic
extensions to the \TeX\ typesetting system.
PDF\TeX\ produces \trmidx{PDF} output directly, rather than using a
\trmidx{PostScript} converter such as \cmdidx{ghostscript}.
\nmidx{Prof.\ Dr.\ Hans Hagen} contributed much to the development of
PDF\TeX. 
% \href{http://davinci.informatik.uni-kl.de}{University of Kaiserslautern}
Hagen runs a private company,
\href{http://www.pragma-ade.com}{Pragma Advanced Document Engineering}, 
that supports a macro package for \TeX\ called \trmidx{Con\TeX{}t}.

PDF\TeX\ tends to give inscrutable error messages.
For more informative diagnostics, invoke with \verb'pdflatex -k 255 foo.tex'.

There are two distinct commands that de-\LaTeX\ files, \cmdidx{detex}
and \cmdidx{untex}. 
These converters are really \LaTeX-strippers---they only do a good
job of preserving unformatted text. 
A good alternative for formatted text is to convert the PDF file
(rather than the \LaTeX\ file), to text.
\cmdidx{pdftotext} accomplishes this.

Foiltex requires the \verb'dvips' option in order to produce landscape
mode foils. 
The \verb'hyperref' package interferes with Foiltex and breaks the 
rotating capability.

\subsubsection[Viewing]{Viewing}\label{sxn:pdf_vw}
Free readers like \cmdidx{acroread}, \cmdidx{ghostview}, 
\cmdidx{kpdf}, and \cmdidx{xpdf} all display PDF\TeX\ files.
The
\href{http://www.adobe.com/products/acrobat/alternate.html}{Adobe-supplied}
\cmdprn{acroread} was intentiaionally crippled to not refresh  
documents on the fly. 
To circumvent this, keep another document open, then use \verb'C-w'
to close the document and \verb'C-o' to open the file again.
However, \cmdidx{acroread} does have the nice ability to convert PDF to
\trmidx{Postscript} using, e.g., 
\begin{verbatim}
acroread -help # Print acroread options
acroread -toPostScript -level2 -size letter -pairs -shrink fl_nm.pdf fl_nm.ps
for fl in `ls *.pdf` ; do
acroread -toPostScript -level2 -size letter -pairs -shrink ${fl} ${fl/pdf/ps}
done
for fl in `ls *.ps` ; do kprinter ${fl}; done
\end{verbatim}

\subsubsection[Forms]{Forms}\label{sxn:pdf_frm}
The problem of filling out PDF document forms has no free solutions.
\trmidx{PDFescape}, a commercial site at
\url{http://www.pdfescape.com/pdf/open}, lets on
upload, edit, and download PDF files.

\subsubsection[hyperref]{hyperref}\label{sxn:hyperref}
The \cmdprn{hyperref} package provides most of the \LaTeX's interface
to the Web-aware features of PDF files supported by \cmdprn{pdflatex}.
PDF files support links within documents, and between the document 
and the \trmidx{World Wide Web} (WWW).
\cmdprn{pdflatex} automatically converts most syntactically useful
document entry points to internal links.
For example, the first section of an \trmidx{article format} document 
will be linked to the internal name \verb'section.1'.
Hence the first section of a document \flidx{foo.pdf} stored at
\url{http://foo.com/doc.pdf} has the global \trmidx{URL}
\url{http://foo.com//doc.pdf#section.1}.

\LaTeX\ documents create active links to such URLs with the
\cmdltxidx{\href} command.
For example, the following links should actively point to this 
particular section of this document 
as an \href{subsubsection.2.5.2}{internal link} and as an
\href{http://dust.ess.uci.edu/doc/ltx/ltx.pdf#subsubsection.2.5.2}{external link},
respectively.
\begin{verbatim}
as an \href{subsubsection.2.5.2}{internal link} and as an
\href{http://dust.ess.uci.edu/doc/ltx/ltx.pdf#subsubsection.2.5.2}{external link},
\end{verbatim}

The \cmdidx{\href} macro can access ``local'' URLs.
This \href{file:///Users/zender/data/ps/ltx.pdf#section.4}{local file link}
accesses the \cmdprn{hyperref}-generated link to Section~\ref{sxn:txt}
through the HTTP \cmdidx{file://} protocol:  
\begin{verbatim}
This \href{file:///Users/zender/data/ps/ltx.pdf#section.4}{active link} ...
\end{verbatim}
When a local file link points to the file being viewed, the viewer
(e.g., \cmdidx{xpdf}) jumps to the link without spawning a new
viewer. 
Local file links are better accomplished with the standard \LaTeX\
\cmdltxidx{\label} and \cmdltxidx{\ref} macros.
Nonetheless, \cmdprn{hyperref} automatically generates labels at
standardized locations in many document types.

The \cmdidx{\href} macro is more useful for accessing remote URLs.
This 
\href{http://dust.ess.uci.edu/doc/ltx/ltx.pdf#section.4}{remote URL link}
should access the \cmdprn{hyperref}-generated link to
Section~\ref{sxn:txt} through the HTTP \cmdidx{http://} protocol: 
\begin{verbatim}
\href{http://dust.ess.uci.edu/doc/ltx/ltx.pdf#section.4}{remote URL link} 
\end{verbatim}
To resolve remote URLs, the current viewer spawns a new viewer process
(probably based on the default handler for the \trmidx{mime-type}, in
this case PDF) to download and display the remote document.
Unfortunately, no viewers seem intelligent enough to open the remote
document to the specified section.
In other words, the viewer downloads the remote document correctly,
and then the new viewer opens the document to Page~1, rather than to
the specified section (Section~\ref{sxn:txt} in this case).

\cmdprn{hyperref} provides the following macros for fine-grained control:
\begin{verbatim}
\hyperdef{category}{name}txt % Mark text with category.name
\hyperref{URL}{category}{name}{text} % Link text to URL#category.name
\hyperbaseurl{URL} % Prepend URL to following URLs
\hypertarget{sxn:abb}{} % fxm
\end{verbatim}
Samples of these macros for generic situations are
\begin{verbatim}
\hyperref{sxn}{}{} % fxm
\hyperref{}{sxn}{abb}{fxm} % Link text to URL#category.name
\hypertarget{sxn:abb}{} % fxm
\end{verbatim}
Actual targets that work in typical situations are
\begin{verbatim}
\hyperbaseurl{file:///Users/zender/data/ps/abb.pdf}
\hypertarget{sxn:abb}{}
\end{verbatim}
The \cmdltxidx{\hypertarget} command sets fxm.

\subsubsection[Manipulating]{Manipulating}\label{sxn:pdf_mnp}
It is often helpful to manipulate a PDF document using native PDF
tools so that no lossy conversions to other formats (e.g., Postscript)
need be done.
The \trmidx{PDF toolkit} \href{http://www.accesspdf.com/pdftk}{pdftk}
is useful for this.
From the \cmdidx{pdftk} homepage:
\begin{quote}
If PDF is electronic paper, then pdftk is an electronic
staple-remover, hole-punch, binder, secret-decoder-ring, and
X-Ray-glasses 
\end{quote}
\trmidx{PDFtk} may be used to merge, split, decrypt, encrypt, burst,
uncompress, and repair PDF files.
To extract subsets of pages from a PDF document,
\begin{verbatim}
pdftk A=${DATA}/ps/prp_itr.pdf cat A2 output ~/prp_itr_smr.pdf
\end{verbatim}

When adding publication quality graphics output (from, e.g., NCL)
in an unnumbered, multi-page file (\flprn{foo\_gph.pdf}) to a 
manuscript PDF (from, e.g., \flprn{foo\_mns.pdf} from Google Docs) to
produce a submittable manuscript (\flprn{ppr\_ZKT12.pdf} for, e.g.,
ACP), one may need to manually number the pages containing the figures.
There is a great FAQ on this at
\url{http://forums.debian.net/viewtopic.php?t=30598}. 
To number an input (unnumbered) 10-page graphics file
\flprn{foo\_gph.pdf} and append it to a 44-page manuscript, do this:
\begin{verbatim}
/bin/cp ~/Téléchargements/figures_09212011.pdf ~/foo_gph.pdf
/bin/cp ~/Téléchargements/ppr_ZKT12_mns.pdf ~/foo_mns.pdf
# Create LaTeX file containing only page numbers
cd;cat > ~/foo_pg.tex << EOF
\documentclass[12pt,letter]{article}
\usepackage{multido}
\usepackage[hmargin=.8cm,vmargin=1.5cm,nohead,nofoot]{geometry}
\begin{document}
\setcounter{page}{45}
%\multido{}{10}{\vphantom{x}\newpage}
\multido{}{10}{\vphantom{x} \addtocounter{figure}{1} Figure~\thefigure \newpage}
\end{document}
EOF
# Produce PDF file containing only page numbers
pdflatex foo_pg.tex 
pdftk foo_gph.pdf burst output foo_gph_%03d.pdf
pdftk foo_pg.pdf  burst output foo_pg_%03d.pdf
# Unite manuscript pages with page numbers
time for i in $(seq -f %03g 1 10) ; do \
pdftk foo_gph_$i.pdf background foo_pg_$i.pdf output foo_mrg_$i.pdf ; done
# Merge one-page files into one multipage file
pdftk foo_mrg_???.pdf output foo_mrg.pdf
# Shrink resulting file by removing font redefitions
gs -sDEVICE=pdfwrite -dCompatibilityLevel=1.4 -dPDFSETTINGS=/screen -dNOPAUSE \
   -dQUIET -dBATCH -sOutputFile=foo_gph_fnl.pdf foo_mrg.pdf
pdftk A=~/foo_mns.pdf B=~/foo_gph_fnl.pdf cat A B output ${DATA}/ppr/ppr_ZKT12.pdf
scp ${DATA}/ppr/ppr_ZKT12.pdf dust.ess.uci.edu:/var/www/html/ppr
\end{verbatim}

\subsubsection[pdfscreen]{pdfscreen}\label{sxn:pdfscreen}
Another useful package is \verb'pdfscreen'.
\verb'pdfscreen' relies heavily on \verb'hyperref' and a number of
other packages.
I have installed the entirety of \verb'pdfscreen' in
\verb'${TEXMFDIR}/tex/latex/pdfscreen'.
% $: re-balance syntax highlighting
However, only \flidx{pdfscreen.sty} and a few other files in the
directory are strictly required for the installation to work.
This should be fixed so the \TeX\ directory does not cruft up.
Printing \verb'pdfscreen' presentations is sometimes useful.
To do this, set 
\begin{verbatim}
pdfpagemode={FullScreen}, % Starts in full screen mode, hit 'Esc' to escape
pdfmenubar=true % Allow access to reader's menubar
\end{verbatim}
in the \verb'\hypersetup' portion of the preamble. 
This will start the presentation in full screen mode and make the
reader (e.g., \cmdidx{acroread}) menubar available once the
presentation is ``escaped'' using, e.g., \verb'Esc'.

\subsubsection[thumbpdf]{thumbpdf}\label{sxn:thumbpdf}
PDF files have the capability to show \trmidx{thumbnails},
reduced-size images of each page.
The various \TeX\ engines do not generate thumbnails automatically
(since they are), so that some intervention is required to insert them
in the final PDF document.
The two freely available methods are the \cmdprn{thumbpdf} package, by
\trmidx[Oberdiek, Heiko]{Heiko Oberdiek}, and \cmdidx{pdfthumb}, a
part of the \cmdidx{PPower4} (P$^{4}$) 
\href{http://www-sp.iti.informatik.tu-darmstadt.de/software/ppower4}{project}. 
The typical usage of \cmdprn{thumbpdf} is
\begin{verbatim}
pdflatex ltx.tex;thumbpdf ltx;pdflatex ltx.tex
\end{verbatim}
The \cmdprn{thumbpdf} package may also be used in Postscript processing
This is accomplished by setting the \cmdprn{dvips} output pipe to PDF
\begin{verbatim}
dvips -Ppdf -G0 -o ${DATA}/ps/ltx.ps ltx.dvi
thumbpdf --modes=dvips --level2 --useps ${DATA}/ps/ltx.ps
dvips -Ppdf -G0 -o ${DATA}/ps/ltx.ps ltx.dvi
ps2pdf ${DATA}/ps/ltx.ps ${DATA}/ps/ltx.pdf
pdf2ps ${DATA}/ps/ltx.pdf ${DATA}/ps/ltx.ps 
\end{verbatim}
% $: re-balance syntax highlighting

\subsubsection[pdfthumb]{pdfthumb}\label{sxn:pdfthumb}
The P$^{4}$
\url{http://www-sp.iti.informatik.tu-darmstadt.de/software/ppower4}{project}  
created the PDF Presentation Post Processor.
Install this package in generic directories then create thumbnail
additions to any PDF file.
\begin{verbatim}
sudo mkdir ${TEXMFDIR}/tex/generic/ppower4
sudo mv *.sty ${TEXMFDIR}/tex/generic/ppower4
sudo mv ppower4 pdfthumb /usr/local/bin
pdfthumb in.pdf out.pdf
pdfthumb nco.pdf nco.pdf2;/bin/mv nco.pdf2 nco.pdf
\end{verbatim}

\subsection[Tables]{Tables}\label{sxn:tbl}
Tables are perhaps the most difficult-to-master aspect of \LaTeX.
Use the \flidx{rotating.sty} package to \trmidx{rotate}
\trmidx{tables}. 
\begin{verbatim}
\begin{sidewaystable}

\end{sidewaystable}
\end{verbatim}
When large portions of text (many pages) are to be printed sideways,
then \trmidx{landscape} mode is called for.
Footnotes in tables are also tricky.

\subsubsection[epstopdf]{epstopdf}\label{sxn:epstopdf}
PDF\TeX\ does not recognize \flprn{*.eps} files directly.
Instead these must be converted to PDF-format with \cmdidx{epstopdf}.
The \cmdprn{epstopdf} package by Heiko Oberdiek is very useful when
using PDF\LaTeX. 
\begin{verbatim}
for fl in `ls *.eps` ; do
epstopdf ${fl}
done
\end{verbatim}
% $: re-balance syntax highlighting
The package \flidx{epstopdf.sty} does this automatically.
It converts Postscript (\flprn{*.ps}) and encapsulated Postscript
(\flprn{*.eps}) files to \PDFacr\ files (\flprn{*.pdf}) automatically 
(using \cmdidx{epstopdf}) if the \PDFacr\ files do not already exist.
However, this capability is intrinsically somewhat insecure since it
involves allowing \LaTeX\ to run shell commands.
Hence, \cmdprn{epstopdf} is not installed by default and special
permissions must be set to activate it.
To enable this feature at run-time use, e.g., 
\verb'pdflatex -shell-escape test.tex'.
To permanently enable this feature for the whole distribution set
\verb'shell_escape = 1' in configuration file \flidx{texmf.cnf}.

\subsubsection[Upgrading]{Upgrading}\label{sxn:pdf_upg}
Upgrading PDF\TeX:
\begin{verbatim}
cd ${DATA}/tmp;mkdir pdftex;cd pdftex
ftp://ftp.muni.cz/pub/tex/local/cstug/thanh/pdftex/snapshots
gunzip pdftex-20010806-linux.zip
./configure --prefix=/usr --datadir=/usr/share
make
sudo mv /usr/bin/pdfetex /usr/bin/pdftex /usr/bin/ttf2afm /data/zender/bck
sudo mv ${TEXMFDIR}/web2c/pdftex.pool ${TEXMFDIR}/web2c/pdfetex.pool /data/zender/bck
sudo cp pdfetex pdftex ttf2afm pdftosrc /usr/bin
sudo cp pdfetex.pool pdftex.pool ${TEXMFDIR}/web2c
sudo texhash

cd ${DATA}/tmp;mkdir pdftex;cd pdftex
sudo mv pdftex-20010806.tgz ${DATA}/tmp/pdftex
tar xvzf pdftex-20010806.tgz
cd src
./configure --prefix=/usr/share
cd texk/web2c
make pdftexbin
cd ${DATA}/tmp/pdftex/src/texk/web2c
sudo cp pdfetex pdftex ttf2afm pdftosrc /usr/bin
sudo cp pdfetex.pool pdftex.pool ${TEXMFDIR}/web2c
\end{verbatim}
Binaries and pool files are located in the directory
\verb'${DATA}/tmp/pdftex/src/texk/web2c'.
% $: re-balance syntax highlighting
You must regenerate \verb'.fmt' files \verb'pdftex.fmt' and
\verb'pdflatex.fmt' after installing a new version
of the PDF\TeX\ binary and \verb'pdftex.pool' files. 
\begin{verbatim}
pdftex -ini -fmt=pdftex plain \\dump
pdftex -ini -fmt=pdflatex latex.ltx
\end{verbatim}
or
\begin{verbatim}
sudo fmtutil --byfmt pdftex
sudo fmtutil --byfmt pdflatex
\end{verbatim}
Apparently, \verb'texconfig init' may do both in one fell swoop.

Then update the graphics \verb'.def' file to the newest version at
\url{http://www.ctan.org/tex-archive/macros/pdftex/graphics/pdftex.def} 
\begin{verbatim}
find /usr/share/ -name pdftex.def
cd ${TEXMFDIR}/tex/latex/graphics/
sudo mv pdftex.def pdftex.def.orig
sudo cp ~/pdftex.def .
\end{verbatim}
% $: re-balance syntax highlighting

\section[Using \LaTeX]{Using \LaTeX}\label{sxn:ltx}

\subsection[NFSS]{NFSS}\label{sxn:nfss}
The font scheme used in \LaTeX\ is known as the 
\trmdfn{New Font Selection Scheme} (\acr{nfss}). 
\acr{nfss} recognizes five distinct font attributes: \trmdfn{encoding},
\trmdfn{family}, \trmdfn{series}, \trmdfn{shape}, and \trmdfn{size}
that may be set individually with the commands
\vrbcmdfmt{fontencoding}{encode}, \vrbcmdfmt{fontfamily}{fam},
\vrbcmdfmt{fontseries}{wt\_wth}, \vrbcmdfmt{fontshape}{form}, and 
\vrbcmdfmt{fontsize}{line\_sp}.
The font \textit{series} consists of two values, weight and width,
concatenated sequentially into one argument, \textit{wt\_wth} of
between one and four characters, e.g., \textsf{ebsc}.  
The font \textit{size} takes two separate arguments, the point size of
the font and the value of \cmdltxidx{\baselineskip}.
Table~\ref{tbl:att} shows sample values for each font attribute.
\begin{table}
\begin{minipage}{\hsize} % Minipage necessary for footnotes KoD95 p. 110 (4.10.4)
\renewcommand{\footnoterule}{\rule{\hsize}{0.0cm}\vspace{-0.0cm}} % KoD95 p. 111
\begin{center}
\caption[Font Attributes]{\textbf{Font Attributes}%
\footnote{\emph{Sources:} \cite{KoD99,KoD04}}%
\label{tbl:att}}
\vspace{\cpthdrhlnskp}
% fxm:\raggedright does not work here, must be between two other columns?
%\begin{tabular}{ r >{\raggedright}p{14.0em}<{} }
\begin{tabular}{ r l }
\hline \rule{0.0ex}{\hlntblhdrskp}% 
Attribute & Value \\[0.0ex]
\hline \rule{0.0ex}{\hlntblntrskp}%
Encoding & \textsf{OT1 OT2 T1 OML OMS OMX U} \\[0.5ex]
Family & \textsf{cmr cmss cmtt} \\[0.5ex]
& Computer modern roman, Computer modern sans serif, Computer modern 
typewriter \\[0.5ex]
Series (weight) & \textsf{ul el l sl m sb b eb ub} \\[0.5ex]
& Ultralight, Extralight, Light, Semilight, Medium, Semibold, Bold,
Extrabold, Ultrabold \\[0.5ex]
Series (width) & \textsf{uc ec c sc m sx x ex ux} \\[0.5ex]
& Ultracondensed, Extracondensed, Condensed, Semicondensed, Medium,
Semiexpanded, Expanded, Extraexpanded, Ultraexpanded \\[0.5ex]
Shape & \textsf{n it sl sc u} \\[0.5ex]
& Normal, Italic, Slanted, Small caps \\[0.5ex]
Size & \\[0.5ex]
\hline
\end{tabular}
\end{center}
\end{minipage}
\end{table}
\LaTeX\ comes with an interactive document which prints font tables
on demand.
\begin{verbatim}
latex ${TEXMFDIR}/tex/latex/base/nfssfont.tex
Name of the font to test = ygoth
\help
\sample
\bye
\end{verbatim}
% $: re-balance syntax highlighting

\subsubsection[\textsf{fontchart}]{\textsf{fontchart}}\label{sxn:fontchart} 
The preferred method for examining \TeX\ fonts is to use the built-in 
\cmdidx{fontchart} program.
The following examples demonstrates the \textsf{cmr12} font
\begin{verbatim}
% tex fontchart
> Name of the font to chart = cmr12
kdvi fontchart &
\end{verbatim}

% fxm: finish this
\subsubsection[\textsf{usefont}]{\textsf{usefont}}\label{sxn:usefont} 
Fonts may be swiftly changed with the \cmdltxidx{\usefont} macro.
This macro takes four arguments: \cmdltxprn{\usefont{}{}{}{}}.

\subsubsection[\textsf{suffixes}]{\textsf{suffixes}}\label{sxn:suffixes}  
Table~\ref{tbl:sfx} shows the conventional meaning of some of the
filename suffixes in the \LaTeX\ universe.
\begin{table}
\begin{minipage}{\hsize} % Minipage necessary for footnotes KoD95 p. 110 (4.10.4)
\renewcommand{\footnoterule}{\rule{\hsize}{0.0cm}\vspace{-0.0cm}} % KoD95 p. 111
\begin{center}
\caption[Filename Suffixes]{\textbf{Filename Suffixes}%
\footnote{\emph{Sources:} \cite{KoD99,KoD04}}%
\label{tbl:sfx}}
\vspace{\cpthdrhlnskp}
% fxm:\raggedright does not work here, must be between two other columns?
%\begin{tabular}{ r >{\raggedright}p{14.0em}<{} }
\begin{tabular}{ r l l}
\hline \rule{0.0ex}{\hlntblhdrskp}% 
Suffix & Example & Meaning \\[0.0ex]
\hline \rule{0.0ex}{\hlntblntrskp}%
.tfm & pplr.tfm & \\[0.5ex]
\hline
\end{tabular}
\end{center}
\end{minipage}
\end{table}
Some characters such as are hard to access. 
For this purpose, use the \cmdltxidx{\symbol} command \cite[][p.~63]{KoD99}.
For example, \verb'\symbol{126}' produces \symbol{126}, the tilde symbol.

\subsection[Dependencies]{Dependencies}\label{sxn:dpn}
Building and maintaining complex \LaTeX\ files is non-trivial and
requires effort similar to maintaining a complex model.
There are two ways to keep track of the dependencies of a source
file. 
The first is to add the command \cmdltxidx{\listfiles} somewhere in the
document preamble.
This causes \LaTeX\ to print the file dependencies to screen during
processing. 
The second is to add the command \verb'\RequirePackage{snapshot}'
\textit{before} the \cmdltxidx{\documentclass} command.
This will generate a dependency file (\verb'*.dep') file.

\subsection[Citations]{Citations}\label{sxn:ctt}
Here we try some typical \LaTeX\ bibliographic citations
\begin{enumerate}
\item Recommended format for citing \trmidx{URLs} is demonstrated by
  invoking 
\verb'\cite{Zen01b}': \cite{Zen01b}
\verb'\cite{WZJ073}': \cite{WZJ073}
\end{enumerate}

\subsection[Dates and Times]{Dates and Times}\label{sxn:tm}
The \flidx{datetime.sty} package provides fine control for printing
\trmidx{dates} and \trmidx{times}.
It provides the \cmdltxidx{\xxivtime} and \cmdltxidx{\ordinal}
commands. 

\subsection[Indicating Definitions, Commands, Files]{Indicating Definitions, Commands, Files}\label{sxn:ind}
It is helpful to indicate what type of object text refers to.
For example, the \TeX{}Info documentation system discriminates between
15 types of text object.
Not all those types are relevent here since, e.g., \LaTeX\ has superb 
citation handling ability already.
Table~\ref{tbl:txi} list the types of text objects that appear in
this documentation, and the commands necessary to indicate that type
of text.
\begin{table}
\begin{minipage}{\hsize} % Minipage necessary for footnotes KoD95 p. 110 (4.10.4)
\renewcommand{\footnoterule}{\rule{\hsize}{0.0cm}\vspace{-0.0cm}} % KoD95 p. 111
\begin{center}
\caption[Text Object Types]{\textbf{Text Object Types}%
\footnote{\emph{Sources:} \TeX{}Info manual}%
\label{tbl:txi}}
\vspace{\cpthdrhlnskp}
% fxm:\raggedright does not work here, must be between two other columns?
%\begin{tabular}{ r >{\raggedright}p{14.0em}<{} } 
\begin{tabular}{ r l l l l }
\hline \rule{0.0ex}{\hlntblhdrskp}% 
Command & Alternate & Example & Result & Meaning \\[0.0ex]
\hline \rule{0.0ex}{\hlntblntrskp}%
\cmdprn{\acronym} & \cmdprn{\acr} & & Acronym \\[0.5ex]
\cmdprn{\cite} & & & Reference \\[0.5ex]
\cmdprn{\code} & & & Program code \\[0.5ex]
\cmdprn{\command} & \cmdprn{\cmdprn} & & Command name \\[0.5ex]
& \cmdprn{\cmdltxprn} & & \LaTeX\ Command name \\[0.5ex]
\cmdprn{\dfn} & \cmdprn{\trmdfn} & & Definition \\[0.5ex]
\cmdprn{\email} & & & Electronic mail address \\[0.5ex]
\cmdprn{\env} & & & Environment variable \\[0.5ex]
\cmdprn{\file} & \cmdprn{\flprn} & & File name \\[0.5ex]
\cmdprn{\kbd} & & & Keyboard input \\[0.5ex]
\cmdprn{\key} & & & Specifying keys \\[0.5ex]
& \cmdprn{\mchprn} & & Machine name \\[0.5ex]
& \cmdprn{\nmprn} & & Proper name \\[0.5ex]
\cmdprn{\option} & & & Option name \\[0.5ex]
\cmdprn{\samp} & \cmdprn{\smpprn} & & Literal character sequence \\[0.5ex]
\cmdprn{\url} & & & World Wide Web location \\[0.5ex]
\cmdprn{\var} & & & Metasyntactic variable \\[0.5ex]
\hline
\end{tabular}
\end{center}
\end{minipage}
\end{table}
For consistency, most object types are defined exactly as in
\TeX{}Info. 

\subsection[\TeXInfo]{\TeXInfo}\label{sxn:texinfo}
A large suite of tools is available for manipulating \LaTeX\ and
\TeXInfo\index{TeXInfo@\TeXInfo} documents.
\subsubsection[Inserting \TeX\ into \TeXInfo]{Inserting \TeX\ into \TeXInfo}\label{sxn:texintexinfo}
\begin{verbatim}
@tex
% Define TeX macros to roughly correspond to LaTeX style files
\def\dfr{d} % [frc] Math differential
@end tex
\end{verbatim}

\subsubsection[Texi2html]{Texi2html}\label{sxn:texi2html}
The \cmdidx{texi2html} command converts \TeXInfo-format manuals into
\trmidx{HTML} format.
The sources are available from the \trmidx{CVS} server
\begin{verbatim}
cvs -d :pserver:t2h-anon@urmel.mathematik.uni-kl.de:/usr/local/Singular/cvsroot login
cvs -d :pserver:t2h-anon@urmel.mathematik.uni-kl.de:/usr/local/Singular/cvsroot co Texi2html
\end{verbatim}
with the password \verb'texi2html'.

\subsubsection[Texi2latex]{Texi2latex}\label{sxn:texi2latex}
The \cmdidx{texi2latex} command converts \TeXInfo-format manuals into 
\LaTeX\ format.
The sources are available from the \trmidx{GNU} project server
\url{http://www.nongnu.org/texi2latex}.
\cmdprn{texi2latex} requires an \trmidx{XSLT} processor such as
\trmidx{Saxon}. 
\begin{verbatim}
sudo apt-get install libsaxon-java
cd ${DATA}/tmp/texi2latex-0.9.4
export CXX=g++-3.4
make
sudo make install
texi2latex ~/nco/doc/nco.texi
# First, install saxon version 6.5.3
cd ${DATA}/tmp
mkdir saxon-6.5.3
cp saxon6_5_3.zip saxon-6.5.3
cd saxon-6.5.3
unzip saxon6_5_3.zip
\end{verbatim}

\subsubsection[Tth]{Tth}\label{sxn:tth}
The \cmdidx{tth} (\TeX-to-HTML) command converts \LaTeX-format
documents into \trmidx{HTML} format.
\begin{verbatim}
tth -a -Lltx -p${TEXINPUTS}:${BIBINPUTS} < ltx.tex > ltx.html
\end{verbatim}
The \verb'-a' switch tells \cmdprn{tth} to \textit{automatically} run
commands like \cmdprn{latex}, \cmdprn{bibtex}, etc., to generate any
necessary auxiliary files.
Tell \cmdprn{tth} the root word for the auxiliary files with the \verb'-L' switch.
\cmdprn{tth} searches for files on the path specified by the \verb'-p' switch.
Full documentation is available at
\flprn{file:///usr/share/doc/tth/tth\_manual.html}. 
According to the manual,
\begin{quote}
TTH does not recognize the \cmdltxidx{\usepackage} command by default
because the \LaTeX\ macros that are input by this command almost
always contain catcode changes or other usages incompatible with TTH. 
oTTH requires that personal packages use \cmdltxidx{\input} instead of
\cmdltxidx{\usepackage}.
This usually imposes non-generic, and thus unacceptable, constraints
on the source \LaTeX\ document.
\end{quote}

\subsubsection[Tex4ht]{Tex4ht}\label{sxn:tex4ht}
\cmdidx{tex4ht} is both a package and a program that converts
\LaTeX-format documents into \trmidx{HTML}.
It is best to invoke the \cmdprn{tex4ht} suite of programs using the 
\cmdidx{htlatex} script.
\begin{verbatim}
htlatex ltx.tex
\end{verbatim}
This produces HTML files of the form \flprn{ltx.css},
\flprn{ltx[0-9].html}, \flprn{ltx.xref},  \flprn{ltx.lg},
\flprn{ltx.idv},  \flprn{ltx.4ct},  \flprn{ltx.tc},

If interested in displaying pages with \cmdidx{mozilla}, in
particular, use the \cmdidx{mzlatex} script:
\begin{verbatim}
mzlatex ltx.tex
\end{verbatim}
This produces XML files of the form \flprn{ltx[0-9].xml}.

\subsection{HTML/SGML}\label{sxn:html}
\trmidx{Hypertext Markup Language} (\trmidx{HTML}) is used to encode
documents displayed by most Web browsers.
Table~\ref{tbl:html} summarizes the most common ways to implement of
\trmidx{special characters} in HTML, and their \LaTeX\ equivalents.
\begin{landscape}
\begin{longtable}{ >{\ttfamily}l<{} >{\ttfamily}l<{} >{\ttfamily}l<{} l }
& & & \kill % NB: longtable requires caption as table entry
\caption[HTML/SGML Special Characters]{\textbf{HTML/SGML Special Characters}% 
\footnote{\textit{Source}: \url{http://www.w3.org/TR/html401/sgml/entities.html}}%  
\label{tbl:html2}} \\
\hline\hline \rule{0.0ex}{\hlntblhdrskp}% 
\textrm{Mnemonic HTML} & \textrm{Numeric HTML} & {\sffamily\LaTeX} & Result \\[0.0ex]
\hline \rule{0.0ex}{\hlntblntrskp}%
\endfirsthead % Lines between and \endfirsthead appear at top of table
\caption[]{(continued)} \\ % Set label for following pages
\textrm{Mnemonic HTML} & \textrm{Numeric HTML} & {\sffamily\LaTeX} & Result \\[0.0ex]
\hline \rule{0.0ex}{\hlntblntrskp}%
\endhead % Previous block appears at top of every page
\endlastfoot % Previous block appears at end of table
\verb'&lt;foo@bar&gt;' & \verb'&#60;foo@bar&#62;' & \verb'$<$foo@bar$>$' & $<$\texttt{foo@bar}$>$ \\[0.1ex]
\verb'&ccedil;' & \verb'&#231;' & \verb'\c{c}' & \texttt{\c{c}} \\[0.5ex]
\verb'&egrave;' & \verb'&#232;' & \verb'\`{e}' & \texttt{\`{e}} \\[0.5ex]
\verb'&eacute;' & \verb'&#233;' & \verb|\'{e}| & \texttt{\'{e}} \\[0.5ex]
\verb'&ograve' & \verb'&#242;' & \verb'\`{o}' & \texttt{\`{o}} \\[0.5ex]
\verb'&oacute' & \verb'&#243;' & \verb|\'{o}| & \texttt{\'{o}} \\[0.5ex]
\verb'&ocirc' & \verb'&#244;' & \verb'\^{o}' & \texttt{\^{o}} \\[0.5ex]
\verb'&otilde' & \verb'&#245;' & \verb'\~{o}' & \texttt{\~{o}} \\[0.5ex]
\verb'&ouml' & \verb'&#246' & \verb'\"{o}' & \texttt{\"{o}} \\[0.5ex]
\verb'NONE' & \verb'&#333;' & \verb'\={o}' & \texttt{\={o}} \\[0.5ex]
\verb'NONE' & \verb'NONE' & \verb'\.{o}' & \texttt{\.{o}} \\[0.5ex]
\verb'NONE' & \verb'NONE' & \verb'\u{o}' & \texttt{\u{o}} \\[0.5ex]
\verb'NONE' & \verb'NONE' & \verb'\v{o}' & \texttt{\v{o}} \\[0.5ex]
\verb'NONE' & \verb'NONE' & \verb'\H{o}' & \texttt{\H{o}} \\[0.5ex]
\verb'NONE' & \verb'NONE' & \verb'\t{oo}' & \texttt{\t{oo}} \\[0.5ex]
\verb'NONE' & \verb'NONE' & \verb'\c{o}' & \texttt{\c{o}} \\[0.5ex]
\verb'NONE' & \verb'NONE' & \verb'\d{o}' & \texttt{\d{o}} \\[0.5ex]
\verb'NONE' & \verb'NONE' & \verb'\b{o}' & \texttt{\b{o}} \\[0.5ex]
\verb'NONE' & \verb'NONE' & \verb'\r{o}' & \texttt{\r{o}} \\[0.5ex]
\verb'FXM' & \verb'FXM' & \verb'\oe' & \texttt{\oe} \\[0.5ex]
\verb'FXM' & \verb'FXM' & \verb'\OE' & \texttt{\OE} \\[0.5ex]
\verb'FXM' & \verb'FXM' & \verb'\ae' & \texttt{\ae} \\[0.5ex]
\verb'FXM' & \verb'FXM' & \verb'\AE' & \texttt{\AE} \\[0.5ex]
\verb'FXM' & \verb'FXM' & \verb'\aa' & \texttt{\aa} \\[0.5ex]
\verb'FXM' & \verb'FXM' & \verb'\AA' & \texttt{\AA} \\[0.5ex]
\verb'&oslash;' & \verb'&#248;' & \verb'\o' & \texttt{\o} \\[0.5ex]
\verb'&Oslash;' & \verb'&#248;' & \verb'\O' & \texttt{\O} \\[0.5ex]
\verb'FXM' & \verb'FXM' & \verb'\l' & \texttt{\l} \\[0.5ex]
\verb'FXM' & \verb'FXM' & \verb'\L' & \texttt{\L} \\[0.5ex]
\verb'FXM' & \verb'FXM' & \verb'\ss' & \texttt{\ss} \\[0.5ex]
\verb'FXM' & \verb'FXM' & \verb'\SS' & \texttt{\SS} \\[0.5ex]
\verb'FXM' & \verb'FXM' & \verb'!`' & \texttt{!`} \\[0.5ex]
\verb'FXM' & \verb'FXM' & \verb'?`' & \texttt{?`} \\[0.5ex]
\verb'&yacute;' & \verb'&#253;' & \verb|\'{y}| & \texttt{\'{y}} \\[0.5ex]
\verb'&nbsp;' & \verb'&#160;' & \verb'non-breakable space' & non-breakable space \\[0.5ex]
\verb'C.&nbsp;S. Zender' & \verb'C.&#160;S. Zender' & \verb'C.~S. Zender' & \texttt{C.~S. Zender} \\[0.5ex]
\verb'&Alpha;' & \&\#913; & \verb'NONE' & (Greek capital letter alpha) \\[0.5ex]
\verb'&alpha;' & \&\#945; & \verb'$\alpha$' & $\mathtt{\alpha}$ (Greek small letter alpha) \\[0.5ex]
\verb'&#176;' & \&\#176; & \verb'\circ' & $\mathtt{\circ}$ (ring accent) \\[0.5ex]
\verb'&#186;' & \&\#186; & \verb'^{\circ}' & $\mathtt{^{\circ}}$ (degree symbol) \\[0.5ex]
\verb'&minus;' & \&\#8722; & \verb'$-$' & $\mathtt{-}$ (Minus sign) \\[0.5ex]
\verb'&times;' & \&\#215; & \verb'$\times$' & $\mathtt{\times}$ (Multiplication sign) \\[0.5ex]
\verb'&divide;' & \&\#247; & \verb'$/$' & $\mathtt{/}$ (Division sign) \\[0.5ex]
\verb'&plusmn;' & \&\#177; & \verb'$\pm$' & $\mathtt{\pm}$ (Plus-minus sign) \\[0.5ex]
\verb'&part;' & \&\#8706; & \verb'$\partial$' & $\mathtt{\partial}$ (Partial differential) \\[0.5ex]
\verb'&micro;' & \&\#181; & \verb'$\mu$' & $\mathtt{\mu}$ (Micro sign) \\[0.5ex]

\verb'&euro;' & \&\#8364; & \verb'\texteuro' & \texttt{\texteuro} (euro sign) \\[0.5ex] % KoD99 p. 452

\verb'&copy;' & \verb'&#169;' & \verb|\copyright| & \texttt{\copyright} \\[0.5ex]
\verb'&emsp;' & \verb'&#8195;' & \verb'foo\;bar' & foo\;bar (em space) \\[0.5ex]
\verb'&ensp;' & \verb'&#8194;' & \verb'foo\:bar' & foo\:bar (en space) \\[0.5ex]
\verb'&thinsp;' & \verb'&#8201;' & \verb'foo\,bar' & foo\,bar (thin space) \\[0.5ex]
\verb'&mdash;' & \verb'&#8212;' & \verb'---' & --- (em dash) \\[0.5ex]
\verb'&ndash;' & \verb'&#8211;' & \verb'--' & -- (en dash) \\[0.5ex]
\verb'&lsquo;' & \verb'&#8216;' & \verb|`foo`| & `foo` (left single quotation mark) \\[0.5ex]
\verb'&rsquo;' & \verb'&#8217;' & \verb|'foo'| & 'foo' (right single quotation mark) \\[0.5ex]
\verb'&ldquo;' & \verb'&#8220;' & \verb|``foo``| & ``foo`` (left double quotation mark) \\[0.5ex]
\verb'&rdquo;' & \verb'&#8221;' & \verb|''foo''| & ''foo'' (right double quotation mark) \\[0.5ex]
\verb'&hellip;' & \verb'&#133;' & \verb'\ldots' & \ldots (low horizontal ellipsis) \\[0.5ex]
\verb'&harr;' & \verb'&#8590;' & \verb'\leftarrow' & $\leftrightarrow$ (horizontal left-right arrow) \\[0.5ex]
\verb'&larr;' & \verb'&#8592;' & \verb'\leftarrow' & $\leftarrow$ (left arrow) \\[0.5ex]
\verb'&rarr;' & \verb'&#8594;' & \verb'\rightarrow' & $\rightarrow$ (right arrow) \\[0.5ex]
%\verb'&foo;' & \verb'&#;' & \verb'foo' & foo \\[0.5ex]
% & & \\[0.5ex]
\hline
\end{longtable} % end tbl:html2
\end{landscape}

\subsection[Indexing]{Indexing}\label{sxn:idx}
Here we show some typical \LaTeX\ indexing and cross-indexing
which make use of the \cmdltxidx{\index} command and derivations
thereof.  
\begin{enumerate}
\item The following instance of the word ``quark'' is indexed using
\verb'\trmdfn{quark}': \trmdfn{quark}. 
When processed with the \verb'\usepackage[hyperindex]{hyperref}' 
package, the index should point back to this instance (as long as
\verb'backref=true').
\item The following instance of the word ``quark'' is indexed using
\verb'\trmidx{quark}': \trmidx{quark}.
\begin{enumerate}
\item The following index of the author ``Homer Simpson'' uses\\
\verb'\trmdfn[Simpson, Homer]{Homer Simpson}' so that ``Homer
Simpson'' appears verbatim in the text but is indexed by the optional 
argument ``Simpson, Homer'': \trmdfn[Simpson, Homer]{Homer Simpson}.
Compare this to \verb'\trmidx[Simpson, Homer]{Homer Simpson}': 
\trmidx[Simpson, Homer]{Homer Simpson} and to
\verb'\trmidx{Homer Simpson}': \trmidx{Homer Simpson}.
\end{enumerate}
\item Related concepts may be cross-indexed and sub-sorted as well.
For example
\verb'\index{Frodo|see{Baggins}}'\index{Frodo|see{Baggins}} places
'Frodo, \textit{see} Baggins' in the index.
Nesting related item is accomplished by, e.g.,
\verb'\index{Baggins!Frodo}'\index{Baggins!Frodo},
\verb'\index{Baggins!Bilbo}'\index{Baggins!Bilbo},
\verb'\index{Baggins!Bilbo!adventures}'\index{Baggins!Bilbo!adventures},
\verb'\index{Baggins!Bilbo!rings}'\index{Baggins!Bilbo!rings}.
\item The index requires some help to determine where to place
non-standard symbols.
For example, \MF\index{Metafont@\MF} should be indexed alphbetically
as ``Metafont'', not as ``\\Metafont'', while \cmdltxidx{\MF} should be
indexed alphabetically as ``MF''. 
This is accomplished by the command \verb'\index{Metafont@\MF}'. 
The argument before the \verb'@' is the lexicographical value of the
entry used for alphabetization, and the argument after the \verb'@'
is the value actually written to the index. 
By the same token, the command \verb'\index{index@\verb+\index+}'
places the word \verb'\index'\index{index@\verb+\index+} under the 
letter ``i''. 
Note that this method is tedious and prone to error since it involves
typing the word ``index'' three times.
Heiko Oberdiek has provided me with the command, \cmdltxidx{\cmdltxidx}
which prints \verb'\cmdltxidx' in the text and in the index at the
correct alphabetical location.
\end{enumerate}

\subsection[Line Numbering]{Line Numbering}\label{sxn:nbr}
\linenumbers % Turn on linenumbers
To number each line in a document, use the package
\flidx{lineno.sty}. 
The \cmdltxidx{\linenumbers} command turns on line-numbering,
and the \cmdltxidx{\nolinenumbers} turns it off.
To demonstrate, the next few sections have line-numbering turned~on.
The package is very flexible and supports options like
\cmdltxidx{\pagewiselinenumbering}, \cmdltxidx{\rightlinenumbers},
etc. 
\flidx{lineno.sty} even supports labeling lines and refering to
specific lines with the standard reference commands, e.g.,
\cmdltxidx{\ref}. 
Unfortunately, line-numbering is expensive in terms of processing time
and contents of the auxiliary file, so expect \LaTeX\ to slow down
when large amounts of material are numbered.

When printing source code, it may be desirable to nicely format
certain keywords in the language.
The \LaTeX\ command \cmdltxidx{\cxx} prints ``\cxx'' instead of the
unformatted ``\trmidx{C++}''.

\subsection[Vertical Spacing]{Vertical Spacing}\label{sxn:spc_vrt}
To change the vertical spacing in a document, use the package
\flidx{setspace.sty}.
%\usepackage{setspace} % \singlespacing, \onehalfspacing, \doublespacing

\subsection[Horizontal Spacing]{Horizontal Spacing}\label{sxn:spc_hrz}
Spacing between numbers and dimensions is a complicated issue.
The issue is most often ignored by novices, and, for those who care,
there are no universal solutions.
A~related question is the spacing between multiple physical dimensions
in a single unit, so-called \trmdfn{interdimensional spacing}.
% comp.text.tex thread 20030709 
The \textit{J.~Fluid.\ Mech.} convention appears to be a
\cmdltxidx{\;} space between the value and dimension, and then a 
\cmdltxidx{\,} space for interdimensional spacing, e.g.,
\verb'5\;cm\,s$^{-1}$' produces 5\;cm\,s$^{-1}$.
\trmidx{AGU} journals appear to use \verb'\,' in both positions, e.g.,
\verb'5\,cm\,s$^{-1}$' produces 5\,cm\,s$^{-1}$.

\subsection[Verbatim]{Verbatim}\label{sxn:vrb}
The package \flidx{listings.sty} contains extensive options to nicely
format input code.
First one must define the current language environment with the
\cmdltxidx{\lstset} command, e.g.,
\lstset{% Set up listings.sty environment
language=[LaTeX]TeX, % Language for listings.sty \lstinline
stringstyle=\ttfamily,
keywordstyle=\ttfamily}
\begin{lstlisting}{}
\lstset{% Set up listings.sty environment
language=[LaTeX]TeX, % Language for listings.sty \lstinline
stringstyle=\ttfamily,
keywordstyle=\ttfamily}
\end{lstlisting}
Once \verb'\lstset' has been initialized, the \cmdltxidx{\lstinline}
command is used to format code inline, and the \cmdidx{lstlisting}
environment is available to place longer code fragments in display
style. 

The \verb'srcltx' package enables \cmdidx{xdvi} to display \LaTeX\ code 
synchronized with the display:
Also consider the \flidx{fancyvrb.sty} package by 
\trmidx[van~Zandt, Timothy]{Timothy van Zandt}.

\nolinenumbers % Turn off linenumbers
\subsection[CVS]{CVS}\label{sxn:cvs}
Using CVS in \LaTeX\ documents is not straightforward.
Norman Gray recommended the following on \verb'comp.text.tex'
\begin{verbatim}
\def\CVS$#1: #2 ${\expandafter\def\csname CVS#1\endcsname{#2}}
\CVS$Revision$
\CVS$Date$
\end{verbatim}
and use it like \verb'\date{\CVSDate, version \CVSRevision}'.
This recipe only works when CVS versioning is turned on, i.e.,
when the \verb'-kk' option is not specified.
The above definition must appear in the document being
tracked, i.e., the \verb'.tex' file, rather than a style file.
Otherwise the version information will reflect the evolution of the 
style file and not the document being tracked.
Unfortunately this contributes to preamble bloat.
\begin{comment}
Following this procedure in the present document leads to
\CVS$Revision$
\CVS$Date$
The date and CVS info are \cmdltxidx{\CVSDate} = \CVSDate and
\cmdltxidx{\CVSRevision} = \CVSRevision.
\end{comment}

\subsection[Strikeout]{Strikeout}\label{sxn:strikeout}
Collaboratively revising text documents is prone to be difficult.
\LaTeX\ is text-based and, in conjunction with revision control
software like \trmidx{CVS}, this makes it easy to allow multiple
authors to work on one source document without mutual interference.
However, indicating the revisions one has made to a shared document
is probably more difficult with \LaTeX\ than with \trmidx{WYSIWYG}
editors. 

One way to indicate revisions with \LaTeX\ is to \trmidx{strikeout} 
text using the \cmdltxidx{\sout} command provided by the \verb'ulem'
package. 
For example,
\begin{verbatim}
This text \sout{has been} is striked out.
\end{verbatim}
produces ``This text \sout{has been} is striked out.''

\subsection[Watermarks]{Watermarks}\label{sxn:wtr}
Printing a message on each page is useful to watermark draft work,
so it is a \acr{FAQ}.
To do this in \LaTeX, use the \cmdltxidx{\draftcopy} package by Juergen
Vollmer. 
The word ``DRAFT'' (assuming an English document) will be blazed
across the specified pages. 
\verb'ghostview' has troubles showing \textsl{DRAFT}
watermark in the on-screen window.
Ghostscript (\verb'gs') works fine with watermarks.

Watermarking \PDFTeX-generated files requires a more sophisticated
package called \verb'eso-pic' which is contained in Martin
Schro\"eder's \verb'ms' bundle.
For \acr{pdf}, try the following:
\begin{verbatim}
\usepackage[pdftex]{graphics,color} % 
\usepackage{eso-pic} % Required for Draft (\AddToShipoutPicture)
\AddToShipoutPicture{\resizebox{0.9\pdfpagewidth}{0.9\pdfpageheight}%           
{\rotatebox{60}{\color[gray]{0.8}\hspace*{5mm}\textsc{Sample Paper}}}}
\end{verbatim}

\subsection[Symbols]{Symbols}\label{sxn:sym}
The variety of symbols \LaTeX\ can produce is astounding.
The definitive source of these symbols is the Symbols document by
M. Scott Pakin, available from
\url{ftp://cam.ctan.org/tex-archive/info/symbols/comprehensive/}.
Some of the more frequently used \TeX-related symbols are listed
in Table~\ref{tbl:sym}.
The \verb'\usepackage{texnames}' makes the proper logos available,
but, according to Robin Fairbairns, it is an out-of-date, poorly coded
package which should be avoided if possible (i.e., use \verb'mflogo'
instead for \MF).
\begin{table}
\begin{minipage}{\hsize} % Minipage necessary for footnotes KoD95 p. 110 (4.10.4)
\renewcommand{\footnoterule}{\rule{\hsize}{0.0cm}\vspace{-0.0cm}} % KoD95 p. 111
\begin{center}
\caption[\TeX\ Family Symbols]{\textbf{\TeX\ Family Symbols}%
\footnote{\emph{Sources:} \cite{KoD99,KoD04}}%
\label{tbl:sym}}
\vspace{\cpthdrhlnskp}
\begin{tabular}{ r l l }
\hline \rule{0.0ex}{\hlntblhdrskp}% 
Symbol & Command & Package \\[0.0ex]
\hline \rule{0.0ex}{\hlntblntrskp}%
\TeX & \cmdltxidx{\TeX} & none\\[0.5ex]
\TeXInfo & \verb'\TeX{}Info', \cmdltxidx{\TeXInfo} & \textsf{csz} \\[0.5ex]
\LaTeX & \cmdltxidx{\LaTeX}, \verb'\LATEX' & none, \textsf{texnames} \\[0.5ex]
\LaTeXe & \cmdltxidx{\LaTeXe} & none \\[0.5ex]
\MF\index{Metafont@\verb'\MF'} & \cmdltxidx{\MF} & \textsf{mflogo} \\[0.5ex]
%\METAFONT & \verb'\METAFONT' & \textsf{texnames} \\[0.5ex]
%\AMSTEX & \verb'\AMSTEX', \verb'\AmSTeX', \cmdltxidx{\AMSTeX} & \textsf{texnames} \\[0.5ex]
%\BIBTEX & \verb'\BIBTEX', \verb'\BIBTeX', \cmdltxidx{\BibTeX} & \textsf{texnames} \\[0.5ex]
% texnames.sty defines \AMSTEX, \AMSTeX, \AmSTeX, \BIBTEX, \BIBTeX, \BibTeX, \LAMSTeX, \LAmSTeX, \LaTeX, \LATEX, \METAFONT, \MF, \SLITEX, \SLITeX, \SLiTeX, \SliTeX
\hline
\end{tabular}
\end{center}
\end{minipage}
\end{table}
According to the UK \TeX\ FAQ, ``For those who don't wish to acquire
the 'proper' logos, the canonical thing to do is to say \verb'AMS-\TeX' for
AMS-\TeX, \verb'Pic\TeX' for Pic\TeX, \verb'Bib\TeX' for Bib\TeX, and
so on.''
Extending this advice to 
More recent packages have their own styles.
A few of the more useful are \verb'\textsc{pdf}\TeX'
for \textsc{pdf}\TeX, and \verb'\textsc{pdf}\LaTeX' for
\textsc{pdf}\LaTeX. 

%Other symbols of interest are shown in Table~\ref{tbl:sym2}
% fxm: Must install txfonts and remove symbolname clashes between
% wasysym and dchem packages, e.g., \saturn
\begin{comment} 
\begin{table}
\begin{minipage}{\hsize} % Minipage necessary for footnotes KoD95 p. 110 (4.10.4)
\renewcommand{\footnoterule}{\rule{\hsize}{0.0cm}\vspace{-0.0cm}} % KoD95 p. 111
\begin{center}
\caption[Miscellaneous Symbols]{\textbf{Miscellaneous Symbols}%
\footnote{\emph{Sources:} \cite{KoD99,KoD04}}%
\label{tbl:sym2}}
\vspace{\cpthdrhlnskp}
\begin{tabular}{ r l l }
\hline \rule{0.0ex}{\hlntblhdrskp}% 
Symbol & Command & Package \\[0.0ex]
\hline \rule{0.0ex}{\hlntblntrskp}%
\Blitz & \cmdltxidx{\Blitz} & \textsf{ifsym} \\[0.5ex]
\Sun & \cmdltxidx{\Sun} & \textsf{ifsym} \\[0.5ex]
\FilledRainCloud & \cmdltxidx{\FilledRainCloud} & \textsf{ifsym} \\[0.5ex]
\SunCloud & \cmdltxidx{\SunCloud} & \textsf{ifsym} \\[0.5ex]
\Thermo{1} & \verb'\Thermo{1}' & \textsf{ifsym} \\[0.5ex]
\Snow & \cmdltxidx{\Snow} & \textsf{ifsym} \\[0.5ex]
\piup & \cmdltxidx{\piup} & \textsf{txfonts} \\[0.5ex]
\tauup & \cmdltxidx{\tauup} & \textsf{txfonts} \\[0.5ex]
\neptune & \cmdltxidx{\neptune} & \textsf{wasysym} \\[0.5ex]
\FemaleMale & \cmdltxidx{\FemaleMale} & \textsf{marvosym} \\[0.5ex]
\Aries & \cmdltxidx{\Aries} & \textsf{marvosym} \\[0.5ex]
\Radioactivity & \cmdltxidx{\Radioactivity} & \textsf{marvosym} \\[0.5ex]
\photon & \cmdltxidx{\photon} & \textsf{wasysym} \\[0.5ex]
\vernal & \cmdltxidx{\vernal} & \textsf{wasysym} \\[0.5ex]
%\foo & \verb'\foo' & \textsf{ifsym} \\[0.5ex]
\hline
\end{tabular}
\end{center}
\end{minipage}
\end{table}
\end{comment}
Often a font is named after its creator, whether a person or a
company.
Thus learning the names of font creators is a good mnemonic for the
package names.
The \textsf{marvosym} fonts are named for ``Martin Vogel's Symbols''.

\subsection[Graphics]{Graphics}\label{sxn:gph}
Inclusion of graphics images is controlled by the \textsf{graphix}
package.
The \cmdltxidx{\includegraphics} macro imports Postscript (\verb'*.ps') and
encapsulated Postscript (\verb'*.eps') files automatically.

Normally graphics \trmdfn{float} in documents, i.e., their
position varies so as to flow well with the surrounding text.
These floats are usually placed within the \verb'figure' environment,
so that a caption may be easily added.
With sizing and caption, the skeleton code to include a graphical
figure looks like
\begin{verbatim}
\begin{figure*}
\centering
\includegraphics[width=0.5\hsize]{SWCF_x_01}\vfill
\includegraphics[width=0.5\hsize]{SWCF_x_07}%
\caption[Shortwave Cloud Forcing]{SWCF for (a) January and (b) July.  
\label{fgr:SWCF_x1}}
\end{figure*}
\end{verbatim}
\begin{figure*}
\centering
\includegraphics[width=0.5\hsize]{/Users/zender/data/ps/erbe_b_sld012d_8589_01_x_SWCF}\vfill
\includegraphics[width=0.5\hsize]{/Users/zender/data/ps/erbe_b_sld012d_8589_07_x_SWCF}%
\caption[Shortwave Cloud Forcing]{SWCF for (a) January and (b) July. 
\label{fgr:SWCF_x2}}
\end{figure*}
Similar commands produced Figure~\ref{fgr:SWCF_x3}.
The \verb'figure' environment only spans a single text column.
The \verb'figure*' environment spans all columns in a multi-column
document. 
The location of the float is an optional argument to the \verb'figure'
environment, e.g., \verb'\begin{figure*}[b]' to place the float at the
bottom of the current or the next page.

\begin{comment}
Subfigures may be explicitly accounted for when using the
\verb'subfigure' package.
\begin{figure}
\centering
\caption[Shortwave Cloud Forcing]{SWCF for (a) January and (b) July. 
\label{fgr:SWCF_x3}}
%\addtocounter{figure}{-1}% Decrement figure counter when caption is above figure
\subfigure[January]{\includegraphics[width=0.5\hsize]{/Users/zender/data/ps/erbe_b_sld012d_8589_01_x_SWCF}\label{fgr:SWCF_x_01}}
\subfigure[July]{\includegraphics[width=0.5\hsize]{/Users/zender/data/ps/erbe_b_sld012d_8589_07_x_SWCF}\label{fgr:SWCF_x_07}}
\end{figure}
\end{comment}

It is not straightforward to import other graphics types.
Raster graphics such as \GIFacr\ and \JPEGacr\ must first be converted 
to Postscript.
\LaTeX\ will do this automatically if it knows the conversion rules.
This may be accomplished with the \cmdltxidx{\DeclareGraphicsRule} command.
The prerequisite is that a shell-executable exist which can produce 
Postscript.
The command \verb'gif2eps' used to exist and could handle \GIFacr\
images. 
Here is how a \GIFacr\ figure might be indicated
\begin{verbatim}
\DeclareGraphicsRule{.gif}{eps}{}{`gif2eps #1 -} % Convert GIF files to eps
\begin{figure*}
\centering
\includegraphics[width=0.5in,height=0.5in]{/Users/zender/data/fgr/dmr/r2d2.gif}
\caption[R2D2]{R2D2.\label{fgr:r2d2}}
\end{figure*}
\end{verbatim}
\begin{figure*}
\centering
\includegraphics[width=0.5in,height=0.5in]{/Users/zender/data/fgr/dmr/r2d2.eps}
\caption[R2D2]{R2D2.\label{fgr:r2d2}}
\end{figure*}
These commands produced Figure~\ref{fgr:r2d2}.
Note the use of the \verb'width' and \verb'height' options to size the
figure. 
If only one dimension were given, then \textsf{graphicx} would
automatically maintain the natural aspect ratio of the figure.

\subsubsection[Clipping and Rotation]{Clipping and Rotation}\label{sxn:clp}
To trim images, use the \cmdidx{trim} and \cmdidx{clip} options to
\cmdltxidx{includegraphics}.
Specify the viewport margins in the order Left, Bottom, Right, Top.
``Liberty'' is a good mnenomonic for the \cmdprn{lbrt} ordering.
\begin{verbatim}
\includegraphics[width=1.0\hsize,clip=true,trim=0.75in 4.1in 0.1in 4.9in]{/Users/zender/data/fgr/ess_atm/ppr_Rud05_fgr1}%
\end{verbatim}
 
\subsection[Trademarks]{Trademarks}\label{sxn:trd}
Identifying intellectual property correctly requires some knowledge of 
copyrights and trademarks.
What distinguishes a trademark from a registered trademark or a
service mark is not clear.
\LaTeX\ automatically makes available the symbols \cmdltxidx{\copyright}, 
\cmdltxidx{\texttrademark}, and \cmdltxidx{\textregistered} to denote these
marks \cite[][p.~239]{KoD99}.
For example, this work is Copyright \copyright\ 2000--2003 by me.
\verb'Linux\textregistered' produces Linux\textregistered.
\verb'Linux\texttrademark' produces Linux\texttrademark\ (where the
superscript is automatic).
However, the text companion package \textsf{textcomp} provides nicer
versions of \verb'\texttrademark' and \verb'\textregistered' than the
defaults, so remember to load it for important documents.
\textsf{textcomp} also provides some symbols that are unique, such as 
\cmdltxidx{\textservicemark} (e.g., Linux\textservicemark),
\cmdltxidx{\textonehalf} (\textonehalf), \cmdltxidx{\textonequarter}
(\textonequarter), and \cmdltxidx{\textthreequarters} (\textthreequarters).

\subsection[Phonetic Alphabet]{Phonetic Alphabet}\label{sxn:phn}
Linguists use the \trmidx{International Phonetic Alphabet} (\trmidx{IPA}) 
to indicate the correct pronunciation of words.
Many consanants are relatively straightforward in the IPA---they are 
pronounced as an English-speaker expects.
More interesting typographically and phonetically are vowels, elisions,
and aspirations that require special symbols.

The \trmidx{TIPA} package provides \LaTeX\ support for IPA text.
Table~\ref{tbl:ipa} show the results of various typesetting various
IPA strings.
% fxm:\raggedright does not work in long tables, must be between two other columns?
% Begin Table~\ref{tbl:ipa}
\begin{longtable}{ >{\raggedright}p{25.0em}<{} l l } % KoD95 p. 94 describes '*' notation
& \kill % NB: longtable requires caption as table entry
% longtables seem to choke htlatex
\caption[Typesetting Phonetics]{\textbf{Typesetting Phonetics (IPA)}%
\footnote{\emph{Source:} \href{http://en.wikipedia.org/wiki/TIPA}{Wikipedia}}%
\label{tbl:ipa}} \\
\hline\hline \rule{0.0ex}{\hlntblhdrskp}% 
\LaTeX\ Command & Result & \\[0.0ex]
\hline \rule{0.0ex}{\hlntblntrskp}%
\endfirsthead % Lines between and \endfirsthead appear at top of table
\caption[]{(continued)} \\ % Set label for following pages
\LaTeX\ Command & Result & \\[0.0ex]
\hline \rule{0.0ex}{\hlntblntrskp}%
\endhead % Previous block appears at top of every page
\endlastfoot % Previous block appears at end of table
\verb'\textipa{[""Ekspl@"neIS@n]}' & \textipa{[""Ekspl@"neIS@n]} & \\[0.5ex]
\verb'[\textsecstress\textepsilon kspl\textschwa'  
\verb'\textprimstress ne\textsci\textesh\textschwa n]' & [\textsecstress\textepsilon kspl\textschwa \textprimstress ne\textsci\textesh\textschwa n] & \\[0.5ex]
\verb'\textipa{/f@"nEtIks/}' & \textipa{/f@"nEtIks/} & \\[0.5ex]
% \verb'' & & \\[0.5ex]
\hline
\end{longtable}
% End Table~\ref{tbl:ipa}

\subsection[Currencies]{Currencies}\label{sxn:crr}
The \trmidx{euro}, the official currency of the European Union,
may be typeset with \cmdltxidx{\texteuro} (\texteuro).
Other currency symbols include\ldots

\subsection[Acronyms and Abbreviation]{Acronyms and Abbreviation}\label{sxn:acr}
Abbreviations are contractions of phrases into a sequence of letters
each usually representing the first letter of a word.
Abbreviations are not intended to be pronounced as a single word.
For example, USA is an abbreviation.
Acronyms are abbreviations which are pronounceable without spelling
the abbreviation letter-by-letter.
Thus IBM is not an acronym, and NATO is.

Typographical conventions set acronyms in a ``smallcaps'' font, i.e.,
a font where capital letters are smaller than the regular text font.
In \TeX, smallcaps is one of the font shapes (\S\ref{sxn:nfss}).
Periodicals which follow this convention include \textsl{The
Economist}. 
Thus acronyms may be correctly typeset ``on the fly'' using the
\cmdltxidx{\textsc} macro.
For example, \verb'\textsc{nco}' produces \textsc{nco},
while \verb'\textsc{Unix}' produces \textsc{Unix}.
There is also a homebrew macro, \cmdltxidx{\acr}, which is currently
just a wrapper for \verb'\textsc'.
For example, \verb'\acr{nco}' produces \acr{nco}.

After raising the question ``When to use smallcaps for acronyms?''
on the \verb'comp.text.tex' \textsc{usenet} list, a variety of 
answers and rationales were proposed.
Robin Fairbairns noted that Barbara Beeton specified a not-quite-small
caps variant for use in Tugboat.
His code for that is
\begin{verbatim}
\usepackage{relsize}
\def\acro#1{\textsmaller{#1}\@}
\acro{TUG} conferences aren't much like \acro{SOSP}s.
\end{verbatim}
Others simply use a small text font
\begin{verbatim}
\newcommand{\cap}[1]{{\small{#1}}}
\newcommand{\capRB}[1]{\raisebox{1pt}{\small{(}}\small{#1}\raisebox{1pt}{\small{)}}} 
\TeX\ User Group \capRB{TUG} conferences aren't much like \cap{SOSP}s.
\end{verbatim}

% PDF bookmarks disallow math formatting so use \texorpdfstring instead
\subsection[Text Samples \texorpdfstring{with math $\alpha$}{}]{Text Samples}\label{sxn:txt_xmp}
Here we try some typical \LaTeX\ text formatting.
\begin{enumerate}
\item Here we use the custom \cmdltxidx{\dgr} (degree) macro, first in text
mode, then in math mode. 
\verb'\dgr' uses \cmdltxidx{\ensuremath} so there should be no discernible
difference: 
\verb'10\dgr S--10\dgr N' gives ``10\dgr S--10\dgr N''.
\verb'10$\dgr$S--10$\dgr$N' gives ``10$\dgr$S--10$\dgr$N''.
\item \cmdltxidx{\includegraphics} will automatically scale a graphic to
fit within a given box while maintaining aspect ratio with this
argument \verb'width=xxx,height=yyy,keepaspectratio'
\end{enumerate}

Two commands useful for typesetting \trmidx{URLs} and 
\trmidx{e-mail addresses} are \cmdidx{\url} and \cmdidx{\href}. 
Table~\ref{tbl:txt} show the results of various typesetting techniques
on URLs, including problematic long URLs.
% fxm:\raggedright does not work in long tables, must be between two other columns?
\begin{longtable}{ *{2}{>{\raggedright}p{15.0em}<{}} l } % KoD95 p. 94 describes '*' notation
& \kill % NB: longtable requires caption as table entry
% longtables seem to choke htlatex
\caption[Typesetting Text]{\textbf{Typesetting Text}%
\footnote{\emph{Source:} \cite{KoD95,KoD99,KoD04}}%  
\label{tbl:txt}} \\
\hline\hline \rule{0.0ex}{\hlntblhdrskp}% 
\LaTeX\ Command & Result & \\[0.0ex]
\hline \rule{0.0ex}{\hlntblntrskp}%
\endfirsthead % Lines between and \endfirsthead appear at top of table
\caption[]{(continued)} \\ % Set label for following pages
\LaTeX\ Command & Result & \\[0.0ex]
\hline \rule{0.0ex}{\hlntblntrskp}%
\endhead % Previous block appears at top of every page
\endlastfoot % Previous block appears at end of table
\verb'\url{nco.sourceforge.net}' & \url{nco.sourceforge.net} & \\[0.5ex]
\verb'\href{http://nco.sourceforge.net}{NCO Homepage}' &
\href{http://nco.sourceforge.net}{NCO Homepage} & \\[0.5ex]
\verb'\url{zender@uci.edu}' & \url{zender@uci.edu} & \\[0.5ex]
\verb'\url{http://some/really/long/URL/that/wants/to/wrap}' & \url{http://some/really/long/URL/that/wants/to/wrap} & \\[0.5ex]
\verb'\href{http://some/really/long/URL/that/wants/to/wrap}{Short Name}' &
\href{http://some/really/long/URL/that/wants/to/wrap}{Short Name} & \\[0.5ex]
% \verb'' & & \\[0.5ex]
\hline
\end{longtable}
Unfortunately, the \cmdidx{dvips} driver is unable to automatically
break URLs across lines.
On the other hand, \cmdidx{pdflatex} intelligently breaks URLs
whenever necessary. 
A~\cmdidx{comp.text.tex} thread initiated on Oct.~24, 2001 by Olive
Moeller discusses the reasons for this.
Thus, as demonstrated in Table~\ref{tbl:txt}, documents may be
formatted differently depending on whether they are produced with
\cmdprn{dvips} or \cmdidx{pdflatex}. 

\section[Text Typography]{Text Typography}\label{sxn:txt}

\subsection[Gothic]{Gothic}\label{sxn:gth}
%\bigdrop{0pt}{3}{yinit}{G}othic 
%\lettrine[lines=3]{G}{othic}
\dropletter{G}{othic}
fonts occupy a special place in the history of typography since they
appear in the oldest typeset texts in Western civilization.
Yannis Haralambous used \MF\ to design four interesting Old German
fonts. 
The Gothic letter initiating this paragraph if from the 
``Yannis initial'' font called \textsf{yinit}.
As mentioned above, the initial letter of paragraphs is ``dropped''
with the \textsf{lettrine} package, in this case
%\verb'\bigdrop{0pt}{3}{yinit}{G}othic'\ldots. 
\verb'\lettrine[lines=3]{G}{othic}'\ldots. 
Unfortunately, accessing non-default fonts in \LaTeX\ generally
involves manipulating a very detailed and complex specification
scheme, \acr{nfss} (\S\ref{sxn:nfss}). 
\begin{itemize}
\item Yannis Fraktur: \verb'\usefont{U}{yfrak}{m}{n}'
\usefont{U}{yfrak}{m}{n} Hello, World. \normalfont
\item Yannis Gothic: \verb'\usefont{U}{ygoth}{m}{n}'
\usefont{U}{ygoth}{m}{n} Hello, World. \normalfont
\item Yannis Schwabacher: \verb'\usefont{U}{yswab}{m}{n}'
\usefont{U}{yswab}{m}{n} Hello, World. \normalfont
\item Yannis Initial: \verb'\usefont{U}{yinit}{m}{n}'
\usefont{U}{yinit}{m}{n} Hello, World. \normalfont
\end{itemize}
\begin{comment} % Lettrine options:
lines: Number of lines the drop cap spans
findent: Controls the horizontal gap between the drop cap and the text
nindent: Shifts indented lines starting with the second
slope: Adapts the slope of the text lines to match letters like A, V, etc.
ante: Produces text before the drop cap
image=true: Loads an image (requires graphicx) as drop cap
\end{comment} % Lettrine options
% 20140531 Absolute path /home/zender incompatible with Mac OS X directory structure
% Need working relative pathnames
%\lettrine[image=true,lines=3,findent=3pt,nindent=0pt]{/home/zender/texmf/fonts/csz/saints_initial_O.jpg}{nce} 
\lettrine[image=true,lines=3,findent=3pt,nindent=0pt]{/Users/zender/texmf/fonts/csz/saints_initial_O.jpg}{nce} 
we assemble a complete alphabet of initial letters from illuminated
manuscripts, it will be a pleasure to typeset manuscripts.
Imagine the beautiful cadences of paragraphs illustrated with initials.

\subsection[Text Fonts]{Text Fonts}\label{sxn:txt_fnt}
Table~\ref{tbl:txt_sty} shows different series and styles of the
default text font.
\begin{longtable}{ l l }
& \kill % NB: longtable requires caption as table entry
\caption[Text Series and Styles]{\textbf{Text Series and Styles}%
\footnote{\emph{Source:} \cite{KoD99,KoD04}}%  
\label{tbl:txt_sty}} \\
\hline\hline \rule{0.0ex}{\hlntblhdrskp}% 
\LaTeX\ Font & Result \\[0.0ex]
\hline \rule{0.0ex}{\hlntblntrskp}%
\endfirsthead % Lines between and \endfirsthead appear at top of table
\caption[]{(continued)} \\ % Set label for following pages
\LaTeX\ Font & Result \\[0.0ex]
\hline \rule{0.0ex}{\hlntblntrskp}%
\endhead % Previous block appears at top of every page
\endlastfoot % Previous block appears at end of table
\cmdltxidx{\textrm} & \textrm{ABCDEFGHIJKLMNOPQRSTUVWXYZabcdefghijklmnopqrstuvwxyz0123456789}\\[0.5ex]
\cmdltxidx{\textsf} & \textsf{ABCDEFGHIJKLMNOPQRSTUVWXYZabcdefghijklmnopqrstuvwxyz0123456789}\\[0.5ex]
\cmdltxidx{\texttt} & \texttt{ABCDEFGHIJKLMNOPQRSTUVWXYZabcdefghijklmnopqrstuvwxyz0123456789}\\[0.5ex]
\cmdltxidx{\textup} & \textup{ABCDEFGHIJKLMNOPQRSTUVWXYZabcdefghijklmnopqrstuvwxyz0123456789}\\[0.5ex]
\cmdltxidx{\textit} & \textit{ABCDEFGHIJKLMNOPQRSTUVWXYZabcdefghijklmnopqrstuvwxyz0123456789}\\[0.5ex]
\cmdltxidx{\textsl} & \textsl{ABCDEFGHIJKLMNOPQRSTUVWXYZabcdefghijklmnopqrstuvwxyz0123456789}\\[0.5ex]
\cmdltxidx{\textsc} & \textsc{ABCDEFGHIJKLMNOPQRSTUVWXYZabcdefghijklmnopqrstuvwxyz0123456789}\\[0.5ex]
\cmdltxidx{\textmd} & \textmd{ABCDEFGHIJKLMNOPQRSTUVWXYZabcdefghijklmnopqrstuvwxyz0123456789}\\[0.5ex]
\cmdltxidx{\textbf} & \textbf{ABCDEFGHIJKLMNOPQRSTUVWXYZabcdefghijklmnopqrstuvwxyz0123456789}\\[0.5ex]
\cmdltxidx{\tiny} & \tiny{ABCDEFGHIJKLMNOPQRSTUVWXYZabcdefghijklmnopqrstuvwxyz0123456789}\\[0.5ex]
\cmdltxidx{\scriptsize} & \scriptsize{ABCDEFGHIJKLMNOPQRSTUVWXYZabcdefghijklmnopqrstuvwxyz0123456789}\\[0.5ex]
\cmdltxidx{\footnotesize} & \footnotesize{ABCDEFGHIJKLMNOPQRSTUVWXYZabcdefghijklmnopqrstuvwxyz0123456789}\\[0.5ex]
\cmdltxidx{\small} & \small{ABCDEFGHIJKLMNOPQRSTUVWXYZabcdefghijklmnopqrstuvwxyz0123456789}\\[0.5ex]
\cmdltxidx{\normalsize} & \normalsize{ABCDEFGHIJKLMNOPQRSTUVWXYZabcdefghijklmnopqrstuvwxyz0123456789}\\[0.5ex]
\cmdltxidx{\large} & \large{ABCDEFGHIJKLMNOPQRSTUVWXYZabcdefghijklmnopqrstuvwxyz0123456789}\\[0.5ex]
\cmdltxidx{\Large} & \Large{ABCDEFGHIJKLMNOPQRSTUVWXYZabcdefghijklmnopqrstuvwxyz0123456789}\\[0.5ex]
\cmdltxidx{\LARGE} & \LARGE{ABCDEFGHIJKLMNOPQRSTUVWXYZabcdefghijklmnopqrstuvwxyz0123456789}\\[0.5ex]
\cmdltxidx{\huge} & \huge{ABCDEFGHIJKLMNOPQRSTUVWXYZabcdefghijklmnopqrstuvwxyz0123456789}\\[0.5ex]
\cmdltxidx{\Huge} & \Huge{ABCDEFGHIJKLMNOPQRSTUVWXYZabcdefghijklmnopqrstuvwxyz0123456789}\\[0.5ex]
% \verb'\foo' & \foo{ABCDEFGHIJKLMNOPQRSTUVWXYZabcdefghijklmnopqrstuvwxyz0123456789} \\[0.5ex]
\hline
\end{longtable}

Table~\ref{tbl:txt_fml} shows different families of text fonts.
\begin{longtable}{ l l }
& \kill % NB: longtable requires caption as table entry
\caption[Text Fonts]{\textbf{Text Fonts}%
\footnote{\emph{Source:} \cite{KoD99,KoD04}}%  
\label{tbl:txt_fml}} \\
\hline\hline \rule{0.0ex}{\hlntblhdrskp}% 
\LaTeX\ Font & Result \\[0.0ex]
\hline \rule{0.0ex}{\hlntblntrskp}%
\endfirsthead % Lines between and \endfirsthead appear at top of table
\caption[]{(continued)} \\ % Set label for following pages
\LaTeX\ Font & Result \\[0.0ex]
\hline \rule{0.0ex}{\hlntblntrskp}%
\endhead % Previous block appears at top of every page
\endlastfoot % Previous block appears at end of table
\cmdltxidx{\textgoth} & \textgoth{ABCDEFGHIJKLMNOPQRSTUVWXYZabcdefghijklmnopqrstuvwxyz0123456789}\\[0.5ex]
\cmdltxidx{\textfrak} & \textfrak{ABCDEFGHIJKLMNOPQRSTUVWXYZabcdefghijklmnopqrstuvwxyz0123456789}\\[0.5ex]
\cmdltxidx{\textswab} & \textswab{ABCDEFGHIJKLMNOPQRSTUVWXYZabcdefghijklmnopqrstuvwxyz0123456789}\\[0.5ex]
% \verb'\foo' & \foo{ABCDEFGHIJKLMNOPQRSTUVWXYZabcdefghijklmnopqrstuvwxyz0123456789} \\[0.5ex]
\hline
\end{longtable}

\section[Math]{Math Typography}\label{sxn:mth}

\subsection[Math Conventions]{Math Conventions}\label{sxn:mth_cnv}
The ISO has established conventions regarding symbols appearing in
mathematical documents \cite[][p.~142]{KoD99}. 
The most often violated convention, which is also easy avoided with
\LaTeX, concerns the use of upright fonts for symbols denoting fixed,
constant values.
So, for example, the base of the \trmidx{natural logarithm} should be
denoted $\me$ rather than simply $e$. 
This is accomplished by using \verb'$\me$' rather than \verb'$e$'.
Likewise the imaginary number is $\mi$ (\cmdltxidx{\mi}) rather than
$i$ (\verb'$i$'); \trmidx{pi} is $\mpi$ (\cmdltxidx{\mpi}) rather than
$\pi$ (\cmdltxidx{\pi}), and the symbol for a \trmidx{differential}
element is, e.g., $\dfr x$ (\verb'$\dfr x$'), rather than $dx$
(\verb'$dx$'). 
Uppercase upright Greek symbols may be obtained from the
\textsf{symbols} font or from the \textsf{txfonts} package. 
\cmdidx{\mathbf} is a bold upright mathematical font, and contains
only Latin and uppercase Greek letters.
The package \flidx{upgreek.sty} supplies upright Greek letters when
the normal letter command is prefixed with ``up'', e.g., \verb'\uppi'
produces $\uppi$ whereas \verb'\pi' produces $\pi$.
Unfortunately, the letters produced by \flprn{upgreek.sty} appear to
be boldface.
The package \flidx{bm.sty} boldfaces Greek letters (as does the
\cmdltxidx{\pmb} or ``Poor man's bold'' from \flidx{amsbsy.sty}).
The packages \flidx{mathptmx.sty} and \flidx{mathpazo.sty} create
full, bold-italic, alphabets in the \trmidx{Times} and
\trmidx{Palatino} fonts, respectively. 
Whether this convention applies to Greek numerical prefixes is
unclear.
For example, a \trmdfn{micron}, $10^{-6}$\,\m, is often written 
``$\mu$m'' (\verb'$\mu$m').
The~$\mu$ in~$\mu$m, however, has a fixed value~($10^{-6}$).
It is, in essence, a universal constant much like $\mpi$.
Therefore microns should be written with an upright mu,
i.e.,~$\upmu$m.
Most journals, including \trmidx{AGU} journals, adhere to this
format. 

According to \cite{KoD99}, p.~142 and to \cite{KoD04}, p.~145:
\begin{quote}
\begin{enumerate}
\item Simple variables are represented by italic letters, $a b c x y z$.
\item Vectors are written in bold face italic, as $\boldsymbol{B v
    \omega}$.
\item Tensors of 2nd order and matrices may appear in a sans serif
  font, as $\mathsf{M D I}$
\item The special numbers $\me$, $\mi$, $\mpi$, as well as the
  differential operator $\dfr$, are to be \textit{written in an
  upright font} to emphasize that they are not variables.
\item A measurement consisting of a number plus a dimension is an
indivisible unit, with a smaller than normal space between them, as 
5.3\,\km\ and 62\,\um. 
The dimension is set in an upright font.
\end{enumerate}
\end{quote}
\cite{KoD04} recommend using fxm.

\subsection[Math Equations]{Math Equations}\label{sxn:mth_eqn}
Lengthy derivations may require breaking the derivation across a page
boundary. 
It is generally considered smarter to diallow page breaks between
equation lines by default, and to require the author to specifically
enable them where necessary.
Thus normal display environments, e.g., the \verb'eqnarray'
environment, do not, by default, allow breaking across pages.
The \cmdltxidx{\displaybreak} command causes a break in a particular
equation on a particular page.
The optimal position for the \verb'\displaybreak' is just before the
\verb'\\' where it should take effect.
\verb'\displaybreak' takes an optional integer argument valued 0
to~4 to indicate the degree of penalty associated with breaking a
display environment, with 0 merely allowing a break to take place,
and~4 requiring that the page break, e.g., \verb'\displaybreak[4]'.

AMS\TeX\ also supplies the \cmdltxidx{\allowdisplaybreaks} macro which
changes the default to permit displayed equations to flow smoothly
from one page to the next. 
This macro should be placed in the document preamble.
\verb'\allowdisplaybreaks' takes an optional integer argument valued 1
to 4 to indicate the degree of penalty associated with breaking a
display environment, with 4 being the most permissive, e.g.,
\verb'\allowdisplaybreaks[4]'.  

\subsection[Math Fonts]{Math Fonts}\label{sxn:mth_fnt}
Table~\ref{tbl:mth_fnt} shows the different mathematical fonts.
The \cmdltxidx{\mathscr} command is defined by multiple packages.
When \verb'\usepackage[mathscr]{eucal}' is employed, \verb'\mathscr'
prints a vertical script font, but when \verb'\usepackage{mathrsfs}'
is in effect \verb'\mathscr' prints an ornate and very curvy script
font reminiscent of wedding invitations. 
\begin{longtable}{ l >{$}l<{$} }
& \kill % NB: longtable requires caption as table entry
\caption[Math Fonts]{\textbf{Math Fonts}%
\footnote{\emph{Source:} \cite{KoD99,KoD04}}%  
\label{tbl:mth_fnt}} \\
\hline\hline \rule{0.0ex}{\hlntblhdrskp}% 
\LaTeX\ Command & \mbox{Result} \\[0.0ex]
\hline \rule{0.0ex}{\hlntblntrskp}%
\endfirsthead % Lines between and \endfirsthead appear at top of table
\caption[]{(continued)} \\ % Set label for following pages
\LaTeX\ Command & \mbox{Result} \\[0.0ex]
\hline \rule{0.0ex}{\hlntblntrskp}%
\endhead % Previous block appears at top of every page
\endlastfoot % Previous block appears at end of table
\trmidx{Greek} & \Gamma \Delta \Theta \Lambda \Xi \Pi \Sigma \Upsilon \Phi \Psi
\Omega \alpha \beta \gamma \delta \epsilon \varepsilon \zeta \eta
\theta \vartheta \iota \kappa \lambda \mu \nu \xi o \pi \varpi \rho
\varrho \sigma \varsigma \tau \upsilon \phi \varphi \chi \psi \omega \\[0.5ex]
\cmdltxidx{\bm} & \bm{\Gamma \Delta \Theta \Lambda \Xi \Pi \Sigma \Upsilon \Phi \Psi
\Omega \alpha \beta \gamma \delta \epsilon \varepsilon \zeta \eta
\theta \vartheta \iota \kappa \lambda \mu \nu \xi o \pi \varpi \rho
\varrho \sigma \varsigma \tau \upsilon \phi \varphi \chi \psi \omega} \\[0.5ex]
\flidx{upgreek.sty} & \Upgamma \Updelta \Uptheta \Uplambda \Upxi \Uppi \Upsigma \Upupsilon \Upphi \Uppsi
\Upomega \upalpha \upbeta \upgamma \updelta \upepsilon \upvarepsilon \upzeta \upeta
\uptheta \upvartheta \upiota \upkappa \uplambda \upmu \upnu \upxi o \uppi \upvarpi \uprho
\upvarrho \upsigma \upvarsigma \uptau \upupsilon \upphi \upvarphi \upchi \uppsi \upomega \\[0.5ex]
\cmdltxidx{\mathnormal} & \mathnormal{ABCDEFGHIJKLMNOPQRSTUVWXYZabcdefghijklmnopqrstuvwxyz0123456789} \\[0.5ex]
\cmdltxidx{\mathrm} & \mathrm{ABCDEFGHIJKLMNOPQRSTUVWXYZabcdefghijklmnopqrstuvwxyz0123456789} \\[0.5ex]
\cmdltxidx{\mathsf} & \mathsf{ABCDEFGHIJKLMNOPQRSTUVWXYZabcdefghijklmnopqrstuvwxyz0123456789} \\[0.5ex]
\cmdltxidx{\mathtt} & \mathtt{ABCDEFGHIJKLMNOPQRSTUVWXYZabcdefghijklmnopqrstuvwxyz0123456789} \\[0.5ex]
\cmdltxidx{\mathit} & \mathit{ABCDEFGHIJKLMNOPQRSTUVWXYZabcdefghijklmnopqrstuvwxyz0123456789} \\[0.5ex]
\cmdltxidx{\mathbf} & \mathbf{ABCDEFGHIJKLMNOPQRSTUVWXYZabcdefghijklmnopqrstuvwxyz0123456789} \\[0.5ex]
\cmdltxidx{\mathcal} & \mathcal{ABCDEFGHIJKLMNOPQRSTUVWXYZabcdefghijklmnopqrstuvwxyz0123456789} \\[0.5ex]
\cmdltxidx{\mathbb} & \mathbb{ABCDEFGHIJKLMNOPQRSTUVWXYZabcdefghijklmnopqrstuvwxyz0123456789} \\[0.5ex]
\cmdltxidx{\mathscr} & \mathscr{ABCDEFGHIJKLMNOPQRSTUVWXYZabcdefghijklmnopqrstuvwxyz0123456789} \\[0.5ex]
\cmdltxidx{\mathfrak} & \mathfrak{ABCDEFGHIJKLMNOPQRSTUVWXYZabcdefghijklmnopqrstuvwxyz0123456789} \\[0.5ex]
%\cmdltxidx{\pmb} & \pmb{ABCDEFGHIJKLMNOPQRSTUVWXYZabcdefghijklmnopqrstuvwxyz0123456789} \\[0.5ex]
& \mbox{fxm 20030224, 20050827: Produces ``Too many math alphabets used in ``version normal'' error here after upgreek added, thus some fonts are not being printed} \\[0.5ex]
%\cmdltxidx{\mathpzc} & \mathpzc{ABCDEFGHIJKLMNOPQRSTUVWXYZabcdefghijklmnopqrstuvwxyz0123456789} \\[0.5ex]
%\cmdltxidx{\mathds} & \mathds{ABCDEFGHIJKLMNOPQRSTUVWXYZabcdefghijklmnopqrstuvwxyz0123456789} \\[0.5ex]
%\cmdltxidx{\varmathbb} & \varmathbb{ABCDEFGHIJKLMNOPQRSTUVWXYZabcdefghijklmnopqrstuvwxyz0123456789} \\[0.5ex]
\hline
\end{longtable}

\subsection[Math Samples]{Math Samples}\label{sxn:mth_smp}

\LaTeX\ contains many useful, but hard-to-remember, commands and
symbols for formatting mathematics.
It is easy to forget when to use uncommon symbols like
\cmdidx{\lesssim}, \cmdidx{\gtrapprox}, and \cmdidx{\,}.
Table~\ref{tbl:mth} shows results of different mathematical
typesetting techniques.
\begin{longtable}{ l >{$}l<{$} }
& \kill % NB: longtable requires caption as table entry
\caption[Typesetting Math]{\textbf{Typesetting Math}%
\footnote{\emph{Source:} \cite{KoD95}}%  
\label{tbl:mth}} \\
\hline\hline \rule{0.0ex}{\hlntblhdrskp}% 
\LaTeX\ Command & \mbox{Result} \\[0.0ex]
\hline \rule{0.0ex}{\hlntblntrskp}%
\endfirsthead % Lines between and \endfirsthead appear at top of table
\caption[]{(continued)} \\ % Set label for following pages
\LaTeX\ Command & \mbox{Result} \\[0.0ex]
\hline \rule{0.0ex}{\hlntblntrskp}%
\endhead % Previous block appears at top of every page
\endlastfoot % Previous block appears at end of table
\verb'\frac{\partial \prs}{\partial \tpt}|_{\vlm}' &
\dpysty \frac{\partial \prs}{\partial \tpt}|_{\vlm} \\[2.0ex]
\verb'\frac{\partial \prs}{\partial \tpt}\big|_{\vlm}' &
\dpysty \frac{\partial \prs}{\partial \tpt}\big|_{\vlm} \\[2.0ex]
\verb'\frac{\partial \prs}{\partial \tpt}\Big|_{\vlm}' &
\dpysty \frac{\partial \prs}{\partial \tpt}\Big|_{\vlm} \\[2.0ex]
\verb'\frac{\partial \prs}{\partial \tpt}\bigg|_{\vlm}' &
\dpysty \frac{\partial \prs}{\partial \tpt}\bigg|_{\vlm} \\[2.0ex]
\verb'\frac{\partial \prs}{\partial \tpt}\Bigg|_{\vlm}' &
\dpysty \frac{\partial \prs}{\partial \tpt}\Bigg|_{\vlm} \\[2.0ex]
\verb'10\dgr \times 10\dgr' & 10\dgr \times 10\dgr \\[0.5ex]
\verb'\tilde{\dmt}_{\nbrsbs}' & \tilde{\dmt}_{\nbrsbs} \\[0.5ex]
\verb'\tilde{\dmt_{\nbrsbs}}' & \tilde{\dmt_{\nbrsbs}} \\[0.5ex]
\verb'\tilde{\dmt}_{\nbrsbs}^{2}' & \tilde{\dmt}_{\nbrsbs}^{2} \\[0.5ex]
\verb'\tilde{\dmt_{\nbrsbs}^{2}}' & \tilde{\dmt_{\nbrsbs}^{2}} \\[0.5ex]
\verb'\bar{\bar{\dmt}}_{\nbrsbs}' & \bar{\bar{\dmt}}_{\nbrsbs} \\[0.5ex]
\verb'\bar{\dmt}_{\nbrsbs}' & \bar{\dmt}_{\nbrsbs} \\[0.5ex]
\verb'\bar{\dmt_{\nbrsbs}}' & \bar{\dmt_{\nbrsbs}} \\[0.5ex]
\verb'\bar{\dmt}_{\nbrsbs}^{2}' & \bar{\dmt}_{\nbrsbs}^{2} \\[0.5ex]
\verb'\bar{\dmt_{\nbrsbs}^{2}}' & \bar{\dmt_{\nbrsbs}^{2}} \\[0.5ex]
\verb'\hat{\dmt}_{\nbrsbs}' & \hat{\dmt}_{\nbrsbs} \\[0.5ex]
\verb'\hat{\dmt_{\nbrsbs}}' & \hat{\dmt_{\nbrsbs}} \\[0.5ex]
\verb'\hat{\dmt}_{\nbrsbs}^{2}' & \hat{\dmt}_{\nbrsbs}^{2} \\[0.5ex]
\verb'\hat{\dmt_{\nbrsbs}^{2}}' & \hat{\dmt_{\nbrsbs}^{2}} \\[0.5ex]
\verb'\overline{\dmt}_{\nbrsbs}' & \overline{\dmt}_{\nbrsbs} \\[0.5ex]
\verb'\overline{\dmt_{\nbrsbs}}' & \overline{\dmt_{\nbrsbs}} \\[0.5ex]
\verb'\overline{\dmt}_{\nbrsbs}^{2}' & \overline{\dmt}_{\nbrsbs}^{2} \\[0.5ex]
\verb'\overline{\dmt_{\nbrsbs}^{2}}' & \overline{\dmt_{\nbrsbs}^{2}} \\[0.5ex]
\verb'\underline{\dmt}_{\nbrsbs}' & \underline{\dmt}_{\nbrsbs} \\[0.5ex]
\verb'\underline{\dmt_{\nbrsbs}}' & \underline{\dmt_{\nbrsbs}} \\[0.5ex]
\verb'\underline{\dmt}_{\nbrsbs}^{2}' & \underline{\dmt}_{\nbrsbs}^{2} \\[0.5ex]
\verb'\underline{\dmt_{\nbrsbs}^{2}}' & \underline{\dmt_{\nbrsbs}^{2}} \\[0.5ex]
\verb'\dot{\dmt}_{\nbrsbs}' & \dot{\dmt}_{\nbrsbs} \\[0.5ex]
\verb'\dot{\dmt_{\nbrsbs}}' & \dot{\dmt_{\nbrsbs}} \\[0.5ex]
\verb'\dot{\dmt}_{\nbrsbs}^{2}' & \dot{\dmt}_{\nbrsbs}^{2} \\[0.5ex]
\verb'\dot{\dmt_{\nbrsbs}^{2}}' & \dot{\dmt_{\nbrsbs}^{2}} \\[0.5ex]
\verb'\ddot{\dmt}_{\nbrsbs}' & \ddot{\dmt}_{\nbrsbs} \\[0.5ex]
\verb'\ddot{\dmt_{\nbrsbs}}' & \ddot{\dmt_{\nbrsbs}} \\[0.5ex]
\verb'\ddot{\dmt}_{\nbrsbs}^{2}' & \ddot{\dmt}_{\nbrsbs}^{2} \\[0.5ex]
\verb'\ddot{\dmt_{\nbrsbs}^{2}}' & \ddot{\dmt_{\nbrsbs}^{2}} \\[0.5ex]
\verb'\vec{\dmt}_{\nbrsbs}' & \vec{\dmt}_{\nbrsbs} \\[0.5ex]
\verb'\vec{\dmt_{\nbrsbs}}' & \vec{\dmt_{\nbrsbs}} \\[0.5ex]
\verb'\vec{\dmt}_{\nbrsbs}^{2}' & \vec{\dmt}_{\nbrsbs}^{2} \\[0.5ex]
\verb'\vec{\dmt_{\nbrsbs}^{2}}' & \vec{\dmt_{\nbrsbs}^{2}} \\[0.5ex]
\verb'\imath' & \imath \\[0.5ex]
\verb'\vec{\imath}' & \vec{\imath} \\[0.5ex]
\verb'\hat{\imath}' & \hat{\imath} \\[0.5ex]
\verb'\mathbf{\vec{\iii}}' & \mathbf{\vec{\iii}}\\[0.5ex]
\verb'\mathbf{\iii}' & \mathbf{\iii}\\[0.5ex]
\verb'\mathbf{\hat{i}}' & \mathbf{\hat{i}}\\[0.5ex]
\verb'\mbox{\boldmath$\hat{\imath}$}' & \mbox{\boldmath$\hat{\imath}$} \\[0.5ex]
\verb'\mbox{\boldmath$\hat{\iii}$}' & \mbox{\boldmath$\hat{\iii}$} \\[0.5ex]
\verb'\jmath' & \jmath \\[0.5ex]
\verb'\vec{\jmath}' & \vec{\jmath} \\[0.5ex]
\verb'\hat{\jmath}' & \hat{\jmath} \\[0.5ex]
\verb'\mathbf{\hat{j}}' & \mathbf{\hat{j}}\\[0.5ex]
\verb'\mbox{\boldmath$\hat{\jmath}$}' & \mbox{\boldmath$\hat{\jmath}$} \\[0.5ex]
\verb'\vec{k}' & \vec{k} \\[0.5ex]
\verb'\hat{k}' & \hat{k} \\[0.5ex]
\verb'\mathbf{\hat{k}}' & \mathbf{\hat{k}}\\[0.5ex]
\verb'\mbox{\boldmath$\hat{k}$}' & \mbox{\boldmath$\hat{k}$} \\[0.5ex]
\verb'\int_{0}^{\infty} \dstnbrrds \; \dfr\rds' & \int_{0}^{\infty} \dstnbrrds \; \dfr\rds \\[0.5ex]
\verb'\int_{0}^{\infty} \dstnbrrds \: \dfr\rds' & \int_{0}^{\infty} \dstnbrrds \: \dfr\rds \\[0.5ex]
\verb'\int_{0}^{\infty} \dstnbrrds \, \dfr\rds' & \int_{0}^{\infty} \dstnbrrds \, \dfr\rds \\[0.5ex]
\verb'\int_{0}^{\infty} \dstnbrrds \dfr\rds' & \int_{0}^{\infty} \dstnbrrds \dfr\rds \\[0.5ex]
\verb'\int_{0}^{\infty} \dstnbrrds \! \dfr\rds' & \int_{0}^{\infty} \dstnbrrds \! \dfr\rds \\[0.5ex]
\verb'\dstfnc_{\nbrsbs}(\dmt)' & \dstfnc_{\nbrsbs}(\dmt) \\[0.5ex]
\verb'\dstfnc^{\mathrm{o}}_{\nbrsbs}(\dmt)' & \dstfnc^{\mathrm{o}}_{\nbrsbs}(\dmt) \\[0.5ex]
\verb'\dstfnc^{o}_{\nbrsbs}(\dmt)' & \dstfnc^{o}_{\nbrsbs}(\dmt) \\[0.5ex]
\verb'\dstfnc^{\mathrm{e}}_{\nbrsbs}(\dmt)' & \dstfnc^{\mathrm{e}}_{\nbrsbs}(\dmt) \\[0.5ex]
\verb'\dstfnc^{e}_{\nbrsbs}(\dmt)' & \dstfnc^{e}_{\nbrsbs}(\dmt) \\[0.5ex]
\verb'\tilde{\gsd}^{2}' & \tilde{\gsd}^{2} \\[0.5ex]
\verb'\tilde{\gsd}^{\!\!2}' & \tilde{\gsd}^{\!\!2}\\[0.5ex]
\verb'\tilde{\gsd^{2}}' & \tilde{\gsd^{2}}\\[0.5ex]
\verb'\tilde{\gsd^{\!\!2}}' & \tilde{\gsd^{\!\!2}}\\[0.5ex]
\verb'\widetilde{\gsd}^{\!\!2}' & \widetilde{\gsd}^{\!\!2}\\[0.5ex]
\verb'\dmtnaa' & \dmtnaa \\[0.5ex]
\verb'\dmtnar' & \dmtnar \\[0.5ex]
\verb'\dmtnma' & \dmtnma \\[0.5ex]
\verb'\dmtnmr' & \dmtnmr \\[0.5ex]
\verb'\dmtnwa' & \dmtnwa \\[0.5ex]
\verb'\dmtnwr' & \dmtnwr \\[0.5ex]
\verb'\mbox{\textonehalf}', \verb'\mbox{\textonequarter}' & \mbox{\textonehalf} \mbox{\textonequarter} \\[0.5ex]
\verb'\frac{1}{2} \frac{2}{3}' & \frac{1}{2} \frac{2}{3} \\[0.5ex]
\verb'\dfrac{1}{2} \dfrac{2}{3}' & \dfrac{1}{2} \dfrac{2}{3} \\[0.5ex]
\verb'\tfrac{1}{2} \tfrac{2}{3}' & \tfrac{1}{2} \tfrac{2}{3} \\[0.5ex]
\verb'\sfrac{1}{2} \sfrac{2}{3}' & \sfrac{1}{2} \sfrac{2}{3} \\[0.5ex]
\verb'1 / 2 \ 2 / 3' & 1 / 2 \ 2 / 3 \\[0.5ex]
\verb'1 / 2 \ 2 / 3' & 1 / 2 \ 2 / 3 \\[0.5ex]
\verb'x^{1/2}' & x^{1/2} \\[0.5ex]
\verb'x^{1\!/2}' & x^{1\!/2} \\[0.5ex]
\verb'x^{1\negthinspace/2}' & x^{1\negthinspace/2} \\[0.5ex]
\verb'x^{1\!/\!2}' & x^{1\!/\!2} \\[0.5ex]
\verb'x^{1\!/\mspace{-2mu}2}' & x^{1\!/\mspace{-2mu}2} \\[0.5ex]
\verb'x^{\frac{1}{2}}' & x^{\frac{1}{2}} \\[0.5ex]
\verb'x^{\sfrac{1}{2}}' & x^{\sfrac{1}{2}} \\[0.5ex]
\verb'x^{\tfrac{1}{2}}' & x^{\tfrac{1}{2}} \\[0.5ex]
\verb'a \lesssim b' & a \lesssim b \\[0.5ex]
\verb'a \lessapprox b' & a \lessapprox b \\[0.5ex]
\verb'a \cong b' & a \cong b \\[0.5ex]
\verb'a \approxeq b' & a \approxeq b \\[0.5ex]
\verb'a \gtrsim b' & a \gtrsim b \\[0.5ex]
\verb'a \gtrapprox b' & a \gtrapprox b \\[0.5ex]
\verb'\pi \Pi \prod' & \pi \Pi \prod \\[0.5ex]
\verb'\sigma \Sigma \sum' & \sigma \Sigma \sum \\[0.5ex]
\verb'\mpi' & \mpi \\[0.5ex]
\verb'\mathbf{\pi \Pi}' & \mathbf{\pi \Pi} \\[0.5ex]
\verb'\bm{\pi \Pi}' & \bm{\pi \Pi} \\[0.5ex]
\verb'\hm{\pi \Pi}' & \hm{\pi \Pi} \\[0.5ex]
\verb'\uppi' & \uppi \\[0.5ex]
\verb'\mbox{\boldmath$\pi \uppi \Pi$}' & \mbox{\boldmath$\pi \uppi \Pi$} \\[0.5ex]
\verb'\pmb{\pi}' & \pmb{\pi} \\[0.5ex]
\verb'\partial' & \partial \\[0.5ex]
\verb'\nabla \times \cccbld' & \nabla \times \cccbld \\[0.5ex]
\verb'\nabla \cross \cccbld' & \nabla \cross \cccbld \\[0.5ex]
\verb'\nabla_{\iii\iii}' & \nabla_{\iii\iii} \\[0.5ex]
\verb'\nabla_{\!\iii\iii}' & \nabla_{\!\iii\iii} \\[0.5ex]
\verb'\textsl{\textsf{\bfseries{M}}}' & \textsl{\textsf{\bfseries{M}}} \\[0.5ex]
\verb'\textsl{\textsf{M}}' & \textsl{\textsf{M}} \\[0.5ex]
\verb'\mathsf{M}' & \mathsf{M} \\[0.5ex]
\verb'\mathbf{M}' & \mathbf{M} \\[0.5ex]
\verb'\mathsfsl{M}' & \mathsfsl{M} \\[0.5ex]
\verb'\mathit{\mathsf{M}}' & \mathit{\mathsf{M}} \\[0.5ex]
% \verb'' & \\[0.5ex]
\hline
\end{longtable}

% Bibliography
%\renewcommand\refname{\normalsize Publications}
\nocite{KoD95,KoD99,KoD04}
\bibliographystyle{agu04}
\bibliography{bib}
\printindex % Requires makeidx KoD95 p. 221
\addcontentsline{toc}{section}{Index}

\begin{comment}
% Transfer required figures to local machine
for fl_stb in \
erbe_b_sld012d_8589_01_x_SWCF \
erbe_b_sld012d_8589_07_x_SWCF \
sld012d_8589_01_x_SWCF \
sld012d_8589_07_x_SWCF \
; do 
scp 'dust.ess.uci.edu:${DATA}/ps/'${fl_stb}'.eps' ${DATA}/ps
epstopdf ${DATA}/ps/${fl_stb}.eps
done
\end{comment}
% $: rebalance syntax highlighting

\begin{comment} 
% Usage: Place usage here at end of file so comment character % not needed
cd ~/tex;make -W ltx.tex ltx.dvi ltx.ps ltx.pdf ltx.txt;cd -

# cd ~/tex;pdflatex ltx.tex;thumbpdf ltx;pdflatex ltx.tex;cd -
# cd ~/tex;texcln ltx;make ltx.pdf;bibtex ltx;makeindex ltx;make ltx.pdf;bibtex ltx;makeindex ltx;make ltx.pdf;cd -
# cd ~/tex;texcln ltx;latex ltx;bibtex ltx;makeindex ltx;latex ltx;bibtex ltx;makeindex ltx;dvips -Ppdf -G0 -o ${DATA}/ps/ltx.ps ~/tex/ltx.dvi;ps2pdf ${DATA}/ps/ltx.ps ${DATA}/ps/ltx.pdf;cd -
scp ${HOME}/tex/ltx.tex ${HOME}/tex/ltx.txt ${DATA}/ps/ltx.ps ${DATA}/ps/ltx.pdf ${HOME}/tex/ltx.dvi dust.ess.uci.edu:/var/www/html/doc/ltx

# NB: latex2html chokes on math in tables in \csznote{}s
latex2html -dir /var/www/html/doc/ltx ltx.tex
# NB: tth works OK on ltx.tex: does no math, ignores epigraphs
cd ${HOME}/tex;tth -a -Lltx -p./:${TEXINPUTS}:${BIBINPUTS} < ${HOME}/tex/ltx.tex > ltx.html
scp ltx.html dust.ess.uci.edu:/var/www/html/doc/ltx
# NB: tex4ht chokes on ltx.tex (longtable problem?)
cd ${HOME}/tex;htlatex ltx.tex
scp ltx*.css ltx*.html dust.ess.uci.edu:/var/www/html/doc/ltx
# NB: tex4moz chokes on ltx.tex
cd ${HOME}/tex;/usr/share/tex4ht/mzlatex ltx.tex
scp ltx*.css ltx*.html ltx*.xml dust.ess.uci.edu:/var/www/html/doc/ltx
\end{comment}

\end{document}

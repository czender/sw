%***********************************************************************
% RCS Identification
%***********************************************************************
% $Author: zender $
% $Date$
% $Id$
% $Revision$
% $Locker:  $
% $RCSfile: ppr_tpl.tex,v $
% $Source: /home/zender/cvs/tex/ppr_tpl.tex,v $
% $Id$
% $State: Exp $
%
% Purpose: Template for quickie papers.
%
% $Log: not supported by cvs2svn $
% Revision 1.1.1.1  1998-09-15 01:49:58  zender
% Imported sources
%
% Revision 1.2  1993/10/11  20:37:20  zender
% lots of tiny mods made.
%
% Revision 1.1  1993/05/30  23:41:14  zender
% Initial revision
%
%***********************************************************************

%***********************************************************************
% Begin TeX notes
%***********************************************************************
% This is written for preprocessing by tib, the citation handler.
% To use all the features of tib the following commands are necessary:
% 
% alias tib       'tib -d /u4/boettner/tib ' 		;in the .cshrc file 
% alias tiblist   'tiblist -d /u4/boettner/tib '	;in the .cshrc file 
%
% tibdex clouds.ref		;creates INDEX
% tib -s jgr clouds.tex		;creates clouds-t.tex
%				;(-s = style, jgr = J. Geophys. Res. format)
% tib clouds.tex		;uses default style
% tex clouds-t.tex		;creates clouds-t.log and clouds-t.dvi
% xdvi clouds-t.dvi		;X window previewer
% dvips clouds-t.dvi		;creates clouds-t.ps
% tiblist -s jgr clouds.ref	;creates tex file directly from references
% tiblook <keyword>		;locates entry from reference file
%
% In order to include citations in the text do the following:
% <, ramaswamy detwiler ,> includes the unique citation which has both
% ramaswamy and detwiler as keywords in running text with
% the year in parenthesis, while [, reference ,] (where the ``,'' is a ``.'')
% puts the whole citation in parenthesis.
%
% This also uses the AMS TeX eplain.tex macro package for equation
% numbering.  To define an equation number, place \eqdef{eqname} right
% AFTER the equation.  To reference it in the text, use \eqref{eqname}
% The same goes for figures, only use \figdef and \figref. CAUTION:
% The figdef and eqdef macros use the same registers so naming a
% figure the same as an equation will confuse it.
%
% Note the following style points: the word ``figure'' should always
% be capitalized in text, i.e., and spelled in full at the beginning
% of a sentence but abbreviated in the middle of a sentence. ``Table''
% and ``Plate'' are the same but should never be abbreviated. And
% letters identifying subgraphs, i.e., Fig.~7{\it a\/} are italicized 
% in JGR but not JAS.
% 
% {\it Italicized\/} words should be followed by ``\/'' to adjust the
% next space correctly (except when the following character is a
% period or a comma). 
% 
% Forced spaces in math mode can be made by using the \ character,
% although that allows line breaks (unlike ~ which doesn't work in
% math mode), e.g., $\lambda = 8\ \mu$ instead of $\lambda = 8$~$\mu$.
%
%$$
%e^{i\pi}=-1
%\eqdef{eqn_name}
%$$
%
%Figure~\figref{fig_name}{\it a\/} is on the next page.
%
%$$
%e^{i\pi}=-1 \qquad\hbox{equation with text for $0 < w < 20$, 
%{\it IWP} in g/m$^2$, $w$ in cm/s} 
%\eqdef{eqn_name2}
%$$
%
% Here's an example of a bulleted outline:
%\vskip 1em
%\item{a.} First
%
%\item{b.} Second
%\itemitem{} sub-item
%\itemitem{} sub-item
%
%\vskip 1em
%
%***********************************************************************
% End TeX notes
%***********************************************************************

%***********************************************************************
% Begin Preamble
%***********************************************************************
% Define general page, paragraph, and line formatting here.
% macros for equation numbering
\input /cgd/home/zender/tex/eplain
% personal TeX definitions of item and month name
\input /cgd/home/zender/tex/utilities
% for inclusion of postscript figures /usr/local/lib/tex_family/tex/inputs/epsf.tex
\input epsf
% have tex report size of picture in output
\epsfverbosetrue
%
\magnification=\magstep1
\font\ninerm=cmr9
\font\tenrm=cmr10
\font\titlefont=cmr10 scaled\magstep1
\font\slanttitlefont=cmsl10 scaled\magstep1
\baselineskip=12pt
\null
%
%Definitions, hyphenations, and macros.
\def\reff{ {$r_{\rm e}$} }
\def\IWP{ {\it IWP\/} }
\hyphenation{va-por}
\hyphenation{depo-sition}
\hyphenation{short-wave}
\hyphenation{long-wave}
\hyphenation{anal-yzed}
\hyphenation{envi-ron-mental}
\hyphenation{cond-itions}
\hyphenation{trop-ical}
\hyphenation{meso-scale}
\hyphenation{near-ly}
\hyphenation{}
\hyphenation{}
\hyphenation{}
\hyphenation{}
%
%***********************************************************************
% End Preamble
%***********************************************************************

%***********************************************************************
% Begin Title Page
%***********************************************************************
%Skip the page number on the title page
\pageno=-1
\headline={\tenrm {\it Draft of \today~~\timestring} \hfil \folio}
\footline={\hfil}
\nopagenumbers
\baselineskip=20pt
\ {\rm }
\vskip 2.truein
\centerline{Title of Paper}
\centerline{Charles S. Zender}
\vskip .5truein
{\obeylines\smallskip
\centerline{National Center for Atmospheric Research%
\footnote*{The National Center for Atmospheric Research is sponsored by the National Science Foundation.}}
\centerline{Boulder, Colorado 80307-3000}
\vskip 1.truein
\centerline{\today}}
\centerline{{\it To be Submitted to JGR Atmospheres 1994}}
\vfill\eject
%
%***********************************************************************
% End Title Page
%***********************************************************************

%***********************************************************************
% Begin Abstract 
%***********************************************************************
%Create the abstract:
\bigskip
\centerline{\it Abstract}
\smallskip
\baselineskip=20pt
{\parindent=0truein \narrower \smallskip \ninerm
\noindent
Abstractulate here
\smallskip}
\baselineskip=20pt
\vfill\eject
%
%***********************************************************************
% End Abstract 
%***********************************************************************

%***********************************************************************
% Begin Body of Paper
%***********************************************************************
\beginsection{1. Introduction} %the next line must be blank!

The intro begins here. 
%This will restore page numbering for rest of the manuscript. It
%should go somewhere before page two text begins.
\pageno=1
\headline={\tenrm {\it Draft of \today~~\timestring} \hfil [\oldstyle\folio\tenrm]}
%\headline={\tenrm \hfil [\oldstyle\folio\tenrm]}
\footline={\hfil}
%
Ending line of a section.
\medskip
\noindent{\sl Sub-section heading}\nobreak

Subsection begins here.

\beginsection{2. Second Section} %the next line must be blank!

%
\medskip
\baselineskip=20pt
\noindent{\it Acknowledgements.}
The authors wish to thank everybody.
This work was supported in part by Earth Observing System project
W-17,661. 
\baselineskip=20pt
%
\vfill\eject
%***********************************************************************
% End Body of Paper
%***********************************************************************

%***********************************************************************
% Begin Appendices
%***********************************************************************
%\centerline{APPENDIX}
%\bigskip
%\noindent{\bf Appendix name here}
%\bigskip
%
%\vfill\eject
%***********************************************************************
% End Appendices
%***********************************************************************

%***********************************************************************
% Begin References
%***********************************************************************
% A percent followed immediately by tib citations means the citations
% will be added to the reference list but won't be cited by tib
% anywhere in the paper.  this is useful for citations which had to
% be formatted by hand.
% the following .[] is used by tib to generate the reference list
%[,extra references,],[,more references,] 
\centerline{REFERENCES}
\bigskip
.[]
\vfill\eject
%***********************************************************************
% End References
%***********************************************************************

%***********************************************************************
% Begin Tables
%***********************************************************************
\nopagenumbers
\baselineskip=16pt
% The '#' means ``stick the text of each column entry in this place''.
% The '&' is like the the tab key on the typewriter, TeX backs up to
% the beginning of the current column then advances one column exactly.
% \cr signifies the end of the row, all unfilled columns are assumed
% blank (TeXbook, p. 231). 
% Here's an example of a formatted table placed in the middle of the text:
%\midinsert
%$$
%\vbox{
%\halign{
% Here is the template for the table:
%# \hfil \qquad & # \hfil \qquad \cr
%
%\noalign{\hrule}
%\noalign{\vskip-2pt}
%\noalign{\hrule}
%\noalign{\vskip6pt}
%\multispan2{\hfil Initial Properties of Standard Cloud \hfil} \cr
%\noalign{\vskip4pt}
%\noalign{\hrule}
%\noalign{\vskip4pt}
%Cloud base (km) & 11\cr
%Maximum vertical updraft (cm-s$^{-1}$) & 5\cr
%\noalign{\vskip4pt}
%\noalign{\hrule}
%
%} % end halign
%} % end vbox
%$$
\baselineskip=12pt
%{\bf Table~1.} The initial cloud properties are shown above. 
%The {\it mean\/} properties are vertical averages within the cloud. 
\baselineskip=16pt
%\endinsert
%
%
\baselineskip=16pt
\vfill\eject
%***********************************************************************
% End Tables
%***********************************************************************

%***********************************************************************
% Begin Figure Captions
%***********************************************************************
%Print out the figure captions on their own pages in their own font.
%Skip the page number on the figure caption pages.
\nopagenumbers
\baselineskip=20pt
%
%Here's a typical figure caption:
%\bigskip
%\vbox{
%\noindent{\bf  Figure~\figdef{fig_name}}.
%({\it a\/}) Figure blurb goes here.
%({\it b\/}) 
%}
%
%Reset the font from the figure caption settings to the text body:
\baselineskip=20pt
\tenrm
\vfill\eject
%***********************************************************************
% End Figure Captions
%***********************************************************************

%***********************************************************************
% Begin Figure Graphics w/o captions
%***********************************************************************
%
%\epsfxsize=6.5in
%\null\bigskip\centerline{\epsffile{/home/zender/web/papers/cir_rad_sens/figure_1.ps}}
%\bigskip\vbox{\centerline{\noindent{\bf Figure~1}.}}
%
%***********************************************************************
% End Figure Graphics w/o captions
%***********************************************************************

%***********************************************************************
% End Document
%***********************************************************************
%
\bye










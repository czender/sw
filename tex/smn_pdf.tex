% $Id$ -*-LaTeX-*-

% Purpose: Test file for LaTeX based PDF seminar
% File should be clean and depend only on generic packages if possible

% Other LaTeX resources:
% http://www.cgd.ucar.edu/stats/LATEX
% www.utopiatype.com

% Usage: 
% cd ~/tex;make -W smn_pdf.tex smn_pdf.dvi smn_pdf.ps smn_pdf.pdf;cd -
% cd ~/tex;pdflatex smn_pdf.tex;thumbpdf smn_pdf;pdflatex smn_pdf.tex;cd -
% cd ~/tex;make -W smn_pdf.tex smn_pdf.pdf;cd -

\documentclass[12pt]{article}

% Standard packages
\usepackage{ifpdf} % Define \ifpdf
\ifpdf % PDFLaTeX
\pdfcompresslevel=9
\usepackage{thumbpdf} % Generate thumbnails
\usepackage{epstopdf} % Convert .eps, if found, to .pdf when required
\fi % !PDFLaTeX
\usepackage{xspace} % Unknown
\usepackage{colortbl} % Unknown

% pdfscreen uses hyperref to set margins for screen mode
% pdfscreen calls hyperref internally if hyperref is not already invoked
% pdfscreen also loads packages graphicx, color, calc, and comment
% print 	Print version
% screen 	Screen version
% panelright	Navigation panel on RHS
% paneltoc	Table of Contents in panel
% sectionbreak	Introduces pagebreak before a sections
% code		Commands producing fancy verbatim-like effects
\usepackage[screen,code,panelright,paneltoc,sectionbreak]{pdfscreen}

% Personal packages
\usepackage{csz} % Library of personal definitions (\ifpdf...)
\usepackage{abc} % Alphabet as three letter macros
\usepackage{dmn} % Dimensional units
\usepackage{chm} % Chemistry
\usepackage{dyn} % Fluid dynamics
\usepackage{aer} % Aerosol physics
\usepackage{rt} % Commands specific to radiative transfer

% Commands which must be executed in preamble

% Commands specific to this file
\newcommand{\pdfscreen}{\texttt{\small\color{section1}pdfscreen}\xspace} % Pretty format for pdfscreen string

% Float placement
% NB: Placement of figures is very sensitive to \textfraction
\renewcommand\textfraction{0.0} % Minimum fraction of page that is text
\setcounter{totalnumber}{73} % Maximum number of floats per page
\setcounter{topnumber}{73} % Maximum number of floats at top of page
\setcounter{dbltopnumber}{73} % Maximum number of floats at top of two-column page
\setcounter{bottomnumber}{73} % Maximum number of floats at bottom of page
\renewcommand\topfraction{1.0} % Maximum fraction of top of page occupied by floats
\renewcommand\dbltopfraction{1.0} % Maximum fraction of top of two-column page occupied by floats
\renewcommand\bottomfraction{1.0} % Maximum fraction of bottom of page occupied by floats
\renewcommand\floatpagefraction{1.0} % Fraction of float page filled with floats
\renewcommand\dblfloatpagefraction{1.0} % Fraction of double column float page filled with floats

% Screen geometry
\begin{screen}
% These commands are all defined by pdfscreen
\margins{0.65in}{0.65in}{0.65in}{0.65in} % Left, right, top, bottom margins
%\emblema{logoLUC.png} % Location of graphic file for pdfscreen navigation panel
\screensize{6.25in}{8in} % Screen dimensions of PDF output
%\urlid{http://www.ess.uci.edu/~zender} % Homepage button in navigation panel points here
\changeoverlay % Cycle through default overlays, repeating every tenth section
\paneloverlay{but.pdf} % Overlay for navigation panel
\overlay{logo.pdf} % Overlay for screen area
\def\pfill{\vskip6pt}
\end{screen}

% Commands performed only in print section
\begin{print}
\notesname{Notes:} % String appearing in notes ovals adjacent to slides
% Redefine section formatting in print mode
\makeatletter
\def\@seccntformat#1{\llap{\scshape\color{section\thesection@level}\csname the#1\endcsname.\hspace*{6pt}}}
\makeatother
\end{print}

\hypersetup{ % A command provided by \hyperref
pdftitle={Foo},
pdfsubject={Foo},
pdfauthor={Charlie Zender},
pdfkeywords={aer mnr dst}
pdfpagemode={FullScreen}
} % end \hypersetup

\begin{document} % End preamble

% Title for screen mode
\begin{screen}
\title{\color{section0}\Huge Seminar on Something}
\end{screen}

% Title for print mode
\begin{print}
\title{\Huge\texttt{pdfscreen.sty} --- Seminar on Something}
\end{print}

\author{
\color{section1}\Large Charles~S.~Zender\\
\href{http://www.ess.uci.edu/~zender}{Department of Earth System Science}\\
\href{http://www.ess.uci.edu/~zender}{Homepage}\\
{\small\href{mailto:zender@uci.edu}{\color{section1}\texttt{zender@uci.edu}}}
} % end author

\pagedissolve{Split /D 2 /Dm /H /M /O} % pagedissolve options in pdfscreen manual p. 9

\date{} % Empty braces turns off date
\maketitle
\begin{screen}
\vfill
\end{screen}

\begin{abstract}
\noindent \pdfscreen package helps typeset seminars
\end{abstract}

% Do not use \tableofcontents command in document when paneltoc is option
\begin{print}
\tableofcontents
\end{print}
\begin{screen}
\vfill
\end{screen}

\section[The Atmosphere]{The Atmosphere}
\begin{itemize}
\item The Earth system % Annual mean surface temperature, precipitation
\item What is the atmosphere? % Composition
\item How the atmosphere ``works'' % Energy and moisture balance 
\item Weather vs. Climate % Winter/Summer contrast Day/Night contrast Sea breezes
\item History % Egyptians Satellites Weather prediction
\item Societal impacts % Why the atmosphere matters
\end{itemize}
\vfill

\section[RT]{What is RT?}
\pagedissolve{Glitter} % pagedissolve options in pdfscreen manual p. 9
\begin{itemize}
\item RT is Radiative Transfer
\end{itemize}
% decl is used to format LaTeX syntax:
% Vertical bars delineate command names
% \arg places optional argument brackets around text
\begin{decl}
|\margins|\arg{left}\arg{right}\arg{top}\arg{bottom}
\end{decl}
\vfill

\section{Dry Deposition Parameterization}
\pagedissolve{} % pagedissolve options in pdfscreen manual p. 9
\begin{eqnarray}
\flxdps & = & - \vlcdps \cnc \\ % SeP97 p. 958 (19.1)
\label{eqn:flx_dps_dfn}
\vlcdps & = & \vlctrb + \vlcgrv \nonumber \\
\label{eqn:vlc_dps_dfn}
& = & \frac{1}{\rssaer + \rsslmn + \rssaer \rsslmn \vlcgrv } + \vlcgrv \\
\label{eqn:rss_dps_dfn}
\rsslmn & = & \left\{
\begin{array}{ >{\displaystyle}l<{} @{\quad:\quad}l}
\frac{ 1}{\wndfrc ( \shmnbr^{-2/3} + 10^{-3/\stknbr} ) } & \mbox{Solid surfaces} \\
\frac{ 1}{\wndfrc ( \shmnbr^{-1/2} + 10^{-3/\stknbr} ) } & \mbox{Liquid surfaces}
\end{array} \right.
\label{eqn:rss_lmn_dfn}
\end{eqnarray}
\vfill

\section{Size Sensitivity of Dry Deposition}
Dry deposition velocity $\vlcdps$ (\cmxs) as a function of aerosol
size $\dmtprt$ (\um), and surface roughness length $\rghmmn$ (cm).
\begin{figure*}
% Compare to SeP97 p. 970 Figure 19.3, p. 971 Figure 19.4
% Seh84 p. 553 Figure 12.3, Seh80 p. 999 Figure 5
% Figure created by mie.pro:scv_gph()
\includegraphics[height=0.5\vsize]{/data/zender/ps/vlc_dps_aer.pdf}\vfill
\caption[Dry Deposition Velocity]{
Dry deposition velocity $\vlcdps$ (\cmxs) as a function of aerosol
size $\dmtprt$ (\um), and surface roughness length $\rghmmn$ (cm).
Shown are the total dry deposition velocity $\vlcdps$ (solid), 
the gravitational settling velocity $\vlcgrv$ (dashed), and the
turbulent velocity $\vlctrb$ (dotted).
\label{fgr:vlc_dps_aer}}
\end{figure*}
\vfill
\clearpage

\begin{slide}
\section[Slides]{Slides}
Slide format--do things look bigger?
\begin{itemize}
\item Yes, but slides in screen format have this distracting box next
to them
\end{itemize}
\end{slide}
\vfill

\section[Resources]{RT resources}
\begin{itemize}
\item \href{http://www.tug.org}{\TeX{} Users Group homepage}
\item \href{http://paos.colorado.edu/5235/frames/index.html}{CU course
PAOS 5235 by Judy Curry}
\item \href{http://www.dri.edu/GradPrograms/ASC/CourseInfo/Atms749/index.html}{DRI course AS 749 by Arnott}
\item \href{http://topex.ucsd.edu/rs}{UCSD course SIO 236 by Dave Sandwell}
\end{itemize}
\vfill

\end{document}

